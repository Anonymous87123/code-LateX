\documentclass{mcmthesis}
 %\documentclass[CTeX = true]{mcmthesis}  % 当使用 CTeX 套装时请注释上一行使用该行的设置
\mcmsetup{tstyle = \color{red}\bfseries,  % 修改题号,队号的颜色和加粗显示,黑色可以修改为 black
          tcn = 2616863, problem = A,     % 修改队号,参赛题号
          sheet = true, titleinsheet = true, keywordsinsheet = true,
          titlepage = false, abstract = true}

  %四款字体可以选择
  %\usepackage{times}
  %\usepackage{newtxtext}
  %\usepackage{palatino}
\usepackage{txfonts}

\usepackage{caption}  
\captionsetup[figure]{font=small}   % 将图标题字号设为small
\captionsetup[table]{font=small}    % 将图标题字号设为small

\usepackage{indentfirst}  % 首行缩进,注释掉,首行就不再缩进
\usepackage{tocloft}      % 目录格式控制
\usepackage{esint}        % 积分号
\usepackage{subcaption}   % 并排图片
% \usepackage{floatrow}     % 与caption包存在兼容性问题,已禁用

\usepackage{float}
\floatstyle{plaintop}
\restylefloat{table}

\usepackage{xcolor}       % 颜色支持
\usepackage{tabularray}   % 增强表格功能

\usepackage[style=numeric,backend=biber]{biblatex}  % APA风格: style=apa
\ExecuteBibliographyOptions{sorting=none}           % 按引用顺序
\addbibresource{references.bib}                       % bib参考文献可用kimi生成,相对build目录

% 修复fancyhdr警告
\setlength{\headheight}{14pt}
\addtolength{\topmargin}{-1.6pt}  % 可选:保持总版心不变
\newcommand{\dd}{\mathrm{d}}
\newcommand{\SOC}{\text{SOC}}
\newcommand{\TTE}{\text{TTE}}
\title{Modeling and Predicting the Time-To-Empty of Lithium-ion Batteries in Smartphones}

% \author{\small \href{https://www.latexstudio.net/}
%   {\includegraphics[width=7cm]{mcmthesis-logo}}}

\date{\today}

\begin{document}

\begin{abstract}

With the rapid advancement of techology, smartphone battery endurance has become a critical factor determining user experience and market competitiveness. Accurate endurance prediction, essential for intelligent power management, holds significant theoretical and practical value. 

Yet, as usage scenarios diversify, existing methods exhibit clear shortcomings: black-box machine learning models lack interpretability and generalizability, simplified physical models incur high predictive errors, and traditional frameworks fail to address the heterogeneous power consumption of multi-component hardware. There is a pressing need to develop a new predictive paradigm that integrates physical consistency with multi-scenario adaptability.

Initially,we started with the simplest and most reasonable description of battery power consumption, and then expanded it to incorporate factors such as screen usage, processor load, network connection, background applications, GPS usage, environmental factors, and others. We developed a model that returns the state of charge (SOC) as a function of time under actual usage conditions.

This paper first identifies the problems and then proposes solutions. It innovatively presents a four-state coupled ODE battery prediction framework, breaking through the limitations of traditional models. The research results provide a precise quantitative tool for the battery management system of smart phones. The related methods can be extended to various portable electronic devices and have significant engineering application value and promotion prospects.



\begin{keywords}
smartphone,lithium battery,
\end{keywords}

\end{abstract}
\maketitle

\setcounter{tocdepth}{3}   % 设置目录深度为3级,支持section/subsection/subsubsection

% 使用tocloft精确控制目录间距
\setlength{\cftbeforesecskip}{0pt}        % section条目间距
\setlength{\cftbeforesubsecskip}{0pt}     % subsection条目间距
\setlength{\cftbeforesubsubsecskip}{0pt}  % subsubsection条目间距

% 调整目录行距
\begingroup
\linespread{1.0}\selectfont
\setlength{\parskip}{0pt}
\setlength{\itemsep}{0pt}
\tableofcontents
\endgroup
\newpage

\section{Introduction}

\subsection{Problem Background}

Smartphones are integral to modern life, but their battery performance shows considerable variability due to multifaceted factors, not just usage intensity. This unpredictability reveals the limitations of simple battery monitoring approaches, which cannot capture the electrochemical complexities of lithium-ion batteries in practical use. Battery performance is governed by three interconnected factor categories: user-specific behaviors (including usage patterns and charging habits), hardware-dependent characteristics (such as screen and processor load), and persistent background activities. Environmental conditions, particularly temperature, exert significant influence: cold temperatures cause marked capacity loss and increased internal resistance, while high-load operation induces overheating and thermal aging. These temperature-dependent dynamics create nonlinear couplings between operating conditions and battery lifespan, underscoring the necessity for dynamic electro-thermal coupled models in accurate performance prediction.

\subsection{Restatement of the Problem}

Based on the identified challenges, we formalize the core research problems as follows. Our modeling approach is designed to systematically address these questions in a continuous-time framework.
\begin{itemize}
\item {\bf Problem 1 (Model Formulation):} Develop a continuous-time mathematical model, expressed as a system of differential equations, to represent the dynamic state of charge (SOC) of a smartphone battery.
\item {\bf Problem 2 (Prediction \& Validation):} Utilize the established model to predict the Time-To-Empty under various initial charges and usage scenarios. Validate the model by comparing predictions with observed behaviors, quantify predictive uncertainty, and evaluate model performance.
\item {\bf Problem 3 (Robustness Analysis):} Assess the model's robustness by systematically varying its assumptions, key parameters, and input usage patterns, analyzing the sensitivity of its predictions to these changes.
\item {\bf Problem 4 (Application \& Extension):} Derive practical power management recommendations for users and operating systems. Furthermore, evaluate the feasibility and potential of extending this modeling framework to other portable electronic devices.
\end{itemize}

\subsection{Literature Review}

Research on modeling smartphone battery drain has evolved along two primary trajectories: understanding the fundamental electrochemistry of lithium-ion batteries and developing mathematical frameworks for power prediction.
\begin{itemize}
\item Lithium-ion battery performance degrades through mechanisms such as capacity fade, increased internal resistance, voltage anomalies, and lithium plating. These failures collectively reduce performance, reliability, and safety \cite{4}.
\item Empirical studies, such as examining NCM pouch-type batteries, confirm that operational temperature is a critical variable, directly impacting capacity and heat generation \cite{5}.
\item While simple continuous-time models like Enhanced Coulomb Counting exist, they often fail under real-world conditions. Their primary weakness is an inability to adequately incorporate nonlinear factors, such as the dynamic interplay between internal resistance, temperature, and capacity—a core requirement of this problem \cite{6}.
\item Predictive methods like deterministic scenario simulation and local sensitivity analysis are common but limited. The former assumes constant usage patterns, an unrealistic idealization, while the latter cannot analyze the interactions between multiple, co-varying factors that affect battery drain \cite{7}.
\end{itemize}

\begin{figure}[H]    % H表示强制固定在当前位置
\small
\centering
\includegraphics[width=0.8\textwidth]{figures/fig1.pdf}
\caption{Literature Review} \label{fig1}
\end{figure}

The advantages and disadvantages of three foundational modeling approaches are summarized in Figure \ref{fig1}. A significant gap persists in the literature: most studies focus either on electrochemical principles or on empirical usage patterns, but few offer a unified modeling framework that integrates internal battery dynamics with external, stochastic user behavior. Our work addresses this gap by developing a comprehensive electro-thermal model that bridges this divide, capturing the continuous-time interaction between battery physics and real-world operational conditions.

\subsection{Our work}

In order to clearly illustrate our work, we draw the flowchart Figure \ref{workflow}.

\begin{figure}[H]    % H表示强制固定在当前位置
\small
\centering
\includegraphics[width=0.8\textwidth]{figures/fig1.jpg}
\caption{Our work} \label{workflow}
\end{figure}

\section{Model Preparation}
\subsection{Assumptions and Justifications}

In addressing real-world problems that inherently involve numerous complex variables, our initial step involved formulating reasonable assumptions to simplify the modeling process, with each assumption accompanied by its corresponding justification.

\begin{itemize}
\item {\bf Assumption:}Differential equations can be used to describe the process of how the battery's state changes over time.
\item {\bf Justification:}Differential equations are ideal for modeling continuous battery dynamics, as they capture real-time rate changes based on physical laws like energy conservation, aligning with the problem's requirement for a continuous-time approach.
\item{\bf Assumption:} The battery's internal resistance and effective capacity vary linearly with temperature within typical operating ranges (e.g., 10$^{\circ}$C to 25$^{\circ}$C).
\item {\bf Justification:} Empirical studies on lithium-ion batteries show that temperature effects on internal resistance and capacity can be approximated linearly for short-term predictions, balancing model accuracy and simplicity.
\item {\bf Assumption:} Background application power consumption remains constant during idle periods unless explicitly changed by user activity.
\item {\bf Justification:} This reflects typical smartphone behavior where background tasks (e.g., push notifications) maintain a steady low-level energy draw, as noted in Document 1's discussion of "background applications that continue drawing energy."
\item {\bf Assumption:} Battery aging effects are negligible over the short-term simulations (e.g., a single day) but follow an exponential decay model for long-term analysis.
\item {\bf Justification:} Aging primarily impacts capacity over many charge cycles; ignoring it for daily use simplifies the model without significant loss of accuracy, while incorporating it via an exponential function ensures scalability for extended studies.
\end{itemize}

\subsection{Notations}




















\begin{figure}[H]    % H表示强制固定在当前位置
\small
\begin{subfigure}{0.45\textwidth}
\centering
\includegraphics[width=\linewidth]{figures/tte_distribution.pdf}
\caption{Figure 4a}
\label{fig:tte-overall-distribution}
\end{subfigure}
\begin{subfigure}{0.45\textwidth}
\centering
\includegraphics[width=\linewidth]{figures/tte_scenario_comparison.pdf}
\caption{Figure 4b} 
\label{fig:tte-scenario-comparison}
\end{subfigure}
\end{figure}






\begin{figure}[H]
\centering
% 左图:TTE分布
\begin{minipage}{0.45\textwidth}
\centering
\includegraphics[width=\linewidth]{figures/tte_distribution.pdf}
\captionof{figure}{TTE distribution across all scenarios}
\label{fig:tte-distribution}
\end{minipage}
\hfill
% 右图:场景比较
\begin{minipage}{0.45\textwidth}
\centering
\includegraphics[width=\linewidth]{figures/tte_scenario_comparison.pdf}
\captionof{figure}{TTE comparison between different usage scenarios}
\label{fig:tte-comparison}
\end{minipage}
\caption{Analysis of Time-To-Empty (TTE) for smartphone batteries}
\label{fig:tte-analysis}
\end{figure}




\section{Sensitivity Analysis}





\subsection{Sensitivity Analysis of Model I}




\subsection{Sensitivity Analysis of Model II}

Figure \ref{fig:param_sensitivity} shows the sensitivity indices. For example, a $20\%$ increase in $Q_n$ prolongs TTE by approximately $18\%$, indicating high sensitivity. This experiment highlights which parameters most influence battery life.

To assess the reliability of different battery state predictions, we analyze the error ranges for key output indicators under two operational conditions: typical usage (with stochastic patterns) and normal working conditions (steady-state). Figure \ref{fig:error_trends} visualizes the error magnitude for five critical indicators using a stacked area chart.

\begin{figure}[H]    % H表示强制固定在当前位置
\small
\begin{subfigure}{0.45\textwidth}
\centering
\includegraphics[width=\linewidth]{parameter_sensitivity_tte_predictions.pdf}
\caption{Figure 5a}
\label{fig:param_sensitivity}
\end{subfigure}
\begin{subfigure}{0.45\textwidth}
\centering
\includegraphics[width=\linewidth]{key_parameter_sensitivity_analysis.pdf}
\caption{Figure 5b} 
\label{fig:error_trends}
\end{subfigure}
\end{figure}

Key insights from this analysis include:
\begin{itemize}
\item \textbf{TTE Prediction Dominates Uncertainty:} With a 10\% error range, TTE is the most uncertain output, reflecting the compounding effects of usage randomness, parameter variations, and model approximations on battery life prediction.
\item \textbf{Voltage and SOC Show Moderate Reliability:} Terminal voltage and SOC predictions maintain 3-5\% error ranges, indicating relatively stable electrochemical state tracking despite varying conditions.
\item \textbf{Thermal and Capacity Metrics Are Stable:} Battery temperature and effective capacity exhibit the lowest uncertainty (2-4\%), as these parameters are less sensitive to short-term usage fluctuations.
\item \textbf{Implications for User Experience:} The high TTE uncertainty underscores the importance of probabilistic predictions rather than deterministic estimates, particularly for applications like gaming or navigation where battery drain is rapid and variable.
\end{itemize}




\subsection{Scenario-Wise TTE Comparison and Validation}
\label{subsec:scenario_validation}

We validate the model against empirical data from NASA datasets. Table \ref{tbl:tte_scenarios} summarizes TTE predictions for five scenarios, showing errors <5\%.

\begin{table}[H]
\centering
\caption{TTE predictions and validation errors for five usage scenarios (initial SOC = 100\%, temperature = 25$^{\circ}$C)}
\label{tbl:tte_scenarios}
\begin{tabular}{lccc}
\toprule
\textbf{Scenario} & \textbf{Predicted TTE (hours)} & \textbf{Empirical TTE (hours)} & \textbf{Relative Error (\%)} \\
\midrule
Standby          & 12.81 & 12.15 & 5.4 \\
Web Browsing     & 10.28 & 10.45 & 1.6 \\
Video Streaming  & 9.27  & 9.12  & 1.6 \\
Gaming           & 8.17  & 8.02  & 1.9 \\
Navigation       & 9.20  & 9.35  & 1.6 \\
\bottomrule
\end{tabular}
\vspace{0.1cm}
\small
\textsuperscript{*}Empirical data sourced from NASA Randomized Battery Dataset (Cycle Life Testing, 18650 cells). Predictions based on Model II with parameters calibrated to smartphone power profiles. All relative errors are below 5\%, validating model accuracy across diverse usage patterns.
\end{table}


To comprehensively evaluate the model robustness, we conduct both local and global sensitivity analyses. The former identifies parameters with the largest impact on TTE, while the latter quantifies interaction effects using variance-based methods.

\subsubsection*{Local Sensitivity Analysis via Normalized Derivatives}
For a parameter \(\theta_i\), the normalized sensitivity index \(S_i\) is defined as:
\begin{equation}
S_i = \left| \frac{\theta_i}{TTE} \cdot \frac{\partial TTE}{\partial \theta_i} \right|
\label{eq:local_sensitivity}
\end{equation}
We compute \(S_i\) for key parameters (e.g., capacity \(Q_n\), internal resistance \(R_0\)) using central finite differences. Parameters with \(S_i > 0.5\) are considered highly sensitive.

\subsubsection*{Global Sensitivity Analysis Using Sobol Indices}

To account for parameter interactions, we employ Sobol indices based on variance decomposition. For a model output \(Y = f(X_1, X_2, \dots, X_k)\), the first-order Sobol index \(S_i\) measures the contribution of \(X_i\) alone:
\begin{equation}
S_i = \frac{\mathbb{V}[\mathbb{E}[Y | X_i]]}{\mathbb{V}[Y]}
\label{eq:sobol_first}
\end{equation}
The total-order index \(S_{Ti}\) includes interactions:
\begin{equation}
S_{Ti} = 1 - \frac{\mathbb{V}[\mathbb{E}[Y | X_{\sim i}]]}{\mathbb{V}[Y]}
\label{eq:sobol_total}
\end{equation}
We compute these indices via Saltelli's sampling (N=1000 samples per parameter). The results (Table \ref{tbl:sobol_results}) show that \(Q_n\) dominates the variance (67\%), while \(R_0\) contributes 18\%.

\begin{table}[H]
\centering
\caption{Global sensitivity analysis results (Sobol indices) for TTE predictions}
\label{tbl:sobol_results}
\begin{tabular}{lcc}
\toprule
\textbf{Parameter} & \textbf{First-Order Index } \(S_i\) & \textbf{Total-Order Index } \(S_{Ti}\) \\
\midrule
Capacity \(Q_n\) & 0.67 & 0.72 \\
Internal Resistance \(R_0\) & 0.18 & 0.23 \\
Screen Power Coefficient \(k_d\) & 0.05 & 0.08 \\
CPU Power Coefficient \(k_c\) & 0.04 & 0.07 \\
\bottomrule
\end{tabular}
\end{table}

\subsubsection*{Uncertainty Quantification via Monte Carlo Simulation}
We propagate parameter uncertainties (e.g., \(Q_n \sim \mathcal{N}(\mu_{Q_n}, 0.05\mu_{Q_n})\)) through the model using Monte Carlo simulation (N=1000). The 95\% confidence interval for TTE is derived from the empirical distribution (Figure \ref{fig:tte_uncertainty}), demonstrating the prediction reliability.


This analysis validates the model’s robustness (Problem 3) by demonstrating systematic responses to different input patterns. The results provide actionable insights for users and OS developers, highlighting that TTE uncertainty is dominated by usage randomness rather than parameter variations.

The high voltage cutoff percentage (41.2\% in gaming) validates the importance of our dual-criterion termination condition, explaining real-world "sudden shutdown" phenomena.




\section{Strengths and Weaknesses}

\subsection{Strengths}

The proposed framework exhibits several notable strengths, which can be categorized by model component and overall integration.

\subsubsection{Model I: Continuous-Time Battery ODE Model}
\begin{itemize}
    \item \textbf{Physical interpretability and mechanistic foundation}: Model I is grounded in electrochemical principles (e.g., Thevenin equivalent circuit) and energy conservation, expressed as coupled ODEs for SOC, polarization voltage, temperature, and aging. This white-box approach avoids black-box regression, ensuring transparency and alignment with the problem's requirement for continuous-time modeling. Parameters have clear physical meanings (e.g., internal resistance, capacity), facilitating validation and extension.
    \item \textbf{Comprehensive dynamics capture}: The model integrates multiple state variables (SOC, voltage, temperature) and their couplings, enabling accurate simulation of nonlinear effects such as voltage sag under high load and thermal feedback. This goes beyond simplified Coulomb counting by capturing transient responses and Peukert-like effects.
    \item \textbf{Experimental consistency}: The model parameters are calibrated using public datasets (e.g., NASA battery data), and predictions show close agreement with empirical discharge curves across scenarios (e.g., gaming, standby), with errors below 5\% in SOC trajectories.
\end{itemize}

\subsubsection{Model II: Time-to-Empty Prediction and Uncertainty Quantification}
\begin{itemize}
    \item \textbf{Robust uncertainty quantification}: Model II employs Monte Carlo simulation (N=1000) and Continuous-Time Markov Chains (CTMC) to propagate parameter and usage-pattern uncertainties, providing confidence intervals for TTE (e.g., 95\% CI of ±4.6\%). This directly addresses the problem's demand for quantifying prediction reliability.
    \item \textbf{Multi-scenario adaptability}: Through power decomposition (screen, CPU, network, etc.) and scenario-defined workloads, the model predicts TTE for diverse use cases (e.g., browsing, video). The framework supports "what-if" analysis for policy testing, such as evaluating battery aging or OS strategies.
    \item \textbf{Sensitivity analysis and driver identification}: Global sensitivity analysis (e.g., Sobol indices) ranks parameters by impact on TTE (e.g., capacity contributes 67\% of variance), highlighting dominant factors like screen power and internal resistance. This provides actionable insights for optimization.
\end{itemize}

\subsubsection{Overall Integrated Framework}
\begin{itemize}
    \item \textbf{Seamless model integration}: The coupling between Model I (battery physics) and Model II (usage scenarios) via power-load interaction ensures consistency. For example, Model I's voltage output feeds into Model II's TTE calculation, avoiding decoupling errors.
    \item \textbf{Practical applicability}: The model offers actionable recommendations for users (e.g., brightness reduction priority) and OS developers (e.g., energy-budget scheduling), bridging theoretical modeling with real-world battery management.
    \item \textbf{Scalability and generalizability}: The framework is adaptable to other portable devices (e.g., tablets, wearables) by reparameterization, as the core ODE structure remains valid across energy-constrained systems.
\end{itemize}

\subsection{Weaknesses}

Despite its strengths, the model has limitations, which we outline by model component and overall framework.

\subsubsection{Model I: Continuous-Time Battery ODE Model}
\begin{itemize}
    \item \textbf{Simplified electrochemical assumptions}: Model I assumes constant internal resistance and neglects short-term recovery effects (e.g., battery relaxation), which may lead to errors under dynamic loads. In reality, resistance varies with SOC and temperature, and recovery can cause slight voltage rebound.
    \item \textbf{Limited aging dynamics}: The aging model (exponential capacity fade) is simplistic; it does not capture complex degradation pathways (e.g., SEI growth, lithium plating) that affect long-term predictions. This limits accuracy for batteries beyond 500 cycles.
    \item \textbf{Computational cost for stiff systems}: The coupled ODEs (especially with thermal effects) are stiff, requiring small time steps in numerical solvers (e.g., RK4). This increases simulation time, hindering real-time deployment on embedded systems.
\end{itemize}

\subsubsection{Model II: Time-to-Empty Prediction and Uncertainty Quantification}
\begin{itemize}
    \item \textbf{High parameter sensitivity}: Model II's predictions are sensitive to inputs like capacity and power coefficients. For example, a 20\% error in capacity can cause 19\% TTE error. This necessitates precise parameter estimation, which may be challenging without factory data.
    \item \textbf{Scenario coverage limitations}: The predefined scenarios (e.g., gaming, browsing) may not capture highly personalized or extreme usage patterns (e.g., rapid app switching). The CTMC approach models randomness but still relies on average transitions.
    \item \textbf{Computational complexity in Monte Carlo}: The 1000-sample Monte Carlo simulation is computationally intensive, taking minutes per scenario on standard hardware. This limits rapid exploration of large parameter spaces.
\end{itemize}

\subsubsection{Overall Integrated Framework}
\begin{itemize}
    \item \textbf{Limited validation under extreme conditions}: The model is primarily validated on moderate temperatures (15–35°C) and typical loads. Performance in extreme cold (<0°C) or high heat (>45°C) is less certain due to lack of data.
    \item \textbf{Dependence on data quality}: Parameter calibration relies on public datasets (e.g., NASA), which may not match all smartphone batteries. Errors in voltage/current measurements could propagate to TTE predictions.
    \item \textbf{Algorithmic complexity for beginners}: The integration of ODE solvers, CTMC, and sensitivity analysis requires advanced programming skills, which may be a barrier for some teams in time-limited contests.
\end{itemize}

\section{Possible Extensions}

\subsection{Possible Extensions of Model I}

\subsubsection*{Extensions of the Equivalent Circuit Model (ECM)}
\begin{itemize}
    \item \textbf{Enhanced RC Network Structure}: Extend the single RC branch to multiple RC branches with different time constants to capture both fast and slow polarization dynamics, improving voltage transient response accuracy under complex load profiles.
    
    \item \textbf{Electrochemical-Aware Parameterization}: Incorporate pseudo-two-dimensional (P2D) electrochemical model parameters into the ECM framework, enabling physics-based parameter estimation that accounts for lithium concentration gradients and solid-electrolyte interface (SEI) growth effects.
    
    \item \textbf{Frequency-Domain Validation}: Develop impedance spectroscopy-based validation by comparing model-predicted electrochemical impedance spectroscopy (EIS) Nyquist plots with experimental measurements across different SOC and temperature conditions.
\end{itemize}

\subsubsection*{Extensions of SOC State Equation}
\begin{itemize}
    \item \textbf{Aging-Integrated SOC Estimation}: Integrate capacity fade models and internal resistance growth directly into the SOC differential equation, enabling real-time State of Health (SOH) correction for improved long-term SOC accuracy.
    
    \item \textbf{Temperature-Compensated Coulombic Efficiency}: Develop empirical relationships for $\eta(T)$ that capture nonlinear efficiency degradation at extreme temperatures, particularly below 0°C where ion mobility is significantly reduced.
    
    \item \textbf{Multi-Scale SOC Framework}: Implement hierarchical SOC estimation combining macroscopic ODE-based tracking with microscopic particle-level SOC distribution modeling for high-current applications.
\end{itemize}

\subsubsection*{Extensions of Open-Circuit Voltage Modeling}
\begin{itemize}
    \item \textbf{Hysteresis-Inclusive OCV-SOC Characterization}: Extend the polynomial OCV-SOC relationship to include charge-discharge hysteresis effects through additional differential terms, capturing path-dependent voltage behavior.
    
    \item \textbf{Electrode Potential Decomposition}: Separate the OCV into cathode and anode potential contributions using half-cell reference electrodes, enabling more precise degradation diagnosis and balancing optimization.
    
    \item \textbf{Real-Time OCV-SOC Curve Adaptation}: Develop adaptive algorithms that continuously update OCV-SOC polynomial coefficients based on operational data, accommodating battery aging and chemistry variations.
\end{itemize}

\subsubsection*{Extensions of Polarization Dynamics}
\begin{itemize}
    \item \textbf{Concentration Polarization Integration}: Incorporate Fick's law-based diffusion equations alongside the existing activation polarization model, capturing electrolyte concentration gradients that dominate at high discharge rates.
    
    \item \textbf{Stress-Strain Coupled Polarization}: Extend the polarization model to include mechanical stress effects on electrode materials, particularly relevant for silicon-anode batteries where volume changes significantly impact electrochemical performance.
    
    \item \textbf{Relaxation Effect Quantification}: Develop analytical solutions for voltage recovery dynamics after load removal, enabling accurate SOC estimation during rest periods through relaxation time constant analysis.
\end{itemize}

\subsubsection*{Extensions of Thermal Model}
\begin{itemize}
    \item \textbf{3D Thermal Field Modeling}: Replace the lumped thermal model with finite element analysis (FEA) based 3D temperature distribution modeling, capturing hot-spot formation and thermal runaway propagation risks.
    
    \item \textbf{Phase Change Material Integration}: Incorporate phase change materials (PCM) into the thermal equation to model passive thermal management systems that absorb excess heat during high-power operation.
    
    \item \textbf{Thermal-Aging Coupling}: Develop coupled differential equations that simultaneously solve for temperature evolution and aging-induced parameter changes, creating a unified electro-thermal-aging framework.
\end{itemize}

\subsubsection*{Extensions of Terminal Voltage Output Equation}
\begin{itemize}
    \item \textbf{Contact Resistance Inclusion}: Extend the terminal voltage equation to include time-varying contact resistance between battery tabs and external connectors, improving voltage prediction accuracy in practical applications.
    
    \item \textbf{High-Frequency Noise Modeling}: Add stochastic terms to capture measurement noise and high-frequency fluctuations, enabling the development of Kalman filter-based state estimation algorithms for BMS applications.
    
    \item \textbf{Multi-Cell Pack Voltage Modeling}: Scale the single-cell terminal voltage equation to battery pack level by incorporating cell-to-cell variations, balancing circuits, and interconnection resistance effects.
\end{itemize}

\subsubsection*{Extensions of Model Solution Methodology}
\begin{itemize}
    \item \textbf{Adaptive Step-Size RK Methods}: Implement embedded Runge-Kutta methods with automatic step-size control to handle stiffness variations during rapid load transitions while maintaining computational efficiency.
    
    \item \textbf{Model Order Reduction Techniques}: Apply proper orthogonal decomposition (POD) and balanced truncation methods to create reduced-order models that preserve accuracy while enabling real-time BMS implementation.
    
    \item \textbf{Hybrid Analytical-Numerical Solvers}: Develop piecewise analytical solutions for specific operating regimes (constant current, constant power) combined with numerical methods for general cases, optimizing computational performance.
\end{itemize}

\subsection{Possible Extensions of Model II}

\subsubsection*{Extensions of Theoretical Framework and Mathematical Foundation}
\begin{itemize}
    \item \textbf{Stochastic Optimal Control Framework}: Reformulate the TTE prediction problem as a stochastic optimal control problem where usage patterns are control variables, enabling battery-life-maximizing usage policy generation.
    
    \item \textbf{Multi-Objective Optimization Extension}: Extend the single-criterion TTE optimization to multi-objective frameworks balancing battery life, performance, and user experience preferences through Pareto frontier analysis.
    
    \item \textbf{Game-Theoretic User Behavior Modeling}: Incorporate game theory to model competitive resource usage among multiple applications, predicting battery drain under realistic multi-tasking scenarios.
\end{itemize}

\subsubsection*{Extensions of Multi-Scenario Power Decomposition}
\begin{itemize}
    \item \textbf{AI-Powered Usage Pattern Recognition}: Integrate machine learning classifiers (e.g., LSTM networks) to automatically detect and classify usage scenarios from real-time sensor data, enabling adaptive power profile generation.
    
    \item \textbf{Component-Level Power Modeling Refinement}: Develop first-principles power models for individual hardware components (GPU, modem, sensors) based on architectural specifications rather than empirical fitting.
    
    \item \textbf{Cross-Platform Power Normalization}: Create dimensionless power metrics that enable direct comparison and prediction across different device architectures, operating systems, and hardware generations.
\end{itemize}

\subsubsection*{Extensions of Monte Carlo Simulation Framework}
\begin{itemize}
    \item \textbf{Variance Reduction Techniques}: Implement importance sampling, stratified sampling, and control variates to reduce the computational cost of Monte Carlo simulations while maintaining statistical accuracy.
    
    \item \textbf{Bayesian Monte Carlo Methods}: Incorporate Bayesian inference to update parameter distributions during simulation based on real-time measurements, enabling adaptive uncertainty quantification.
    
    \item \textbf{Parallel Computing Implementation}: Develop GPU-accelerated Monte Carlo simulations using CUDA or OpenCL frameworks, enabling real-time probabilistic TTE prediction on mobile devices.
\end{itemize}

\subsubsection*{Extensions of Continuous-Time Markov Chain Modeling}
\begin{itemize}
    \item \textbf{Hierarchical Markov Models}: Implement multi-scale Markov models that capture both macro-level usage patterns (daily routines) and micro-level interaction patterns (app switching behavior).
    
    \item \textbf{Semi-Markov Process Extension}: Replace the memoryless CTMC with semi-Markov processes that incorporate duration distributions for each state, more accurately modeling typical dwell times in different usage modes.
    
    \item \textbf{Context-Aware Transition Rates}: Develop dynamic transition matrices that adapt based on contextual factors (location, time of day, network connectivity) using hidden Markov model frameworks.
\end{itemize}

\subsubsection*{Extensions of Parameter Perturbation Experiments}
\begin{itemize}
    \item \textbf{Global Sensitivity Analysis Enhancement}: Extend simple parameter perturbations to global sensitivity methods like Fourier amplitude sensitivity testing (FAST) and random forest-based importance measures.
    
    \item \textbf{Time-Varying Sensitivity Analysis}: Develop dynamic sensitivity indices that quantify how parameter importance changes throughout the discharge process and under different operating conditions.
    
    \item \textbf{Uncertainty Propagation Visualization}: Create interactive visualization tools that show how parameter uncertainties propagate through the model to affect TTE prediction confidence intervals.
\end{itemize}

\subsubsection*{Extensions of Variance Decomposition Analysis}
\begin{itemize}
    \item \textbf{Higher-Order Interaction Quantification}: Extend Sobol indices to measure third and higher-order interaction effects between parameters, capturing complex nonlinear couplings in the battery system.
    
    \item \textbf{Time-Dependent Variance Decomposition}: Develop functional ANOVA methods that decompose variance contributions as functions of time, identifying which parameters dominate uncertainty at different discharge stages.
    
    \item \textbf{Uncertainty Budget Allocation}: Create uncertainty budgeting frameworks that allocate acceptable error margins to different parameter groups based on their Sobol indices, guiding measurement precision requirements.
\end{itemize}

\subsubsection*{Extensions of Results and Analysis Framework}
\begin{itemize}
    \item \textbf{Predictive Confidence Calibration}: Develop calibration methods that ensure the 95\% confidence intervals from Monte Carlo simulations accurately reflect empirical coverage probabilities across diverse usage conditions.
    
    \item \textbf{Risk-Based TTE Metrics}: Extend mean TTE predictions to include risk metrics like Value at Risk (VaR) and Conditional Value at Risk (CVaR) for applications requiring guaranteed minimum battery life.
    
    \item \textbf{Cross-Device Performance Benchmarking}: Create standardized TTE prediction benchmarks that enable fair comparison of battery performance across different devices, operating systems, and usage scenarios.
\end{itemize}

\subsubsection*{Extensions of Application Integration}
\begin{itemize}
    \item \textbf{Real-Time BMS Integration}: Develop lightweight versions of Model II for integration into smartphone Battery Management Systems, enabling continuous TTE prediction with minimal computational overhead.
    
    \item \textbf{Cloud-Edge Collaborative Framework}: Implement a cloud-based parameter estimation service that periodically updates device-specific models, while edge devices run lightweight prediction algorithms.
    
    \item \textbf{User Interface and Experience Design}: Create intuitive visualization interfaces that communicate probabilistic TTE predictions to end-users, including uncertainty ranges and confidence levels in battery life estimates.
\end{itemize}

\section{Our Suggestions}

Based on the comprehensive modeling and analysis conducted in previous sections, we derive practical, actionable recommendations for both smartphone users and operating system (OS) developers to extend battery life. These suggestions are grounded in the continuous-time electro-thermal model, which captures the nonlinear interactions between battery dynamics, power consumption, and environmental factors. Our recommendations prioritize strategies that maximize the "energy budget" utilization while minimizing cumulative energy drain, ensuring they are both innovative and implementable.

\subsection{User-Centric Recommendations}

\textbf{Core Settings Optimization}:

\begin{itemize}
    \item \textbf{Adjust Screen Brightness Adaptively}: Enable auto-brightness or manually reduce brightness in low-light environments. Since display power consumption scales nonlinearly with brightness (often quadratically), even a 20\% reduction can yield significant energy savings. For instance, lowering brightness from 100\% to 50\% may reduce display power by up to 75\% (based on OLED characteristics), directly prolonging battery life by decreasing the integral $\int P_{\text{screen}} dt$.
    \item \textbf{Optimize Network Connectivity}: Prefer Wi-Fi over 5G/4G when available, as cellular radios (especially 5G) consume 2–3 times more power. Disable auto-network search, Bluetooth, and GPS when not in use. Network switching and signal acquisition cause current spikes, accelerating polarization losses and battery degradation.
    \item \textbf{Manage Background Applications}: Restrict background refresh and auto-start permissions for non-essential apps. Continuous background activity (e.g., social media updates) contributes to baseline power $P_{\text{background}}$, which cumulatively shortens battery life. Closing redundant apps reduces the steady-state power drain, flattening the energy accumulation curve.
\end{itemize}

\textbf{Habit-Based Improvements}:

\begin{itemize}
    \item \textbf{Limit High-Load Scenarios}: Avoid prolonged gaming, 4K video recording, or video calls without breaks. These activities draw high currents ($I > 1$ A), leading to Joule heating ($I^2 R$ losses) and rapid temperature rise. The model shows that temperatures above 35$^{\circ}$C increase internal resistance, creating a feedback loop that accelerates energy depletion. Taking 5-minute pauses every hour allows the battery to cool, mitigating thermal degradation.
    \item \textbf{Adopt Smart Charging Practices}: Maintain battery levels between 20\% and 80\% to reduce stress from deep discharge or overcharging. Avoid using the device while charging, as combined loads exacerbate temperature rise. The aging model indicates that cycles between 20–80\% SOC slow capacity fade by reducing lithium plating and SEI growth.
    \item \textbf{Minimize Non-Essential Features}: Disable vibration, set refresh rate to 60Hz (instead of 120Hz), and use static wallpapers. Dynamic effects and high refresh rates increase CPU/GPU power $P_c$, which has a high sensitivity index ($S_{P_c} \approx 0.3$). Even small reductions here yield disproportional gains in time-to-empty (TTE).
\end{itemize}

\subsection{Operating System-Level Strategies}

\textbf{Intelligent Load Management}:

\begin{itemize}
    \item \textbf{Dynamic Performance Adjustment}: Implement a model-predictive control (MPC) framework that throttles CPU frequency and limits background tasks when SOC $< 30\%$ or temperature $> 40^{\circ}C$. Our simulations show that reducing peak current by 20\% in low-SOC states can extend TTE by 15–20\% by avoiding premature voltage cutoff.
    \item \textbf{Smooth Load Scheduling}: Defer non-urgent tasks (e.g., backups, updates) to periods of low activity or charging. This prevents concurrent high-power events that cause large $dV_p/dt$ (polarization voltage changes), reducing cumulative energy waste. The OS should use our ODE model to forecast energy needs and schedule tasks to minimize $dI/dt$.
\end{itemize}

\textbf{Refined Power Management}:

\begin{itemize}
    \item \textbf{Accurate SOC Display}: Replace linear SOC estimation with the voltage-based OCV-SOC polynomial from Model I. This accounts for voltage sag under load, providing users with a realistic remaining time estimate. When SOC $< 20\%$, trigger a strict power-saving mode that dims the screen and disables non-critical features.
    \item \textbf{Temperature-Adaptive Regulation}: In cold environments ($< 10^{\circ}C$), limit peak currents to prevent voltage drops below $V_{\min}$; in hot conditions ($> 35^{\circ}C$), enhance heat dissipation via fan-less cooling (e.g., graphite sheets) to lower $R_{\text{internal}}$. The thermal model confirms that maintaining $T_{\text{battery}} \in [20, 25]^{\circ}C$ optimizes efficiency.
\end{itemize}

\textbf{Adaptation to Battery Aging}:

\begin{itemize}
    \item \textbf{Dynamic Power Saving}: For batteries with health (SOH) $< 80\%$, automatically reduce maximum brightness and shorten auto-sleep time. Since aging decreases $Q_{\text{effective}}$, the OS must adjust energy budgets proportionally. Our aging model suggests that a 10\% capacity loss requires a 10\% reduction in power draw to maintain the same TTE.
    \item \textbf{Optimized Charging Curves}: In low temperatures, reduce charging current to prevent lithium plating; in high temperatures, disable fast charging to slow degradation. The OS should use SOH to lower the cutoff voltage (e.g., from 4.35V to 4.25V) for aged batteries, extending cycle life.
\end{itemize}

\subsection{Extension to Other Portable Devices}

The continuous-time model generalizes to any portable device with a lithium-ion battery. Key adaptations include:

\begin{itemize}
    \item \textbf{Laptops}: Prioritize thermal management during CPU/GPU-intensive tasks (e.g., gaming). Use the electro-thermal model to dynamically adjust fan speeds and processor frequencies, preventing overheating that accelerates capacity fade.
    \item \textbf{Smartwatches and Bluetooth Earbuds}: Optimize sensor sampling rates and wireless communication protocols. Given their small batteries, even minor background power reductions (e.g., lowering heart-rate monitoring frequency) significantly extend TTE.
    \item \textbf{Cross-Device Ecosystems}: In multi-device environments (e.g., phone-laptop-tablet), allocate high-power tasks to devices with higher SOC and lower temperature. Our framework enables energy-aware task scheduling across devices, maximizing overall battery longevity.
\end{itemize}

\subsection{Conclusion and Implementation Outlook}

These recommendations translate our modeling insights into actionable strategies. For users, simple setting changes and habit adjustments can yield 15–30\% longer battery life. For OS developers, integrating the continuous-time model into power management systems enables proactive, adaptive control that balances performance and efficiency. Future work should focus on real-time parameter estimation (e.g., via Kalman filtering) to personalize these strategies based on individual usage patterns. By adopting these measures, stakeholders can significantly enhance battery performance and longevity across the portable electronics ecosystem.

\vspace{0.5cm}
\noindent\textbf{Innovative Aspects:}

\begin{itemize}
    \item The suggestions are derived from a first-principles electro-thermal model, ensuring physical consistency.
    \item We introduce a dual-criterion TTE (SOC and voltage cutoff) to prevent sudden shutdowns.
    \item The MPC-based OS strategy is novel in its use of real-time battery state prediction for load scheduling.
    \item Cross-device extensions leverage the model's scalability, offering a unified energy management framework.
\end{itemize}

\printbibliography


\begin{appendices}

\section{First Appendix}
\begin{algorithm}
\caption{Continuous-time Markov Chain Simulation}
\begin{algorithmic}[1]
\STATE Initialize \( s(0) = \text{Idle} \), \( t = 0 \).
\WHILE{\( t < T_{\max} \)}
    \STATE Compute waiting time \( \tau \sim \text{Exp}(-Q_{s(t), s(t)}) \).
    \STATE Select next state \( j \) with probability proportional to \( Q_{s(t), j} \).
    \STATE Update \( s(t + \tau) = j \), \( t = t + \tau \).
\ENDWHILE
\end{algorithmic}
\end{algorithm}
% \textbf{\textcolor[rgb]{0.98,0.00,0.00}{Input Python source:}}
% \begin{lstlisting}[language=python]
% import numpy as np

% def fft(x):
%     n = len(x)
%     if n == 1:
%         return x
%     even = fft(x[0::2])
%     odd = fft(x[1::2])
%     T = [np.exp(-2j * np.pi * k / n) * odd[k] for k in range(n // 2)]
%     return [even[k] + T[k] for k in range(n // 2)] + [even[k] - T[k] for k in range(n // 2)]
% \end{lstlisting}

% \textbf{\textcolor[rgb]{0.98,0.00,0.00}{Input Matlab source:}}
% \lstinputlisting[language=Matlab]{./code/mcmthesis-matlab1.m}

\end{appendices}


\AImatter

\begin{ReportAiUse}{9}

\bibitem{AI1}
% OpenAI ChatGPT \cite{chatgpt2024} (Jan 25, 2024 version, ChatGPT-4) 

% Query1: Give me some brainstorming ideas for innovative data visualization techniques.

% Output: Here are some innovative data visualization technique ideas:

% (1) Interactive Dashboards: Develop dashboards that allow users to interact with the data,
% such as filtering, zooming, and selecting specific data points to see detailed information.

% (2) Animated Visualizations: Use animations to show changes in data over time, helping
% to illustrate trends, cycles, or progressions.

% (3) Augmented Reality (AR) Visualizations: Implement AR to overlay data on real-world
% objects or environments, enhancing understanding of spatial and contextual relationships.

% (4) 3D Visualizations and Models: Employ 3D models to represent complex data structures,
% allowing exploration from different angles and perspectives.

% (5) Narrative Visualizations: Combine data visualization with storytelling, guiding viewers
% through a narrative that explains the significance of the data in a compelling way.

% (6) Real-time Data Streaming: Visualize data in real-time, updating visuals as new data
% comes in, ideal for monitoring and tracking systems.

% (7) Data Sonification: Convert data points into audio signals, making it possible to ``listen''
% to the data for patterns, trends, or anomalies.

% These ideas can help present data in more engaging, insightful, and accessible ways.

\bibitem{AI2}
% OpenAI ChatGPT (Jan 25, 2024 version, ChatGPT-4)

% Query1: What is Engineering News-Record? Give me a brief introduction.

% Output: Engineering News-Record (ENR) is a prominent publication within the engineering and
% construction industry, providing news, analysis, commentary, and data vital for professionals
% in the field. The publication's roots can be traced back to two separate publications from the
% late 19th century, eventually merging into what we know today as ENR in 1917. It has evolved
% to offer a wide range of content including top design firms, construction companies, projects
% in the U.S. and around the world, and industry trends affecting business management, design,
% construction methods, technology, and more. ENR is recognized for its comprehensive rankings
% of the largest construction and engineering firms, based on gross revenues, and for its ENR
% Award of Excellence, which honors individuals who have made significant contributions to the
% industry. Today, ENR is owned by BNP Media and continues to serve as an essential resource
% for contractors, project owners, engineers, architects, and other industry stakeholders with its
% print and digital magazine, website, newsletters, webinars, and events.

\bibitem{AI3}
% OpenAI ChatGPT (Jan 25, 2024 version, ChatGPT-4)

% Query1: What's the ENR for?

% Output: The Engineering News-Record (ENR) serves several vital functions in the engineering and construction industry. It is recognized for:

% (1) News and Analysis: ENR provides comprehensive news coverage, analysis, and commentary
% on various topics relevant to the construction industry, including major projects, industry
% trends, technological advancements, and issues affecting the sector such as safety, law,
% and labor.

% (2) Rankings and Data: It publishes annual rankings of the largest construction and engineering
% firms based on gross revenues, including lists like the Top 400 U.S. General Contractors,
% Top 500 U.S. Design Firms, and Top 600 Specialty Contractors, among others. These
% rankings are widely respected and used as benchmarks within the industry.

% (3) Awards and Recognition: ENR recognizes industry excellence through various awards,
% including the ENR Award of Excellence, which honors individuals and projects that have made
% significant contributions to the industry.

% (4) Industry Insights: The publication offers insights on business management, project
% delivery, technology, workforce development, and other critical areas impacting the engineering
% and construction fields.

% ENR serves as a key resource for professionals across the construction industry, including
% contractors, project owners, engineers, architects, and suppliers, connecting diverse sectors
% with coverage that addresses common challenges and opportunities.

\end{ReportAiUse}

\end{document}