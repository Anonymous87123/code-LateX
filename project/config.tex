% ========== 基本包加载 ==========
\usepackage{amsmath, amsthm, amssymb}
\usepackage{graphicx}
\usepackage{geometry}
\usepackage[colorlinks, linkcolor=black]{hyperref}
\usepackage{bookmark}
\usepackage{tikz}
\usepackage[most]{tcolorbox}
\tcbuselibrary{theorems}
\usepackage{fontspec} % 添加字体支持

% 设置英文字体
\setmainfont{Times New Roman} % 英文主字体:Times New Roman
\setsansfont{Arial} % 英文无衬线字体:Arial
\setmonofont{Consolas} % 等宽字体:Consolas

% 设置中文字体
\setCJKmainfont{SimSun} % 中文主字体:宋体(正文)
\setCJKsansfont{SimHei} % 中文无衬线字体:黑体(标题)
\setCJKmonofont{KaiTi} % 中文等宽字体:楷体(特殊强调)

% ========== 页面设置 ==========
\geometry{
  top=25.4mm,
  bottom=25.4mm,
  left=20mm,
  right=20mm,
  headheight=2.17cm,
  headsep=4mm,
  footskip=12mm
}
\linespread{1.5} % 行间距

% ========== 数学字体优化 ==========
\usepackage{unicode-math} % 数学字体支持
\setmathfont{Latin Modern Math} % 数学字体:Latin Modern Math(简洁清晰)

% ========== 定理环境设置 ==========
% 使用楷体显示定理内容
\newtheoremstyle{chinese}% 名称
  {3pt}% 上方空白
  {3pt}% 下方空白
  {\kaishu}% 正文字体(楷体)
  {}% 缩进
  {\heiti\bfseries}% 标题字体(黑体加粗)
  {.}% 标题后标点
  { }% 标题后空白
  {}% 标题说明

% 应用定理样式
\theoremstyle{chinese}
\newtheorem{definition}{定义}[section]
\newtheorem{solution}{解}[section]
\newtheorem{axiom}{公理}[section]
\newtheorem{property}{性质}[section]
\newtheorem{proposition}{命题}[section]
\newtheorem{theorem}{定理}[section]
\newtheorem{lemma}{引理}[section]
\newtheorem{corollary}{推论}[section]
\newtheorem{example}{例}[section]
\newtheorem{remark}{注}[section]

% ========== 自定义命令 ==========
\newcommand{\R}{\mathbb{R}} % 实数集
\newcommand{\C}{\mathbb{C}} % 复数集
\newcommand{\Z}{\mathbb{Z}} % 整数集
\newcommand{\N}{\mathbb{N}} % 自然数集
\newcommand{\dif}{\mathrm{d}} % 微分符号

% ========== 图形路径设置 ==========
\graphicspath{{./flg/}} % 图片路径