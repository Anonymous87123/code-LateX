% !TEX root = main.tex
\documentclass[12pt,fontset=windows, a4paper, oneside,UTF8]{ctexbook}
% ========== 基本包加载 ==========
% 1. 基础宏包
\usepackage{geometry}
\usepackage{fontspec}
\usepackage{amsmath, amsthm, amssymb}
\usepackage{extarrows}
\allowdisplaybreaks
\usepackage{mathrsfs}
\usepackage{enumitem}
\usepackage{graphicx}
\usepackage{array}
\usepackage{ulem}
\usepackage{caption}
\usepackage{tocloft}
% 2. 图形和颜色宏包
\usepackage[dvipsnames]{xcolor}
\usepackage{tikz}
\usetikzlibrary{shapes.geometric}
\usepackage[most]{tcolorbox}
\tcbuselibrary{theorems}
% 3. 页眉页脚宏包
\usepackage{fancyhdr}
\usepackage{lastpage}
% 4. 超链接和书签
\usepackage{hyperref}
\usepackage{bookmark}
% 5. 智能引用
\usepackage{cleveref}
% ========================
% 1. 页面布局设置(窄边距)
\geometry{
  a4paper,
  top=15mm,
  bottom=10mm,
  left=15mm,
  right=15mm,
  headheight=25pt,
  headsep=8mm,
  footskip=15mm,
  includehead,
  includefoot
}
% 2. 字体设置(修改部分)
% 设置全局英文字体
\setmainfont{Times New Roman}
\setsansfont{Arial}
\setmonofont{Consolas}
% 定义字体命令(保持不变)
\newcommand{\kt}{\kaishu} % 楷体
\newcommand{\st}{\songti} % 宋体
\newcommand{\htbf}{\heiti\bfseries} % 黑体加粗
\newcommand{\e}{\text{e}}
\newcommand{\dd}{\text{d}}
\definecolor{deepblue}{HTML}{003366}
\definecolor{lightblue}{HTML}{E5F6FF}
\definecolor{deepgreen}{HTML}{006633}
\definecolor{lightgreen}{HTML}{E5F6E5}
\definecolor{deeppurple}{HTML}{660066}
\definecolor{lightpurple}{HTML}{F6E5F6}
\definecolor{deepred}{HTML}{990000}
\definecolor{lightred}{HTML}{FFE5E5}
\definecolor{deeporange}{HTML}{CC6600}
\definecolor{lightorange}{HTML}{FFF6E5}
\definecolor{deepgray}{HTML}{333333}
\definecolor{lightgray}{HTML}{F6F6F6}
\definecolor{deepbrown}{HTML}{8B4513}
\definecolor{lightbrown}{HTML}{FFE4B5}
\definecolor{lightyellow}{HTML}{FFFFE0}
\definecolor{darkyellow}{HTML}{CCCC00}
\definecolor{lightcyan}{HTML}{E0FFFF}
\definecolor{darkcyan}{HTML}{008B8B}
\definecolor{lightpink}{HTML}{FFB6C1}
\definecolor{darkpink}{HTML}{FF1493}
\definecolor{lavender}{HTML}{E6E6FA}
\definecolor{thistle}{HTML}{D8BFD8}
\definecolor{lightblue}{HTML}{ADD8E6}
\definecolor{darkblue}{HTML}{00008B}
\definecolor{lightgray}{HTML}{D3D3D3}
\definecolor{darkgray}{HTML}{A9A9A9}
\newcounter{theoremcounter}[section]
% 定理环境
\newtcbtheorem[use counter=theoremcounter,number within=section]{theorem}{定理}{
    colback=lightgreen, %背景颜色
    colframe=deepgreen, %边框颜色
    colbacktitle=deepgreen, %标题框颜色
    coltitle=white,  %标题颜色
    fonttitle=\upshape\color{white}, %标题字体
    fontupper=\rmfamily, %标题内容字体
    separator sign none,
    top=8pt,
    attach boxed title to top left={yshift=-2mm,xshift=5mm},
    description delimiters={ }{ },
    enhanced % 边框粗细
}{thm}
% 引理环境
\newtcbtheorem[use counter=theoremcounter,number within=section]{lemma}{引理}{
    colback=lightpurple,
    colframe=deeppurple,
    colbacktitle=deeppurple,
    coltitle=white,
    fonttitle=\upshape\color{white},
    fontupper=\rmfamily,
    separator sign none,
    top=8pt,
    attach boxed title to top left={yshift=-2mm,xshift=5mm},
    description delimiters={ }{ },
    enhanced
}{lem}
% 定义环境
\newtcbtheorem[number within=section,number within=section]{definition}{定义}{
    colback=lightred,
    colframe=deepred,
    colbacktitle=deepred,
    coltitle=white,
    fonttitle=\upshape\color{white},
    fontupper=\rmfamily,
    separator sign none,
    top=8pt,
    attach boxed title to top left={yshift=-2mm,xshift=5mm},
    description delimiters={ }{ },
    enhanced
}{def}

% 例题和解共享计数器
\newcounter{examplecounter}[section]
% 例题环境
\newtcbtheorem[use counter=examplecounter,number within=section]{example}{例题}{
    colback=lightblue,
    colframe=deepblue,
    colbacktitle=deepblue,
    coltitle=white,
    fonttitle=\upshape\color{white},
    fontupper=\rmfamily,
    separator sign none,
    top=8pt,
    attach boxed title to top left={yshift=-2mm,xshift=5mm},
    description delimiters={ }{ },
    enhanced  
}{ex}
% 结论环境(宋体)
\newtcbtheorem[number within=section]{conclusion}{结论}{
    colback=lightorange,
    colframe=deeporange,
    colbacktitle=deeporange,
    coltitle=white,
    fonttitle=\upshape\color{white},
    fontupper=\rmfamily,
    separator sign none,
    top=8pt,
    attach boxed title to top left={yshift=-2mm,xshift=5mm},
    description delimiters={ }{ },
    enhanced
}{con}
\newtcbtheorem[number within=section]{property}{性质}{
    colback=lightgray,
    colframe=deepgray,
    colbacktitle=deepgray,
    coltitle=white,
    fonttitle=\upshape\color{white},
    fontupper=\rmfamily,
    separator sign none,
    top=8pt,
    attach boxed title to top left={yshift=-2mm,xshift=5mm},
    description delimiters={ }{ },
    enhanced       % 标题加粗与其他环境统一
}{prop}
\newtcbtheorem[number within=section]{criterion}{准则}{
    colback=lightbrown,
    colframe=deepbrown,
    colbacktitle=deepbrown,
    coltitle=white,
    fonttitle=\upshape\color{white},
    fontupper=\rmfamily,
    separator sign none,
    top=8pt,
    attach boxed title to top left={yshift=-2mm,xshift=5mm},
    description delimiters={ }{ },
    enhanced
}{crit}
\newtcbtheorem[use counter=theoremcounter,number within=section]{corollary}{推论}{
    colback=lightcyan,
    colframe=darkcyan,
    colbacktitle=darkcyan,
    coltitle=white,
    fonttitle=\upshape\color{white},
    fontupper=\rmfamily,
    separator sign none,
    top=8pt,
    attach boxed title to top left={yshift=-2mm,xshift=5mm},
    description delimiters={ }{ },
    enhanced       % 标题加粗与其他环境统一
}{cor}
% 传统样式环境(无背景色)
\newtheoremstyle{plain-chinese}% 名称
  {6pt}% 上方空白
  {6pt}% 下方空白
  {\st}% 正文字体(宋体)
  {}% 缩进
  {\heiti}% 标题字体(黑体加粗)
  {.}% 标题后标点
  { }% 标题后空白
  {}% 标题说明
\theoremstyle{plain-chinese}
\newtheorem{solution}{解}[section] % 使用与例题相同的计数器
\newtheorem{remark}{注}[section]
% 4. 智能引用设置(保持不变)
% ========================
\crefname{theorem}{定理}{定理}
\crefname{lemma}{引理}{引理}
\crefname{definition}{定义}{定义}
\crefname{example}{例题}{例题}
\crefname{conclusion}{结论}{结论}
\crefname{solution}{解}{解}
\crefname{remark}{注}{注}
\crefname{property}{性质}{性质}
\crefname{criterion}{准则}{准则}
\crefname{proof}{证明}{证明}

% 5. 数学字体设置(修改)
\usepackage{unicode-math} % 更好的数学字体支持
\setmathfont{Latin Modern Math} % 使用默认数学字体
% ========== 自定义命令(保持不变) ==========
\newcommand{\R}{\mathbb{R}} % 实数集
\newcommand{\C}{\mathbb{C}} % 复数集
\newcommand{\Z}{\mathbb{Z}} % 整数集
\newcommand{\N}{\mathbb{N}} % 自然数集

% ========== 图形路径设置(保持不变) ==========
\graphicspath{{./flg/}} % 图片路径

% 6. 页眉页脚设置
% ========================
\usepackage{fancyhdr}
\usepackage{lastpage} % 获取总页数
\pagestyle{fancy}
\fancyhf{} % 清除所有页眉页脚设置

% 通用设置
\fancyhead[L]{\small\kt 工科数学分析作业} % 左边:书名(楷体)
\fancyhead[C]{\small\st 华南理工大学}
\fancyhead[R]{\small\st 夏同202530451676} % 中间:声明(宋体)
\fancyfoot[C]{\thepage} % 居中页码(自定义样式)

% 正文部分设置
\fancyhead[R]{\small\st\rightmark} % 右边:章节名称(宋体)

% 前言部分设置(使用罗马数字页码)
\fancypagestyle{frontmatter}{
    \fancyhf{}
    \fancyhead[L]{\small\kt 工科数学分析作业}
    \fancyhead[C]{\small\st 前言}
    \fancyhead[R]{} % 前言部分无章节名称
    \fancyfoot[C]{\thepage}
    \renewcommand{\headrulewidth}{0.4pt} % 页眉线
    \renewcommand{\footrulewidth}{0pt} % 无页脚线
    \pagenumbering{Roman} % 罗马数字页码
}

% 目录部分设置(使用罗马数字页码,延续前言页码)          % 去掉标题下方的横线
\fancypagestyle{tocmatter}{
    \fancyhf{}
    \fancyhead[L]{\small\kt 工科数学分析作业}
    \fancyhead[C]{\small\st 目录} % 居中显示"目录"
    \fancyhead[R]{\small\st 夏同} 
    \fancyfoot[C]{\thepage}
    \renewcommand{\headrulewidth}{0.4pt}
    \renewcommand{\footrulewidth}{0pt}
    % 注意:这里不重置页码,延续前面的罗马数字
}
% 正文部分设置(使用阿拉伯数字页码)
\fancypagestyle{mainmatter}{
    \fancyhf{}
    \fancyhead[L]{\small\kt 工科数学分析作业}
    \fancyhead[C]{\small\st 计算机科学与工程学院~计类1班~夏同}
    \fancyhead[R]{\small\st\rightmark}
    \fancyfoot[C]{\thepage}
    \renewcommand{\headrulewidth}{0.4pt} % 页眉线
    \renewcommand{\footrulewidth}{0pt} % 无页脚线
    \pagenumbering{arabic} % 阿拉伯数字页码
}
% 设置章节标记格式
\renewcommand{\chaptermark}[1]{\markboth{#1}{}}
\renewcommand{\sectionmark}[1]{\markright{\thesection.\ #1}}

% 8. 章节标题字体设置
\ctexset{
    chapter = {
        format = \centering, % 整体居中
        nameformat = \kaishu\LARGE, % 编号部分黑体加粗
        titleformat = \songti\LARGE, % 标题部分宋体
        aftername = \quad, % 编号和标题之间的间距
        beforeskip = 30pt, % 标题前的垂直间距
        afterskip = 20pt, % 标题后的垂直间距
        name = {第,章}, % 中文章节编号格式
        number = \chinese{chapter}, % 使用中文数字
    },
    section = {
        format = \raggedright, % 左对齐
        nameformat = \heiti\bfseries\large, % 编号部分黑体加粗
        titleformat = \songti\large, % 标题部分宋体
        aftername = \quad, % 编号和标题之间的间距
        beforeskip = 15pt, % 标题前的垂直间距
        afterskip = 6pt, % 标题后的垂直间距
    },
    subsection = {
        format = \raggedright, % 左对齐
        nameformat = \heiti\bfseries\normalsize, % 编号部分黑体加粗
        titleformat = \songti\normalsize, % 标题部分宋体
        aftername = \quad, % 编号和标题之间的间距
        beforeskip = 6pt, % 标题前的垂直间距
        afterskip = 3pt, % 标题后的垂直间距
    }
}
\begin{document}
\begin{titlepage}
    \centering
    % 顶部留白
    \vspace*{0.5cm}
        \includegraphics[width=\textwidth]{flg/mylogo.png} % 调整图片大小
    \par\vspace{1cm}
    {\fontsize{75}{75}\selectfont\songti 简单导数 \par}
    \vspace{1cm} % 减少间距
    {\Large \kaishu{作者:} 夏同 \par}
    \vspace{0.1cm}
    {\Large \kaishu{学院:} 计算机科学与工程学院 \par}
    \vspace{0.1cm}
    {\Large \kaishu{参考:} 群友 \par}
    \vspace{0.1cm}
    {\Large \kaishu{日期:} \today \par}
    \vspace{0.5cm}
        \includegraphics[width=0.7\textwidth]{flg/saying.png} % 调整图片大小
    \par\vspace{1cm}
\end{titlepage}
% 前言部分
\frontmatter
\pagestyle{frontmatter} % 应用前言样式
% 前言内容
\chapter*{前言}
本书内容全部使用\LaTeX{}进行排版,作者"还在尬黑"是一位准大一学生,高中毕业于广东深圳中学,高三数学各次大考平均排名位于前5\%,高考应该也不例外。"还在尬黑"拥有知乎(同名),微信公众号(同名),小红书号(同名)等账号,头像是一个右手三叶结。以及不同名不同头像的GitHub账号,发表原创优质内容百余篇,且固定更新频率,堪称发文的“螺丝钉”。

“还在尬黑”对圆锥曲线的解题研究有着浓厚兴趣,并在书中将其总结成了一套完整的解析几何教程。本书适合高中解析几何解题体系未成熟的高二高三学生,以及前来自学的高一学生以及初中生,也可作为高中数学教材。笔者衷心希望本书能够帮助读者提高圆锥曲线解题速度和解题能力,并能够准确地识别班内的"大佬"是用什么东西来装逼的,当然,本书和市面上的某些书不同,不会直接甩给学生们根本用不明白也不懂从何而来的技巧大招,而是会侧重解析一种方法的产生过程,以及如何恰当的选择方法解决具体问题。

本书后面有些部分为选学内容,难度可能较高,属于高考不怎么考的范畴,这部分留给同学们进行自我提高和兴趣拓展。当然,建议读者先打牢必学内容的基础,再来进行进一步的学习。

在创作本书的过程中,笔者也得到了朋友们和热心群众的帮助,在此向他们表示感谢!

十分感谢读者朋友们的支持和赞助!祝大家健康进步,高考成功!
\begin{flushright}
    \vspace{2\baselineskip} % 在署名前留出两行空白
    \kt 还在尬黑 \\ % 楷体斜体
    \today
\end{flushright}
\begin{figure*}[htbp]
	\centering
    \includegraphics[width=0.25\textwidth]{flg/logo.png}%
	\caption{我的头像}
	\label{fig0-1}
\end{figure*}

% 目录部分(关键修改)
\newpage
\CTEXoptions[tocdepth=2] % 设置目录深度为章和节
\tableofcontents
\thispagestyle{tocmatter} 
% 正文部分
\mainmatter
\pagestyle{mainmatter} % 应用正文样式

% 章节内容
\chapter{行列式}
\section{第1周作业}
习题一第一大题的第(1)(3)(5)问解答如下:
\begin{example}{(习题一第一大题)}{}
计算行列式的值
$\begin{vmatrix}
    \sin x & -\cos x \\
    \cos x & \sin x
\end{vmatrix}$
,
$\begin{vmatrix}
    1 & 2 & 3 \\
    4 & 5 & 6 \\
    7 & 8 & 9
\end{vmatrix}$
,
$\begin{vmatrix}
    x & y & y \\
    y & x & y \\
    y & y & x
\end{vmatrix}$
\end{example}
\begin{solution}{}{}
    (1)原式$=\sin^2x-(-\cos^2x)=1$.\\
    (2)由沙路法则:原式\vspace{-10pt}\\
    \vspace{-10pt}
    \begin{align*}
        =&1\times5\times9+2\times6\times7+3\times4\times8\\
        &-1\times6\times8-2\times4\times9-3\times5\times7\\
        =&45+84+96-48-72-105\\
        =&225-225=0.\square
    \end{align*}
    实际上第三行是第二行各数的两倍减去第一行各数得到的,因此第三行是第一行和第二行的线性组合,所以矩阵:
    \[
    \begin{pmatrix} 
        1 & 2 & 3 \\
        4 & 5 & 6 \\
        7 & 8 & 9
    \end{pmatrix}
    \]的行向量线性相关,行列式的值为0.\\
    (3)仍然由沙路法则:原式\vspace{=-10pt}
    \begin{align*}
        &=x^3+y^3+y^3-xy^2-xy^2-xy^2\\
        &=x^3+2y^3-3xy^2.\square
    \end{align*}
    实际上这个结果可以推广的。
\end{solution}
\begin{theorem}{}{}
    将$n$阶行列式$D$中每个元$a_{ij},(i,j=1,2,...,n)$都加上参数$t$,得到的行列式记为$D(t)$,则:
    \[D(t)=D+\sum_{i=1}^n\sum_{j=1}^nA_{ij}.\]
    其中$A_{ij}$是$a_{ij}$的代数余子式.
\end{theorem}
\begin{proof}{}{}
    将$D(t)$中的第一列拆开,得到的新行列式记为$D_1(t)$,则:
    \[D_1(t)=\begin{vmatrix}a_{11}&a_{12}+t&\cdots&a_{1n}+t\\
        a_{21}&a_{22}+t&\cdots&a_{2n}+t\\
        \vdots&\vdots&&\vdots\\
        a_{n1}&a_{n2}+t&\cdots&a_{nn}+t\end{vmatrix}\]
    \begin{align*}
        D(t)&=D_1(t)+\begin{vmatrix}t&a_{12}+t&\cdots&a_{1n}+t\
        t&a_{22}+t&\cdots&a_{2n}+t\\\vdots&\vdots&&\vdots\\
        t&a_{n2}+t&\cdots&a_{nn}+t\end{vmatrix}\\
        &=D_1(t)+
        \begin{vmatrix}t&a_{12}&\cdots&a_{1n}\\
            t&a_{22}&\cdots&a_{2n}\\
            \vdots&\vdots&&\vdots\\
            t&a_{n2}&\cdots&a_{nn}\end{vmatrix}\\
        &=t\sum_{i=1}^nA_{i1}+D_1(t).\\
        &=t\sum_{i=1}^nA_{i1}+t\sum_{i=1}^nA_{i2}+D_2(t).\\
        &=t\sum_{i=1}^nA_{i1}+t\sum_{i=1}^nA_{i2}+t\sum_{i=1}^nA_{i3}+D_3(t).\\
        &=\dots\\
        &=t\sum_{i=1}{n}\sum_{j=1}^nA_{ij}+D.
        \end{align*}
\end{proof}

\chapter{极限}
\newpage
\section{第2周作业}
\begin{example}{给出下列极限的精确定义}{}
    (1)$\displaystyle\lim_{x\to 0}f(x)=A$\quad (2)$\displaystyle\lim_{x\to 0}(1+x)^{\frac{1}{x}}=e$
\end{example}
\begin{solution}
    (1)对于任意$\varepsilon >0$,存在$\delta>0$使得当$0<|x|<\delta$时,$|f(x)|<\varepsilon$.

    (2)对于任意$\varepsilon>0$,存在$\delta>0$使得当$0<|x|<\delta$时,$|(1+x)^{\frac{1}{x}}-e|<\varepsilon$.
\end{solution}
\begin{example}{利用极限的精确定义证明下列函数的极限}{}
    (1)$\displaystyle\lim_{x\to 3}(x^2+5x)=24$\quad (2)$\displaystyle\lim_{x\to 1}\frac{x^2-1}{x-1}=2$
\end{example}
\begin{solution}
    (1)要证对于任意$\varepsilon>0$,存在$\delta>0$使得当$0<|x-3|<\delta$时,$|(x^2+5x)-24|=|(x+8)(x-3)|<\varepsilon$。已经出现了$|x-3|$,所以现在只需限定$|x+8|$,先限定$|x-3|<1$,那么$|x+8|<12$,此时还需满足$|(x+8)(x-3)|<12|x-3|<\varepsilon$,得$|x-3|<\dfrac{\varepsilon}{12}$,故取$\delta=\min\{1,\dfrac{\varepsilon}{12}\}$,当$0<|x-3|<\delta$时,$|(x^2+5x)-24|=|(x+8)(x-3)|<\varepsilon$.

    (2)要证对于任意$\varepsilon>0$,存在$\delta>0$使得当$0<|x-1|<\delta$时,$|\frac{x^2-1}{x-1}-2|=|x-1|<\varepsilon$。取$\delta=\varepsilon$,当$0<|x-1|<\delta$时,$|\frac{x^2-1}{x-1}-2|<\varepsilon$.
\end{solution}
\begin{example}{证明}{}
    由$\displaystyle\lim_{x\to a}f(x)=A$能推出$\displaystyle\lim_{x\to a}|f(x)|=|A|$,但反之不然。
\end{example}
\begin{solution}
    对于任意$\varepsilon>0$,存在$\delta>0$使得当$0<|x-a|<\delta$时,$|f(x)-A|=<\varepsilon$,所以由绝对值不等式得到$\displaystyle|f(x)-A|>||f(x)-|-A||=||f(x)|-|A||>0$,故$||f(x)|-|A||<\varepsilon$,所以由$\displaystyle\lim_{x\to a}f(x)=A$能推出$\displaystyle\lim_{x\to a}|f(x)|=|A|$.然后反过来,考虑定义在实数域上的函数$f(x)=\begin{cases}1,x\in Q\\-1,x\notin Q\end{cases}$,其极限$\displaystyle\lim_{x\to a}|f(x)|=1$,但是$\displaystyle\lim_{x\to a}f(x)$不存在。
\end{solution}
\begin{example}{利用极限的精确证明}{}
   $\displaystyle\lim_{x\to a}\sin x=\sin a$
\end{example}
\begin{solution}
    要证$\displaystyle\lim_{x\to a}\sin x=\sin a$,只需证对于任意$\varepsilon>0$,存在$\delta>0$使得当$0<|x-a|<\delta$时,$\displaystyle\lim_{x\to a}|\sin x-\sin a|=2|\cos\dfrac{x+a}{2}||\sin\dfrac{x-a}{2}|<\varepsilon$。又因为$2|\cos\dfrac{x+a}{2}||\sin\dfrac{x-a}{2}|<2|\sin\dfrac{x-a}{2}|<2|\dfrac{x-a}{2}|=|x-a|$,所以取$\delta=\varepsilon$,当$0<|x-a|<\delta$时,$\displaystyle\lim_{x\to a}|\sin x-\sin a|<\varepsilon$.
\end{solution}

\end{document}