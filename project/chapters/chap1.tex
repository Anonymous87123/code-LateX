\chapter{先导课程}
\chaptermark{先导课程}
\section{写在前面}
解析几何是高中数学的重要学习内容,在高考中分值占比较高。

不少教辅会以“圆锥曲线”作为替代性的表述,这可能是因为圆锥曲线是解析几何中的重难点。但笔者认为“解析几何”更为贴切:第一,从应试角度考虑,圆锥曲线是解析几何的子集,现在考试有的题也会出直线和圆,新定义曲线(如3次曲线等)进行考察;第二,笔者打算从不单单只谈3种圆锥曲线,而是想在其基础之上,更多地普及一些考试常用的几何知识和背景;第三,笔者愿意先从最基本的直线开始说起,帮助读者搭建完整的解析几何体系。

首先,笔者来讲一讲怎么进行计算。这似乎是一个很简单的问题,但是谁又能保证在紧张刺激的考场环境下不会犯错误?一旦出现计算错误,检查就需要花费一定的时间,所以不如挑选合适的计算方法,从源头上减少失误。本节中,笔者会结合自己的一些实战经验,尽量告诉大家一些计算过程中减小失误,提升速度的技巧和方法,以及解析几何中计算的基本方向----整体代换。

鉴于本书的覆盖群体,笔者会尽量避免过多的公式推导和过于严谨的学术表述,而是从直观的角度出发,用一些具体的例子来说明计算的过程。希望大家能从中受益。
\section{轮换,对称}
在此之前,请允许我先介绍一些基本的概念,我们不妨先来看一些看起来很整齐的式子,这些式子平时很常见,大家在备考强基计划的过程中也会遇到比较多这样的式子:
\begin{example}{}{}
    观察以下代数式,并尝试在心里面归纳出它们的特点:\vspace{-10pt}
    \begin{gather*}
        abc\\
        a+b+c \\
        ab+bc+ca \\
        a^2+b^2+c^2 \\
        (a+b)(b+c)(c+a)\\
        a²b + a²c + b²a + b²c + c²a + c²b
    \end{gather*}
\end{example}
\begin{solution}
    相信大部分同学是通过自己的直觉来归纳的,直观的感受就是这些式子很“整齐”,而且很有规律可循。那么问题来了:“整齐”是怎么体现的?更进一步地,有没有手段验证一个代数式是“整齐”的?至于“很有规律可循”,那么规律是什么?
    
    这些问题循序渐进,如果理清这些问题,那么读者便掌握了学习数学时地最基本的关注点:定义,性质,判定。这些式子中的$a,b,c$​​结构权重是均等的​​,它们地位相同,没有“特权变量”,也没有“次序”之分。
    
    而且,眼尖的读者可以发现,这些表达式中的项往往成组出现​​,覆盖所有可能的组合,比如$a+b+c$中全为一次项,如果$a$出现了,不用猜也知道$b$和$c$也出现了;再比如$a²b + a²c + b²a + b²c + c²a + c²b$中,$a^2b$出现了,其中$a$被平方了,那不用猜也知道在其他的项中,$b$和$c$也会被平方,而且后面一定会跟着其他没有被平方的字母。它们出现的次数相同。
    
    {\heiti 事实上,由于乘法和加法的交换律和结合律,我们可以发现,对于上面任意一个式子,我们都可以挑选任意两个变量交换位置,而多项式本身保持不变。}大家不妨想象一下阅兵式的场景,即使我们偷偷调换两个兵的位置,你也看不出来有什么异样,这是阅兵队伍“整齐”的体现。同样地,这个代数式也可以这样操作,来验证这个代数式是“整齐”的,“规律可循”的。这样我们便可以引出对称式的概念。
\end{solution}
\begin{definition}{对称式}{}
    对于一个 $n$ 元多项式 $P(x_1, x_2, \dots, x_n)$,若对于数 $1, 2, \dots, n$ 的任意一个排列 $(i_1, i_2, \dots, i_n)$,都有
    \[
    P(x_{i_1}, x_{i_2}, \dots, x_{i_n}) = P(x_1, x_2, \dots, x_n),
    \]
    则称 $P(x_1, x_2, \dots, x_n)$ 为对称式。
\end{definition}
“对称”​​ 体现在字母地位平等,没有哪个字母是特殊的。只要式子中包含某个由特定字母组成的项(例如$a^2b$),那么它一定包含由所有其他字母以同样的方式组成的项(即$a^2c,b^2a,b^2c,c^2a,c^2b$)。

这样我们就认识了对称式的概念,这样当读者听到别人说“对称式”的时候,不会至于一脸懵逼,或者一边点头,假装听懂,然后用直觉去理解这个概念(这样的情况长期发展下去,是不利于学习数学的)。当然,读者也许会发现,像“$a²b + a²c + b²a + b²c + c²a + c²b$”这样的式子其实比较长,占用了较大的空间,也显得不够简洁。因此我们不妨规定以下记号:
\begin{definition}{循环和}{}

\end{definition}

\begin{property}{对称式的性质}{}
    (1)基本对称多项式的基础性​​:

\end{property}

那么下面我们乘胜追击,再来看一组式子:
\begin{example}{}{}
    观察以下代数式,并尝试在心里面归纳出它们的特点:\vspace{-10pt}
    \begin{gather*}
        a^2b+b^2c+c^2a\\
        a^3b+b^3c+c^3a\\
        \dfrac{a}{b+c}+\dfrac{b}{c+a}+\dfrac{c}{a+b}
    \end{gather*}
\end{example}
\begin{solution}
    和刚才的对称式不同,如果我们这里调换某两个字母的位置,那么结果也会发生变化。比如$a^2b+b^2c+c^2a$中,如果我们把$a$和$b$调换位置,那么结果也会发生变化,比如说新出现了$b^2a$项,这是原来所没有的。

    但是读者会发现,这个式子看上去也是有规律可循的,比如$a^3b+b^3c+c^3a$中,$a^3$项出现了,那不用猜也知道$b^3$和$c^3$也会在其它部分出现,而且出现的次数相同,但是和上文的规律不一样,$a^3$后面只会跟着$b$,却没有$c$,即没有$a^2c$项。
\end{solution}
\begin{example}{}{}
    将$(a+b)(b+c)(c+a)$进行展开,并尽己所能地保证结果的每个部分都是由$a,b,c$三个元同时出现且地位相同的式子:
\end{example}
\begin{solution}
    先展开,再重组:\vspace{-10pt}
    \begin{align*}
    \vspace{-15pt}
    (a+b)(b+c)(c+a)&=(b^2+ac+ab+bc)(a+c)\\
    &=ab^2+b^2c+a^2c+ac^2+a^2b+abc+abc+bc^2\\
    &=(a+b+c)(ab+bc+ca)-abc
    \vspace{-10pt}
    \end{align*}
\end{solution}
最后为什么

\begin{theorem}{}{}
    $$(a+b)^2=a^2+2ab+b^2$$

\end{theorem}