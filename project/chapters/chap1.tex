\chapter{先导课程}
\section{认识解析几何}
解析几何是高中数学的重要学习内容,不少教辅会以“圆锥曲线”作为替代表述,但笔者认为“解析几何”更为贴切:第一,从应试角度考虑,圆锥曲线是解析几何的子集,现在考试有的题也会出直线和圆,新定义曲线(如3次曲线等)进行考察;第二,笔者打算从不单单只谈3种圆锥曲线,而是想在其基础之上,更多地普及一些考试常用的几何知识和背景;第三,笔者愿意先从最基本的直线开始说起,帮助读者搭建完整的解析几何体系。

\begin{definition}{函数的极限}
设函数 \( f(x) \) 在点 \( x_0 \) 的某个去心邻域内有定义,如果存在常数 \( A \),对于任意给定的正数 \( \varepsilon \)(无论多么小),总存在正数 \( \delta \) 使得当 \( 0 < |x - x_0| < \delta \) 时,有 \( |f(x) - A| < \varepsilon \),那么常数 \( A \) 就叫做函数 \( f(x) \) 当 \( x \to x_0 \) 时的极限,记作
\[ \lim_{x \to x_0} f(x) = A \]
\end{definition}

\begin{example}
证明 \( \lim_{x \to 1} (2x + 1) = 3 \).
\end{example}

\begin{solution}
对于任意 \( \varepsilon > 0 \),取 \( \delta = \frac{\varepsilon}{2} \),则当 \( 0 < |x - 1| < \delta \) 时,
\[ |(2x + 1) - 3| = |2x - 2| = 2|x - 1| < 2\delta = \varepsilon \]
因此,由定义可知 \( \lim_{x \to 1} (2x + 1) = 3 \).
\end{solution}

\section{连续函数}

\begin{definition}{连续}
设函数 \( f(x) \) 在点 \( x_0 \) 的某个邻域内有定义,如果
\[ \lim_{x \to x_0} f(x) = f(x_0) \]
则称函数 \( f(x) \) 在点 \( x_0 \) 处连续。
\end{definition}

\begin{theorem}{连续函数的性质}
如果函数 \( f(x) \) 和 \( g(x) \) 都在点 \( x_0 \) 处连续,则它们的和、差、积、商(分母不为零)都在点 \( x_0 \) 处连续。
\end{theorem}
