\chapter{先导课程}
\chaptermark{先导课程}
\section{写在前面}
解析几何是高中数学的重要学习内容,在高考中分值占比较高。

不少教辅会以“圆锥曲线”作为替代表述,这可能是因为圆锥曲线是解析几何中的重难点。但笔者认为“解析几何”更为贴切:第一,从应试角度考虑,圆锥曲线是解析几何的子集,现在考试有的题也会出直线和圆,新定义曲线(如3次曲线等)进行考察;第二,笔者打算从不单单只谈3种圆锥曲线,而是想在其基础之上,更多地普及一些考试常用的几何知识和背景;第三,笔者愿意先从最基本的直线开始说起,帮助读者搭建完整的解析几何体系。

\section{常见的计算式子}
首先,笔者来讲一讲怎么进行计算。这似乎是一个很简单的问题,但是谁又能保证在紧张刺激的考场环境下不会犯错误?一旦出现计算错误,检查就需要花费一定的时间,所以不如挑选合适的计算方法,从源头上减少失误。我们不妨先来看一些很整齐的式子:
\begin{example}{}{}
    展开$(a+b)(b+c)(c+a)$
\end{example}
\begin{solution}
    我们先按部就班地算一算:\vspace{-10pt}
    \begin{align*}
    \vspace{-15pt}
    (a+b)(b+c)(c+a)&=(b^2+ac+ab+bc)(a+c)\\
    &=ab^2+b^2c+a^2c+ac^2+a^2b+abc+abc+bc^2\\
    &=(a+b+c)(ab+bc+ca)-abc
    \vspace{-10pt}
    \end{align*}
\end{solution}
\begin{theorem}{}{}
    $$(a+b)^2=a^2+2ab+b^2$$
\end{theorem}