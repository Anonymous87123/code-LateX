\chapter{杂题}
\section{数学问题}
\newpage
\begin{example}{}{}
    $a,d\geq 0,b,c>0,b+c\geq a+d$,求$\dfrac{b}{c+d}+\dfrac{c}{a+b}$的最小值.
\end{example}
\begin{solution}
    观察到$b,c$在两个部分都出现,而$a,b$只出现了一次,固定$b,c$,左边关于$a$单调递减,右边关于$d$单调递减,又$b+c\geq a+d$,故不断扩大$a,d$直到$a+d=b+c$使得原式变小:
    \[\dfrac{b}{c+d}+\dfrac{c}{a+b}=\dfrac{b}{c+d}+\dfrac{c}{2b+c-d}=\dfrac{1}{\frac{c}{b}+\frac{d}{b}}+\dfrac{\frac{c}{b}}{2+\frac{c}{b}-\frac{d}{b}}\]
    换元$t=\dfrac{c}{b},x=\dfrac{d}{b}$,则需要研究这个函数的最小值
    \begin{align*}f(x)=\dfrac{1}{x+t}+\dfrac{t}{2+t-x},x\in[0,t+1],t>0\quad
        f'(x)=\dfrac{t}{(2+t-x)^2}-\dfrac{1}{(x+t)^2}\\
        f'(x)>0\Leftrightarrow \dfrac{\sqrt{t}}{2+t-x}>\dfrac{1}{x+t}
        \Rightarrow x>x_0=\dfrac{2+t-t\sqrt{t}}{\sqrt{t}+1}=\dfrac{2+u^2-u^3}{u+1},u^2=t
    \end{align*}
    且$2+u^2-u^3=0$只有一根$u_0>1$,$ 2u^3+u-1=0$只有一根$u_1<1$,当极值点在定义域内
    \[x_0\in(0,1+u^2)\Leftrightarrow\begin{cases}x_0=\frac{2+u^2-u^3}{u+1}<u^2+1\\x_0=\frac{2+u^2-u^3}{u+1}>0\end{cases}\Leftrightarrow\begin{cases}2u^3+u-1>0\\2+u^2-u^3>0\end{cases}\Leftrightarrow \begin{cases}u>u_1\\u<u_0\end{cases}\]
    所以此时$f(x)$最小值在$x=x_0$处取到
    \[f_1(u_0)=\dfrac{1}{x_0+u^2}+\dfrac{u^2}{2+u^2-x_0}=g_1(u)=\dfrac{(1+u)^2}{2(1+u^2)},u_1<u<u_0\]
    当极值点在定义域的右侧
    \[x_0>1+u^2\Leftrightarrow\begin{cases}\frac{2+u^2-u^3}{u+1}>u^2+1\\\frac{2+u^2-u^3}{u+1}>0\end{cases}\Leftrightarrow \begin{cases}2u^3+u-1<0\\2+u^2-u^3>0\end{cases}\Leftrightarrow u_0>u_1>u\]
    所以$f(x)$单减,在$x=u^2+1$处取到最小值
    \[f_2(u_0)=\dfrac{1}{2u^2+1}+\dfrac{u^2}{2+u^2-u^2-1}=u^2+\dfrac{1}{2u^2+1}=g_2(u),u<u_1\]
    然而还有极值点在定义域的左侧的情况,当
    \[x_0<0<1+u^2\Leftrightarrow2+u^2-u^3<0\Leftrightarrow u>u_0\]时有$f(x)$单增,则$f(x)$在$x=0$处取到最小值
    \[f_3(0)=\dfrac{1}{0+u^2}+\dfrac{u^2}{2+u^2-0}=\dfrac{1}{u^2}+\dfrac{u^2}{2+u^2}=g_3(u),u>u_0\]
    分别求导得到$g_1(u)\geq g_1(u_1),g_2(u)\geq \sqrt2-\dfrac12,g_3(u)\geq\sqrt{2+\sqrt2}>1$
    因此淘汰$g_3(u)$,比较前两个,由于$\begin{cases}g_1(u_1)=\dfrac{(1+u_1)^2}{2(1+u_1^2)}>g\bigg(\dfrac47\bigg)>\sqrt2-\frac12\\2u_1^3+u_1-1=0\Rightarrow u_1>\dfrac47\end{cases}$发现成立,得到最小值$\sqrt2-\dfrac12$.
\end{solution}
\begin{example}{}{}
    证明$\left(\dfrac{1}{n}\right)^n+\left(\dfrac{2}{n}\right)^n+\left(\dfrac{3}{n}\right)^n+\cdots+\left(\dfrac{n-1}{n}\right)^n+\left(\dfrac{n}{n}\right)^n>\dfrac{2(n-1)}{n+1}$.
\end{example}
\begin{solution}
    这个不等式有明显的几何意义,首先分母都有$n$,然后分子每次自增,长得很像一个个矩形的面积求和,然后右边是个分式,感觉就是积分后的结果,所以我们根据题目中的$n$次方结构,找到函数$f(x)=x^n$,然后将其积分区间$[0,1]$分成$n$等分,并由$f(x)$下凸,将每一个曲边梯形“扩大”成直边梯形(为什么不先放成矩形?因为放成梯形明显更紧,而且求和的难度不会显著变大),根据题中的$k=1,2,3,\cdots,n$:
    \[\dfrac{1}{2}\left(\left(\dfrac{k-1}{n}\right)^n+\left(\dfrac{k}{n}\right)^n\right)\left(\dfrac{k}{n}-\dfrac{k-1}{n}\right)>\int_{\frac{k-1}{n}}^{\frac{k}{n}}x^n\dd x\]
    求和得到
    \begin{align*}
    &\left(\dfrac{1}{n}\right)^n+2\left(\dfrac{2}{n}\right)^n+2\left(\dfrac{3}{n}\right)^n+\cdots+2\left(\dfrac{n-1}{n}\right)^n+\left(\dfrac{n}{n}\right)^n>n\int_{0}^1x^n\dd x\\
    &\Leftrightarrow 2\sum_{k=1}^n\left(\dfrac{k}{n}\right)^n-\left(\dfrac{0}{n}\right)^n-\left(\dfrac{n}{n}\right)^n>\left.\dfrac{x^{n+1}}{n+1}\right|^{1}_0=\dfrac{n}{n+1}\\
    &\Leftrightarrow \sum_{k=1}^n\left(\dfrac{k}{n}\right)^n>\dfrac{2(n-1)}{n+1}\\
    \end{align*}
\end{solution}
\begin{example}{}{}
    证明$f(x)=\ln x-8\dfrac{x-1}{(x^{\frac13}+1)^3}$单调递增
\end{example}
\begin{solution}
    转化为$\displaystyle F(t)=3\ln t-8\cdot\frac{t^3-1}{(t+1)^3}$单增,$\displaystyle F^{\prime}(t)=\frac{3}{t}-8\cdot\frac{3(t^2+1)}{(t+1)^4}=\frac{3}{t}-\frac{24(t^2+1)}{(t+1)^4}$
    链式法则$f^{\prime}(x)=\frac{dF}{dt}\cdot\frac{dt}{dx}=F^{\prime}(t)\cdot\frac{1}{3t^2}$,所以\[f^{\prime}(x)=\left(\frac{3}{t}-\frac{24(t^2+1)}{(t+1)^4}\right)\cdot\frac{1}{3t^2}=\frac{1}{t^3}-\frac{8(t^2+1)}{t^2(t+1)^4}=\frac{(t-1)^4}{t^3(t+1)^4}>0.\]
\end{solution}
