\chapter{曲线系}
\section{曲线系引入}
\begin{example}{(椭圆上的中点弦直线)}{}
设椭圆$\dfrac{x^2}{a^2}+\dfrac{y^2}{b^2}=1$内有一点$P(x_0,y_0)$,求以$P$为弦中点的直线方程.
\end{example}
\begin{solution}
    构造两个椭圆\[\begin{cases}\dfrac{x^2}{a^2}+\dfrac{y^2}{b^2}=1\\\dfrac{(2x_0-x)^2}{a^2}+\dfrac{(2y_0-y)^2}{b^2}=1\end{cases}\]
    相减得到答案\[\dfrac{x_0x}{a^2}+\dfrac{y_0y}{b^2}=\dfrac{x_0^2}{a^2}+\dfrac{y_0^2}{b^2}\]
\end{solution}
\begin{example}{(椭圆上的直线两点式)}{}
设椭圆$\dfrac{x^2}{a^2}+\dfrac{y^2}{b^2}=1$上有两点$A(x_1,y_1),B(x_2,y_2)$,求$AB$方程.
\end{example}
\begin{solution}
    构造以$A(x_1,y_1),B(x_2,y_2)$为直径的相似椭圆\[\dfrac{(x-x_1)(x-x_2)}{a^2}+\dfrac{(y-y_1)(y-y_2)}{b^2}=1\]则与原来的椭圆方程相减
    \[\dfrac{x^2}{a^2}+\dfrac{y^2}{b^2}=\dfrac{(x_1+x_2)^2}{a^2}+\dfrac{(y_1+y_2)^2}{b^2}\]就可以得到$AB$的方程为
    \[\dfrac{x_1+x_2}{a^2}x+\dfrac{y_1+y_2}{b^2}y=1+\dfrac{x_1x_2}{a^2}+\dfrac{y_1y_2}{b^2}\]
    再对这个方程使用万能代换便可以得到参数方程直线两点式的形式
\end{solution}
\begin{example}{(抛物线上的直线两点式)}{}
设$y^2=2px$上有两点$A(x_1,y_1),B(x_2,y_2)$,求$AB$方程.
\end{example}
\begin{solution}
    构造直线系$(y-y_1)(y-y_2)=0$,与$y^2=2px$相减得到\[y^2-(y-y_1)(y-y_2)=2px\Leftrightarrow 2px-(y_1+y_2)y+y_1y_2=0\]
\end{solution}
\begin{example}{(利用梯形构造四点共圆)}{}
    二次函数$y=x^2+(a+1)x+a-2$与坐标轴交于$A,B,C$三点,求$\triangle ABC$的外接圆的方程,并验证圆心是否在定直线上,以及外接圆是否过定点.
\end{example}
\begin{solution}
    构造直线系$(y-0)(y-(a-2))=y(y-a+2)$再与抛物线$x^2+(a+1)x-y+a-2=0$相加得到:
    \[x^2+y^2+(a+1)x-y-ay+2y+a-2=x^2+y^2+(a+1)x+(1-a)y+a-2=0\]
    圆心$\bigg(\dfrac{a+1}{-2},\dfrac{1-a}{-2}\bigg)$一定在定直线上。将方程改写为
    \[a(x-y+1)+x^2+y^2+x+y-2=0\]发现当$x=0,y=1$和$x=-2,y=-1$符合要求,这样就得到定点$(0,1),(-2,-1)$
\end{solution}
\begin{example}{(2025八省联考)}{}
    椭圆$\dfrac{x^2}{9}+\dfrac{y^2}{8}=1$,$A(-3,0),B(2,0)$,过$B$的弦$MN$与点$A$形成的三角形$AMN$外心为$D$,证明$k_{MN}k_{OD}$为定值.
\end{example}
\begin{solution}
    由斜率相反出共圆的结论,可以构造直线系$(x-ty-2)(x-ty+3)=0$,然后与椭圆联立:
    \[\begin{cases}(x-ty-2)(x-ty+3)=0\\8x^2+9y^2-72=0\end{cases}\]
    然后待定$x^2,y^2$系数相等得到椭圆的系数是$(t^2+1)$,所以相加得到:
    \begin{align*}&(t^{2}+1)(8x^{2}+9y^{2}-72)+(x-ty-2)(x+ty+3)=0\\&\Leftrightarrow{}(8t^{2}+9)x^{2}+(8t^{2}+9)y^{2}+x-5ty-72t^{2}-78=0\end{align*}
    则$k_{OD}=-5t,k_{MN}=\dfrac1{t}$,定值为$-5$.
\end{solution}
\begin{example}{(直径圆的构造)}{}
    已知抛物线$y^2=2px$和弦$AB$所在的直线$x=ty+m$,求$A,B$的直径圆方程.
\end{example}
\begin{solution}
    消去$x$得到\[y^2=2p(ty+m)=2pty+2pm\Rightarrow y^2-2pty-2pm=0\]
    消去$y$得到\[t^2y^2=(x-m)^2=2p^2t^2x\Rightarrow x^2-(2m+2p^2t^2)x+m^2=0\]
    加起来就可以得到直径圆方程\[x^2+y^2-(2m+2p^2t^2)x-2pty+m^2-2pm=0\]
    原因是消去$x,y$得到的式子都可以变为$(x-x_1)(x-x_2)=0$和$(y-y_1)(y-y_2)=0$的形式,而$(x_1,y_1),(x_2,y_2)$是弦$AB$的两端点,所以直径圆的方程就是这两个直线的交点的方程。
\end{solution}
\begin{example}{(过两点的圆系方程例题)}{}
    已知$C:y^2=4x,l:y=x+1$交于$A,B$两点,求经过$A,B$且与$x=-1$相切的圆的方程.
\end{example}
\begin{solution}
    联立有:\[\begin{cases}y^2=4(y-1)\\(x+1)^2=4x\end{cases}\Rightarrow 
    \begin{cases}y^2-4y+4=0\\x^2-2x+1=0\end{cases}\]
    相加就有:\[x^2+y^2-2x-4y+5=0\]这是直径圆,接下来配凑圆系方程:
    \[x^{2}+y^{2}-6x-4y-3+\lambda(x-y-1)=0\]
    代入$x=-1$,得到\[y^{2}-4y+4+\lambda(-2-y)=0\]
    并让$\Delta=0$,得到:\[\Delta=(4+\lambda)^{2}-4(4-2\lambda)=0\Rightarrow 
    \lambda_1=0,\lambda_2=-16\]
    所以圆方程有两个:$x^2+y^2-2x-4y+5=0$和$x^2+y^2-18x+12y+21=0$,本题目是易错题,容易漏解.
\end{solution}
\begin{example}{(直径圆习题)}{}
    已知椭圆$\dfrac{x^2}{a^2}+\dfrac{y^2}{b^2}=1$,弦$AB$过定点$(m,0)$,其直径圆与椭圆交于点$C,D$,证明直线$CD$过定点.
\end{example}
\begin{solution}
    设$AB:x=ty+m$,其中$m$已知,联立韦达:\[\begin{cases}3x^{2}+4y^{2}-12=0\\x=ty+m&\end{cases}\Rightarrow\begin{cases}(3t^{2}+4)y^{2}+6tmy+3m^{2}-12=0\\(3t^{2}+4)x^{2}-8mx+4m^{2}-12t^{2}=0\end{cases}\]
    两式相加得到:\[(3t^2+4)x^2+(3t^2+4)y^2-8mx+6tmy+7m^2-12-12t^2=0\]
    引入$\lambda(3x^2+4y^2=12)=0$并配凑$(x+ty+...)(x-ty+...)$结构,就有
    \[-(t^2+1)(3x^{2}+4y^{2}-12)+(3t^{2}+4)(x^{2}+y^{2})-8mx+6tmy+7m^{2}-12-12t^{2}=0\]
    因式分解成\[(x-ty-m)(x+ty-7m)=0\],所以$CD$过定点$(7m,0)$.
\end{solution}
\begin{example}{(角平分线)}
    已知椭圆$\dfrac{x^2}{a^2}+\dfrac{y^2}{b^2}=1$,弦$AB$过定点$P(1,0)$,$C(4,0)$,证明$x$轴平分$\angle{ACB}$
\end{example}
\begin{solution}
    由$AB:x=ty+1$,由对称性可以设$A'B':x=-ty+1$,得到退化二次曲线\[(x-ty-1)(x+ty-1)=0\]
    待定$\lambda(3x^2+4y^2-12)=0$,再根据要配凑的结构
    \[(x-?y-4)(x+?y-4)=0\]中的常数项,得到$\lambda=-\dfrac14$
    \[-(3x^{2}+4y^{2}-12)+4(x-ty-1)(x+ty-1)=0\]
    得到$(x-4)^2=(4+t^2)y^2$,这等价于\[(x+\sqrt{4+t^2}y-4)(x-\sqrt{4+t^2}y-4)=0\]
    这个方程的根显然为$(x_1,y_1),(x_2,y_2)$,所以斜率相反,角平分线得证.
\end{solution}
\newpage
\begin{example}{(抹茶奶绿)}{}
    椭圆$x^2+4y^2=4$的右焦点为$(\sqrt3,0)$,点$M,N$在$x$轴上方的椭圆上,且$\triangle{MNF}$的外心$D$在$x$轴上,$\sqrt{3}|FM||FN|=2|MN|$,求$S_{\triangle{MNF}}$
\end{example}
\begin{solution}
    设直线方程$MN:y=kx+m\Leftrightarrow kx=y-m$,并联立
    \vspace{-5pt}
    \[\begin{cases}(1):(4k^2+1)x^2+8kmx+4m^2-4=0\\(2):(4k^2+1)y^2-2my+m^2-4k^2=0\\(3):y-kx-m=0\end{cases}\vspace{-5pt}\]
    由$(1)+(2)+2m(3)$相加得到过$A,B$两点,且圆心在$x$轴上的圆\vspace{-5pt}
    \[(4k^2+1)(x^2+y^2)+6kmx+3m^2-4k^2-4=0\vspace{-5pt}\]
    代入$F(\sqrt3,0)$,得到直线$MN$到焦点$(\sqrt3,0)$的距离:$d$(后面要根据这个式子求斜率):\vspace{-5pt}
    \[8k^2+6\sqrt{3}km+3m^2=1\Leftrightarrow d=\dfrac{\sqrt{3}k+m}{\sqrt{k^2+1}}=\dfrac{1}{\sqrt3}\vspace{-5pt}\]
    此时设圆半径为$R$,并由等面积法推算得:\vspace{-5pt}\[\begin{cases}S_{\triangle{MNF}}=\dfrac12|FM||FN|\sin\angle{MFN}=\dfrac12|MN|d=\dfrac{1}{2}\dfrac{1}{\sqrt3}|MN|\\\sqrt{3}|FM||FN|=2|MN|\end{cases}\Rightarrow \sin\angle{MFN}=\dfrac12\vspace{-5pt}\]
    于是设$M(x_1,y_1),N(x_2,y_2)$由向量公式得到
    \begin{align*}
        \cos\angle{MFN}=&\dfrac{\sqrt3}{2}=\dfrac{\overrightarrow{FM}\cdot\overrightarrow{FN}}{|FM||FN|}=\dfrac{(x_1-\sqrt3)(x_2-\sqrt3)+y_1y_2}{(2-\frac{\sqrt3}{2}x_1)(2-\frac{\sqrt3}{2}x_2)}=\dfrac{(x_1-\sqrt3)(x_2-\sqrt3)+y_1y_2}{\frac34(\frac{4}{\sqrt3}-x_1)(\frac{4}{\sqrt3}-x_2)}\\
        =&\dfrac{4}{3}\dfrac{12k^2+8\sqrt{3}km+4m^2-1+m^2-4k^2}{(4k^2+1)\frac{16}{3}+\frac{32}{\sqrt3}km+4m^2-4}=\dfrac{8k^2+8\sqrt{3}km+5m^2-1}{16k^2+8\sqrt{3}km+3m^2+1}\\
        =&\frac{8k^2+8\sqrt{3}km+5m^2-(8k^2+6\sqrt{3}km+3m^2)}{16k^2+8\sqrt{3}km+3m^2+8k^2+6\sqrt{3}km+3m^2}=\frac{\sqrt3km+m^2}{12k^2+7\sqrt3km+3m^2}
    \end{align*}
    齐次化后,设$t=\dfrac{m}{k}$,得到$(3\sqrt3-2)t^2+(21-2\sqrt3)t+12\sqrt3=0$解得横截距,另一个根舍去:\vspace{-5pt}
    \[t_1=\dfrac{2\sqrt3-21-2\sqrt3-3}{2(3\sqrt3-2)}=\dfrac{12}{2-3\sqrt3},t_1^2=\dfrac{144}{31-12\sqrt3},t_2=-\sqrt3\vspace{-5pt}\]
    然后根据前面求出的距离,利用斜率的定义计算:
    \[k^2=\dfrac{d^2}{(-\frac{m}{k}-\sqrt3)^2-d^2}=\dfrac{1}{3(\frac{3+2\sqrt3}{3\sqrt3-2})^2-1}=\dfrac{31-12\sqrt3}{3(3+2\sqrt3)^2-(3\sqrt3-2)^2}=\dfrac{1}{16}\dfrac{31-12\sqrt3}{2+3\sqrt3}\vspace{-5pt}\]
    \[m^2=k^2\dfrac{m^2}{k^2}=\dfrac{1}{16}\dfrac{31-12\sqrt3}{2+3\sqrt3}\dfrac{144}{31-12\sqrt3}=\dfrac{9}{2+3\sqrt3},km=\dfrac34\dfrac{2-3\sqrt3}{2+3\sqrt3}\]\vspace{-10pt}
    最后得到面积:
    \begin{align*}
        S_{\triangle{MNF}}&=\dfrac12\sin\angle{MFN}|FM||FN|=\dfrac{1}{4}|FM||FN|=\dfrac{16k^2+8\sqrt{3}km+3m^2+1}{4k^2+1}\\
        &=\dfrac{(31-12\sqrt3)+6\sqrt{3}(2-3\sqrt3)+27+2+3\sqrt3}{\frac14(31-12\sqrt3)+2+3\sqrt3}=\dfrac{2+\sqrt3}{13}
    \end{align*}
\end{solution}
\newpage
\begin{example}{(抹茶奶绿)}{}
    椭圆$x^2+4y^2=4$的右焦点为$(\sqrt3,0)$,点$M,N$在$x$轴上方的椭圆上,且$\triangle{MNF}$的外心$D$在$x$轴上,$\sqrt{3}|FM||FN|=2|MN|$,求$S_{\triangle{MNF}}$
\end{example}
\begin{solution}
    由于本题中圆和椭圆都能削掉半边变成“函数”,所以我们可以直接联立它们:
    \[\begin{cases}(x-m)^2+y^2=r^2\\\dfrac{x^2}{4}+y^2=1\end{cases}\Rightarrow \dfrac{3}{4}x^2-2mx+2\sqrt{3}m-2=0\]
    设$M(x_1,y_1),N(x_2,y_2)$,由$\sqrt{3}|FM||FN|=2|MN|$得到:
    \begin{align*}|MN|=&\dfrac{\sqrt3}{2}\left(2-\dfrac{\sqrt3}{2}x_1\right)\left(2-\dfrac{\sqrt3}{2}x_2\right)=\dfrac{3\sqrt3}{8}\left(\dfrac{4}{\sqrt3}-x_1\right)\left(\dfrac{4}{\sqrt3}-x_2\right)\\=&\dfrac{3\sqrt3}{8}\dfrac43\left(\dfrac{3}{4}\dfrac{16}{3}-\dfrac{8m}{\sqrt3}+2m-2\right)
    =\dfrac{\sqrt3}{2}(2-\dfrac{2}{\sqrt{3}}m)=\sqrt3-m\end{align*}
    由于圆经过椭圆焦点,有\[(\sqrt3-m)^2=r^2\Rightarrow \sqrt3-m=|MN|=r=\dfrac{\sqrt{3}}{2}|FM||FN|\]
    设圆心为$D$,有等边三角形$MDN$,圆周角定理得到$\angle{MFN}=\dfrac{\pi}{6}$,所求面积为
    \[S_{\triangle{MNF}}=\dfrac12\sin\angle{MFN}|FM||FN|=\dfrac{1}{4}|FM||FN|=\dfrac{\sqrt3}{6}r\]
    我们要求出$r$,所以还少一个方程,利用三角形$MNF$余弦定理:
    \[|MF|^2+|NF|^2-\sqrt3|MF||NF|=|MF|^2+|NF|^2-2r=|MN|^2=r^2\Rightarrow |MF|^2+|NF|^2=r^2+2r\]
    为了引入$m$,利用恒等式$|MF|^2+|NF|^2+2|MF||NF|=(|MF|+|NF|)^2$,韦达定理得:
    \[|MF|+|NF|=4-\dfrac{\sqrt3}{2}(x_1+x_2)=4-\dfrac{\sqrt3}{2}\dfrac{8}{3}m=4-\dfrac{4}{\sqrt3}m\]
    那么\[|MF|^2+|NF|^2=(|MF|+|NF|)^2-2|MF||NF|=(4-\dfrac{4}{\sqrt3}m)^2-\dfrac{4}{\sqrt3}r=r^2+2r\]
    代入$\sqrt3-m=r$得到\[\left(4-\dfrac{4}{\sqrt3}(\sqrt3-r)\right)^2-\dfrac{4}{\sqrt3}r=r^2+2r\Rightarrow r=\dfrac{6+4\sqrt3}{13}\]
    得到$S_{\triangle{MNF}}=\dfrac{\sqrt3}{6}r=\dfrac{2+\sqrt3}{13}$
\end{solution}
