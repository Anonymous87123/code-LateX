\chapter{神经网络与深度学习基础}

\section{启发性思考}
深度学习的核心思想是模仿人脑处理信息的方式。人脑中约有860亿个神经元,每个神经元通过树突接收信号,在细胞体整合,若超过某个阈值则通过轴突释放信号(电脉冲)至其他神经元。这一“全有或全无”的激发模式启发了最早的数学模型——\textbf{人工神经元}(或称感知器)。

本章将系统学习:
\begin{itemize}
    \item 人工神经元的基本组成:\textbf{权重}、\textbf{偏置}、\textbf{激活函数}
    \item 几种经典激活函数(Sigmoid, tanh, ReLU)的特性与对比
    \item \textbf{前馈神经网络}的结构与矩阵表示
    \item \textbf{Softmax函数}及其温度系数的妙用
    \item 神经网络训练的完整流程:\textbf{前向传播}、\textbf{损失函数}、\textbf{梯度下降}与核心算法——\textbf{反向传播(Backpropagation, BP)}
\end{itemize}

\section{神经元:深度学习的基石}

\subsection{数学模型}
一个神经元接收 $n$ 个输入 $x_1, x_2, \ldots, x_n$,每个输入对应一个权重 $w_i$,并有一个偏置 $b$。神经元首先计算加权和(称为\textbf{预激活值}):
\[
z = \sum_{i=1}^{n} w_i x_i + b
\]
然后将 $z$ 送入一个非线性函数 $f$,得到该神经元的\textbf{激活输出}:
\[
a = f(z)
\]

\subsection{关键组件详解}

\begin{definition}[权重(Weight)]
权重 $w_i$ 衡量第 $i$ 个输入对神经元输出的\textbf{重要性}或\textbf{贡献度}。在训练过程中,权重会被不断调整,以学习输入与输出之间的映射关系。
\begin{itemize}
    \item \textbf{理性理解}:权重是线性变换的参数,决定了输入空间到输出空间的投影方向与尺度。
    \item \textbf{感性理解}:好比人际交往中,不同朋友说的话对你决策的影响程度不同,权重就是这种“信任度”或“影响力”的量化。
\end{itemize}
\end{definition}

\begin{definition}[偏置(Bias)]
偏置 $b$ 是一个常数项,它\textbf{平移}了激活函数的阈值。即使所有输入为零,神经元也可能被激活(若 $b > 0$)。
\begin{itemize}
    \item \textbf{理性理解}:偏置提供了模型的平移自由度,使决策边界不必通过原点,增强了模型的表达能力。
    \item \textbf{感性理解}:可以理解为一个人的“先天倾向”或“初始立场”。例如,即使没有任何外部信息(输入为零),一个人也可能因为内在性格(偏置)而倾向于做出某个决定。
\end{itemize}
\end{definition}

\begin{definition}[激活函数(Activation Function)]
激活函数 $f$ 引入了\textbf{非线性}。如果没有非线性,无论堆叠多少层,整个网络仍然等价于一个线性模型,无法学习复杂的模式。
\begin{itemize}
    \item \textbf{理性理解}:非线性变换是神经网络能够成为“通用函数逼近器”的关键。
    \item \textbf{感性理解}:就像人脑神经元对输入信号的响应不是简单的线性叠加,而是有一个“激发”或“抑制”的非线性过程。
\end{itemize}
\end{definition}

\section{激活函数:从Sigmoid到ReLU}

\subsection{经典激活函数}

\subsubsection{Sigmoid函数}
\[
\sigma(z) = \frac{1}{1 + e^{-z}} \in (0, 1)
\]
\begin{itemize}
    \item \textbf{优点}:输出平滑、连续,导数易于计算 $\sigma'(z) = \sigma(z)(1-\sigma(z))$。输出范围 $(0,1)$ 适合表示概率。
    \item \textbf{缺点}:
    \begin{enumerate}
        \item \textbf{梯度消失(Vanishing Gradient)}:当 $|z|$ 很大时,导数接近0。在深层网络中,梯度反向传播时连乘会导致前面层的梯度极小,参数更新缓慢甚至停滞。
        \item 输出不是零中心的(均值>0),可能导致后续层的输入发生偏移,影响梯度下降效率。
        \item 计算涉及指数,相对较慢。
    \end{enumerate}
    \item \textbf{用途}:二分类问题的输出层(表示概率)。
\end{itemize}

\subsubsection{双曲正切函数(tanh)}
\[
\tanh(z) = \frac{e^{z} - e^{-z}}{e^{z} + e^{-z}} \in (-1, 1)
\]
\begin{itemize}
    \item \textbf{优点}:输出是\textbf{零中心}的(均值为0),这通常有助于加速收敛。同样平滑连续。
    \item \textbf{缺点}:同样存在梯度消失问题。
    \item \textbf{用途}:适合需要输出有界且零中心的场景,如RNN的隐藏状态。
\end{itemize}

\subsubsection{线性整流单元(ReLU)}
\[
\text{ReLU}(z) = \max(0, z)
\]
\begin{itemize}
    \item \textbf{优点}:
    \begin{enumerate}
        \item \textbf{计算高效}:只需比较和取最大值。
        \item \textbf{缓解梯度消失}:在正半轴导数为1,梯度可以无衰减地反向传播。
        \item 在实践中,ReLU网络通常比Sigmoid/tanh网络\textbf{收敛快得多}。
    \end{enumerate}
    \item \textbf{缺点}:
    \begin{enumerate}
        \item \textbf{死亡ReLU问题(Dying ReLU)}:一旦输入落入负半轴(例如,由于大的负偏置或激烈的权重更新),输出和梯度都为零,神经元“死亡”,不再参与后续学习。
        \item 输出非零中心,且无上界,可能导致梯度爆炸(需配合适当初始化与归一化)。
    \end{enumerate}
\end{itemize}

\subsection{激活函数对比与进阶选择}

\begin{table}[h]
\centering
\caption{激活函数对比}
\begin{tabular}{ccccc}
\textbf{函数} & \textbf{输出范围} & \textbf{零中心} & \textbf{梯度消失} & \textbf{计算速度} \\
Sigmoid & (0,1) & 否 & 严重 & 慢 \\
tanh & (-1,1) & \textbf{是} & 严重 & 慢 \\
ReLU & [0, $\infty$) & 否 & \textbf{缓解} & \textbf{极快} \\
\end{tabular}
\end{table}

\subsubsection{ReLU的改进变体}
\begin{itemize}
    \item \textbf{Leaky ReLU}:$f(z) = \max(\alpha z, z)$,其中 $\alpha$ 是一个小的正数(如0.01)。在负半轴保留一个微小斜率,避免神经元完全死亡。
    \item \textbf{PReLU(Parametric ReLU)}:将 $\alpha$ 作为可学习参数,让网络自行决定负半轴的斜率。
    \item \textbf{ELU(Exponential Linear Unit)}:$f(z) = \begin{cases} z, & z>0 \\ \alpha(e^z-1), & z \leq 0 \end{cases}$。负半轴平滑渐近,输出接近零中心。
    \item \textbf{GELU / SiLU(Sigmoid Linear Unit)}:$f(z) = z \cdot \sigma(z)$。这是一种平滑的非单调激活函数,被BERT、GPT等Transformer模型采用,性能优异。
\end{itemize}

\textbf{实践建议}:现代深度网络中,ReLU及其变体(尤其是GELU)已成为隐藏层的默认选择。Sigmoid多用于输出层(二分类概率),tanh在特定架构(如LSTM)中仍有应用。

\section{前馈神经网络:层叠的威力}

\subsection{网络结构}
前馈神经网络(Feed-forward Neural Network)由多层神经元组成,信息\textbf{单向流动}:从输入层,经过若干隐藏层,到达输出层。
\begin{itemize}
    \item \textbf{不存在循环或反馈}(有循环的模型是循环神经网络RNN)。
    \item 每一层的神经元接收前一层所有神经元的输出作为输入,并输出给下一层所有神经元(全连接)。
\end{itemize}

\subsection{优雅高效的矩阵表示}
设网络共 $L$ 层(输入层为第0层,输出层为第 $L$ 层)。第 $l$ 层有 $d^{(l)}$ 个神经元。
\begin{itemize}
    \item 第 $l$ 层的\textbf{权重矩阵} $W^{(l)} \in \mathbb{R}^{d^{(l)} \times d^{(l-1)}}$:$W_{ij}^{(l)}$ 表示第 $l-1$ 层第 $j$ 个神经元到第 $l$ 层第 $i$ 个神经元的连接权重。
    \item 第 $l$ 层的\textbf{偏置向量} $b^{(l)} \in \mathbb{R}^{d^{(l)}}$。
    \item 第 $l$ 层的\textbf{输入}(即前一层的激活输出)记为 $a^{(l-1)} \in \mathbb{R}^{d^{(l-1)}}$。
\end{itemize}
则第 $l$ 层的\textbf{预激活值} $z^{(l)}$ 和\textbf{激活输出} $a^{(l)}$ 为:
\[
z^{(l)} = W^{(l)} a^{(l-1)} + b^{(l)}, \quad a^{(l)} = f^{(l)}(z^{(l)})
\]
其中 $f^{(l)}$ 是第 $l$ 层的激活函数。这种矩阵形式极其适合GPU并行计算。

\subsection{Softmax函数:多分类的“裁判”}

\subsubsection{标准Softmax}
对于 $K$ 类的分类问题,输出层通常有 $K$ 个神经元,输出一个未经归一化的得分向量(logits)$z \in \mathbb{R}^K$。Softmax将其转换为一个合法的概率分布:
\[
\text{Softmax}(z)_i = \frac{e^{z_i}}{\sum_{j=1}^{K} e^{z_j}}, \quad i=1,\ldots,K
\]
\begin{itemize}
    \item \textbf{指数作用}:将实数映射到正数,并\textbf{放大分数差距}(高分更高,低分更低)。
    \item \textbf{归一化}:确保所有输出之和为1,符合概率公理。
    \item \textbf{用途}:多分类任务的输出层,常与交叉熵损失函数搭配。
\end{itemize}

\subsubsection{温度系数(Temperature)}
引入温度系数 $T>0$,控制概率分布的“软硬”程度:
\[
\text{Softmax}(z, T)_i = \frac{e^{z_i / T}}{\sum_{j=1}^{K} e^{z_j / T}}
\]
\begin{itemize}
    \item $T=1$:标准Softmax。
    \item $T>1$(高温):概率分布更“平坦”,模型对各类别的不确定性增加。常用于\textbf{知识蒸馏}(Teacher模型使用高温度产生软标签,Student模型学习之)。
    \item $T<1$(低温):概率分布更“尖锐”,模型置信度更高。当 $T \to 0^+$ 时,Softmax退化为argmax(即one-hot向量)。
\end{itemize}

\section{神经网络的训练:让模型学会思考}

训练神经网络本质是寻找一组参数(所有权重和偏置),使模型在训练数据上的预测损失最小。这是一个典型的优化问题。

\subsection{步骤一:定义网络结构}
\begin{itemize}
    \item \textbf{输入层维度}:由数据特征决定。
    \item \textbf{隐藏层数量与宽度}:决定模型容量。层数越多、每层神经元越多,模型拟合复杂函数的能力越强,但也更容易过拟合、计算成本更高。
    \item \textbf{输出层维度}:由任务决定(如二分类为1,多分类为类别数,回归为1)。
\end{itemize}
\textbf{数据预处理}:通常将输入特征标准化(均值0,方差1)或归一化到 $[0,1]$,以加速收敛、提升性能。

\subsection{步骤二:定义损失函数(Loss Function)}
损失函数量化模型预测 $\hat{y}$ 与真实标签 $y$ 之间的差距。
\begin{itemize}
    \item \textbf{均方误差(MSE)}:用于回归任务。
    \[
    J(W,b) = \frac{1}{2m} \sum_{i=1}^{m} (h_{W,b}(x^{(i)}) - y^{(i)})^2
    \]
    其中 $m$ 是样本数,乘以 $\frac{1}{2}$ 是为了后续求导方便(与平方项的导数2抵消)。
    \item \textbf{交叉熵损失(Cross-Entropy)}:用于分类任务,尤其是与Softmax输出层结合。
    \[
    J(W,b) = -\frac{1}{m} \sum_{i=1}^{m} \sum_{k=1}^{K} y_k^{(i)} \log(\hat{y}_k^{(i)})
    \]
    其中 $K$ 是类别数,$y_k^{(i)}$ 是样本 $i$ 的真实标签(one-hot向量),$\hat{y}_k^{(i)}$ 是模型预测的概率。
    \begin{remark}
        交叉熵衡量两个概率分布之间的“距离”。当预测分布与真实分布完全一致时,交叉熵最小(等于真实分布的熵)。
    \end{remark}
\end{itemize}

\subsection{步骤三:梯度下降优化}
目标:最小化 $J(W,b)$。梯度下降(Gradient Descent)是核心优化算法。
\[
W^{(l)} := W^{(l)} - \alpha \frac{\partial J}{\partial W^{(l)}}, \quad b^{(l)} := b^{(l)} - \alpha \frac{\partial J}{\partial b^{(l)}}
\]
其中 $\alpha$ 是\textbf{学习率(Learning Rate)},控制更新步长。
\begin{itemize}
    \item \textbf{理性}:沿着损失函数梯度的反方向(最速下降方向)更新参数。
    \item \textbf{感性}:好比在山区蒙眼下山,用脚试探周围最陡的下坡方向,然后迈一小步,重复此过程。
\end{itemize}
\textbf{核心挑战}:如何高效计算所有参数的梯度 $\frac{\partial J}{\partial W^{(l)}}$ 和 $\frac{\partial J}{\partial b^{(l)}}$?这正是反向传播要解决的。

\section{反向传播:深度学习的引擎}

反向传播(Backpropagation, BP)不是独立的优化算法,而是\textbf{利用链式法则高效计算梯度}的方法。它是神经网络训练的核心。

\subsection{误差项(Error Term)的定义}
对于第 $l$ 层的第 $i$ 个神经元,定义其误差项 $\delta_i^{(l)}$ 为损失函数对该神经元预激活值 $z_i^{(l)}$ 的偏导数:
\[
\delta_i^{(l)} = \frac{\partial J}{\partial z_i^{(l)}}
\]
误差项衡量了该神经元对最终损失的“责任”大小。

\subsection{反向传播的四步流程}

\subsubsection{1. 前向传播}
给定输入 $x$,依次计算并保存每一层的 $z^{(l)}$ 和 $a^{(l)}$($a^{(0)} = x$)。

\subsubsection{2. 计算输出层误差}
\[
\delta^{(L)} = \nabla_{a^{(L)}} J \odot f'(z^{(L)})
\]
其中 $\nabla_{a^{(L)}} J$ 是损失对输出层激活的梯度,$\odot$ 表示逐元素乘法(Hadamard积)。
\begin{itemize}
    \item 对于MSE损失和线性输出:$\nabla_{a^{(L)}} J = a^{(L)} - y$。
    \item 对于交叉熵损失和Softmax输出:$\nabla_{a^{(L)}} J = a^{(L)} - y$(这是一个美妙的巧合,简化了计算)。
\end{itemize}

\subsubsection{3. 反向传播误差(关键递推公式)}
从 $l = L-1$ 到 $l=1$,逐层计算:
\[
\boxed{\delta^{(l)} = \left( (W^{(l+1)})^{\top} \delta^{(l+1)} \right) \odot f'(z^{(l)}) }
\]
\begin{itemize}
    \item \textbf{理性解释}:后一层的误差 $\delta^{(l+1)}$ 通过权重矩阵的转置 $(W^{(l+1)})^{\top}$ 传播回前一层,再乘以当前层激活函数的导数,进行调制。
    \item \textbf{感性解释}:将最终错误(损失)归咎于每一层神经元的“失误”。高层神经元的错误分摊给其输入连接的底层神经元,分摊比例就是连接的权重,同时还要考虑该底层神经元当时是否处于“敏感”状态(由 $f'(z^{(l)})$ 体现)。
\end{itemize}

\subsubsection{4. 计算参数梯度}
一旦得到 $\delta^{(l+1)}$,第 $l$ 层参数的梯度就非常简洁:
\[
\boxed{\frac{\partial J}{\partial W^{(l)}} = \delta^{(l+1)} (a^{(l)})^{\top}}, \quad \boxed{\frac{\partial J}{\partial b^{(l)}} = \delta^{(l+1)}}
\]
\begin{itemize}
    \item \textbf{权重梯度}:等于后一层的误差 $\delta^{(l+1)}$ 与前一层的激活 $a^{(l)}$ 的外积。
    \item \textbf{偏置梯度}:就是后一层的误差 $\delta^{(l+1)}$ 本身。
\end{itemize}

\subsubsection{5. 更新参数}
使用梯度下降(或其变体如Adam)更新所有参数:
\[
W^{(l)} \leftarrow W^{(l)} - \alpha \frac{\partial J}{\partial W^{(l)}}, \quad b^{(l)} \leftarrow b^{(l)} - \alpha \frac{\partial J}{\partial b^{(l)}}
\]

\subsubsection{6. 批处理与梯度累加}
实际中我们使用小批量(mini-batch)数据计算梯度。步骤4中计算出的梯度通常是该批次内所有样本梯度的\textbf{平均值}。这既利用了向量化计算的效率,又引入了噪声,有助于逃离局部极小点。

\subsection{反向传播推导(选读)}
理解推导有助于深入掌握BP本质。核心是链式法则。我们欲求 $\frac{\partial J}{\partial W_{ij}^{(l)}}$。注意 $J$ 通过 $z_i^{(l+1)}$ 依赖于 $W_{ij}^{(l)}$。
\begin{align*}
\frac{\partial J}{\partial W_{ij}^{(l)}} &= \frac{\partial J}{\partial z_i^{(l+1)}} \cdot \frac{\partial z_i^{(l+1)}}{\partial W_{ij}^{(l)}} \\
&= \delta_i^{(l+1)} \cdot \frac{\partial}{\partial W_{ij}^{(l)}} \left( \sum_k W_{ik}^{(l)} a_k^{(l)} + b_i^{(l)} \right) \\
&= \delta_i^{(l+1)} \cdot a_j^{(l)}
\end{align*}
这正是 $\frac{\partial J}{\partial W^{(l)}} = \delta^{(l+1)} (a^{(l)})^{\top}$ 的第 $(i, j)$ 个元素。

误差反向传播公式的推导类似,考虑 $J$ 通过所有 $z_j^{(l+1)}$ 依赖于 $z_i^{(l)}$:
\begin{align*}
\delta_i^{(l)} = \frac{\partial J}{\partial z_i^{(l)}} &= \sum_j \frac{\partial J}{\partial z_j^{(l+1)}} \cdot \frac{\partial z_j^{(l+1)}}{\partial z_i^{(l)}} \\
&= \sum_j \delta_j^{(l+1)} \cdot \frac{\partial}{\partial z_i^{(l)}} \left( \sum_k W_{jk}^{(l+1)} f(z_k^{(l)}) + b_j^{(l+1)} \right) \\
&= \sum_j \delta_j^{(l+1)} \cdot W_{ji}^{(l+1)} \cdot f'(z_i^{(l)}) \\
&= \left( \sum_j W_{ji}^{(l+1)} \delta_j^{(l+1)} \right) f'(z_i^{(l)})
\end{align*}
括号内即为 $( (W^{(l+1)})^{\top} \delta^{(l+1)} )_i$,因此有 $\delta^{(l)} = ((W^{(l+1)})^{\top} \delta^{(l+1)}) \odot f'(z^{(l)})$。

\section{引言:从拟合数据到泛化世界}

深度学习的核心目标是构建能够在\textbf{未见过的数据}上表现良好的模型。这就像学生在期末考试中遇到新题时能灵活运用所学知识,而不是只会做练习题。本章将系统学习如何评估模型性能、诊断常见问题、优化模型泛化能力,并探索深度学习从神经网络到现代架构的发展历程。

\section{模型性能评估:不只是看分数}

\subsection{数据集划分的艺术}

为了模拟模型在真实世界中的表现,我们需要将数据划分为三个互斥的部分:
\begin{itemize}
    \item \textbf{训练集(Training Set)}:模型的"教科书",用于学习参数(权重和偏置)。
    \item \textbf{验证集(Validation Set)}:模型的"模拟考试",用于:
    \begin{enumerate}
        \item \textbf{超参数调优}:选择学习率、网络层数、神经元数量等。
        \item \textbf{模型选择}:比较不同架构的性能。
        \item \textbf{早停法(Early Stopping)}的参考依据。
    \end{enumerate}
    \item \textbf{测试集(Test Set)}:模型的"期末考试",在整个训练和调优过程中只使用一次,用于给出模型性能的\textbf{无偏估计}。
\end{itemize}

\textbf{黄金法则}:测试集必须与训练/验证集完全隔离,在整个流程的最后才使用一次。任何基于测试集结果的模型调整都会导致对泛化能力的\textbf{乐观估计}。

\subsection{K-折交叉验证:小数据集的救星}

当数据量有限时,简单的单次划分可能因随机性导致评估结果不稳定。K-折交叉验证提供了更稳健的解决方案。

\textbf{算法流程}:
\begin{enumerate}
    \item 将训练数据(不包含测试集)随机划分为K个大小相似的互斥子集("折"),通常K=5或10。
    \item 进行K次循环,在第i次迭代中:
    \begin{itemize}
        \item 使用第i折作为验证集
        \item 使用其余K-1折作为训练集
    \end{itemize}
    \item 将K次迭代得到的性能指标取平均值,作为模型性能的最终评估。
\end{enumerate}

\begin{remark}
    \textbf{优点}:
    \begin{itemize}
        \item 充分利用有限数据:每个样本都既当过训练数据也当过验证数据。
        \item 减少评估结果的随机性:基于多次评估的平均结果更可靠。
    \end{itemize}
    \textbf{缺点}:计算成本是单次划分的K倍。
\end{remark}

\section{诊断模型问题:欠拟合与过拟合}

模型性能不佳通常源于两个根本问题:

\begin{figure}[h]
\centering
\begin{subfigure}{0.45\textwidth}
\centering
\begin{tikzpicture}[scale=0.8]
\draw[->] (0,0) -- (4,0) node[right] {模型复杂度};
\draw[->] (0,0) -- (0,3) node[above] {误差};
\draw[blue, thick] plot[smooth] coordinates {(0.2,2.5) (2,0.8) (3.8,2.3)};
\draw[red, thick] plot[smooth] coordinates {(0.2,0.7) (1.5,0.4) (3.8,1.8)};
\node[blue] at (3.5,2.5) {训练误差};
\node[red] at (3.5,1.3) {测试误差};
\draw[dashed] (2,0) -- (2,3);
\node at (2, -0.3) {最佳复杂度};
\end{tikzpicture}
\caption{误差随模型复杂度的变化}
\end{subfigure}
\hfill
\begin{subfigure}{0.45\textwidth}
\centering
\begin{tikzpicture}[scale=0.8]
\draw[->] (0,0) -- (4,0) node[right] {训练轮数};
\draw[->] (0,0) -- (0,3) node[above] {误差};
\draw[blue, thick] plot[smooth] coordinates {(0.2,2.2) (1,1.5) (2,1.2) (3,0.9) (3.8,0.8)};
\draw[red, thick] plot[smooth] coordinates {(0.2,2.3) (1,1.7) (1.5,1.5) (2,1.6) (2.5,1.9) (3.8,2.4)};
\draw[dashed] (1.5,0) -- (1.5,3);
\node[blue] at (3.2,1.1) {训练误差};
\node[red] at (3.2,2.1) {测试误差};
\node at (1.5, -0.3) {早停点};
\end{tikzpicture}
\caption{早停法的原理}
\end{subfigure}
\caption{欠拟合与过拟合的图示}
\end{figure}

\subsection{欠拟合(Underfitting)}
\begin{itemize}
    \item \textbf{表现}:训练误差和测试误差都很高。
    \item \textbf{原因}:模型太简单,无法捕捉数据中的基本规律。
    \item \textbf{类比}:用线性方程去拟合抛物线数据。
    \item \textbf{解决方案}:
    \begin{enumerate}
        \item 增加模型复杂度(更多层、更多神经元)。
        \item 使用更强大的特征。
        \item 减少正则化强度。
        \item 延长训练时间。
    \end{enumerate}
\end{itemize}

\subsection{过拟合(Overfitting)}
\begin{itemize}
    \item \textbf{表现}:训练误差很低,但测试误差很高。
    \item \textbf{原因}:模型过于复杂,记住了训练数据中的噪声和偶然性。
    \item \textbf{类比}:死记硬背所有例题,但遇到新题不会做。
    \item \textbf{解决方案}:
    \begin{enumerate}
        \item 获取更多训练数据(最有效但成本高)。
        \item 使用正则化技术。
        \item 简化模型结构。
        \item 早停法。
        \item Dropout(仅限神经网络)。
    \end{enumerate}
\end{itemize}

\subsection{偏差-方差权衡:统计学习理论的视角}

泛化误差可以分解为三个部分:
\[
\underbrace{\mathbb{E}[(y - \hat{f}(x))^2]}_{\text{泛化误差}} = \underbrace{[\text{Bias}(\hat{f}(x))]^2}_{\text{偏差}} + \underbrace{\text{Var}(\hat{f}(x))}_{\text{方差}} + \underbrace{\sigma^2}_{\text{不可约减误差}}
\]

\begin{itemize}
    \item \textbf{偏差(Bias)}:模型预测的期望值与真实值之间的差距,反映了模型的\textbf{系统性错误}。
    \begin{itemize}
        \item 高偏差:模型过于简单,无法拟合数据的基本模式(欠拟合)。
        \item \textbf{感性理解}:就像用直尺测量曲面,无论测多少次都有系统误差。
    \end{itemize}
    
    \item \textbf{方差(Variance)}:模型对训练数据变化的敏感度,反映了模型的\textbf{不稳定性}。
    \begin{itemize}
        \item 高方差:模型过于复杂,对训练数据中的随机噪声敏感(过拟合)。
        \item \textbf{感性理解}:就像用过于灵敏的天平,每次读数都因微小扰动而不同。
    \end{itemize}
    
    \item \textbf{不可约减误差(Irreducible Error)}:数据本身的噪声,任何模型都无法消除。
\end{itemize}

\textbf{权衡}:增加模型复杂度通常降低偏差但增加方差,减少模型复杂度则相反。理想的模型需要在两者之间找到平衡点。

\section{正则化技术:防止过拟合的利器}

正则化通过在损失函数中添加\textbf{惩罚项}来约束模型复杂度,防止过拟合。

\subsection{L1和L2正则化}

在损失函数中添加参数的惩罚项:
\[
J_{\text{reg}}(W, b) = J(W, b) + \lambda \cdot R(W)
\]
其中$\lambda > 0$是正则化强度超参数。

\begin{table}[h]
\centering
\caption{L1 vs L2正则化对比}
\begin{tabular}{lccc}
\textbf{类型} & \textbf{惩罚项} $R(W)$ & \textbf{数学形式} & \textbf{效果} \\
L1正则化(Lasso) & $\lambda \sum_{i=1}^n |w_i|$ & 绝对值之和 & \textbf{稀疏化}:将不重要特征的权重压缩到0 \\
L2正则化(Ridge) & $\frac{\lambda}{2} \sum_{i=1}^n w_i^2$ & 平方和 & \textbf{权重衰减}:使所有权重更小、更平均 \\
\end{tabular}
\end{table}

\subsubsection{几何解释}
\begin{figure}[h]
\centering
\begin{tikzpicture}[scale=0.8]
% L1正则化
\begin{scope}[shift={(0,0)}]
\draw[->] (-2,0) -- (2,0) node[right] {$w_1$};
\draw[->] (0,-2) -- (0,2) node[above] {$w_2$};
\draw[blue, thick] plot[smooth] coordinates {(-1.5,-0.8) (-0.5,0.2) (0.5,0.8) (1.5,1.5)};
\draw[red, thick] (-1.5,-1.5) -- (1.5,1.5) node[right] {$|w_1|+|w_2|=C$};
\draw[red, thick] (1.5,-1.5) -- (-1.5,1.5);
\fill[red] (0.5,0.8) circle (2pt);
\node at (0,-2.5) {L1正则化(菱形约束)};
\end{scope}

% L2正则化
\begin{scope}[shift={(5,0)}]
\draw[->] (-2,0) -- (2,0) node[right] {$w_1$};
\draw[->] (0,-2) -- (0,2) node[above] {$w_2$};
\draw[blue, thick] plot[smooth] coordinates {(-1.5,-0.8) (-0.5,0.2) (0.5,0.8) (1.5,1.5)};
\draw[red, thick] (0,0) circle (1.2);
\fill[red] (0.7,0.7) circle (2pt);
\node at (0,-2.5) {L2正则化(圆形约束)};
\end{scope}
\end{tikzpicture}
\caption{L1和L2正则化的几何解释}
\end{figure}

L1正则化的约束区域是菱形,最优解往往在角点上(某些权重为0),从而实现特征选择。L2正则化的约束区域是圆形,最优解使所有权重都较小且分布均匀。

\subsection{Dropout:训练时随机丢弃神经元}

Dropout是一种在\textbf{训练阶段}使用的正则化技术:
\begin{itemize}
    \item 每个训练步骤中,以概率$p$随机"丢弃"(设为0)每个神经元的输出。
    \item 迫使网络不依赖于任何单个神经元,学习更鲁棒的特征。
    \item 测试阶段使用所有神经元,但输出要乘以$1-p$(或训练时缩放)。
\end{itemize}

\begin{figure}[h]
\centering
\begin{tikzpicture}[
    neuron/.style={circle, draw, minimum size=0.8cm},
    dropout/.style={circle, draw, minimum size=0.8cm, fill=red!20}
]
% 原始网络
\begin{scope}[shift={(0,0)}]
\foreach \i in {1,...,4} {
    \node[neuron] (in\i) at (0,-\i*1.2) {};
    \node[neuron] (out\i) at (2,-\i*1.2) {};
    \draw[->] (in\i) -- (out\i);
}
\node at (1,-6) {标准网络};
\end{scope}

% Dropout网络
\begin{scope}[shift={(4,0)}]
\foreach \i in {1,...,4} {
    \ifnum\i=2
        \node[dropout] (in\i) at (0,-\i*1.2) {};
    \else
        \node[neuron] (in\i) at (0,-\i*1.2) {};
    \fi
    \ifnum\i=3
        \node[dropout] (out\i) at (2,-\i*1.2) {};
    \else
        \node[neuron] (out\i) at (2,-\i*1.2) {};
    \fi
    \ifnum\i=2
        \draw[dashed, gray] (in\i) -- (out\i);
    \else
        \draw[->] (in\i) -- (out\i);
    \fi
}
\node at (1,-6) {Dropout(训练时)};
\node[red] at (0,-2.4) {×};
\node[red] at (2,-3.6) {×};
\end{scope}
\end{tikzpicture}
\caption{Dropout示意图}
\end{figure}

\subsection{早停法(Early Stopping)}

监控验证集误差,当验证误差连续多个epoch不再下降(甚至上升)时停止训练。
\begin{itemize}
    \item \textbf{优点}:简单有效,不需要改变模型结构。
    \item \textbf{缺点}:需要额外的验证集,停止时机不易确定。
\end{itemize}

\section{从神经网络到深度学习:历史脉络}

\subsection{发展历程}

\subsubsection{早期探索(1940s-1980s)}
\begin{itemize}
    \item 1943年:McCulloch和Pitts提出M-P神经元模型,模拟生物神经元。
    \item 1958年:Rosenblatt提出感知机(Perceptron),可视为单层神经网络。
    \item 1969年:Minsky和Papert出版《Perceptrons》,指出单层感知机无法解决XOR等非线性问题,导致神经网络研究进入寒冬。
\end{itemize}

\subsubsection{复兴前夜(1980s-2006)}
\begin{itemize}
    \item 1986年:Rumelhart等人提出反向传播算法,解决了多层网络训练问题。
    \item 但仍受限于:数据不足、计算力有限、优化困难(梯度消失/爆炸)。
\end{itemize}

\subsubsection{复兴之年(2006)}
\begin{itemize}
    \item Hinton等人发表《A Fast Learning Algorithm for Deep Belief Nets》,提出:
    \begin{enumerate}
        \item \textbf{逐层贪婪预训练}:逐层训练受限玻尔兹曼机(RBM)。
        \item \textbf{微调}:用反向传播微调整个网络。
    \end{enumerate}
    \item 解决了深度网络训练难题,开启了深度学习新时代。
\end{itemize}

\subsubsection{爆发之年(2012)}
\begin{itemize}
    \item AlexNet在ImageNet竞赛中夺冠,将Top-5错误率从26\%降至15\%。
    \item 关键创新:使用ReLU、Dropout、GPU加速。
    \item 证明学习到的特征可超越手工设计特征。
\end{itemize}

\subsection{深度学习的四大支柱}

\begin{table}[h]
\centering
\caption{深度学习的四大支柱}
\begin{tabular}{p{0.22\textwidth}p{0.35\textwidth}p{0.35\textwidth}}
\toprule
\textbf{支柱} & \textbf{内容} & \textbf{意义} \\
\midrule
\textbf{数据} & 
大规模标注数据(ImageNet等),互联网产生海量数据 &
燃料,数据量决定模型性能上限 \\
\textbf{算力} &GPU并行计算,TPU专用芯片,云计算平台
 &
引擎,使训练深层网络成为可能 \\
\textbf{算法} &
新激活函数(ReLU),正则化(Dropout),优化器(Adam),新架构(ResNet)
 &
智慧,提升模型性能与训练稳定性 \\
\textbf{开源生态} &TensorFlow, PyTorch &
土壤,加速研究与应用普及 \\
\bottomrule
\end{tabular}
\end{table}

\subsection{为什么叫"深度学习"?}

\textbf{深}指网络层数多(通常>3层)。与传统机器学习相比:

\begin{table}[h]
\centering
\caption{传统机器学习 vs 深度学习}
\begin{tabular}{p{0.45\textwidth}p{0.45\textwidth}}
\toprule
\textbf{传统机器学习} & \textbf{深度学习} \\
\midrule
\textbf{特征工程}:人工设计特征 & \textbf{特征学习}:自动从数据中学习特征 \\
\textbf{流水线式}:多个独立步骤 & \textbf{端到端}:单个模型完成所有任务 \\
\textbf{浅层模型}:1-2层 & \textbf{深层模型}:数十至数百层 \\
\textbf{可解释性强} & \textbf{可解释性弱}(黑盒) \\
\textbf{数据需求少} & \textbf{数据需求大} \\
\bottomrule
\end{tabular}
\end{table}

\textbf{深度学习的关键优势}:通过多层非线性变换,自动学习\textbf{层次化特征表示}:
\begin{itemize}
    \item 浅层:边缘、纹理、颜色
    \item 中层:部件、形状
    \item 深层:语义概念、对象
\end{itemize}

\section{关键模型架构}

\subsection{自编码器(Autoencoders, AEs)}

\begin{figure}[h]
\centering
\begin{tikzpicture}[
    node distance=1cm,
    encoder/.style={rectangle, draw, minimum width=1.5cm, minimum height=0.8cm},
    decoder/.style={rectangle, draw, minimum width=1.5cm, minimum height=0.8cm},
    latent/.style={rectangle, draw, minimum width=1cm, minimum height=0.6cm, fill=gray!20}
]
% 输入
\node[encoder] (input) {输入 $x$};
\node[encoder, below=of input] (enc1) {编码器};
\node[latent, below=of enc1] (latent) {潜在表示 $z$};
\node[decoder, below=of latent] (dec1) {解码器};
\node[decoder, below=of dec1] (output) {重建 $\hat{x}$};

\draw[->] (input) -- (enc1);
\draw[->] (enc1) -- (latent);
\draw[->] (latent) -- (dec1);
\draw[->] (dec1) -- (output);
\draw[->, dashed] (output.south) -- ++(0,-0.5) -| ++(-2,0) |- (input.north) node[pos=0.25, above] {};

\node[left=0.5cm of latent] {瓶颈层};
\end{tikzpicture}
\caption{自编码器结构示意图}
\end{figure}

\textbf{核心思想}:学习数据的压缩表示(编码),然后重建输入。
\[
\begin{aligned}
\text{编码器:} & z = f(x) = \sigma(Wx + b) \\
\text{解码器:} & \hat{x} = g(z) = \sigma(W'z + b') \\
\text{损失函数:} & L(x, \hat{x}) = \|x - \hat{x}\|^2
\end{aligned}
\]

\textbf{变体与应用}:
\begin{itemize}
    \item \textbf{稀疏自编码器}:在损失函数中添加稀疏约束。
    \item \textbf{去噪自编码器}:输入加噪声,要求重建原始干净数据。
    \item \textbf{变分自编码器(VAE)}:学习数据的概率分布,可生成新样本。
\end{itemize}

\subsection{卷积神经网络(CNN)}

专门处理网格状数据(如图像),核心思想:\textbf{局部连接}、\textbf{权重共享}、\textbf{平移不变性}。

\begin{figure}[h]
\centering
\begin{tikzpicture}[scale=0.7]
% 输入图像
\draw[fill=blue!20] (0,0) rectangle (3,3);
\foreach \i in {0,...,2} {
    \foreach \j in {0,...,2} {
        \draw (\i,\j) rectangle (\i+1,\j+1);
    }
}
\node at (1.5, -0.5) {输入图像};

% 卷积核
\draw[fill=red!20] (4,1) rectangle (6,3);
\foreach \i in {0,...,1} {
    \foreach \j in {0,...,1} {
        \draw (4+\i,1+\j) rectangle (4+\i+1,1+\j+1);
    }
}
\node at (5, 0.5) {卷积核};

% 卷积操作
\draw[->] (3.5,1.5) -- (4,1.5);
\node at (3.75, 2) {卷积};

% 特征图
\draw[fill=green!20] (7,0) rectangle (9,2);
\foreach \i in {0,...,1} {
    \foreach \j in {0,...,1} {
        \draw (7+\i,\j) rectangle (7+\i+1,\j+1);
    }
}
\node at (8, -0.5) {特征图};

% 池化
\draw[->] (9.5,1) -- (10,1);
\draw[fill=yellow!20] (10.5,0) rectangle (11.5,1);
\draw (10.5,0) rectangle (11.5,1);
\node at (11, -0.5) {池化};
\end{tikzpicture}
\caption{CNN基本操作:卷积与池化}
\end{figure}

\subsubsection{核心组件}
\begin{enumerate}
    \item \textbf{卷积层}:使用卷积核在输入上滑动,计算局部加权和。
    \[
    (I * K)_{i,j} = \sum_{m} \sum_{n} I_{i+m, j+n} \cdot K_{m,n}
    \]
    
    \item \textbf{池化层}:下采样,减少参数量,增加平移不变性。
    \begin{itemize}
        \item 最大池化:$ \text{MaxPool}(x)_{i,j} = \max(x_{2i:2i+1, 2j:2j+1}) $
        \item 平均池化:$ \text{AvgPool}(x)_{i,j} = \text{mean}(x_{2i:2i+1, 2j:2j+1}) $
    \end{itemize}
    
    \item \textbf{全连接层}:将特征图展平后分类。
\end{enumerate}

\subsubsection{经典架构}
\begin{itemize}
    \item \textbf{AlexNet}(2012):首次使用ReLU、Dropout、GPU训练。
    \item \textbf{VGG}(2014):使用3×3小卷积核堆叠深层网络。
    \item \textbf{ResNet}(2015):残差连接解决梯度消失,训练极深网络(152层)。
\end{itemize}

\subsection{循环神经网络(RNN)与LSTM}

专门处理序列数据,具有\textbf{记忆能力}。

\subsubsection{标准RNN}
\begin{figure}[h]
\centering
\begin{tikzpicture}[
    node distance=1.2cm,
    neuron/.style={circle, draw, minimum size=0.8cm},
    unfold/.style={rectangle, draw, minimum width=1cm, minimum height=0.8cm}
]
% 折叠表示
\node[neuron] (A) at (0,0) {A};
\node[left=0.5cm of A] {$x_t$};
\node[above=0.2cm of A] {$h_{t-1}$};
\node[right=0.5cm of A] {$h_t$};
\draw[->] (-0.8,0) -- (A);
\draw[->] (0,0.8) -- (A);
\draw[->] (A) -- (0.8,0);
\draw[->] (A) edge[loop below] node[below] {循环} (A);

\end{tikzpicture}
\caption{RNN的表示}
\end{figure}

RNN的状态更新公式:
\[
h_t = \tanh(W_{hh}h_{t-1} + W_{xh}x_t + b_h)
\]

\textbf{问题}:标准RNN使用tanh激活函数,虽然能缓解梯度爆炸,但仍有梯度消失问题,难以学习长期依赖。

\subsubsection{LSTM:长短期记忆网络}

LSTM通过门控机制解决长期依赖问题。

\begin{figure}[h]
\centering
\begin{tikzpicture}[
    node distance=0.8cm,
    gate/.style={rectangle, draw, minimum width=1.2cm, minimum height=0.6cm},
    point/.style={circle, fill=black, inner sep=1.5pt},
    op/.style={circle, draw, minimum size=0.6cm}
]
% 细胞状态线
\draw[thick] (0,0) -- (10,0) node[right] {$C_t$};

% 门结构
\node[gate] (forget) at (2,1.5) {遗忘门 $f_t$};
\node[gate] (input) at (2,0.5) {输入门 $i_t$};
\node[gate] (output) at (2,-0.5) {输出门 $o_t$};
\node[op] (tanh) at (4,0.5) {$\tanh$};

% 输入
\node[point] (x) at (0,2) {};
\node[left=0.2cm of x] {$x_t$};
\node[point] (h) at (0,1) {};
\node[left=0.2cm of h] {$h_{t-1}$};

% 连接
\draw[->] (x) -- (0,2) |- (forget);
\draw[->] (x) -- (0,2) |- (input);
\draw[->] (x) -- (0,2) |- (output);
\draw[->] (h) -- (0,1) |- (forget);
\draw[->] (h) -- (0,1) |- (input);
\draw[->] (h) -- (0,1) |- (output);

% 乘法点
\node[op] (mult1) at (6,1) {$\times$};
\node[op] (mult2) at (6,-1) {$\times$};
\node[op] (mult3) at (8,0) {$\times$};

% 连接门输出
\draw[->] (forget) -- (4,1.5) -- (4,1.2) -- (mult1);
\draw[->] (input) -- (tanh);
\draw[->] (tanh) -- (5,0.5) -- (5,-0.8) -- (mult2);
\draw[->] (output) -- (4,-0.5) -- (4,-0.8) -- (mult3);

% 细胞状态更新
\draw[->] (0,0) -- (mult1);
\draw[->] (mult1) -- (7,0) -- (7,-1) -- (mult2);
\draw[->] (mult2) -- (8,-1) -- (8,-0.2) -- (mult3);
\draw[->] (mult3) -- (9,0) -- (10,0);

% 输出
\node[point] (ht) at (10,-1) {};
\draw[->] (9,0) -- (9,-1) -- (ht);
\node[right=0.2cm of ht] {$h_t$};
\end{tikzpicture}
\caption{LSTM单元结构}
\end{figure}

\textbf{LSTM的三个门}:
\begin{enumerate}
    \item \textbf{遗忘门}:决定从细胞状态中丢弃什么信息
    \[
    f_t = \sigma(W_f \cdot [h_{t-1}, x_t] + b_f)
    \]
    
    \item \textbf{输入门}:决定将哪些新信息存入细胞状态
    \[
    \begin{aligned}
    i_t &= \sigma(W_i \cdot [h_{t-1}, x_t] + b_i) \\
    \tilde{C}_t &= \tanh(W_C \cdot [h_{t-1}, x_t] + b_C)
    \end{aligned}
    \]
    
    \item \textbf{输出门}:决定输出什么信息
    \[
    \begin{aligned}
    o_t &= \sigma(W_o \cdot [h_{t-1}, x_t] + b_o) \\
    h_t &= o_t \odot \tanh(C_t)
    \end{aligned}
    \]
\end{enumerate}

\textbf{细胞状态更新}:
\[
C_t = f_t \odot C_{t-1} + i_t \odot \tilde{C}_t
\]

\begin{remark}
    \textbf{为什么LSTM能解决长期依赖?}
    \begin{itemize}
        \item 细胞状态$C_t$像一条"传送带",信息可以几乎不变地流过。
        \item 门控机制选择性地让信息通过。
        \item 梯度在细胞状态上可以稳定传播,不易消失。
    \end{itemize}
\end{remark}

\subsection{生成对抗网络(GANs)}

\begin{figure}[h]
\centering
\begin{tikzpicture}[
    node distance=1.5cm,
    gen/.style={rectangle, draw, rounded corners, fill=blue!20, minimum width=2cm, minimum height=1cm},
    disc/.style={rectangle, draw, rounded corners, fill=red!20, minimum width=2cm, minimum height=1cm},
    data/.style={rectangle, draw, fill=green!20, minimum width=1.5cm, minimum height=0.8cm}
]
\node[gen] (generator) at (0,0) {生成器 $G$};
\node[disc] (discriminator) at (5,0) {判别器 $D$};
\node[data] (real) at (5,2) {真实数据};
\node[data] (fake) at (0,-1.5) {生成数据};

% 噪声输入
\draw[<-] (generator.west) -- ++(-1.5,0) node[left] {随机噪声 $z$};

% 生成过程
\draw[->] (generator.east) -- node[above] {$G(z)$} (discriminator.west);

% 判别过程
\draw[->] (real.south) -- (discriminator.north);
\draw[->] (generator.south) -- (fake.north);

% 对抗训练
\draw[<->, thick, red] (generator.north) to[bend left] node[above] {对抗训练} (discriminator.north);

% 输出
\draw[->] (discriminator.east) -- ++(1.5,0) node[right] {真/假概率};
\end{tikzpicture}
\caption{GAN的基本结构}
\end{figure}

\textbf{核心思想}:两个神经网络相互博弈
\begin{itemize}
    \item \textbf{生成器 $G$}:学习真实数据分布,生成以假乱真的样本。
    \item \textbf{判别器 $D$}:区分真实数据与生成数据。
\end{itemize}

\textbf{目标函数(极小极大博弈)}:
\[
\min_G \max_D V(D, G) = \mathbb{E}_{x \sim p_{\text{data}}(x)}[\log D(x)] + \mathbb{E}_{z \sim p_z(z)}[\log(1 - D(G(z)))]
\]

\textbf{训练过程}:
\begin{enumerate}
    \item 固定$G$,训练$D$:最大化区分真实与假数据的能力。
    \item 固定$D$,训练$G$:最小化$\log(1 - D(G(z)))$,即让生成数据更易骗过$D$。
\end{enumerate}

\subsection{扩散模型(Diffusion Models)}

\begin{figure}[h]
\centering
\begin{tikzpicture}[
    node distance=1.2cm,
    box/.style={rectangle, draw, minimum width=2.5cm, minimum height=1cm},
    arrow/.style={->, >=stealth, thick}
]
% 前向过程
\node[box, fill=blue!10] (x0) at (0,0) {$x_0$ (真实数据)};
\node[box, fill=blue!10] (xt) at (4,0) {$x_t$};
\node[box, fill=blue!10] (xT) at (8,0) {$x_T$ (纯噪声)};

\draw[arrow] (x0) -- node[above] {加噪} (xt);
\draw[arrow] (xt) -- node[above] {加噪} (xT);

\node[below=0.3cm of xt] {前向过程(固定)};

% 反向过程
\node[box, fill=red!10] (x0hat) at (0,-3) {$\hat{x}_0$ (生成数据)};
\node[box, fill=red!10] (xthat) at (4,-3) {$\hat{x}_t$};
\node[box, fill=red!10] (xThat) at (8,-3) {$x_T \sim \mathcal{N}(0, I)$};

\draw[arrow] (xThat) -- node[above] {去噪} (xthat);
\draw[arrow] (xthat) -- node[above] {去噪} (x0hat);

\node[below=0.3cm of xthat] {反向过程(学习)};

% 连接
\draw[arrow, dashed] (xT.south) -- (xThat.north);
\end{tikzpicture}
\caption{扩散模型的前向与反向过程}
\end{figure}

\textbf{核心思想}:
\begin{enumerate}
    \item \textbf{前向过程}:逐步向数据添加高斯噪声,直到变成纯噪声。
    \[
    q(x_t | x_{t-1}) = \mathcal{N}(x_t; \sqrt{1-\beta_t}x_{t-1}, \beta_t I)
    \]
    
    \item \textbf{反向过程}:学习去噪过程,从噪声中重建数据。
    \[
    p_\theta(x_{t-1} | x_t) = \mathcal{N}(x_{t-1}; \mu_\theta(x_t, t), \Sigma_\theta(x_t, t))
    \]
\end{enumerate}

\textbf{优势}:生成质量高,训练稳定,无需对抗训练。

\section{Transformer:注意力就是全部}

\subsection{自注意力机制(Self-Attention)}

传统RNN/LSTM的问题是顺序处理、难以并行、长距离依赖衰减。Transformer的解决方案:完全基于注意力。

\begin{figure}[h]
\centering
\begin{tikzpicture}[
    node distance=0.8cm,
    token/.style={rectangle, draw, minimum width=1.2cm, minimum height=0.6cm},
    vector/.style={rectangle, draw, fill=blue!10, minimum width=1cm, minimum height=0.5cm}
]
% 输入
\node[token] (x1) at (0,4) {猫};
\node[token] (x2) at (0,3) {喜欢};
\node[token] (x3) at (0,2) {吃};
\node[token] (x4) at (0,1) {鱼};

% 线性变换
\foreach \i in {1,...,4} {
    \node[vector] (q\i) at (2,4.5-\i) {$q_\i$};
    \node[vector] (k\i) at (3.5,4.5-\i) {$k_\i$};
    \node[vector] (v\i) at (5,4.5-\i) {$v_\i$};
    \draw[->] (x\i.east) -- (q\i.west);
    \draw[->] (x\i.east) -- (k\i.west);
    \draw[->] (x\i.east) -- (v\i.west);
}

% 注意力计算
\node at (7.5,4) {注意力得分:};
\node at (8,3.5) {$\alpha_{ij}  $};

\node at (11,3.5) {= $0.8v_1 + 0.1v_2 + 0.1v_3 + 0.0v_4$};
\node at (11,2.5) {= $0.2v_1 + 0.6v_2 + 0.1v_3 + 0.1v_4$};
\node at (11,1.5) {= $0.1v_1 + 0.2v_2 + 0.6v_3 + 0.1v_4$};
\node at (11,0.5) {= $0.0v_1 + 0.1v_2 + 0.1v_3 + 0.8v_4$};

\draw[->, blue, thick] (q1.east) to[bend left] (k1.east);
\draw[->, red, thick] (q1.east) to[bend left] (k3.east);
\end{tikzpicture}
\caption{自注意力机制示例:"吃"与"鱼"有强关联}
\end{figure}

\textbf{Query, Key, Value概念}:
\begin{itemize}
    \item \textbf{Query(查询)}:当前要计算注意力的位置。
    \item \textbf{Key(键)}:所有位置,用于与Query计算相关性。
    \item \textbf{Value(值)}:所有位置的实际信息。
\end{itemize}

\textbf{计算过程}:
\begin{enumerate}
    \item 对每个输入$x_i$,通过可学习权重矩阵计算$q_i = W^Q x_i$, $k_i = W^K x_i$, $v_i = W^V x_i$。
    \item 计算注意力分数:$\alpha_{ij} = \text{softmax}\left(\frac{q_i \cdot k_j}{\sqrt{d_k}}\right)$,其中$d_k$是Key的维度(缩放因子防止梯度消失)。
    \item 加权求和:$c_i = \sum_j \alpha_{ij} v_j$。
\end{enumerate}

\textbf{矩阵形式}:
\[
\text{Attention}(Q, K, V) = \text{softmax}\left(\frac{QK^\top}{\sqrt{d_k}}\right)V
\]

\subsection{多头注意力(Multi-Head Attention)}

单一注意力机制可能只关注特定类型的模式。多头注意力允许模型同时关注不同子空间的信息。

\[
\begin{aligned}
\text{MultiHead}(Q, K, V) &= \text{Concat}(\text{head}_1, \ldots, \text{head}_h)W^O \\
\text{head}_i &= \text{Attention}(QW_i^Q, KW_i^K, VW_i^V)
\end{aligned}
\]

\subsection{Transformer架构}

\begin{figure}[h]
\centering
\begin{tikzpicture}[
    block/.style={rectangle, draw, minimum width=3cm, minimum height=1cm},
    layer/.style={rectangle, draw, dashed, minimum width=3.5cm, minimum height=2cm}
]
% 编码器
\node[layer, fill=blue!5] (encoder) at (0,-1) {编码器};
\node[block] (enc1) at (0,1.5) {自注意力 + 前馈网络};
\node[block] (enc2) at (0,0) {自注意力 + 前馈网络};
\node[block] (enc3) at (0,-1.5) {自注意力 + 前馈网络};

% 解码器
\node[layer, fill=red!5] (decoder) at (5,0) {解码器};
\node[block] (dec1) at (5,2.5) {掩码自注意力 + 前馈网络};
\node[block] (dec2) at (5,1) {交叉注意力 + 前馈网络};
\node[block] (dec3) at (5,-0.5) {交叉注意力 + 前馈网络};

% 输入输出
\node[left=0.5cm of enc3] {输入序列};
\draw[->] (-1.5,-1.5) -- (enc3);
\node[right=0.5cm of dec3] {输出序列};
\draw[->] (dec3) -- (6.5,-0.5);

% 编码器-解码器注意力
\draw[->, dashed] (enc1.east) -- (dec2.west);
\draw[->, dashed] (enc2.east) -- (dec2.west);
\draw[->, dashed] (enc3.east) -- (dec2.west);

% 位置编码
\node[above=0.2cm of enc1] {位置编码};
\end{tikzpicture}
\caption{Transformer架构}
\end{figure}

\subsubsection{编码器(Encoder)}
\begin{itemize}
    \item 由N个相同层堆叠。
    \item 每层包含:多头自注意力 + 前馈神经网络 + 残差连接 + 层归一化。
\end{itemize}

\subsubsection{解码器(Decoder)}
\begin{itemize}
    \item 也由N个相同层堆叠。
    \item 每层包含:
    \begin{enumerate}
        \item \textbf{掩码多头自注意力}:防止当前位置看到未来信息(因果掩码)。
        \item \textbf{多头交叉注意力}:Query来自解码器,Key/Value来自编码器。
        \item 前馈神经网络 + 残差连接 + 层归一化。
    \end{enumerate}
\end{itemize}

\subsection{位置编码(Positional Encoding)}

自注意力机制本身不包含位置信息,需要额外注入位置编码:
\[
\begin{aligned}
PE_{(pos, 2i)} &= \sin(pos / 10000^{2i/d_{\text{model}}}) \\
PE_{(pos, 2i+1)} &= \cos(pos / 10000^{2i/d_{\text{model}}})
\end{aligned}
\]

其中$pos$是位置,$i$是维度索引。这种正弦编码能够捕捉相对位置关系。

\subsection{Transformer vs RNN/LSTM}

\begin{table}[h]
\centering
\caption{Transformer与RNN/LSTM对比}
\begin{tabular}{p{0.3\textwidth}p{0.3\textwidth}p{0.3\textwidth}}
\textbf{特性} & \textbf{Transformer} & \textbf{RNN/LSTM} \\
\textbf{核心机制} & 自注意力(全局依赖) & 循环结构(顺序依赖) \\
\textbf{并行能力} & \textbf{高}(所有位置并行计算) & \textbf{低}(必须顺序计算) \\
\textbf{长距离依赖} & \textbf{O(1)}路径长度 & \textbf{O(n)}路径长度,易梯度消失 \\
\textbf{计算复杂度} & $O(L^2 \cdot d)$,$L$为序列长度 & $O(L \cdot d^2)$ \\
\textbf{显存占用} & 高(需存储注意力矩阵) & 低 \\
\textbf{可解释性} & \textbf{高}(注意力权重可视化) & 低(隐藏状态难解释) \\
\textbf{位置信息} & 需显式添加位置编码 & 天然包含位置信息 \\
\end{tabular}
\end{table}

\section{为什么Transformer能够"通吃"?}

\subsection{从专家模型到通用模型}

\begin{itemize}
    \item \textbf{旧范式}:各司其职的专家模型
    \begin{itemize}
        \item 图像:CNN(卷积核捕捉局部特征)
        \item 序列:RNN/LSTM(循环结构捕捉时序)
        \item 归纳偏置强,但领域受限。
    \end{itemize}
    
    \item \textbf{新范式}:万物皆序列的通用模型
    \begin{itemize}
        \item 文本:词序列 $[w_1, w_2, \ldots, w_n]$
        \item 图像:切分为patch序列
        \item 音频:帧序列
        \item 视频:帧序列的序列
    \end{itemize}
    
    \item \textbf{统一处理}:所有数据都转化为标记序列,用Transformer处理。
\end{itemize}

\subsection{Transformer的核心优势}

\begin{enumerate}
    \item \textbf{全局依赖建模}:自注意力直接建立任意两个位置的联系。
    \item \textbf{并行计算}:极大加速训练。
    \item \textbf{可扩展性}:模型规模(参数量、数据量、计算量)增加带来稳定性能提升。
    \item \textbf{多模态统一}:相同的架构处理不同模态数据。
\end{enumerate}

\section{总结与思考}

本章从模型评估和优化出发,探讨了深度学习的关键问题:
\begin{itemize}
    \item \textbf{评估}:通过数据集划分和交叉验证获得可靠的性能估计。
    \item \textbf{诊断}:识别欠拟合和过拟合,理解偏差-方差权衡。
    \item \textbf{优化}:使用正则化、Dropout、早停法提升泛化能力。
    \item \textbf{发展}:从简单神经网络到深度学习的演进,四大支柱的支撑。
    \item \textbf{架构}:CNN、RNN/LSTM、GAN、扩散模型、Transformer等各有所长。
    \item \textbf{未来}:Transformer展示了通用人工智能架构的潜力,大模型时代已经到来。
\end{itemize}

\textbf{思考}:深度学习的发展不仅是算法的进步,更是数据、算力、算法和开源生态协同发展的结果。理解这些基础原理,才能更好地应用和创新。