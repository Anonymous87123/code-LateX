\chapter{深度强化学习之连续控制}
离散动作空间 (Discrete Action Space):动作是有限的、可数的集合。
- Atari 游戏中的“上、下、左、右”(4选1),
- 棋类游戏中的落子位置(从N个位置中选1个)。
连续动作空间 (Continuous Action Space):动作是在一个或多个维度上取值的连续实数。
- 自动驾驶:我们需要同时控制方向盘和油门/刹车。
  - 动作可以是一个二维向量:$a = [\text{方向盘转角}, \text{加速度}]$,例如,$a = [-15.5, 0.75]$ 表示方向盘向左转 15.5 度,同时施加 75% 的油门。
- 机器人控制:一个拥有6个关节的机械臂(6自由度),需要同时控制每个关节的目标角度或力矩。
  - 动作就是一个六维向量:$a = [\theta_1, \theta_2, \theta_3, \theta_4, \theta_5, \theta_6]$
价值学习方法的困境
像 Q-Learning 或 DQN 这类算法的核心是计算在状态 $s$ 下,所有可能动作 $a$ 的价值 $Q(s,a)$,然后通过取最大值来选择最优动作:
$$a^* = \arg\max_a Q(s,a)$$

在离散空间中,我们可以为有限的几个动作分别计算 Q 值,然后比较大小。但在连续空间中,动作 $a$ 是无限的!我们无法遍历所有可能的动作来找到那个使 Q 值最大的动作。因此,执行 $\arg\max$ 操作变得不可行。

传统策略梯度方法的困境

对于离散动作,策略网络 $\pi(a|s)$ 的输出层通常是一个 Softmax 函数。它输出一个概率分布,对应每个可行动作被选择的概率。例如,对于 4 个动作,输出可能是 $[0.1, 0.7, 0.1, 0.1]$。

这种方法的本质是为每一个动作都赋予一个概率。当动作是无限多个时,这个思路显然也无法实现。我们不可能设计一个能输出无穷多个概率的神经网络。
离散化 (Discretization)
一个最直观、最简单的想法是离散化 (Discretization),试图将无限的连续空间改造为有限的离散空间。
1. 单维划分:将一个连续的动作维度切分成有限个“档位”或“区间”。
○ 例子:对于自动驾驶中的方向盘转角 (范围 $[-90^\circ, +90^\circ]$),我们可以将其粗略地离散化为5个固定动作:$\{-45^\circ, -15^\circ, 0^\circ, +15^\circ, +45^\circ\}$。
2. 多维组合:当动作是多维时,最终的离散动作集是所有维度离散点集合的笛卡尔积 (Cartesian Product)。这意味着总动作数是每个维度离散点数量的乘积。
○ 维度1 (方向盘):离散化为5个档位 $D_1 = \{-45^\circ, -15^\circ, 0^\circ, +15^\circ, +45^\circ\}$。
○ 维度2 (油门/刹车):离散化为3个档位 $D_2 = \{\text{全力加速}, \text{保持速度}, \text{紧急刹车}\}$。
○ 最终动作集:最终的离散动作空间 $A$ 是 $D_1$ 和 $D_2$ 的所有可能组合。总动作数量为 $|A| = |D_1| \times |D_2| = 5 \times 3 = 15$。
■ 这15个动作包括:$(-45^\circ, \text{全力加速})$、$(-45^\circ, \text{保持速度})$、$(-45^\circ, \text{紧急刹车})$、$(-15^\circ, \text{全力加速})$ 等
离散化的困境
1. 维度灾难 (Curse of Dimensionality)
随着动作维度的增加,离散动作的总数会呈指数级爆炸式增长。
○ 一个常见的6自由度机械臂,需要同时控制6个关节。如果我们仅将每个关节的力矩粗略地划分为10个级别,那么总的离散动作数量将是:
$$Total\ Actions = 10 \times 10 \times 10 \times 10 \times 10 \times 10 = 10^6 = 1,000,000$$
让一个算法从一百万个动作中学习并选择最优动作,在计算上是极其低效甚至不可行的。
2. 精度损失
离散化本质上是一种“粗暴”的近似,它使得智能体失去了在“档位”之间进行精细微调的能力。
○ 也许最优的转向角度是 $-17.3^\circ$,但在我们上面5档位的设定中,智能体只能在 $-15^\circ$ 和 $-45^\circ$ 之间选择一个,永远无法达到真正的最优解。这不仅导致策略的次优性,还可能让智能体在两个离散点之间来回震荡,无法稳定。
离散化是一种“治标不治本”的思路,它回避了问题的核心。这种方法仅在动作维度极低(通常1-2维)且对控制精度要求不高的极少数场景下或许可用。