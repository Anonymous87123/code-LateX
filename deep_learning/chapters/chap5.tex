\chapter{深度强化学习之连续控制}

\section{引言:连续控制问题的挑战}

\subsection{离散动作空间 vs 连续动作空间}

在强化学习中,动作空间(Action Space)是智能体(Agent)可以执行的所有可能动作的集合。根据动作的性质,我们可以将动作空间分为两大类:

\begin{definition}[离散动作空间(Discrete Action Space)]
离散动作空间由有限个、可数的动作组成。智能体在每个时间步从这些动作中选择一个执行。
\begin{itemize}
    \item \textbf{特点}:动作数量有限,可以枚举
    \item \textbf{示例}:
    \begin{itemize}
        \item Atari游戏:上、下、左、右(4个动作)
        \item 棋类游戏:棋盘上的合法落子位置(几十到几百个动作)
        \item 导航问题:前进、后退、左转、右转、停止
    \end{itemize}
    \item \textbf{数学表示}:$\mathcal{A} = \{a_1, a_2, \dots, a_n\}$,其中$n$是有限正整数
\end{itemize}
\end{definition}

\begin{definition}[连续动作空间(Continuous Action Space)]
连续动作空间中的动作是在一个或多个维度上取值的连续实数。智能体在每个时间步需要指定每个动作维度的具体数值。
\begin{itemize}
    \item \textbf{特点}:动作是连续的,不可枚举
    \item \textbf{示例}:
    \begin{itemize}
        \item 自动驾驶:方向盘转角(-90°到+90°的连续值)、油门/刹车(0到1的连续值)
        \item 机器人控制:机械臂的6个关节角度,每个关节在特定范围内连续变化
        \item 无人机控制:三个方向的推力大小,每个推力是连续值
    \end{itemize}
    \item \textbf{数学表示}:$\mathcal{A} \subseteq \mathbb{R}^d$,其中$d$是动作维度
\end{itemize}
\end{definition}

\subsection{连续控制问题的实例}

为了更直观地理解连续控制问题,让我们看几个具体例子:

\begin{example}[自动驾驶]
自动驾驶汽车需要同时控制多个连续变量:
\[
a = \begin{bmatrix}
\text{方向盘转角} \\ \text{加速度}
\end{bmatrix} = \begin{bmatrix}
-15.5^\circ \\ 0.75
\end{bmatrix}
\]
其中:
\begin{itemize}
    \item 方向盘转角:通常在$[-90^\circ, +90^\circ]$范围内连续变化
    \item 加速度:在$[-1, 1]$范围内连续变化,负值表示刹车
\end{itemize}
这是一个2维连续动作空间。
\end{example}

\begin{example}[6自由度机械臂控制]
工业机器人需要精确控制每个关节的角度或力矩:
\[
a = \begin{bmatrix}
\theta_1 \\ \theta_2 \\ \theta_3 \\ \theta_4 \\ \theta_5 \\ \theta_6
\end{bmatrix} = \begin{bmatrix}
0.5 \text{ rad} \\ 1.2 \text{ rad} \\ -0.3 \text{ rad} \\ 0.8 \text{ rad} \\ 0.1 \text{ rad} \\ -0.5 \text{ rad}
\end{bmatrix}
\]
其中每个$\theta_i$表示第$i$个关节的角度,这是一个6维连续动作空间。
\end{example}

\begin{example}[人形机器人行走]
人形机器人需要控制多个关节的力矩来实现稳定行走:
\[
a = \begin{bmatrix}
\tau_1 \\ \tau_2 \\ \vdots \\ \tau_{12}
\end{bmatrix}
\]
其中$\tau_i$表示第$i$个关节的力矩,通常有12-20个关节需要控制。
\end{example}

\section{价值学习方法在连续动作空间中的困境}

\subsection{Q-Learning/DQN的核心机制回顾}

在离散动作空间中,Q-Learning和DQN等价值学习方法的核心是计算状态-动作价值函数$Q(s,a)$,然后通过最大化操作选择最优动作:

\[
a^* = \arg\max_{a \in \mathcal{A}} Q(s,a)
\]

这个机制在离散动作空间中运行良好,因为:
\begin{enumerate}
    \item 可以枚举所有动作$a \in \mathcal{A}$
    \item 为每个动作计算$Q(s,a)$
    \item 比较所有$Q(s,a)$值,选择最大的
\end{enumerate}

\subsection{连续动作空间中的挑战}

在连续动作空间中,$\arg\max$操作面临根本性困难:

\begin{enumerate}
    \item \textbf{无限动作数量}:动作空间是无限的,无法枚举所有可能的动作
    \item \textbf{连续优化问题}:$\arg\max_{a \in \mathbb{R}^d} Q(s,a)$是一个连续优化问题
    \item \textbf{计算复杂度}:每次选择动作都需要解决一个优化问题,计算代价高昂
    \item \textbf{局部最优}:$Q(s,a)$函数可能非凸,难以找到全局最优
\end{enumerate}

\textbf{具体分析}:
假设$Q(s,a)$是一个关于$a$的复杂非线性函数,找到使$Q(s,a)$最大的$a$需要:
\[
\nabla_a Q(s,a) = 0
\]
然后检查二阶条件$\nabla_a^2 Q(s,a) \prec 0$(负定)。这在实际中几乎不可能实时完成。

\subsection{传统策略梯度方法的困境}

对于离散动作空间,策略网络$\pi_\theta(a|s)$通常使用Softmax输出层,为每个离散动作输出一个概率:

\[
\pi_\theta(a|s) = \frac{\exp(f_\theta(s,a))}{\sum_{a' \in \mathcal{A}} \exp(f_\theta(s,a'))}
\]

其中$f_\theta(s,a)$是网络为动作$a$计算的得分。

在连续动作空间中,这种方法失效,因为:
\begin{enumerate}
    \item 无法计算分母中的归一化常数(需要对无限动作积分)
    \item 无法为无限个动作分别计算得分
\end{enumerate}

\section{离散化方法:一种直观的解决方案}

\subsection{离散化的基本思想}

面对连续动作空间的挑战,一个最直观的想法是\textbf{离散化}(Discretization):将连续的无限动作空间转化为离散的有限动作空间,然后应用已有的离散动作算法。

\begin{definition}[离散化]
离散化是将连续动作空间$\mathcal{A} \subseteq \mathbb{R}^d$划分为有限个离散区域,每个区域用一个代表性动作(通常是中心点)表示的过程。
\end{definition}

\subsection{一维离散化}

对于一维连续动作,离散化过程很简单:

\begin{example}[方向盘转角离散化]
假设方向盘转角范围是$[-90^\circ, +90^\circ]$,我们可以将其离散化为5个档位:
\[
\mathcal{A}_{\text{disc}} = \{-45^\circ, -15^\circ, 0^\circ, +15^\circ, +45^\circ\}
\]
智能体只能选择这5个角度之一。
\end{example}

\subsection{多维离散化}

对于多维连续动作,离散化需要更复杂的处理:

\begin{example}[自动驾驶的二维动作离散化]
考虑自动驾驶的两个连续动作维度:
\begin{itemize}
    \item 方向盘转角$D_1$:离散化为5个档位:$\{-45^\circ, -15^\circ, 0^\circ, +15^\circ, +45^\circ\}$
    \item 油门/刹车$D_2$:离散化为3个档位:$\{\text{全力加速}, \text{保持速度}, \text{紧急刹车}\}$
\end{itemize}

总动作空间是这两个维度的笛卡尔积:
\[
\mathcal{A}_{\text{disc}} = D_1 \times D_2
\]
总动作数为$5 \times 3 = 15$,包括$(-45^\circ, \text{全力加速})$、$(-45^\circ, \text{保持速度})$等所有组合。
\end{example}

\subsection{离散化的数学表示}

更一般地,对于$d$维连续动作空间,将每个维度$i$离散化为$k_i$个档位,则总动作数为:
\[
|\mathcal{A}_{\text{disc}}| = \prod_{i=1}^{d} k_i
\]

这是一个组合爆炸问题:随着维度增加,动作数量呈指数增长。

\subsection{离散化的困境与局限性}

尽管离散化方法直观简单,但它面临严重问题:

\begin{table}[H]
\centering
\caption{离散化方法的局限性}
\begin{tabular}{p{0.3\textwidth}p{0.65\textwidth}}
\toprule
\textbf{问题} & \textbf{详细分析} \\
\midrule
\textbf{维度灾难(Curse of Dimensionality)} & 
\begin{itemize}
    \item 动作数量随维度指数增长:$k^d$
    \item 例如:6自由度机械臂,每个维度离散化为10个级别
    \[
    |\mathcal{A}_{\text{disc}}| = 10^6 = 1,000,000
    \]
    \item 从一百万个动作中选择最优动作计算量巨大
    \item 存储$Q(s,a)$表需要大量内存
\end{itemize} \\
\hline
\textbf{精度损失} & 
\begin{itemize}
    \item 离散化丢失了连续控制能力
    \item 真正的最优动作可能在两个离散点之间
    \item 例如:最优转向角可能是$-17.3^\circ$,但离散化后只能选择$-15^\circ$或$-45^\circ$
    \item 导致策略次优,控制抖动
\end{itemize} \\
\hline
\textbf{探索效率低} & 
\begin{itemize}
    \item 离散动作之间的跳跃导致探索不连续
    \item 难以学习平滑的控制策略
    \item 在边界附近可能产生振荡
\end{itemize} \\
\hline
\textbf{不适用于高精度控制} & 
\begin{itemize}
    \item 机器人手术、精密加工等需要高精度连续控制
    \item 离散化无法满足精度要求
    \item 可能导致系统不稳定
\end{itemize} \\
\bottomrule
\end{tabular}
\end{table}

\subsection{离散化的适用场景}

尽管有诸多限制,离散化在特定情况下仍有价值:
\begin{itemize}
    \item 动作维度很低(1-2维)
    \item 控制精度要求不高
    \item 快速原型验证
    \item 与现有离散算法兼容
\end{itemize}

但对于大多数实际连续控制问题,我们需要更优雅的解决方案。

\section{随机策略梯度方法:参数化概率分布}

\subsection{核心思想}

随机策略梯度(Stochastic Policy Gradient)方法的核心理念是:策略网络不直接输出动作,而是输出一个概率分布的参数,然后从这个分布中采样得到动作。

\begin{figure}[H]
\centering
\includegraphics[width=0.7\textwidth]{picture/strategy.png}
\caption{随机策略网络:输出概率分布参数,从中采样动作}
\end{figure}

\subsection{为什么选择高斯分布?}

高斯分布(正态分布)是连续动作空间中最常用的概率分布,原因如下:

\begin{enumerate}
    \item \textbf{参数简单}:只需均值和标准差两个参数
    \begin{itemize}
        \item 均值$\mu$:策略认为"最优"的动作
        \item 标准差$\sigma$:探索的程度
    \end{itemize}
    
    \item \textbf{数学性质良好}:
    \begin{itemize}
        \item 中心极限定理保证许多随机变量的和近似服从高斯分布
        \item 具有最大熵性质,是最"随机"的分布
    \end{itemize}
    
    \item \textbf{计算方便}:
    \begin{itemize}
        \item 概率密度函数有解析表达式
        \item 采样、计算对数概率都很简单
    \end{itemize}
    
    \item \textbf{物理意义明确}:
    \begin{itemize}
        \item 许多物理系统的噪声是高斯分布
        \item 符合"大部分情况下接近均值,偶尔有较大偏差"的直觉
    \end{itemize}
\end{enumerate}

\subsection{单变量高斯分布}

对于一维连续动作,策略网络输出均值和标准差:

\[
a \sim \mathcal{N}(\mu_\theta(s), \sigma_\theta(s)^2)
\]

其中:
\begin{itemize}
    \item $\mu_\theta(s)$:策略网络在状态$s$下输出的均值
    \item $\sigma_\theta(s)$:策略网络在状态$s$下输出的标准差(必须为正)
\end{itemize}

概率密度函数为:
\[
\pi_\theta(a|s) = \frac{1}{\sigma_\theta(s)\sqrt{2\pi}} \exp\left(-\frac{(a - \mu_\theta(s))^2}{2\sigma_\theta(s)^2}\right)
\]

\subsection{多变量高斯分布与独立性假设}

对于$d$维连续动作,理论上应该使用多元高斯分布:
\[
a \sim \mathcal{N}(\mu_\theta(s), \Sigma_\theta(s))
\]
其中$\Sigma_\theta(s)$是$d \times d$的协方差矩阵。

但实际中通常做\textbf{独立性假设}:各动作维度相互独立。此时协方差矩阵是对角矩阵:
\[
\Sigma_\theta(s) = \text{diag}(\sigma_1^2, \sigma_2^2, \dots, \sigma_d^2)
\]

概率密度函数简化为各维度概率密度的乘积:
\[
\pi_\theta(a|s) = \prod_{i=1}^{d} \frac{1}{\sigma_i\sqrt{2\pi}} \exp\left(-\frac{(a_i - \mu_i)^2}{2\sigma_i^2}\right)
\]

\subsection{策略网络的输出设计}

在实现中,策略网络需要输出概率分布的参数:

\begin{enumerate}
    \item \textbf{均值$\mu_\theta(s)$}:网络直接输出$d$维向量
    \item \textbf{标准差$\sigma_\theta(s)$}:两种常见设计
    \begin{itemize}
        \item \textbf{状态相关}:网络输出另一个$d$维向量,通过$\exp$函数确保正性
        \item \textbf{状态无关}:$\sigma$作为可学习参数,不依赖状态
    \end{itemize}
\end{enumerate}

为了保证标准差为正,通常输出$\log\sigma$,然后通过指数函数转换:
\[
\sigma = \exp(\log\sigma)
\]

\subsection{动作选择过程}

在状态$s_t$下,策略网络输出参数后,动作选择过程为:

\begin{algorithm}[H]
\caption{随机策略的动作选择}
\begin{algorithmic}[1]
\REQUIRE 状态$s_t$,策略网络参数$\theta$
\ENSURE 动作$a_t$
\STATE 前向传播:$[\mu_\theta(s_t), \log\sigma_\theta(s_t)] = \text{PolicyNetwork}(s_t)$
\STATE 计算标准差:$\sigma_\theta(s_t) = \exp(\log\sigma_\theta(s_t))$
\STATE 从标准正态分布采样:$\epsilon \sim \mathcal{N}(0, I)$
\STATE 计算动作:$a_t = \mu_\theta(s_t) + \sigma_\theta(s_t) \odot \epsilon$,其中$\odot$是逐元素乘法
\STATE 执行动作$a_t$,获得奖励$r_{t+1}$和下一状态$s_{t+1}$
\end{algorithmic}
\end{algorithm}

这个过程称为"重参数化技巧"(Reparameterization Trick),它使采样过程可微分,便于梯度传播。

\subsection{计算对数概率}

在策略梯度方法中,需要计算$\log \pi_\theta(a|s)$及其梯度。对于单变量高斯分布:

\begin{align*}
\log \pi_\theta(a|s) &= \log\left(\frac{1}{\sigma\sqrt{2\pi}} \exp\left(-\frac{(a - \mu)^2}{2\sigma^2}\right)\right) \\
&= -\frac{(a - \mu)^2}{2\sigma^2} - \log\sigma - \log\sqrt{2\pi}
\end{align*}

对于$d$维独立高斯分布:
\begin{align*}
\log \pi_\theta(a|s) &= \log\left(\prod_{i=1}^{d} \pi_\theta(a_i|s)\right) \\
&= \sum_{i=1}^{d} \log \pi_\theta(a_i|s) \\
&= -\frac{1}{2}\sum_{i=1}^{d}\left(\frac{(a_i - \mu_i)^2}{\sigma_i^2} + 2\log\sigma_i + \log(2\pi)\right)
\end{align*}

\subsection{数值稳定性问题与解决方案}

计算概率时可能遇到数值下溢问题:多个小概率相乘结果趋近于0。解决方案是在对数空间计算:

\textbf{问题}:
\[
\pi_\theta(a|s) = \prod_{i=1}^{d} \pi_\theta(a_i|s) \to 0 \quad \text{当$d$很大时}
\]

\textbf{解决方案}:使用对数概率
\[
\log \pi_\theta(a|s) = \sum_{i=1}^{d} \log \pi_\theta(a_i|s)
\]

在PPO中计算重要性采样比率时:
\[
r_t(\theta) = \frac{\pi_\theta(a_t|s_t)}{\pi_{\theta_k}(a_t|s_t)} = \exp\left(\log \pi_\theta(a_t|s_t) - \log \pi_{\theta_k}(a_t|s_t)\right)
\]

\subsection{PPO在连续动作空间中的实现}

将上述技术整合到PPO框架中:

\begin{algorithm}[H]
\caption{PPO for Continuous Control}
\begin{algorithmic}[1]
\REQUIRE 初始化策略参数$\theta$,价值参数$\phi$,裁剪参数$\epsilon$,学习率$\alpha$
\FOR{迭代$k=0,1,2,\dots$}
    \STATE 使用当前策略$\pi_{\theta_k}$收集轨迹数据$\mathcal{D}_k$
    \STATE 计算优势估计$\hat{A}_t$(使用GAE)
    \FOR{epoch $=1$ to $M$}
        \STATE 从$\mathcal{D}_k$中采样小批量数据
        \STATE 计算对数概率:$\log \pi_\theta(a_t|s_t)$和$\log \pi_{\theta_k}(a_t|s_t)$
        \STATE 计算重要性采样比率:$r_t(\theta) = \exp(\log \pi_\theta(a_t|s_t) - \log \pi_{\theta_k}(a_t|s_t))$
        \STATE 计算裁剪目标函数:
        \[
        L^{\text{CLIP}}(\theta) = \mathbb{E}_t\left[\min\left(r_t(\theta)\hat{A}_t, \text{clip}(r_t(\theta), 1-\epsilon, 1+\epsilon)\hat{A}_t\right)\right]
        \]
        \STATE 更新策略参数:$\theta \leftarrow \theta + \alpha \nabla_\theta L^{\text{CLIP}}(\theta)$
    \ENDFOR
\ENDFOR
\end{algorithmic}
\end{algorithm}

\section{确定性策略梯度方法}

\subsection{确定性策略与随机策略的对比}

\begin{table}[H]
\centering
\caption{随机策略 vs 确定性策略}
\begin{tabular}{p{0.45\textwidth}p{0.45\textwidth}}
\toprule
\textbf{随机策略(Stochastic Policy)} & \textbf{确定性策略(Deterministic Policy)} \\
\midrule
输出动作的概率分布$\pi_\theta(a|s)$ & 直接输出动作$a = \mu_\theta(s)$ \\
从分布中采样得到动作 & 动作被唯一确定 \\
自带探索能力 & 需要手动添加噪声探索 \\
通常用于在线策略算法 & 天然适合离线策略算法 \\
示例:PPO、TRPO、A3C & 示例:DPG、DDPG、TD3 \\
\bottomrule
\end{tabular}
\end{table}

\begin{figure}[H]
\centering
\includegraphics[width=0.7\textwidth]{picture/dpg.png}
\caption{确定性策略:直接输出动作值}
\end{figure}

\subsection{确定性策略梯度定理}

\begin{theorem}[确定性策略梯度定理]
对于确定性策略$a = \mu_\theta(s)$,目标函数$J(\theta) = \mathbb{E}_{s \sim \rho^\mu}[Q^\mu(s, \mu_\theta(s))]$的梯度为:
\[
\nabla_\theta J(\theta) = \mathbb{E}_{s \sim \rho^\mu}[\nabla_\theta \mu_\theta(s) \nabla_a Q^\mu(s, a)|_{a=\mu_\theta(s)}]
\]
其中$\rho^\mu$是策略$\mu$下的状态分布。
\end{theorem}

\textbf{直观理解}:
\begin{itemize}
    \item $\nabla_a Q^\mu(s, a)$:Critic告诉Actor动作应该如何变化以提高Q值
    \item $\nabla_\theta \mu_\theta(s)$:Actor的参数如何影响输出动作
    \item 两者结合:通过链式法则,确定如何更新Actor参数以提高Q值
\end{itemize}

\subsection{探索策略:手动添加噪声}

确定性策略本身没有探索能力,需要在训练时手动添加噪声:

\begin{algorithm}[H]
\caption{确定性策略的探索}
\begin{algorithmic}[1]
\REQUIRE 状态$s_t$,确定性策略$\mu_\theta$,探索噪声$\mathcal{N}$
\ENSURE 动作$a_t$
\STATE 计算确定性动作:$a_{\text{det}} = \mu_\theta(s_t)$
\STATE 添加探索噪声:$a_t = a_{\text{det}} + \mathcal{N}$
\STATE 执行动作$a_t$,收集经验存入回放缓冲区
\end{algorithmic}
\end{algorithm}

常用的噪声类型:
\begin{itemize}
    \item 高斯噪声:$\mathcal{N} \sim \mathcal{N}(0, \sigma^2)$
    \item Ornstein-Uhlenbeck过程:具有时间相关性的噪声,适合物理系统
    \item 参数空间噪声:在策略参数中添加噪声
\end{itemize}

\subsection{Critic指导Actor更新}

在确定性策略梯度中,Critic网络$Q_\phi(s,a)$指导Actor更新:

\begin{figure}[H]
\centering
\includegraphics[width=0.8\textwidth]{picture/8.png}
\caption{Critic指导Actor更新:通过Q函数对动作的梯度}
\end{figure}

更新过程:
\begin{enumerate}
    \item Actor输出动作:$a = \mu_\theta(s)$
    \item Critic评估动作:$Q = Q_\phi(s, a)$
    \item 计算Q对动作的梯度:$\nabla_a Q_\phi(s, a)$
    \item 通过链式法则更新Actor:
    \[
    \nabla_\theta J(\theta) = \mathbb{E}[\nabla_\theta \mu_\theta(s) \nabla_a Q_\phi(s, a)]
    \]
\end{enumerate}

\section{DDPG:深度确定性策略梯度}

\subsection{DDPG算法框架}

DDPG(Deep Deterministic Policy Gradient)结合了DQN和DPG的优点:

\begin{figure}[H]
\centering
\begin{tikzpicture}[scale=0.8]
\node[draw, rectangle, rounded corners, fill=blue!20, minimum width=3cm, minimum height=1cm] (actor) at (0,2) {Actor $\mu_\theta(s)$};
\node[draw, rectangle, rounded corners, fill=red!20, minimum width=3cm, minimum height=1cm] (critic) at (0,0) {Critic $Q_\phi(s,a)$};
\node[draw, rectangle, rounded corners, fill=blue!10, minimum width=3cm, minimum height=1cm] (target_actor) at (5,2) {Target Actor $\mu_{\theta^-}(s)$};
\node[draw, rectangle, rounded corners, fill=red!10, minimum width=3cm, minimum height=1cm] (target_critic) at (5,0) {Target Critic $Q_{\phi^-}(s,a)$};
\node[draw, rectangle, fill=green!20, minimum width=4cm, minimum height=1.5cm] (replay) at (2.5,-2) {经验回放缓冲区};

\draw[->, thick] (actor) -- node[right] {动作$a$} (critic);
\draw[->, thick] (critic.north) -- node[left] {Q值} (actor.south);
\draw[<->, thick, dashed] (actor) -- node[above] {软更新} (target_actor);
\draw[<->, thick, dashed] (critic) -- node[above] {软更新} (target_critic);
\draw[->] (replay.north) -- (critic.south);
\draw[->] (replay.north) -- (target_critic.south);
\end{tikzpicture}
\caption{DDPG算法框架:Actor-Critic结构+目标网络+经验回放}
\end{figure}

\subsection{DDPG算法流程}

\begin{algorithm}[H]
\caption{DDPG算法}
\begin{algorithmic}[1]
\REQUIRE 学习率$\alpha_\theta, \alpha_\phi$,软更新系数$\tau$,折扣因子$\gamma$,探索噪声$\mathcal{N}$,批量大小$N$
\STATE 随机初始化Actor网络$\mu_\theta(s)$和Critic网络$Q_\phi(s,a)$的参数
\STATE 初始化目标网络:$\theta^- \leftarrow \theta$,$\phi^- \leftarrow \phi$
\STATE 初始化经验回放缓冲区$\mathcal{D}$
\FOR{回合$=1$ to $M$}
    \STATE 初始化环境,获得初始状态$s_1$
    \FOR{时间步$t=1$ to $T$}
        \STATE 选择动作:$a_t = \mu_\theta(s_t) + \mathcal{N}_t$,其中$\mathcal{N}_t$是探索噪声
        \STATE 执行动作$a_t$,获得奖励$r_t$和下一状态$s_{t+1}$
        \STATE 存储转移$(s_t, a_t, r_t, s_{t+1})$到$\mathcal{D}$
        \STATE 从$\mathcal{D}$中采样$N$个转移的批量$\{(s_i, a_i, r_i, s_{i+1})\}$
        
        \STATE 计算目标Q值:$y_i = r_i + \gamma Q_{\phi^-}(s_{i+1}, \mu_{\theta^-}(s_{i+1}))$
        \STATE 更新Critic:最小化损失$L = \frac{1}{N}\sum_i (y_i - Q_\phi(s_i, a_i))^2$
        \STATE 更新Actor:使用确定性策略梯度
        \[
        \nabla_\theta J \approx \frac{1}{N}\sum_i \nabla_a Q_\phi(s_i, a)|_{a=\mu_\theta(s_i)} \nabla_\theta \mu_\theta(s_i)
        \]
        
        \STATE 软更新目标网络:
        \[
        \theta^- \leftarrow \tau\theta + (1-\tau)\theta^-
        \]
        \[
        \phi^- \leftarrow \tau\phi + (1-\tau)\phi^-
        \]
    \ENDFOR
\ENDFOR
\end{algorithmic}
\end{algorithm}

\subsection{DDPG的关键技术}

\subsubsection{目标网络(Target Networks)}
\begin{itemize}
    \item 解决"追逐移动目标"问题
    \item 使用软更新:$\theta^- \leftarrow \tau\theta + (1-\tau)\theta^-$
    \item $\tau$通常很小(如0.001),使目标网络变化缓慢
\end{itemize}

\subsubsection{经验回放(Experience Replay)}
\begin{itemize}
    \item 打破数据相关性
    \item 提高数据利用率
    \item 支持离线策略学习
\end{itemize}

\subsubsection{探索噪声}
\begin{itemize}
    \item 时间相关噪声:Ornstein-Uhlenbeck过程
    \[
    d\mathcal{N}_t = \theta(\mu - \mathcal{N}_t)dt + \sigma dW_t
    \]
    \item 随时间衰减:训练后期减少探索
\end{itemize}

\subsection{DDPG的改进:TD3}

TD3(Twin Delayed DDPG)解决了DDPG的三个问题:

\begin{enumerate}
    \item \textbf{过估计问题}:使用两个Critic网络,取最小值
    \[
    y = r + \gamma \min_{j=1,2} Q_{\phi_j^-}(s', \mu_{\theta^-}(s') + \epsilon)
    \]
    \item \textbf{方差累积}:添加目标策略平滑
    \[
    a' = \mu_{\theta^-}(s') + \text{clip}(\mathcal{N}(0, \sigma), -c, c)
    \]
    \item \textbf{策略更新延迟}:Critic更新多次后,Actor才更新一次
\end{enumerate}

\subsection{DDPG的改进:SAC}

SAC(Soft Actor-Critic)结合了随机策略和离线策略学习的优点:

\begin{itemize}
    \item \textbf{最大化熵目标}:
    \[
    J(\theta) = \mathbb{E}\left[\sum_{t} r(s_t, a_t) + \alpha \mathcal{H}(\pi_\theta(\cdot|s_t))\right]
    \]
    \item \textbf{自动调整温度参数}:$\alpha$控制熵的重要性
    \item \textbf{随机策略}:输出高斯分布参数
    \item \textbf{离线策略}:使用经验回放
    \item \textbf{双Q网络}:防止过估计
\end{itemize}

\section{方法对比与总结}

\subsection{三种方法全面对比}

\begin{table}[H]
\centering
\caption{连续控制方法全面对比}
\begin{tabular}{p{0.25\textwidth}p{0.2\textwidth}p{0.2\textwidth}p{0.2\textwidth}}
\toprule
\textbf{特性} & \textbf{离散化方法} & \textbf{随机策略梯度} & \textbf{确定性策略梯度} \\
\midrule
\textbf{核心思想} & 连续空间离散化 & 学习概率分布 & 学习确定性映射 \\
\textbf{策略输出} & 离散动作的Q值 & 分布参数$(\mu, \sigma)$ & 动作向量$a$ \\
\textbf{动作选择} & $\arg\max Q(s,a)$ & 从分布中采样 & 直接输出+噪声 \\
\textbf{探索机制} & $\epsilon$-greedy等 & 分布采样 & 手动添加噪声 \\
\textbf{学习方式} & 价值学习 & 策略梯度 & 确定性策略梯度 \\
\textbf{样本效率} & 中等 & 较低(在线) & 高(离线) \\
\textbf{收敛速度} & 慢 & 中等 & 快 \\
\textbf{稳定性} & 中等 & 高 & 较低 \\
\textbf{实现难度} & 简单 & 中等 & 复杂 \\
\textbf{代表算法} & DQN(离散化) & PPO, TRPO, A3C & DDPG, TD3, SAC \\
\textbf{适用场景} & 低维简单控制 & 需要探索的任务 & 高维精确控制 \\
\bottomrule
\end{tabular}
\end{table}

\subsection{选择指南}

\begin{table}[H]
\centering
\caption{方法选择指南}
\begin{tabular}{p{0.5\textwidth}p{0.5\textwidth}}
\toprule
\textbf{考虑因素} & \textbf{推荐方法} \\
\midrule
动作维度低(1-2维),精度要求不高 & 离散化方法 \\
需要充分探索,环境随机性大 & 随机策略梯度(PPO) \\
样本效率重要,有大量历史数据 & 确定性策略梯度(DDPG/TD3/SAC) \\
控制精度要求高,需要平滑控制 & 确定性策略梯度 \\
安全关键应用,需要稳定训练 & 随机策略梯度(PPO/TRPO) \\
计算资源有限 & 离散化或PPO \\
需要快速原型验证 & 离散化或PPO \\
高维复杂控制(如人形机器人) & SAC或TD3 \\
\bottomrule
\end{tabular}
\end{table}

\subsection{实际应用建议}

\subsubsection{网络架构设计}
\begin{itemize}
    \item \textbf{输入归一化}:状态输入标准化到$[-1, 1]$
    \item \textbf{输出缩放}:动作输出缩放到实际范围
    \item \textbf{激活函数}:中间层使用ReLU,输出层使用tanh
    \item \textbf{参数初始化}:最后一层小权重初始化
\end{itemize}
\subsubsection{训练技巧}
\begin{enumerate}
    \item \textbf{热身阶段}:开始时随机探索收集数据
    \item \textbf{探索衰减}:训练后期减少探索噪声
    \item \textbf{学习率衰减}:训练后期降低学习率
    \item \textbf{监控指标}:
    \begin{itemize}
        \item 回合奖励
        \item Q值范围
        \item 策略熵
        \item 梯度范数
    \end{itemize}
    \item \textbf{正则化}:
    \begin{itemize}
        \item 权重衰减
        \item 梯度裁剪
        \item 批归一化
    \end{itemize}
\end{enumerate}

\subsection{常见问题与解决方案}

\begin{table}[H]
\centering
\caption{常见问题与解决方案}
\begin{tabular}{p{0.3\textwidth}p{0.65\textwidth}}
\toprule
\textbf{问题} & \textbf{解决方案} \\
\midrule
训练不稳定,奖励震荡 & 减小学习率,增加批量大小,使用目标网络 \\
探索不足,陷入局部最优 & 增加探索噪声,提高熵奖励,重启探索 \\
过拟合,泛化能力差 & 使用dropout,数据增强,早停法 \\
梯度爆炸/消失 & 梯度裁剪,合适的初始化,批归一化 \\
收敛速度慢 & 调整学习率,优化网络结构,并行收集数据 \\
Q值过高估计 & 使用双Q学习,延迟更新,目标策略平滑 \\
探索与利用不平衡 & 自适应探索策略,好奇心驱动探索 \\
\bottomrule
\end{tabular}
\end{table}

\section{前沿发展与未来方向}

\subsection{当前研究趋势}

\begin{enumerate}
    \item \textbf{样本效率提升}
    \begin{itemize}
        \item 模型基强化学习
        \item 元学习
        \item 模仿学习
    \end{itemize}
    
    \item \textbf{探索策略改进}
    \begin{itemize}
        \item 好奇心驱动探索
        \item 内在动机
        \item 基于不确定性的探索
    \end{itemize}
    
    \item \textbf{多任务与迁移学习}
    \begin{itemize}
        \item 共享表示学习
        \item 渐进式网络
        \item 蒸馏学习
    \end{itemize}
    
    \item \textbf{安全与稳健性}
    \begin{itemize}
        \item 约束强化学习
        \item 鲁棒强化学习
        \item 安全探索
    \end{itemize}
    
    \item \textbf{分布式与并行训练}
    \begin{itemize}
        \item 大规模分布式训练
        \item 异步优化
        \item 联邦强化学习
    \end{itemize}
\end{enumerate}

\subsection{重要算法进展}

\begin{table}[H]
\centering
\caption{近年重要连续控制算法}
\begin{tabular}{p{0.2\textwidth}p{0.3\textwidth}p{0.4\textwidth}}
\toprule
\textbf{算法} & \textbf{年份} & \textbf{主要贡献} \\
\midrule
DDPG & 2015 & 深度确定性策略梯度,结合DQN和DPG \\
PPO & 2017 & 简单高效的策略梯度算法,成为标准基准 \\
TD3 & 2018 & 解决DDPG的过估计问题,更稳定 \\
SAC & 2018 & 最大熵强化学习,随机策略+离线策略 \\
MPO & 2018 & 相对熵策略搜索,更理论化方法 \\
SAC-v2 & 2019 & 自动调整温度参数,更易使用 \\
REDQ & 2021 & 随机集成双Q学习,高样本效率 \\
DrQ & 2021 & 数据增强强化学习,改进样本效率 \\
\bottomrule
\end{tabular}
\end{table}