\chapter{深度强化学习之价值学习}
\section{价值学习:从评估到决策}

\subsection{价值学习的核心思想}

在强化学习中,价值学习(Value-Based Learning)是一类重要的方法。与直接学习策略(策略学习)不同,价值学习的核心思想是学习一个\textbf{价值函数}(Value Function),这个函数用于评估在某个状态(或状态-动作对)下的"好坏"程度。

\begin{definition}[最优动作价值函数]
最优动作价值函数 $Q^*(s, a)$ 定义为在所有可能的策略中,从状态 $s$ 开始、执行动作 $a$ 后能获得的最大期望回报:
\[
Q^*(s, a) = \max_{\pi} Q_{\pi}(s, a), \quad \forall s \in \mathcal{S}, a \in \mathcal{A}
\]
其中 $Q_{\pi}(s, a)$ 是在策略 $\pi$ 下的动作价值函数。
\end{definition}

\textbf{关键洞察}:一旦我们知道了最优动作价值函数 $Q^*(s, a)$,最优策略就可以直接得到:在每个状态 $s$ 下,选择使 $Q^*(s, a)$ 最大的动作:
\[
\pi^*(s) = \arg\max_{a \in \mathcal{A}} Q^*(s, a)
\]

\textbf{价值学习的优点}:
\begin{itemize}
    \item \textbf{直观性}:Q值直接反映了动作的"好坏"。
    \item \textbf{稳定性}:相比策略学习方法,价值学习通常更稳定。
    \item \textbf{可解释性}:Q值表或Q网络提供了对决策过程的直观理解。
\end{itemize}

\textbf{价值学习的挑战}:
\begin{itemize}
    \item \textbf{维度灾难}:当状态空间或动作空间很大时,存储所有状态-动作对的Q值变得不可行。
    \item \textbf{连续空间}:对于连续状态或动作空间,无法枚举所有可能情况。
    \item \textbf{泛化能力}:需要从有限的经验中泛化到未见过的状态。
\end{itemize}

\section{Q-Learning:经典的价值学习算法}

\subsection{算法原理}

Q-Learning是强化学习中最经典、最重要的算法之一。它是一种\textbf{离线策略}(off-policy)的\textbf{时序差分}(Temporal Difference, TD)学习方法。

\begin{definition}[Q-Learning更新规则]
Q-Learning通过以下规则更新Q值估计:
\[
Q(s, a) \leftarrow Q(s, a) + \alpha \left[ r + \gamma \max_{a'} Q(s', a') - Q(s, a) \right]
\]
其中:
\begin{itemize}
    \item $\alpha \in (0, 1]$ 是学习率(步长)
    \item $\gamma \in [0, 1]$ 是折扣因子
    \item $r$ 是执行动作 $a$ 后获得的即时奖励
    \item $s'$ 是转移后的新状态
    \item $\max_{a'} Q(s', a')$ 是在新状态 $s'$ 下所有可能动作的最大Q值
\end{itemize}
\end{definition}

\textbf{直观理解}:
\begin{itemize}
    \item $Q(s, a)$ 是当前对在状态 $s$ 下执行动作 $a$ 的价值的估计。
    \item $r + \gamma \max_{a'} Q(s', a')$ 是TD目标,包含两部分:
    \begin{enumerate}
        \item 即时奖励 $r$
        \item 折扣后的未来最大可能价值 $\gamma \max_{a'} Q(s', a')$
    \end{enumerate}
    \item $r + \gamma \max_{a'} Q(s', a') - Q(s, a)$ 是TD误差,表示当前估计与目标之间的差距。
    \item 学习率 $\alpha$ 控制更新步长:$\alpha$ 越大,更新越快,但可能不稳定;$\alpha$ 越小,更新越慢,但更稳定。
\end{itemize}

\subsection{表格Q-Learning}

在状态空间和动作空间都很小的情况下,我们可以使用表格形式存储所有状态-动作对的Q值。

\begin{table}[H]
\centering
\caption{表格Q-Learning中的Q值表示例}
\begin{tabular}{c|cccc}
\hline
\textbf{状态} & \textbf{动作1} & \textbf{动作2} & \textbf{动作3} & \textbf{动作4} \\
\hline
状态0 & $Q(0,1)$ & $Q(0,2)$ & $Q(0,3)$ & $Q(0,4)$ \\
状态1 & $Q(1,1)$ & $Q(1,2)$ & $Q(1,3)$ & $Q(1,4)$ \\
状态2 & $Q(2,1)$ & $Q(2,2)$ & $Q(2,3)$ & $Q(2,4)$ \\
$\vdots$ & $\vdots$ & $\vdots$ & $\vdots$ & $\vdots$ \\
状态m & $Q(m,1)$ & $Q(m,2)$ & $Q(m,3)$ & $Q(m,4)$ \\
\hline
\end{tabular}
\end{table}

\begin{algorithm}[H]
\caption{表格Q-Learning算法}
\begin{algorithmic}[1]
\REQUIRE 学习率 $\alpha$,折扣因子 $\gamma$,探索率 $\epsilon$
\ENSURE 最优动作价值函数 $Q^*(s, a)$
\STATE 初始化 $Q(s, a)$ 为任意值(通常为0或随机小值)
\FOR{每个回合(episode)}
    \STATE 初始化状态 $s$
    \WHILE{$s$ 不是终止状态}
        \STATE 根据 $\epsilon$-贪婪策略从 $s$ 选择动作 $a$:
        \[
        a = \begin{cases}
        \text{随机动作} & \text{以概率 } \epsilon \\
        \arg\max_{a'} Q(s, a') & \text{以概率 } 1-\epsilon
        \end{cases}
        \]
        \STATE 执行动作 $a$,观察奖励 $r$ 和新状态 $s'$
        \STATE 更新Q值:$Q(s, a) \leftarrow Q(s, a) + \alpha [r + \gamma \max_{a'} Q(s', a') - Q(s, a)]$
        \STATE $s \leftarrow s'$
    \ENDWHILE
\ENDFOR
\end{algorithmic}
\end{algorithm}

\subsection{探索与利用的平衡:$\epsilon$-贪婪策略}

在Q-Learning中,$\epsilon$-贪婪策略用于平衡探索(尝试新动作)和利用(选择已知最佳动作):

\[
\pi(a|s) = \begin{cases}
1 - \epsilon + \frac{\epsilon}{|\mathcal{A}|} & \text{如果 } a = \arg\max_{a'} Q(s, a') \\
\frac{\epsilon}{|\mathcal{A}|} & \text{其他动作}
\end{cases}
\]

\begin{itemize}
    \item $\epsilon \in [0, 1]$ 是探索率
    \item 以概率 $1-\epsilon$ 选择当前认为最好的动作(利用)
    \item 以概率 $\epsilon$ 随机选择动作(探索)
    \item 通常随着训练进行,$\epsilon$ 会逐渐减小(从高探索到高利用)
\end{itemize}

\subsection{Q-Learning与SARSA的比较}

\begin{table}[H]
\centering
\caption{Q-Learning与SARSA的比较}
\begin{tabular}{p{0.45\textwidth}p{0.45\textwidth}}
\toprule
\textbf{Q-Learning(离线策略)} & \textbf{SARSA(在线策略)} \\
\midrule
\textbf{更新公式}:$Q(s,a) \leftarrow Q(s,a) + \alpha[r + \gamma \max_{a'} Q(s',a') - Q(s,a)]$ & \textbf{更新公式}:$Q(s,a) \leftarrow Q(s,a) + \alpha[r + \gamma Q(s',a') - Q(s,a)]$ \\
\hline
\textbf{目标策略}:贪婪策略 $\pi(s) = \arg\max_{a} Q(s,a)$ & \textbf{目标策略}:与行为策略相同(通常为$\epsilon$-贪婪) \\
\hline
\textbf{行为策略}:$\epsilon$-贪婪策略 & \textbf{行为策略}:$\epsilon$-贪婪策略 \\
\hline
\textbf{TD目标}:$r + \gamma \max_{a'} Q(s',a')$ & \textbf{TD目标}:$r + \gamma Q(s',a')$ \\
\hline
\textbf{学习对象}:最优Q函数 $Q^*$ & \textbf{学习对象}:当前策略的Q函数 $Q_\pi$ \\
\hline
\textbf{探索影响}:行为策略探索不影响目标策略 & \textbf{探索影响}:探索直接影响学习策略 \\
\hline
\textbf{收敛性}:收敛到最优策略(在适当条件下) & \textbf{收敛性}:收敛到$\epsilon$-贪婪策略的最优策略 \\
\bottomrule
\end{tabular}
\end{table}

\textbf{关键区别}:
\begin{itemize}
    \item Q-Learning在计算TD目标时使用$\max_{a'} Q(s',a')$,假设下一个动作是最优的。
    \item SARSA在计算TD目标时使用实际采取的下一个动作$Q(s',a')$,更"谨慎"。
    \item 这导致Q-Learning更"乐观",SARSA更"保守"。
\end{itemize}

\section{深度Q网络:当Q-Learning遇见深度学习}

\subsection{表格方法的局限性}

表格Q-Learning虽然直观,但面临严重限制:

\begin{enumerate}
    \item \textbf{维度灾难}:状态空间随维度指数增长。例如:
    \begin{itemize}
        \item 围棋:$10^{170}$ 个状态,无法存储
        \item Atari游戏:$256^{84\times84}$ 个可能状态,天文数字
        \item 连续状态空间:无限多个状态
    \end{itemize}
    
    \item \textbf{泛化能力差}:每个状态-动作对独立学习,无法泛化到相似状态。
    
    \item \textbf{内存需求巨大}:存储所有Q值需要大量内存。
    
    \item \textbf{学习效率低}:每个状态需要单独学习,无法利用状态间的相似性。
\end{enumerate}

\subsection{深度Q网络的基本思想}

深度Q网络(Deep Q-Network, DQN)的核心思想是用神经网络来近似Q函数,从而解决维度灾难问题。

\begin{definition}[深度Q网络]
深度Q网络是一个参数化的函数 $Q(s, a; \theta)$,其中 $\theta$ 是神经网络的参数。对于给定的状态 $s$,网络输出所有可能动作的Q值:
\[
f(s; \theta) \approx [Q(s, a_1), Q(s, a_2), \ldots, Q(s, a_n)]
\]
其中 $n$ 是动作空间的大小。
\end{definition}

\begin{figure}[H]
\centering
\includegraphics[width=0.8\textwidth]{D:/code LateX/deep_learning/picture/net.png}
\caption{深度Q网络结构示意图:输入状态,输出每个动作的Q值}
\end{figure}

\textbf{网络架构}:
\begin{itemize}
    \item \textbf{输入}:状态表示(如图像像素、特征向量等)
    \item \textbf{隐藏层}:多个全连接层或卷积层,用于提取特征
    \item \textbf{输出层}:每个输出节点对应一个动作的Q值
\end{itemize}

\subsection{DQN的训练目标与损失函数}

DQN的训练目标是找到参数 $\theta$,使得 $Q(s, a; \theta)$ 近似最优Q函数 $Q^*(s, a)$。

\begin{definition}[DQN损失函数]
使用均方误差(MSE)作为损失函数:
\[
L(\theta) = \mathbb{E}_{(s,a,r,s') \sim \mathcal{D}} \left[ \left( Q(s, a; \theta) - y \right)^2 \right]
\]
其中:
\begin{itemize}
    \item $(s, a, r, s')$ 是从经验回放缓冲区 $\mathcal{D}$ 中采样的转移
    \item $y$ 是TD目标:$y = r + \gamma \max_{a'} Q(s', a'; \theta^-)$
    \item $\theta^-$ 是目标网络的参数(与在线网络 $\theta$ 不同)
\end{itemize}
\end{definition}

\textbf{梯度计算}:
\[
\nabla_{\theta} L(\theta) = \mathbb{E}_{(s,a,r,s') \sim \mathcal{D}} \left[ \left( Q(s, a; \theta) - y \right) \cdot \nabla_{\theta} Q(s, a; \theta) \right]
\]

\textbf{参数更新}:
\[
\theta \leftarrow \theta - \alpha \cdot \nabla_{\theta} L(\theta)
\]
其中 $\alpha$ 是学习率。

\subsection{DQN的训练流程}

\begin{algorithm}[H]
\caption{深度Q网络(DQN)算法}
\begin{algorithmic}[1]
\REQUIRE 经验回放缓冲区容量 $N$,目标网络更新频率 $C$,折扣因子 $\gamma$,探索率 $\epsilon$
\STATE 初始化在线网络 $Q(\cdot; \theta)$ 的参数 $\theta$ 随机
\STATE 初始化目标网络 $Q(\cdot; \theta^-)$ 的参数 $\theta^- \leftarrow \theta$
\STATE 初始化经验回放缓冲区 $\mathcal{D}$ 为空,容量为 $N$
\FOR{每个回合(episode)}
    \STATE 初始化状态 $s_1$ 和预处理 $\phi_1 = \phi(s_1)$
    \FOR{$t = 1$ 到 $T$}
        \STATE 以概率 $\epsilon$ 选择随机动作 $a_t$,否则 $a_t = \arg\max_{a} Q(\phi(s_t), a; \theta)$
        \STATE 执行动作 $a_t$,观察奖励 $r_t$ 和下一状态 $s_{t+1}$
        \STATE 预处理 $\phi_{t+1} = \phi(s_{t+1})$
        \STATE 存储转移 $(\phi_t, a_t, r_t, \phi_{t+1})$ 到 $\mathcal{D}$
        \STATE 从 $\mathcal{D}$ 中随机采样小批量转移 $(\phi_j, a_j, r_j, \phi_{j+1})$
        \STATE 计算TD目标:
        \[
        y_j = \begin{cases}
        r_j & \text{如果 } \phi_{j+1} \text{ 是终止状态} \\
        r_j + \gamma \max_{a'} Q(\phi_{j+1}, a'; \theta^-) & \text{否则}
        \end{cases}
        \]
        \STATE 计算损失:$L = \frac{1}{m} \sum_{j=1}^{m} (y_j - Q(\phi_j, a_j; \theta))^2$
        \STATE 使用梯度下降更新 $\theta$
        \STATE 每 $C$ 步更新目标网络:$\theta^- \leftarrow \theta$
    \ENDFOR
\ENDFOR
\end{algorithmic}
\end{algorithm}

\section{DQN的核心技术}

\subsection{经验回放(Experience Replay)}

经验回放是DQN成功的关键技术之一,解决了两个核心问题:

\begin{enumerate}
    \item \textbf{数据效率}:每个转移可以被多次使用,提高数据利用率。
    \item \textbf{相关性破坏}:随机采样打破了连续状态之间的相关性,使训练更稳定。
\end{enumerate}

\begin{figure}[H]
\centering
\includegraphics[width=0.7\textwidth]{D:/code LateX/deep_learning/picture/recall.png}
\caption{经验回放机制:存储历史转移并随机采样用于训练}
\end{figure}

\begin{definition}[经验回放缓冲区]
经验回放缓冲区 $\mathcal{D}$ 是一个固定大小的循环缓冲区,存储转移元组 $(s, a, r, s', \text{done})$,其中:
\begin{itemize}
    \item $s$:当前状态
    \item $a$:执行的动作
    \item $r$:获得的奖励
    \item $s'$:转移后的新状态
    \item $\text{done}$:是否到达终止状态
\end{itemize}
\end{definition}

\textbf{经验回放的工作原理}:
\begin{enumerate}
    \item \textbf{收集}:智能体与环境交互,将每个转移存储到缓冲区。
    \item \textbf{采样}:训练时随机从缓冲区采样小批量转移。
    \item \textbf{学习}:使用采样到的转移计算损失并更新网络。
\end{enumerate}

\textbf{经验回放的优势}:
\begin{itemize}
    \item \textbf{打破相关性}:连续的状态高度相关,直接用于训练会导致梯度更新方向高度相关,训练不稳定。
    \item \textbf{数据重用}:每个转移可以被多次用于训练,提高数据效率。
    \item \textbf{平滑分布}:随机采样使数据分布更平稳,减少方差。
\end{itemize}

\subsection{目标网络(Target Network)}

目标网络是DQN的另一项关键技术,解决了训练中的不稳定性问题。

\begin{definition}[目标网络]
目标网络 $Q(\cdot; \theta^-)$ 是与在线网络 $Q(\cdot; \theta)$ 结构相同的网络,但参数更新更慢:
\begin{itemize}
    \item \textbf{硬更新}:每 $C$ 步将在线网络的参数复制给目标网络:$\theta^- \leftarrow \theta$
    \item \textbf{软更新}:每次迭代按比例更新:$\theta^- \leftarrow \tau \theta + (1-\tau) \theta^-$,其中 $\tau \ll 1$
\end{itemize}
\end{definition}

\textbf{目标网络的作用}:
\begin{enumerate}
    \item \textbf{稳定训练}:TD目标 $y = r + \gamma \max_{a'} Q(s', a'; \theta^-)$ 在一段时间内固定,减少了目标的波动。
    \item \textbf{避免发散}:防止Q值估计的"追逐尾巴"现象(目标随估计不断变化,导致训练发散)。
    \item \textbf{缓解过估计}:目标网络使用旧参数,减少了最大化操作带来的过估计问题。
\end{enumerate}

\begin{figure}[H]
\centering
\includegraphics[width=0.8\textwidth]{D:/code LateX/deep_learning/picture/dqnloss.png}
\caption{DQN损失计算:使用目标网络计算TD目标}
\end{figure}

\textbf{为什么需要目标网络?}

考虑没有目标网络的情况:
\[
y = r + \gamma \max_{a'} Q(s', a'; \theta)
\]
损失函数:
\[
L(\theta) = \mathbb{E}[(Q(s, a; \theta) - y)^2]
\]
问题:目标 $y$ 依赖于正在优化的参数 $\theta$,导致:
\begin{enumerate}
    \item 目标不断变化,训练不稳定。
    \item Q值可能发散("追逐尾巴"问题)。
\end{enumerate}

使用目标网络后:
\[
y = r + \gamma \max_{a'} Q(s', a'; \theta^-)
\]
目标网络参数 $\theta^-$ 更新较慢,提供了更稳定的学习目标。

\section{DQN的改进与扩展}

\subsection{优先经验回放(Prioritized Experience Replay)}

标准经验回放均匀采样,但不同转移的重要性不同。优先经验回放根据TD误差的绝对值赋予不同采样优先级。

\begin{definition}[采样优先级]
转移 $i$ 的采样概率为:
\[
P(i) = \frac{p_i^{\alpha}}{\sum_k p_k^{\alpha}}
\]
其中:
\begin{itemize}
    \item $p_i$ 是转移 $i$ 的优先级
    \item $\alpha \in [0, 1]$ 控制优先程度的强度($\alpha=0$ 时退化为均匀采样)
\end{itemize}
\end{definition}

\textbf{优先级计算方法}:
\begin{enumerate}
    \item \textbf{基于TD误差}:$p_i = |\delta_i| + \epsilon$,其中 $\delta_i$ 是转移 $i$ 的TD误差,$\epsilon$ 是小的正常数。
    \item \textbf{基于排名}:$p_i = \frac{1}{\text{rank}(i)}$,其中 $\text{rank}(i)$ 是基于 $|\delta_i|$ 的排名。
\end{enumerate}

\textbf{重要性采样权重}:
由于非均匀采样引入偏差,需要重要性采样权重校正:
\[
w_i = \left( \frac{1}{N} \cdot \frac{1}{P(i)} \right)^{\beta}
\]
其中 $\beta \in [0, 1]$ 控制校正程度,训练中从 $\beta_{\text{初始}}$ 线性增加到1。

\textbf{损失函数}:
\[
L(\theta) = \sum_{i=1}^{m} w_i \cdot (y_i - Q(s_i, a_i; \theta))^2
\]

\textbf{为什么重要性采样有效?}

考虑两种学习方式:
\begin{enumerate}
    \item \textbf{均匀采样}:学习率 $\alpha$,用样本 $(s_j, a_j, r_j, s_{j+1})$ 更新一次。
    \item \textbf{优先采样}:学习率 $\alpha_j = w_j \cdot \alpha$,用重要样本更新多次。
\end{enumerate}

虽然学习率减小了,但重要样本被更频繁地使用,相当于用更多计算量从重要样本中提取更多信息。

\subsection{Double DQN}

标准DQN存在过估计(overestimation)问题:$\max$ 操作倾向于选择被高估的动作。

\begin{theorem}[过估计定理]
设 $X_1, \ldots, X_n$ 是独立同分布的随机变量,均值为 $\mu$,方差为 $\sigma^2$。则:
\[
\mathbb{E}[\max(X_1, \ldots, X_n)] \geq \mu
\]
等号成立当且仅当 $n=1$ 或 $\sigma^2=0$。
\end{theorem}

\textbf{证明}:由Jensen不等式,$\max$ 是凸函数,所以 $\mathbb{E}[\max(X_i)] \geq \max(\mathbb{E}[X_i]) = \mu$。

在DQN中,Q值估计包含噪声(来自函数近似、环境随机性等),$\max$ 操作会放大正误差,导致过估计。

\begin{definition}[Double DQN]
Double DQN(DDQN)解耦动作选择和价值评估:
\[
y = r + \gamma Q\left(s', \arg\max_{a'} Q(s', a'; \theta); \theta^-\right)
\]
其中:
\begin{itemize}
    \item 使用在线网络 $\theta$ 选择动作:$a^* = \arg\max_{a'} Q(s', a'; \theta)$
    \item 使用目标网络 $\theta^-$ 评估价值:$Q(s', a^*; \theta^-)$
\end{itemize}
\end{definition}

\begin{table}[H]
\centering
\caption{不同Q-Learning变种的比较}
\begin{tabular}{p{0.25\textwidth}p{0.25\textwidth}p{0.2\textwidth}p{0.2\textwidth}}
\toprule
\textbf{算法} & \textbf{动作选择} & \textbf{价值评估} & \textbf{过估计程度} \\
\midrule
原始Q-Learning & DQN网络 & DQN网络 & 严重 \\
DQN+目标网络 & 目标网络 & 目标网络 & 中等 \\
Double DQN & DQN网络 & 目标网络 & 轻微 \\
\bottomrule
\end{tabular}
\end{table}

\textbf{Double DQN的优势}:
\begin{enumerate}
    \item 减少过估计,提高价值估计的准确性。
    \item 在某些任务上性能显著优于标准DQN。
    \item 实现简单,只需修改TD目标计算方式。
\end{enumerate}

\subsection{Dueling DQN}

Dueling DQN改进了网络结构,将Q值分解为状态价值 $V(s)$ 和优势函数 $A(s, a)$。

\begin{definition}[Dueling架构]
\[
Q(s, a; \theta, \alpha, \beta) = V(s; \theta, \beta) + A(s, a; \theta, \alpha) - \frac{1}{|\mathcal{A}|} \sum_{a'} A(s, a'; \theta, \alpha)
\]
其中:
\begin{itemize}
    \item $V(s; \theta, \beta)$:状态价值函数,评估状态 $s$ 的好坏
    \item $A(s, a; \theta, \alpha)$:优势函数,评估动作 $a$ 相对于平均水平的优势
    \item 减去的项确保优势函数的平均值为0,解决可识别性问题
\end{itemize}
\end{definition}

\textbf{可识别性问题}:如果直接定义 $Q(s, a) = V(s) + A(s, a)$,那么对于常数 $c$,有:
\[
Q(s, a) = [V(s) + c] + [A(s, a) - c]
\]
即 $V(s)$ 和 $A(s, a)$ 不唯一。通过减去优势函数的平均值解决这一问题。

\begin{figure}[H]
\centering
\includegraphics[width=0.8\textwidth]{D:/code LateX/deep_learning/picture/2.png}
\caption{Dueling DQN网络结构:将Q值分解为状态价值和优势函数}
\end{figure}

\textbf{Dueling DQN的优势}:
\begin{enumerate}
    \item \textbf{更好的泛化}:可以学习哪些状态有价值,而不需要了解每个动作的影响。
    \item \textbf{更稳定的学习}:状态价值提供基线,减少方差。
    \item \textbf{更好的策略评估}:在状态价值相似但动作价值不同的情况下表现更好。
\end{enumerate}

\subsection{Multi-Step DQN}

标准DQN使用单步TD目标,只考虑一步奖励。Multi-Step DQN考虑多步奖励,减少短视行为。

\begin{definition}[n步TD目标]
n步TD目标考虑未来n步的奖励:
\[
y_t^{(n)} = \sum_{i=0}^{n-1} \gamma^i r_{t+i} + \gamma^n \max_{a'} Q(s_{t+n}, a'; \theta^-)
\]
特别地:
\begin{itemize}
    \item $n=1$:标准DQN,$y_t^{(1)} = r_t + \gamma \max_{a'} Q(s_{t+1}, a'; \theta^-)$
    \item $n=\infty$:蒙特卡洛方法,考虑整个回合的回报
\end{itemize}
\end{definition}

\textbf{多步学习的权衡}:
\begin{itemize}
    \item \textbf{偏差-方差权衡}:n越大,偏差越小(使用更多真实奖励),方差越大(依赖更多随机步骤)。
    \item \textbf{更新延迟}:需要等待n步后才能更新,学习延迟增加。
\end{itemize}

\textbf{实现方式}:
\begin{enumerate}
    \item 存储n步转移:$(s_t, a_t, r_t, \ldots, r_{t+n-1}, s_{t+n})$
    \item 计算n步TD目标
    \item 用标准DQN方式更新
\end{enumerate}

\subsection{Noisy DQN}

传统$\epsilon$-贪婪探索在动作空间添加噪声,Noisy DQN在参数空间添加噪声,实现更高效的探索。

\begin{definition}[Noisy DQN]
Noisy DQN在网络的权重中添加可学习的噪声:
\[
\tilde{Q}(s, a, \xi; \mu, \sigma) = Q(s, a; \mu + \sigma \odot \xi)
\]
其中:
\begin{itemize}
    \item $\mu$ 和 $\sigma$ 是可学习的参数
    \item $\xi$ 是随机噪声,从标准正态分布采样:$\xi \sim \mathcal{N}(0, 1)$
    \item $\odot$ 表示逐元素乘法
\end{itemize}
\end{definition}

\textbf{噪声注入方式}:
\begin{enumerate}
    \item \textbf{因子化高斯噪声}:减少参数数量,提高效率。
    \item \textbf{独立高斯噪声}:每个参数独立添加噪声。
\end{enumerate}

\textbf{训练过程}:
\begin{enumerate}
    \item 每个回合开始时采样噪声 $\xi$,并在整个回合中固定。
    \item 使用带噪声的网络选择动作:$a_t = \arg\max_{a} \tilde{Q}(s_t, a, \xi; \mu, \sigma)$
    \item 收集经验并计算损失。
    \item 通过损失函数同时优化 $\mu$ 和 $\sigma$。
\end{enumerate}

\begin{table}[H]
\centering
\caption{$\epsilon$-贪婪与Noisy DQN的比较}
\begin{tabular}{p{0.45\textwidth}p{0.45\textwidth}}
\toprule
\textbf{$\epsilon$-贪婪探索} & \textbf{Noisy DQN探索} \\
\midrule
\textbf{噪声位置}:动作空间 & \textbf{噪声位置}:参数空间 \\
\hline
\textbf{更新频率}:每个时间步独立决定 & \textbf{更新频率}:每个回合固定 \\
\hline
\textbf{探索连贯性}:低(独立随机) & \textbf{探索连贯性}:高(连贯的探索方向) \\
\hline
\textbf{探索与策略耦合}:解耦 & \textbf{探索与策略耦合}:强耦合(噪声是网络一部分) \\
\hline
\textbf{学习稳定性}:较低(频繁切换) & \textbf{学习稳定性}:较高(一致的方向) \\
\hline
\textbf{超参数}:需要手动调整 $\epsilon$ & \textbf{超参数}:噪声参数自动学习 \\
\bottomrule
\end{tabular}
\end{table}

\subsection{Distributional DQN}

传统DQN学习Q值的期望,Distributional DQN学习Q值的完整分布。

\begin{definition}[价值分布]
价值分布 $Z(s, a)$ 是一个随机变量,表示从状态 $s$ 执行动作 $a$ 后获得的随机回报。
传统Q值是价值分布的期望:
\[
Q(s, a) = \mathbb{E}[Z(s, a)]
\]
\end{definition}

\begin{figure}[H]
\centering
\includegraphics[width=0.7\textwidth]{D:/code LateX/deep_learning/picture/RL.png}
\caption{Distributional DQN:学习价值分布而非期望值}
\end{figure}

\subsubsection{C51算法}

C51(Categorical 51)是Distributional DQN的一种实现,使用51个类别的分类分布表示价值分布。

\textbf{核心思想}:
\begin{enumerate}
    \item 将价值范围 $[V_{\min}, V_{\max}]$ 离散化为 $N=51$ 个等间距的支持点(atoms):$z_i = V_{\min} + i \cdot \Delta z$,其中 $\Delta z = \frac{V_{\max} - V_{\min}}{N-1}$。
    \item 对于每个状态-动作对,网络输出这51个点的概率:$p_i(s, a) = \mathbb{P}(Z(s, a) = z_i)$。
    \item 价值分布的期望为:$Q(s, a) = \sum_{i=1}^{N} p_i(s, a) \cdot z_i$。
\end{enumerate}

\textbf{投影贝尔曼更新}:
对于转移 $(s, a, r, s')$,目标分布为:
\[
\mathcal{T} z_j = r + \gamma z_j
\]
将 $\mathcal{T} z_j$ 投影到原始支持点上,得到目标概率分布。

\textbf{损失函数}:使用交叉熵损失:
\[
L(\theta) = - \sum_{i=1}^{N} \hat{p}_i \log p_i(s, a; \theta)
\]
其中 $\hat{p}_i$ 是目标分布在第 $i$ 个atom上的概率。

\textbf{C51的优势}:
\begin{enumerate}
    \item \textbf{更丰富的学习信号}:分布提供比标量更多的信息。
    \item \textbf{更好的风险意识}:可以区分高方差和低方差情况。
    \item \textbf{更稳定的训练}:分布学习通常更稳定。
\end{enumerate}

\subsubsection{QR-DQN和IQN}

除了C51,还有其他分布RL算法:
\begin{itemize}
    \item \textbf{QR-DQN(Quantile Regression DQN)}:学习价值分布的分位数。
    \item \textbf{IQN(Implicit Quantile Networks)}:隐式学习分位数,更灵活。
\end{itemize}

\section{Rainbow:集成多种改进}

Rainbow算法集成了DQN的六种主要改进:
\begin{enumerate}
    \item 经验回放
    \item 目标网络
    \item Double DQN
    \item Dueling网络
    \item Multi-Step学习
    \item Noisy网络
    \item Distributional RL
\end{enumerate}

\textbf{Rainbow的关键设计}:
\begin{enumerate}
    \item \textbf{多步损失}:使用n步TD目标。
    \item \textbf{优先经验回放}:使用分布TD误差作为优先级。
    \item \textbf{Noisy网络}:用于探索。
    \item \textbf{Distributional RL}:学习价值分布。
    \item \textbf{Dueling架构}:分解状态价值和优势。
    \item \textbf{Double DQN}:解耦动作选择和评估。
\end{enumerate}

\textbf{消融实验结果}:
\begin{itemize}
    \item \textbf{最大贡献}:多步学习和优先经验回放。
    \item \textbf{中等贡献}:Distributional RL和Noisy网络。
    \item \textbf{较小贡献}:Dueling网络和Double DQN。
    \item \textbf{组合效应}:所有改进组合的效果远超单个改进。
\end{itemize}

\section{总结与比较}

\subsection{DQN变种对比}

\begin{table}[H]
\centering
\caption{DQN主要变种的比较}
\begin{tabular}{p{0.2\textwidth}p{0.25\textwidth}p{0.25\textwidth}p{0.2\textwidth}}
\toprule
\textbf{算法} & \textbf{核心创新} & \textbf{解决的问题} & \textbf{实现复杂度} \\
\midrule
\textbf{DQN} & 神经网络近似Q函数 & 维度灾难,连续状态空间 & 中等 \\
\textbf{优先经验回放} & 按TD误差优先级采样 & 样本效率低 & 低 \\
\textbf{Double DQN} & 解耦动作选择与评估 & Q值过估计 & 低 \\
\textbf{Dueling DQN} & Q=V+A架构 & 状态价值与动作优势分离 & 低 \\
\textbf{Multi-Step DQN} & n步TD目标 & 短视偏差 & 低 \\
\textbf{Noisy DQN} & 参数空间噪声探索 & 探索效率低 & 中等 \\
\textbf{Distributional DQN} & 学习价值分布 & 风险意识,丰富信号 & 高 \\
\textbf{Rainbow} & 集成所有改进 & 单一改进的局限性 & 高 \\
\bottomrule
\end{tabular}
\end{table}

\subsection{实际应用建议}

\begin{enumerate}
    \item \textbf{从标准DQN开始}:实现基础DQN,包含经验回放和目标网络。
    \item \textbf{逐步添加改进}:按优先级添加:Double DQN → 优先经验回放 → Multi-Step → 其他。
    \item \textbf{超参数调优}:
    \begin{itemize}
        \item 学习率:通常 $10^{-4}$ 到 $10^{-3}$
        \item 折扣因子 $\gamma$:0.99(长期任务)或 0.95(短期任务)
        \item 回放缓冲区大小:$10^5$ 到 $10^6$
        \item 批量大小:32 到 256
    \end{itemize}
    \item \textbf{监控训练}:
    \begin{itemize}
        \item 观察回报曲线
        \item 监控Q值范围(避免发散)
        \item 检查探索率衰减
    \end{itemize}
\end{enumerate}

\subsection{未来方向}

\begin{enumerate}
    \item \textbf{分布式强化学习}:多个智能体并行收集经验,加速训练。
    \item \textbf{元强化学习}:学习快速适应新任务的能力。
    \item \textbf{基于模型的DQN}:结合模型预测与价值学习。
    \item \textbf{多任务学习}:一个智能体学习多个相关任务。
    \item \textbf{安全约束}:在价值学习中加入安全约束。
\end{enumerate}

价值学习作为强化学习的核心方法,从经典的Q-Learning到现代的Rainbow,经历了显著的发展。理解这些算法的原理、优势和局限性,对于在实际问题中选择合适的算法至关重要。
