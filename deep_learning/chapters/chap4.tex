\chapter{深度强化学习之策略学习}

\section{引言:为什么需要策略学习?}

\subsection{价值学习方法的局限性}

在前面的学习中,我们探讨了价值学习(Value-Based Learning)方法,如Q-Learning、DQN及其变种。这些方法通过学习价值函数(Value Function)来间接得到策略,其核心思想是:找到最优动作价值函数$Q^*(s,a)$,然后通过贪心策略$\pi^*(s) = \arg\max_a Q^*(s,a)$得到最优策略。

然而,价值学习方法在某些场景下存在明显局限性:

\begin{table}[H]
\centering
\caption{价值学习方法的局限性}
\begin{tabular}{p{0.3\textwidth}p{0.65\textwidth}}
\toprule
\textbf{局限性类型} & \textbf{详细说明与示例} \\
\midrule
\textbf{大动作空间问题} & 
\begin{itemize}
    \item 当动作空间很大时,需要计算每个动作的Q值,计算成本高
    \item 例如:机器人控制可能有几十个连续关节,每个关节有多个自由度
    \item 在Atari游戏中,虽然动作空间有限,但某些游戏动作空间也很大
\end{itemize} \\
\hline
\textbf{连续动作空间} & 
\begin{itemize}
    \item 连续动作空间中无法通过$\arg\max$选择动作
    \item 例如:自动驾驶的转向角度是连续值
    \item 机械臂控制的关节角度是连续值
    \item 需要额外的优化过程来找到最大化Q值的动作
\end{itemize} \\
\hline
\textbf{随机策略表示困难} & 
\begin{itemize}
    \item 某些问题的最优策略本质上是随机的
    \item 例如:石头剪刀布游戏中,最优策略是以1/3概率选择每个动作
    \item 部分可观察马尔可夫决策过程(POMDP)中可能需要随机策略
    \item 探索时也需要一定随机性
\end{itemize} \\
\hline
\textbf{策略表示受限} & 
\begin{itemize}
    \item 通常只能表示确定性策略或$\epsilon$-贪婪策略
    \item 难以表示更复杂的策略形式
    \item 策略缺乏明确的概率解释
\end{itemize} \\
\bottomrule
\end{tabular}
\end{table}

\subsection{策略学习的优势}

策略学习(Policy Learning)或策略梯度(Policy Gradient)方法直接对策略进行参数化和优化,具有以下优势:

\begin{itemize}
    \item \textbf{直接优化目标}:直接优化累积奖励,不通过价值函数作为中间步骤
    \item \textbf{自然处理连续动作}:策略网络可以直接输出连续动作或动作分布参数
    \item \textbf{支持随机策略}:策略网络可以直接输出动作的概率分布
    \item \textbf{更好的收敛性}:在某些问题中,策略梯度方法有更好的收敛保证
    \item \textbf{探索与利用的自然平衡}:随机策略本身带有探索性质
\end{itemize}

\subsection{策略学习的基本框架}

策略学习的核心思想是:使用参数化的函数(通常是神经网络)直接表示策略$\pi_\theta(a|s)$,然后通过优化参数$\theta$来最大化期望累积奖励。

\begin{figure}[H]
\centering
\includegraphics[width=0.7\textwidth]{picture/3.png}
\caption{策略网络:输入状态,输出动作的概率分布}
\end{figure}

策略网络$\pi_\theta(a|s)$接收状态$s$作为输入,输出动作$a$的概率分布(对于离散动作空间)或动作分布的参数(对于连续动作空间)。

\section{策略梯度定理:理论基础}

\subsection{目标函数的定义}

我们的目标是最大化期望累积奖励。考虑折扣奖励情况,定义目标函数为:

\[
J(\theta) = \mathbb{E}_{\tau \sim p_\theta(\tau)} \left[ \sum_{t=0}^{T} \gamma^t r_{t+1} \right]
\]

其中:
\begin{itemize}
    \item $\tau = (s_0, a_0, r_1, s_1, a_1, r_2, \dots, s_T, a_T, r_{T+1})$是一条轨迹
    \item $p_\theta(\tau)$是在策略$\pi_\theta$下生成轨迹$\tau$的概率
    \item $\gamma \in [0,1]$是折扣因子
    \item $T$可以是有限的(分幕式任务)或无限的(持续式任务)
\end{itemize}

\subsection{轨迹概率的分解}

根据马尔可夫决策过程的性质,轨迹概率可以分解为:

\[
p_\theta(\tau) = p(s_0) \prod_{t=0}^{T} \pi_\theta(a_t|s_t) p(s_{t+1}|s_t, a_t)
\]

其中:
\begin{itemize}
    \item $p(s_0)$是初始状态分布
    \item $\pi_\theta(a_t|s_t)$是策略在状态$s_t$下选择动作$a_t$的概率
    \item $p(s_{t+1}|s_t, a_t)$是状态转移概率
\end{itemize}

\subsection{策略梯度推导}

我们需要计算目标函数$J(\theta)$关于参数$\theta$的梯度。直接计算梯度很困难,因为目标函数中包含对轨迹的期望。策略梯度定理提供了计算这个梯度的方法。

\begin{theorem}[策略梯度定理]
对于参数化的策略$\pi_\theta(a|s)$,目标函数$J(\theta)$的梯度为:
\[
\nabla_\theta J(\theta) = \mathbb{E}_{\tau \sim p_\theta(\tau)} \left[ \left( \sum_{t=0}^{T} \nabla_\theta \log \pi_\theta(a_t|s_t) \right) \left( \sum_{t=0}^{T} \gamma^t r_{t+1} \right) \right]
\]
\end{theorem}

\textbf{证明}:

\begin{align*}
\nabla_\theta J(\theta) &= \nabla_\theta \mathbb{E}_{\tau \sim p_\theta(\tau)} \left[ \sum_{t=0}^{T} \gamma^t r_{t+1} \right] \\
&= \nabla_\theta \int p_\theta(\tau) R(\tau) d\tau \quad \text{(其中 } R(\tau) = \sum_{t=0}^{T} \gamma^t r_{t+1} \text{)} \\
&= \int \nabla_\theta p_\theta(\tau) R(\tau) d\tau \\
&= \int p_\theta(\tau) \nabla_\theta \log p_\theta(\tau) R(\tau) d\tau \quad \text{(使用对数导数技巧)} \\
&= \mathbb{E}_{\tau \sim p_\theta(\tau)} \left[ \nabla_\theta \log p_\theta(\tau) R(\tau) \right]
\end{align*}

接下来计算$\nabla_\theta \log p_\theta(\tau)$:

\begin{align*}
\log p_\theta(\tau) &= \log p(s_0) + \sum_{t=0}^{T} \left[ \log \pi_\theta(a_t|s_t) + \log p(s_{t+1}|s_t, a_t) \right] \\
\nabla_\theta \log p_\theta(\tau) &= \sum_{t=0}^{T} \nabla_\theta \log \pi_\theta(a_t|s_t)
\end{align*}

因为$\log p(s_0)$和$\log p(s_{t+1}|s_t, a_t)$与$\theta$无关,所以它们的梯度为零。代入得:

\[
\nabla_\theta J(\theta) = \mathbb{E}_{\tau \sim p_\theta(\tau)} \left[ \left( \sum_{t=0}^{T} \nabla_\theta \log \pi_\theta(a_t|s_t) \right) R(\tau) \right]
\]

证毕。

\subsection{直观理解}

策略梯度定理可以直观地理解:
\begin{itemize}
    \item $\nabla_\theta \log \pi_\theta(a_t|s_t)$是指数族分布的性质,表示在当前参数下,增加动作$a_t$在状态$s_t$下概率的方向
    \item $R(\tau)$是整条轨迹的回报,作为权重
    \item 如果一条轨迹的回报$R(\tau)$很高,就增加这条轨迹中所有动作的概率
    \item 如果一条轨迹的回报$R(\tau)$很低,就减少这条轨迹中所有动作的概率
\end{itemize}

\section{REINFORCE算法:蒙特卡洛策略梯度}

\subsection{基本REINFORCE算法}

REINFORCE(REward Increment = Nonnegative Factor × Offset Reinforcement × Characteristic Eligibility)是最基础的策略梯度算法。它直接使用蒙特卡洛采样来估计策略梯度。

\begin{algorithm}[H]
\caption{REINFORCE算法}
\begin{algorithmic}[1]
\REQUIRE 策略参数$\theta$,学习率$\alpha$,折扣因子$\gamma$
\ENSURE 最优策略参数$\theta^*$
\STATE 初始化策略参数$\theta$
\FOR{每个回合(episode)}
    \STATE 使用当前策略$\pi_\theta$生成一条轨迹:$\tau = (s_0, a_0, r_1, s_1, a_1, r_2, \dots, s_T, a_T, r_{T+1})$
    \STATE 计算轨迹回报:$R(\tau) = \sum_{t=0}^{T} \gamma^t r_{t+1}$
    \FOR{$t = 0$ 到 $T$}
        \STATE 计算梯度:$g_t = \nabla_\theta \log \pi_\theta(a_t|s_t)$
    \ENDFOR
    \STATE 估计梯度:$\nabla_\theta J(\theta) \approx \left( \sum_{t=0}^{T} g_t \right) R(\tau)$
    \STATE 更新参数:$\theta \leftarrow \theta + \alpha \nabla_\theta J(\theta)$
\ENDFOR
\end{algorithmic}
\end{algorithm}

\subsection{REINFORCE的改进:因果关系修正}

基本REINFORCE算法有一个问题:在时刻$t$采取的动作$a_t$不应该影响$t$时刻之前的奖励。因此,我们使用从时刻$t$开始的折扣回报:

\[
G_t = \sum_{k=t}^{T} \gamma^{k-t} r_{k+1}
\]

改进后的策略梯度为:

\[
\nabla_\theta J(\theta) = \mathbb{E}_{\tau \sim p_\theta(\tau)} \left[ \sum_{t=0}^{T} \nabla_\theta \log \pi_\theta(a_t|s_t) G_t \right]
\]

对应的算法更新为:

\begin{algorithm}[H]
\caption{REINFORCE with Causality}
\begin{algorithmic}[1]
\REQUIRE 策略参数$\theta$,学习率$\alpha$,折扣因子$\gamma$
\STATE 初始化策略参数$\theta$
\FOR{每个回合(episode)}
    \STATE 使用当前策略$\pi_\theta$生成一条轨迹:$\tau = (s_0, a_0, r_1, s_1, a_1, r_2, \dots, s_T, a_T, r_{T+1})$
    \STATE 计算每个时刻的回报$G_t$:
    \FOR{$t = T$ 到 $0$(反向计算)}
        \IF{$t = T$}
            \STATE $G_T = r_{T+1}$
        \ELSE
            \STATE $G_t = r_{t+1} + \gamma G_{t+1}$
        \ENDIF
    \ENDFOR
    \FOR{$t = 0$ 到 $T$}
        \STATE 计算梯度:$g_t = \nabla_\theta \log \pi_\theta(a_t|s_t)$
        \STATE 更新参数:$\theta \leftarrow \theta + \alpha g_t G_t$
    \ENDFOR
\ENDFOR
\end{algorithmic}
\end{algorithm}

\subsection{REINFORCE的改进:引入基线}

REINFORCE算法的另一个问题是高方差。即使使用因果关系修正,回报$G_t$的方差仍然很大。为了降低方差,我们引入基线(baseline)$b(s_t)$:

\[
\nabla_\theta J(\theta) = \mathbb{E}_{\tau \sim p_\theta(\tau)} \left[ \sum_{t=0}^{T} \nabla_\theta \log \pi_\theta(a_t|s_t) (G_t - b(s_t)) \right]
\]

基线$b(s_t)$可以是任何不依赖于动作$a_t$的函数。常用的基线是状态值函数$V(s_t)$的估计。可以证明,只要基线不依赖于动作,引入基线不会改变梯度的期望值:

\begin{proof}
\begin{align*}
\mathbb{E}_{\tau \sim p_\theta(\tau)} \left[ \sum_{t=0}^{T} \nabla_\theta \log \pi_\theta(a_t|s_t) b(s_t) \right]
&= \sum_{t=0}^{T} \mathbb{E}_{s_t} \left[ \mathbb{E}_{a_t \sim \pi_\theta(\cdot|s_t)} \left[ \nabla_\theta \log \pi_\theta(a_t|s_t) b(s_t) \right] \right] \\
&= \sum_{t=0}^{T} \mathbb{E}_{s_t} \left[ b(s_t) \mathbb{E}_{a_t \sim \pi_\theta(\cdot|s_t)} \left[ \nabla_\theta \log \pi_\theta(a_t|s_t) \right] \right] \\
&= \sum_{t=0}^{T} \mathbb{E}_{s_t} \left[ b(s_t) \cdot 0 \right] = 0
\end{align*}
\end{proof}

最简单的基线是平均回报:$b = \frac{1}{N} \sum_{i=1}^{N} R(\tau^i)$。改进后的REINFORCE算法:

\begin{algorithm}[H]
\caption{REINFORCE with Baseline}
\begin{algorithmic}[1]
\REQUIRE 策略参数$\theta$,学习率$\alpha$,折扣因子$\gamma$
\STATE 初始化策略参数$\theta$
\FOR{每个回合(episode)}
    \STATE 使用当前策略$\pi_\theta$生成一条轨迹,计算$G_t$
    \STATE 计算基线$b$(可以使用多个轨迹的平均回报)
    \FOR{$t = 0$ 到 $T$}
        \STATE 计算优势:$A_t = G_t - b$
        \STATE 计算梯度:$g_t = \nabla_\theta \log \pi_\theta(a_t|s_t)$
        \STATE 更新参数:$\theta \leftarrow \theta + \alpha g_t A_t$
    \ENDFOR
\ENDFOR
\end{algorithmic}
\end{algorithm}

\subsection{REINFORCE的优缺点}

\begin{table}[H]
\centering
\caption{REINFORCE算法的优缺点}
\begin{tabular}{p{0.45\textwidth}p{0.45\textwidth}}
\toprule
\textbf{优点} & \textbf{缺点} \\
\midrule
简单直观,易于实现 & 高方差,训练不稳定 \\
可以处理连续动作空间 & 样本效率低(需要完整轨迹) \\
可以学习随机策略 & 更新频率低(每回合更新一次) \\
理论上有收敛保证 & 对超参数敏感 \\
探索充分(通过随机策略) & 可能收敛到局部最优 \\
\bottomrule
\end{tabular}
\end{table}

\section{Actor-Critic方法:结合价值函数}

\subsection{Actor-Critic基本思想}

REINFORCE算法使用蒙特卡洛方法估计回报$G_t$,虽然无偏但方差大。Actor-Critic方法引入价值函数(Critic)来估计优势函数,从而降低方差。

\begin{figure}[H]
\centering
\includegraphics[width=0.8\textwidth]{picture/4.png}
\caption{Actor-Critic框架示意图}
\end{figure}

Actor-Critic方法包含两个组件:
\begin{itemize}
    \item \textbf{Actor(演员)}:策略网络$\pi_\theta(a|s)$,负责选择动作
    \item \textbf{Critic(评论家)}:价值网络$V_\phi(s)$,负责评估状态的价值
\end{itemize}

\subsection{优势函数(Advantage Function)}

在Actor-Critic中,我们使用优势函数$A(s_t, a_t)$替代回报$G_t$:

\[
A(s_t, a_t) = Q(s_t, a_t) - V(s_t)
\]

其中$Q(s_t, a_t)$是动作价值函数,$V(s_t)$是状态价值函数。优势函数表示在状态$s_t$下执行动作$a_t$相对于平均水平的优势。

在实际中,我们使用时序差分(TD)误差来估计优势函数:

\[
A(s_t, a_t) \approx r_{t+1} + \gamma V(s_{t+1}) - V(s_t)
\]

\subsection{Actor-Critic算法}

\begin{algorithm}[H]
\caption{Actor-Critic算法}
\begin{algorithmic}[1]
\REQUIRE 策略参数$\theta$,价值参数$\phi$,学习率$\alpha_\theta, \alpha_\phi$,折扣因子$\gamma$
\STATE 初始化策略参数$\theta$和价值参数$\phi$
\FOR{每个时间步}
    \STATE 在状态$s_t$,根据策略$\pi_\theta(\cdot|s_t)$选择动作$a_t$
    \STATE 执行动作$a_t$,获得奖励$r_{t+1}$,转移到新状态$s_{t+1}$
    \STATE 计算TD误差:$\delta_t = r_{t+1} + \gamma V_\phi(s_{t+1}) - V_\phi(s_t)$
    \STATE 更新Critic:$\phi \leftarrow \phi + \alpha_\phi \delta_t \nabla_\phi V_\phi(s_t)$
    \STATE 更新Actor:$\theta \leftarrow \theta + \alpha_\theta \nabla_\theta \log \pi_\theta(a_t|s_t) \delta_t$
\ENDFOR
\end{algorithmic}
\end{algorithm}

\subsection{Actor-Critic的优缺点}

\begin{table}[H]
\centering
\caption{Actor-Critic算法的优缺点}
\begin{tabular}{p{0.45\textwidth}p{0.45\textwidth}}
\toprule
\textbf{优点} & \textbf{缺点} \\
\midrule
方差低,训练稳定 & 有偏差(依赖价值函数估计) \\
样本效率高(单步更新) & 需要同时训练两个网络 \\
可以处理连续任务 & 可能不稳定(两个网络相互影响) \\
更新频率高(每一步更新) & 超参数更多(两个学习率) \\
\bottomrule
\end{tabular}
\end{table}

\section{A2C和A3C:并行化Actor-Critic}

\subsection{优势Actor-Critic(A2C)}

A2C(Advantage Actor-Critic)是Actor-Critic的改进版本,主要改进包括:

\begin{enumerate}
    \item \textbf{明确使用优势函数}:使用$A(s_t, a_t) = r_{t+1} + \gamma V(s_{t+1}) - V(s_t)$
    \item \textbf{多步回报}:使用n步回报来估计优势函数
    \item \textbf{并行训练}:多个环境并行收集数据
\end{enumerate}

\begin{figure}[H]
\centering
\includegraphics[width=0.25\textwidth]{picture/A2C.png}
\caption{A2C架构示意图:多个Worker并行收集数据,同步更新}
\end{figure}

\subsection{A2C算法}

\begin{algorithm}[H]
\caption{A2C算法}
\begin{algorithmic}[1]
\REQUIRE 全局策略参数$\theta$,全局价值参数$\phi$,学习率$\alpha_\theta, \alpha_\phi$,折扣因子$\gamma$,Worker数量$N$
\STATE 初始化全局参数$\theta, \phi$
\REPEAT
    \STATE 重置所有Worker的梯度:$d\theta \leftarrow 0$, $d\phi \leftarrow 0$
    \STATE 同步所有Worker参数:$\theta_i \leftarrow \theta$, $\phi_i \leftarrow \phi$
    \FOR{每个Worker $i = 1$ 到 $N$}
        \STATE Worker $i$运行策略$\pi_{\theta_i}$收集轨迹数据
        \STATE 计算优势估计$A_t$(可以使用n步回报或GAE)
        \STATE 计算策略梯度:$d\theta_i = \sum_t \nabla_{\theta_i} \log \pi_{\theta_i}(a_t|s_t) A_t$
        \STATE 计算价值梯度:$d\phi_i = \sum_t \nabla_{\phi_i} (V_{\phi_i}(s_t) - R_t)^2$
    \ENDFOR
    \STATE 汇总梯度:$d\theta = \frac{1}{N} \sum_i d\theta_i$, $d\phi = \frac{1}{N} \sum_i d\phi_i$
    \STATE 更新全局参数:$\theta \leftarrow \theta + \alpha_\theta d\theta$, $\phi \leftarrow \phi + \alpha_\phi d\phi$
\UNTIL{收敛}
\end{algorithmic}
\end{algorithm}

\subsection{异步优势Actor-Critic(A3C)}

A3C(Asynchronous Advantage Actor-Critic)是A2C的异步版本。在A3C中,每个Worker异步地更新全局网络参数,不需要等待其他Worker。

\begin{figure}[H]
\centering
\includegraphics[width=0.6\textwidth]{picture/A3C.png}
\caption{A3C架构示意图:多个Worker异步更新全局网络}
\end{figure}

A3C的关键特点:
\begin{itemize}
    \item \textbf{异步更新}:Worker独立运行,不需要同步
    \item \textbf{探索多样性}:不同Worker可能处于不同状态,增加探索
    \item \textbf{硬件效率}:充分利用多核CPU
\end{itemize}

\subsection{A2C vs A3C比较}

\begin{table}[H]
\centering
\caption{A2C与A3C比较}
\begin{tabular}{p{0.3\textwidth}p{0.3\textwidth}p{0.3\textwidth}}
\toprule
\textbf{特性} & \textbf{A2C} & \textbf{A3C} \\
\midrule
\textbf{更新方式} & 同步 & 异步 \\
\textbf{通信开销} & 高(需要同步) & 低(异步) \\
\textbf{实现复杂度} & 较低 & 较高 \\
\textbf{收敛速度} & 稳定但可能较慢 & 可能更快但波动大 \\
\textbf{硬件利用} & 需要同步,效率较低 & 充分利用多核,效率高 \\
\textbf{探索多样性} & 一般 & 好(不同Worker不同状态) \\
\bottomrule
\end{tabular}
\end{table}

\section{TRPO:置信域策略优化}

\subsection{TRPO的基本思想}

TRPO(Trust Region Policy Optimization)的核心思想是:在策略更新时,限制新策略与旧策略的差异,确保每次更新都在一个"可信赖"的区域内,从而保证策略性能单调提升。

\begin{figure}[H]
\centering
\includegraphics[width=0.7\textwidth]{picture/5.png}
\caption{TRPO的置信域优化:限制策略更新幅度}
\end{figure}

\subsection{TRPO的数学形式}

TRPO将策略优化问题形式化为带约束的优化问题:

\begin{align*}
\max_{\theta} & \quad \mathbb{E}_{s \sim \rho_{\theta_{\text{old}}}, a \sim \pi_{\theta_{\text{old}}}} \left[ \frac{\pi_\theta(a|s)}{\pi_{\theta_{\text{old}}}(a|s)} A_{\theta_{\text{old}}}(s, a) \right] \\
\text{s.t.} & \quad \mathbb{E}_{s \sim \rho_{\theta_{\text{old}}}} \left[ D_{\text{KL}}(\pi_{\theta_{\text{old}}}(\cdot|s) \| \pi_\theta(\cdot|s)) \right] \leq \delta
\end{align*}

其中:
\begin{itemize}
    \item $\rho_{\theta_{\text{old}}}$是旧策略下的状态分布
    \item $D_{\text{KL}}$是KL散度,度量两个策略的差异
    \item $\delta$是置信域半径,限制策略更新的幅度
\end{itemize}

\subsection{重要性采样(Importance Sampling)}

TRPO使用重要性采样来利用旧策略收集的数据评估新策略:

\[
\mathbb{E}_{a \sim \pi_\theta(\cdot|s)} [A(s,a)] = \mathbb{E}_{a \sim \pi_{\theta_{\text{old}}}(\cdot|s)} \left[ \frac{\pi_\theta(a|s)}{\pi_{\theta_{\text{old}}}(a|s)} A(s,a) \right]
\]

重要性采样比率$\frac{\pi_\theta(a|s)}{\pi_{\theta_{\text{old}}}(a|s)}$衡量了新策略与旧策略在动作选择上的差异。

\subsection{代理目标函数(Surrogate Objective)}

TRPO优化的是代理目标函数:

\[
L(\theta) = \mathbb{E}_{s \sim \rho_{\theta_{\text{old}}}, a \sim \pi_{\theta_{\text{old}}}} \left[ \frac{\pi_\theta(a|s)}{\pi_{\theta_{\text{old}}}(a|s)} A_{\theta_{\text{old}}}(s,a) \right]
\]

这个代理目标函数是原始目标函数$J(\theta)$的一阶近似。

\subsection{TRPO的求解}

TRPO的约束优化问题通过以下步骤求解:

\begin{enumerate}
    \item 线性化目标函数:$L(\theta) \approx g^T (\theta - \theta_{\text{old}})$
    \item 二次化约束:$\frac{1}{2} (\theta - \theta_{\text{old}})^T H (\theta - \theta_{\text{old}}) \leq \delta$
    \item 求解约束优化问题:$\theta^* = \theta_{\text{old}} + \sqrt{\frac{2\delta}{g^T H^{-1} g}} H^{-1} g$
\end{enumerate}

其中$g = \nabla_\theta L(\theta)|_{\theta=\theta_{\text{old}}}$是梯度,$H$是KL散度的Hessian矩阵(Fisher信息矩阵)。

在实际实现中,使用共轭梯度法计算$H^{-1} g$,避免直接计算和存储$H^{-1}$。

\subsection{TRPO算法}

\begin{algorithm}[H]
\caption{TRPO算法}
\begin{algorithmic}[1]
\REQUIRE 初始策略参数$\theta_0$,置信域半径$\delta$,回溯系数$\alpha$
\FOR{$k = 0, 1, 2, \dots$}
    \STATE 使用策略$\pi_{\theta_k}$收集轨迹数据
    \STATE 估计优势函数$A_{\theta_k}(s,a)$
    \STATE 计算代理目标$L(\theta)$和其梯度$g$
    \STATE 使用共轭梯度法计算更新方向$v \approx H^{-1} g$
    \STATE 计算步长$\beta = \sqrt{\frac{2\delta}{v^T H v}}$
    \STATE 候选参数:$\theta_{\text{candidate}} = \theta_k + \beta v$
    \STATE 回溯线性搜索:找到最大的$\alpha^j$使得$L(\theta_k + \alpha^j \beta v) \geq L(\theta_k)$且满足KL约束
    \STATE 更新参数:$\theta_{k+1} = \theta_k + \alpha^j \beta v$
\ENDFOR
\end{algorithmic}
\end{algorithm}

\begin{figure}[H]
\centering
\includegraphics[width=0.5\textwidth]{picture/trpo.png}
\caption{TRPO的置信域优化过程}
\end{figure}

\subsection{TRPO的优缺点}

\begin{table}[H]
\centering
\caption{TRPO算法的优缺点}
\begin{tabular}{p{0.25\textwidth}p{0.45\textwidth}}
\toprule
\textbf{优点} & \textbf{缺点} \\
\midrule
理论保证单调提升 & 计算复杂,需要二阶优化 \\
更新稳定,避免策略崩溃 & 实现难度大 \\
适用于复杂任务 & 与某些网络结构不兼容(如dropout) \\
样本效率较高 & 超参数敏感(置信域半径$\delta$) \\
\bottomrule
\end{tabular}
\end{table}

\section{PPO:近端策略优化}

\subsection{PPO的基本思想}

PPO(Proximal Policy Optimization)是TRPO的简化版本,通过修改目标函数来实现策略更新的约束,避免了TRPO中复杂的二阶优化。

\begin{figure}[H]
\centering
\includegraphics[width=0.3\textwidth]{picture/6.png}
\caption{PPO的裁剪机制:限制重要性采样比率}
\end{figure}

\subsection{PPO-Clip目标函数}

PPO-Clip通过裁剪重要性采样比率来限制策略更新:

\[
L^{\text{CLIP}}(\theta) = \mathbb{E}_{t} \left[ \min\left( r_t(\theta) A_t, \text{clip}(r_t(\theta), 1-\epsilon, 1+\epsilon) A_t \right) \right]
\]

其中:
\begin{itemize}
    \item $r_t(\theta) = \frac{\pi_\theta(a_t|s_t)}{\pi_{\theta_{\text{old}}}(a_t|s_t)}$是重要性采样比率
    \item $A_t$是优势函数估计
    \item $\epsilon$是裁剪参数,通常取0.1或0.2
    \item $\text{clip}(x, a, b)$将$x$限制在$[a, b]$区间内
\end{itemize}

\subsection{PPO-Clip的直观理解}

\begin{itemize}
    \item 当$A_t > 0$时,动作$a_t$优于平均水平,我们希望增加其概率
    \begin{itemize}
        \item 如果$r_t(\theta) < 1+\epsilon$,使用$r_t(\theta)A_t$
        \item 如果$r_t(\theta) \geq 1+\epsilon$,使用$(1+\epsilon)A_t$,限制更新幅度
    \end{itemize}
    \item 当$A_t < 0$时,动作$a_t$劣于平均水平,我们希望减少其概率
    \begin{itemize}
        \item 如果$r_t(\theta) > 1-\epsilon$,使用$r_t(\theta)A_t$
        \item 如果$r_t(\theta) \leq 1-\epsilon$,使用$(1-\epsilon)A_t$,限制更新幅度
    \end{itemize}
\end{itemize}

\begin{figure}[H]
\centering
\includegraphics[width=0.3\textwidth]{picture/7.png}
\caption{PPO-Clip目标函数的两种情况}
\end{figure}

\subsection{PPO-Penalty目标函数}

PPO-Penalty通过在目标函数中添加KL散度惩罚项来约束策略更新:

\[
L^{\text{KL}}(\theta) = \mathbb{E}_{t} \left[ r_t(\theta) A_t - \beta D_{\text{KL}}(\pi_{\theta_{\text{old}}}(\cdot|s_t) \| \pi_\theta(\cdot|s_t)) \right]
\]

其中$\beta$是自适应参数,根据KL散度的大小调整。

\subsection{广义优势估计(GAE)}

GAE(Generalized Advantage Estimation)是一种权衡偏差和方差的优势估计方法:

\[
\hat{A}_t^{\text{GAE}(\gamma, \lambda)} = \sum_{l=0}^{\infty} (\gamma \lambda)^l \delta_{t+l}
\]

其中$\delta_t = r_t + \gamma V(s_{t+1}) - V(s_t)$是TD误差。

GAE的参数$\lambda$控制偏差-方差权衡:
\begin{itemize}
    \item $\lambda = 0$:$\hat{A}_t = \delta_t$,方差小但偏差大
    \item $\lambda = 1$:$\hat{A}_t = \sum_{l=0}^{\infty} \gamma^l \delta_{t+l}$,偏差小但方差大
    \item 通常取$\lambda = 0.95$或$0.97$,平衡偏差和方差
\end{itemize}

\subsection{PPO算法}

\begin{algorithm}[H]
\caption{PPO-Clip算法}
\begin{algorithmic}[1]
\REQUIRE 初始策略参数$\theta_0$,初始价值参数$\phi_0$,裁剪参数$\epsilon$,学习率$\alpha$
\FOR{$k = 0, 1, 2, \dots$}
    \STATE 使用策略$\pi_{\theta_k}$收集轨迹数据$\mathcal{D}_k$
    \STATE 计算优势估计$\hat{A}_t$(使用GAE)
    \STATE 计算目标价值$\hat{V}_t = \hat{A}_t + V_{\phi_k}(s_t)$
    \STATE 优化PPO-Clip目标函数(多个epoch):
    \[
    \theta_{k+1} = \arg\max_{\theta} \frac{1}{|\mathcal{D}_k|T} \sum_{\tau \in \mathcal{D}_k} \sum_{t=0}^{T} \min\left( r_t(\theta) \hat{A}_t, \text{clip}(r_t(\theta), 1-\epsilon, 1+\epsilon) \hat{A}_t \right)
    \]
    \STATE 优化价值函数(多个epoch):
    \[
    \phi_{k+1} = \arg\min_{\phi} \frac{1}{|\mathcal{D}_k|T} \sum_{\tau \in \mathcal{D}_k} \sum_{t=0}^{T} \left( V_\phi(s_t) - \hat{V}_t \right)^2
    \]
\ENDFOR
\end{algorithmic}
\end{algorithm}

\subsection{联合损失函数}

在实际实现中,PPO通常使用联合损失函数,同时优化策略和价值函数:

\[
L(\theta, \phi) = L^{\text{CLIP}}(\theta) - c_1 L^{\text{VF}}(\phi) + c_2 S[\pi_\theta](s_t)
\]

其中:
\begin{itemize}
    \item $L^{\text{CLIP}}(\theta)$是PPO-Clip策略损失
    \item $L^{\text{VF}}(\phi) = (V_\phi(s_t) - V_t^{\text{target}})^2$是价值函数损失
    \item $S[\pi_\theta](s_t)$是策略熵,鼓励探索
    \item $c_1, c_2$是系数,平衡不同项
\end{itemize}

\subsection{PPO的优缺点}

\begin{table}[H]
\centering
\caption{PPO算法的优缺点}
\begin{tabular}{p{0.45\textwidth}p{0.45\textwidth}}
\toprule
\textbf{优点} & \textbf{缺点} \\
\midrule
实现简单,计算效率高 & 理论保证较弱 \\
性能稳定,适用于多种任务 & 超参数较多($\epsilon, c_1, c_2, \lambda$等) \\
样本效率高 & 对超参数敏感 \\
与多种网络结构兼容 & 可能收敛到局部最优 \\
支持连续和离散动作空间 & 需要调整学习率调度 \\
\bottomrule
\end{tabular}
\end{table}

\section{策略学习方法的比较与应用}

\subsection{方法对比}

\begin{table}[H]
\centering
\caption{策略学习方法对比}
\begin{tabular}{p{0.15\textwidth}p{0.2\textwidth}p{0.2\textwidth}p{0.2\textwidth}p{0.2\textwidth}}
\toprule
\textbf{方法} & \textbf{核心思想} & \textbf{优点} & \textbf{缺点} & \textbf{适用场景} \\
\midrule
REINFORCE & 蒙特卡洛策略梯度 & 简单,无偏 & 高方差,低效 & 简单任务,探索算法 \\
Actor-Critic & 价值函数引导 & 方差低,高效 & 有偏差,不稳定 & 中等复杂度任务 \\
A2C/A3C & 并行化AC & 稳定,高效 & 实现复杂 & 需要并行化的任务 \\
TRPO & 置信域优化 & 理论保证,稳定 & 计算复杂 & 高精度控制任务 \\
PPO & 裁剪目标函数 & 简单高效,稳定 & 超参数多 & 通用强化学习任务 \\
\bottomrule
\end{tabular}
\end{table}

\subsection{选择指南}

\begin{enumerate}
    \item \textbf{简单任务}:可以从REINFORCE或Actor-Critic开始
    \item \textbf{中等复杂度任务}:推荐使用A2C或PPO
    \item \textbf{复杂控制任务}:考虑TRPO或PPO
    \item \textbf{需要高样本效率}:使用PPO with GAE
    \item \textbf{计算资源有限}:使用PPO(实现简单)
    \item \textbf{需要理论保证}:使用TRPO
\end{enumerate}

\subsection{实际应用建议}

\begin{itemize}
    \item \textbf{网络架构}:
    \begin{itemize}
        \item 使用共享主干网络提取特征
        \item 策略头输出动作分布参数
        \item 价值头输出状态价值
        \item 适当使用归一化技术
    \end{itemize}
    \item \textbf{超参数调优}:
    \begin{itemize}
        \item 学习率:通常$10^{-4}$到$10^{-3}$
        \item 折扣因子$\gamma$:0.99(长期任务)或0.95(短期任务)
        \item PPO裁剪参数$\epsilon$:0.1到0.3
        \item GAE参数$\lambda$:0.9到0.99
        \item 批量大小:128到2048
    \end{itemize}
    \item \textbf{训练技巧}:
    \begin{itemize}
        \item 使用多个环境并行收集数据
        \item 适当增加熵奖励鼓励探索
        \item 使用学习率衰减
        \item 监控KL散度避免策略崩溃
    \end{itemize}
\end{itemize}

\section{总结与展望}

\subsection{策略学习的发展历程}

策略学习方法经历了从简单到复杂的发展过程:

\begin{itemize}
    \item \textbf{早期}:REINFORCE算法提出策略梯度基本思想
    \item \textbf{发展}:Actor-Critic方法引入价值函数降低方差
    \item \textbf{成熟}:TRPO提供理论保证的稳定优化
    \item \textbf{普及}:PPO简化实现,成为实际应用标准
    \item \textbf{前沿}:分布策略梯度,元策略学习等
\end{itemize}

\subsection{关键技术创新}

\begin{enumerate}
    \item \textbf{重要性采样}:允许重用旧数据,提高样本效率
    \item \textbf{优势函数}:降低梯度方差,提高训练稳定性
    \item \textbf{置信域方法}:限制策略更新幅度,避免策略崩溃
    \item \textbf{裁剪技术}:简化置信域约束,提高计算效率
    \item \textbf{并行化}:提高数据收集效率,加速训练
    \item \textbf{熵奖励}:鼓励探索,防止过早收敛
\end{enumerate}

\subsection{未来发展方向}

\begin{enumerate}
    \item \textbf{样本效率提升}:结合模型学习,元学习
    \item \textbf{探索策略改进}:好奇心驱动,内在动机
    \item \textbf{多任务学习}:共享表示,迁移学习
    \item \textbf{分层策略}:技能学习,选项发现
    \item \textbf{安全约束}:约束策略优化,安全探索
    \item \textbf{分布式训练}:大规模并行,分布式优化
\end{enumerate}

\subsection{学习建议}

\begin{enumerate}
    \item \textbf{理论基础}:深入理解策略梯度定理,重要性采样
    \item \textbf{实践能力}:实现基本算法,调试超参数
    \item \textbf{工具掌握}:熟练使用RL框架(如Stable Baselines3)
    \item \textbf{代码阅读}:学习开源实现,理解细节
    \item \textbf{论文跟踪}:关注最新进展,学习新方法
\end{enumerate}

策略学习作为强化学习的核心方法之一,已经在游戏、机器人控制、自然语言处理等领域取得了显著成功。理解策略学习的原理和方法,掌握PPO等现代算法,对于解决实际强化学习问题具有重要意义。

