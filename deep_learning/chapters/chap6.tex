\chapter{进化计算之起源与遗传算法}
\section{引言:两种哲学思想的碰撞}

\subsection{理性主义与经验主义的对立}

在哲学史上,关于人类如何获取知识,存在两种根本对立的观点:理性主义(Rationalism)和经验主义(Empiricism)。

\begin{table}[H]
\centering
\caption{理性主义与经验主义的对比}
\begin{tabular}{p{0.45\textwidth}p{0.45\textwidth}}
\toprule
\textbf{理性主义(Rationalism)} & \textbf{经验主义(Empiricism)} \\
\midrule
\textbf{代表人物}:笛卡尔、斯宾诺莎、莱布尼茨 & \textbf{代表人物}:培根、洛克、牛顿 \\
\hline
\textbf{核心思想}:真正的知识来源于理性和逻辑推理,而非感官经验。存在不证自明的"公理"或"天赋观念"。 & \textbf{核心思想}:知识来源于感官经验。通过观察、实验、归纳总结形成知识。 \\
\hline
\textbf{方法论}:从公理出发,通过严密的演绎推理构建知识体系。 & \textbf{方法论}:通过观察现象,归纳总结出一般规律。 \\
\hline
\textbf{优点}:逻辑严谨,体系严密。 & \textbf{优点}:能够产生新知识,适用范围广,是现代科学的基石。 \\
\hline
\textbf{缺点}:如果公理错误,整个体系崩溃;适用范围有限。 & \textbf{缺点}:归纳法的结论不保证永远正确。 \\
\hline
\textbf{经典名言}:我思故我在(笛卡尔) & \textbf{经典名言}:知识就是力量(培根) \\
\bottomrule
\end{tabular}
\end{table}

\subsection{数学优化与进化计算的哲学对应}

在优化理论中,我们同样可以看到这两种哲学思想的体现:

\begin{table}[H]
\centering
\caption{数学优化与进化计算的哲学对应}
\begin{tabular}{p{0.45\textwidth}p{0.45\textwidth}}
\toprule
\textbf{数学优化(对应理性主义)} & \textbf{进化计算(对应经验主义)} \\
\midrule
依赖于精确的数学模型(类似于"公理") & 不依赖于精确的数学模型 \\
\hline
使用严格的数学推导和证明(演绎法) & 模仿自然界的"物竞天择,适者生存"(归纳法) \\
\hline
通过梯度下降、牛顿法等一步步找到最优解 & 通过"选择、交叉、变异"等随机操作,不断尝试和迭代 \\
\hline
\textbf{优点}:一旦模型建立,求解过程严谨,通常能保证找到最优解或近似最优解。 & \textbf{优点}:几乎可以应用于任何可以定义"好坏"(适应度)的优化问题,适用性极广。 \\
\hline
\textbf{缺点}:对问题要求高,很多现实问题难以建立精确的数学模型。 & \textbf{缺点}:无法从数学上证明一定能找到全局最优解,性能边界难以严格保证。 \\
\bottomrule
\end{tabular}
\end{table}

\subsection{现实世界优化问题的复杂性}

现实世界中,许多优化问题具有多个特性,使得传统数学优化方法难以应用:

\begin{enumerate}
    \item \textbf{非线性(Nonlinear)}:目标函数或约束条件不是线性的
    \item \textbf{不可导/导数不连续(Non-differentiable/Discontinuous Gradient)}:目标函数在某些点不可导或导数不连续
    \item \textbf{高维度(High dimensionality)}:决策变量数量很多
    \item \textbf{非凸、大量的局部最优解(Non-convex, Numerous local optima)}:存在多个局部最优解
    \item \textbf{黑箱优化(Black-Box Optimization, BBO)}:目标函数是黑箱,只能通过输入得到输出,无法获取内部信息
\end{enumerate}

\subsubsection{黑箱优化问题举例}

\begin{itemize}
    \item \textbf{交通流优化}:目标是最小化平均通勤时间。输入是交通信号灯时序方案。黑箱是通过SUMO等软件运行一次交通流仿真。
    \item \textbf{计算流体力学(CFD)}:目标是找到阻力最小的飞机翼型。输入是翼型的几何参数。黑箱是运行一次昂贵的CFD仿真。
    \item \textbf{药物/材料发现}:目标是找到性能最好的分子/合金。输入是化学结构/配方。黑箱是运行分子动力学模拟或真实的物理实验。
\end{itemize}

\begin{figure}[H]
\centering
\includegraphics[width=0.3\textwidth]{picture/bbo.png}
\caption{黑箱优化问题示意图:只能通过输入得到输出,无法了解内部结构}
\end{figure}

\section{遗传算法的提出}

\subsection{自然进化的启示}

世界上最好的解决问题的工具是什么?

\begin{enumerate}
    \item \textbf{人类大脑}:创造了车轮、纽约和战争...(Douglas Adams)→ 神经计算(Neural Computation)
    \item \textbf{进化机制}:创造了人类大脑...(Charles Robert Darwin)→ 进化计算(Evolutionary Computation)
\end{enumerate}

遗传算法(Genetic Algorithm, GA)是由美国密歇根大学的John Holland教授在20世纪60年代末到70年代初提出的一种概率性群体搜索算法。它以孟德尔的遗传学说和达尔文的生物进化论为基础,通过模拟生物的进化过程来求解优化问题。

\begin{figure}[H]
\centering
\includegraphics[width=0.3\textwidth]{picture/ga.png}
\caption{生物进化过程与遗传算法的对应关系}
\end{figure}

\subsection{生物进化与遗传算法的类比}

\begin{table}[H]
\centering
\caption{生物进化过程与遗传算法的对应关系}
\begin{tabular}{p{0.3\textwidth}p{0.3\textwidth}p{0.3\textwidth}}
\toprule
\textbf{生物学术语} & \textbf{遗传算法术语} & \textbf{解释} \\
\midrule
基因(gene) & 决策变量/解的元素 & 问题的基本构成单元 \\
\hline
个体/染色体(individual/chromosome) & 候选解 & 待求解问题的一个可能解 \\
\hline
群体(population) & 候选解集合 & 一个候选解的有限集合 \\
\hline
适应度(fitness) & 适应值/质量评估值 & 候选解的质量评估值 \\
\hline
适者生存(survival of the fittest) & 选择操作 & 保留较好解,淘汰较差解 \\
\hline
交叉(crossover) & 交叉操作 & 两个个体交换基因信息产生新个体 \\
\hline
变异(mutation) & 变异操作 & 个体基因发生随机变化 \\
\hline
进化(evolution) & 迭代优化 & 种群逐代改进,逼近最优解 \\
\bottomrule
\end{tabular}
\end{table}

\section{遗传算法的总体流程}

\begin{figure}[H]
\centering
\includegraphics[width=0.9\textwidth]{picture/flow.png}
\caption{遗传算法的总体流程}
\end{figure}

\begin{algorithm}[H]
\caption{遗传算法基本流程}
\begin{algorithmic}[1]
\REQUIRE 种群大小$N$,交叉概率$P_c$,变异概率$P_m$,最大迭代次数$G_{\max}$
\ENSURE 至今最优解
\STATE 随机生成初始种群$P_0$,大小为$N$
\STATE 评估$P_0$中每个个体的适应度
\STATE 记录至今最优解
\FOR{$g = 1$ to $G_{\max}$}
    \STATE 选择操作:从当前种群中选择$N$个个体形成交配池
    \STATE 交叉操作:以概率$P_c$对交配池中的个体进行交叉,产生新个体
    \STATE 变异操作:以概率$P_m$对新个体进行变异
    \STATE 评估新个体的适应度
    \STATE 执行精英策略:用至今最优解替换当前最差解
    \STATE 更新至今最优解
\ENDFOR
\RETURN 至今最优解
\end{algorithmic}
\end{algorithm}

\section{遗传算法的关键组成部分}

\subsection{编码策略(Encoding Strategy)}

编码策略是指如何将问题的一个解映射为遗传算法中的个体(染色体)。编码策略的设计强调全面性和唯一性,与算法的性能密切相关。

\begin{table}[H]
\centering
\caption{常见的编码策略}
\begin{tabular}{p{0.25\textwidth}p{0.3\textwidth}p{0.35\textwidth}}
\toprule
\textbf{编码类型} & \textbf{表示方法} & \textbf{适用问题} \\
\midrule
二进制编码 & 每个决策变量用二进制数表示 & 离散优化问题,特别是0-1规划问题 \\
\hline
整数编码 & 每个决策变量用整数表示 & 离散优化问题,如调度问题 \\
\hline
实数编码 & 每个决策变量用实数表示 & 连续优化问题,如函数优化 \\
\hline
排列编码 & 将有限集合内的元素进行排列 & 组合优化问题,如TSP、调度问题 \\
\bottomrule
\end{tabular}
\end{table}

\subsubsection{编码策略的选择原则}
\begin{enumerate}
    \item \textbf{合法性}:编码应能表示所有可能的合法解
    \item \textbf{唯一性}:每个解应有唯一的编码表示
    \item \textbf{完备性}:编码应包含问题的所有信息
    \item \textbf{简洁性}:编码应尽量简洁,减少冗余
    \item \textbf{可操作性}:编码应便于遗传操作(选择、交叉、变异)
\end{enumerate}

\subsection{适应度函数(Fitness Function)}

适应度函数用于评价个体所对应解的优劣,是遗传算法选择的依据。

\subsubsection{适应度函数的设计原则}
\begin{enumerate}
    \item 能直观反映解的质量
    \item 能体现解质量的差异
    \item 与遗传算子的特性相匹配
\end{enumerate}

\subsubsection{约束处理}
对于约束优化问题,常用的方法是使用带惩罚项的适应度函数:
\[
f(\mathbf{x}) = \text{fitness}(\mathbf{x}) + \lambda \cdot \text{penalty}(\mathbf{x})
\]
其中:
\begin{itemize}
    \item $\text{fitness}(\mathbf{x})$是原始适应度
    \item $\text{penalty}(\mathbf{x})$是惩罚项,反映解违反约束的程度
    \item $\lambda$是惩罚系数
\end{itemize}

惩罚函数的设计应保证:最好不合法解的适应度值差于最差合法解的适应度值。

\subsection{选择算子(Selection Operator)}

选择算子体现生物进化过程中的"优胜劣汰"原则,适应度好的个体有较大的机会被选中。

\subsubsection{轮盘赌选择(Roulette Wheel Selection)}

轮盘赌选择是最经典的选择算子。每个个体的选择概率与其适应度成正比:

\[
P_i = \frac{f(\mathbf{x}_i)}{\sum_{j=1}^{N} f(\mathbf{x}_j)}
\]

其中$f(\mathbf{x}_i)$是个体$\mathbf{x}_i$的适应度。

\textbf{问题}:
\begin{enumerate}
    \item 对适应度绝对值敏感
    \item 出现"超级个体"会垄断种群
    \item 无法直接处理负适应度值
\end{enumerate}

\textbf{改进方案}:使用Softmax函数计算选择概率:
\[
P_i = \frac{\exp(f(\mathbf{x}_i)/T)}{\sum_{j=1}^{N} \exp(f(\mathbf{x}_j)/T)}
\]
其中$T$是温度参数,控制选择压力。

\textbf{优点}:
\begin{enumerate}
    \item 不受适应度尺度影响(平移不变性)
    \item 可调节选择压力(通过温度参数$T$)
    \item 天然兼容负适应度值
\end{enumerate}

\subsubsection{锦标赛选择(Tournament Selection)}

锦标赛选择是另一种常用的选择算子:

\begin{algorithm}[H]
\caption{锦标赛选择}
\begin{algorithmic}[1]
\REQUIRE 种群$P$,种群大小$N$,锦标赛规模$K$
\ENSURE 选择出的个体
\STATE 初始化空列表$selected$
\WHILE{$|selected| < N$}
    \STATE 从种群中随机选择$K$个个体
    \STATE 从这$K$个个体中选择适应度最高的个体
    \STATE 将该个体加入$selected$列表
\ENDWHILE
\RETURN $selected$
\end{algorithmic}
\end{algorithm}

\textbf{特点}:
\begin{enumerate}
    \item 选择概率与适应度的取值范围无关(只与排序相关)
    \item 需要设置参数$K$(锦标赛规模)
    \item 较差个体被选中的机会较小
    \item 好个体可能会在种群中重复出现
\end{enumerate}

\subsection{交叉算子(Crossover Operator)}

交叉算子将两个或多个个体重新组合,期望将优势基因"强强联合",产生适应度更高的新个体。

\subsubsection{单点交叉(One-Point Crossover)}

\begin{figure}[H]
\centering
\includegraphics[width=0.5\textwidth]{picture/onepoint.png}
\caption{单点交叉示意图:随机选择一个交叉点,交换两个父代个体交叉点后的部分}
\end{figure}

对于二进制编码,单点交叉的操作过程:
\begin{enumerate}
    \item 随机选择一个交叉点$k$($1 \leq k < L$,$L$为染色体长度)
    \item 父代1:$A = a_1 a_2 \dots a_k a_{k+1} \dots a_L$
    \item 父代2:$B = b_1 b_2 \dots b_k b_{k+1} \dots b_L$
    \item 子代1:$A' = a_1 a_2 \dots a_k b_{k+1} \dots b_L$
    \item 子代2:$B' = b_1 b_2 \dots b_k a_{k+1} \dots a_L$
\end{enumerate}

\subsubsection{两点交叉(Two-Point Crossover)}

\begin{figure}[H]
\centering
\includegraphics[width=0.5\textwidth]{picture/twopoint.png}
\caption{两点交叉示意图:随机选择两个交叉点,交换两个父代个体交叉点间的部分}
\end{figure}

对于二进制编码,两点交叉的操作过程:
\begin{enumerate}
    \item 随机选择两个交叉点$k_1, k_2$($1 \leq k_1 < k_2 < L$)
    \item 父代1:$A = a_1 \dots a_{k_1} a_{k_1+1} \dots a_{k_2} a_{k_2+1} \dots a_L$
    \item 父代2:$B = b_1 \dots b_{k_1} b_{k_1+1} \dots b_{k_2} b_{k_2+1} \dots b_L$
    \item 子代1:$A' = a_1 \dots a_{k_1} b_{k_1+1} \dots b_{k_2} a_{k_2+1} \dots a_L$
    \item 子代2:$B' = b_1 \dots b_{k_1} a_{k_1+1} \dots a_{k_2} b_{k_2+1} \dots b_L$
\end{enumerate}

\subsubsection{算术交叉(Arithmetic Crossover)}

对于实数编码,常用的交叉算子是算术交叉:
\[
\begin{aligned}
\text{子代1} &= \alpha \cdot \text{父代1} + (1-\alpha) \cdot \text{父代2} \\
\text{子代2} &= (1-\alpha) \cdot \text{父代1} + \alpha \cdot \text{父代2}
\end{aligned}
\]
其中$\alpha \in [0,1]$是随机数或固定值。

\subsection{变异算子(Mutation Operator)}

变异算子旨在为算法带来新的基因材料,增加种群的多样性,防止算法过早收敛。

\subsubsection{基本变异算子}

对于每个基因,以变异概率$P_m$决定是否进行变异:

\begin{algorithm}[H]
\caption{基本变异操作}
\begin{algorithmic}[1]
\FOR{每个个体$\mathbf{x} = (x_1, x_2, \dots, x_D)$}
    \FOR{每个基因$i = 1$ to $D$}
        \STATE 生成随机数$r \sim U(0,1)$
        \IF{$r < P_m$}
            \STATE 对$x_i$实施变异操作
        \ENDIF
    \ENDFOR
\ENDFOR
\end{algorithmic}
\end{algorithm}

\subsubsection{均匀变异(Uniform Mutation)}
\[
x_i' = \text{random}(l_i, u_i)
\]
其中$l_i$和$u_i$是基因$x_i$的下界和上界。

\subsubsection{高斯变异(Gaussian Mutation)}
\[
x_i' = x_i + N(0, \sigma^2)
\]
其中$N(0, \sigma^2)$是均值为0、方差为$\sigma^2$的正态分布随机数。

\subsection{精英策略(Elitist Strategy)}

精英策略旨在将优势基因保存在群体中,避免交叉、变异操作导致优势基因流失。

\begin{algorithm}[H]
\caption{精英策略}
\begin{algorithmic}[1]
\REQUIRE 当前种群$P_g$,至今最优解$\mathbf{x}_{\text{best}}$
\ENSURE 更新后的种群
\STATE 找到当前种群中的最差解$\mathbf{x}_{\text{worst}}$
\STATE 用至今最优解$\mathbf{x}_{\text{best}}$替换$\mathbf{x}_{\text{worst}}$
\RETURN 更新后的种群
\end{algorithmic}
\end{algorithm}

\section{遗传算法的参数设置}

\subsection{种群大小(Population Size, $N$)}

在固定的总计算预算(最大函数评估次数MaxFEs)下:
\[
\text{MaxFEs} = \text{种群大小}(N) \times \text{世代代数}(G)
\]

两种策略的权衡:

\begin{table}[H]
\centering
\caption{种群大小设置的两种策略}
\begin{tabular}{p{0.45\textwidth}p{0.45\textwidth}}
\toprule
\textbf{大种群策略($N \uparrow$, $G \downarrow$)} & \textbf{小种群策略($N \downarrow$, $G \uparrow$)} \\
\midrule
\textbf{思路}:广度优先,并行探索大量不同的解 & \textbf{思路}:深度优先,对少数解进行深度挖掘 \\
\hline
\textbf{优点}:多样性好,能覆盖更广的搜索空间,有效避免早熟 & \textbf{优点}:收敛速度快,能快速迭代优化 \\
\hline
\textbf{缺点}:进化不充分,可能来不及对优良个体进行精细优化 & \textbf{缺点}:多样性差,极易陷入局部最优且无法跳出 \\
\hline
\textbf{适用场景}:复杂、多峰问题,搜索空间大 & \textbf{适用场景}:相对简单、单峰问题,计算资源有限 \\
\bottomrule
\end{tabular}
\end{table}

\textbf{设置建议}:常用范围为30-200。对于复杂问题,通常需要更大的$N$来保证足够的探索。

\subsection{交叉概率(Crossover Probability, $P_c$)}

交叉概率控制着个体间进行基因重组的频率。

\begin{itemize}
    \item \textbf{$P_c$过高(如$>0.95$)}:可能频繁破坏已有的优良基因组合,导致解的质量不稳定
    \item \textbf{$P_c$过低(如$<0.5$)}:搜索停滞,产生新解的能力不足,探索效率低下
\end{itemize}

\textbf{设置建议}:通常设置一个较高的值,以保证算法有足够的探索能力。经验范围:0.5-0.95。

\subsection{变异概率(Mutation Probability, $P_m$)}

变异概率控制着个体发生随机变异的概率。

\begin{itemize}
    \item \textbf{$P_m$过高(如$>0.1$)}:算法退化为随机搜索,优良基因被持续破坏,难以收敛
    \item \textbf{$P_m$过低(如$<0.005$)}:无法有效引入新基因,一旦陷入局部最优便难以逃脱
\end{itemize}

\textbf{设置建议}:必须设置一个很低的值,它应是一个辅助性的小概率事件。经验范围:0.005-0.1。

一个常用的经验法则是$P_m$与染色体长度$D$成反比:$P_m \propto 1/D$。

\section{遗传算法的特点}

\begin{table}[H]
\centering
\caption{遗传算法的特点}
\begin{tabular}{p{0.3\textwidth}p{0.65\textwidth}}
\toprule
\textbf{特点} & \textbf{详细说明} \\
\midrule
\textbf{群体性随机搜索} & 从点集开始,以点集结束,通过群体进行搜索 \\
\hline
\textbf{灵活性强} & 只在设计适应度函数和处理约束时需要先验知识 \\
\hline
\textbf{长于全局搜索} & 以适应度为导向,但不排斥接纳较差的新解,有助于跳出局部最优 \\
\hline
\textbf{隐含并行性} & 群体中每个个体的操作相似,易于并行实现 \\
\hline
\textbf{局部搜索能力弱} & 精度较低,通常需要与局部搜索算子相结合 \\
\hline
\textbf{实时性差} & 需要多次迭代,收敛速度可能较慢 \\
\hline
\textbf{参数敏感} & 需要设置控制参数且与问题相关性较大 \\
\bottomrule
\end{tabular}
\end{table}

\section{应用1:函数优化(4D Rastrigin函数)}

\subsection{问题定义}

目标函数:4D Rastrigin函数
\[
f(x_1, x_2, x_3, x_4) = 40 + \sum_{i=1}^{4} [x_i^2 - 10 \cos(2\pi x_i)]
\]
约束条件:每个决策变量$x_i$的取值范围为$[-5.12, 5.12]$

适应度函数:由于是求最小值,适应度与目标函数值成反比
\[
\text{Fitness} = \frac{1}{f(\mathbf{x})}
\]

\subsection{实例演示}

\subsubsection{步骤1:初始化种群$P_0$及适应度评估}

随机生成一个大小为4的初始种群,并评估其适应度:

\begin{table}[H]
\centering
\caption{初始种群$P_0$}
\begin{tabular}{|c|c|c|c|c|}
\hline
编号 & 染色体$(x_1, x_2, x_3, x_4)$ & 目标函数值$f(\mathbf{x})$ & 适应度 \\
\hline
个体1 & $[4.11, -3.22, 2.55, -4.81]$ & 114.31 & 0.0087 \\
\hline
个体2 & $[0.10, -0.20, 0.15, -0.05]$ & 3.97 & 0.2519 \\
\hline
个体3 & $[-3.11, 3.88, -4.01, 1.23]$ & 108.80 & 0.0092 \\
\hline
个体4 & $[1.10, 0.80, -1.20, -0.90]$ & 19.98 & 0.0500 \\
\hline
\end{tabular}
\end{table}

\textbf{观察}:个体2表现最优,其适应度(0.2519)远超其他个体。记录至今最优解(精英):个体2,其解为$[0.10, -0.20, 0.15, -0.05]$。

\subsubsection{步骤2:选择操作(轮盘赌选择)}

计算总适应度:$F_{\text{total}} = 0.0087 + 0.2519 + 0.0092 + 0.0500 = 0.3198$

计算每个个体的选择概率和累积概率:

\begin{table}[H]
\centering
\caption{选择概率计算}
\begin{tabular}{|c|c|c|c|}
\hline
编号 & 适应度 & 选择概率 & 累积概率 \\
\hline
个体1 & 0.0087 & 0.027 & 0.027 \\
\hline
个体2 & 0.2519 & 0.788 & 0.815 \\
\hline
个体3 & 0.0092 & 0.029 & 0.844 \\
\hline
个体4 & 0.0500 & 0.156 & 1.000 \\
\hline
\end{tabular}
\end{table}

通过轮盘赌选择,假设选中的个体为:个体2、个体4、个体2、个体1。

\subsubsection{步骤3:交叉操作(算术交叉)}

设定交叉概率$P_c = 0.9$。

\begin{enumerate}
    \item 配对1:(个体2,个体4)。生成随机数$r=0.85<0.9$,执行交叉。
    
    设随机数$\alpha = 0.8$:
    \[
    \begin{aligned}
    \text{后代1} &= 0.8 \cdot \text{个体2} + 0.2 \cdot \text{个体4} = [0.30, 0.00, -0.21, -0.25] \\
    \text{后代2} &= 0.2 \cdot \text{个体2} + 0.8 \cdot \text{个体4} = [0.90, 0.60, -0.93, -0.73]
    \end{aligned}
    \]
    
    \item 配对2:(个体2,个体1)。生成随机数$r=0.75<0.9$,执行交叉。
    
    设随机数$\alpha = 0.9$:
    \[
    \begin{aligned}
    \text{后代3} &= 0.9 \cdot \text{个体2} + 0.1 \cdot \text{个体1} = [0.50, -0.50, 0.39, -0.53] \\
    \text{后代4} &= 0.1 \cdot \text{个体2} + 0.9 \cdot \text{个体1} = [3.71, -2.92, 2.31, -4.33]
    \end{aligned}
    \]
\end{enumerate}

\subsubsection{步骤4:变异操作(高斯变异)}

设定变异概率$P_m = 0.1$,变异强度$\sigma = 0.1$。

遍历4个后代的共16个基因,假设只有一个随机数小于$P_m$,且该事件发生在后代2的第4个基因$x_4$上:
\[
x_4' = -0.73 + N(0, 0.1^2) = -0.73 + 0.08 = -0.65
\]
最终后代2':$[0.90, 0.60, -0.93, -0.65]$

\subsubsection{步骤5:形成临时新种群并评估}

\begin{table}[H]
\centering
\caption{临时新种群}
\begin{tabular}{|c|c|c|c|}
\hline
编号 & 染色体$(x_1, x_2, x_3, x_4)$ & 目标函数值$f(\mathbf{x})$ & 适应度 \\
\hline
后代1 & $[0.30, 0.00, -0.21, -0.25]$ & 2.03 & 0.4926 \\
\hline
后代2' & $[0.90, 0.60, -0.93, -0.65]$ & 12.01 & 0.0833 \\
\hline
后代3 & $[0.50, -0.50, 0.39, -0.53]$ & 8.85 & 0.1130 \\
\hline
后代4 & $[3.71, -2.92, 2.31, -4.33]$ & 100.25 & 0.0100 \\
\hline
\end{tabular}
\end{table}

\subsubsection{步骤6:执行精英策略}

\begin{enumerate}
    \item 找到当代最差解:后代4(适应度0.0100)
    \item 获取至今最优解:个体2(来自$P_0$,适应度0.2519)
    \item 执行替换:后代4被个体2替换
\end{enumerate}

\subsubsection{步骤7:最终新一代种群$P_1$}

\begin{table}[H]
\centering
\caption{新一代种群$P_1$}
\begin{tabular}{|c|c|c|c|c|}
\hline
编号 & 染色体$(x_1, x_2, x_3, x_4)$ & 来源 & 目标函数值$f(\mathbf{x})$ & 适应度 \\
\hline
新个体1 & $[0.30, 0.00, -0.21, -0.25]$ & 后代1 & 2.03 & 0.4926 \\
\hline
新个体2 & $[0.90, 0.60, -0.93, -0.65]$ & 后代2' & 12.01 & 0.0833 \\
\hline
新个体3 & $[0.50, -0.50, 0.39, -0.53]$ & 后代3 & 8.85 & 0.1130 \\
\hline
新个体4 & $[0.10, -0.20, 0.15, -0.05]$ & 精英替换 & 3.97 & 0.2519 \\
\hline
\end{tabular}
\end{table}

\textbf{观察}:新一代种群的最优适应度从0.2519提升到0.4926,说明算法在向更好的方向进化。

\section{应用2:旅行商问题(Traveling Salesman Problem, TSP)}

\subsection{问题定义}

旅行商问题:给定一个城市集合,找到一条访问每个城市一次且仅一次,并最终返回出发点的最短回路。

\begin{itemize}
    \item 城市可以代表:客户、传感器、DNA片段、焊接点等
    \item 距离可以表示:时间、成本、相似性度量等
\end{itemize}

\subsubsection{数学形式化}

设城市集合为$C = \{1, 2, \dots, n\}$,距离矩阵为$D = (d_{ij})_{n \times n}$。目标是找到城市的一个排列$\pi = (\pi(1), \pi(2), \dots, \pi(n))$,最小化总路径长度:
\[
\text{Minimize } Z = \sum_{i=1}^{n-1} d_{\pi(i), \pi(i+1)} + d_{\pi(n), \pi(1)}
\]

\subsubsection{实例数据}

距离矩阵:
\[
D = \begin{bmatrix}
\infty & 20 & 42 & 35 \\
20 & \infty & 30 & 34 \\
42 & 30 & \infty & 12 \\
35 & 34 & 12 & \infty
\end{bmatrix}
\]

\subsection{编码策略:排列编码(Permutation Encoding)}

对于TSP,最有效的编码方式是排列编码:
\begin{itemize}
    \item 染色体是一个长度为$n$的数组,包含从1到$n$的所有城市ID,每个ID只出现一次
    \item 数组的顺序代表旅行的顺序
    \item 示例:染色体$[1, 4, 3, 2]$代表路径$1 \to 4 \to 3 \to 2 \to 1$
\end{itemize}

这种编码方式天然满足TSP"访问每个城市一次且仅一次"的核心约束。

\subsection{遗传算子设计}

\subsubsection{种群初始化}

\begin{enumerate}
    \item \textbf{随机初始化}:多次随机打乱一个包含所有城市ID的数组
    \begin{itemize}
        \item 创建有序数组,如$[1, 2, 3, 4]$
        \item 对该数组执行随机置换算法,得到随机排列,如$[3, 1, 4, 2]$
        \item 重复此过程,直到生成足够数量的初始个体
    \end{itemize}
    
    \item \textbf{贪心初始化}:提高初始种群的平均质量
    \begin{itemize}
        \item 从一个随机城市出发
        \item 每一步都选择前往所有未访问城市中最近的一个
        \item 直到所有城市都被访问
    \end{itemize}
    
    示例(从城市1出发):
    \begin{enumerate}
        \item 路径:$[1]$
        \item 下一站(距1最近):2 $\to$ 路径:$[1, 2]$
        \item 下一站(距2最近且未访问):3 $\to$ 路径:$[1, 2, 3]$
        \item 下一站(只剩4):4 $\to$ 路径:$[1, 2, 3, 4]$
        \item 总距离:$d_{12} + d_{23} + d_{34} + d_{41} = 20 + 30 + 12 + 35 = 97$
    \end{enumerate}
\end{enumerate}

最终种群可以由1个贪心解和多个随机解组成,以兼顾开局优势和种群多样性。

\subsubsection{适应度与锦标赛选择}

\begin{itemize}
    \item \textbf{适应度}:在TSP中,目标是最小化距离,因此个体的"适应性"与距离成反比
    \item \textbf{锦标赛选择}:这种选择策略只关心个体间的优劣排序,不需要将距离转换为适应度值
\end{itemize}

锦标赛选择流程(以锦标赛大小$K=2$为例):
\begin{enumerate}
    \item 从当前种群中随机选出2个个体
    \item 直接比较这两个个体的总旅行距离
    \item 总距离更短的那个个体"获胜",被复制到下一代的交配池中
    \item 重复以上步骤,直到交配池满员
\end{enumerate}

\subsubsection{维持合法性的交叉算子:部分匹配交叉(PMX)}

\begin{figure}[H]
\centering
\includegraphics[width=0.4\textwidth]{picture/pmx.png}
\caption{部分匹配交叉(PMX)示意图}
\end{figure}

PMX操作步骤:
\begin{enumerate}
    \item 随机选择两个交叉点
    \item 交换两个父代个体交叉点间的片段
    \item 根据映射关系修复冲突,确保每个城市只出现一次
\end{enumerate}

示例:
\begin{itemize}
    \item 父代1:$[1, 2, 3, 4, 5, 6, 7, 8, 9]$
    \item 父代2:$[4, 5, 2, 1, 8, 7, 6, 9, 3]$
    \item 交叉点:3和6
    \item 交换中间片段:
    \begin{itemize}
        \item 子代1中间:$[1, 8, 7, 6]$
        \item 子代2中间:$[3, 4, 5, 6]$
    \end{itemize}
    \item 修复冲突,得到合法子代
\end{itemize}

\subsubsection{维持合法性的变异算子}

\begin{enumerate}
    \item \textbf{交换变异(Swap Mutation)}:随机选择染色体上的两个位置,并交换这两个位置上的城市
    \[
    1\ 2\ 3\ 4\ 5\ 6\ 7\ 8\ 9\ 0 \to 1\ 6\ 3\ 4\ 5\ 2\ 7\ 8\ 9\ 0
    \]
    
    \item \textbf{乱序变异(Scramble Mutation)}:随机选择染色体的一个子集,然后将这个子集中的城市进行随机打乱
    \[
    0\ 1\ 2\ 2\ 3\ 4\ 5\ 6\ 7\ 8\ 9 \to 0\ 1\ 3\ 6\ 2\ 4\ 5\ 7\ 8\ 9
    \]
    
    \item \textbf{倒置变异(Inversion Mutation)}:随机选择染色体中的一个子段,然后将其完全逆序
    \[
    0\ 1\ 2\ 3\ 4\ 5\ 6\ 7\ 8\ 9 \to 0\ 1\ 6\ 5\ 4\ 3\ 2\ 7\ 8\ 9
    \]
\end{enumerate}

\subsubsection{局部搜索增强:2-Opt算法}

为了结合GA的全局探索能力和局部搜索的精细挖掘能力,可以引入2-Opt算法:

\begin{algorithm}[H]
\caption{2-Opt局部搜索}
\begin{algorithmic}[1]
\REQUIRE 当前路径$\pi$
\ENSURE 改进后的路径
\STATE 设置$improved \gets \text{True}$
\WHILE{$improved$}
    \STATE $improved \gets \text{False}$
    \FOR{$i = 1$ to $n-2$}
        \FOR{$j = i+2$ to $n$}
            \STATE 计算当前边$(i, i+1)$和$(j, j+1)$的长度$d_1$
            \STATE 计算交换后边$(i, j)$和$(i+1, j+1)$的长度$d_2$
            \IF{$d_2 < d_1$}
                \STATE 交换路径中$i+1$到$j$之间的城市顺序
                \STATE $improved \gets \text{True}$
            \ENDIF
        \ENDFOR
    \ENDFOR
\ENDWHILE
\RETURN 改进后的路径
\end{algorithmic}
\end{algorithm}

2-Opt操作:随机移除两条不相交的边,用唯一可能的方式重连,若路径变短则接受新路径。

\subsection{GA求解TSP的整体流程}

\begin{algorithm}[H]
\caption{GA for TSP with Local Search}
\begin{algorithmic}[1]
\REQUIRE 城市数量$n$,距离矩阵$D$,种群大小$N$,最大迭代次数$G_{\max}$
\ENSURE 至今最优路径
\STATE 初始化种群$P_0$(包含随机解和贪心解)
\STATE 评估$P_0$中每个个体的路径长度
\STATE 记录至今最优解
\FOR{$g = 1$ to $G_{\max}$}
    \STATE 锦标赛选择:从当前种群中选择$N$个个体形成交配池
    \STATE PMX交叉:以概率$P_c$对交配池中的个体进行交叉
    \STATE 变异操作:以概率$P_m$对新个体进行变异(交换、倒置等)
    \STATE 评估新个体的路径长度
    \STATE 执行精英策略:用至今最优解替换当前最差解
    \STATE 局部搜索:对本代最优解执行2-Opt优化
    \STATE 更新至今最优解
\ENDFOR
\RETURN 至今最优解
\end{algorithmic}
\end{algorithm}

\section{总结}

遗传算法作为一种基于自然选择和遗传机制的优化算法,具有以下特点:

\begin{enumerate}
    \item \textbf{哲学基础}:体现了经验主义的哲学思想,通过"尝试-评估-选择"的迭代过程逼近最优解
    
    \item \textbf{算法特点}:
    \begin{itemize}
        \item 群体性搜索,避免陷入局部最优
        \item 适用于黑箱优化、非凸、不可导等复杂问题
        \item 灵活性强,只需定义适应度函数
        \item 隐含并行性,易于并行实现
    \end{itemize}
    
    \item \textbf{关键组件}:
    \begin{itemize}
        \item 编码策略:将问题解映射为染色体
        \item 适应度函数:评价解的质量
        \item 遗传算子:选择、交叉、变异
        \item 精英策略:保留历史最优解
    \end{itemize}
    
    \item \textbf{参数设置}:
    \begin{itemize}
        \item 种群大小$N$:权衡探索广度与深度
        \item 交叉概率$P_c$:控制基因重组频率
        \item 变异概率$P_m$:控制引入新基因的概率
    \end{itemize}
    
    \item \textbf{应用领域}:
    \begin{itemize}
        \item 函数优化:连续、离散、混合优化问题
        \item 组合优化:TSP、调度、布局等问题
        \item 机器学习:特征选择、参数调优
        \item 工程设计:结构优化、电路设计等
    \end{itemize}
    
    \item \textbf{改进方向}:
    \begin{itemize}
        \item 自适应参数调整
        \item 多种群协同进化
        \item 与局部搜索结合(模因算法)
        \item 并行化实现
    \end{itemize}
\end{enumerate}

遗传算法虽然不能保证找到全局最优解,但在许多复杂优化问题上表现出了良好的性能。它的成功验证了自然进化机制作为一种优化策略的有效性,也为解决复杂优化问题提供了新的思路和方法。
