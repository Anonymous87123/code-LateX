\chapter{进化计算之拓展篇}
\section{引言:从连续优化到组合优化与程序生成}

在前面的学习中,我们深入探讨了进化计算在连续优化问题上的应用,特别是差分进化(DE)、粒子群优化(PSO)和协方差矩阵自适应进化策略(CMA-ES)这三种经典算法。然而,现实世界中的优化问题不仅仅局限于连续空间,还存在大量的组合优化问题,如旅行商问题(TSP)、调度问题、路径规划等。同时,我们也不仅仅满足于优化参数,有时我们希望自动生成解决问题的程序或策略。本章将介绍进化计算的两个重要拓展方向:蚁群优化(Ant Colony Optimization, ACO)和遗传编程(Genetic Programming, GP)。

\subsection{连续优化与组合优化的区别}

\begin{table}[H]
\centering
\caption{连续优化与组合优化的对比}
\begin{tabular}{p{0.3\textwidth}p{0.3\textwidth}p{0.3\textwidth}}
\toprule
\textbf{特征} & \textbf{连续优化} & \textbf{组合优化} \\
\midrule
决策变量 & 连续实数 & 离散取值(整数、排列、集合等) \\
\hline
搜索空间 & 无限、连续、通常有界 & 有限、离散、通常规模巨大 \\
\hline
典型问题 & 函数优化、参数调优 & TSP、调度、装箱、图着色 \\
\hline
算法特点 & 基于梯度或随机搜索 & 基于启发式、局部搜索、元启发式 \\
\hline
解的质量评估 & 目标函数值 & 目标函数值,但可能涉及约束满足 \\
\hline
挑战 & 局部最优、高维度、多峰性 & 组合爆炸、约束复杂、NP难 \\
\bottomrule
\end{tabular}
\end{table}

组合优化问题通常具有以下特点:
\begin{itemize}
    \item \textbf{有限但巨大的搜索空间}:虽然理论上解的数量有限,但实际中往往无法枚举
    \item \textbf{NP难问题}:大多数有实际意义的组合优化问题都是NP难的
    \item \textbf{约束复杂}:通常包含多种类型的约束条件
    \item \textbf{离散结构}:解通常具有组合结构,如排列、划分、匹配等
\end{itemize}

\subsection{自然启发的优化思想}

进化计算的核心思想是受自然现象启发,通过模拟自然界的智能行为来解决复杂的优化问题。我们已经学习了模拟生物进化的遗传算法(GA)、模拟鸟群觅食的粒子群优化(PSO)等。本章将介绍另外两种受自然启发的算法:

\begin{itemize}
    \item \textbf{蚁群优化(ACO)}:模拟蚂蚁群体觅食行为,通过信息素的正反馈机制寻找最优路径
    \item \textbf{遗传编程(GP)}:模拟生物进化过程,自动生成计算机程序
\end{itemize}

这两种算法分别针对不同类型的优化问题:ACO主要解决组合优化问题,GP则用于自动程序设计。

\section{蚁群优化算法:模拟蚂蚁觅食的智能优化}

\subsection{自然界的蚂蚁觅食行为}

在自然界中,蚂蚁群体展现出令人惊叹的集体智能。尽管单个蚂蚁的能力有限,但蚁群却能高效地找到从巢穴到食物源的最短路径。这种能力主要依赖于一种称为\textbf{信息素(Pheromone)}的化学物质。

\subsubsection{双桥实验}

著名的双桥实验揭示了蚂蚁觅食的智能行为:

\textbf{实验过程}:
\begin{enumerate}
    \item 在蚁巢和食物源之间设置两条不同长度的路径
    \item 初始时,蚂蚁随机选择路径
    \item 由于短路径上的往返时间更短,信息素积累更快
    \item 随着时间推移,短路径上的信息素浓度越来越高
    \item 最终,大多数蚂蚁都选择短路径
\end{enumerate}

\textbf{关键机制}:
\begin{itemize}
    \item \textbf{正反馈}:更多的蚂蚁选择信息素浓度高的路径
    \item \textbf{自组织}:没有中央控制,通过局部相互作用形成全局模式
    \item \textbf{随机性}:蚂蚁的选择有一定随机性,避免陷入局部最优
\end{itemize}

\subsection{蚁群优化算法的提出}

蚁群优化算法由Marco Dorigo于1992年在其博士论文中首次提出,随后在1996年完善并正式发表。ACO是群体智能(Swarm Intelligence)的代表性算法之一。

\begin{table}[H]
\centering
\caption{自然界蚂蚁觅食行为与ACO算法的对应关系}
\begin{tabular}{p{0.3\textwidth}p{0.3\textwidth}p{0.3\textwidth}}
\toprule
\textbf{自然现象} & \textbf{ACO算法概念} & \textbf{数学表示/解释} \\
\midrule
觅食空间 & 问题的搜索空间 & 所有可能解的集合 \\
\hline
蚂蚁 & 搜索代理 & 算法中的个体,负责构建解 \\
\hline
蚂蚁构建的路径 & 一个候选解 & 如TSP中的城市访问序列 \\
\hline
信息素 & 人工信息素 & 记录路径优劣的数值,通常表示为矩阵 \\
\hline
信息素挥发 & 信息素更新机制 & 避免算法早熟,保持探索能力 \\
\hline
正反馈 & 强化学习机制 & 优秀解的信息素增强,吸引更多蚂蚁 \\
\hline
最优路径 & 问题的最优解 & 算法找到的最佳解 \\
\bottomrule
\end{tabular}
\end{table}

\subsection{基本蚁群系统(Ant System, AS)}

基本蚁群系统是最早的ACO算法,以TSP(旅行商问题)为典型应用场景。TSP问题可以形式化描述为:

给定$n$个城市和城市间的距离矩阵$D = (d_{ij})_{n \times n}$,寻找一条访问每个城市恰好一次并回到起点的最短回路。

\subsubsection{AS算法的核心要素}

\begin{enumerate}
    \item \textbf{信息素矩阵$\tau$}:$\tau_{ij}$表示边$(i,j)$上的信息素浓度
    \item \textbf{启发式信息$\eta$}:$\eta_{ij} = 1/d_{ij}$,表示从城市$i$到$j$的启发式吸引力
    \item \textbf{蚂蚁的路径构建}:每只蚂蚁按照概率选择下一个城市
    \item \textbf{信息素更新}:包括挥发和增强两个过程
\end{enumerate}

\subsubsection{路径构建:随机比例规则}

蚂蚁$k$在城市$i$选择下一个城市$j$的概率为:

\[
p_k(i,j) = 
\begin{cases}
\dfrac{[\tau(i,j)]^\alpha [\eta(i,j)]^\beta}{\sum\limits_{u \in J_k(i)} [\tau(i,u)]^\alpha [\eta(i,u)]^\beta}, & \text{if } j \in J_k(i) \\
0, & \text{otherwise}
\end{cases}
\]

其中:
\begin{itemize}
    \item $J_k(i)$:蚂蚁$k$在城市$i$可以访问的、且未访问过的城市集合
    \item $\alpha$:信息素重要程度参数,控制信息素的影响
    \item $\beta$:启发式信息重要程度参数,控制启发式信息的影响
    \item $\tau(i,j)$:边$(i,j)$上的信息素浓度
    \item $\eta(i,j)$:启发式信息,通常$\eta(i,j) = 1/d_{ij}$
\end{itemize}

\textbf{参数$\alpha$和$\beta$的作用}:
\begin{itemize}
    \item $\alpha = 0$:算法退化为贪心算法,只考虑距离因素
    \item $\beta = 0$:算法只依赖信息素,可能陷入局部最优
    \item 通常设置:$\alpha = 1$,$\beta = 2 \sim 5$
\end{itemize}

\subsubsection{信息素更新机制}

信息素更新包括两个过程:挥发和增强。

\begin{enumerate}
    \item \textbf{信息素挥发}:所有边上的信息素按一定比例挥发
    \[
    \tau(i,j) \leftarrow (1 - \rho) \cdot \tau(i,j)
    \]
    其中$\rho \in (0,1]$是挥发系数,通常$\rho = 0.5$。
    
    \item \textbf{信息素增强}:蚂蚁在其经过的边上释放信息素
    \[
    \tau(i,j) \leftarrow \tau(i,j) + \sum_{k=1}^m \Delta \tau_k(i,j)
    \]
    其中$m$是蚂蚁数量,$\Delta \tau_k(i,j)$是蚂蚁$k$在边$(i,j)$上释放的信息素量:
    \[
    \Delta \tau_k(i,j) = 
    \begin{cases}
    Q / L_k, & \text{if 蚂蚁}k\text{经过边}(i,j) \\
    0, & \text{otherwise}
    \end{cases}
    \]
    其中$Q$是常数,$L_k$是蚂蚁$k$构建路径的长度。
\end{enumerate}

\subsubsection{初始信息素设置}

初始信息素浓度$\tau_0$的设置对算法性能有重要影响:
\[
\tau_0 = \frac{m}{L_{\text{nn}}}
\]
其中$L_{\text{nn}}$是使用最近邻贪心算法得到的路径长度。

\textbf{设置原理}:
\begin{itemize}
    \item 与问题规模相关:$L_{\text{nn}}$反映了问题难度
    \item 与蚂蚁数量相关:$m$越大,初始信息素相对越小
    \item 量纲一致性:与信息素增强公式中的$Q/L_k$量纲一致
\end{itemize}

\subsubsection{AS算法流程}

\begin{algorithm}[H]
\caption{基本蚁群系统(Ant System, AS)算法}
\begin{algorithmic}[1]
\REQUIRE TSP实例,蚂蚁数量$m$,参数$\alpha, \beta, \rho, Q$,最大迭代次数$T_{\max}$
\ENSURE 最优路径$T^*$和其长度$L^*$
\STATE 初始化:计算启发式信息$\eta_{ij} = 1/d_{ij}$,初始化信息素$\tau_{ij} = \tau_0$
\STATE 计算最近邻路径长度$L_{\text{nn}}$,设置$\tau_0 = m/L_{\text{nn}}$
\STATE 设置最优路径$T^* = \emptyset$,$L^* = \infty$
\FOR{$t = 1$ to $T_{\max}$}
    \FOR{$k = 1$ to $m$}
        \STATE 随机选择起始城市$s_k$
        \STATE 初始化已访问城市列表$visited_k = \{s_k\}$
        \STATE 当前城市$c = s_k$
        \WHILE{$|visited_k| < n$}
            \STATE 根据公式计算选择每个未访问城市的概率$p_k(c,j)$
            \STATE 使用轮盘赌选择下一个城市$j$
            \STATE 将$j$加入$visited_k$,更新当前城市$c = j$
        \ENDWHILE
        \STATE 计算路径长度$L_k$,包括返回起点的距离
        \IF{$L_k < L^*$}
            \STATE 更新$L^* = L_k$,$T^* = visited_k$
        \ENDIF
    \ENDFOR
    \STATE 信息素挥发:$\tau_{ij} = (1-\rho)\tau_{ij}$,对所有$i,j$
    \STATE 信息素增强:对每只蚂蚁$k$,对其路径上的每条边$(i,j)$,$\tau_{ij} = \tau_{ij} + Q/L_k$
\ENDFOR
\RETURN $T^*, L^*$
\end{algorithmic}
\end{algorithm}

\begin{figure}[H]
\centering
\includegraphics[width=0.3\textwidth]{picture/as.png}
\caption{AS算法流程示意图:包括路径构建和信息素更新两个主要阶段}
\end{figure}

\subsection{AS算法实例:四城市TSP问题}

\begin{example}[四城市TSP问题]
距离矩阵为:
\[
D = \begin{bmatrix}
\infty & 3 & 1 & 2 \\
3 & \infty & 5 & 4 \\
1 & 5 & \infty & 2 \\
2 & 4 & 2 & \infty
\end{bmatrix}
\]

参数设置:$m=3$,$\alpha=1$,$\beta=2$,$\rho=0.5$,$Q=1$。
\end{example}

\subsubsection{步骤1:初始化}

\begin{enumerate}
    \item 计算最近邻路径:从A出发,最近邻为C(1),然后D(2),然后B(4),返回A(3),总长$L_{\text{nn}}=10$
    \item 初始信息素:$\tau_0 = m/L_{\text{nn}} = 3/10 = 0.3$
    \item 信息素矩阵:
    \[
    \tau^{(0)} = \begin{bmatrix}
    0 & 0.3 & 0.3 & 0.3 \\
    0.3 & 0 & 0.3 & 0.3 \\
    0.3 & 0.3 & 0 & 0.3 \\
    0.3 & 0.3 & 0.3 & 0
    \end{bmatrix}
    \]
    \item 启发式信息矩阵:
    \[
    \eta = \begin{bmatrix}
    0 & 1/3 & 1 & 1/2 \\
    1/3 & 0 & 1/5 & 1/4 \\
    1 & 1/5 & 0 & 1/2 \\
    1/2 & 1/4 & 1/2 & 0
    \end{bmatrix}
    \]
\end{enumerate}

\subsubsection{步骤2:路径构建(以蚂蚁1为例)}

假设蚂蚁1从城市A出发:
\begin{enumerate}
    \item 当前在城市A,可访问城市:B, C, D
    \item 计算选择概率:
    \begin{align*}
    p(A,B) &= \frac{0.3^1 \times (1/3)^2}{0.3^1 \times (1/3)^2 + 0.3^1 \times 1^2 + 0.3^1 \times (1/2)^2} = \frac{0.0333}{0.4083} = 0.0815 \\
    p(A,C) &= \frac{0.3}{0.4083} = 0.7348 \\
    p(A,D) &= \frac{0.075}{0.4083} = 0.1837
    \end{align*}
    \item 轮盘赌选择:假设随机数0.05,选择城市B
    \item 以此类推,最终得到路径A→B→D→C→A,长度$L_1=3+4+2+1=10$
\end{enumerate}

\subsubsection{步骤3:信息素更新}

\begin{enumerate}
    \item 信息素挥发:$\tau_{ij} = 0.5 \times \tau_{ij}$
    \item 信息素增强:假设三只蚂蚁的路径长度分别为10, 10, 12
    \begin{itemize}
        \item 蚂蚁1在边(A,B)上增强:$\Delta \tau_1(A,B) = 1/10 = 0.1$
        \item 蚂蚁2在边(A,B)上增强:$\Delta \tau_2(A,B) = 0.1$
        \item 蚂蚁3未经过边(A,B)
    \end{itemize}
    \item 边(A,B)更新后:$\tau_{AB} = 0.5\times0.3 + 0.1 + 0.1 = 0.35$
\end{enumerate}

\subsection{蚁群系统(Ant Colony System, ACS)的改进}

基本AS算法存在收敛速度慢、易陷入局部最优等问题。Dorigo和Gambardella于1997年提出了改进的蚁群系统(ACS)。

\subsubsection{ACS的三项改进}

\begin{enumerate}
    \item \textbf{伪随机比例规则}:平衡探索与开发
    \item \textbf{局部信息素更新}:增加探索能力
    \item \textbf{全局信息素更新}:只更新最优路径
\end{enumerate}

\subsubsection{伪随机比例规则}

蚂蚁$k$在城市$i$选择下一个城市$j$的规则:
\[
j = 
\begin{cases}
\arg\max_{u \in J_k(i)} \{[\tau(i,u)]^\alpha [\eta(i,u)]^\beta\}, & \text{if } q \leq q_0 \\
S, & \text{otherwise}
\end{cases}
\]
其中:
\begin{itemize}
    \item $q$是$[0,1]$均匀分布的随机数
    \item $q_0$是参数,控制开发与探索的平衡
    \item $S$是按照AS的概率分布随机选择的结果
\end{itemize}

\textbf{物理意义}:
\begin{itemize}
    \item 以概率$q_0$选择当前最优的边(开发)
    \item 以概率$1-q_0$按照概率分布随机选择(探索)
    \item 通常$q_0 = 0.7 \sim 0.99$
\end{itemize}

\subsubsection{局部信息素更新}

蚂蚁在构建路径的过程中,每经过一条边$(i,j)$,立即进行局部更新:
\[
\tau(i,j) = (1 - \xi) \cdot \tau(i,j) + \xi \cdot \tau_0
\]
其中$\xi \in (0,1)$是局部挥发系数,通常$\xi = 0.1$。

\textbf{作用}:
\begin{itemize}
    \item 降低已访问边的信息素,鼓励探索新路径
    \item 避免算法过早收敛
    \item 增加解的多样性
\end{itemize}

\subsubsection{全局信息素更新}

在每轮迭代结束后,只对全局最优路径$T_{\text{gb}}$进行更新:
\[
\tau(i,j) = (1 - \rho) \cdot \tau(i,j) + \rho \cdot \Delta \tau(i,j), \quad \forall (i,j) \in T_{\text{gb}}
\]
其中$\Delta \tau(i,j) = 1/L_{\text{gb}}$,$L_{\text{gb}}$是全局最优路径长度。

\textbf{与AS的区别}:
\begin{itemize}
    \item AS更新所有蚂蚁经过的边
    \item ACS只更新全局最优路径上的边
    \item 计算复杂度从$O(n^2)$降到$O(n)$
\end{itemize}

\subsubsection{信息素边界限制}

ACS中信息素被限制在$[\tau_{\min}, \tau_{\max}]$范围内:
\begin{itemize}
    \item $\tau_{\min} = \tau_0$,通常$\tau_0 = 1/(nL_{\text{nn}})$
    \item $\tau_{\max} = 1/(\rho L_{\text{gb}})$
\end{itemize}

\textbf{作用}:
\begin{itemize}
    \item 避免某些边的信息素过高导致早熟
    \item 保证所有边都有被选择的机会
\end{itemize}

\subsubsection{ACS算法流程}

\begin{algorithm}[H]
\caption{蚁群系统(Ant Colony System, ACS)算法}
\begin{algorithmic}[1]
\REQUIRE TSP实例,蚂蚁数量$m$,参数$\alpha, \beta, \rho, \xi, q_0$,最大迭代次数$T_{\max}$
\ENSURE 最优路径$T_{\text{gb}}$和其长度$L_{\text{gb}}$
\STATE 初始化:计算启发式信息$\eta_{ij} = 1/d_{ij}$,初始化信息素$\tau_{ij} = \tau_0$
\STATE 计算最近邻路径长度$L_{\text{nn}}$,设置$\tau_0 = 1/(nL_{\text{nn}})$
\STATE 设置全局最优路径$T_{\text{gb}} = \emptyset$,$L_{\text{gb}} = \infty$
\FOR{$t = 1$ to $T_{\max}$}
    \STATE 初始化每只蚂蚁的起始城市
    \FOR{$k = 1$ to $m$}
        \STATE 当前城市$i$ = 起始城市
        \STATE 初始化已访问城市列表$visited_k = \{i\}$
        \WHILE{$|visited_k| < n$}
            \STATE 按照伪随机比例规则选择下一个城市$j$
            \STATE 局部信息素更新:$\tau(i,j) = (1-\xi)\tau(i,j) + \xi\tau_0$
            \STATE 将$j$加入$visited_k$,更新当前城市$i = j$
        \ENDWHILE
        \STATE 计算路径长度$L_k$
        \IF{$L_k < L_{\text{gb}}$}
            \STATE 更新$L_{\text{gb}} = L_k$,$T_{\text{gb}} = visited_k$
        \ENDIF
    \ENDFOR
    \STATE 全局信息素更新:对$T_{\text{gb}}$上的每条边$(i,j)$,$\tau(i,j) = (1-\rho)\tau(i,j) + \rho/L_{\text{gb}}$
    \STATE 限制信息素范围:$\tau(i,j) = \max(\tau_{\min}, \min(\tau_{\max}, \tau(i,j)))$
\ENDFOR
\RETURN $T_{\text{gb}}, L_{\text{gb}}$
\end{algorithmic}
\end{algorithm}

\subsection{ACO参数设置与调优}

\begin{table}[H]
\centering
\caption{ACO算法参数设置指南}
\begin{tabular}{p{0.2\textwidth}p{0.2\textwidth}p{0.25\textwidth}p{0.25\textwidth}}
\toprule
\textbf{参数} & \textbf{符号} & \textbf{推荐范围} & \textbf{影响} \\
\midrule
蚂蚁数量 & $m$ & $n$(城市数) & 探索能力,计算成本 \\
\hline
信息素重要性 & $\alpha$ & 1 & 信息素的影响程度 \\
\hline
启发式重要性 & $\beta$ & 2-5 & 启发式信息的影响程度 \\
\hline
全局挥发率 & $\rho$ & AS: 0.5, ACS: 0.1 & 信息素挥发速度,影响收敛 \\
\hline
局部挥发率 & $\xi$ & ACS: 0.1 & 增加探索,防止早熟 \\
\hline
开发概率 & $q_0$ & ACS: 0.7-0.99 & 控制开发与探索平衡 \\
\hline
初始信息素 & $\tau_0$ & AS: $m/L_{\text{nn}}$, ACS: $1/(nL_{\text{nn}})$ & 影响初期搜索行为 \\
\bottomrule
\end{tabular}
\end{table}

\textbf{调优建议}:
\begin{enumerate}
    \item 从默认参数开始
    \item 根据问题规模调整蚂蚁数量
    \item 通过实验调整$\alpha$和$\beta$的平衡
    \item 收敛过快时减小$q_0$或增加$\xi$
    \item 收敛过慢时增加$q_0$或减小$\xi$
\end{enumerate}

\subsection{ACO与其他算法的对比}

\subsubsection{ACO vs GA}

\begin{table}[H]
\centering
\caption{ACO与GA的对比}
\begin{tabular}{p{0.3\textwidth}p{0.3\textwidth}p{0.3\textwidth}}
\toprule
\textbf{维度} & \textbf{蚁群优化(ACO)} & \textbf{遗传算法(GA)} \\
\midrule
灵感来源 & 蚂蚁群体觅食行为 & 生物进化理论 \\
\hline
解生成方式 & 逐步构建(自回归) & 全局重构(非自回归) \\
\hline
约束处理 & 预防式:构建时避免违反 & 修复式:生成后修复或惩罚 \\
\hline
信息共享 & 间接共享:通过信息素 & 直接混合:通过交叉操作 \\
\hline
环境适应性 & 适应动态环境 & 静态优化为主 \\
\hline
计算特点 & 正反馈、分布式 & 选择、交叉、变异 \\
\hline
适用问题 & 路径规划、调度、分配 & 函数优化、参数调优、组合优化 \\
\bottomrule
\end{tabular}
\end{table}

\subsubsection{ACO vs Q-Learning}

\begin{table}[H]
\centering
\caption{ACO与Q-Learning的对比}
\begin{tabular}{p{0.3\textwidth}p{0.3\textwidth}p{0.3\textwidth}}
\toprule
\textbf{维度} & \textbf{蚁群优化(ACO)} & \textbf{Q-Learning} \\
\midrule
抽象本质 & 基于经验的迭代优化 & 基于经验的迭代优化 \\
\hline
主体性质 & 群体经验的分布式存储 & 单智能体的价值函数 \\
\hline
适应环境 & 调整信息素矩阵 & 调整Q-table \\
\hline
更新触发 & 一轮迭代后全局更新 & 单步动作后立即更新 \\
\hline
核心数学 & 正反馈+挥发衰减 & 贝尔曼方程+时序差分 \\
\hline
决策依据 & 启发式信息+群体经验 & 纯价值函数 \\
\hline
探索控制 & 随机比例规则+挥发率 & $\epsilon$-greedy或Boltzmann \\
\hline
优化侧重 & 全局优化效果好 & 局部决策灵敏度高 \\
\bottomrule
\end{tabular}
\end{table}

\subsection{ACO的应用领域}

ACO算法已在多个领域成功应用:

\begin{enumerate}
    \item \textbf{旅行商问题(TSP)及变种}:车辆路径问题、带时间窗的VRP
    \item \textbf{调度问题}:作业车间调度、流水车间调度
    \item \textbf{网络路由}:通信网络路由、无线传感器网络
    \item \textbf{数据挖掘}:聚类分析、特征选择
    \item \textbf{图像处理}:边缘检测、图像分割
\end{enumerate}

\section{遗传编程:自动程序生成的进化方法}

\subsection{遗传编程的基本思想}

遗传编程由John Koza于1992年提出,是遗传算法的自然扩展。与GA优化参数不同,GP优化的是计算机程序的结构。

\subsubsection{核心思想}

通过模拟自然选择过程,自动生成解决特定问题的计算机程序:
\begin{itemize}
    \item 程序表示为树状结构
    \item 通过适应度评估程序的性能
    \item 使用遗传操作(选择、交叉、变异)进化程序种群
    \item 最终得到高性能的程序
\end{itemize}

\subsubsection{与GA的关系}

\begin{itemize}
    \item \textbf{共同点}:都基于进化计算框架,使用选择、交叉、变异操作
    \item \textbf{不同点}:
    \begin{itemize}
        \item GA优化固定长度的字符串(参数)
        \item GP优化可变大小的树结构(程序)
        \item GA的交叉是等位基因交换
        \item GP的交叉是子树交换
    \end{itemize}
\end{itemize}

\subsection{GP的个体表示:程序树}

GP使用树状结构表示程序,称为程序树(Program Tree)或语法树(Syntax Tree)。

\subsubsection{基元集(Primitive Set)}

基元集是GP可用的基本构建块,包括:

\begin{enumerate}
    \item \textbf{函数集(Function Set)}:内部节点,有子节点
    \begin{itemize}
        \item 算术运算:+, -, *, /, sin, cos, exp, log
        \item 布尔运算:AND, OR, NOT
        \item 条件运算:IF-THEN-ELSE
        \item 循环运算:WHILE, FOR
        \item 其他函数:取决于具体问题
    \end{itemize}
    
    \item \textbf{终止符集(Terminal Set)}:叶节点,无子节点
    \begin{itemize}
        \item 变量:x, y, z
        \item 常量:1, 2, 3.14, true, false
        \item 零参数函数:rand(), time()
    \end{itemize}
\end{enumerate}

\subsubsection{基元集的设计原则}

\begin{enumerate}
    \item \textbf{充分性}:基元集应能表达解空间中的所有可能解
    \item \textbf{封闭性}:任何函数节点的返回值类型应与子节点类型匹配
    \item \textbf{适当性}:基元集应与问题相关,不过大也不过小
    \item \textbf{效率性}:避免不必要的基元,减少搜索空间
\end{enumerate}

\subsection{GP的基本步骤}

GP遵循进化计算的基本框架,但操作针对树结构:

\begin{algorithm}[H]
\caption{遗传编程基本流程}
\begin{algorithmic}[1]
\REQUIRE 问题描述,适应度函数,基元集,参数设置
\ENSURE 最优程序
\STATE 初始化:生成初始种群(随机程序树)
\STATE 评估:计算每个个体的适应度
\WHILE{未满足终止条件}
    \STATE 选择:根据适应度选择父代
    \STATE 繁殖:对父代应用遗传操作生成子代
    \begin{itemize}
        \item 交叉:交换两个父代的子树
        \item 变异:替换父代的子树
        \item 复制:直接复制父代
    \end{itemize}
    \STATE 评估:计算子代的适应度
    \STATE 替换:用子代替换部分或全部父代
\ENDWHILE
\RETURN 最优程序
\end{algorithmic}
\end{algorithm}

\subsubsection{种群初始化}

有三种常用的初始化方法:

\begin{enumerate}
    \item \textbf{增长法(Grow Method)}
    \begin{itemize}
        \item 从函数集随机选择根节点
        \item 对每个参数位置,从整个基元集随机选择
        \item 如果选择终止符,则该分支停止生长
        \item 如果选择函数,则继续生长
        \item 达到最大深度时,必须选择终止符
    \end{itemize}
    
    \item \textbf{完全法(Full Method)}
    \begin{itemize}
        \item 从函数集随机选择根节点
        \item 直到深度$d_{\max}-1$,都从函数集选择
        \item 在深度$d_{\max}$,从终止符集选择
        \item 生成的都是深度为$d_{\max}$的完整树
    \end{itemize}
    
    \item \textbf{分层对半分法(Ramped Half-and-Half)}
    \begin{itemize}
        \item 将种群按深度分层
        \item 每层中,50\%用增长法,50\%用完全法
        \item 结合两种方法的优点,增加多样性
    \end{itemize}
\end{enumerate}

\subsubsection{适应度评估}

适应度函数衡量程序解决目标问题的能力:

\begin{itemize}
    \item \textbf{符号回归}:预测值与真实值的误差
    \item \textbf{分类问题}:分类准确率
    \item \textbf{控制问题}:控制性能指标
    \item \textbf{程序生成}:通过测试用例的比例
\end{itemize}

常用适应度度量:
\[
\text{MSE} = \frac{1}{n} \sum_{i=1}^n (y_i - f(x_i))^2
\]
\[
\text{准确率} = \frac{\text{正确分类数}}{\text{总样本数}}
\]

\subsubsection{选择操作}

与GA类似,常用选择方法:
\begin{itemize}
    \item \textbf{轮盘赌选择}:按适应度比例选择
    \item \textbf{锦标赛选择}:随机选择$k$个个体,选择最优的
    \item \textbf{精英选择}:保留最优个体直接进入下一代
\end{itemize}

\subsubsection{交叉操作}

GP交叉是子树交换:
\begin{enumerate}
    \item 随机选择两个父代个体
    \item 在每个父代中随机选择一个交叉点
    \item 交换以交叉点为根的子树
    \item 生成两个子代个体
\end{enumerate}


\subsubsection{变异操作}

两种主要变异方式:
\begin{enumerate}
    \item \textbf{子树变异(Subtree Mutation)}
    \begin{itemize}
        \item 随机选择变异点
        \item 删除以该点为根的子树
        \item 用随机生成的子树替换
    \end{itemize}
    
    \item \textbf{单点变异(Point Mutation)}
    \begin{itemize}
        \item 随机选择变异点
        \item 用同类型的基元替换
        \item 保持树结构不变
    \end{itemize}
\end{enumerate}

\subsubsection{复制操作}

以一定概率直接复制父代到子代,保留优秀个体。

\subsection{GP的参数设置}

\begin{table}[H]
\centering
\caption{GP算法参数设置指南}
\begin{tabular}{p{0.25\textwidth}p{0.25\textwidth}p{0.4\textwidth}}
\toprule
\textbf{参数} & \textbf{推荐范围} & \textbf{说明} \\
\midrule
种群大小 & 500-5000 & 越大探索能力越强,计算成本越高 \\
\hline
最大深度 & 3-10 & 控制程序复杂度,防止膨胀 \\
\hline
交叉率 & 0.8-0.95 & 主要搜索操作,促进多样性 \\
\hline
变异率 & 0.01-0.1 & 引入新基因,防止早熟 \\
\hline
复制率 & 0-0.1 & 保留优秀个体 \\
\hline
选择方法 & 锦标赛 & 常用锦标赛大小$k=7$ \\
\hline
精英数量 & 1-5 & 保留最优个体 \\
\hline
最大代数 & 50-1000 & 根据问题复杂度调整 \\
\bottomrule
\end{tabular}
\end{table}

\subsection{GP的应用案例:符号回归}

符号回归是GP的经典应用,目标是发现数据背后的数学表达式。

\subsubsection{问题描述}

给定数据集$\{(x_i, y_i)\}_{i=1}^n$,寻找数学表达式$f(x)$,使得$f(x_i) \approx y_i$。

\begin{example}[二次多项式回归]
目标函数:$f(x) = x^2 + x + 1$
数据范围:$x \in [-1, 1]$
采样点:$x = -1.0, -0.9, \dots, 0.9, 1.0$
\end{example}

\subsubsection{GP设置}

\begin{enumerate}
    \item \textbf{终止符集}:$T = \{x, \mathbb{R}\}$,其中$\mathbb{R}$是随机常数
    \item \textbf{函数集}:$F = \{+, -, \times, \%\}$,其中\%是保护除法(除0返回1)
    \item \textbf{适应度函数}:MSE
    \[
    \text{MSE} = \frac{1}{n} \sum_{i=1}^n (y_i - f(x_i))^2
    \]
    \item \textbf{参数设置}:种群大小4,最大深度2,交叉率0.5,变异率0.25,复制率0.25
    \item \textbf{终止条件}:MSE < 0.1
\end{enumerate}

\subsubsection{GP求解过程}

\begin{figure}[H]
\centering
\includegraphics[width=0.9\textwidth]{picture/14.png}
\caption{GP求解符号回归问题的完整过程:从初始化到找到最优解}
\end{figure}

\textbf{第1代}:
\begin{itemize}
    \item 个体1: $x+1$,MSE=7.7
    \item 个体2: $x^2+1$,MSE=11.00
    \item 个体3: 2,MSE=17.98
    \item 个体4: $x$,MSE=28.70
\end{itemize}

\textbf{遗传操作}:
\begin{itemize}
    \item 复制:个体1直接复制
    \item 变异:个体3变异为$x\%x$(即1)
    \item 交叉1:个体1和4交叉,得到$x$和$x+1$
    \item 交叉2:个体1和2交叉,得到$x^2+x+1$
\end{itemize}

\textbf{第2代}:
\begin{itemize}
    \item 个体1: $x+1$,MSE=7.7
    \item 个体2: 1,MSE=11.00
    \item 个体3: $x$,MSE=28.70
    \item 个体4: $x^2+x+1$,MSE=0.0
\end{itemize}

找到最优解$f(x) = x^2 + x + 1$,终止。

\subsubsection{不同回归方法对比}


\begin{table}[H]
\centering
\caption{不同回归方法的对比}
\begin{tabular}{p{0.2\textwidth}p{0.25\textwidth}p{0.25\textwidth}p{0.2\textwidth}}
\toprule
\textbf{方法} & \textbf{模型形式} & \textbf{优点} & \textbf{缺点} \\
\midrule
线性回归 & $y = \beta_0 + \beta_1 x$ & 简单、可解释 & 只能拟合线性关系 \\
\hline
多项式回归 & $y = \sum_{i=0}^k \beta_i x^i$ & 可拟合非线性 & 阶数选择困难,易过拟合 \\
\hline
神经网络 & $y = f(Wx+b)$ & 强大拟合能力 & 黑箱、需大量数据、易过拟合 \\
\hline
符号回归 & 任意表达式 & 可解释、自动发现 & 计算复杂、可能膨胀 \\
\bottomrule
\end{tabular}
\end{table}

\subsection{GP的优缺点与挑战}

\subsubsection{优点}
\begin{itemize}
    \item \textbf{自动程序设计}:无需预先指定程序结构
    \item \textbf{可解释性}:生成的程序易于理解
    \item \textbf{灵活性}:适用于多种问题类型
    \item \textbf{创造性}:可能发现人类未想到的解决方案
\end{itemize}

\subsubsection{缺点与挑战}
\begin{enumerate}
    \item \textbf{计算复杂度高}:程序评估和进化成本高
    \item \textbf{膨胀现象(Bloat)}:程序无意义地增长
    \item \textbf{过拟合}:在训练集上表现好,测试集差
    \item \textbf{参数敏感}:需要仔细调参
    \item \textbf{局部最优}:可能陷入局部最优解
\end{enumerate}

\subsubsection{膨胀现象的原因与对策}

\textbf{膨胀原因}:
\begin{itemize}
    \item 搜索偏差:大程序有更多交叉点
    \item 无用代码:不影响适应度的代码不被淘汰
    \item 适应度地形:大程序可能在小邻域内有更好解
\end{itemize}

\textbf{对策}:
\begin{itemize}
    \item 深度限制:限制程序最大深度
    \item 大小限制:限制程序节点数
    \item 简化操作:定期简化程序
    \item 适应度惩罚:对复杂程序进行惩罚
\end{itemize}

\subsection{GP与其他算法的对比}

\subsubsection{GP vs 深度强化学习(DRL)}

\begin{table}[H]
\centering
\caption{GP与DRL的对比}
\begin{tabular}{p{0.3\textwidth}p{0.3\textwidth}p{0.3\textwidth}}
\toprule
\textbf{维度} & \textbf{遗传编程(GP)} & \textbf{深度强化学习(DRL)} \\
\midrule
策略表示 & 树结构(符号表达式) & 神经网络(数值函数) \\
\hline
可微性 & 不可微 & 可微 \\
\hline
训练方式 & 进化算法(选择、交叉、变异) & 梯度下降(反向传播) \\
\hline
先验知识 & 不依赖,从基元集构建 & 依赖网络结构设计 \\
\hline
可解释性 & 强,符号表达式易于理解 & 弱,黑箱模型难以解释 \\
\hline
探索机制 & 通过变异和交叉探索 & 通过动作随机性探索 \\
\hline
样本效率 & 通常较低 & 可以较高(有重用机制) \\
\hline
计算需求 & 大量程序评估 & 大量环境交互 \\
\hline
适用问题 & 符号回归、程序生成、控制 & 游戏、机器人控制、决策 \\
\hline
优势 & 可解释、无需梯度、创造性 & 强大表示能力、端到端学习 \\
\hline
劣势 & 计算昂贵、易膨胀、参数敏感 & 需要大量数据、不稳定、难调参 \\
\bottomrule
\end{tabular}
\end{table}

\subsubsection{作为策略的GP}

GP可以视为一种策略表示方法:
\[
\pi(s) = \text{GP-Program}(s)
\]
其中程序以状态$s$为输入,输出动作$a$。

\textbf{增强GP处理序列决策}:
\begin{itemize}
    \item 引入状态变量作为终止符
    \item 添加记忆机制
    \item 使用迭代或递归结构
\end{itemize}

\subsection{GP的应用领域}

\begin{enumerate}
    \item \textbf{符号回归}:发现数据背后的数学规律
    \item \textbf{程序生成}:自动生成算法、函数
    \item \textbf{控制器设计}:机器人控制、过程控制
    \item \textbf{电路设计}:数字电路、模拟电路
    \item \textbf{游戏AI}:游戏策略、NPC行为
    \item \textbf{数据挖掘}:特征构建、模式发现
    \item \textbf{图像处理}:滤波器设计、特征提取
\end{enumerate}

\section{总结与展望}

\subsection{算法对比总结}

\begin{table}[H]
\centering
\caption{进化计算算法综合对比}
\begin{tabular}{p{0.15\textwidth}p{0.2\textwidth}p{0.2\textwidth}p{0.2\textwidth}p{0.15\textwidth}}
\toprule
\textbf{算法} & \textbf{灵感来源} & \textbf{表示方式} & \textbf{适用问题} & \textbf{特点} \\
\midrule
GA & 生物进化 & 固定长度字符串 & 优化、搜索 & 通用、易实现 \\
\hline
DE & 向量差分 & 实数向量 & 连续优化 & 简单、高效 \\
\hline
PSO & 鸟群觅食 & 位置速度 & 连续优化 & 收敛快、参数少 \\
\hline
CMA-ES & 自然进化 & 高斯分布 & 连续优化 & 自适应、旋转不变 \\
\hline
ACO & 蚂蚁觅食 & 路径/排列 & 组合优化 & 正反馈、分布式 \\
\hline
GP & 生物进化 & 树结构 & 程序生成 & 自动设计、可解释 \\
\bottomrule
\end{tabular}
\end{table}

\subsection{选择指南}

\begin{itemize}
    \item \textbf{连续参数优化}:DE、PSO、CMA-ES
    \item \textbf{组合优化}:ACO、GA
    \item \textbf{自动程序设计}:GP
    \item \textbf{需要可解释性}:GP、符号回归
    \item \textbf{动态环境}:ACO、PSO
    \item \textbf{高维问题}:DE、CMA-ES
    \item \textbf{计算资源有限}:PSO、简单GA
\end{itemize}

\subsection{未来发展方向}

\begin{enumerate}
    \item \textbf{混合算法}:结合不同算法的优势
    \begin{itemize}
        \item ACO+局部搜索:增强局部搜索能力
        \item GP+神经网络:结合符号和连接主义
        \item 进化+强化学习:进化神经网络结构
    \end{itemize}
    
    \item \textbf{大规模并行}:利用GPU、分布式计算加速
    \item \textbf{自适应机制}:自动调整参数和策略
    \item \textbf{理论分析}:深入理解算法收敛性和复杂性
    \item \textbf{新应用领域}:量子计算、生物信息、金融科技
    \item \textbf{可解释AI}:GP在可解释机器学习中的应用
    \item \textbf{自动机器学习(AutoML)}:自动选择算法、调参
\end{enumerate}

\subsection{实践建议}

\begin{enumerate}
    \item \textbf{理解问题本质}:分析问题特点,选择合适算法
    \item \textbf{从简单开始}:先尝试简单算法和默认参数
    \item \textbf{逐步优化}:根据结果调整参数和策略
    \item \textbf{多次运行}:统计多次运行结果,评估稳定性
    \item \textbf{可视化分析}:可视化搜索过程,理解算法行为
    \item \textbf{记录实验}:详细记录实验设置和结果
    \item \textbf{理论结合实践}:理解算法原理,指导实践应用
\end{enumerate}

进化计算作为一类受自然启发的优化方法,在解决复杂问题中展现出强大能力。ACO和GP分别针对组合优化和程序生成问题,扩展了进化计算的应用范围。理解这些算法的原理、特点和应用场景,对于解决实际问题具有重要意义。随着计算技术的进步和理论的发展,进化计算将继续在科学和工程领域发挥重要作用。