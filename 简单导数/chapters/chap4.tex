\chapter{恒成立}
\section{不等式题(对应高一基本不等式模块)}
\begin{example}{来自数海漫游考前100题}{}
    若 $x - 3\sqrt{y} = \sqrt{x - 3y}$,求 $x$ 的取值范围.
\end{example}
\begin{solution}
令 $u = \sqrt{x - 3y}, \quad v = \sqrt{y}$。为了符合算术平方根,必须满足可行域约束 $u \geqslant 0, v \geqslant 0$。代入推导有:$u^2 + 3v^2 = (x - 3y) + 3y = x$。原方程化为:$x - 3v = u \implies x = u + 3v$。联立两者消去 $x$ 得到约束等式:$u^2 + 3v^2 = u + 3v \implies u^2 - u + 3v^2 - 3v = 0$。配方得隐函数椭圆方程:$$\left(u - \frac{1}{2}\right)^2 + 3\left(v - \frac{1}{2}\right)^2 = 1$$问题几何化转为:在第一象限及原点(即 $u \geqslant 0, v \geqslant 0$)内,求目标函数 $x = u + 3v$ 约束在该椭圆上的取值范围。【致命陷阱揭秘】:如果解题者在这里盲目套用拉格朗日乘数法或反演隐函数求导求 $x = u+3v$ 的极值,会轻易解得最大值在 $(1, 1)$ 处取 $4$,最小值在 $(0, 0)$ 处取 $0$。如果不加证明地默认“函数在极小值和极大值之间连续取遍所有的值”,直接就会写出答案 $[0, 4]$。但这完全忽略了物理约束条件 $u \geqslant 0, v \geqslant 0$。由于这个约束,可行域曲线在这里发生了拓扑断裂!我们通过三角换元严密参数化该椭圆:设 $u = \frac{1}{2} + \cos\theta, \quad v = \frac{1}{2} + \frac{1}{\sqrt{3}}\sin\theta$目标函数变为:$x = u + 3v = 2 + 2\cos\left(\theta - \frac{\pi}{3}\right)$我们来严格筛查同时满足 $u \geqslant 0, v \geqslant 0$ 的角度 $\theta \in [-\pi, \pi]$:$u \geqslant 0 \implies \cos\theta \geqslant -\frac{1}{2} \implies \theta \in \left[-\frac{2\pi}{3}, \frac{2\pi}{3}\right]$$v \geqslant 0 \implies \sin\theta \geqslant -\frac{\sqrt{3}}{2} \implies \theta \in \left[-\pi, -\frac{2\pi}{3}\right] \cup \left[-\frac{\pi}{3}, \pi\right]$两者取严格的交集,得到合法的参数区间竟然是:$\theta \in \left\{-\frac{2\pi}{3}\right\} \cup \left[-\frac{\pi}{3}, \frac{2\pi}{3}\right]$请注意那个孤零零的点!椭圆在第一象限(包括边界)的图像并不是一条完整的闭合连通曲线,而是被硬生生扯成了互相不连通的两部分:第一块(孤立点):当且仅当 $\theta = -\frac{2\pi}{3}$ 时,刚好 $u=0, v=0$。此时代入算出唯一孤立解 $x = 0$。第二块(连续实弧):当 $\theta \in \left[-\frac{\pi}{3}, \frac{2\pi}{3}\right]$ 时。此时相位角 $\left(\theta - \frac{\pi}{3}\right) \in \left[-\frac{2\pi}{3}, \frac{\pi}{3}\right]$。在这段连续的区间内,余弦函数连续覆盖了 $\left[-\frac{1}{2}, 1\right]$。代入算出这段连续弧贡献的 $x$ 范围为 $2 + 2\left(-\frac{1}{2}\right) \leqslant x \leqslant 2 + 2(1)$,即确切的 $x \in [1, 4]$。至于 $x \in (0, 1)$ 的空白区间,此时对应的椭圆轨迹钻入了第四象限($u>0, v<0$),导致 $\sqrt{y}$ 成了负数,这在实数范畴内是不可能完成的计算。所以$x \in \{0\} \cup [1, 4]$。
\end{solution}

\section{幂指对恒成立问题}
\begin{example}{虚调子}{}
证明不等式 $4x^2(1-x)^2 + x(1-x) \leqslant x^{1-x}\cdot(1-x)^x$ 在 $x \in [0, 1]$ 上成立
\end{example}
\begin{solution}
0,1处显然取等,下证在 $x \in (0, 1)$ 内不等式恒成立。此时左端可提取公因式 $x(1-x)$,变形为 $x(1-x)[4x(1-x) + 1]$。右端改写为 $x \cdot x^{-x} \cdot (1-x) \cdot (1-x)^{x-1}$,即 $\frac{x(1-x)}{x^x(1-x)^{1-x}}$。由于在该区间内 $x(1-x) > 0$,原不等式等价于:
$$4x(1-x) + 1 \leqslant \frac{1}{x^x(1-x)^{1-x}}$$
由于两端均为正实数,对两端取自然对数,不等号方向保持不变:
$$\ln[4x(1-x) + 1] \leqslant -x\ln x - (1-x)\ln(1-x)$$
整理得待证目标函数 $f(x) \leqslant 0$:
$$f(x) = x\ln x + (1-x)\ln(1-x) + \ln[1 + 4x(1-x)] \leqslant 0$$
考虑到函数关于 $x = 1/2$ 的对称性,令 $x = \frac{1+u}{2}$,其中 $u \in (-1, 1)$。代入可得:
$$x(1-x) = \frac{1-u^2}{4}, \quad 1 + 4x(1-x) = 2 - u^2$$
将上述分量代入 $f(x)$ 构造关于 $u$ 的偶函数 $F(u)$:
$$F(u) = \frac{1+u}{2}\ln(1+u) + \frac{1-u}{2}\ln(1-u) - \ln 2 + \ln(2-u^2)$$
对 $F(u)$ 求关于 $u$ 的一阶导数:
$$F'(u) = \frac{1}{2}\ln(1+u) + \frac{1}{2} - \frac{1}{2}\ln(1-u) - \frac{1}{2} - \frac{2u}{2-u^2} = \frac{1}{2}\ln\left(\frac{1+u}{1-u}\right) - \frac{2u}{2-u^2}$$
令辅助函数 $G(u) = 2F'(u) = \ln\left(\frac{1+u}{1-u}\right) - \frac{4u}{2-u^2}$。对其继续求导:
$$G'(u) = \frac{1}{1+u} + \frac{1}{1-u} - \frac{4(2-u^2) - 4u(-2u)}{(2-u^2)^2} = \frac{2}{1-u^2} - \frac{8+4u^2}{(2-u^2)^2}$$
通分化简分子部分:
$$2(2-u^2)^2 - (1-u^2)(8+4u^2) = 2(4-4u^2+u^4) - (8-4u^2-4u^4) = 6u^4 - 4u^2 = 2u^2(3u^2-2)$$
由此得 $G'(u) = \frac{2u^2(3u^2-2)}{(1-u^2)(2-u^2)^2}$。在 $u \in (0, 1)$ 范围内,分母始终大于零,其符号由 $3u^2-2$ 决定。当 $u \in (0, \sqrt{2/3})$ 时,$G'(u) < 0$,$G(u)$ 单调递减;当 $u \in (\sqrt{2/3}, 1)$ 时,$G'(u) > 0$,$G(u)$ 单调递增。

分析 $G(u)$ 在区间 $[0, 1)$ 上的表现:$G(0) = 0$,随 $u$ 增大,$G(u)$ 先减小至负值。当 $u \to 1^-$ 时,$G(u) \to +\infty$。根据连续函数的零点存在定理,存在唯一的 $u_0 \in (\sqrt{2/3}, 1)$ 使得 $G(u_0) = 0$。

相应地,在 $u \in (0, u_0)$ 上,$G(u) < 0$ 从而 $F'(u) < 0$,$F(u)$ 单调递减;在 $u \in (u_0, 1)$ 上,$G(u) > 0$ 从而 $F'(u) > 0$,$F(u)$ 单调递增。因此 $F(u)$ 在区间端点取得最大值。计算端点值 $F(0) = \frac{1}{2}\ln 1 + \frac{1}{2}\ln 1 - \ln 2 + \ln 2 = 0$。考察右边界极限,由 $\lim_{t\to 0^+} t\ln t = 0$,得 $\lim_{u \to 1^-} \frac{1-u}{2}\ln(1-u) = 0$,则:
$$\lim_{u \to 1^-} F(u) = \frac{2}{2}\ln 2 + 0 - \ln 2 + \ln(2-1) = 0$$
综上所述,在 $u \in (-1, 1)$ 内 $F(u) \leqslant 0$ 恒成立,等价于原不等式在 $x \in (0, 1)$ 内成立。结合边界点情况,原不等式在 $x \in [0, 1]$ 上成立。
\end{solution}

\begin{example}{虚调子}{}
证明不等式:$4x^2/(x+1)^3 + x/(x+1) \le x^{1/(x+1)}$对任意$x\in(0,1)$恒成立
\end{example}
\begin{solution}
    令 $s=\frac{x-1}{x+1}$,由于 $x>0$,可得 $s\in(-1,1)$,此变换为一一对应。由此反解出 $x=\frac{1+s}{1-s}$,进而有 $x+1=\frac{2}{1-s}$ 以及 $\frac{1}{x+1}=\frac{1-s}{2}$。将原不等式左边化简并代入以 $s$ 为变量的表达式可得:
$$ \frac{x}{x+1} = \frac{1+s}{2} ,\frac{4x^2}{(x+1)^3} = 4\left(\frac{1+s}{1-s}\right)^2 \left(\frac{1-s}{2}\right)^3 = \frac{1}{2}(1+s)^2(1-s),x^{\frac{1}{x+1}} = \left(\frac{1+s}{1-s}\right)^{\frac{1-s}{2}} $$
于是:
$$ \frac{4x^2}{(x+1)^3}+\frac{x}{x+1} = \frac{1}{2}\left[(1+s)+(1+s)^2(1-s)\right] = \frac{1}{2}(2+2s-s^2-s^3) = (1+s)\left(1-\frac{s^2}{2}\right) $$
故在 $s\in(-1,1)$ 上,原不等式等价于:
$$ (1+s)\left(1-\frac{s^2}{2}\right) \le \left(\frac{1+s}{1-s}\right)^{\frac{1-s}{2}} $$
显然当 $s\in(-1,1)$ 时两边均大于 0,同时取对数,做差构造函数 $Q(s)$:
$$ Q(s) = \frac{1-s}{2}\ln\left(\frac{1+s}{1-s}\right) - \ln(1+s) - \ln\left(1-\frac{s^2}{2}\right) $$
于是等价于证明对任意 $s\in(-1,1)$ 均有 $Q(s) \ge 0$。显然 $Q(s)$ 为偶函数。故只需证明 $Q(s) \ge 0$ 在 $s\in[0,1)$ 上成立即可。
对 $Q(s)$ 求导:
\begin{align*}
Q'(s) &= -\frac{1}{2}\ln\left(\frac{1+s}{1-s}\right) + \frac{1-s}{2}\left(\frac{1}{1+s}+\frac{1}{1-s}\right) - \frac{1}{1+s} + \frac{s}{1-\frac{s^2}{2}}=  \frac{2(2+s^2)}{(2-s^2)^2} - \frac{1}{1-s^2}\\
&= \frac{2(2+s^2)(1-s^2)-(2-s^2)^2}{(2-s^2)^2(1-s^2)} = \frac{(4-2s^2-2s^4)-(4-4s^2+s^4)}{(2-s^2)^2(1-s^2)} = \frac{s^2(2-3s^2)}{(2-s^2)^2(1-s^2)} 
\end{align*}
对于 $s\in(0,1)$,分母 $(2-s^2)^2(1-s^2) > 0$ 恒成立,故 $Q''(s)$ 的符号由分子中的 $2-3s^2$ 决定。当 $0 < s < \sqrt{\frac{2}{3}}$ 时,$Q''(s) > 0$;当 $\sqrt{\frac{2}{3}} < s < 1$ 时,$Q''(s) < 0$。这表明导函数 $Q'(s)$ 在 $\left(0, \sqrt{\frac{2}{3}}\right)$ 上单调递增,在 $\left(\sqrt{\frac{2}{3}}, 1\right)$ 上单调递减。
由于 $Q'(0) = 0$,且当 $s \to 1^-$ 时:
$$ \lim_{s\to 1^-} Q'(s) = \lim_{s\to 1^-} \left( \frac{2s}{2-s^2} - \frac{1}{2}\ln\frac{1+s}{1-s} \right) = 2 - \infty = -\infty < 0 $$
故存在唯一的 $s_0 \in \left(\sqrt{\frac{2}{3}}, 1\right)$ 使得 $Q'(s_0) = 0$。并且当 $0 < s < s_0$ 时,$Q'(s) > 0$;当 $s_0 < s < 1$ 时,$Q'(s) < 0$。由此可知,函数 $Q(s)$ 在 $[0, s_0]$ 上单调递增,在 $[s_0, 1)$ 上单调递减。

进一步考察 $Q(s)$ 在区间端点的值与极限:
显然 $Q(0) = \frac{1}{2}\ln 1 - \ln 1 - \ln 1 = 0$。当 $s\to 1^-$ 时,令 $t=1-s \to 0^+$,则有:
$$ \frac{1-s}{2}\ln\frac{1+s}{1-s} = \frac{t}{2}\ln\frac{2-t}{t} \sim -\frac{1}{2}t\ln t \to 0 $$
同时 $-\ln(1+s) - \ln\left(1-\frac{s^2}{2}\right) \to -\ln 2 - \ln\frac{1}{2} = 0$。因此 $\lim_{s\to 1^-} Q(s) = 0$。
    
综上所述,因为 $Q(s)$ 在 $[0,1)$ 上先增后减,且两端点的值(或极限)均为 0,故在整个区间 $[0,1)$ 上恒有 $Q(s) \ge 0$。由偶函数性质推知,在 $s\in(-1,1)$ 上恒有 $Q(s) \ge 0$,等号仅在 $s=0$ 时取得。将 $s=0$ 代回原变量得 $x=1$。

因此,在常规定义域 $x>0$ 内原不等式恒成立,解集为 $(0, +\infty)$,等号仅在 $x=1$ 处取到。若允许 $0^1=0$ 成立,则解集为 $[0, +\infty)$,等号在 $x=0$ 及 $x=1$ 处取到。
\end{solution}

\begin{example}{}{}
    对于一切$x\geq0$证明\[\ln(x+1)+\frac{x}{\sqrt{\frac{1}{2}x^2+1}}+\frac{1}{2}x^2-2x \geq 0\]
\end{example}
\begin{solution}
设函数 $f(x) = \ln(x+1) + \frac{x}{\sqrt{\frac{1}{2}x^2+1}} + \frac{1}{2}x^2 - 2x$。
由于对数函数的真数必须严格大于 $0$,故 $f(x)$ 的定义域为 $(-1, +\infty)$。
经代入检验,$f(0) = \ln(1) + 0 + 0 - 0 = 0$。

对 $f(x)$ 求导,可得:
$$f'(x) = \frac{1}{x+1} + \frac{1 \cdot \sqrt{\frac{1}{2}x^2+1} - x \cdot \frac{x}{2\sqrt{\frac{1}{2}x^2+1}}}{\frac{1}{2}x^2+1} + x - 2$$
化简中间的无理项,得一阶导数表达式为:
$$f'(x) = \frac{1}{x+1} + \frac{1}{\left(\frac{1}{2}x^2+1\right)^{\frac{3}{2}}} + x - 2$$

为判断 $f'(x)$ 的符号,先引入并证明以下引理:
对于任意实数 $y \ge 0$,恒有 $(1+y)^{\frac{3}{2}} \le 1 + \frac{3}{2}y + \frac{3}{8}y^2$。
证明如下:因 $y \ge 0$,不等式两侧均大于 $0$,对两侧同时平方,比较大小即可。
左式的平方为:$\left( (1+y)^{\frac{3}{2}} \right)^2 = 1 + 3y + 3y^2 + y^3$;
右式的平方为:$\left( 1 + \frac{3}{2}y + \frac{3}{8}y^2 \right)^2 = 1 + \frac{9}{4}y^2 + \frac{9}{64}y^4 + 3y + \frac{3}{4}y^2 + \frac{9}{8}y^3 = 1 + 3y + 3y^2 + \frac{9}{8}y^3 + \frac{9}{64}y^4$。
两式相减可得:
$$\left( 1 + \frac{3}{2}y + \frac{3}{8}y^2 \right)^2 - \left( (1+y)^{\frac{3}{2}} \right)^2 = \frac{1}{8}y^3 + \frac{9}{64}y^4$$
由于 $y \ge 0$,该差值显然恒大于等于 $0$,故引理成立。

对于任意实数 $x$,均有 $\frac{1}{2}x^2 \ge 0$。令 $y = \frac{1}{2}x^2$,代入上述引理可得:
$$\left(1+\frac{1}{2}x^2\right)^{\frac{3}{2}} \le 1 + \frac{3}{4}x^2 + \frac{3}{32}x^4$$
对该不等式两侧取倒数,得:
$$\frac{1}{\left(\frac{1}{2}x^2+1\right)^{\frac{3}{2}}} \ge \frac{1}{1 + \frac{3}{4}x^2 + \frac{3}{32}x^4} = \frac{32}{3x^4 + 24x^2 + 32}$$
将此下界代回一阶导数 $f'(x)$ 的表达式中,并对前两项通分:
$$f'(x) \ge \frac{1}{x+1} + x - 2 + \frac{32}{3x^4 + 24x^2 + 32} = \frac{x^2 - x - 1}{x+1} + \frac{32}{3x^4 + 24x^2 + 32}$$

进一步将右侧通分,公分母为 $(x+1)(3x^4 + 24x^2 + 32)$。在定义域 $x > -1$ 内,该分母恒大于 $0$。
记分子为 $N(x)$,展开并合并同类项:
$$N(x) = (x^2 - x - 1)(3x^4 + 24x^2 + 32) + 32(x+1)$$
$$= (3x^6 + 24x^4 + 32x^2) - (3x^5 + 24x^3 + 32x) - (3x^4 + 24x^2 + 32) + 32x + 32$$
$$= 3x^6 - 3x^5 + 21x^4 - 24x^3 + 8x^2$$
提取 $x^2$,得 $N(x) = x^2(3x^4 - 3x^3 + 21x^2 - 24x + 8)$。

设 $H(x) = 3x^4 - 3x^3 + 21x^2 - 24x + 8$,对其进行配方:
$$H(x) = 3\left(x^4 - x^3 + \frac{1}{4}x^2\right) - \frac{3}{4}x^2 + 21x^2 - 24x + 8$$
$$= 3\left(x^2 - \frac{1}{2}x\right)^2 + \frac{81}{4}x^2 - 24x + 8$$
$$= 3\left(x^2 - \frac{1}{2}x\right)^2 + \frac{81}{4}\left(x^2 - \frac{32}{27}x\right) + 8$$
$$= 3\left(x^2 - \frac{1}{2}x\right)^2 + \frac{81}{4}\left[ \left(x - \frac{16}{27}\right)^2 - \frac{256}{729} \right] + 8$$
$$= 3\left(x^2 - \frac{1}{2}x\right)^2 + \frac{81}{4}\left(x - \frac{16}{27}\right)^2 - \frac{64}{9} + 8$$
$$= 3\left(x^2 - \frac{1}{2}x\right)^2 + \frac{81}{4}\left(x - \frac{16}{27}\right)^2 + \frac{8}{9}$$
由于实数的平方非负,且 $\frac{8}{9} > 0$,因此对于任意实数 $x$,恒有 $H(x) \ge \frac{8}{9} > 0$。

综上所述,当 $x > -1$ 时,分子 $N(x) = x^2 H(x) \ge 0$(当且仅当 $x=0$ 时等号成立),分母恒正。
因此 $f'(x) \ge 0$ 恒成立,且 $f'(x)$ 仅在 $x=0$ 处为 $0$。
这表明原函数 $f(x)$ 在 $(-1, +\infty)$ 上严格单调递增。
又因 $f(0) = 0$,故当且仅当 $x \ge 0$ 时,有 $f(x) \ge f(0) = 0$。
从而使原不等式成立的解集为 $\{x \mid x \ge 0\}$。解答完毕。
\end{solution}


\begin{example}{}{}
    证明函数 $f(x) = \frac{[\ln(x+1)]^x}{x^{\ln x}}$ 在 $x \in (0, +\infty)$ 上严格单调递增
\end{example}
\begin{solution}
由于 $f(x) > 0$,转化为证明 $g(x) = \ln f(x)$ 单调递增,即证 $g'(x) > 0$ 恒成立。对 $f(x)$ 取自然对数得到 $g(x) = x \ln(\ln(x+1)) - (\ln x)^2$。对其求导可得:
$$g'(x) = \ln(\ln(x+1)) + \frac{x}{(x+1)\ln(x+1)} - \frac{2\ln x}{x}$$

作变量代换,令 $y = \ln(x+1)$。由于 $x > 0$,必有 $y > 0$ 且 $x = e^y - 1$。将 $g'(x)$ 转化为只关于 $y$ 的函数 $P(y)$:
$$P(y) = g'(e^y - 1) = \ln y + \frac{1 - e^{-y}}{y} - \frac{2\ln(e^y - 1)}{e^y - 1}$$
要证 $g'(x) > 0$,即证对于所有 $y > 0$ 均有 $P(y) > 0$。

利用双曲正弦函数恒等式 $e^y - 1 = 2e^{y/2}\sinh(y/2) = y e^{y/2} \frac{\sinh(y/2)}{y/2}$,对其两边取自然对数得:
$$\ln(e^y - 1) = \ln y + \frac{y}{2} + \ln\left(\frac{\sinh(y/2)}{y/2}\right)$$
引入引理:对于任意 $u > 0$,恒有 $\ln\left(\frac{\sinh u}{u}\right) < \frac{u^2}{6}$。
证明如下:设 $h(u) = \frac{u^2}{6} - \ln\left(\frac{\sinh u}{u}\right)$,易知 $h(0^+) = 0$。考察其导数 $h'(u) = \frac{1}{u}\left(1 + \frac{u^2}{3} - u \coth u\right)$。利用 $\coth u$ 的无穷部分分式展开式 $\coth u = \frac{1}{u} + \sum_{k=1}^\infty \frac{2u}{u^2 + k^2\pi^2}$,两边同乘 $u$ 得到 $u \coth u = 1 + \sum_{k=1}^\infty \frac{2u^2}{u^2 + k^2\pi^2}$。对于所有 $u > 0$,恒有 $\frac{2u^2}{u^2 + k^2\pi^2} < \frac{2u^2}{k^2\pi^2}$,故 $u \coth u < 1 + \sum_{k=1}^\infty \frac{2u^2}{k^2\pi^2} = 1 + \frac{2u^2}{\pi^2} \frac{\pi^2}{6} = 1 + \frac{u^2}{3}$。由此可知 $1 + \frac{u^2}{3} - u \coth u > 0$,从而 $h'(u) > 0$,即 $h(u) > 0$ 恒成立,引理得证。

在引理中代入 $u = y/2$,可得不等式 $\ln(e^y - 1) < \ln y + \frac{y}{2} + \frac{y^2}{24}$。将此上界代入 $P(y)$ 的表达式中,由于前置系数 $-\frac{2}{e^y-1}$ 为负,代入上界将构成 $P(y)$ 的严格下界 $B(y)$:
$$P(y) > \ln y + \frac{1 - e^{-y}}{y} - \frac{2(\ln y + y/2 + y^2/24)}{e^y - 1} = \frac{e^y - 3}{e^y - 1}\ln y + \frac{1 - e^{-y}}{y} - \frac{y + y^2/12}{e^y - 1} \equiv B(y)$$
为证 $B(y) > 0$,注意到 $\frac{e^y - 3}{e^y - 1}$ 在 $y = \ln 3$ 处变号,可将含 $\ln y$ 的项孤立,化归为研究函数 $Q(y)$:
$$Q(y) = \ln y - \frac{y^2 e^y + \frac{1}{12}y^3 e^y - (e^y - 1)^2}{y e^y (e^y - 3)}$$
证明 $B(y) > 0$ 等价于:当 $y > \ln 3$ 时,需证 $Q(y) > 0$;当 $0 < y < \ln 3$ 时,需证 $Q(y) < 0$。

对 $Q(y)$ 求导,得到:
$$Q'(y) = \frac{1}{y} - \frac{d}{dy}\left[ \frac{y^2 e^y + \frac{1}{12}y^3 e^y - (e^y - 1)^2}{y e^y (e^y - 3)} \right] = \frac{Z_1(y)}{y (e^y - 3)^2}$$
其中分子核心部分 $Z_1(y)$ 展开为:
$$Z_1(y) = 2 - 2y + 3y^2 + \frac{1}{2}y^3 + (3y + 3)e^{-y} - \left(1 + y + y^2 - \frac{5}{6}y^3 - \frac{1}{12}y^4\right)e^y$$
为分析 $Z_1(y)$ 的符号,同乘 $e^y$ 构建函数 $W(y) = e^y Z_1(y)$。由于对于 $y > 0$ 恒有 $e^y > 0$,$W(y)$ 的符号与 $Z_1(y)$ 及 $Q'(y)$ 的符号完全一致:
$$W(y) = \left(2 - 2y + 3y^2 + \frac{1}{2}y^3\right)e^y + 3y + 3 - \left(1 + y + y^2 - \frac{5}{6}y^3 - \frac{1}{12}y^4\right)e^{2y}$$

对 $W(y)$ 进行连续求导以判断其单调性与根的分布。
一阶导数:
$$W'(y) = \left(\frac{1}{2}y^3 + \frac{9}{2}y^2 + 4y\right)e^y + 3 - \left(3 + 4y - \frac{1}{2}y^2 - 2y^3 - \frac{1}{6}y^4\right)e^{2y}$$
二阶导数:
$$W''(y) = \left(\frac{1}{2}y^3 + 6y^2 + 13y + 4\right)e^y - \left(10 + 7y - 7y^2 - \frac{14}{3}y^3 - \frac{1}{3}y^4\right)e^{2y}$$
提取指数因子,令 $W_2(y) = W''(y)e^{-y}$:
$$W_2(y) = \frac{1}{2}y^3 + 6y^2 + 13y + 4 - \left(10 + 7y - 7y^2 - \frac{14}{3}y^3 - \frac{1}{3}y^4\right)e^y$$
对 $W_2(y)$ 继续求导,记为 $W_3(y)$:
$$W_2'(y) = \frac{3}{2}y^2 + 12y + 13 - \left(17 - 7y - 21y^2 - 6y^3 - \frac{1}{3}y^4\right)e^y \equiv W_3(y)$$
再对 $W_3(y)$ 求导两次:
$$W_3'(y) = 3y + 12 - \left(10 - 49y - 39y^2 - \frac{22}{3}y^3 - \frac{1}{3}y^4\right)e^y$$
$$W_3''(y) = 3 + \left(39 + 127y + 61y^2 + \frac{26}{3}y^3 + \frac{1}{3}y^4\right)e^y$$

观察 $W_3''(y)$ 可见,由于 $y > 0$,其展开项的所有系数均为正,故 $W_3''(y) > 0$ 恒成立。由此进行逆推分析:
由于 $W_3''(y) > 0$,$W_3'(y)$ 严格单调递增。由 $W_3'(0) = 2 > 0$ 可知 $W_3'(y) > 0$ 恒成立。
由于 $W_3'(y) > 0$,$W_3(y)$(即 $W_2'(y)$)严格单调递增。已知 $W_2'(0) = -4 < 0$ 且当 $y \to +\infty$ 时趋于正无穷,故 $W_2'(y)$ 在 $(0, +\infty)$ 上存在唯一根 $\alpha$。
因此,$W_2(y)$ 在 $(0, \alpha)$ 上单调递减,在 $(\alpha, +\infty)$ 上单调递增。由 $W_2(0) = -6 < 0$ 且趋于正无穷,可知 $W_2(y)$ 存在唯一根 $\beta$。
由于 $W''(y) = e^y W_2(y)$,$W''(y)$ 的符号与 $W_2(y)$ 相同,即先负后正。考虑到 $W'(0) = 0$,必有 $W'(y)$ 先减小至负数再递增,穿过 $x$ 轴于唯一根 $\gamma$。
由此推知 $W(y)$ 在 $(0, +\infty)$ 上先单调递减后单调递增,存在唯一极小值。

计算端点与特定值:起点 $W(0) = 4 > 0$,$W(\ln 3) \approx -0.052 < 0$,且当 $y \to +\infty$ 时 $W(y) \to +\infty$。根据零点定理,$W(y)$ 在 $(0, +\infty)$ 上恰有两个根:$y_1 \in (0, \ln 3)$ 和 $y_2 \in (\ln 3, +\infty)$。
因为 $Q'(y)$ 与 $W(y)$ 同号,可知 $Q(y)$ 的单调性分布为:在 $(0, y_1)$ 单调递增,在 $(y_1, \ln 3)$ 单调递减;在 $(\ln 3, y_2)$ 单调递减,在 $(y_2, +\infty)$ 单调递增。
结合渐近线行为 $Q(0^+) \to -\infty$ 和 $Q(\ln 3^-) \to -\infty$,可知区间 $(0, \ln 3)$ 上的极大值 $Q(y_1)$ 必定小于 $0$,故当 $y \in (0, \ln 3)$ 时 $Q(y) < 0$ 恒成立。
同理,结合 $Q(\ln 3^+) \to +\infty$ 和 $Q(+\infty) \to +\infty$,可知区间 $(\ln 3, +\infty)$ 上的极小值 $Q(y_2)$ 必定大于 $0$,故当 $y \in (\ln 3, +\infty)$ 时 $Q(y) > 0$ 恒成立。

综上所述,不等式 $B(y) > 0$ 在 $y \in (0, +\infty)$ 上恒成立。这等价于 $P(y) > 0$,即证明了导数 $g'(x) > 0$ 在 $(0, +\infty)$ 上恒成立。因此,$f(x) = \frac{[\ln(x+1)]^x}{x^{\ln x}}$ 在 $(0, +\infty)$ 上严格单调递增。证明完毕。
\end{solution}

\begin{example}{}{}
给定 $a > 0$。对每个正整数 $n$,定义
$$f_n(x) = (a + n - 1)^x + (a + n)^x - (a + n + 1)^x.$$
记其零点为 $x_n$(即 $f_n(x_n) = 0$)。设数列 $\{x_n\}$ 的前 $n$ 项和为
$$
S_n = \sum_{j=1}^n x_j.$$
固定题目中的正整数 $n$,若
$$\begin{cases}
S_n + S_{3n} + S_{8n} + S_{12n} - S_{4n} - S_{6n} - S_{9n} - S_{11n} = m, \\
S_{2n} + S_{4n} + S_{7n} + S_{11n} - S_n - S_{5n} - S_{8n} - S_{10n} = s, \\
S_{3n} + S_{5n} + S_{10n} - S_{2n} - S_{7n} - S_{9n} = r,
\end{cases}$$
证明:$s^2 < mr$。
\end{example}
\begin{solution}
对给定的正整数 $n$ 与常数 $a>0$,令 $b=a+n>1$。方程 $f_n(x)=0$ 等价于 $(b-1)^x+b^x=(b+1)^x$。将其两边同除以 $(b+1)^x$,得到 $(\frac{b-1}{b+1})^x+(\frac{b}{b+1})^x=1$。构造函数 $g(x)=(\frac{b-1}{b+1})^x+(\frac{b}{b+1})^x$。因为底数 $\frac{b-1}{b+1}, \frac{b}{b+1} \in (0,1)$,其导数 $g'(x)<0$,故 $g(x)$ 在 $[0,\infty)$ 上严格递减。结合 $g(0)=2>1$ 及 $\lim_{x\to\infty}g(x)=0<1$,可知方程 $g(x)=1$ 存在唯一的正实数解,即 $f_n(x)$ 的零点 $x_n$ 存在且唯一。

定义序列的分块和 $A_k = S_{kn} - S_{(k-1)n} = \sum_{j=(k-1)n+1}^{kn} x_j$($k=1,2,\dots,12$)。由定义有 $S_{kn} = \sum_{i=1}^k A_i$。将题设中 $m, s, r$ 的关系式用 $A_k$ 展开并作代数恒等变形,可得:
$$m=A_{12}-A_{2}-A_{3}-2A_{4}-A_{5}-A_{6}-A_{9},$$
$$s=A_{2}-A_{5}-A_{8}+A_{11},$$
$$r=A_{3}-A_{6}-A_{7}+A_{10}.$$

进一步定义序列 $\{A_k\}$ 的二阶差分 $\mu_k = -(A_{k+2}-2A_{k+1}+A_k)$。由离散连续两次求和公式,可得展开式 $A_k = A_1 + (k-1)(A_2-A_1) - \sum_{j=1}^{k-2}(k-1-j)\mu_j$(对于 $k \ge 2$)。将此关系代入 $s$ 与 $r$ 的表达式中化简,得到:
$$-s=\sum_{k=1}^{10}S_{k}\mu_{k}, \quad -r=\sum_{k=1}^{10}R_{k}\mu_{k},$$
其中相应的常数系数序列为 $(S_{k})_{k=1}^{10}=(0,1,2,3,3,3,3,2,1,0)$, $(R_{k})_{k=1}^{10}=(0,0,1,2,3,3,2,1,0,0)$。
同理,对 $m$ 展开并合并同类项,可得:
$$-m = \sum_{k=1}^{10}M_{k}\mu_{k} + 6A_1 + 15(A_{12}-A_{11}),$$
其中 $(M_{k})_{k=1}^{10}=(6,11,15,17,18,18,18,18,17,16)$。由于方程的根函数随参数单调递增,故序列 $\{x_j\}$ 递增,从而分块和序列 $\{A_k\}$ 单调递增,保证了仿射项 $6A_1+15(A_{12}-A_{11}) \ge 0$。由此得出不等式 $-m \ge \sum_{k=1}^{10}M_{k}\mu_{k}$。令 $M = \sum_{k=1}^{10}M_{k}\mu_{k}$, $S = -s$, $R = -r$。若能证明 $S^2 < MR$ 且 $R>0$,结合 $-m \ge M$ 便可推导出 $s^2 < mr$。

为了考察 $\mu_k$ 的性质,将根扩展为定义在 $t>1$ 上的连续变量函数 $X(t)$,其满足 $(t-1)^{X(t)}+t^{X(t)}=(t+1)^{X(t)}$。由于 $X(t)$ 满足 Bernstein 函数的性质,根据 Lévy-Khintchine 表示定理,存在常数 $c_0, c_1 \ge 0$ 及 $(0,\infty)$ 上的正测度 $\Pi$ 使得:
$$X(x)=c_{0}+c_{1}x+\int_{0}^{\infty}(1-e^{-xs})d\Pi(s).$$
对于固定步长 $h>0$ 与起点 $b>0$,抽样令 $a_k = X(b+kh)$。计算其二阶差分,并利用差分与积分的交换性,得到:
$$-\Delta^{2}a_{k}=\int_{0}^{\infty}e^{-(b+kh)s}(1-e^{-hs})^{2}d\Pi(s).$$
作积分变量代换 $t=e^{-hs} \in (0,1)$,将测度推前至 $[0,1]$,可知 $-\Delta^{2}a_{k}$ 构成 Hausdorff 矩序列。应用到本问题中,固定分块长度 $n$,对每个块内偏移 $j=1,\dots,n$,定义 $a_k^{(j)} = X(a+j+(k-1)n)$。由前述结论,存在 $[0,1]$ 上的正测度 $\nu_j$ 使得 $-\Delta^2 a_k^{(j)} = \int_0^1 t^{k-1} d\nu_j(t)$。因 $A_k = \sum_{j=1}^n a_k^{(j)}$,令总测度 $\nu = \sum_{j=1}^n \nu_j$,由差分算子的线性性可得:
$$\mu_{k}=-\Delta^{2}A_{k}=\int_{0}^{1}t^{k-1}d\nu(t) \quad (k \ge 1).$$

基于该积分表示,可将 $M, S, R$ 写为关于测度 $\nu$ 的积分式:
$$M=\int_{0}^{1}M(t)d\nu(t), \quad S=\int_{0}^{1}S(t)d\nu(t), \quad R=\int_{0}^{1}R(t)d\nu(t),$$
其中被积函数对应的多项式分别为:
$$M(t)=6+11t+15t^{2}+17t^{3}+18t^{4}+18t^{5}+18t^{6}+18t^{7}+17t^{8}+16t^{9},$$
$$S(t)=t(t+1)(t^{2}-t+1)(t^{2}+t+1)^{2},$$
$$R(t)=t^{2}(t+1)(t^{2}+1)(t^{2}+t+1).$$

接下来分析这些多项式的点态正定性。设 $D(t) = M(t)R(t) - S(t)^2$,经代数分解化简得:
$$D(t)=t^{2}(t+1)(t^{2}+t+1)P(t),$$
其中 $P(t)=15t^{11}+15t^{10}+30t^{9}+30t^{8}+29t^{7}+28t^{6}+27t^{5}+26t^{4}+23t^{3}+17t^{2}+9t+5$。当 $t \in (0,1)$ 时,由于 $P(t)$ 的各项系数均为正,故 $P(t)>0$,且 $t^2(t+1)(t^2+t+1)>0$ 显然成立。因此对于任意 $t \in (0,1)$,恒有 $M(t)R(t)-S(t)^2>0$。

上述严格不等式说明,对于每个 $t \in (0,1)$,矩阵 $\begin{pmatrix}M(t)&S(t)\\ S(t)&R(t)\end{pmatrix}$ 严格正定。据此可构造向量函数:
$$u(t)=(\sqrt{M(t)},0), \quad v(t)=\left(\frac{S(t)}{\sqrt{M(t)}},\sqrt{R(t)-\frac{S(t)^{2}}{M(t)}}\right),$$
其满足 $|u(t)|^2 = M(t)$, $|v(t)|^2 = R(t)$,且 $u(t) \cdot v(t) = S(t)$。由积分形式的 Cauchy-Schwarz 不等式,有:
$$S^{2}=\left(\int_{0}^{1}u(t)\cdot v(t)d\nu(t)\right)^{2} \le \left(\int_{0}^{1}|u(t)|^{2}d\nu(t)\right)\left(\int_{0}^{1}|v(t)|^{2}d\nu(t)\right)=MR.$$
该不等式等号成立的条件是 $u(t)$ 与 $v(t)$ 在 $\nu$ 测度下几乎处处共线,即要求 $M(t)R(t)-S(t)^2=0$ 几乎处处成立。但如前所证,在积分区域 $(0,1)$ 上 $M(t)R(t)-S(t)^2>0$ 严格成立,故等号不可取,从而 $S^2 < MR$。代回变量 $M \le -m$、$S=-s$、$R=-r$,最终得到 $s^2 < mr$。证明完毕。
\end{solution}

\begin{example}{}{}
已知函数 $f(x)=\sqrt{\frac{x^{a}}{a}}+\sqrt{\frac{a^{x}}{x}}-2$ 其中 $a>0$, $x>0$. \\
(I)当 $a=1$ 时,讨论函数 $f(x)$ 的单调性;\\
(II)证明: $f(x)\ge0$
\end{example}
\begin{solution}
当 $a=1$ 时, 函数化简为 $f(x)=\sqrt{x}+\frac{1}{\sqrt{x}}-2$. 对其求导可得:
$$f^{\prime}(x)=\frac{1}{2\sqrt{x}}-\frac{1}{2\sqrt{x^{3}}}=\frac{x-1}{2\sqrt{x^{3}}}.$$
易知当 $x \in (0,1)$ 时, $f^{\prime}(x)<0$, 函数 $f(x)$ 单调递减;当 $x \in (1,+\infty)$ 时, $f^{\prime}(x)>0$, 函数 $f(x)$ 单调递增.

对于 $f(x) \ge 0$ 的证明, 首先利用基本不等式对原函数进行放缩, 有:
$$f(x)\ge2\sqrt{\sqrt{\frac{x^{a}}{a}}\cdot\sqrt{\frac{a^{x}}{x}}}-2=2\sqrt[4]{x^{a-1}a^{x-1}}-2.$$
当 $a>1$ 且 $x>1$, 或 $0<a<1$ 且 $0<x<1$ 时, 均有 $x^{a-1}a^{x-1}>1$, 此时 $f(x)>0$ 显然成立. 由 $a=1$ 时已证的单调性可知 $f(x)\ge f(1)=0$. 结合表达式关于 $a$ 和 $x$ 的对称性, 后续只需证明当 $0<a<1<x$ 时的情形.

为便于处理指数形式, 考虑通过换元分离变量. 令 $x=p^{2}$, $a=q^{2}$, 其中 $p>1>q>0$, 则原不等式转化为证明:
\[\frac{p^{q^{2}}}{q}+\frac{q^{p^{2}}}{p}\ge2\]
首先对 $p^{q^{2}}$ 进行放缩. 构造不等式 $p^{q^{2}}\ge1+q\left(\frac{p}{p+q-pq}-1\right)$, 两边取对数, 即证 $q^{2}\ln p\ge \ln\left[1+q\left(\frac{p}{p+q-pq}-1\right)\right]$.
设 $g(p)=q^{2}\ln p-\ln\left[1+q\left(\frac{p}{p+q-pq}-1\right)\right]$, 对其求导可得:
$$g^{\prime}(p)=\frac{q^{2}}{p}-\frac{q\frac{(p+q-pq)-p(1-q)}{(p+q-pq)^{2}}}{1+q\left(\frac{p}{p+q-pq}-1\right)}=\frac{q^{2}}{p}-\frac{q^{2}}{(p+q-pq)^{2}+(p+q-pq)(pq^{2}-q^{2})}$$
$$=\frac{q^{2}(1-q)(p-1)[q^{2}(p-1)+p(1-q)]}{p(p+q-pq)(pq^{2}-q^{2}-pq+p+q)}.$$
由于 $p>1>q>0$, 易知 $1-q>0$ 且 $p-1>0$, 导函数表达式中各项因式均大于零, 故在 $(1,+\infty)$ 上恒有 $g^{\prime}(p)>0$. 因此 $g(p)>g(1)=0$, 上述关于 $p^{q^{2}}$ 的放缩不等式成立.

同理, 对 $q^{p^{2}}$ 进行放缩, 验证 $q^{p^{2}}\ge1+p\left(\frac{q}{p+q-pq}-1\right)$ 是否成立. 因不等式左侧恒正, 仅需考虑右侧大于零的情形, 即 $1+p\left(\frac{q}{p+q-pq}-1\right)>0$, 解得 $\frac{p^{2}-p}{p^{2}-p+1}<q<\frac{p}{p-1}$.
设 $h(q)=p^{2}\ln q-\ln\left[1+p\left(\frac{q}{p+q-pq}-1\right)\right]$, 求导得:
$$h^{\prime}(q)=\frac{p^{2}(p-1)(1-q)[p^{2}(q-1)+q(1-p)]}{q(p+q-pq)(p^{2}q-p^{2}-pq+p+q)}.$$
由 $p>1>q>\frac{p^{2}-p}{p^{2}-p+1}$ 判定各因式符号, 可知在该区间内恒有 $h^{\prime}(q)<0$. 从而 $h(q)>h(1)=0$, 该不等式亦成立.

综合上述两项放缩结果, 可得:
$$\frac{p^{q^{2}}}{q}+\frac{q^{p^{2}}}{p}-2\ge \frac{1}{q}\left[1+q\left(\frac{p}{p+q-pq}-1\right)\right] + \frac{1}{p}\left[1+p\left(\frac{q}{p+q-pq}-1\right)\right] - 2$$
$$=\frac{1}{p}+\frac{1}{q}+\frac{p+q}{p+q-pq}-4=\frac{p+q}{pq}+\frac{p+q}{p+q-pq}-4$$
$$=\frac{(p+q)(p+q-pq)+(p+q)pq-4pq(p+q-pq)}{pq(p+q-pq)}$$
$$=\frac{p^{2}+q^{2}+4p^{2}q^{2}+2pq-4p^{2}q-4pq^{2}}{pq(p+q-pq)}=\frac{(p+q-2pq)^{2}}{pq(p+q-pq)}\ge0.$$
至此, 原不等式得证.

此外, 本题也可通过琴生不等式进行证明. 当 $a\ne1$ 时, 原不等式等价于证明:
$$\frac{x}{x+a}x^{\frac{a-1}{2}}+\frac{a}{x+a}a^{\frac{x-1}{2}}\ge\frac{2\sqrt{xa}}{x+a}.$$
由琴生不等式可得:
$$\frac{x}{x+a}x^{\frac{a-1}{2}}+\frac{a}{x+a}a^{\frac{x-1}{2}}\ge e^{\frac{x(a-1)\ln x+a(x-1)\ln a}{2(x+a)}},$$
因此只需证明 $\frac{x(a-1)\ln x+a(x-1)\ln a}{2(x+a)}\ge \ln\frac{2\sqrt{xa}}{x+a}$.
由对称性不妨设 $t=\frac{a}{x}\ge1$, 则上式等价于:
$$(x-1)t\ln(tx)+(tx-1)\ln x+2(t+1)\ln\frac{t+1}{2\sqrt{t}}\ge0.$$
记左侧函数为 $G(x)$, 则 $G^{\prime}(x)=t\ln t+2t+2t\ln x-\frac{t+1}{x}$. 易知 $G^{\prime}(x)$ 单调递增, 且 $G^{\prime}(1)=t\ln t+t-1\ge0$, $G^{\prime}\left(\frac{1}{t}\right)=t-t^{2}-t\ln t\le0$. 故存在唯一的 $x_{0}\in\left[\frac{1}{t},1\right]$ 使 $G^{\prime}(x_{0})=0$.
从而有:
$$G(x)\ge G(x_{0})=t+1-2tx_{0}-t\ln t+2(t+1)\ln\frac{t+1}{2\sqrt{tx_{0}}}.$$
由 $G^{\prime}(x_{0})=0$ 结合对数不等式放缩有 $2t\left(1+\ln(\sqrt{t}x_{0})-\frac{t+1}{2tx_{0}}\right)\ge2t\left(2-\frac{1}{\sqrt{t}x_{0}}-\frac{t+1}{2tx_{0}}\right)$, 可推得 $4tx_{0}\le t+2\sqrt{t+1}$.
将其代入 $G(x_{0})$ 的表达式中放缩得:
$$G(x_{0})\ge\frac{t+1-2\sqrt{t}}{2}-t\ln t+2(t+1)\ln\frac{t+1}{\sqrt{t}+1}.$$
为书写简便,将 $\sqrt{t}$ 记为 $u$ (其中 $u\ge1$), 构造函数 $H(u)=\frac{u^{2}+1-2u}{2}-2u^{2}\ln u+2(u^{2}+1)\ln\frac{u^{2}+1}{u+1}$, 现只需证当 $u\ge1$ 时 $H(u)\ge0$.
求导得 $H^{\prime}(u)=4u\left[\frac{u^{2}+2u-3}{4u(u+1)}+\ln\frac{u^{2}+1}{u^{2}+u}\right]$. 记 $\varphi(u)=\frac{u^{2}+2u-3}{4u(u+1)}+\ln\frac{u^{2}+1}{u^{2}+u}$, 则:
$$\varphi^{\prime}(u)=\frac{(u-1)^{2}(3u^{2}+8u+3)}{4u^{2}(u^{2}+1)(u+1)^{2}}\ge0.$$
因此 $\varphi(u)$ 在 $[1, +\infty)$ 上单调递增, $\varphi(u)\ge\varphi(1)=0$, 进而 $H^{\prime}(u)\ge0$. 故 $H(u)$ 在 $[1,+\infty)$ 上单调递增, $H(u)\ge H(1)=0$, 原不等式得证.

(注:该不等式可进一步推广加强为:对任意的 $x>0, y>0$, 均有 $\sqrt{\frac{x^{y}}{y}}\ge\frac{x-1}{x+1}-\frac{y-1}{y+1}+1$.)
\end{solution}


\begin{example}{}{}
设函数 $f(x)=e^{ax}+e^{bx}-x$,且 $a>0$,$b>0$\\
(1) 证明:$f(x)$ 在 $\mathbf{R}$ 上不单调\\
(2) 若 $f(x)$ 有且仅有一个零点 $x_{0}$. \\
(i) 证明:$e+1<x_{0}\leqslant 2e$~~~(ii) 当 $a=(e+2)b$ 时,求 $x_{0}$.
\end{example}
\begin{solution}
(1)已知函数 $f(x) = e^{ax} + e^{bx} - x$,对其求导得 $f'(x) = ae^{ax} + be^{bx} - 1$,再次求导得 $f''(x) = a^2e^{ax} + b^2e^{bx}$。因为 $a > 0$ 且 $b > 0$,对任意 $x \in \mathbf{R}$ 都有 $f''(x) > 0$,故 $f'(x)$ 在 $\mathbf{R}$ 上严格单调递增。
又因 $\lim_{x \to -\infty} f'(x) = -1 < 0$,$\lim_{x \to +\infty} f'(x) = +\infty > 0$,根据零点存在定理,必存在唯一的实数 $x_1$ 使得 $f'(x_1) = 0$。当 $x < x_1$ 时,$f'(x) < 0$,$f(x)$ 单调递减;当 $x > x_1$ 时,$f'(x) > 0$,$f(x)$ 单调递增。因此 $f(x)$ 存在单调递减区间和单调递增区间,在 $\mathbf{R}$ 上不单调。\\
(2)(i)由(1)可知,$f(x)$ 在 $x = x_1$ 处取得全局最小值。由于 $f(x)$ 有且仅有一个零点 $x_0$,该零点必为函数的最小值点,即 $x_0 = x_1$。于是有 $f(x_0) = 0$ 且 $f'(x_0) = 0$,即 $e^{ax_0} + e^{bx_0} = x_0$ 且 $ae^{ax_0} + be^{bx_0} = 1$。令 $u = ax_0$,$v = bx_0$。由 $x_0 = e^{ax_0} + e^{bx_0} > 0$ 及 $a, b > 0$ 可知 $u, v > 0$。上述方程组可重写为 $e^u + e^v = x_0$ 与 $ue^u + ve^v = x_0$,消去 $x_0$ 后得到核心关系式 $(u-1)e^u + (v-1)e^v = 0$。
构造函数 $h(t) = (t-1)e^t \ (t > 0)$,因 $h'(t) = te^t > 0$,故 $h(t)$ 在 $(0, +\infty)$ 上严格单调递增,且 $h(1) = 0$。由 $h(u) + h(v) = 0$,我们不妨设 $v \leqslant u$,必有 $h(v) \leqslant 0 \leqslant h(u)$,从而推知 $0 < v \leqslant 1 \leqslant u$(当且仅当 $u=v=1$ 时取等号)。
由于 $u \geqslant 1$ 且 $v > 0$,故 $x_0 = e^u + e^v > e^1 + e^0 = e+1$。
另一方面,将 $x_0$ 视为关于 $v$ 的函数 $S(v) = e^u + e^v \ (0 < v \leqslant 1)$。对 $(u-1)e^u + (v-1)e^v = 0$ 两边关于 $v$ 求导,得 $u e^u \cdot u' + v e^v = 0$,即 $u' = -\frac{v e^v}{u e^u}$。从而 $S'(v) = e^u \cdot u' + e^v = e^v(1 - \frac{v}{u})$。因 $0 < v \leqslant 1 \leqslant u$,有 $\frac{v}{u} \leqslant 1$,即 $S'(v) \geqslant 0$,这说明 $S(v)$ 在 $(0, 1]$ 上单调递增。当 $v = 1$ 时(此时 $u = 1$)取得最大值,故 $x_0 \leqslant S(1) = 2e$。综上所述,$e+1 < x_0 \leqslant 2e$ 获证。\\
(ii)当 $a=(e+2)b$ 时,代入 $u = ax_0, v = bx_0$ 得到 $u = (e+2)v$。将其代入 $(u-1)e^u + (v-1)e^v = 0$ 中,方程两边同除以 $e^v$,化简得到 $((e+2)v - 1)e^{(e+1)v} + v - 1 = 0$。
令 $k(v) = ((e+2)v - 1)e^{(e+1)v} + v - 1 \ (v > 0)$,求导得 $k'(v) = e^{(e+1)v} [ (e+1)(e+2)v + 1 ] + 1$。显然当 $v > 0$ 时 $k'(v) > 0$ 恒成立,说明 $k(v)$ 在 $(0, +\infty)$ 上严格单调递增,若存在零点则必定唯一。
观察可知 $k\left(\frac{1}{e+1}\right) = \left(\frac{e+2}{e+1} - 1\right)e^1 + \frac{1}{e+1} - 1 = \frac{e}{e+1} - \frac{e}{e+1} = 0$,故 $v = \frac{1}{e+1}$ 为该方程的唯一解,此时 $u = (e+2)v = \frac{e+2}{e+1}$。
最后代回 $x_0$ 的表达式,解得 $x_0 = e^u + e^v = e^{\frac{e+2}{e+1}} + e^{\frac{1}{e+1}} = e^{\frac{1}{e+1}}(e + 1)$。
\end{solution}
\newpage
\begin{example}{来自“港”}{}
若$t>0$,证明: \[\e^{\frac{t+1+\ln t}{(t+1)^2}\mathrm{e}^{\frac{t\ln t-t-1}{t+1}}x}+\mathrm{e}^{\frac{t+1-t\ln t}{(t+1)^2}\mathrm{e}^{\frac{t\ln t-t-1}{t+1}}x}-x\geqslant0\]
\end{example}
\begin{solution}
证明:当 $t=1$ 时等号显然成立。下设 $t>0$ 且 $t \neq 1$。
引入 $x_1 = \frac{\ln t}{t-1}, x_2 = \frac{t\ln t}{t-1}$,以及
$$u = \frac{\ln t}{t+1}=\frac{x_1(x_2-x_1)}{x_1+x_2}, \quad v = \frac{t\ln t}{t+1}=\frac{x_2(x_2-x_1)}{x_1+x_2}$$
由此可得 $u + v = \frac{(t+1)\ln t}{t+1} = \ln t$,且 $\frac{v}{u} = t$。
考察原不等式的指数部分,有 $\frac{t\ln t - t - 1}{t+1} = \frac{t\ln t}{t+1} - 1 = v - 1$。
考察第一项与第二项的系数,注意到 $\frac{1}{t+1} = \frac{x_1}{x_1+x_2}$,则:
$$\frac{t+1+\ln t}{(t+1)^2} = \frac{1}{t+1} \left( 1 + \frac{\ln t}{t+1} \right) = \frac{x_1}{x_1+x_2}(1+u)$$
$$\frac{t+1-t\ln t}{(t+1)^2} = \frac{1}{t+1} \left( 1 - \frac{t\ln t}{t+1} \right) = \frac{x_1}{x_1+x_2}(1-v)$$
将上述关系代入原不等式 $\frac{t+1+\ln t}{(t+1)^2} \mathrm{e}^{\frac{t\ln t - t - 1}{t+1}x} + \frac{t+1-t\ln t}{(t+1)^2} \mathrm{e}^{\frac{t\ln t - t - 1}{t+1}x} - x \geqslant 0$,整理为关于 $x_1, x_2$ 的形式:
$$\mathrm{e}^{\frac{x_1}{x_1+x_2}(1+u)\mathrm{e}^{v-1}x} + \mathrm{e}^{\frac{x_1}{x_1+x_2}(1-v)\mathrm{e}^{v-1}x} - x \geqslant 0$$
观察到 $Y = \frac{x_1}{x_1+x_2}\mathrm{e}^{v-1}x$为多次出现的结构,解得
$$ x = \frac{x_1+x_2}{x_1} \mathrm{e}^{1-v} Y = (t+1)\mathrm{e}^{1-v} Y= (\mathrm{e}^{u+v} + 1)\mathrm{e}^{1-v} Y = (\mathrm{e}^{1+u} + \mathrm{e}^{1-v})Y$$
此时原不等式等价于:
\begin{align*}
&\mathrm{e}^{(1+u)Y} + \mathrm{e}^{(1-v)Y} - (\mathrm{e}^{1+u} + \mathrm{e}^{1-v})Y \geqslant 0 \Leftrightarrow\mathrm{e}^{(1+u)Y} + \mathrm{e}^{(1-v)Y} \geqslant  (\mathrm{e}^{1+u} + \mathrm{e}^{1-v})Y\\
\text{考虑}&\mathrm{e}^{(1+u)Y} + \mathrm{e}^{(1-v)Y} \geqslant \mathrm{e}^{1+u} + \mathrm{e}^{1-v} + (({1+u})\mathrm{e}^{1+u} + ({1-v})\mathrm{e}^{1-v})(Y-1),\text{切线放缩}\\
\Leftrightarrow&\mathrm{e}^{(1+u)Y} + \mathrm{e}^{(1-v)Y} \geqslant \mathrm{e}^{1+u} + \mathrm{e}^{1-v} + (\mathrm{e}^{1+u} + \mathrm{e}^{1-v} + \mathrm{e}(u\mathrm{e}^u - v\mathrm{e}^{-v}))(Y-1)\\
\Leftrightarrow&\mathrm{e}^{(1+u)Y} + \mathrm{e}^{(1-v)Y} \geqslant \mathrm{e}^{1+u} + \mathrm{e}^{1-v} + (\mathrm{e}^{1+u} + \mathrm{e}^{1-v} )(Y-1)\\
\Leftrightarrow&\mathrm{e}^{(1+u)Y} + \mathrm{e}^{(1-v)Y} \geqslant  (\mathrm{e}^{1+u} + \mathrm{e}^{1-v})Y
\end{align*}
于是得证。注:切线放缩是根据取等条件$Y=1$,先注意到取等条件再构造放缩。
\end{solution}


\begin{example}{虚调子}{}
    对于所有的 $x > 0$,均有
$$e^{-x} + e^{-\frac{1}{x}} \leq 1 - \frac{3\left(1 - \frac{2}{e}\right)}{1 + x + \frac{1}{x}}$$
\end{example}
\begin{solution}
观察原不等式,若将变量$x$替换为$\frac{1}{x}$,不等式左侧$e^{-\frac{1}{x}} + e^{-x}$与右侧分母$1 + \frac{1}{x} + x$均保持不变。因此,该不等式在变换$x \leftrightarrow \frac{1}{x}$下具有对称性。对于$x \in (0, 1)$的情形,可通过代换转化为$x > 1$的情形。故只需证明$x \ge 1$时的情形即可,$(0, 1)$区间可由对称性自然推导得出。

记常数$c = 3\left(1 - \frac{2}{e}\right)$。当$x \ge 1$时,不等式右侧分母通分为$1 + x + \frac{1}{x} = \frac{x^2+x+1}{x}$,且显然有$\frac{x^2+x+1}{x} > 0$。在不等式两边同乘$x^2+x+1$,得到等价不等式:
$$(x^2+x+1)\left(e^{-x} + e^{-\frac{1}{x}}\right) \leq x^2+x+1 - cx$$
移项后,构造辅助函数$P(x)$:
$$P(x) = (x^2+x+1)\left(1 - e^{-x} - e^{-\frac{1}{x}}\right) - cx$$
原命题即转化为证明:对于所有的$x \ge 1$,$P(x) \ge 0$恒成立。

检验端点$x = 1$的值:
$$P(1) = 3(1 - 2e^{-1}) - c = 3\left(1 - \frac{2}{e}\right) - 3\left(1 - \frac{2}{e}\right) = 0$$
由于$P(1) = 0$,为了证明$P(x) \ge 0$,只需证明$P(x)$在$[1, +\infty)$上严格单调递增,即只需证明其一阶导数$P'(x) > 0$对一切$x > 1$恒成立。

应用乘积求导法则,对$P(x)$求导:
$$P'(x) = (2x+1)\left(1 - e^{-x} - e^{-\frac{1}{x}}\right) + (x^2+x+1)\left(e^{-x} - \frac{1}{x^2}e^{-\frac{1}{x}}\right) - c$$
将其展开并按常数项、含$e^{-x}$的项、含$e^{-1/x}$的项分别合并同类项:
常数项为$2x+1 - c$;
含$e^{-x}$的项为$-(2x+1)e^{-x} + (x^2+x+1)e^{-x} = (x^2-x)e^{-x}$;
含$e^{-1/x}$的项为$-(2x+1)e^{-\frac{1}{x}} - \frac{x^2+x+1}{x^2}e^{-\frac{1}{x}} = -\left(2x+2 + \frac{1}{x} + \frac{1}{x^2}\right)e^{-\frac{1}{x}}$。
组合后可得导数表达式:
$$P'(x) = 2x+1 - c + (x^2-x)e^{-x} - \left(2x+2+\frac{1}{x}+\frac{1}{x^2}\right)e^{-\frac{1}{x}}$$

为便于判断符号,令$s = \frac{1}{x}$。由$x > 1$知$s \in (0, 1)$。代入$P'(x)$并提取公因子$\frac{1}{s}$:
$$P'\left(\frac{1}{s}\right) = \frac{1}{s} \left[ 2 + s - cs + \left(\frac{1}{s} - 1\right)e^{-\frac{1}{s}} - (2+2s+s^2+s^3)e^{-s} \right]$$
将$c = 3 - \frac{6}{e}$代入前三项化简:
$$2 + s - cs = 2 + s - \left(3 - \frac{6}{e}\right)s = 2(1-s) + \frac{6s}{e}$$
从而导数表达式化为:
$$P'\left(\frac{1}{s}\right) = \frac{1}{s} \left[ 2(1-s) + \frac{6s}{e} + \frac{1-s}{s}e^{-\frac{1}{s}} - (2+2s+s^2+s^3)e^{-s} \right]$$

观察方括号内的$\frac{1-s}{s}e^{-\frac{1}{s}}$一项。由于$s \in (0, 1)$,有$1-s > 0$、$s > 0$且$e^{-\frac{1}{s}} > 0$,故$\frac{1-s}{s}e^{-\frac{1}{s}} > 0$恒成立。为了证明$P'\left(\frac{1}{s}\right) > 0$,只需证明去掉该正项后的剩余部分大于等于$0$。
为此,构造辅助函数$f(s)$:
$$f(s) = 2(1-s) + \frac{6s}{e} - (2+2s+s^2+s^3)e^{-s}$$
下证对于$s \in (0, 1)$,$f(s) > 0$。

先计算$f(s)$在闭区间$[0, 1]$两端点的值:
$$f(0) = 2 - 2 = 0$$
$$f(1) = \frac{6}{e} - (2+2+1+1)e^{-1} = 0$$
对$f(s)$求一阶导数:
$$f'(s) = -2 + \frac{6}{e} - \left[ (2+2s+3s^2)e^{-s} - (2+2s+s^2+s^3)e^{-s} \right] = -2 + \frac{6}{e} + (s^3 - 2s^2)e^{-s}$$
继续对$f'(s)$求二阶导数:
$$f''(s) = (3s^2 - 4s)e^{-s} - (s^3 - 2s^2)e^{-s} = e^{-s} (-s^3 + 5s^2 - 4s)$$
提取因式$-s$进行分解:
$$f''(s) = -s(s^2 - 5s + 4)e^{-s} = -s(s-1)(s-4)e^{-s}$$
对于探讨的开区间$s \in (0, 1)$,有$-s < 0$、$s-1 < 0$、$s-4 < 0$。三个负数相乘结果为负,加上$e^{-s} > 0$,得出在$(0, 1)$内$f''(s) < 0$恒成立。

由$f''(s) < 0$可知,$f'(s)$在区间$[0, 1]$上严格单调递减。
考察$f'(s)$在边界处的值:
$$f'(0) = -2 + \frac{6}{e} > 0$$
$$f'(1) = -2 + \frac{6}{e} - e^{-1} = -2 + \frac{5}{e} < 0$$
由于$f'(s)$严格单调递减,且由正变负,根据零点定理,在$(0, 1)$内部存在唯一的点$s_0$,使得$f'(s_0) = 0$。
由此可知,在$[0, s_0]$上$f'(s) > 0$,函数$f(s)$严格单调递增;在$[s_0, 1]$上$f'(s) < 0$,函数$f(s)$严格单调递减。
结合$f(s)$全程先增后减的形态且两端$f(0) = f(1) = 0$,可知对于整个区间$(0, 1)$内部,严格有$f(s) > 0$成立。

回顾前述逻辑链,由$f(s) > 0$及$\frac{1-s}{s}e^{-\frac{1}{s}} > 0$恒成立,可推得$P'\left(\frac{1}{s}\right) > 0$对$s \in (0, 1)$严格成立。
还原变量$x = \frac{1}{s}$,得知导函数$P'(x) > 0$对所有$x > 1$恒成立,故原函数$P(x)$在$[1, +\infty)$上严格单调递增。结合$P(1) = 0$,必然有对于所有的$x \ge 1$均有$P(x) \ge 0$,等号当且仅当$x=1$时取得。
根据对称性,上述结论自然延拓至$0 < x < 1$的情形。

综上所述,不等式$e^{-x} + e^{-\frac{1}{x}} \leq 1 - \frac{3\left(1 - \frac{2}{e}\right)}{1 + x + \frac{1}{x}}$对于所有的$x > 0$均成立,等号当且仅当$x = 1$时取得,证明完毕。
\end{solution}


\begin{example}{}{}
\[\e^x\geqslant\frac{3\sqrt{\e}}{4}x+\frac{\sqrt{5\e}}{4}\sqrt{x^2+1}\]
\end{example}
\begin{solution}
原命题等价于证明:
\[
1 \geqslant \mathrm{e}^{\frac{1}{2}-x} \left(\frac{3}{4}x + \frac{\sqrt{5}}{4}\sqrt{x^2+1}\right)
\]
考虑到不等式中含有 $\sqrt{\mathrm{e}}$(即 $\mathrm{e}^{\frac{1}{2}}$),我们猜测等号可能在 $x=\frac{1}{2}$ 处取得。为了证明该结论,构造辅助函数:
\[
f(x) = \mathrm{e}^{\frac{1}{2}-x}\left(\frac{3}{4}x + \frac{\sqrt{5}}{4}\sqrt{x^{2}+1}\right)
\]
我们的目标转化为证明:对于任意实数 $x$,恒有 $f(x) \leqslant 1$。对 $f(x)$ 求导,应用乘法法则可得:
\[
\begin{aligned}
f'(x) &= -\mathrm{e}^{\frac{1}{2}-x}\left(\frac{3}{4}x+\frac{\sqrt{5}}{4}\sqrt{x^2+1}\right) + \mathrm{e}^{\frac{1}{2}-x}\left(\frac{3}{4} + \frac{\sqrt{5}x}{4\sqrt{x^2+1}}\right) \\
&= \frac{\mathrm{e}^{\frac{1}{2}-x}}{4\sqrt{x^2+1}} \left( -3x\sqrt{x^2+1} - \sqrt{5}(x^2+1) + 3\sqrt{x^2+1} + \sqrt{5}x \right) \\
&= \frac{\mathrm{e}^{\frac{1}{2}-x}}{4\sqrt{x^2+1}} \left[ \sqrt{x^2+1}(3-3x) - \sqrt{5}(x^2-x+1) \right]
\end{aligned}
\]
令 $f'(x) = 0$,只需方括号内部分为零,即:$\sqrt{x^2+1}(3-3x) = \sqrt{5}(x^2-x+1)$,注意到等式右侧 $x^2-x+1 = \left(x-\frac{1}{2}\right)^2 + \frac{3}{4} > 0$ 恒成立,这就要求等式左侧必须满足 $3-3x > 0$,从而得到该方程成立的\textbf{隐含条件} $x < 1$。在满足 $x < 1$ 的前提下,将上述等式两边平方以消除根号:
\[
(x^2+1)(3-3x)^2 = 5(x^2-x+1)^2\Rightarrow 5(x^2-x+1)^2 - 9(1-x)^2(x^2+1) = 0
\]
展开整理后,可以得到一个关于 $x$ 的一元四次方程:
\[
4x^4 - 8x^3 + 3x^2 - 8x + 4 = (2x-1)(x-2)(2x^2+x+2) = 0
\]
解得 $x = \frac{1}{2}$ 或 $x = 2$(二次因式 $2x^2+x+2=0$ 无实根)。结合前面的隐含条件 $x < 1$,平方产生的增根 $x=2$ 必须舍去,故导函数 $f'(x)$ 存在唯一的零点 $x = \frac{1}{2}$。接下来分析 $f'(x)$ 的符号:
\begin{itemize}
    \item 当 $x \in \left(-\infty, \frac{1}{2}\right)$ 时,$f'(x) > 0$,函数 $f(x)$ 单调递增;
    \item 当 $x \in \left(\frac{1}{2}, +\infty\right)$ 时,$f'(x) < 0$,函数 $f(x)$ 单调递减。
\end{itemize}
因此,$f(x)$ 在 $x = \frac{1}{2}$ 处取得全局最大值:
\[
f\left(\frac{1}{2}\right) = \mathrm{e}^{0}\left(\frac{3}{4} \cdot \frac{1}{2} + \frac{\sqrt{5}}{4}\sqrt{\frac{1}{4}+1}\right) = \frac{3}{8} + \frac{\sqrt{5}}{4} \cdot \frac{\sqrt{5}}{2} = \frac{3}{8} + \frac{5}{8} = 1
\]
即对任意实数 $x$,都有 $f(x) \leqslant f\left(\frac{1}{2}\right) = 1$,也就是:
\[
\mathrm{e}^{\frac{1}{2}-x}\left(\frac{3}{4}x + \frac{\sqrt{5}}{4}\sqrt{x^{2}+1}\right) \leqslant 1
\Leftrightarrow\mathrm{e}^x \geqslant \frac{3\sqrt{\mathrm{e}}}{4}x + \frac{\sqrt{5\mathrm{e}}}{4}\sqrt{x^2+1}
\]
证毕。
\end{solution}
\begin{example}{(2019年浙江导数)}{}
    对任意$x\geq \dfrac{1}{e^2}$均有$f(x)=a\ln x+\sqrt{1+x}-\dfrac{\sqrt{x}}{2a}\leq 0$,求$a$的取值范围.
\end{example}
\begin{solution}
    必要性探路,什么极点效应,内点效应啥的,都是一个函数与$x$轴相切的不同情形,所以我们可以直接研究$f(x)$与$x$轴的相切问题,从而规避使用端点效应带来的潜在风险。
    \[\begin{cases}f(x)=a\ln x+\sqrt{1+x}-\dfrac{\sqrt{x}}{2a}=0\\[1.2ex]f'(x)=\dfrac{a}{x}+\dfrac{1}{2\sqrt{1+x}}-\dfrac{1}{4a\sqrt{x}}=0\end{cases}
    \Rightarrow \begin{cases}a\ln x+\sqrt{1+x}-\dfrac{\sqrt{x}}{2a}=0\\[1.2ex]\dfrac{1}{x_0}a^2+\dfrac{1}{2\sqrt{x_0+1}}a-\dfrac{1}{4\sqrt{x_0}}=0\end{cases}\]
    我们虽然难以直接解出第一个方程,但是却可以从第二个方程解出$a$,利用求根公式得到:
    \[a=\sqrt{\left(\frac{x_0}{4\sqrt{x_0+1}}\right)^2+\dfrac{\sqrt{x_0}}{4}}-\frac{x_0}{4\sqrt{x_0+1}}>0\]
    代入第一个方程得到一个很浸泡的式子:\[\ln x_0=\frac{x_0}{2\left(\sqrt{\left(\frac{x_0}{4\sqrt{x_0+1}}\right)^2+\frac{\sqrt{x_0}}{4}}-\frac{x_0}{4\sqrt{x_0+1}}\right)^2}-\dfrac{\sqrt{1+x_0}}{\left(\sqrt{\left(\frac{x_0}{4\sqrt{x_0+1}}\right)^2+\frac{\sqrt{x_0}}{4}}-\frac{x_0}{4\sqrt{x_0+1}}\right)}\]
    显然这里只能取$x=1$得到$a\in\left(0,\dfrac{\sqrt{2}}{4}\right]$,这样就转化为证明充分性:\[f(x)\leq0\Rightarrow-2\ln x-\frac{2\sqrt{x+1}}{a}+\frac{\sqrt{x}}{a^2}\geq0\]
    直接上求根公式好了,懒得讨论了:\[\Leftrightarrow \dfrac{1}{a}\geq\sqrt{1+\frac{1}{x}}+\sqrt{1+\frac{1}{x}+\frac{2\ln x}{\sqrt x}}\Rightarrow 2\sqrt2\geq\sqrt{1+\frac{1}{x}}+\sqrt{1+\frac{1}{x}+\frac{2\ln x}{\sqrt x}}\]
    \[\Leftrightarrow\left(2\sqrt{2}-\sqrt{1+\frac{1}{x}}\right)^{2}\geq1+\frac{1}{x}+\frac{2\ln x}{\sqrt{x}}\Leftrightarrow \ln x\leq4\sqrt{x}-2\sqrt{2}\sqrt{x+1}\]
    换元$t=x^2,t\in\left(\dfrac{1}{e},1\right)$,转化为证明$\ln t\leq 2t-\sqrt2\sqrt{t^2+1}$,并利用飘带放缩:即$\forall x\in\left(\dfrac{1}{e},1\right]$
    \[g(x)=2x-\sqrt{2x^2+2}-\ln x=\dfrac{\sqrt2(x-1)(x+1)}{\sqrt{2}x+\sqrt{1+x^2}}-\ln x\geq \dfrac{\sqrt2(x-1)(x+1)}{\sqrt{2}x+\sqrt{1+x^2}}-2\dfrac{x-1}{x+1}\geq 0\]
    最后一个不等号成立,要求$\dfrac{(x+1)^2}{\sqrt{2}x+\sqrt{1+x^2}}\leq \sqrt2\Leftrightarrow \sqrt{1+x^2}\leq \sqrt{2}$,成立!因此我们知道充分性也成立,所以任意$x\in\left(0,\dfrac{1}{\e^2}\right]$,都有$a\in\left(0,\dfrac{\sqrt{2}}{4}\right]$满足要求.
\end{solution}
\newpage

\begin{example}{(2008年江西浸泡压轴题)}{}
    已知$\displaystyle f \left( x \right)= \frac{1}{ \sqrt{1+x}}+ \frac{1}{ \sqrt{1+a}}+ \sqrt{ \frac{ax}{ax+8}},x \in \left( 0,+ \infty \right)$
\vspace{5pt}

    (1)当$a=8$时,求$f(x)$的单调区间.

    (2)对于任意正数$a$,求证$1<f(x)<2$.
\end{example}
\begin{solution}
    (1)当$a=8$时,$\displaystyle f(x)= \frac{1}{ \sqrt{1+x}}+ \frac{1}{3}+ \sqrt{ \frac{x}{x+1}}$,观察式子不难想到换元$x=\tan^2\theta$:
    \[f(x)=\frac{1}{ \sqrt{1+\tan^2\theta}}+ \frac{1}{3}+ \sqrt{ \frac{\tan^2\theta}{\tan^2\theta+1}}\\=\cos \theta+ \frac{1}{3}+ \sin \theta\\=\sqrt{2}\sin(\theta+\frac{\pi}{4})+\frac13\]
    由$\theta\in(0,\frac{\pi}{2})$,所以$\sin(\theta+\frac{\pi}{4})$在$(0,\frac{\pi}{4})$上单调递增,在$(\frac{\pi}{4},\frac{\pi}{2})$上单调递减,所以$f(x)$在$(0,1)$上单调递增,在$(1,+\infty)$上单调递减。

    (2)这道题难就难在题目给了函数,似乎是暗示学生走求导的路子,但求导异常难做,反而看成是不等式问题却能找到方向。由于函数结构不对称,我们引入第三个变元就可以将本题条件化为对称形式:令$f \left( x \right)= \frac{1}{ \sqrt{1+x}}+ \frac{1}{ \sqrt{1+y}}+ \sqrt{ \frac{xy}{xy+8}},x,y \in \left( 0,+ \infty \right)$:
    \[
    \begin{cases}a=\dfrac{1}{\sqrt{1+x}}\in(0,1)\\b=\dfrac{1}{ \sqrt{1+y}}\in(0,1)\\c=\sqrt{ \dfrac{xy}{xy+8}}\in(0,1)\end{cases}\Rightarrow 
    \begin{cases}a^2=\dfrac{1}{1+x},x=\dfrac{1}{a^2}-1\\[1.2ex]b^2=\dfrac{1}{1+y},y=\dfrac{1}{b^2}-1\\[1.2ex]c^2=\dfrac{ax}{ax+8}=1-\dfrac{8}{ax+8}\end{cases}
    \]
现在要找到三个元之间的关系式,消元法就够了:
\[
c^2=1-\dfrac{8}{ax+8}=1-\dfrac{8}{(\dfrac{1}{a^2}-1)(\dfrac{1}{b^2}-1)+8}\Rightarrow (1-a^2)(1-b^2)(1-c^2)=8a^2b^2c^2
\]先假设$a+b+c>=2$,列出已知条件:
\[\begin{cases}a,b,c\in(0,1),\quad a+b+c\geq 2\\(1-a^2)(1-b^2)(1-c^2)=8a^2b^2c^2\end{cases}\]
我们现在要证明的是“同时满足这几个条件的题目是一个错题”。其中最后一个条件看似很强,给出了三元关系,但是如果不拿来消元的话很难用上,而消元又回到了原题的情形,这就很尴尬了。那我们不妨尝试删除$(1-a^2)(1-b^2)(1-c^2)=8a^2b^2c^2$并尝试通过剩余条件导出$(1-a^2)(1-b^2)(1-c^2)\ne8a^2b^2c^2$,即$(1-a^2)(1-b^2)(1-c^2)>8a^2b^2c^2$或$(1-a^2)(1-b^2)(1-c^2)<8a^2b^2c^2$。而这个式子是对称的,我们不妨拆成$(1-a^2)>2bc$或$(1-a^2)<2bc$来证明,虽然这个转换之后的式子是原来式子成立的充分而非必要条件,但确实是可以尝试的方向,是不是可以叫“充分性探路”呢?
\[\begin{cases}1-a^2=(1-a)(1+a)<2(1-a)=2(b+c-1)\\(1-b)(1-c)\in(0,1)\Rightarrow bc+1<b+c\end{cases}\Rightarrow 1-a^2<2bc\]
这样同理得到$1-b^2<2ac,1-c^2<2ab$,这样就有$(1-a^2)(1-b^2)(1-c^2)<8a^2b^2c^2$,矛盾,所以$a+b+c<2$。然后我们紧接着设$a+b+c\leq 1$,同样列出已知条件:
\[\begin{cases}a,b,c\in(0,1),\quad a+b+c<=1\\(1-a^2)(1-b^2)(1-c^2)=8a^2b^2c^2\end{cases}\]
同样的套路,我们依然考虑证明$(1-a^2)>2bc$或$(1-a^2)<2bc$,由
\[\begin{cases}1-a^2=(1-a)(1+a)>1-a\geq b+c\\(1-b)(1-c)\in(0,1)\Rightarrow bc<b+c\end{cases}\Rightarrow 1-a^2>2bc\]
这样同理得到$1-b^2>2ac,1-c^2>2ab$,这样就有$(1-a^2)(1-b^2)(1-c^2)>8a^2b^2c^2$,矛盾。所以只能是$1<a+b+c<2$.

\noindent(3)当然本题解法不唯一,比如说我们根据
\[f \left( x \right)= \frac{1}{ \sqrt{1+x}}+ \frac{1}{ \sqrt{1+y}}+ \sqrt{ \frac{xy}{xy+8}}=\frac{1}{ \sqrt{1+x}}+ \frac{1}{\sqrt{1+y}}+ \frac{1}{1+\sqrt{\frac{8}{xy}}}\]
来换元$z=\dfrac{8}{xy}$,则有$xyz=8$要证明$ 1<\dfrac{1}{ \sqrt{1+x}}+ \dfrac{1}{ \sqrt{1+y}}+ \dfrac{1}{\sqrt{1+z}}<2$,已知条件是$x,y,z>0$,那么$(x-2)(y-2)(z-2)<8$,不妨假设数量关系$x\leq y\leq z$,就有
\[\begin{cases}2\leq z\\xyz=8\end{cases}\Rightarrow xy\leq 4\]
然后运用对偶式的思想证明出:\[
\frac{1}{\sqrt{1+x}}+\frac{1}{\sqrt{1+y}}>\frac{1}{1+\frac{x}{2}}+\frac{1}{1+\frac{y}{2}}\geq\frac{1}{1+\frac{x}{2}}+\frac{1}{1+\frac{2}{x}}=1
\]
由$z\geq 2$得到\[\frac{1}{\sqrt{1+z}}<\frac{1}{1+\frac{z}{8}}=\frac{1}{\sqrt{\frac{1}{xy}}}=\frac{\sqrt{xy}}{1+\sqrt{xy}}\]
以及\[(1-\frac{1}{\sqrt{1+x}})(1-\frac{1}{\sqrt{1+y}})>0\Rightarrow \frac{1}{\sqrt{1+x}}+\frac{1}{\sqrt{1+y}}<1+\frac{1}{\sqrt{(1+x)(1+y)}}<1+\frac{1}{1+\sqrt{xy}}\]
合起来就是:
\[\dfrac{1}{ \sqrt{1+x}}+ \dfrac{1}{ \sqrt{1+y}}+ \dfrac{1}{\sqrt{1+z}}<1+\frac{1}{1+\sqrt{xy}}+\frac{\sqrt{xy}}{1+\sqrt{xy}}=2\]
这个$\dfrac{1}{\sqrt{1+z}}<\dfrac{1}{1+\dfrac{z}{8}}$是事后诸葛,高考生不必深究。
\end{solution}
\begin{example}{(2008年江西导数压轴)}{}
已知函数$f \left( x \right)= \dfrac{1}{ \sqrt{1+x}}+ \dfrac{1}{ \sqrt{1+a}}+ \dfrac{1}{ \sqrt{1+ \dfrac{8}{ax}}}.$证明:$1<f(x)<2.$
\end{example}
\begin{solution}
将$a$视作参数,直接暴力求导:
\begin{align*}f^{\prime}(x)&=-\frac{1}{2(1+x)^{\frac{3}{2}}}+\frac{1}{2(1+\frac{8}{ax})^{\frac{3}{2}}}\cdot\frac{8}{ax^2}\\&=-\frac{1}{2(1+x)^\frac{3}{2}}+\frac{4\sqrt{a}}{\sqrt{x}(ax+8)^\frac{3}{2}}=\frac{8\sqrt{a}(1+x)^\frac{3}{2}-\sqrt{x}(ax+8)^\frac{3}{2}}{2\sqrt{x}(1+x)^\frac{3}{2}(ax+8)^\frac{3}{2}}\\&=\frac{64a(1+x)^3-x(ax+8)^3}{2\sqrt{x}(1+x)^{\frac{3}{2}}(ax+8)^{\frac{3}{2}}\left[8\sqrt{a}(1+x)^{\frac{3}{2}}+\sqrt{x}(ax+8)^{\frac{3}{2}}\right]}\\&=\frac{-a^3x^4-24a^2x^3+64ax^3+192ax+64a-512x}{2\sqrt{x}(1+x)^{\frac{3}{2}}(ax+8)^{\frac{3}{2}}\left[8\sqrt{a}(1+x)^{\frac{3}{2}}+\sqrt{x}(ax+8)^{\frac{3}{2}}\right]}\\
&=\frac{-a(a^2x^4+(24a-64)x^3-8(24-\frac{64}{a})x-64)}{2\sqrt{x}(1+x)^{\frac{3}{2}}(ax+8)^{\frac{3}{2}}\left[8\sqrt{a}(1+x)^{\frac{3}{2}}+\sqrt{x}(ax+8)^{\frac{3}{2}}\right]}\\
&=\frac{-a\bigg((ax^2-8)(ax^2+8)+(24-\frac{64}{a})(ax^3-8x)\bigg)}{2\sqrt{x}(1+x)^{\frac{3}{2}}(ax+8)^{\frac{3}{2}}\left[8\sqrt{a}(1+x)^{\frac{3}{2}}+\sqrt{x}(ax+8)^{\frac{3}{2}}\right]}\\
&=\frac{\left(ax^{2}-8 \right) \left(-x^{2}a^{2}-24xa-8a+64x \right)}{2\sqrt{x}(1+x)^{\frac{3}{2}}(ax+8)^{\frac{3}{2}}\left[8\sqrt{a}(1+x)^{\frac{3}{2}}+\sqrt{x}(ax+8)^{\frac{3}{2}}\right]}\\
&=\frac{(\sqrt{a}x-2\sqrt2)(\sqrt{a}x+2\sqrt2)(-a^2x^2+(64-24a)x-8a)}{2\sqrt{x}(1+x)^{\frac{3}{2}}(ax+8)^{\frac{3}{2}}\left[8\sqrt{a}(1+x)^{\frac{3}{2}}+\sqrt{x}(ax+8)^{\frac{3}{2}}\right]}
\end{align*}
这个分子有理化(目的是提取恒正的项)肯定是套路式的,但是这个因式分解,笔者认为有点小技巧,我们提取$a$使得常数项真正地为常数,然后由于二次项为0,以及一次项和三次项之间有倍数关系,采取凑平方差,分解成了两个二次函数,由于分母大于0,现在要对分子的正负性进行讨论,方便起见,我们看一下后面那个丑陋的二次函数的判别式长什么样:
\[\Delta = (64-24a)^2-32a^3=32(-a^3+18a^2-96a+128)=32(2-a)(a-8)^2\]
所以如果限定$a\geq 2$那么$\Delta\leq 0$,再加上二次项为负数,所以这个二次函数小于0,但是我们能不能做这样的限定呢?其实是可以的,因为$f(x)$根号的下面主要有$a,x,\frac{8}{ax}$,这三个东西乘起来是$8$,是对称的三个变元,我们显然可以给它们规定大小顺序,所以想一想不难知道规定$a\geq 2$是合理的,那么规定另外两个大于等于2行不行呢?也行,但是会导致解题变得更加复杂,所以不推荐。

现在,我们只需关注$\sqrt{a}x-2\sqrt2$的正负性了,这东西是一次函数,容易知道$f'(x)$在$(0,\sqrt{\dfrac{8}{a}})$
大于0,在$(\sqrt{\dfrac{8}{a}},+\infty)$小于0,所以$f(x)$在$(0,\sqrt{\dfrac{8}{a}})$单调递增,在$(\sqrt{\dfrac{8}{a}},+\infty)$单调递减。所以$f(x)$的最大值是\[f(\sqrt{\dfrac{8}{a}})=\frac{2}{\sqrt{1+\sqrt{\frac{8}{a}}}}+\frac{1}{\sqrt{1+a}}\]
下面考虑最小值,发现当$x\rightarrow 0$时,$f(x)\rightarrow 1+\frac{1}{\sqrt{1+a}}$,当$x\rightarrow +\infty$时,$f(x)\rightarrow 1+\frac{1}{\sqrt{1+a}}$,所以\[1+\frac{1}{\sqrt{1+a}}<f(x)<\frac{2}{\sqrt{1+\sqrt{\frac{8}{a}}}}+\frac{1}{\sqrt{1+a}}\]
此时$1<f(x)$已经得证,下面只需证明\[g(a)=\frac{2}{\sqrt{1+\sqrt{\frac{8}{a}}}}+\frac{1}{\sqrt{1+a}}<2\]
再次求导:
\begin{align*}g'(a)&=-\frac{1}{\left(1+\sqrt{\frac{8}{a}}\right)^{\frac{3}{2}}}\cdot\frac{-2\sqrt2}{2a^{\frac{3}{2}}}-\frac{1}{2(1+a)^{\frac{3}{2}}}\\&=\frac{2\sqrt{2}(1+a)^{\frac{3}{2}}-a^{\frac{3}{2}}\left(1+\sqrt{\frac{8}{a}}\right)^{\frac{3}{2}}}{2a^{\frac{3}{2}}\left(1+\sqrt{\frac{8}{a}}\right)^{\frac{3}{2}}(1+a)^{\frac{3}{2}}}\\&=\frac{8(1+a)^3-\left(a+2\sqrt{2a}\right)^3}{2a^{\frac{3}{2}}\left(1+\sqrt{\frac{8}{a}}\right)^{\frac{3}{2}}(1+a)^{\frac{3}{2}}\left[2\sqrt{2}(1+a)^{\frac{3}{2}}+a^{\frac{3}{2}}\left(1+\sqrt{\frac{8}{a}}\right)^{\frac{3}{2}}\right]}\\&=\frac{(a-2\sqrt{2a}+2)\left[4(1+a)^2+2(1+a)(a+2\sqrt{2a})+(a+2\sqrt{2a})^2\right]}{2a^{\frac{3}{2}}\left(1+\sqrt{\frac{8}{a}}\right)^{\frac{3}{2}}(1+a)^{\frac{3}{2}}\left[2\sqrt{2}(1+a)^{\frac{3}{2}}+a^{\frac{3}{2}}\left(1+\sqrt{\frac{8}{a}}\right)^{\frac{3}{2}}\right]}\\&=\frac{(\sqrt{a}-\sqrt{2})^2\left[4(1+a)^2+2(1+a)(a+2\sqrt{2a})+(a+2\sqrt{2a})^2\right]}{2a^{\frac{3}{2}}\left(1+\sqrt{\frac{8}{a}}\right)^{\frac{3}{2}}(1+a)^{\frac{3}{2}}\left[2\sqrt{2}(1+a)^{\frac{3}{2}}+a^{\frac{3}{2}}\left(1+\sqrt{\frac{8}{a}}\right)^{\frac{3}{2}}\right]}\geq0\end{align*}
所以原函数$g(a)$单调递增,考虑到\[\lim_{x\rightarrow +\infty}g(a)=0+2=2\]所以$f(x)<2$也得证。
\end{solution}

\newpage
\section{三角函数恒成立问题}
\begin{example}{经典题}{}
求$f(x) = \sin x \sin 2x \sin 3x$的值域.
\end{example}
\begin{solution}
反复积化和差
\begin{align*}
    f(x) &= \frac{1}{2}(\sin 2x \cos 2x - \sin 2x \cos 4x)= \frac{1}{4}(\sin 2x + \sin 4x - \sin 6x)\\
    f'(x)&= \frac{1}{4}(2\cos 2x + 4\cos 4x - 6\cos 6x) = 0
\end{align*}
令 $v = \cos 2x$,用二倍角和三倍角公式($\cos 4x = 2v^2 - 1$,$\cos 6x = 4v^3 - 3v$)代入$f'(x)=0$得到
\[6v^3 - 2v^2 - 5v + 1 = (v - 1)(6v^2 + 4v - 1) = 0\Leftrightarrow v =\frac{-2 \pm \sqrt{10}}{6}\]
此时$f(x) = \frac{1}{4}\sin 2x (1 + 2\cos 2x - (4\cos^2 2x - 1)) = \frac{1}{2}\sin 2x (1 + v - 2v^2)$,将$\sin 2x$换成$v$的式子,考虑平方:$$f(x)^2 = \frac{1}{4}\sin^2 2x (1 + v - 2v^2)^2 = \frac{1}{4}(1 - v^2)(1 + v - 2v^2)^2$$用$6v^2 = 1 - 4v \implies v^2 = \frac{1 - 4v}{6}$反复降次得到极值$ \frac{1039 + 680v}{972}$,此时$f(x)^2 = \frac{2437 + 340\sqrt{10}}{11664}$,开方得到值域\[\left[- \frac{34\sqrt{2} + 5\sqrt{5}}{108}, \frac{34\sqrt{2} + 5\sqrt{5}}{108}\right]\]
\end{solution}


\begin{example}{来自“扁头耄耋”}{}
    证明对于任意 $x > 0$,有:$$x + \frac{1}{x} + 1 \geqslant \cot\left(\frac{\pi}{2(x^2+x+1)}\right)\cot\left(\frac{\pi x^2}{2(x^2+x+1)}\right)$$
\end{example}
\begin{solution}
设 $A = \frac{\pi}{2(x^2+x+1)}$,$B = \frac{\pi x^2}{2(x^2+x+1)}$,并构造辅助角 $C = \frac{\pi x}{2(x^2+x+1)}$。由于 $x > 0$,显然有 $A, B, C \in (0, \frac{\pi}{2})$。注意到 $A + B + C = \frac{\pi}{2}$,且满足 $C^2 = \frac{\pi^2 x^2}{4(x^2+x+1)^2} = AB$,即 $C = \sqrt{AB}$。

为了化简不等式右侧 $\text{RHS} = \cot A \cot B$,利用和角公式及 $A + B = \frac{\pi}{2} - C$,可得:
$$ \tan(A+B) = \frac{\tan A + \tan B}{1 - \tan A \tan B} = \tan\left(\frac{\pi}{2} - C\right) = \cot C = \frac{1}{\tan C} $$
整理得 $\tan C(\tan A + \tan B) = 1 - \tan A \tan B$,等式两边同除以 $\tan A \tan B$ 并重排项,得到:
$$ \cot A \cot B = 1 + \tan C(\cot A + \cot B) $$
对于不等式左侧 $\text{LHS} = x + \frac{1}{x} + 1 = \frac{x^2+x+1}{x}$,结合 $C$ 的定义式可知 $\text{LHS} = \frac{\pi}{2C}$。由 $A+B+C = \frac{\pi}{2}$,可将其进一步表示为 $\text{LHS} = \frac{A+B+C}{C} = \frac{A+B}{C} + 1$。此时,原不等式 $\text{LHS} \ge \text{RHS}$ 等价于:
$$ \frac{A+B}{C} + 1 \ge 1 + \tan C(\cot A + \cot B) \iff \frac{\cot A + \cot B}{A+B} \le \frac{\cot C}{C} $$
代入 $C = \sqrt{AB}$,命题转化为证明对于任意 $A, B \in (0, \frac{\pi}{2})$,恒有 $\frac{\cot A + \cot B}{A+B} \le \frac{\cot \sqrt{AB}}{\sqrt{AB}}$。

利用$\cot z = \frac{1}{z} - \sum_{n=1}^\infty \frac{2z}{n^2\pi^2 - z^2}$。将其代入目标不等式左侧得:
$$ \frac{\cot A + \cot B}{A+B} = \frac{1}{AB} - \sum_{n=1}^\infty \frac{2}{A+B} \left( \frac{A}{n^2\pi^2 - A^2} + \frac{B}{n^2\pi^2 - B^2} \right) $$
目标不等式右侧展开为:
$$ \frac{\cot \sqrt{AB}}{\sqrt{AB}} = \frac{1}{AB} - \sum_{n=1}^\infty \frac{2}{n^2\pi^2 - AB} $$
只需证明级数中每一对应项均满足:
$$ \frac{1}{A+B} \left( \frac{A}{a - A^2} + \frac{B}{a - B^2} \right) \ge \frac{1}{a - AB} \quad (\text{其中 } a = n^2\pi^2, n \ge 1) $$
由于 $A, B \in (0, \frac{\pi}{2})$,分母 $a-A^2, a-B^2, a-AB$ 均为正数。将上式两边同乘分母并展开:
$$ [A(a - B^2) + B(a - A^2)](a - AB) \ge (A+B)(a - A^2)(a - B^2) $$
左侧化简为 $(A+B)(a - AB)^2$,右侧展开为 $(A+B)[a^2 - a(A^2+B^2) + A^2B^2]$。消去正数因子 $(A+B)$ 并展开平方式:
$$ a^2 - 2aAB + A^2B^2 \ge a^2 - a(A^2+B^2) + A^2B^2 \iff a(A^2+B^2) \ge 2aAB $$
由于 $a > 0$,上式等价于 $(A-B)^2 \ge 0$,该式显然成立。故原级数每一项的不等关系均成立,累加后得证。

当且仅当 $A = B$ 时等号成立,即 $x^2 = 1$,结合 $x > 0$ 知取等条件为 $x = 1$。综上所述,原不等式得证。
\end{solution}

\begin{example}{来自“扁头耄耋”}{}
    证明对于任意 $x > 0$,有:$$ \cot\left(\frac{\pi}{2(x^2+x+1)}\right)\cot\left(\frac{\pi x^2}{2(x^2+x+1)}\right)\geqslant 3$$
\end{example}
\begin{solution}
    设$A=\frac{\pi}{2(x^2+x+1)}$,$B=\frac{\pi x^2}{2(x^2+x+1)}$,并继续沿用前文构造的辅助角$C=\frac{\pi x}{2(x^2+x+1)}$。由于$x>0$,显然有$A, B, C \in (0, \frac{\pi}{2})$,且满足$A+B+C=\frac{\pi}{2}$及$C=\sqrt{AB}$。利用前文已证明的恒等式进行变形:$$\cot A \cot B = 1 + \tan C(\cot A + \cot B)$$将括号内式子用正余弦展开并通分,同时代入和角关系$A+B=\frac{\pi}{2}-C$:$$\cot A + \cot B = \frac{\sin(A+B)}{\sin A \sin B} = \frac{\sin(\frac{\pi}{2} - C)}{\sin A \sin B} = \frac{\cos C}{\sin A \sin B}$$将其代回原式,即可将目标等式右侧化简为:$$\cot A \cot B = 1 + \frac{\sin C}{\cos C} \cdot \frac{\cos C}{\sin A \sin B} = 1 + \frac{\sin C}{\sin A \sin B}$$为证明原式$\geqslant 3$,我们需要对分母$\sin A \sin B$寻找上限。由于$C$是$A, B$的几何平均数(即$C=\sqrt{AB}$),自然联想到利用函数的凹凸性。
构造函数$f(t)=\ln(\sin e^t)$,其定义域为$e^t \in (0, \frac{\pi}{2})$。对该函数求导:
$$f'(t) = \frac{\cos e^t \cdot e^t}{\sin e^t} = e^t \cot e^t,f''(t) = e^t \cot e^t - e^{2t} \csc^2 e^t = \frac{\frac{1}{2}e^t \sin(2e^t) - e^{2t}}{\sin^2 e^t}$$
对于任意$u=e^t > 0$,由于$\sin(2u) < 2u$恒成立,故分子$\frac{1}{2}u \sin(2u) - u^2 < u^2 - u^2 = 0$,这意味着$f''(t) < 0$。因此$f(t)$在对应区间上为严格凹函数。
根据Jensen不等式(琴生不等式),有:
$$\frac{f(\ln A) + f(\ln B)}{2} \leqslant f\left(\frac{\ln A + \ln B}{2}\right) = f(\ln \sqrt{AB}) = f(\ln C)$$
即$\frac{1}{2}(\ln \sin A + \ln \sin B) \leqslant \ln \sin C$,脱去对数后得到关键不等式:$$\sin A \sin B \leqslant \sin^2 C$$(当且仅当$A=B$时取等号)。将此结论代入前面的化简式中:$$\cot A \cot B = 1 + \frac{\sin C}{\sin A \sin B} \geqslant 1 + \frac{\sin C}{\sin^2 C} = 1 + \frac{1}{\sin C}\geqslant 1 + \frac{1}{\sin(\frac{\pi}{6})} = 1 + 2 = 3$$综上所述,$\cot A \cot B \geqslant 3$恒成立。等号成立的条件为$A=B$且$C=\frac{\pi}{6}$,即$x=1$。原不等式另一侧得证。
\end{solution}

\begin{example}{来自“许你坚强”}{}
    证明对于任意 $A, B \in (0, \frac{\pi}{2})$,恒有 $\displaystyle\frac{\cot A + \cot B}{A+B} \le \frac{\cot \sqrt{AB}}{\sqrt{AB}}$。
\end{example}
\begin{solution}
已知 $A, B \in (0, \frac{\pi}{2})$,由于待证不等式关于 $A, B$ 对称,不妨设 $A \geq B$。令 $x = \sqrt{AB}$,则 $x \in (0, \frac{\pi}{2})$ 且 $A \geq x \geq B > 0$。由 $B = \frac{x^2}{A}$ 可知,在固定常数 $x$ 的前提下,$A$ 的取值范围为 $[x, \frac{\pi}{2})$。

构造关于 $A$ 的单变量函数:
$$f(A) = \frac{\cot A + \cot(\frac{x^2}{A})}{A + \frac{x^2}{A}}$$
对其求导可得:
$$f'(A) = \frac{(A + \frac{x^2}{A})(-\csc^2 A + \frac{x^2}{A^2}\csc^2 \frac{x^2}{A}) - (1 - \frac{x^2}{A^2})(\cot A + \cot \frac{x^2}{A})}{(A + \frac{x^2}{A})^2}$$
将 $B = \frac{x^2}{A}$ 代回上述导数公式,并将分子与分母同乘 $A$,得:
$$f'(A) = \frac{(A + B)(-A \csc^2 A + B \csc^2 B) - (A - B)(\cot A + \cot B)}{A(A + B)^2}$$
令导数公式的分子部分为 $\Delta(A, B)$,即:
$$\Delta(A, B) = (A + B)(B \csc^2 B - A \csc^2 A) - (A - B)(\cot A + \cot B)$$
由于分母 $A(A + B)^2 > 0$,$f'(A)$ 的符号完全由 $\Delta(A, B)$ 决定。要证明原不等式,需证当 $A \geq B$ 时,恒有 $\Delta(A, B) \leq 0$。

引入辅助函数 $R(y) = y \cot y$,$y \in (0, \frac{\pi}{2})$,其导数为 $R'(y) = \cot y - y \csc^2 y$。进一步令 $Q(y) = -R'(y) = y \csc^2 y - \cot y$。代入 $\Delta(A, B)$ 的第一项中有 $y \csc^2 y = Q(y) + \cot y$,故:
$$B \csc^2 B - A \csc^2 A = Q(B) - Q(A) + \cot B - \cot A$$
将其代入 $\Delta(A, B)$ 并展开,可化简为:
$$\Delta(A, B) = (A + B)[Q(B) - Q(A)] - 2(A\cot A - B\cot B) = (A + B)[Q(B) - Q(A)] - 2[R(A) - R(B)]$$
由微积分基本定理,连续函数 $R(y)$ 在区间上的增量可表示为积分:
$$R(A) - R(B) = \int_{B}^{A} R'(t) dt = -\int_{B}^{A} Q(t) dt$$
为证明 $\Delta(A, B) \leq 0$,等价于证明如下构造的二元函数 $F(A, B) \geq 0$:
$$F(A, B) = -\Delta(A, B) = (A + B)[Q(A) - Q(B)] - 2\int_{B}^{A} Q(t) dt$$
对 $F(A, B)$ 关于 $A$ 求一阶偏导:
$$\frac{\partial F}{\partial A} = [Q(A) - Q(B)] + (A + B)Q'(A) - 2Q(A) = (A + B)Q'(A) - Q(A) - Q(B)$$
令 $G(A, B) = \frac{\partial F}{\partial A}$,继续对 $A$ 求偏导:
$$\frac{\partial G}{\partial A} = Q'(A) + (A + B)Q''(A) - Q'(A) = (A + B)Q''(A)$$

为确定偏导数的符号,考察 $Q(y)$ 的各阶导数($y \in (0, \frac{\pi}{2})$):
$$Q'(y) = 2\csc^2 y(1 - y \cot y)$$
$$Q''(y) = 2\csc^2 y [y - 3 \cot y + 3y \cot^2 y]$$
将 $Q''(y)$ 方括号内的部分转化为正余弦表达式,令 $u = 2y \in (0, \pi)$:
$$y - 3 \cot y + 3y \cot^2 y = \frac{y(1 + 2 \cos^2 y) - 1.5 \sin(2y)}{\sin^2 y} = \frac{\frac{u}{2}(2 + \cos u) - 1.5 \sin u}{\sin^2 y}$$
只需考察分子的符号,设 $z(u) = u + \frac{u}{2} \cos u - 1.5 \sin u$。对其求导:
$$z'(u) = 1 - \cos u - \frac{u}{2} \sin u$$
$$z''(u) = \frac{1}{2}(\sin u - u \cos u)$$
当 $u \in (0, \frac{\pi}{2}]$ 时,由正切函数的不等式性质 $\tan u > u$ 可知 $\sin u > u \cos u$;当 $u \in (\frac{\pi}{2}, \pi)$ 时,$\cos u < 0$,显然有 $\sin u - u \cos u > 0$。因此在整个区间 $(0, \pi)$ 内恒有 $z''(u) > 0$。由此可知 $z'(u)$ 单调递增,结合 $z'(0) = 0$ 得 $z'(u) > 0$;进而 $z(u)$ 单调递增,结合 $z(0) = 0$ 推得 $z(u) > 0$。这表明对于 $y \in (0, \frac{\pi}{2})$,恒有 $Q''(y) > 0$。

基于 $Q''(A) > 0$ 及 $A + B > 0$,可得 $\frac{\partial G}{\partial A} > 0$,即 $G(A, B)$ 关于 $A$ 在 $[B, \frac{\pi}{2})$ 上单调递增。考察起点值 $G(B, B) = 2(BQ'(B) - Q(B))$,设 $H(y) = yQ'(y) - Q(y)$,求导得 $H'(y) = yQ''(y) > 0$,故 $H(y)$ 单调递增。
利用泰勒展开分析 $y \to 0^+$ 时的极限行为:
$$Q(y) = \frac{y - \frac{1}{2}\sin(2y)}{\sin^2 y} = \frac{y - \frac{1}{2}(2y - \frac{4y^3}{3} + \dots)}{y^2 - \frac{y^4}{3} + \dots} = \frac{2y}{3} + \frac{4y^3}{45} + O(y^5)$$
求导得 $Q'(y) = \frac{2}{3} + \frac{4y^2}{15} + O(y^4)$。代入 $H(y)$ 中计算:
$$H(y) = y\left(\frac{2}{3} + \frac{4y^2}{15}\right) - \left(\frac{2y}{3} + \frac{4y^3}{45}\right) + O(y^5) = \frac{8y^3}{45} + O(y^5)$$
因此 $\lim_{y \to 0^+} H(y) = 0$。由于 $H(y)$ 严格单调递增,故当 $y > 0$ 时 $H(y) > 0$,从而起点值 $G(B, B) > 0$。

因为 $G(B, B) > 0$ 且 $\frac{\partial G}{\partial A} > 0$,故对于 $A \geq B$ 恒有 $G(A, B) > 0$。这表明 $\frac{\partial F}{\partial A} > 0$,函数 $F(A, B)$ 关于 $A$ 单调递增。结合 $F(B, B) = 0$,得出当 $A \geq B$ 时 $F(A, B) \geq 0$ 恒成立。
由 $F(A, B) \geq 0$ 可得 $\Delta(A, B) \leq 0$,进而 $f'(A) \leq 0$。这意味着在保持 $x = \sqrt{AB}$ 不变的情况下,函数 $f(A)$ 在区间 $[x, \frac{\pi}{2})$ 上严格单调递减。因此,函数在边界点 $A = x$(即 $A = B = \sqrt{AB}$)处取得最大值:
$$f(A) \leq f(x) = \frac{\cot x + \cot(\frac{x^2}{x})}{x + \frac{x^2}{x}} = \frac{2\cot x}{2x} = \frac{\cot x}{x}$$
将 $x = \sqrt{AB}$ 代回即证得原不等式:
$$\frac{\cot A + \cot B}{A + B} \leq \frac{\cot \sqrt{AB}}{\sqrt{AB}}$$
当且仅当 $A = B$ 时,等号成立。证毕。
\end{solution}


\begin{example}{}{}
    设函数$f\left ( x\right ) = \cos 3x- \cos x\sin ^3x$ .\\
(1)求$f(x)$在$[0,\pi]$上的单调区间; (2)求$|f(x)|$的最大值.
\end{example}
\begin{solution}

\end{solution}
\begin{example}{}{}
$0<x<\frac{\pi}{2},k\geqslant\frac{1}{2}$,证明\[\frac{\tan x}{k}\left(\sin\frac{x}{k}\right)^{2k^2}>\left(\frac{x}{k}\right)^{2k^2+1}\]
\end{example}
\begin{solution}
由于 $0<x<\frac{\pi}{2}$ 且 $k \geqslant \frac{1}{2}$,分离变量 $x$ 和 $k$:
$$ \frac{\tan x}{k} \left(\sin\frac{x}{k}\right)^{2k^2} > \left(\frac{x}{k}\right)^{2k^2} \frac{x}{k} \Leftrightarrow\frac{\tan x}{x} > \left( \frac{\frac{x}{k}}{\sin \frac{x}{k}} \right)^{2k^2} $$
考虑证明右侧关于 $k$ 的单调递减。若证得则右侧在 $k \geqslant \frac{1}{2}$ 时的最大值在 $k = \frac{1}{2}$ 处取得,随后只需证明 $k = \frac{1}{2}$ 时的右侧小于左侧即可。令右侧括号内的变量 $t = \frac{x}{k}$。由于 $0<x<\frac{\pi}{2}$ 且 $k \geqslant \frac{1}{2}$,可知 $t \in (0, \pi)$。构造函数 $f(t) = \frac{\ln\left(\frac{t}{\sin t}\right)}{t^2}$,若能证明 $f(t)$ 在 $(0, \pi)$ 上单调递增,则由于 $t = \frac{x}{k}$ 随 $k$ 的增大而减小,即可得出 $\exp(2x^2 f(\frac{x}{k}))$ 随 $k$ 的增大而减小。对 $f(t)$ 求导:
$$ f'(t) = \frac{\frac{\sin t}{t} \cdot \frac{\sin t - t \cos t}{\sin^2 t} \cdot t^2 - 2t \ln\left(\frac{t}{\sin t}\right)}{t^4} = \frac{1}{t^3} \left[ \frac{\sin t - t \cos t}{\sin t} - 2 \ln\left(\frac{t}{\sin t}\right) \right] $$
令 $p(t) = \frac{\sin t - t \cos t}{\sin t} - 2 \ln\left(\frac{t}{\sin t}\right) = 1 - t \cot t - 2 \ln\left(\frac{t}{\sin t}\right)$,对其求导:
$$ p'(t) = -\cot t + t \csc^2 t - \frac{2}{t} + 2 \cot t = \cot t + t \csc^2 t - \frac{2}{t} $$
通分并利用倍角公式 $2\sin t \cos t = \sin 2t$ 以及 $2\sin^2 t = 1 - \cos 2t$ 化简:
$$ p'(t) = \frac{t \sin t \cos t + t^2 - 2 \sin^2 t}{t \sin^2 t} = \frac{t^2 + \frac{t}{2}\sin 2t + \cos 2t - 1}{t \sin^2 t} $$
令分子为 $r(t) = t^2 + \frac{t}{2}\sin 2t + \cos 2t - 1$,对其求导:
$$ r'(t) = 2t + \frac{1}{2}\sin 2t + t \cos 2t - 2\sin 2t = 2t + t \cos 2t - \frac{3}{2}\sin 2t = t(2 + \cos 2t) - \frac{3}{2}\sin 2t $$
下证:对于任意 $u > 0$,均有 $\frac{\sin u}{\cos u + 2} < \frac{u}{3}$。构造函数 $H(u) = u(2 + \cos u) - 3\sin u$,对其求导:
$$ H'(u) = 2 + \cos u - u \sin u - 3 \cos u = 2 - 2 \cos u - u \sin u $$
$$ H''(u) = 2 \sin u - \sin u - u \cos u = \sin u - u \cos u $$
当 $u \in (0, \frac{\pi}{2})$ 时,由于 $\tan u > u$,即 $\sin u > u \cos u$,所以 $H''(u) > 0$。当 $u \in [\frac{\pi}{2}, \pi)$ 时,$\sin u > 0$ 且 $\cos u \leqslant 0$,所以 $H''(u) > 0$ 显然成立。因此,$H'(u)$ 在 $(0, \pi)$ 上单调递增。结合 $H'(0) = 0$,可知当 $u \in (0, \pi)$ 时 $H'(u) > 0$。这说明 $H(u)$ 在 $(0, \pi)$ 上单调递增,由于 $H(0) = 0$,故当 $u \in (0, \pi)$ 时 $H(u) > 0$。对于 $u \geqslant \pi$,由于 $\frac{\sin u}{\cos u + 2} \leqslant 1$ 且 $\frac{u}{3} > 1$,不等式自然成立。综上,对于任意 $u > 0$,引理 $\frac{\sin u}{\cos u + 2} < \frac{u}{3}$ 均成立。利用此不等式,令 $u = 2t$,则有 $\frac{\sin 2t}{\cos 2t + 2} < \frac{2t}{3}$,即 $t(2 + \cos 2t) > \frac{3}{2}\sin 2t$。
这说明 $r'(t) > 0$,故 $r(t)$ 单调递增,$r(t) > r(0) = 0$。
进而 $p'(t) > 0$,所以 $p(t)$ 单调递增,$p(t) > \lim_{t\to 0} p(t) = 0$。
最终得到 $f'(t) = \frac{p(t)}{t^3} > 0$,即 $f(t)$ 在 $(0, \pi)$ 上单调递增。因为 $f(\frac{x}{k})$ 是关于 $k$ 的递减函数,所以在 $k \geqslant \frac{1}{2}$ 时:
$$ \left( \frac{\frac{x}{k}}{\sin \frac{x}{k}} \right)^{2k^2} = \exp\left(2x^2 f\left(\frac{x}{k}\right)\right) \leqslant \exp(2x^2 f(2x)) = \left(\frac{2x}{\sin 2x}\right)^{\frac{1}{2}} $$
最后,我们只需证明 $\left(\frac{2x}{\sin 2x}\right)^{\frac{1}{2}} < \frac{\tan x}{x}$。不等式两边均为正,平方等价于:
$$ \frac{2x}{\sin 2x} < \frac{\tan^2 x}{x^2} \Longleftrightarrow \frac{x}{\sin x \cos x} < \frac{\sin^2 x}{x^2 \cos^2 x} \Longleftrightarrow x^3 < \sin^2 x \tan x $$
对两边取对数,令 $m(x) = 2 \ln \sin x + \ln \tan x - 3 \ln x$,求导:
$$ m'(x) = 2 \cot x + \frac{\sec^2 x}{\tan x} - \frac{3}{x} = \frac{2\cos^2 x + 1}{\sin x \cos x} - \frac{3}{x} = \frac{2(\cos 2x + 2)}{\sin 2x} - \frac{3}{x} $$
再次使用 $\frac{\sin 2x}{\cos 2x + 2} < \frac{2x}{3}$,对其取倒数可得 $\frac{\cos 2x + 2}{\sin 2x} > \frac{3}{2x}$。
因此 $m'(x) > 2\left(\frac{3}{2x}\right) - \frac{3}{x} = 0$。
这说明 $m(x)$ 在 $(0, \frac{\pi}{2})$ 上单调递增,由于 $\lim_{x\to 0} m(x) = 0$,故 $m(x) > 0$,从而 $x^3 < \sin^2 x \tan x$ 成立。至此,$\left(\frac{x}{k}\right)^{2k^2+1} < \frac{\tan x}{k}\left(\sin\frac{x}{k}\right)^{2k^2}$ 证明完毕。
\end{solution}

\newpage
\begin{example}{三角函数}{}
    求函数 $f_a(x) = a^{\sin x} + a^{\cos x}$ 的取值范围,其中 $a > 1$.
\end{example}
\begin{solution}
对于函数 $f_a(x) = a^{\sin x} + a^{\cos x}$($a > 1$),首先分析其最小值的取得。根据均值不等式,有 $a^{\sin x} + a^{\cos x} \geq 2\sqrt{a^{\sin x + \cos x}}$。当且仅当 $a^{\sin x} = a^{\cos x}$,即 $\sin x = \cos x$ 时,等号成立。由辅助角公式可知 $\sin x + \cos x = \sqrt{2}\sin\left(x + \frac{\pi}{4}\right)$,其最小值为 $-\sqrt{2}$。考虑到指数函数 $y = a^u$($a > 1$)单调递增,故 $\sqrt{a^{\sin x + \cos x}}$ 的最小值为 $\sqrt{a^{-\sqrt{2}}} = a^{-1/\sqrt{2}}$。当 $x = 2k\pi - \frac{3\pi}{4}$($k \in \mathbb{Z}$)时,$\sin x = \cos x = -\frac{\sqrt{2}}{2}$,上述等号均能成立,故函数 $f_a(x)$ 的最小值为 $2a^{-1/\sqrt{2}}$。

分析函数 $f_a(x)$ 的最大值。作平移变换 $x = y + \frac{\pi}{4}$,则原函数可化为 $f_a(x) = a^{\frac{\sin y + \cos y}{\sqrt{2}}} + a^{\frac{\cos y - \sin y}{\sqrt{2}}}$。记 $b = \frac{\ln a}{\sqrt{2}} > 0$,利用双曲余弦函数 $\cosh u = \frac{e^u + e^{-u}}{2}$,原函数可改写为 $f_a(x) = 2e^{b \cos y} \cosh(b \sin y)$。对 $\frac{f_a(x)}{2}$ 取对数得 $\ln\frac{f_a(x)}{2} = b \cos y + \ln \cosh(b \sin y)$。由于 $\cos y \leq \sqrt{1 - \sin^2 y}$,当 $\cos y \geq 0$ 时等号成立,故有 $\ln\frac{f_a(x)}{2} \leq \sqrt{b^2 - (b\sin y)^2} + \ln \cosh(b \sin y)$。

令 $t = b \sin y$,则 $t \in [-b, b]$。构造辅助函数 $g(t) = \sqrt{b^2 - t^2} + \ln \cosh t$。由于 $g(t)$ 是偶函数,仅需讨论 $t \in [0, b]$ 上的单调性。对 $g(t)$ 求导得 $g'(t) = \tanh t - \frac{t}{\sqrt{b^2 - t^2}}$,继续求二阶导数得 $g''(t) = \operatorname{sech}^2 t - \frac{b^2}{(b^2 - t^2)^{3/2}}$。观察可知 $\operatorname{sech}^2 t$ 与 $-\frac{b^2}{(b^2 - t^2)^{3/2}}$ 在 $[0, b)$ 上均严格递减,故 $g''(t)$ 在 $[0, b]$ 上严格递减。由于 $g''(0) = 1 - \frac{1}{b}$,最大值的取得需分两种情况讨论:

若 $b \leq 1$(即 $1 < a \leq e^{\sqrt{2}}$),则 $g''(0) \leq 0$。由 $g''(t)$ 严格递减可知 $g''(t) < 0$ 对一切 $t \in (0, b]$ 成立,故 $g'(t)$ 在 $[0, b]$ 上严格递减。又 $g'(0) = 0$,故在 $t \in (0, b]$ 内 $g'(t) < 0$,函数 $g(t)$ 在 $[0, b]$ 上严格递减。此时最大值在 $t = 0$ 处取得,即 $g(0) = b$。对应 $x = 2k\pi + \frac{\pi}{4}$ 时,原函数取得最大值 $2e^b = 2a^{1/\sqrt{2}}$。

若 $b > 1$(即 $a > e^{\sqrt{2}}$),则 $g''(0) > 0$。由于 $g''(t)$ 严格递减且当 $t \to b$ 时 $g''(t) \to -\infty$,存在唯一的 $t_1 \in (0, b)$ 使得 $g''(t_1) = 0$。此时 $g'(t)$ 在 $[0, t_1]$ 上严格递增,在 $[t_1, b]$ 上严格递减。结合 $g'(0) = 0$ 且当 $t \to b$ 时 $g'(t) \to -\infty$,可知在 $(0, b)$ 内存在唯一的 $t_0$ 使得 $g'(t_0) = 0$。此时 $g(t)$ 在 $t_0$ 处取得极大值,亦为最大值。由于 $g(t_0) > g(0) = b$,此情形下最大值大于 $2a^{1/\sqrt{2}}$。

综上所述,当 $1 < a \leq e^{\sqrt{2}}$ 时(如 $a=3$),函数 $f_a(x)$ 的取值范围为 $[2a^{-1/\sqrt{2}}, 2a^{1/\sqrt{2}}]$;当 $a > e^{\sqrt{2}}$ 时,最小值为 $2a^{-1/\sqrt{2}}$,最大值为 $2\exp\left(\sqrt{b^2 - t_0^2} + \ln \cosh t_0\right)$,其中 $t_0$ 为方程 $\tanh t = \frac{t}{\sqrt{b^2 - t^2}}$ 在 $(0, b)$ 内的唯一根。
\end{solution}

\begin{example}{港梦杯第18题原稿}{}
    求$\displaystyle\min_{a\in\mathbb{Q}^+}\{\max_{x\in R}\{\sin x+\sin ax\}\}$.
\end{example}
\begin{solution}
$a$ 是正有理数,设 $a = \frac{p}{q}$,其中 $p, q$ 为互素的正整数。$F(x) = \sin x + \sin\left(\frac{p}{q}x\right)$可以转换为求函数 $f(x)= \sin(qx) + \sin(px)$ 全局最大值的最小值。即我们记 $M(p,q) = \displaystyle\max_{x \in \mathbb{R}} (\sin px + \sin qx)$。目标是求 $\displaystyle\min_{(p,q)} M(p,q)$。和差化积公式得$f(x) = 2\sin\left(\frac{p+q}{2}x\right)\cos\left(\frac{p-q}{2}x\right)$

我们可以通过对不同的 $(p,q)$ 组合进行分类,以类似“采样”的方式,寻找使得正弦项为 1 的特定点列,筛选出唯一的极小值候选点。任意两个互素的正整数 $p, q$,其奇偶性必然属于以下三种情况之一:

情况 1:$p, q$ 均为奇数,且 $p \equiv q \pmod 4$。此时,若 $p \equiv q \equiv 1 \pmod 4$,我们取 $x = \frac{\pi}{2}$。由于 $p = 4n+1$,则 $\sin(px) = \sin\left(2n\pi + \frac{\pi}{2}\right) = 1$。同理 $\sin(qx) = 1$。
此时 $f\left(\frac{\pi}{2}\right) = 1 + 1 = 2$。若 $p \equiv q \equiv 3 \pmod 4$,我们取 $x = \frac{3\pi}{2}$。
由于 $p = 4n+3$,则 $\sin(px) = \sin\left(6n\pi + \frac{9\pi}{2}\right) = \sin\left(2\pi + \frac{\pi}{2}\right) = 1$。同理 $\sin(qx) = 1$。此时 $f\left(\frac{3\pi}{2}\right) = 2$。因此,在此类情况下,始终有 $M(p,q) \ge 2$。

情况 2:$p, q$ 奇偶性不同(一奇一偶)。此时我们构造特定的采样点列 $x_k = \frac{4k+1}{p+q}\pi \quad (k \in \mathbb{Z})$。此时正弦项的相位为 $\frac{p+q}{2}x_k = 2k\pi + \frac{\pi}{2}$,显然 $\sin\left(\frac{p+q}{2}x_k\right) = 1$ 恒成立。简化为:$f(x_k) = 2\cos\left(\frac{p-q}{p+q} \cdot \frac{4k+1}{2}\pi\right)$,记和值 $S = p+q$,差值 $D = p-q$。我们要考察余弦项的相位距离 $2m\pi$ 有多近,定义相位偏差 $\Delta$:
$$\Delta = \frac{D(4k+1)}{2S}\pi - 2m\pi = \frac{\pi}{2S} \left[ 4Dk - 4mS + D \right]$$
因为一奇一偶,故 $S$ 与 $D$ 均为奇数。又因 $\gcd(p,q)=1$,易知 $\gcd(D, S) = 1$。
根据裴蜀定理,当 $k, m$ 遍历整数时,$(Dk - Sm)$ 取遍所有整数。因此 $4(Dk - Sm) + D$ 能够取遍所有模 4 余 $D$ 的整数。
由于 $D$ 是奇数,一个奇数距离 4 的倍数最近的距离必定是 1(即存在整数 $X$ 使 $|4X+D| = 1$)。
因此,必定存在某个 $k$,使得最小绝对相差为 $\Delta_{\min} = \frac{\pi}{2S} \times 1 = \frac{\pi}{2(p+q)}$。
所以,在此类情况下:
$$M(p,q) \ge 2\cos\left(\frac{\pi}{2(p+q)}\right)$$
由于一奇一偶的正整数组合最小的和为 $S = 1+2 = 3$,因此下界为:
$$M(p,q) \ge 2\cos\left(\frac{\pi}{6}\right) = \sqrt{3} \approx 1.732$$

情况 3:$p, q$ 均为奇数,且 $p \not\equiv q \pmod 4$。
此时必定一个是 $4n+1$,另一个是 $4n+3$。
这导致和 $S = p+q$ 必定是 4 的倍数,差 $D = p-q$ 必定是偶数且模 4 余 2。
设 $S = 4S'$,$D = 2D'$(其中 $D'$ 为奇数)。
继续沿用情况 2 中的采样点列 $x_k = \frac{4k+1}{p+q}\pi$,相位偏差转化为:
$$\Delta = \frac{\pi}{2S} \left[ D(4k+1) - 4mS \right] = \frac{\pi}{8S'} \left[ 8D'k + 2D' - 16mS' \right] = \frac{\pi}{4S'} \left[ 4D'k - 8mS' + D' \right]$$
提取公因式得绝对值为 $\frac{\pi}{4S'} |4(D'k - 2mS') + D'|$。
因为 $S=p+q$ 与 $D=p-q$ 均为偶数,且 $p,q$ 互素,可推得 $\gcd\left(\frac{D}{2}, \frac{S}{2}\right) = 1$,即 $\gcd(D', 2S') = 1$。
同理,由裴蜀定理,存在整数使得 $|4(D'k - 2mS') + D'| = 1$。
此时最小相位偏差为 $\Delta_{\min} = \frac{\pi}{4S'} \times 1 = \frac{\pi}{S} = \frac{\pi}{p+q}$。
所以,在此类情况下:
$$M(p,q) \ge 2\cos\left(\frac{\pi}{p+q}\right)$$
由于 $S = p+q$ 必须是 4 的倍数,我们分情况讨论:
\begin{itemize}
    \item 当 $p+q \ge 8$ 时,其下界满足 $M(p,q) \ge 2\cos\left(\frac{\pi}{8}\right) = \sqrt{2+\sqrt{2}} \approx 1.847$。
    \item 当 $p+q = 4$ 时,这是本类别中唯一的例外,此时对应的唯一互素组合为 $\{p, q\} = \{1, 3\}$。
\end{itemize}

综上所述,除了 $\{1, 3\}$ 之外,整个正有理数域内所有其他 $(p,q)$ 组合的 $M(p,q)$ 理论下界均大于 $\sqrt{3} \approx 1.732$。若存在全局最大值的最小值,必定在 $\{1, 3\}$ 处取得。此时对应原问题 $a=3$ 或 $a=\frac{1}{3}$。不难得到答案为$\frac{8\sqrt{3}}{9} $
\end{solution}

\begin{example}{}{}
 $\cos x+\cos y=1$,且 $x,y\in[-\frac{\pi}{2},\frac{\pi}{2}]$。求 $x^2+y^2$ 的取值范围
\end{example}
令$x=\arccos(1-u),y=\arccos u$,则\[x^2+y^2=G(u)=(\arccos(1-u))^2+(\arccos u)^2,\quad u\in[0,1]\]
求导:$$G'(u)=2\arccos(1-u)\cdot\frac{-1}{\sqrt{1-(1-u)^2}}\cdot(-1)+2\arccos u\cdot\frac{-1}{\sqrt{1-u^2}}=\frac{2\arccos(1-u)}{\sqrt{2u-u^2}}-\frac{2\arccos u}{\sqrt{1-u^2}}$$
当 $u\in(0,1)$ 时,对应的 $x, y \in (0,\frac{\pi}{2})$。此时$\sin x = \sqrt{1-\cos^2 x} = \sqrt{1-(1-u)^2} = \sqrt{2u-u^2}$,$\sin y = \sqrt{1-\cos^2 y} = \sqrt{1-u^2}$所以$G'(u)=2\left(\frac{x}{\sin x}-\frac{y}{\sin y}\right)$,根据形式,引入在 $(0,\frac{\pi}{2}]$显然单增的函数 $H(t)=\frac{t}{\sin t}$。当 $u\in(0,\frac{1}{2})$ 时:此时 $1-u > \frac{1}{2} > u$,即 $\cos x > \cos y$,$x < y$, $H(x) < H(y)$,即 $\frac{x}{\sin x} < \frac{y}{\sin y}$。因此 $G'(u) < 0$,$G(u)$ 在区间 $[0,\frac{1}{2}]$ 上严格单调递减。当 $u\in(\frac{1}{2},1)$ 时,此时 $1-u < \frac{1}{2} < u$,即 $\cos x < \cos y$。同理 $x > y$,$H(x) > H(y)$。因此 $G'(u) > 0$,$G(u)$ 在区间 $[\frac{1}{2},1]$ 上单增。

\newpage
\section{零点题}
\begin{example}{“港”命制,零点和小于$2^n-1$}{}
已知函数 $f(x)=x-\ln x-1\ (x>0)$,定义 $f_{n+1}(x)=f_n(f(x))\ (n\in \mathbb{N}^*)$.\\
(1) 求 $f_n(x)-x$ 的零点个数;\\
(2) 证明:$f_n(x)$ 的所有极值点之和不小于 $2^n-1$.
\end{example}
\begin{solution}
由链式法则 $f_n'(x) = \prod_{k=0}^{n-1} f'(f_k(x))$。由于定义域限制,极值点对应的导数为0,必然要求连乘式中某一项为0。但若 $f_k(x)=1\ (k \le n-2)$,点就不在 $f_n$ 的定义域内。因此,能在定义域内使 $f_n'(x)=0$ 的点,当且仅当 $f_{n-1}(x)=1$。(且这些点均满足 $f_n(x)=0$,为极小值点)。令 $A_k = \{x > 0 \mid f_k(x) = 1\}$,则 $A_{n-1}$ 即为 $f_n(x)$ 的所有极值点构成的集合。依题意,$A_0=\{1\}$;$A_k$ 相当于对 $A_{k-1}$ 中的每一个元素,求方程 $f(x)=y$ 的两根。集合 $A_{k-1}$ 的大小为 $2^{k-1}$。
\end{solution}

\begin{example}{邪帝零点题}{}
已知函数 $f(x)=x\ln(x-2)-\ln a$ 的零点为 $x_0$, $g(x)=2ax+\dfrac{x}{\ln x-a}$.  \\
(1) 实数 $x_1$, $x_2$ ($e^a<x_1<x_2$) 满足 $g(x_1)=g(x_2)$,若 $x_1x_2>e^{x_0}$,求 $a$ 的取值范围;  \\
(2) 记函数 $h(x)=f(x)-g(x)$,讨论 $h(x)$ 的零点个数.
\end{example}
\begin{solution}
(1)$f(x) = x\ln(x-2) - \ln a$ 的零点为 $x_0$,故 $x_0 > 2$ 且 $\ln a = x_0\ln(x_0-2)$。函数 $g(x) = 2ax + \frac{x}{\ln x - a}$,且存在 $e^a < x_1 < x_2$ 使得 $g(x_1) = g(x_2)$。出于简化分母的考虑,令 $t = \ln x - a$,由 $x > e^a$ 知 $t > 0$。此时 $x = e^{t+a}$,函数 $g(x)$ 转化为 $g(x) = e^{t+a} \left( 2a + \frac{1}{t} \right) = e^a \cdot e^t \left( 2a + \frac{1}{t} \right)$。

设 $F(t) = e^t \left( 2a + \frac{1}{t} \right)$,则 $F(t_1) = F(t_2)$,其中 $0 < t_1 < t_2$。求导得$F'(t) = e^t \frac{2at^2 + t - 1}{t^2}$。令 $F'(t) = 0$,因 $t>0$ 且 $a>0$,方程 $2at^2+t-1=0$ 有唯一正根 $t_0 = \frac{\sqrt{1+8a}-1}{4a}$。当 $t \in (0, t_0)$ 时,$F'(t) < 0$;当 $t \in (t_0, +\infty)$ 时,$F'(t) > 0$。故 $F(t)$ 在 $t_0$ 处取得极小值,且满足 $0 < t_1 < t_0 < t_2$。

下证 $t_1 + t_2 > 2t_0$。构造 $H(t) = \ln F(t) = t + \ln(2at+1) - \ln t$,则 $H(t_1) = H(t_2)$。求导得:
$$H'(t) = 1 + \frac{2a}{2at+1} - \frac{1}{t} = \frac{2at^2+t-1}{2at^2+t}$$
进一步构造 $\varphi(x) = H(t_0 - x) - H(t_0 + x)$,其中 $0 < x < t_0$。求导并用 $2at_0^2+t_0=1$ 代入得:
$$\varphi'(x) = - \left[ H'(t_0-x) + H'(t_0+x) \right] = -\frac{2(4a^2x^4 - (1+6a)x^2)}{(1+2ax^2)^2 - (1+8a)x^2}$$
由于 $0 < x < t_0$,有 $x^2 < t_0^2 = \frac{1+4a-\sqrt{1+8a}}{8a^2}$。此时 $4a^2x^2 < 4a^2t_0^2 < 1+6a$,可知$\varphi'(x)$分子小于 $0$,分母大于 $0$。从而 $\varphi'(x) > 0$,故 $\varphi(x) > \varphi(0) = 0$,即 $H(t_0 - x) > H(t_0 + x)$。令 $x = t_0 - t_1$,得 $H(t_1) > H(2t_0 - t_1)$。结合 $H(t_1) = H(t_2)$ 可得 $H(t_2) > H(2t_0 - t_1)$。由于 $t_2$ 与 $2t_0 - t_1$ 均位于 $H(t)$ 的单增区间 $(t_0, +\infty)$ 内,故 $t_2 > 2t_0 - t_1$,即 $t_1 + t_2 > 2t_0$。

由 $x_1 x_2 = e^{t_1+a} \cdot e^{t_2+a} = e^{t_1+t_2+2a}> e^{2t_0+2a}$。为使 $x_1 x_2 > e^{x_0}$ 恒成立,必要性探路 $e^{2t_0+2a} \geqslant e^{x_0}$,即 $2t_0 + 2a \geqslant x_0$。显然函数 $x \ln(x-2)$ 在 $(2, +\infty)$ 上单增,条件等价于 $f(2t_0+2a) \geqslant f(x_0) = 0$。

由 $2at_0^2+t_0-1=0$ 得 $a = \frac{1-t_0}{2t_0^2}$,代入得 $2t_0+2a = \frac{2t_0^3 - t_0 + 1}{t_0^2}$。设 $A(t_0) = f(2t_0+2a) = (2t_0+2a) \ln(2t_0+2a-2) - \ln a$,需使得 $A(t_0) \geqslant 0$ 恒成立。对 $A(t_0)$ 求导可知 $A'(t_0) < 0$ 恒成立。当 $a = 1$ 时,对应 $t_0 = \frac{1}{2}$,$2t_0+2a = 3$,此时 $A(1/2) = 0$。为保证 $A(t_0) \geqslant 0$,必须满足 $t_0 \le \frac{1}{2}$。又因 $a = \frac{1-t_0}{2t_0^2}$ 在 $(0, 1)$ 上随 $t_0$ 单调递减,故 $t_0 \le \frac{1}{2}$ 等价于 $a \geqslant 1$。因此,$a$ 的取值范围为 $[1, +\infty)$。
\newpage
(2)原方程等价于 $W(x) = \frac{f(x) - g(x)}{x} = \ln(x-2) - \frac{\ln a}{x} - 2a - \frac{1}{\ln x - a} = 0$。对 $W(x)$ 求导:
$$W'(x) = \frac{1}{x-2} + \frac{\ln a}{x^2} + \frac{1}{x(\ln x - a)^2}$$
先考察定义域避免踩坑,显然函数 $W(x)$ 的定义域需满足 $x > 2$ 且 $x \ne e^a$。

当 $a > \ln 2$ 时,$e^a > 2$,定义域为 $(2, e^a) \cup (e^a, +\infty)$。考察 $W'(x)$ 的前两项通分得 $\frac{x^2 + (x-2)\ln a}{x^2(x-2)}$。设分子为 $p(x) = x^2 + (x-2)\ln a$,其对称轴为 $x = -\frac{\ln a}{2}$。由于 $a > \ln 2$,不难得到对称轴严格小于 $\frac{1}{2}$,故 $p(x)$ 在 $x > 2$ 上严格单调递增。又因 $p(2) = 4 > 0$,故对任意 $x > 2$ 均有 $p(x) > 0$。从而 $W'(x) > 0$ 恒成立,函数 $W(x)$ 在区间 $(2, e^a)$ 和 $(e^a, +\infty)$ 内均严格单调递增。在 $(2, e^a)$ 内,当 $x \to 2^+$ 时 $W(x) \to -\infty$;当 $x \to (e^a)^-$ 时 $W(x) \to +\infty$,存在 $1$ 个零点。在 $(e^a, +\infty)$ 内,当 $x \to (e^a)^+$ 时 $W(x) \to -\infty$;当 $x \to +\infty$ 时 $W(x) \to +\infty$,也存在 $1$ 个零点。故此时 $h(x)$ 存在 $2$ 个零点。

当 $0 < a \leqslant \ln 2$ 时,$e^a \leqslant 2$,此时均有 $\ln x > \ln 2 \geqslant a$,定义域为 $(2, +\infty)$。当 $x \to 2^+$ 时 $W(x) \to -\infty$;当 $x \to +\infty$ 时 $W(x) \to +\infty$。若存在极值点 $x_1$ 使得 $W'(x_1) = 0$,则有 $\frac{\ln a}{x_1} = -\frac{x_1}{x_1-2} - \frac{1}{(\ln x_1 - a)^2}$。将其代入 $W(x_1)$ 得:
$$W(x_1) = \ln(x_1-2) + \frac{x_1}{x_1-2} + \frac{1}{(\ln x_1 - a)^2} - \frac{1}{\ln x_1 - a} - 2a$$
由于 $x_1 - 2 > 0$,有 $\ln(x_1-2) + \frac{x_1-2+2}{x_1-2} = \ln(x_1-2) + \frac{2}{x_1-2} + 1 \geqslant \ln 2 + 2$。又设 $t = \ln x_1 - a > 0$,则 $\frac{1}{t^2} - \frac{1}{t} = \left(\frac{1}{t} - \frac{1}{2}\right)^2 - \frac{1}{4} \geqslant -\frac{1}{4}$。因此对于任意极值点,有 $W(x_1) \geqslant \ln 2 + 2 - \frac{1}{4} - 2a \geqslant \ln 2 + \frac{7}{4} - 2\ln 2 > 0$。这表明 $W(x)$ 的所有极小值均严格大于 $0$。由于 $x \to 2^+$ 时函数从 $-\infty$ 开始单调上升,在到达首个极小值的过程中穿过 $x$ 轴一次,此后函数值恒为正。故此时仅存在 $1$ 个零点。

综上所述,当 $0 < a \leqslant \ln 2$ 时,$h(x)$ 有 $1$ 个零点;当 $a > \ln2$ 时,$h(x)$ 有2个零点.
\end{solution}

\begin{example}{邪帝零点题}{}
已知函数$f(x)=\frac{x-a}{\ln x}$,$g(x)=(x+a)^2+a$,其中$a\in R$.\\
(1)当$a=0$时,若曲线$f(x)$与直线$y=k$有一个交点,求实数$k$的取值范围;\\
(2)设函数$h(x)=f(x)+g(x)$,讨论$h(x)$的零点个数.
\end{example}
\begin{solution}

\end{solution}


\newpage
\begin{example}{来自amare Donata Caesia}{}
    函数 $f(x) = \log_a(\ln(x + a))$, $g(x) = \ln(x + b)$ 共零点.\\
(1)求 $b - a$ 的值;(2)若 $f(x) \geq g(x)$ 在区间 $(-b, 2]$ 上恒成立, 求实数 $a$ 的取值范围.
\end{example}
\begin{solution}
(1)对于函数 $f(x) = \log_a(\ln(x + a))$,满足 $\ln(x + a) > 0$,即 $x > 1 - a$。令 $f(x) = 0$解得其唯一零点 $x_1 = e - a$。对于函数 $g(x) = \ln(x + b)$,其定义域为 $x > -b$,令 $g(x) = 0$ 解得其唯一零点 $x_2 = 1 - b$。共零点则 $e - a = 1 - b$,整理得 $b - a = 1 - e$。\\
(2)令 $t = x + a$,由于 $x \in (e - 1 - a, 2]$,则 $t \in (e - 1, a + 2]$。原不等式可转化为 $H(t) = \frac{\ln(\ln t)}{\ln a} - \ln(t + 1 - e) \ge 0$。显然,当 $t = e$ 时,$H(e) = \log_a(\ln e) - \ln(e + 1 - e) = 0$。

当 $0 < a < 1$ 时,$\ln a < 0$。若 $a \le e - 2$,则区间右端点 $a + 2 \le e$。此时对于任意 $t \in (e - 1, a + 2]$,均有 $t \le e$,则 $\ln(\ln t) \le 0$。由于 $\ln a < 0$,得 $\frac{\ln(\ln t)}{\ln a} \ge 0$。同时,$t \le e$ 导致 $t + 1 - e \le 1$,故 $\ln(t + 1 - e) \le 0$,即 $-\ln(t + 1 - e) \ge 0$。此时 $H(t)$ 为两个非负项之和,恒有 $H(t) \ge 0$。若 $e - 2 < a < 1$,则区间包含 $t > e$ 的部分,此时若取 $t = a + 2 > e$,则 $\ln(\ln t) > 0$ 且 $\ln(t + 1 - e) > 0$,导致 $H(a + 2) < 0$,不满足题意。故此种情况下 $a \in (0, e - 2]$。

当 $a > 1$ 时,区间 $(e - 1, a + 2]$ 包含内部点 $t = e$。若 $H(t) \ge 0$ 在该区间内恒成立且 $H(e) = 0$,则 $t = e$ 必须为 $H(t)$ 的极小值点。根据费马引理,必有 $H'(e) = 0$。对 $H(t)$ 求导得:
$$H'(t) = \frac{1}{t \ln t \ln a} - \frac{1}{t + 1 - e}$$
令 $H'(e) = \frac{1}{e \ln a} - 1 = 0$,解得 $\ln a = \frac{1}{e}$,即 $a = e^{\frac{1}{e}}$。现检验 $a = e^{\frac{1}{e}}$ 的充分性:此时 $H'(t) = \frac{e(t + 1 - e) - t \ln t}{t \ln t (t + 1 - e)}$。设分子为 $u(t) = e(t + 1 - e) - t \ln t$,则 $u'(t) = e - 1 - \ln t$。在 $t \in (e - 1, a + 2]$ 上,由于 $a = e^{\frac{1}{e}} < 2$,故 $t < 4$。而 $e^{e - 1} > e^{1.7} > 4$,故在该区间内 $\ln t < e - 1$,得 $u'(t) > 0$。由于 $u(t)$ 单调递增且 $u(e) = 0$,故当 $t \in (e - 1, e)$ 时 $H'(t) < 0$,当 $t \in (e, a + 2]$ 时 $H'(t) > 0$。因此 $H(t)$ 在 $t = e$ 处取得最小值 $H(e) = 0$,满足 $H(t) \ge 0$ 恒成立。

综上所述,$b - a = 1 - e$,实数 $a$ 的取值范围是 $(0, e - 2] \cup \{ e^{\frac{1}{e}} \}$。
\end{solution}

\newpage
\begin{example}{来自扛把子}{}
已知函数$f(x)=e^{ax}+\ln\left(x+1\right)-2$tan$x-1.$ $x\in\left(-1,\frac\pi2\right).$\\
(1) 当 $a=0$ 时,讨论$f(x)$ 的零点个数.\\
(2) 证明:当 $a\leqslant1$ 时,$f(x)$ 有且仅有两个零点.
\end{example}
\begin{solution}
(1)当 $a=0$ 时, $f(x) = \ln(x+1) - 2\tan x$。定义域 $x \in \left(-1, \frac{\pi}{2}\right)$。显然 $f(0) = \ln 1 - 2\tan 0 = 0$,故 $x=0$ 是函数的一个零点。
求导:$$f'(x) = \frac{\cos^2 x - 2x - 2}{(x+1)\cos^2 x}=\frac{h(x)}{(x+1)\cos^2 x},h'(x) = -2\sin x \cos x - 2 = -\sin(2x) - 2$$
注意到$h'(x) \leqslant -1 < 0$ 恒成立。故 $h(x)$ 在 $\left(-1, \frac{\pi}{2}\right)$ 上单减,$h(-1) = \cos^2(-1) > 0$,$h(0) = 1 - 0 - 2 = -1 < 0$,所以对任意 $x > 0$,都有 $h(x) < 0 \implies f'(x) < 0$,所以$f(x)$ 在 $\left(0, \frac{\pi}{2}\right)$ 上严格单调递减,又 $f(0)=0$,可知该区间$f(x) < 0$,无零点。

显然,存在唯一的实数 $x_0 \in (-1, 0)$ 使得 $h(x_0) = 0$。当 $x \in (-1, x_0)$ 时,$h(x)>0 \implies f'(x) > 0$,$f(x)$ 单调递增;当 $x \in (x_0, 0)$ 时,$h(x)<0 \implies f'(x) < 0$,$f(x)$ 单调递减。因 $f(0) = 0$ 且 $f(x)$ 在 $(x_0, 0)$ 上递减,必然有极大值 $f(x_0) > 0$。又因当 $x \to -1^+$ 时,$\ln(x+1) \to -\infty$,即 $f(-1^+) \to -\infty$。因为 $f(x)$ 在 $(-1, x_0)$ 上从 $-\infty$ 单调递增至 $f(x_0) > 0$,必定且仅必定穿过 $x$ 轴一次。因此该区间有且仅有 $1$ 个零点。综上,当 $a=0$ 时,$f(x)$ 有且仅有 $2$ 个零点。\\
(2)当 $a \leqslant 1$ 时,仍有$f(0) = e^0 + \ln 1 - 0 - 1 = 0$,不难想到分正负区间讨论。对于 $x > 0$ 且 $a \leqslant 1$,有$e^{ax} \leqslant e^x$。从而有:$$f(x) \leqslant e^x + \ln(x+1) - 2\tan x - 1 \triangleq f_1(x)$$
求导发现在$(0,\frac{\pi}2)$区间,有\[f_1'(x) = e^x + \frac{1}{x+1} - \frac{2}{\cos^2 x}\leqslant(1 + x + x^2)+(1 - x + x^2)-(2 + 2x^2)=0\]
所以$f_1(x)$ 在 $\left(0, \frac{\pi}{2}\right)$ 上单减。又 $f_1(0)=0$,所以$(0,\frac{\pi}2)$区间无零点。

现在只需想方设法证明$(-1,0)$区间有一个零点就可以。令\[y(x) = 1 + 2\tan x - \ln(x+1),y'(x) = \frac{2}{\cos^2 x} - \frac{1}{x+1},y''(x) = \frac{4\sin x}{\cos^3 x} + \frac{1}{(x+1)^2}\]
令 $t = -x$,则 $t \in (0, 1)$。此时
\begin{align*}
y''(t) &= \frac{1}{(1-t)^2} - \frac{4\sin t}{\cos^3 t}>\frac{1}{(1-t)^2}-\frac{4t}{(1-\frac{t^2}2)^3}\\
&=\dfrac{1 - 4t + \frac{13}{2}t^2 - 4t^3 + \frac{3}{4}t^4 - \frac{1}{8}t^6}{(1-t)^2(1-\frac{t^2}2)^3}=\frac{(1-t)^4 + \frac{t^2}{8}(4 - 2t^2 - t^4)}{(1-t)^2(1-\frac{t^2}2)^3}>0
\end{align*}
据泰勒中值定理,对于任意 $x \in (-1, 0)$存在中值 $c \in (x, 0)$使得:$$y(x) = y(0) + y'(0)x + \frac{y''(c)}{2}x^2= 1 + x + \frac{y''(c)}{2}x^2>0$$


\end{solution}