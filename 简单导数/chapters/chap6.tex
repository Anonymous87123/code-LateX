

\chapter{伪装导数题}
\section{概率}
\begin{example}{}{}
事件 $A$、$B$ 是随机试验中的两个事件,$P(A)$、$P(B) \neq 0$,且 $P(A)+P(B)=\frac{1}{2}$,\[\frac{1}{P(B|A)}+\frac{1}{P(A|B)}=\frac{1}{P(B)-P(AB)}+\frac{1}{P(A)-P(AB)}+2\]下列说法正确的是\\
\begin{tabular}{@{}p{0.45\textwidth} p{0.45\textwidth}@{}}
A. 事件 $A$、$B$ 一定不相互独立 & B. 若 $P(A)=\frac{1}{4}$,则 $P(AB)=\frac{1}{16}$ \\
C. $P(AB)\in\left(0,\frac{3-2\sqrt{2}}{4}\right]$ & D. $P(AB)\in\left(0,\frac{\sqrt{17}-1}{4}\right]$
\end{tabular}
\end{example}
\begin{solution}
设 $P(A) = x$,$P(B) = y$,$P(AB) = z$。则 $x > 0$,$y > 0$ 且 $x+y = \frac{1}{2}$。条件概率 $P(B|A) = \frac{P(AB)}{P(A)} = \frac{z}{x}$, $P(A|B) = \frac{P(AB)}{P(B)} = \frac{z}{y}$。
由于原等式中包含 $\frac{1}{P(B|A)}$ 等项作分母,可知 $z \neq 0$,即 $z > 0$。同时分母 $P(A)-P(AB) = x-z \neq 0$ 且 $P(B)-P(AB) = y-z \neq 0$,说明 $z < x$ 且 $z < y$。结合 $x+y=\frac{1}{2}$,必然有 $z < \frac{1}{4}$。将上述变量代入原等式:
\[\frac{x}{z} + \frac{y}{z} = \frac{1}{2z}=\frac{1}{y-z} + \frac{1}{x-z} + 2\Leftrightarrow xy = \frac{3}{2}z - z^2\]
已知 $x+y = \frac{1}{2}$ 且 $xy = \frac{3}{2}z - z^2$。根据韦达定理,$x$ 和 $y$ 是关于 $t$ 的一元二次方程 $t^2 - \frac{1}{2}t + \left(\frac{3}{2}z - z^2\right) = 0$ 的两个实数根。
为了保证实数根存在,判别式必须满足 $\Delta \ge 0$,即 $16z^2 - 24z + 1 \ge 0$。
解得 $z \le \frac{3-2\sqrt{2}}{4}$ 或 $z \ge \frac{3+2\sqrt{2}}{4}$。
结合前提 $0 < z < \frac{1}{4}$,且 $\frac{3-2\sqrt{2}}{4} \approx 0.0428 < \frac{1}{4}$,可得 $z$ 的严格取值范围为:
\[z \in \left(0, \frac{3-2\sqrt{2}}{4}\right]\]
\begin{itemize}
    \item \textbf{选项 A:}假设事件 $A, B$ 相互独立,则 $P(AB) = P(A)P(B)$,即 $z = xy$。代入推导出的关系式 $xy = \frac{3}{2}z - z^2$,得 $z = \frac{3}{2}z - z^2 \implies z^2 - \frac{1}{2}z = 0$。解得 $z=0$ 或 $z=\frac{1}{2}$。但这都不在有效定义域 $\left(0, \frac{3-2\sqrt{2}}{4}\right]$ 内。因此假设不成立,事件 $A, B$ 一定不相互独立,\textbf{故 A 正确}。
    \item \textbf{选项 B:}若 $P(A) = \frac{1}{4}$,由 $P(A)+P(B)=\frac{1}{2}$ 得 $P(B) = \frac{1}{4}$。此时 $xy = \frac{1}{16}$。代入关系式有 $\frac{1}{16} = \frac{3}{2}z - z^2$,化简为 $16z^2 - 24z + 1 = 0$。解得 $z = \frac{3-2\sqrt{2}}{4}$(另一根大于 $\frac{1}{4}$ 故舍去)。因为 $\frac{3-2\sqrt{2}}{4} \neq \frac{1}{16}$,\textbf{故 B 错误}。
    \item \textbf{选项 C:}$P(AB) = z$ 的取值范围确为 $\left(0, \frac{3-2\sqrt{2}}{4}\right]$,\textbf{故 C 正确}。
    \item \textbf{选项 D:}由于C正确,\textbf{故 D 正确}。
\end{itemize}
综上所述,正确的说法是 ACD。
\end{solution}


\begin{example}{}{}
已知无穷数列$\left\{a_n\right\}$满足$a_n\in\mathbf{N}^*$,且$\left\{a_n\right\}$前$n$项的中位数为$n.$\\
(1)若$\{a_n\}$是递增数列,求$\{a_n\}$的通项公式;\\
(2)证明:$a_n\geqslant n$ ;\\
(3)给定$k\in\mathbb{N}^*$,求关于$n$的方程$a_n=k$所有可能的解集.
\end{example}
\begin{solution}
(1) 由于 $\{a_n\}$ 是递增数列,当 $n=2k-1$ 时,前 $2k-1$ 项的中位数为第 $k$ 项,故 $a_k = 2k-1$;当 $n=2k$ 时,前 $2k$ 项的中位数为 $\frac{a_k + a_{k+1}}{2} = 2k$。将 $a_k = 2k-1$ 代入解得 $a_{k+1} = 2k+1$。故 $\{a_n\}$ 的通项公式为 $a_n = 2n-1$。

(2) 记 $S_n = \{a_1, a_2, \dots, a_n\}$,将其元素升序排列为 $x_{n, 1} \leqslant x_{n, 2} \leqslant \dots \leqslant x_{n, n}$,并记 $C_n(x)$ 为 $S_n$ 中不超过 $x$ 的元素个数。
对于 $n=2k-1$,中位数为 $x_{2k-1, k} = 2k-1$,说明 $C_{2k-1}(2k-2) \leqslant k-1$ 且 $C_{2k-1}(2k-1) \geqslant k$。
对于 $n=2k$,中位数满足 $x_{2k, k} + x_{2k, k+1} = 4k$。因为 $x_{2k, k} \leqslant 2k \leqslant x_{2k, k+1}$,无论 $x_{2k, k} = 2k$ 还是 $x_{2k, k} \leqslant 2k-1$,均有 $C_{2k}(2k-1) \leqslant k$。
由于 $S_{2k} = S_{2k-1} \cup \{a_{2k}\}$,故 $k \geqslant C_{2k}(2k-1) = C_{2k-1}(2k-1) + I(a_{2k} \leqslant 2k-1) \geqslant k + I(a_{2k} \leqslant 2k-1)$,这迫使 $I(a_{2k} \leqslant 2k-1) = 0$,即 $a_{2k} \geqslant 2k$。
同理,对于 $n=2k-2\ (k \geqslant 2)$,中位数满足 $x_{2k-2, k-1} + x_{2k-2, k} = 4k-4$。若 $x_{2k-2, k} \leqslant 2k-2$,则必有 $x_{2k-2, k-1} = x_{2k-2, k} = 2k-2$,导致 $C_{2k-2}(2k-2) \geqslant k$。但 $S_{2k-1}$ 包含 $S_{2k-2}$,这会导致 $k-1 \geqslant C_{2k-1}(2k-2) \geqslant C_{2k-2}(2k-2) \geqslant k$ 的矛盾。
故 $x_{2k-2, k} \geqslant 2k-1$,使得 $C_{2k-2}(2k-2) = k-1$。进而 $k-1 \geqslant C_{2k-1}(2k-2) = C_{2k-2}(2k-2) + I(a_{2k-1} \leqslant 2k-2) = k-1 + I(a_{2k-1} \leqslant 2k-2)$,迫使 $I(a_{2k-1} \leqslant 2k-2) = 0$,即 $a_{2k-1} \geqslant 2k-1$。结合 $a_1=1 \geqslant 1$,对任意 $n \in \mathbb{N}^*$ 均有 $a_n \geqslant n$。

(3) 由(2)推导可知 $C_{2m}(2m-1) = m$ 且 $x_{2m, m} \leqslant 2m-1$,结合 $x_{2m, m} + x_{2m, m+1} = 4m$ 推得 $x_{2m, m+1} \geqslant 2m+1$。这意味着数列中不可能出现偶数,否则若 $a_n = 2m$,由 $a_n \geqslant n$ 必有 $n \leqslant 2m$,导致 $2m \in S_{2m}$,产生矛盾。
因此,当 $k$ 为偶数时,可能解集仅有 $\emptyset$。
当 $k$ 为奇数时,令 $k = 2m-1$。为满足前 $2m$ 项恰有 $m$ 个数 $\leqslant 2m-1$ 且无偶数,这 $m$ 个数必须是 $1, 3, \dots, 2m-1$ 的无重复排列。同理分析 $n=2m-2$ 时的中位数可知 $x_{2m-2, m-1} = 2m-3$,故 $x_{2m-2, m} = 2m-1$。这说明元素 $2m-1$ 必定已出现在 $S_{2m-2}$ 中,其位置 $n \leqslant 2m-2 = k-1$。
结合无重复元素及 $a_n \geqslant n$ 的约束:当 $k=1$ 时,$n \leqslant 1$,解集仅有 $\{1\}$;当 $k=3$ 时,$n \leqslant 2$ 且 $a_1=1$,解集仅有 $\{2\}$。对于奇数 $k \geqslant 5$,因为 $a_1, a_2$ 已被固定为 $1, 3$,新元素 $n$ 的范围放宽至 $3 \leqslant n \leqslant k-1$。在此范围内,总能通过灵活安排后续冗余奇数来构造合法的数列,使得 $a_n=k$ 成立。
综上,关于 $n$ 的方程 $a_n=k$ 所有可能的解集为:当 $k$ 为偶数时为 $\emptyset$;当 $k=1$ 时为 $\{1\}$;当 $k=3$ 时为 $\{2\}$;当奇数 $k \geqslant 5$ 时为 $\{3\}, \{4\}, \dots, \{k-1\}$。
\end{solution}



\begin{example}{邪帝原创导数题}{}
    已知函数$f(x)=x-(a+1)\ln x-\frac{a}{x},(a>1)$\\
    (1)讨论$f(x)$的单调性;\\
    (2)若$f(x_1)=f(x_2)=f(x_3),(x_1<x_2<x_3)$,证明$f(x_1x_2x_3)>1-a$.
\end{example}

\begin{solution}
第一步:找出隐藏的“完美对称”
原函数 $f(x) = x - (a + 1)\ln x - \frac{a}{x}$ 包含对数和分式,看起来毫无对称性。但我们可以通过一个极其优美的代数换元,将其彻底“拉平”。
令 $x = \sqrt{a}e^t$,这样 $x \in (0, +\infty)$ 就被映射到了 $t \in (-\infty, +\infty)$。我们把这个代入原函数:
$$f(\sqrt{a}e^t) = \sqrt{a}e^t - (a + 1)\left(t + \frac{1}{2}\ln a\right) - \sqrt{a}e^{-t}$$
$$
= 2\sqrt{a}\left(\frac{e^t - e^{-t}}{2}\right) - (a + 1)t - \frac{a + 1}{2}\ln a$$
提取出常数项,我们构造一个新函数 $h(t)$:
$$h(t) = f(\sqrt{a}e^t) + \frac{a + 1}{2}\ln a = 2\sqrt{a}\sinh t - (a + 1)t$$

奇迹出现了:$h(t)$ 是一个完美的奇函数!即 $h(-t) = -h(t)$。

原函数 $f(x)$ 所有的非对称和扭曲,在指数换元下,全都被吸收成了完美的中心对称!
第二步:将“三元乘积”转化为“三根之和”
已知 $f(x_1) = f(x_2) = f(x_3) = c$。
在我们构造的新世界里,这就等价于直线 $y = C$ (其中 $C = c + \frac{a+1}{2} \ln a$)与奇函数 $h(t)$ 交于三个点 $t_1 < t_2 < t_3$。
因为 $x_i = \sqrt{a} e^{t_i}$,所以我们的三个根的乘积变成了:
$$x_1 x_2 x_3 = \left( \sqrt{a} e^{t_1} \right) \left( \sqrt{a} e^{t_2} \right) \left( \sqrt{a} e^{t_3} \right) = a^{\frac{3}{2}} e^{t_1 + t_2 + t_3}$$
再来看目标不等式右侧的 $x_0$。
题目第一问求出极大值为 $f(1) = 1 - a$,$x_0$ 是 $(a, +\infty)$ 上满足 $f(x_0) = 1 - a$ 的根。
在 $t$ 的世界里,$x = 1$ 对应着 $t_A = -\frac{1}{2} \ln a$(这是 $h(t)$ 的极大值点)。
所以 $x_0$ 对应着某个正数 $t_B > 0$,满足 $h(t_B) = h(t_A) = C_{\max}$。
即:$x_0 = \sqrt{a} e^{t_B}$。
现在,我们把要证的目标不等式 $x_1 x_2 x_3 > x_0$ 翻译成 $t$ 的语言:
$$
a^{\frac{3}{2}} e^{t_1 + t_2 + t_3} > \sqrt{a} e^{t_B}$$
两边同除以 $\sqrt{a}$ 并取对数:
$$t_1 + t_2 + t_3 + \ln a > t_B$$
前面说过 $t_A = -\frac{1}{2} \ln a$,所以 $\ln a = -2 t_A$。代入上式,目标不等式精准等价于:
$$
t_1 + t_2 + t_3 > 2 t_A + t_B$$
你看,原本复杂无比的三个根的乘积与 $x_0$ 的比较,在对称换元后,完美地变成了一元奇函数三个根的和与极限极值点的线性比较!$\ln a$ 这个恶心的常数项甚至自动抵消了!
第三步:一元单变量的终极秒杀(拐点偏移)
现在问题变成了纯粹的函数图像性质:
已知奇函数 $h(t) = 2\sqrt{a}\sinh t - (a + 1)t$,其极大值点为 $t_A < 0$。水平线 $y = C$ 与其交于 $t_1, t_2, t_3$ 三点。求证:三根之和 $S(C) = t_1 + t_2 + t_3 > 2t_A + t_B$。
既然 $S(C)$ 是一个关于水平线高度 $C$ 的单变量函数,我们只需要研究它的单调性:
当 $C$ 不断向上平移,逼近极大值 $C_{\max}$ 时,交点会发生什么变化?
显然,$t_1$ 和 $t_2$ 都在向极大值点 $t_A$ 靠拢,而 $t_3$ 在向 $t_B$ 靠拢。
所以当 $C \to C_{\max}$ 时,极限状态刚好就是:
$$S(C_{\max}) = t_A + t_A + t_B = 2t_A + t_B$$
目标等式右侧的 $2t_A + t_B$,竟然就是三根之和 $S(C)$ 的下确界!
由于 $h(t)$ 在 $(-\infty, 0)$ 上是严格上凸的 ($h''(t) = 2\sqrt{a}\sinh t < 0$),根据拐点偏移的经典结论(或者对 $S(C)$ 求导可知 $S'(C) = \sum \frac{1}{h'(t_i)} < 0$),三根之和 $S(C)$ 是关于 $C$ 的严格单调递减函数。
因为 $c < f(1)$,即水平线 $C < C_{\max}$,所以必然有:
$$
S(C) > S(C_{\max}) \implies t_1 + t_2 + t_3 > 2t_A + t_B$$
将其还原回 $x$,即得证 $f(x_1 x_2 x_3) > 1 - a$。
\end{solution}
\begin{example}{}{}
    已知函数 $f(x)=\ln x+ax^{2}+bx(a,b\in R)$.\\
(1) 若 $a=\frac{1}{e^{4}}, b=-\frac{3}{e^{2}}$, 求函数 $y=f(x)$ 的单调递减区间;\\
(2) 若存在实数 $b$, 使得函数 $f(x)$ 有三个不同的零点 $x_{1}, x_{2}, x_{3}$.求 $a$ 的取值范围;若 $x_{1}, x_{2}, x_{3}$ 成等差数列, 求证: $x_{2}^{2}>e^{3}$.
\end{example}
\begin{solution}
(1) 求函数 $y=f(x)$ 的单调递减区间
\\
由题意知,函数 $f(x) = \ln x + ax^2 + bx$ 的定义域为 $x \in (0, +\infty)$。
当 $a = \frac{1}{e^4}, b = -\frac{3}{e^2}$ 时,函数解析式为:
$$f(x) = \ln x + \frac{1}{e^4}x^2 - \frac{3}{e^2}x$$
\\
对其求导,并进行通分整理:
$$f'(x) = \frac{1}{x} + \frac{2}{e^4}x - \frac{3}{e^2} = \frac{2x^2 - 3e^2x + e^4}{e^4 x}$$
由于在定义域内 $x > 0$ 且 $e^4 > 0$,所以分母 $e^4 x > 0$ 恒成立。
\\
令 $f'(x) \le 0$,只需分子满足:
$$2x^2 - 3e^2x + e^4 \le 0$$
利用十字相乘法对左侧二次多项式进行因式分解:
$$(2x - e^2)(x - e^2) \le 0$$
解得两根之间的区域:
$$\frac{e^2}{2} \le x \le e^2$$
\\
结论: 函数 $y=f(x)$ 的单调递减区间为 $\left[ \frac{e^2}{2}, e^2 \right]$。
\\
\\
(2) ① 求 $a$ 的取值范围
\\
命题“存在实数 $b$ 使得函数 $f(x)$ 有三个不同的零点”,等价于方程 $\ln x + ax^2 + bx = 0$ 在 $(0, +\infty)$ 上有三个不同的实数根。
将原方程等式两边同除以 $x$(因为 $x>0$),分离出参数 $b$:
$$-b = \frac{\ln x}{x} + ax$$
令 $g(x) = \frac{\ln x}{x} + ax \quad (x>0)$。
问题转化为:水平直线 $y=-b$ 与函数 $y=g(x)$ 的图像存在 3 个不同的交点。
\\
要使直线与函数图像交于 3 点,函数 $g(x)$ 必须既有局部极大值又有局部极小值。
这就要求其导函数 $g'(x) = 0$ 至少有两个不同的正实数根。
对 $g(x)$ 求导:
$$g'(x) = \frac{\frac{1}{x} \cdot x - \ln x \cdot 1}{x^2} + a = \frac{1 - \ln x}{x^2} + a = \frac{ax^2 - \ln x + 1}{x^2}$$
令分子为 $h(x) = ax^2 - \ln x + 1 \quad (x>0)$,则 $g'(x)$ 的符号完全由 $h(x)$ 决定。
\\
我们要使 $h(x) = 0$ 有两个不同的正根。
对 $h(x)$ 继续求导:
$$h'(x) = 2ax - \frac{1}{x} = \frac{2ax^2 - 1}{x}$$
\\
对参数 $a$ 的符号进行严格分类讨论:
\\
情形一:若 $a \le 0$
对于任意 $x > 0$,$2ax^2 \le 0 \implies 2ax^2 - 1 < 0$,因此 $h'(x) < 0$ 恒成立。
这说明 $h(x)$ 在 $(0, +\infty)$ 上严格单调递减,故 $h(x)=0$ 至多只有 1 个根。
这会导致 $g'(x)=0$ 至多只有 1 个变号零点,这意味着 $g(x)$ 的图像至多只有一次折返,任意水平直线 $y=-b$ 最多只能与 $g(x)$ 产生 2 个交点,绝对无法产生三个零点,矛盾!
因此,必须有 $a > 0$。
\\
情形二:若 $a > 0$
令 $h'(x) = 0$,解得唯一驻点 $x_0 = \frac{1}{\sqrt{2a}}$。
当 $x \in \left(0, \frac{1}{\sqrt{2a}}\right)$ 时,$h'(x) < 0$,$h(x)$ 单调递减;
当 $x \in \left(\frac{1}{\sqrt{2a}}, +\infty\right)$ 时,$h'(x) > 0$,$h(x)$ 单调递增。
因此,$h(x)$ 在 $x = \frac{1}{\sqrt{2a}}$ 处取得全局唯一极小值(也是最小值):
$$h_{min} = h\left(\frac{1}{\sqrt{2a}}\right) = a\left(\frac{1}{2a}\right) - \ln\left(\frac{1}{\sqrt{2a}}\right) + 1 = \frac{3}{2} + \frac{1}{2}\ln(2a)$$
要想让 $h(x)=0$ 能穿过 $x$ 轴产生两个不同的正根,必须要求该最小值严格小于 0:
$$\frac{3}{2} + \frac{1}{2}\ln(2a) < 0 \implies \ln(2a) < -3 \implies 2a < e^{-3} \implies 0 < a < \frac{1}{2e^3}$$
\\
充分性论证(确保完全闭环):
当 $0 < a < \frac{1}{2e^3}$ 时,$h_{min} < 0$。
同时计算边界极限:
当 $x \to 0^+$ 时,$\ln x \to -\infty \implies h(x) \to +\infty$;
当 $x \to +\infty$ 时,因为 $a>0$,多项式增长压倒对数函数 $\implies h(x) \to +\infty$。
由连续函数零点定理,$h(x)$ 恰有两个不同的根 $x_1', x_2'$ 且 $0 < x_1' < x_0 < x_2'$。
此时 $g'(x)$ 在 $(0, x_1')$ 为正,在 $(x_1', x_2')$ 为负,在 $(x_2', +\infty)$ 为正。
由此可知 $g(x)$ 在 $x_1'$ 处取得极大值,在 $x_2'$ 处取得极小值,且 $g(x_1') > g(x_2')$。
再看 $g(x) = \frac{\ln x}{x} + ax$ 的极限:
由于 $a>0$,当 $x \to 0^+$ 时 $g(x) \to -\infty$;当 $x \to +\infty$ 时 $g(x) \to +\infty$。
根据介值定理,只要我们选取 $-b \in (g(x_2'), g(x_1'))$,直线 $y=-b$ 必然与 $y=g(x)$ 恰好交于 3 个不同的点。
\\
结论: $a$ 的取值范围为 $\left( 0, \frac{1}{2e^3} \right)$。
\\
\\
(2) ② 若 $x_{1}, x_{2}, x_{3}$ 成等差数列,求证: $x_{2}^{2}>e^{3}$
\\
这里我们将通过全局恒等变形一举击穿此题的核心难点,不作任何局部逼近。
已知方程有三个正实数根 $x_1, x_2, x_3$ 且成等差数列。不妨设 $x_1 < x_2 < x_3$。
依据等差数列性质,大小居中的 $x_2$ 必为等差中项,必然有:
$$x_1 + x_3 = 2x_2$$
设该等差数列的公差为 $d$ ($d>0$),则有:$x_1 = x_2 - d$,且 $x_3 = x_2 + d$。
因为定义域要求根必须为正($x_1 > 0$),因此有极关键的隐含限制:$0 < d < x_2$。
\\
将三个根分别代入原方程 $f(x)=0$,我们得到三个绝对成立的等式:
(式1) $\ln(x_2 - d) + a(x_2 - d)^2 + b(x_2 - d) = 0$
(式2) $\ln x_2 + ax_2^2 + bx_2 = 0$
(式3) $\ln(x_2 + d) + a(x_2 + d)^2 + b(x_2 + d) = 0$
\\
执行极其美妙的线性组合消元运算:(式1) + (式3) - 2 $\times$ (式2) = 0
即 $f(x_1) + f(x_3) - 2f(x_2) = 0$。
代入展开并合并同类项得:
$$\left[ \ln(x_1) + \ln(x_3) - 2\ln x_2 \right] + a\left[ x_1^2 + x_3^2 - 2x_2^2 \right] + b\left[ x_1 + x_3 - 2x_2 \right] = 0$$
\\
我们逐项化简这个式子:
一次项:$b[2x_2 - 2x_2] = 0$。非常优美,难以处理的干扰参数 $b$ 被彻底消去了!
二次项:$a[ (x_2 - d)^2 + (x_2 + d)^2 - 2x_2^2 ] = a(2x_2^2 + 2d^2 - 2x_2^2) = 2ad^2$。
对数项:$\ln(x_2 - d) + \ln(x_2 + d) - 2\ln x_2 = \ln(x_2^2 - d^2) - \ln(x_2^2) = \ln\left( \frac{x_2^2 - d^2}{x_2^2} \right) = \ln\left(1 - \frac{d^2}{x_2^2}\right)$。
\\
代回原等式,我们得到了一个无懈可击的全局精确恒等式:
$$\ln\left(1 - \frac{d^2}{x_2^2}\right) + 2ad^2 = 0$$
为了剥离变量完成降维,令参量 $t = \frac{d^2}{x_2^2}$。
由于前面推导的 $0 < d < x_2$,我们严格得到 $0 < \frac{d^2}{x_2^2} < 1$,即 $t \in (0, 1)$。
上述恒等式可重写为:
$$\ln(1 - t) + 2a(x_2^2 \cdot t) = 0$$
等式两边同除以 $t$(显然 $t>0$),解出目标变量的解析表达:
$$2ax_2^2 = \frac{-\ln(1 - t)}{t} \quad \dots \text{(*)式}$$
\\
接下来,引入一个全局严格对数不等式:对于任意 $t \in (0, 1)$,恒有 $-\ln(1 - t) > t$。
\\
【引理严密证明】:
构造函数 $\varphi(t) = -\ln(1 - t) - t \quad (t \in [0, 1))$。
求导:$\varphi'(t) = -\frac{1}{1-t} \cdot (-1) - 1 = \frac{1}{1 - t} - 1 = \frac{t}{1 - t}$。
当 $t \in (0, 1)$ 时,由于分母 $1-t > 0$ 且分子 $t > 0$,故恒有 $\varphi'(t) > 0$。
从而 $\varphi(t)$ 在 $[0, 1)$ 上严格单调递增,必定有 $\varphi(t) > \varphi(0) = -\ln 1 - 0 = 0$。
因此 $-\ln(1 - t) - t > 0 \implies -\ln(1 - t) > t$ 全局恒成立。证明完毕。
\\
由于 $-\ln(1 - t) > t$,不等式两边同除以 $t$,得到 $\frac{-\ln(1 - t)}{t} > 1$。
将该引理结论直接应用于我们的 $(*)式$,实施严格放缩:
$$2ax_2^2 = \frac{-\ln(1 - t)}{t} > 1$$
由于在第 ① 问中已经论证过 $a>0$,两边同除以 $2a$,得到了极其漂亮的不等式链中坚节点:
$$x_2^2 > \frac{1}{2a}$$
\\
最后,由本题第 ① 问已经严格论证的“存在三个零点必须满足的前提”,参数 $a$ 的取值被牢牢限制在:
$$0 < a < \frac{1}{2e^3}$$
因为 $a > 0$,对其同乘 2 取倒数,不等号方向反转:
$$2a < \frac{1}{e^3} \implies \frac{1}{2a} > e^3$$
由极度严谨的不等式传递性,我们顺理成章地将两端连接在一起:
$$x_2^2 > \frac{1}{2a} > e^3$$
即 $x_2^2 > e^3$。解答闭环彻底完成,完全排除了所有的逻辑陷阱,原命题得证!
\end{solution}



