\chapter{导数题}
\section{偏移题}
\begin{example}{}{}
    (黎曼杯T18加强) $f(x) = \e^x-x\ln x - kx - 1$.若函数 $f(x)$ 有两个零点,将其分别记为 $a,b$.\\(1)试求 $k$ 的取值范围;\\(2)证明:\[k>a+\ln b+\frac{\sqrt2}{2}\](3)证明:\[a+b>1-\frac{1}{\e^{k+1}}+\ln(k+1)\]
\end{example}
\begin{solution}
\textbf{(1)} 令 $g(x)=\dfrac{\e^x}{x}-\ln x-\dfrac{1}{x}$,则原方程等价于 $g(x)=k$.求导得
\[
g'(x)=\frac{(x-1)(\e^x-1)}{x^2}.
\]
由于 $x>0$ 时 $\e^x-1>0$,故 $g'(x)$ 的符号由 $x-1$ 决定:当 $0<x<1$ 时 $g'(x)<0$,当 $x>1$ 时 $g'(x)>0$.因此 $g(x)$ 在 $(0,1)$ 上单调递减,在 $(1,+\infty)$ 上单调递增,在 $x=1$ 处取得最小值 $g(1)=\e-1$.又 $\lim\limits_{x\to0^+}g(x)=+\infty$,$\lim\limits_{x\to+\infty}g(x)=+\infty$,故对任意 $k>\e-1$,方程 $g(x)=k$ 恰有两个不同的实根;而当 $k\le \e-1$ 时,方程至多有一个实根.因此 $k$ 的取值范围是 $(\e-1,+\infty)$.

\textbf{(2)} 设 $f(x)$ 的两个零点为 $\alpha,\beta$,且 $0<\alpha<1<\beta$.由 $f(\alpha)=0$ 得
\[
k=\frac{\e^\alpha}{\alpha}-\ln\alpha-\frac1\alpha.
\]
要证 $k>\alpha+\ln\beta+\dfrac{\sqrt2}{2}$,即证
\[
\frac{\e^\alpha}{\alpha}-\ln\alpha-\frac1\alpha>\alpha+\ln\beta+\frac{\sqrt2}{2},
\]
整理得
\[
\ln(\alpha\beta)<\frac{\e^\alpha}{\alpha}-\alpha-\frac1\alpha-\frac{\sqrt2}{2}.
\]
构造函数 $G(x)=\dfrac{\e^x}{x}-x-\dfrac1x-\dfrac{\sqrt2}{2}\quad(x>0)$,求导得
\[
G'(x)=\frac{(x-1)(\e^x-x-1)}{x^2}.
\]
由 $\e^x\ge x+1$(当且仅当 $x=0$ 取等),知当 $x>0$ 时 $\e^x-x-1>0$,故 $G'(x)$ 的符号由 $x-1$ 决定:$0<x<1$ 时 $G'(x)<0$,$x>1$ 时 $G'(x)>0$.因此 $G(x)$ 在 $(0,1)$ 上单调递减,在 $(1,+\infty)$ 上单调递增,最小值 $G(1)=\e-2-\dfrac{\sqrt2}{2}>0$,从而 $G(x)>0$ 恒成立.于是欲证原不等式,只需证 $\ln(\alpha\beta)<0$,即 $\alpha\beta<1$.

下证 $\alpha\beta<1$.考虑函数 $F(x)=I(x)-I\!\left(\dfrac1x\right)$,其中 $I(x)=\e^x-x\ln x-kx-1$.对 $x\ge1$ 求导整理得
\[
F'(x)=\frac{(x-1)\bigl(\e^x-x\e^{1/x}+x-1\bigr)}{x^2}.
\]
令 $m(x)=\e^x-x\e^{1/x}+x-1$,则
\[
m'(x)=\e^x-\e^{1/x}+\frac{\e^{1/x}}{x^2}+1>0\quad(x>1),
\]
且 $m(1)=0$,故 $m(x)>0$ 对 $x>1$ 恒成立.因此 $F'(x)>0$($x>1$),$F(x)$ 在 $[1,+\infty)$ 上单调递增,且 $F(1)=0$,从而 $F(x)\ge0$,即 $I(x)\ge I(1/x)$ 对 $x\ge1$ 成立.取 $x=\beta>1$,得 $I(\beta)\ge I(1/\beta)$.由 $I(\beta)=0$ 知 $I(1/\beta)\le0$.

另一方面,由第(1)问定义的 $g(x)=\dfrac{\e^x}{x}-\ln x-\dfrac1x$ 在 $(0,1)$ 上单调递减,且 $g(\alpha)=k$,易知 $I(x)=x\bigl(g(x)-k\bigr)$.因为 $g(x)-k$ 在 $(0,1)$ 上严格递减且恰有一个零点 $x=\alpha$,所以当 $x\in(0,\alpha)$ 时 $g(x)-k>0$,$I(x)>0$;当 $x\in(\alpha,1)$ 时 $g(x)-k<0$,$I(x)<0$.于是由 $I(1/\beta)\le0$ 及 $1/\beta\in(0,1)$ 可得 $1/\beta\ge\alpha$,即 $\alpha\beta\le1$.若 $\alpha\beta=1$,则 $1/\beta=\alpha$,代入得 $I(\alpha)=I(\beta)=0$ 且 $\beta=1/\alpha$,但此时 $g(\alpha)=g(1/\alpha)$.容易验证 $g(x)\neq g(1/x)$ 对 $x\in(0,1)\cup(1,+\infty)$ 成立(例如由 $F(x)$ 的性质可推知),矛盾.因此 $\alpha\beta<1$.

综上,$\ln(\alpha\beta)<0$,从而原不等式得证.
\end{solution}
\newpage
\begin{example}{来自群友“港”}{}
    曲线$\varGamma :x^2+\ln^2y=\left(\dfrac{\e}{\e+1}\right)^2$,$AB$为$\varGamma$的一条弦.\\
    (1)求$\varGamma$的最高点$P$,最低点$Q$的坐标.\\
    (2)若$AB$的斜率为0,求$S_{\triangle{PAB}}$的最大值.\\
    (3)求$S_{\triangle{PAB}}$的最大值.
\end{example}
\begin{solution}
记 $R = \frac{e}{e+1}$,则方程化为 $x^{2} + (\ln y)^{2} = R^{2}$。由此可得参数方程:
$$\begin{cases}
x = R \cos \theta, \\
y = e^{R \sin \theta},
\end{cases}
\quad \theta \in [0, 2\pi).$$
(1) 当$\sin\theta=1$时$y$ 取最大值$e^R$,此时 $x=0$;当$\sin\theta=-1$ 时 $y$ 取最小值$e^{-R}$,此时$x=0$。故最高点为$P(0,e^R)$,最低点为$Q(0,e^{-R})$,即
$P\left(0,e^{{\frac{e}{e+1}}}\right),\quad Q\left(0,e^{{-\frac{e}{e+1}}}\right).$\\
(2) 由于直线 $l$ 斜率为 $0$,即水平线,由曲线关于 $y$ 轴对称,可设 $A$ 在左半平面,$B$ 在右半平面,且 $A$ 与 $B$ 关于 $y$ 轴对称。设 $B$ 对应的参数为 $\theta$($\cos\theta > 0$),则 $B(R\cos\theta, \mathrm{e}^{R\sin\theta})$,$A(-R\cos\theta, \mathrm{e}^{R\sin\theta})$。于是 $\triangle PAB$ 的底边长 $AB = 2R\cos\theta$,高为 $\mathrm{e}^{R} - \mathrm{e}^{R\sin\theta}$,面积
\[
S(\theta) = R\cos\theta\,(\mathrm{e}^{R} - \mathrm{e}^{R\sin\theta}),\quad \theta\in(-\tfrac{\pi}{2},\tfrac{\pi}{2}).
\]
令$t=R\sin\theta$,则
\[
S(t) = \sqrt{R^2-t^{2}}\,(\mathrm{e}^{R} - \mathrm{e}^{t}).
\]
为求最大值,考虑函数 $\varphi(t) = \ln S(t)$:
\begin{align*}
\varphi(t) &= \frac{1}{2}\ln(R^2 - t^2) + \ln(\mathrm{e}^{R} - \mathrm{e}^{t}),~~\varphi'(t) = -\frac{t}{R^2 - t^2} - \frac{\mathrm{e}^{t}}{\mathrm{e}^{R} - \mathrm{e}^{t}}\\
\varphi'(t)=0&\Leftrightarrow t\e^R + \e^t(R^2 - t^2 - t) = 0\Leftrightarrow t\e^{\frac{\e}{\e+1}} + \e^t\left(\left(\frac{\e}{\e+1}\right)^2 - t^2 - t\right) = 0\\
&\Leftrightarrow (\e^t-\e^{-\frac1{\e+1}})\left(\left(\frac{\e}{\e+1}\right)^2 - t^2 - t\right)+\e^{-\frac1{\e+1}}\left(t+\frac1{\e+1}\right)\left(\frac{\e^2}{\e+1}-t\right)=0
\end{align*}
得到驻点$t=-\frac{1}{\e+1}$,再求二阶导:
\[
\varphi''(t) = -\frac{(R^2 - t^2) - t(-2t)}{(R^2 - t^2)^2} - \frac{\mathrm{e}^{t}(\mathrm{e}^{R} - \mathrm{e}^{t}) - \mathrm{e}^{t}(-\mathrm{e}^{t})}{(\mathrm{e}^{R} - \mathrm{e}^{t})^2} = -\frac{R^2 + t^2}{(R^2 - t^2)^2} - \frac{\mathrm{e}^{t}\mathrm{e}^{R}}{(\mathrm{e}^{R} - \mathrm{e}^{t})^2} < 0
\]
因此 $\varphi'(t)$ 在 $(-R, R)$ 上严格单调递减,故方程 $\varphi'(t) = 0$ 至多有一个实根$t=-\frac{1}{\e+1}$。由于 $S(t)$ 在端点 $t = \pm R$ 处为零,在 $(-R, R)$ 内为正,故该驻点即为最大值点。此时 $c = \mathrm{e}^{t} = \mathrm{e}^{-\frac{1}{\mathrm{e}+1}}$,代入得
\[
S_{\max} = \sqrt{R^2 - \left(-\frac{1}{\mathrm{e}+1}\right)^2}\left(\mathrm{e}^{R} - \mathrm{e}^{-\frac{1}{\mathrm{e}+1}}\right) = \sqrt{\frac{\mathrm{e}-1}{\mathrm{e}+1}} \cdot \mathrm{e}^{-\frac{1}{\mathrm{e}+1}}(\mathrm{e}-1) 
\]
(3) 对于一般直线 $l$,设 $A(R\cos\theta_1, \mathrm{e}^{R\sin\theta_1})$,$B(R\cos\theta_2, \mathrm{e}^{R\sin\theta_2})$,则 $\triangle PAB$ 面积为
\[
S(\theta_1,\theta_2) = \frac{R}{2}\left| \cos\theta_1 (\mathrm{e}^{R\sin\theta_2} - \mathrm{e}^{R}) - \cos\theta_2 (\mathrm{e}^{R\sin\theta_1} - \mathrm{e}^{R}) \right|.
\]
由面积表达式及三角不等式得
\[
S \le \frac{R}{2} \left( |\cos\theta_1| (\mathrm{e}^{R} - \mathrm{e}^{R\sin\theta_2}) + |\cos\theta_2| (\mathrm{e}^{R} - \mathrm{e}^{R\sin\theta_1}) \right) \triangleq H(\theta_1,\theta_2),
\]
等号成立当且仅当 \(\cos\theta_1\) 与 \(\cos\theta_2\) 异号。令 \(u=\sin\theta_1\), \(v=\sin\theta_2\),则
\[
H(u,v) = \frac{R}{2} \left( \sqrt{1-u^2}\,(\mathrm{e}^{R} - \mathrm{e}^{Rv}) + \sqrt{1-v^2}\,(\mathrm{e}^{R} - \mathrm{e}^{Ru}) \right).
\]
求偏导数并令其为零,得到方程组
\[
\frac{u}{\sqrt{1-u^2}}(\mathrm{e}^{R} - \mathrm{e}^{Rv}) = -R\sqrt{1-v^2}\,\mathrm{e}^{Ru}, \quad
\frac{v}{\sqrt{1-v^2}}(\mathrm{e}^{R} - \mathrm{e}^{Ru}) = -R\sqrt{1-u^2}\,\mathrm{e}^{Rv}.
\]
由对称性知 \(u,v<0\),令 \(a=-u>0\), \(b=-v>0\),则方程组化为
\[
\frac{a}{\sqrt{1-a^2}}(\mathrm{e}^{R} - \mathrm{e}^{-Rb}) = R\sqrt{1-b^2}\,\mathrm{e}^{-Ra}, \quad
\frac{b}{\sqrt{1-b^2}}(\mathrm{e}^{R} - \mathrm{e}^{-Ra}) = R\sqrt{1-a^2}\,\mathrm{e}^{-Rb}.
\]
两式相除并整理得
\[
\frac{a}{b} \cdot \frac{\mathrm{e}^{R} - \mathrm{e}^{-Rb}}{\mathrm{e}^{R} - \mathrm{e}^{-Ra}} = \mathrm{e}^{-R(a-b)}.
\]
取对数后移项得到
\[
\ln a + \ln(\mathrm{e}^{R} - \mathrm{e}^{-Ra}) + Ra = \ln b + \ln(\mathrm{e}^{R} - \mathrm{e}^{-Rb}) + Rb.
\]
定义函数 \(K(t)=\ln t + \ln(\mathrm{e}^{R} - \mathrm{e}^{-Rt}) + Rt=\ln t + \ln(\mathrm{e}^{\frac{\e}{\e+1}} - \mathrm{e}^{-\frac{\e}{\e+1}t}) + \frac{\e}{\e+1}t\),则由复合函数单调性得到 \(K(t)\) 严格递增,从而 \(a=b\),即 \(u=v\)。因此极值点必满足 \(u=v\),此时
\[
H(u,u)=R\sqrt{1-u^2}(\mathrm{e}^{R} - \mathrm{e}^{Ru}),
\]
与 (2) 中形式相同,由 (2) 知该函数的最大值在 $u = -\frac{1}{\mathrm{e}}$ 处取得,同时,等号条件要求 $\cos\theta_1$ 与 $\cos\theta_2$ 异号,故取 $\theta_2 = \pi - \theta_1$,此时直线为水平线 $y = \mathrm{e}^{Ru} = \mathrm{e}^{-\frac{1}{\mathrm{e}+1}}$,且 $S$ 达到该最大值。由于 $H(u,v)$ 在边界 $u,v=\pm1$ 上为零,而内点值大于零,故该驻点即为全局最大值点。综上,$\triangle PAB$ 面积的最大值为$\displaystyle  \sqrt{\frac{\mathrm{e}-1}{\mathrm{e}+1}} \cdot \mathrm{e}^{-\frac{1}{\mathrm{e}+1}}(\mathrm{e}-1) $
\end{solution}


\newpage
\begin{example}{}{}
    (1)若$\theta\in\left(0,\frac{\pi}2\right)$,判断$2\cos^2\theta$与$\left(\frac{1}{n}\cos 2\theta+1\right)^n$的大小关系,并证明.\\
    (2)证明:对于任意自然数$n$,$\theta\in\left(0,\frac{\pi}2\right)$,有
\begin{gather*}
\left(\sin^2\theta+\sin^3\theta+\cdots+\sin^{2n}\theta+\sin^{2n+1}\theta\right)+\left(\cos^2\theta+\cos^3\theta+\cdots+\cos^{2n}\theta+\cos^{2n+1}\theta\right)\\
\geq\left(\sqrt{2}+2\right)\left(1-\frac{1}{2^n}\right)
\end{gather*}
\end{example}
\begin{solution}
(1)取$\theta=0$探路,显然有$2\leqslant \left(1+\frac1n\right)^n$,于是考虑证明$2\cos^2\theta\leqslant\left(\frac{1}{n}\cos 2\theta+1\right)^n$,即$1+\cos2\theta\leqslant \left(\frac{1}{n}\cos 2\theta+1\right)^n$,取对数即$\ln(1+\cos2\theta)\leqslant n\ln\left(1+\frac{\cos2\theta}{n}\right)$,改造成:
\begin{align*}
    f(x)&=n\ln\left(1+\frac{x}{n}\right)-\ln(1+x)\\
    f'(x)&=\frac{1}{1+\frac{x}{n}}-\frac{1}{1+x}=\frac{(1+x)-\bigl(1+\frac{x}{n}\bigr)}{\bigl(1+\frac{x}{n}\bigr)(1+x)}=\frac{x\bigl(1-\frac1n\bigr)}{\bigl(1+\frac{x}{n}\bigr)(1+x)}.
\end{align*}
因此$f(x)$在$(-1,0]$上单调递减,在$[0,1]$上单调递增,从而$f(x)\geqslant f(0)=0$对任意$x\in(-1,1]$成立,等号仅当$x=0$时取得。于是$1+\cos2\theta\leqslant\bigl(1+\frac{\cos2\theta}{n}\bigr)^n$,即$2\cos^2\theta\leqslant\bigl(\frac{1}{n}\cos2\theta+1\bigr)^n$,等号当$\cos2\theta=0$,即$\theta=\frac\pi4$时成立。\\
(2)由二倍角公式,\(\sin^2\theta = \frac{1-\cos2\theta}{2}\),\(\cos^2\theta = \frac{1+\cos2\theta}{2}\),则对于任意正整数 \(m \geq 2\),
\[
\sin^m\theta + \cos^m\theta = 2^{-\frac{m}{2}} \left[ (1-\cos2\theta)^{\frac{m}{2}} + (1+\cos2\theta)^{\frac{m}{2}} \right].
\]
令 \(x = \cos2\theta \in (-1,1)\),则需证的和式为
\[
S(\theta) = \sum_{m=2}^{2n+1} 2^{-\frac{m}{2}} \left[ (1-x)^{\frac{m}{2}} + (1+x)^{\frac{m}{2}} \right].
\]
对于每个 \(k = \frac{m}{2} \in \{1, 1.5, 2, \dots, n+0.5\}\),考虑函数 \(\varphi_k(x) = (1-x)^k + (1+x)^k\)。由于 \(\varphi_k(x)\) 是偶函数,且当 \(x > 0\) 时,
\[
\varphi_k'(x) = -k(1-x)^{k-1} + k(1+x)^{k-1} = k\left[ (1+x)^{k-1} - (1-x)^{k-1} \right] > 0,
\]
所以 \(\varphi_k(x)\) 在 \([0,1)\) 上单调递增,从而在 \(x=0\) 处取最小值 \(\varphi_k(0)=2\)。因此
\[
(1-x)^{\frac{m}{2}} + (1+x)^{\frac{m}{2}} \geq 2\Rightarrow \sin^m\theta + \cos^m\theta \geq 2 \cdot 2^{-\frac{m}{2}} = 2^{1-\frac{m}{2}}.
\]
对 \(m\) 从 \(2\) 到 \(2n+1\) 求和得
\[
S(\theta) \geq \sum_{m=2}^{2n+1} 2^{1-\frac{m}{2}} = 2 \sum_{m=2}^{2n+1} 2^{-\frac{m}{2}} = 2 \cdot \frac{(\frac{1}{\sqrt{2}})^2 \left(1 - (\frac{1}{\sqrt{2}})^{2n}\right)}{1 - \frac{1}{\sqrt{2}}} = (2+\sqrt{2})\left(1 - \frac{1}{2^n}\right).
\]
等号成立当且仅当 \(x=0\),即 \(\cos2\theta = 0\),亦即 \(\theta = \frac{\pi}{4}\)。因此原不等式得证。
\end{solution}
\newpage
\begin{example}{}{}
    证明:曲线\[y=(x-\ln x)\e^{1-x}\]上不存再不同的两点关于直线$y=x$对称
\end{example}
\begin{solution}

\end{solution}

\newpage
\begin{example}{邪帝导数题}{}
    已知函数$f(x)=x-(a+1)\ln x-\frac{a}{x},(a>1)$\\
    (1)讨论$f(x)$的单调性;\\
    (2)若$f(x_1)=f(x_2)=f(x_3),(x_1<x_2<x_3)$,证明$f(x_1x_2x_3)<1-a$.
\end{example}
\begin{solution}
\end{solution}
\begin{example}{远古偏移题,来自陈语梦}{}
已知 $x_{1} - \ln(\ln x_{1} + 1) = x_{2} - \ln(\ln x_{2} + 1) = m$求证\[m + 1 < x_{1} + x_{2} < \frac{7}{6}m + \frac{5}{6}\]
\end{example}
\begin{solution}
    设函数$f(x)=x-\ln(\ln x+1)-1$,左侧化为\[g(x)=x^2-x\left(m+1\right)+1=x\ln(\ln x+1)-x+1,~~~x_1<1<x_2\Leftrightarrow g(x_1)>g(x_2)\]
    加强证明$g(x)$单调递增,求导得$g'(x)=\ln(\ln x+1)+x\cdot\frac1{\ln x+1}\cdot\frac1x-1=\ln(\ln x+1)+\frac1{\ln x+1}-1$
    又因为任意$x>0$有$\ln x>1-\frac1x$,所以$g'(x)>0$,故$g(x)$单调递增。右侧化为
    \[g(x)=x^2-x\left(\frac{7}{6}f(x) + \frac{5}{6}\right)+1=1-\frac{{x}^{2}}{6}-\frac{5}{6}x+\frac76x\ln(\ln({x})+1)\]
    可以考虑证明函数
    \begin{align*}
    h(x)&=g(x)+\frac16f(x)+\frac{7}{72}f^2(x)=x^2-x\left(\frac{7}{6}f(x) + \frac{5}{6}\right)+1+\frac16f(x)+\frac{7}{72}f^2(x)\\
    &=
    \end{align*}
    单调递减,上界:\[x_1+x_2+x_3<2+(\e-3)m-\frac{m^2}{2}-\frac{2}{3}m^3\]
\end{solution}
\newpage
\begin{example}{来自amare Donata Caesia}{}
    $x_1\ln x_1=x_2\ln x_2=-x_3\ln x_3,x_1<x_2<x_3$,证明$x_1+x_2+x_3>2$
\end{example}
\begin{solution}
考虑加强证明\[\begin{cases}x_1+x_2>1+m+\frac{m^2}2\\x_3>1-m-\frac{m^2}2\end{cases}\]
第一个不等式,考虑加强证明$x_1+x_2>\e(3-\e)m^2+m+1$,化成$x-\ln x$的经典偏移模型,即证\[x_1-\ln x_1=x_2-\ln x_2=a>1,\frac{1}{x_1}+\frac{1}{x_2}>\e^a-1+\dfrac{\e(3-\e)}{\e^a}\]
假设$\frac{1}{x_1}>\frac{1}{x_2},x_1<1<x_2$,即证\[g(x)=\frac{1}{x^2}-\left(\e^a-1+\dfrac{\e(3-\e)}{\e^a}\right)\frac{1}{x}=\frac{1}{x^2}-\left(\frac{\e^x}{x}-1+\e(3-\e)\dfrac{x}{\e^x}\right)\frac{1}{x},g(x_1)>g(x_2)\]则$x=1$必然是极值点,只需要改造$g(x)$使得其单调递减即可完成加强证明,显然地,考虑到$x_1-\ln x_1=x_2-\ln x_2=a$,取指数就有$\frac{\e^{x_1}}{x_1}=\frac{\e^{x_2}}{x_2}=\e^a$,只需证明
\begin{align*}
h(x)&=\frac{1}{x^2}-\frac{\frac{\mathrm{e}^x}{x}-1+\mathrm{e}(3-\mathrm{e})\frac{x}{\mathrm{e}^x}}{x}+1+\mathrm{e}(5-2\mathrm{e})\frac{x}{\mathrm{e}^x}\\
h'(x)&=\frac{\mathrm{e}^x(2-x)-2}{x^3}-\frac{1}{x^2}+\mathrm{e}\mathrm{e}^{-x}\begin{bmatrix}8-3\mathrm{e}-(5-2\mathrm{e})x\end{bmatrix}
\end{align*}
$h(x)$单调递减即可,设$H(x)=x^3\e^xh'(x)=\mathrm{e}^{2x}(2-x)-(x+2)\mathrm{e}^x+\mathrm{e}x^3\left[8-3\mathrm{e}-(5-2\mathrm{e})x\right]$求4次导数把$x$的幂函数导掉:
\begin{align*}
    H'(x)&=e^{2x}(3-2x)-e^x(x+3)+3e(8-3e)x^2-4e(5-2e)x^3\\
H''(x)&=4\e^{2x}(1-x)-\e^x(x+4)+6\e(8-3\e)x-12\e(5-2\e)x^2\\
H'''(x)&=4\e^{2x}(1-2x)-\e^x(x+5)+6\e(8-3e)-24\e(5-2\e)x\\
H''''(x)&=-16x\e^{2x}-(x+6)\e^x-24\e(5-2\e)
\end{align*}
显然对于$x>0$,$H''''(x)$单调递减,且$H''''(0)=-6+24\e(2\e-5)>0,\displaystyle \lim_{x\to+\infty}H''''(x)=-\infty$,故其存在零点$(\xi_1,0)$,则$H'''(x)$在$(0,\xi_1)$单增,在$(\xi_1,+\infty)$单减,由于
\[H'''(0)=-1+6\e(8-3\e)<0,H'''(1)=26\e^2-78\e<0,H'''\left(\frac12\right)=6\e^2-12\e-\frac{11}{2}\sqrt{\e}>0\]
故$H'''(x)$存在零点$(\xi_2,0),(\xi_3,0)$,那么$H''(x)$在$(0,\xi_2)$单减,在$(\xi_2,\xi_3)$单增,在$(\xi_3,+\infty)$单减,由于
\[H''(0)=0,H''\left(\frac12\right)=11\e-\frac{9}{2}\sqrt{\e}-3\e^2>0,H''(1)=6\e^2-17\e<0\]
所以$H''(x)$存在零点$(\xi_4,0),(\xi_5,0)$,所以$H'(x)$在$(0,\xi_4)$单减,在$(\xi_4,\xi_5)$单增,在$(\xi_5,\infty)$单减,由于$H'(0)=H'(1)=0$,所以$H'(x)$存在零点$(\xi_6,0)$,那么$H(x)$在$(0,\xi_6)$单减,在$(\xi_6,1)$单增,在$(1,+\infty)$单减,又因为$H(0)=H(1)=0$,所以$H(x)\leqslant 0$,故$h(x)$在$(0,+\infty)$单减。于是$x_1+x_2>\e(3-\e)m^2+m+1$得证.

下证$x_3>1-m-\frac{m^2}2>1$,对于$m\in(-\frac1{\e},0)$,不难验证后一个“$>$”成立,两边套函数名$-m=x_3\ln x_3>(1-m-\frac{m^2}2)\ln (1-m-\frac{m^2}2)$即证
\[p(m)=\frac{m}{\frac{m^2}2+m-1}-\ln (1-m-\frac{m^2}2)>0,\forall m\in(-\frac1{\e},0)\]
令$t=-m\in(0,\frac{1}{e})$,则不等式化为
$$f(t)=\frac{t}{1+t-\frac{t^{2}}{2}}-\ln \left(1+t-\frac{t^{2}}{2}\right)>0,~~f'(t)=\frac{t^{2}\left(2-\frac{t}{2}\right)}{\left(1+t-\frac{t^{2}}{2}\right)^{2}}>0, \quad \forall t\in\left(0,\frac{1}{e}\right)$$
且$f(0)=0$,故$f(t)>0$在区间内成立,即$x_3>1-m-\frac{m^2}2>1$。这就证明了$x_1+x_2+x_3>2$.
\end{solution}

