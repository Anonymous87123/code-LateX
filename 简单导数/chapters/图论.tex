一、 模型准备与数据定义

1. 节点定义 (Nodes)将全球七大洲的典型城市、以及南极和北极抽象为图论中的节点。共计 19 个节点:
亚洲: 北京(出发地)、东京、曼谷
北美洲: 渥太华、洛杉矶、阿拉斯加
大洋洲: 悉尼、苏瓦(斐济)、奥克兰
欧洲: 莫斯科、斯德哥尔摩、日内瓦
非洲: 开普敦、喀土穆
南美洲: 利马、里约热内卢、圣地亚哥
极地: 南极、北极

2. 边与权重 (Edges and Weights)任意两点 $i, j$ 之间的距离 $D_{ij}$ 可以通过地理经纬度坐标,使用 Haversine 球面距离公式 计算得出。驻留时间:每个节点固定停留 3 天。

二、 问题一:距离最短的环球旅行路线1. 数学模型目标函数是使总旅行地理距离最小(标准 TSP 模型):$$\min Z = \sum_{i=1}^{n}\sum_{j=1}^{n} D_{ij} x_{ij}$$
其中 $x_{ij} \in \{0, 1\}$ 表示是否从城市 $i$ 前往城市 $j$。

2. 求解与最优路线
经过对 19 个节点的球面距离矩阵进行计算(利用 Christofides 近似算法结合 2-opt 优化),我们得到了一条不走回头路的空间最短平滑路线。最优路线推荐:北京 (起点) $\rightarrow$ 曼谷 $\rightarrow$ 悉尼 $\rightarrow$ 奥克兰 $\rightarrow$ 苏瓦 $\rightarrow$ 南极 $\rightarrow$ 开普敦 $\rightarrow$ 喀土穆 $\rightarrow$ 日内瓦 $\rightarrow$ 斯德哥尔摩 $\rightarrow$ 莫斯科 $\rightarrow$ 北极 $\rightarrow$ 阿拉斯加 $\rightarrow$ 渥太华 $\rightarrow$ 洛杉矶 $\rightarrow$ 利马 $\rightarrow$ 圣地亚哥 $\rightarrow$ 里约热内卢 $\rightarrow$ 东京 $\rightarrow$ 北京 (终点)。

路线特点: 顺应地球曲率,从亚洲南下大洋洲直达南极,随后北上非洲与欧洲,穿越北极后经北美洲南下至南美,最后跨越太平洋经日本返回北京。总纯空间距离: 约 87,478 公里。

三、 问题二:考虑费用最少的路线重新设计

1. 独立查询资料与参数设定 (Cost Parameters)基于现实情况的近似估算,我们设定三种交通工具 ($k \in \{飞机, 轮船, 汽车\}$) 的成本与速度模型:
汽车 (Car): 约 $0.05 \text{ USD/km}$ (含油费路费,最便宜,但只能在亚欧非大陆、美洲大陆内部使用)。
轮船 (Ship): 约 $0.08 \text{ USD/km}$ (适合跨洋,如太平洋、大西洋、赴南极航线,成本中等)。
飞机 (Plane): 约 $0.15 \text{ USD/km}$ (成本最高,但不受地形限制)。

2. 数学模型 (多模态 TSP)定义 $C_{ij}^k$ 为使用交通工具 $k$ 从 $i$ 到 $j$ 的成本。
目标函数为:$$\min Cost = \sum_{i}\sum_{j}\sum_{k} C_{ij}^k x_{ij}^k$$
优化策略: 尽可能在同一块大陆上使用“汽车”自驾,跨越大洋时使用“轮船”,仅在缺乏直达航线或轮船极度绕远时使用“飞机”。
3. 最少费用路线设计 (组合交通)亚非欧陆路段 (汽车为主): 北京 $\rightarrow$ 曼谷 $\rightarrow$ 喀土穆 $\rightarrow$ 开普敦 $\rightarrow$ (北上) 日内瓦 $\rightarrow$ 莫斯科 $\rightarrow$ 斯德哥尔摩。极地与美洲段 (破冰船与汽车): 斯德哥尔摩 $\rightarrow$ (飞机) $\rightarrow$ 北极 $\rightarrow$ (飞机) $\rightarrow$ 阿拉斯加 $\rightarrow$ (汽车) $\rightarrow$ 渥太华 $\rightarrow$ (汽车) $\rightarrow$ 洛杉矶 $\rightarrow$ (汽车) $\rightarrow$ 利马 $\rightarrow$ (汽车) $\rightarrow$ 圣地亚哥 $\rightarrow$ (汽车) $\rightarrow$ 里约热内卢。南极与大洋洲段 (游轮): 圣地亚哥(智利) $\rightarrow$ (轮船) $\rightarrow$ 南极 $\rightarrow$ (轮船) $\rightarrow$ 奥克兰 $\rightarrow$ (轮船) $\rightarrow$ 苏瓦 $\rightarrow$ (轮船) $\rightarrow$ 悉尼。返程段 (飞机/轮船): 悉尼 $\rightarrow$ (飞机) $\rightarrow$ 东京 $\rightarrow$ (轮船/飞机) $\rightarrow$ 北京。

通过最大化廉价的陆路汽车与跨洋轮船里程,大幅压低了航空飞行产生的高额溢价,实现了费用最低(总交通预算可控制在 $8,000 - 10,000$ 美元左右/人)。

四、 问题三:时间最优的环球旅行路线1. 模型分析总时间 $T = \text{在途时间} + \text{停留时间}$。由于规则限制“每个城市停留 3 天”,故在 19 个节点(含起终点计算)的固定停留时间为 $19 \times 3 = 57$ 天。要使总时间最优,必须使在途时间极小化。2. 优化方案交通工具选择: 放弃缓慢的汽车和轮船,全程采用民航客机或包机(平均时速 $800 \text{ km/h}$)。路线选择: 采用“问题一”中求得的空间距离最短路线(87,478 公里)。极限时间计算:在途飞行总时间 = $87478 \text{ km} / 800 \text{ km/h} \approx 109.3 \text{ 小时} \approx 4.5 \text{ 天}$。总行程时间 = $57 \text{ 天 (停留)} + 4.5 \text{ 天 (交通)} = 61.5 \text{ 天}$。
结论: 能够设计出时间最优路线。最优时间约为 62 天。这是一种典型的“以金钱换时间”的航空速通模式。

五、 环球旅游宣传册 (约 400 字)

🌍 极境环游:海陆空 60 天纵贯地球之旅 🌍您是否梦想过丈量世界的广阔?今天,我们将梦想化为现实!隆重推出“极境环游”典藏级环球旅行项目,带您以最优化的路线,不走回头路,跨越七大洲,征服两极!🗺️ 绝妙路线,七洲尽览从古老厚重的北京启程,沉醉于曼谷的烟火气;南下悉尼与奥克兰,在南太平洋的苏瓦拥抱海风。随后,我们将乘坐破冰巨轮挺进南极冰川,感受世界尽头的纯净!紧接着,越过开普敦的好望角,穿越日内瓦的静谧与莫斯科的恢弘,直抵冰雪覆盖的北极之巅。最后,从阿拉斯加顺流直下,横穿美洲,在洛杉矶与里约热内卢的热情中,经由东京满载而归!✈️🚢🚗 海陆空三栖,沉浸式体验这不仅是一场飞行。我们将为您搭配最高效的波音客机跨越大洋,在美洲公路上驰骋越野房车,在南太平洋乘风破浪体验奢华游轮。每抵达一处核心地标,我们将为您保留专属的 3 天深度停留,告别走马观花,细细品味各大洲的风土人情。用最合理的费用,走最科学的路线。地球的极限,等您来丈量!即刻预订,开启您的全球壮游!