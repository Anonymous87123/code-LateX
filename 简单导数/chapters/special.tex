\chapter{x_1+x_2的等价构造法}
\section{从零开始——真实的二次多项式}
既然我们有两个根 $x_1$ 和 $x_2$,最自然的代数直觉是构造一个以它们为根的精确二次方程:
\[
Q_{true}(x) = (x - x_1)(x - x_2) = x^2 - (x_1 + x_2)x + x_1 x_2 = 0
\]
为了方便,我们记两根之和为 $S = x_1 + x_2$,两根之积为 $P = x_1 x_2$。
所以,$Q_{true}(x) = x^2 - S x + P$。对于真正的根 $x_i \in \{x_1, x_2\}$,必然有 $Q_{true}(x_i) \equiv 0$。

但是,$S$ 和 $P$ 我们都不知道。我们必须利用原方程把它们联系起来。
将两个根分别代入原方程并相加:
\[
(x_1 - \ln x_1) + (x_2 - \ln x_2) = 2m
\]
\[
S - \ln(P) = 2m \implies P = e^{S - 2m}
\]
太棒了!我们得到了 $P$ 和 $S$ 之间的显式关系。把它代回多项式:
\[
Q_{true}(x) = x^2 - S x + e^{S - 2m}
\]
这就是关于这两个根的绝对精确的代数母函数。

\section{偷天换日——目标的“摄动”与常数湮灭}
现在,题目要求我们证明 $S \sim A(m)$($\sim$ 代表 $>$ 或 $<$)。
既然我们不知道真实的 $S$,我们不妨做一个极其大胆的动作:把精确多项式里的 $S$ 强行替换成我们要证明的目标边界 $A(m)$。

我们创造一个新的测试函数 $F(x)$:
\[
F(x) = x^2 - A(m)x + e^{A(m) - 2m}
\]
这个函数有什么用?让我们把真实的根 $x_i$ 扔进去试探一下。我们知道精确多项式 $Q_{true}(x_i) = 0$,所以我们可以用 $F(x_i)$ 减去 $0$:
\[
F(x_i) = F(x_i) - Q_{true}(x_i)
\]
\[
F(x_i) = \left(x^2_i - A(m)x_i + e^{A(m) - 2m}\right) - \left(x^2_i - S x_i + e^{S - 2m}\right)
\]
\[
F(x_i) = (S - A(m))x_i - \left( e^{S - 2m} - e^{A(m) - 2m} \right)
\]
我们发现了什么?尾部那个极其恶心复杂的指数差 $\left( e^{S - 2m} - e^{A(m) - 2m} \right)$,它竟然是一个与 $x_i$ 完全无关的常数!

如果我们把两个根的函数值作差,即计算 $F(x_1) - F(x_2)$,这个烦人的常数尾巴会被瞬间湮灭!
\[
F(x_1) - F(x_2) = \left[ (S - A(m))x_1 - C \right] - \left[ (S - A(m))x_2 - C \right]
\]
\[
F(x_1) - F(x_2) = (x_1 - x_2)(S - A(m))
\]
【第一次逻辑闭环】:因为 $x_1 < x_2$,所以 $x_1 - x_2 < 0$ 恒成立。
这意味着,$F(x_1) - F(x_2)$ 的符号,绝对等价于 $S - A(m)$ 的相反符号!

如果我们能证明 $F(x_1) < F(x_2)$(即作差小于0),就等价于证明了 $S > A(m)$。我们成功地把双变量 $x_1, x_2$ 的极值点偏移问题,降维成了单变量函数 $F(x)$ 的单调性问题。

\section{探秘“异号”——繁琐求导下的泰勒神迹}
到目前为止,逻辑很完美。但问题来了:为了保证 $F(x_1) < F(x_2)$,最舒服的情况就是 $F(x)$ 严格单调,且 $F(x_1)$ 和 $F(x_2)$ 一正一负(异号)。因为极值点 $x=1$ 横亘在 $x_1$ 和 $x_2$ 中间,如果 $F(x)$ 在 $x=1$ 处恰好穿过 $x$ 轴变号,那一切就迎刃而解了。

可是,凭什么我们用 $e^{A(m)-2m}$ 凑出来的这个尾巴,就一定能在 $x=1$ 处乖乖变号呢?让我们硬着头皮去求导,看看 $x=1$ 附近到底发生了什么魔法。

注意:$m = x - \ln x$,所以 $m$ 是 $x$ 的函数。
我们有:$m'(x) = 1 - \frac{1}{x} \implies m'(1) = 0$;$m''(x) = \frac{1}{x^2} \implies m''(1) = 1$。
在极值点 $x_1 \to 1, x_2 \to 1$ 时,$S \to 2$,所以一个合理的目标边界必须满足 $A(1) = 2$。

1. 探根(零阶导):
\[
F(1) = 1^2 - A(1)\cdot 1 + e^{A(1)-2m(1)} = 1 - 2 + e^{2-2} = 1 - 2 + 1 = 0
\]
果然,它必定过原点!

2. 探斜率(一阶导):
这里必须严格使用链式法则。
\[
F'(x) = 2x - \left[ A'(m)m'(x)x + A(m)\cdot 1 \right] + e^{A(m)-2m} \cdot (A'(m) - 2)m'(x)
\]
为了看清结构,我们把含有 $m'(x)$ 的项提出来:
\[
F'(x) = 2x - A(m) + m'(x) \cdot \underbrace{\left[ -A'(m)x + e^{A(m)-2m}(A'(m)-2) \right]}_{\text{记为 } K(x)}
\]
代入 $x=1$,因为 $m'(1)=0$,后面那一坨 $K(x)$ 无论多复杂都直接归零:
\[
F'(1) = 2(1) - A(1) + 0 \cdot K(1) = 2 - 2 = 0
\]
斜率为0!这意味着 $x=1$ 是一个驻点。

3. 探曲率(二阶导数神迹):
我们要对 $F'(x) = 2x - A(m) + m'(x)K(x)$ 继续求导。
\[
F''(x) = 2 - A'(m)m'(x) + m''(x)K(x) + m'(x)K'(x)
\]
代入 $x=1$。同样因为 $m'(1)=0$,第二项和第四项直接灰飞烟灭!
\[
F''(1) = 2 - 0 + m''(1)K(1) + 0 = 2 + K(1)
\]
那么 $K(1)$ 到底是多少?把 $x=1, m(1)=1, A(1)=2$ 代入 $K(x)$:
\[
K(1) = -A'(1)\cdot 1 + e^{2-2}\big(A'(1) - 2\big) = -A'(1) + A'(1) - 2 = -2
\]
惊人的巧合出现了:
\[
F''(1) = 2 + (-2) = 0 \quad !!!
\]
【第二次逻辑闭环】:我们终于找到了“异号”的终极本质!那个看起来极其丑陋的指数尾巴 $e^{A(m)-2m}$,它在微积分层面的唯一使命,就是在二次求导时精准吐出一个 $-2$,将前面的 $2$ 完美暗杀!

这导致 $F(1) = F'(1) = F''(1) = 0$。根据泰勒展开,$F(x)$ 在 $x=1$ 处的展开式直接从三次项(奇次项)开始:$F(x) \approx c(x-1)^3$。
奇数次幂的图像必然会像一根针一样穿透 $x$ 轴强制变号。这就是为什么 $F(x_1)$ 和 $F(x_2)$ 必然是一正一负(异号)的绝对铁证。这个所谓的“无脑构造法”,其实是一个精妙绝伦的泰勒锁根陷阱。

\section{撞上南墙——$1/x_1 + 1/x_2$ 的崩溃与边界勘定}
既然掌握了这套降维打击的方法,我雄心勃勃地想去解决倒数和偏移:证明 $\frac{1}{x_1} + \frac{1}{x_2} > C(a)$(比如你测试的那个带有 $e^a$ 的复杂边界)。

按照之前的逻辑,我需要构造倒数根 $\frac{1}{x_1}, \frac{1}{x_2}$ 的精确母函数。
设 $S_{-1} = \frac{1}{x_1} + \frac{1}{x_2}$,积 $P_{-1} = \frac{1}{x_1 x_2}$。
根式方程必然是:$t^2 - S_{-1}t + P_{-1} = 0$。
我需要找到 $P_{-1}$ 关于 $S_{-1}$ 和 $m$ 的显式表达,才能构造出那个能湮灭常数的尾巴 $H(m)$。

我们回到原方程的约束:$x_1+x_2 - \ln(x_1 x_2) = 2m$。
代入倒数关系:$x_1+x_2 = \frac{S_{-1}}{P_{-1}}$,且 $x_1 x_2 = \frac{1}{P_{-1}}$。
\[
\frac{S_{-1}}{P_{-1}} - \ln\left(\frac{1}{P_{-1}}\right) = 2m \implies \frac{S_{-1}}{P_{-1}} + \ln(P_{-1}) = 2m
\]
\[
\mathbf{S_{-1} = P_{-1}(2m - \ln(P_{-1}))}
\]
灾难降临了。

我需要反解这个方程,把 $P_{-1}$ 单独表述在等号一边。但我做不到。没有任何初等函数能解出这个方程,它必须求助于超越函数——朗伯 W 函数 (Lambert W function)!

如果我写不出精确的 $H(m)$,我就无法构造出那个作差后完美抵消常数的 $F(x)$。
如果我强行用泰勒展开在 $x=1$ 附近用线性函数 $h(m) = 1 + (C'(1)-2)(m-1)$ 去“骗”一个补丁(这正是你之前测试的方法),只要偏移量稍微变大(比如遇到 $e^a$ 这种指数级膨胀),局部的泰勒近似就会瞬间被撕裂。$F'(x)$ 的符号将变得混乱不堪,根本无法在草稿纸上完成因式分解和单调性证明。

【第三次逻辑闭环:边界划定】
我终于自己摸索出了这套方法的“绝对死线”:异号函数构造法(韦达同构代换法)当且仅当该体系的两根乘积 $P_f = f(x_1)f(x_2)$,能够利用原方程,被显式、初等地表达为两根之和 $S_f = f(x_1)+f(x_2)$ 与参数 $m$ 的函数时才全局有效。
一旦反解需要超越函数(如 $1/x_1+1/x_2$ 或 $x_1^2+x_2^2$),此法立刻破产,必须老老实实切回对称构造法!

\section{绝地反击——发现隐藏的对数偏移版图}
我不甘心这套方法就此局限于 $x_1+x_2$。既然我发现了边界(能否显式反解),那我就去寻找在这个边界内的其他函数群!

我把目光投向了 $\ln x_1 + \ln x_2 \sim C(m)$。能行吗?
设和 $S_{\ln} = \ln x_1 + \ln x_2$,积 $P_{\ln} = \ln x_1 \ln x_2$。
由 $x - \ln x = m$,得 $x = m + \ln x$。
考察原根乘积 $x_1 x_2$:
路线一(指数):$x_1 x_2 = e^{\ln x_1 + \ln x_2} = e^{S_{\ln}}$
路线二(代数):$x_1 x_2 = (m + \ln x_1)(m + \ln x_2) = m^2 + m(\ln x_1 + \ln x_2) + \ln x_1 \ln x_2 = m^2 + m S_{\ln} + P_{\ln}$
将两者画上等号:
\[
e^{S_{\ln}} = m^2 + m S_{\ln} + P_{\ln}
\]
奇迹重现! 我们可以极其完美地显式解出 $P_{\ln}$:
\[
\mathbf{P_{\ln} = e^{S_{\ln}} - m S_{\ln} - m^2}
\]
这意味着,我完全可以依照之前的逻辑,立刻写出对数偏移的无敌异号通式:
\[
F(x) = (\ln x)^2 - C(m)\ln x + \big( e^{C(m)} - m C(m) - m^2 \big)
\]
这个函数必定能在 $x=1$ 处触发 $F''(1)=0$ 的泰勒锁根神迹,且求导后必定可以完美因式分解,降维秒杀一切对数和的极值点偏移问题!

这段探索让我彻底明白:没有什么是真正的“无脑大招”。异号构造法的本质,就是利用真实韦达多项式进行误差摄动,再辅以初等显式代换引发极值点泰勒奇次穿轴的微积分游戏。只有亲手摸到了它的边界,才算真正驾驭了它。


\begin{example}{$x-\ln x$经典偏移模型,证明$x_{1}+x_{2}<\frac{4m+2}{3}$}{}
已知 $x_1, x_2$ 为方程 $x - \ln x = m$ 的两个相异正实根。证明: $x_{1}+x_{2}<\frac{4m+2}{3}.$ 
\end{example}
\begin{solution}
令 $f(x) = x - \ln x\ (x > 0)$,对其求导得 $f'(x) = \frac{x-1}{x}$。当 $x \in (0, 1)$ 时,$f'(x) < 0$,$f(x)$ 单调递减;当 $x \in (1, +\infty)$ 时,$f'(x) > 0$,$f(x)$ 单调递增。因此 $f(x)$ 在 $x=1$ 处取得极小值 $f(1) = 1$。要使方程存在两个相异实根,需满足 $m > 1$。不妨设 $x_1 < x_2$,由函数单调性可知两根必定分布在 $x=1$ 两侧,即满足 $0 < x_1 < 1 < x_2$。构造辅助函数 $F(x) = x^2 - \frac{4(x-\ln x)+2}{3}x + e^{\frac{2}{3}(1-(x-\ln x))}\ (x > 0)$。将其指数部分展开化简,利用 $e^{\frac{2}{3}(1-x+\ln x)} = x^{\frac{2}{3}}e^{\frac{2}{3}(1-x)}$ 并提取恒正因子 $x$,可将 $F(x)$ 等价变形为:$$F(x) = x \left[ \frac{-x + 4\ln x - 2}{3} + x^{-\frac{1}{3}}e^{\frac{2}{3}(1-x)} \right]$$令 $V(x) = \frac{-x + 4\ln x - 2}{3} + x^{-\frac{1}{3}}e^{\frac{2}{3}(1-x)}$。由于 $x>0$,$F(x)$ 的符号与 $V(x)$ 相同,且易知 $V(1) = 0$。对 $V(x)$ 进一步求导:$$V'(x) = \frac{4-x}{3x} - \frac{1+2x}{3x^{\frac{4}{3}}}e^{\frac{2}{3}(1-x)} = \frac{1}{3x^{\frac{4}{3}}} \left[ (4-x)x^{\frac{1}{3}} - (1+2x)e^{\frac{2}{3}(1-x)} \right]$$令 $U(x) = (4-x)x^{\frac{1}{3}} - (1+2x)e^{\frac{2}{3}(1-x)}$,其符号与 $V'(x)$ 相同,且有 $U(1) = 0$。对 $U(x)$ 再次求导:$$U'(x) = \frac{4(1-x)}{3x^{\frac{2}{3}}} - \frac{4(1-x)}{3}e^{\frac{2}{3}(1-x)} = \frac{4(1-x)}{3} \left[ x^{-\frac{2}{3}} - e^{\frac{2}{3}(1-x)} \right]$$由基础对数不等式可知,对于任意 $x > 0$ 且 $x \neq 1$,恒有 $\ln x < x - 1$。不等式两边同乘 $-\frac{2}{3}$ 得到 $-\frac{2}{3}\ln x > \frac{2}{3}(1-x)$。两边同时取指数可得 $x^{-\frac{2}{3}} > e^{\frac{2}{3}(1-x)}$。因此,对于任意 $x \neq 1$,括号内部分 $x^{-\frac{2}{3}} - e^{\frac{2}{3}(1-x)} > 0$ 恒成立。由此可知,$U'(x)$ 的符号完全由 $1-x$ 决定。当 $x \in (0, 1)$ 时,$U'(x) > 0$,$U(x)$ 单调递增;当 $x \in (1, +\infty)$ 时,$U'(x) < 0$,$U(x)$ 单调递减。故 $U(x)$ 在 $x=1$ 处取得全局最大值,即对于任意 $x \neq 1$,恒有 $U(x) < U(1) = 0$。这表明 $V'(x) < 0$ 恒成立,即 $V(x)$ 在 $(0, +\infty)$ 上单调递减。结合 $V(1) = 0$ 易知:当 $x \in (0, 1)$ 时,$V(x) > 0$,进而 $F(x) > 0$;当 $x \in (1, +\infty)$ 时,$V(x) < 0$,进而 $F(x) < 0$。由前述已知 $0 < x_1 < 1 < x_2$,代入该结论可得 $F(x_1) > 0$ 且 $F(x_2) < 0$,必然有:$F(x_1) > F(x_2)$,剩余过程显然。
\end{solution}

\begin{example}{$x-\ln x$经典偏移模型,证明$x_{1}+x_{2}<m+\sqrt{m}.$ }{}
已知 $x_1, x_2$ 为方程 $x - \ln x = m$ 的两个相异正实根。证明: $x_{1}+x_{2}<m+\sqrt{m}.$ 
\end{example}
\begin{solution}
已知 $x_1, x_2$ 为方程 $x-\ln x = m$ 的两个不相等的正实数根。令 $g(x) = x-\ln x \ (x > 0)$,对其求导得 $g'(x) = \frac{x-1}{x}$。当 $x \in (0, 1)$ 时,$g'(x) < 0$,$g(x)$ 单调递减;当 $x \in (1, +\infty)$ 时,$g'(x) > 0$,$g(x)$ 单调递增。因此 $g(x)$ 的最小值为 $g(1) = 1$。要使方程有两个不相等的实数根,必须满足 $m > 1$,且两根必定分布在 $x=1$ 的两侧。不妨设 $0 < x_1 < 1 < x_2$,此时有 $x_1 - \ln x_1 = m$ 且 $x_2 - \ln x_2 = m$。

构造辅助函数 $F(x)=x^{2}-(x-\ln x+\sqrt{x-\ln x})x+e^{\sqrt{x-\ln x}-x+\ln x} \ (x>0)$。提取公因式 $x$ 并利用等式 $e^{\ln x} = x$,可将 $F(x)$ 变形为:
$$F(x) = x \left( \ln x - \sqrt{x-\ln x} + e^{\sqrt{x-\ln x}-x} \right)$$
令 $u = \sqrt{x-\ln x}$。因为 $x-\ln x \ge 1$,故 $u \ge 1$ 恒成立。定义函数 $G(x) = \ln x - u + e^{u-x}$,当 $x=1$ 时,$u=1$,可得 $G(1) = 0$。由于 $x>0$,$F(x)$ 的符号完全由 $G(x)$ 决定。

对于任意 $x>0$,恒有 $x - \ln x - (\ln x)^2 > 0$,从而有 $u > \ln x$,即 $u-\ln x > 0$ 恒成立。为判断 $G(x)$ 的符号,只需比较 $e^{u-x}$ 与 $u-\ln x$ 的大小。对两边取自然对数,构造新函数 $f(x) = \ln(u - \ln x) - u + x$,且 $f(1) = 0$。

对 $f(x)$ 求导。由 $u^2 = x-\ln x$ 两边求导得 $u' = \frac{x-1}{2xu}$。利用复合函数求导法则,有:
$$f'(x) = \frac{u' - \frac{1}{x}}{u - \ln x} - u' + 1 = \frac{u' - \frac{1}{x} + (1 - u')(u - \ln x)}{u - \ln x}$$
将 $u'$ 代入并通分,同时注意到 $u - \ln x = u^2 + u - x$,化简可得:
$$f'(x) = \frac{(2xu - x + 1)(u^2 + u - x) + x - 1 - 2u}{2xu(u-\ln x)}$$
由于分母 $2xu(u-\ln x) > 0$ 恒成立,所以 $f'(x)$ 的符号由分子决定。将分子展开并利用 $u^2 = x-\ln x$ 降次合并同类项,得到分子可表示为 $B(x) - u C(x)$,其中 $B(x) = (x+1)(2x - 1 - \ln x)$,$C(x) = 2x\ln x + x + 1$。显然当 $x > 0$ 时,$B(x) > 0$ 与 $C(x) > 0$ 恒成立。判定分子是否大于 $0$,即判断 $B(x) > u C(x)$,等价于证明 $B(x)^2 - (x - \ln x)C(x)^2 > 0$。
将上述表达式完全展开并令 $y = \ln x$,可得关于 $x$ 和 $y$ 的多项式:
$$D(x, y) = 4x^2 y^3 + (-4x^3 + 5x^2 + 6x + 1) y^2 + (-8x^3 - 9x^2 + 2x + 3) y + (4x^4 + 3x^3 - 5x^2 - 3x + 1)$$
当 $x \to 1$ 即 $y \to 0$ 时,利用泰勒展开可知该多项式的最低阶主元为 $2(x-1)^2 > 0$;并且对于定义域内满足 $x \neq 1$ 的所有 $x$,该多项式全局恒大于 $0$。由此可知分子恒大于 $0$,故 $f'(x) > 0$ 在 $(0, +\infty)$ 上恒成立。

因此,$f(x)$ 在 $(0, +\infty)$ 上单调递增。当 $x < 1$ 时,$f(x) < 0 \implies G(x) > 0 \implies F(x) > 0$;当 $x > 1$ 时,$f(x) > 0 \implies G(x) < 0 \implies F(x) < 0$。回到方程的两根 $0 < x_1 < 1 < x_2$,由上述单调性分析可知必有 $F(x_1) > 0$ 且 $F(x_2) < 0$,从而有不等式 $F(x_1) > F(x_2)$。剩余过程显然。
\end{solution}


\textbf{例7} 已知 $x_1, x_2$ 为方程 $x - \ln x = m$ 的两个相异正实根。证明: $x_{1}+x_{2}>\frac{2}{3}m+\frac{4}{3}\sqrt{m}.$ 
解析:令 $F(x)=x^{2}-(\frac{2}{3}(x-\ln x)+\frac{4}{3}\sqrt{x-\ln x})x+e^{\frac{4}{3}(\sqrt{x-\ln x}-(x-\ln x))}$, 求导分析$F(x)$可得 $\notin F(x_{1})<F(x_{2})$, 消$m$, 因式分解得
$x_{1}+x_{2}>\frac{2}{3}m+\frac{4}{3}\sqrt{m}.$

\textbf{例8} 证明: $x_{1}+x_{2}<m+\frac{m \ln m}{m-1}.$
解析:令 $F(x)=x^{2}-((x-\ln x)+\frac{(x-\ln x)\ln(x-\ln x)}{(x-\ln x)-1})x+e^{\frac{(x-\ln x)\ln(x-\ln x)}{(x-\ln x)-1}-(x-\ln x)}$, 求导分析$F(x)$可得
\[
\begin{cases}
F(x)\le0,&x\ge1\\ 
F(x)>0,&x<1
\end{cases},
\]
故 $(F(x_{1})>F(x_{2})$, 消$m$, 因式分解得 $x_{1}+x_{2}<m+\frac{m \ln m}{m-1}.$

\textbf{例9} 证明: $x_{1}+x_{2}>m+\frac{1}{m}+\ln m$. 
解析:令 $F(x)=x^{2}-((x-\ln x)+\frac{1}{(x-\ln x)}+\ln(x-\ln x))x+e^{\frac{1}{(x-\ln x)}+\ln(x-\ln x)-(x-\ln x)},$ 求导分析$F(x)$ 可得
\[
\begin{cases}
F(x)\ge0,&x\ge1\\ 
F(x)<0,&x<1
\end{cases},
\]
故 $F(x_{1})<F(x_{2})$, 消$m$, 因式分解得$x_1 +x_{2}>m+\frac{1}{m}+\ln m.$

\textbf{例10} 证明: $x_{1}+x_{2}-\frac{2}{x_{1}+x_{2}}>m+\ln m.$ 
解析:即证 $E(x_{1}+x_{2})^{2}-(m+\ln m)(x_{1}+x_{2})-2>0$, 即证 T7导学猫高考学科交流群883431056 $x_{1}+x_{2}>\frac{1}{2}(m+\ln m+\sqrt{(m+\ln m)^{2}+8}).$ 记 $s(x)=\frac{1}{2}(x-\ln x+\ln(x-\ln x)+\sqrt{(x-\ln x+\ln(x-\ln x)^{2}+8})$ 令 求导分析$F(x)$可得 $F(x)=x^{2}-s(x)x+e^{s(x)-2(x-\ln x)}$,
\[
\begin{cases}
F(x)\le0,&x\ge1\\ 
F(x)>0,&x<1
\end{cases},
\]
$\forall\forall F(x_{1})<F(x_{2})$, 因式分解得 $1x_{1}+x_{2}>\frac{1}{2}(m+\ln m+\sqrt{(m+\ln m)^{2}+8}).$

此类方法的关键在于对$F(x)$单调性质的说明, 鉴于写此文章的目的时说明此方法的通用性, 本文并为细致的把每个$F(x)$的单调性都完整证明, 有兴趣的读者可以自行证明; 本方法对于精度低的题, 可以大大降低运算成本, 但对于精度高的题, 与另一通法对称构造的计算量不分伯仲。