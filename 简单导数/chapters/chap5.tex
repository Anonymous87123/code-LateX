\chapter{插入中间函数}
\section{插值+配导数型}
\begin{example}{}{}
    求证:$\e^x-\e x\geqslant (x\ln x)^2$
\end{example}
\begin{solution}
为证明原不等式 $e^x - ex \ge (x\ln x)^2$,设 $L(x) = e^x - ex$,$R(x) = (x\ln x)^2$。由于 $L(1) = R(1) = 0$ 且 $L'(1) = R'(1) = 0$,考虑构造多项式 $P(x) = (x-1)^2(Ax^2 + Bx + C)$ 作为中间过渡函数,以满足 $P(1)=0$ 且 $P'(1)=0$。

考察函数在 $x=2$ 处的值与导数,已知 $R(2) \approx 1.922$,$L(2) \approx 1.952$;$R'(2) \approx 4.694$,$L'(2) \approx 4.671$。为使多项式严格介于两函数之间,选取适当的有理数,令 $P(2) = \frac{31}{16} \in (1.922, 1.952)$,$P'(2) = \frac{75}{16} \in (4.671, 4.694)$。将条件代入多项式及其导数,可得方程组 $4A + 2B + C = \frac{31}{16}$ 与 $12A + 5B + 2C = \frac{75}{16}$。消去 $C$ 得到 $4A + B = \frac{13}{16}$。取 $A = \frac{1}{16}$,解得 $B = \frac{9}{16}, C = \frac{9}{16}$。由此得到多项式 $P(x) = \frac{1}{16}(x-1)^2(x^2 + 9x + 9) = \frac{1}{16}(x^4 + 7x^3 - 8x^2 - 9x + 9)$。以下证明原不等式等价的两个部分:$P(x) \ge (x\ln x)^2$ 且 $e^x - ex \ge P(x)$。

首先证明 $P(x) \ge (x\ln x)^2$ 对 $x \in (0, +\infty)$ 恒成立。
令差值函数 $F(x) = P(x) - x^2\ln^2 x$,易知 $F(1) = 0, F'(1) = 0$。
当 $x \in (0, 1]$ 时,由对数不等式性质,恒有 $-\ln x \le \frac{(1/x)^2 - 1}{2(1/x)}$。原不等式变形为 $\frac{1}{16}(\frac{1}{x}-1)^2(\frac{1}{x^2}+\frac{9}{x}+9) \ge (\frac{-\ln x}{x})^2$。因此只需证明 $(\frac{1}{x}-1)\sqrt{\frac{9}{x^2}+\frac{9}{x}+1} \ge 2((\frac{1}{x})^2 - 1) = 2(\frac{1}{x}-1)(\frac{1}{x}+1)$ 成立。由于 $x \in (0, 1]$,故 $\frac{1}{x} \ge 1$。两边约去非负因子 $\frac{1}{x}-1$($x=1$ 时等式自然成立),只需证 $\sqrt{\frac{9}{x^2}+\frac{9}{x}+1} \ge 2(\frac{1}{x}+1)$。两边皆正,平方并整理得 $\frac{5}{x^2} + \frac{1}{x} - 3 \ge 0$。因 $\frac{1}{x} \ge 1$,上式左端显然大于等于 $3 > 0$,故当 $x \in (0, 1]$ 时,$F(x) \ge 0$ 成立。

当 $x \in [1, +\infty)$ 时,计算四阶导数 $F^{(4)}(x) = \frac{3}{2} + \frac{4\ln x + 2}{x^2}$。因 $x \ge 1$,故 $\ln x \ge 0$,$F^{(4)}(x) > \frac{3}{2} > 0$,即 $F'''(x)$ 单调递增。由端点值 $F'''(1) = -1.875 < 0$ 与 $F'''(2) \approx 1.237 > 0$ 可知,$F'''(x)$ 在 $(1, 2)$ 内有唯一零点。由此推知 $F''(x)$ 先递减后递增。验证 $F''(1) = 0.375 > 0$,$F''(1.5) \approx -0.136 < 0$,$F''(2) \approx 0.123 > 0$,表明 $F''(x)$ 恰有两个零点。进而 $F'(x)$ 从 $F'(1) = 0$ 起先增变正,后减穿过横轴变负,最后递增。由 $F'(2) \approx -0.007 < 0$ 知,$F'(x)$ 在 $x > 2$ 处存在唯一正零点,且 $F(x)$ 在该点取得全局极小值。
对 $x \ge 2$,由于 $F'''(x) > 0$,故恒有 $F''(x) \ge F''(2) > 0.122$。应用带有拉格朗日余项的泰勒公式展开:$F(x) = F(2) + F'(2)(x-2) + \frac{F''(c)}{2}(x-2)^2 \ge F(2) + F'(2)(x-2) + \frac{0.122}{2}(x-2)^2$。该二次函数的最小值为 $F(2) - \frac{(F'(2))^2}{2 \times 0.122}$。代入已知边界估算 $F(2) > 0.0110$,$|F'(2)| < 0.015$,极小值不小于 $0.0110 - \frac{0.000225}{0.244} \approx 0.0101 > 0$。因此,当 $x \in [1, +\infty)$ 时,$F(x) \ge 0$。

其次证明 $e^x - ex \ge P(x)$ 对 $x \in (0, +\infty)$ 恒成立。
令差值函数 $H(x) = e^x - ex - P(x)$,已知 $H(1) = 0, H'(1) = 0$。
当 $x \in (0, 1]$ 时,$H^{(4)}(x) = e^x - 1.5$,其唯一零点为 $\ln 1.5 \approx 0.405$。验证端点 $H'''(0) = -1.625 < 0$ 与 $H'''(1) \approx -1.407 < 0$,可知 $H'''(x)$ 先减后增且在 $(0, 1]$ 内恒负。因此 $H''(x)$ 在 $(0, 1]$ 单调递减,由 $H''(1) \approx 0.343 > 0$,知 $H''(x) > 0$ 恒成立。进而 $H'(x)$ 单调递增,由 $H'(1) = 0$ 推知 $H'(x) \le 0$ 恒成立。故 $H(x)$ 在该区间单调递减,结合 $H(1) = 0$,得 $H(x) \ge 0$。

当 $x \in [1, +\infty)$ 时,$H^{(4)}(x) = e^x - 1.5 > 1.2 > 0$,说明 $H'''(x)$ 单调递增。由 $H'''(1) < 0$ 与 $H'''(2) \approx 1.764 > 0$,知 $H'''(x)$ 在 $(1, 2)$ 内有唯一零点。这推得 $H''(x)$ 先减后增。验证端点 $H''(1) > 0$,$H''(1.5) < 0$,$H''(2) \approx 0.139 > 0$,知其有两个零点。由此可知 $H'(x)$ 先增变正,再减变负,最后递增。由 $H'(2) \approx -0.0165 < 0$ 知,$H'(x)$ 在 $x > 2$ 处存在唯一正零点,且 $H(x)$ 在此处取得极小值。
对 $x \ge 2$,因 $H''(x) \ge H''(2) > 0.137$,再次应用泰勒公式托底:$H(x) \ge H(2) + H'(2)(x-2) + \frac{0.137}{2}(x-2)^2$。该二次函数的最小值为 $H(2) - \frac{(H'(2))^2}{2 \times 0.137}$。结合边界 $H(2) > 0.0115$,$|H'(2)| < 0.0195$,极小值不小于 $0.0115 - \frac{0.00038}{0.274} \approx 0.0101 > 0$。因此,当 $x \in [1, +\infty)$ 时,$H(x) \ge 0$。

综上所述,不等式 $e^x - ex \ge P(x) \ge (x\ln x)^2$ 对任意 $x \in (0, +\infty)$ 均严格成立,原命题得证。
\end{solution}

\begin{example}{}{}
对任意 $x \in (0, 1)$,不等式 $x^{1-x}(1-x)^x\le \frac{\sin\pi x}{2}$ 均成立
\end{example}
\begin{solution}
\textbf{附:中间函数 $M(x)$ 构造的推导解析}

原不等式两端的函数 $L(x) = x^{1-x}(1-x)^x$ 与 $R(x) = \frac{\sin \pi x}{2}$ 在区间 $(0,1)$ 上均关于直线 $x = \frac{1}{2}$ 对称。为使中间函数 $M(x)$ 具备相同的对称性,可设 $t = x(1-x)$,并构造关于 $t$ 的二次多项式 $f(t) = at^2 + bt + c$ 作为中间函数的基本形式。

首先考察函数在端点与对称中心的零阶值匹配。当 $x \to 0$ 时,$t \to 0$。此时 $\lim_{x \to 0} L(x) = 0$ 且 $\lim_{x \to 0} R(x) = 0$,由连续性要求 $f(0) = 0$,解得 $c = 0$。在对称中心 $x = \frac{1}{2}$ 处,$t = \frac{1}{4}$,此时两端函数均达到极值且 $L\left(\frac{1}{2}\right) = R\left(\frac{1}{2}\right) = \frac{1}{2}$。代入中间函数得 $f\left(\frac{1}{4}\right) = \frac{a}{16} + \frac{b}{4} = \frac{1}{2}$,化简得参数约束条件 $a + 4b = 8$。

其次考察端点处的渐近阶匹配。当 $x \to 0$ 时,$t \sim x$,中间函数局部展开为 $f(t) \sim bx$。左侧函数在 $x \to 0$ 时的等价无穷小展开为 $L(x) = x \mathrm{e}^{-x\ln x + x\ln(1-x)} \sim x$,即起步一次项系数为 $1$。右侧函数在 $x \to 0$ 时展开为 $R(x) \sim \frac{\pi}{2}x$,一次项系数为 $\frac{\pi}{2}$。为使 $f(t)$ 在 $x \to 0$ 附近满足局部条件 $L(x) \le f(t) \le R(x)$,其一次项系数必须介于两者之间,即要求 $1 \le b \le \frac{\pi}{2}$。

进一步考察对称中心处的局部曲率匹配(埃米特插值)。在极值点 $x = \frac{1}{2}$ 处,三个函数相交且一阶导数均为 $0$,决定函数局部上下关系的关键在于二阶导数对应的二次项系数。令 $u = x - \frac{1}{2}$,将三个函数在 $u=0$ 处进行泰勒展开。右侧函数展开为 $R\left(u+\frac{1}{2}\right) = \frac{\cos \pi u}{2} = \frac{1}{2} - \frac{\pi^2}{4}u^2 + \mathcal{O}(u^4)$。对于中间函数,将 $t = \frac{1}{4} - u^2$ 及 $a = 8-4b$ 代入 $f(t)$,展开得 $f(t) = \frac{1}{2} - (4-b)u^2 + au^4$。对于左侧函数,先对 $\ln L(x)$ 在 $u=0$ 处展开,有 $\ln L\left(u+\frac{1}{2}\right) = -\ln 2 - 6u^2 - \frac{28}{3}u^4 + \mathcal{O}(u^6)$,还原为指数形式可得 $L\left(u+\frac{1}{2}\right) = \mathrm{e}^{-\ln 2 - 6u^2 - \frac{28}{3}u^4 + \mathcal{O}(u^6)} = \frac{1}{2} - 3u^2 + \frac{13}{3}u^4 + \mathcal{O}(u^6)$。为保证在极值点局部区间内存在 $L(x) \le f(t) \le R(x)$,这三个展开式的二次项系数必须满足不等式链 $-3 \le -(4-b) \le -\frac{\pi^2}{4}$,解得 $1 \le b \le 4 - \frac{\pi^2}{4}$。

最后,通过比较高阶项以确定参数的最简取值。综合渐近阶与局部曲率的要求,$b$ 的取值范围为 $\left[1, 4 - \frac{\pi^2}{4}\right]$。若取下确界 $b = 1$,则 $a = 4$。此时中间函数与左侧函数的二次项系数同为 $-3$,需进一步比较四次项系数。中间函数展开式的四次项系数为 $a = 4$,而左侧函数展开式的四次项系数为 $\frac{13}{3}$。由于 $\frac{13}{3} > 4$,这意味着在离开极值点后,左侧函数的下降速度慢于中间函数,将反超中间函数从而导致局部放缩失效。因此,必须严格满足 $b > 1$。为使多项式参数尽量简明,在合法区间 $\left(1, 4 - \frac{\pi^2}{4}\right]$ 内选取分母较小的有理数 $b = \frac{3}{2}$,代入关系式 $a + 4b = 8$ 解得 $a = 2$。

综上所述,满足所有端点渐近展开和极值点泰勒展开钳制条件的中间过渡多项式为 $f(t) = 2t^2 + \frac{3}{2}t$。将其还原为关于 $x$ 的表达式,即为证明中所构造的中间函数 $M(x) = 2x^2(1-x)^2 + \frac{3}{2}x(1-x)$。

令 $L(x) = x^{1-x}(1-x)^x$,$M(x) = 2x^2(1-x)^2 + \frac{3}{2}x(1-x)$,$R(x) = \frac{\sin \pi x}{2}$。原命题等价于证明对任意 $x \in (0, 1)$,恒有 $L(x) \le M(x) \le R(x)$ 成立。
由于 $L(x)$、$M(x)$ 及 $R(x)$ 在定义域 $(0, 1)$ 上均关于直线 $x = \frac{1}{2}$ 对称,因此只需证明当 $x \in (0, \frac{1}{2}]$ 时,不等式 $L(x) \le M(x) \le R(x)$ 成立即可。

先证明左侧不等式 $L(x) \le M(x)$。
当 $x \in (0, \frac{1}{2}]$ 时,两端函数值均大于 $0$。对不等式两边同时取自然对数,即证 $(1-x)\ln x + x\ln(1-x) \le \ln\left(2x^2(1-x)^2 + \frac{3}{2}x(1-x)\right)$ 成立。
令 $t = x(1-x)$,当 $x \in (0, \frac{1}{2}]$ 时,$t \in (0, \frac{1}{4}]$ 且单调递增。不等式左边可恒等变形为 $\ln t - x\ln x - (1-x)\ln(1-x)$,右侧可化为 $\ln t + \ln\left(2t + \frac{3}{2}\right)$。消去两侧相同的 $\ln t$,原不等式等价于证明 $-x\ln x - (1-x)\ln(1-x) \le \ln\left(2t + \frac{3}{2}\right)$。
构造函数 $H(x) = \ln\left(2t + \frac{3}{2}\right) + x\ln x + (1-x)\ln(1-x)$,目标转化为证明在 $x \in (0, \frac{1}{2}]$ 上 $H(x) \ge 0$ 恒成立。
对 $H(x)$ 求一阶导数,注意到 $\frac{\mathrm{d}t}{\mathrm{d}x} = 1-2x$,有
$$ H'(x) = \frac{2(1-2x)}{2t + \frac{3}{2}} + \ln x + 1 - \ln(1-x) - 1 = \frac{4(1-2x)}{4t + 3} + \ln x - \ln(1-x). $$
继续对 $H'(x)$ 求二阶导数,利用除法求导法则得
$$ H''(x) = \frac{-8(4t+3) - 4(1-2x) \cdot 4(1-2x)}{(4t+3)^2} + \frac{1}{x} + \frac{1}{1-x}. $$
由于 $(1-2x)^2 = 1 - 4x + 4x^2 = 1 - 4t$,且 $\frac{1}{x} + \frac{1}{1-x} = \frac{1}{x(1-x)} = \frac{1}{t}$,可将 $H''(x)$ 完全转化为关于 $t$ 的表达式:
$$ H''(x) = \frac{-32t - 24 - 16(1-4t)}{(4t+3)^2} + \frac{1}{t} = \frac{32t - 40}{(4t+3)^2} + \frac{1}{t}. $$
通分并合并同类项,得
$$ H''(x) = \frac{t(32t - 40) + (4t+3)^2}{t(4t+3)^2} = \frac{32t^2 - 40t + 16t^2 + 24t + 9}{t(4t+3)^2} = \frac{48t^2 - 16t + 9}{t(4t+3)^2}. $$
对于分子二次多项式 $48t^2 - 16t + 9$,其判别式 $\Delta = (-16)^2 - 4 \times 48 \times 9 = 256 - 1728 < 0$,且二次项系数 $48 > 0$,故对任意实数 $t$,恒有 $48t^2 - 16t + 9 > 0$。由于在区间 $x \in (0, \frac{1}{2})$ 内 $t > 0$,因此 $H''(x) > 0$ 恒成立。
这表明一阶导数 $H'(x)$ 在区间 $(0, \frac{1}{2}]$ 上严格单调递增。结合 $H'\left(\frac{1}{2}\right) = 0 + \ln\frac{1}{2} - \ln\frac{1}{2} = 0$,可知当 $x \in (0, \frac{1}{2})$ 时,恒有 $H'(x) < 0$。
从而原函数 $H(x)$ 在 $(0, \frac{1}{2}]$ 上严格单调递减。代入端点值 $H\left(\frac{1}{2}\right) = \ln\left(2 \times \frac{1}{4} + \frac{3}{2}\right) + \frac{1}{2}\ln\frac{1}{2} + \frac{1}{2}\ln\frac{1}{2} = \ln 2 - \ln 2 = 0$,可知对于所有的 $x \in (0, \frac{1}{2}]$,均有 $H(x) \ge 0$ 成立。左侧不等式得证。

接着证明右侧不等式 $M(x) \le R(x)$。
为分析函数性质,作平移代换令 $u = \frac{1}{2} - x$。当 $x \in (0, \frac{1}{2}]$ 时,$u \in [0, \frac{1}{2})$。
此时 $t = x(1-x) = \left(\frac{1}{2}-u\right)\left(\frac{1}{2}+u\right) = \frac{1}{4} - u^2$。代入中间函数 $M(x)$ 展开得:
$$ M(x) = 2\left(\frac{1}{4} - u^2\right)^2 + \frac{3}{2}\left(\frac{1}{4} - u^2\right) = 2\left(\frac{1}{16} - \frac{1}{2}u^2 + u^4\right) + \frac{3}{8} - \frac{3}{2}u^2 = \frac{1}{2} - \frac{5}{2}u^2 + 2u^4. $$
右侧函数转化为 $R(x) = \frac{\sin\left(\pi\left(\frac{1}{2}-u\right)\right)}{2} = \frac{\cos \pi u}{2}$。
原不等式等价于 $\frac{1}{2} - \frac{5}{2}u^2 + 2u^4 \le \frac{\cos \pi u}{2}$。两边同乘 $2$ 并移项,构造判定函数 $g(u) = \cos \pi u - 1 + 5u^2 - 4u^4$。只需证明当 $u \in [0, \frac{1}{2})$ 时,$g(u) \ge 0$ 恒成立。
对 $g(u)$ 依次求各阶导数:
$$ g'(u) = -\pi\sin \pi u + 10u - 16u^3, $$
$$ g''(u) = -\pi^2\cos \pi u + 10 - 48u^2, $$
$$ g'''(u) = \pi^3\sin \pi u - 96u, $$
$$ g^{(4)}(u) = \pi^4\cos \pi u - 96, $$
$$ g^{(5)}(u) = -\pi^5\sin \pi u. $$
当 $u \in (0, \frac{1}{2})$ 时,$g^{(5)}(u) < 0$ 恒成立,因此 $g^{(4)}(u)$ 在该区间内严格单调递减。考虑到 $\pi > 3.14$,有 $\pi^4 > 3.14^4 \approx 97.2 > 96$。因此 $g^{(4)}(0) = \pi^4 - 96 > 0$,又 $g^{(4)}\left(\frac{1}{2}\right) = -96 < 0$。由零点存在定理可知,$g^{(4)}(u)$ 在 $\left(0, \frac{1}{2}\right)$ 内存在唯一零点。这表明 $g'''(u)$ 在该区间内先严格单调递增,后严格单调递减。
考虑到 $\pi < 3.15$,有 $\pi^3 < 3.15^3 \approx 31.3 < 48$。因此 $g'''(0) = 0$,且 $g'''\left(\frac{1}{2}\right) = \pi^3 - 48 < 0$。因 $g'''(u)$ 从 $0$ 上升至正值后再下降至负值,故 $g'''(u)$ 在 $\left(0, \frac{1}{2}\right)$ 内存在唯一零点。同理推知,$g''(u)$ 在该区间内先严格单调递增,后严格单调递减。
考虑到 $\pi^2 < 10$,有 $g''(0) = 10 - \pi^2 > 0$,且 $g''\left(\frac{1}{2}\right) = 10 - 12 = -2 < 0$。因 $g''(u)$ 从正值上升后再下降至负值,故 $g''(u)$ 在 $\left(0, \frac{1}{2}\right)$ 内存在唯一零点。推知 $g'(u)$ 在该区间内亦先严格单调递增,后严格单调递减。
计算端点值得 $g'(0) = 0$,且 $g'\left(\frac{1}{2}\right) = -\pi + 5 - 2 = 3 - \pi < 0$。因 $g'(u)$ 从 $0$ 上升至正值后再下降至负值,故 $g'(u)$ 在 $\left(0, \frac{1}{2}\right)$ 内存在唯一零点。
综合以上分析可知,原函数 $g(u)$ 的导数 $g'(u)$ 在区间内先正后负。因此,$g(u)$ 在区间 $\left[0, \frac{1}{2}\right)$ 内先严格单调递增,后严格单调递减。
验证区间端点值:
$$ g(0) = \cos 0 - 1 + 0 = 0, $$
$$ g\left(\frac{1}{2}\right) = \cos\frac{\pi}{2} - 1 + 5 \times \frac{1}{4} - 4 \times \frac{1}{16} = 0. $$
由于连续函数 $g(u)$ 在区间起点的函数值为 $0$,内部先增后减,且在区间终点落回至 $0$,故对于任意 $u \in \left(0, \frac{1}{2}\right)$,恒有 $g(u) > 0$ 成立。右侧不等式得证。

综上所述,当 $x \in \left(0, \frac{1}{2}\right]$ 时,左中右不等式链成立。由对称性,对任意 $x \in (0, 1)$,不等式 $x^{1-x}(1-x)^x \le 2x^2(1-x)^2 + \frac{3}{2}x(1-x) \le \frac{\sin\pi x}{2}$ 均成立。证毕。
\end{solution}


\section{估阶+配导数型}
\begin{example}{}{}
    对于任意$x\in(0,1)$,恒成立$2\ln 2 \cdot \ln(\sqrt{x}+\sqrt{1-x}) \geq \ln x \cdot \ln(1-x)$
\end{example}
\begin{solution}
要证明原不等式 $2\ln 2 \cdot \ln(\sqrt{x}+\sqrt{1-x}) \geq \ln x \cdot \ln(1-x)$,可将其分别记为左侧函数 $L(x)$ 与右侧函数 $R(x)$。寻找中间函数 $f(x)$ 的过程主要基于渐近线估阶、对称性及切线插值原理。

\textbf{渐近估阶与对称性确定函数结构} \\
当 $x \to 0$ 时,利用泰勒展开分析两侧函数的渐近行为。对于左侧,$\ln(1 + \sqrt{x} - \frac{x}{2} + \cdots) \sim \sqrt{x}$,故 $L(x)$ 的衰减速度为 $O(\sqrt{x})$。对于右侧,$\ln(1-x) \sim -x$,故 $R(x)$ 的衰减速度为 $O(x\ln x)$。
在 $x \to 0$ 附近,$\sqrt{x}$ 的量级大于 $-x\ln x$。要使中间函数 $f(x)$ 满足 $L(x) \geq f(x) \geq R(x)$,其阶数必须与左侧同阶,即保持在 $O(\sqrt{x})$ 级别。由于原不等式关于直线 $x = 1/2$ 完美对称,中间函数的骨架自然被设定为具备相同对称性的形式:$f(x) = C \sqrt{x(1-x)}$。

\textbf{埃米特插值确定参数} \\
利用中心对称点 $x = 1/2$ 确定系数 $C$。计算两侧函数在该点的值,可得 $L(1/2) = (\ln 2)^2$ 且 $R(1/2) = (\ln 2)^2$。这表明两侧函数在对称中心处取得等号。
为了使中间函数无缝夹在两者之间,$f(x)$ 必须同样经过该点,即 $f(1/2) = (\ln 2)^2$。代入骨架公式得到 $C\sqrt{1/4} = (\ln 2)^2$,解得 $C = 2(\ln 2)^2$。由此,中间函数被唯一确定为 $f(x) = 2(\ln 2)^2 \sqrt{x(1-x)}$。

\textbf{二阶导数曲率验证} \\
由于三者关于 $x = 1/2$ 对称,其在该点的一阶导数均为 $0$。进一步计算二阶导数:
$$
\begin{aligned}
L''(1/2) &= -2\ln 2 \approx -1.386 \\
f''(1/2) &= -4(\ln 2)^2 \approx -1.922 \\
R''(1/2) &= 8\ln 2 - 8 \approx -2.455
\end{aligned}
$$
上述结果满足 $L''(1/2) > f''(1/2) > R''(1/2)$。这在几何上意味着在 $x=1/2$ 处,左侧开口最平缓,中间函数次之,右侧开口最陡峭,从而在局部严格保证了不交叉的上下界关系,为全局放缩提供了坚实基础。

\textbf{证明:} 对任意 $x \in (0, 1)$,恒有 $2\ln 2 \cdot \ln(\sqrt{x}+\sqrt{1-x}) \geq 2(\ln 2)^2 \sqrt{x(1-x)} \geq \ln x \cdot \ln(1-x)$。

证明左半部分:$L(x) \geq f(x)$:令$w = 2\sqrt{x(1-x)}$。由 $x \in (0, 1)$ 及均值不等式可知,$w \in (0, 1]$,且当且仅当 $x = 1/2$ 时 $w = 1$。
对左侧真数项进行平方变形:$(\sqrt{x}+\sqrt{1-x})^2 = 1 + 2\sqrt{x(1-x)} = 1 + w$,故 $\sqrt{x}+\sqrt{1-x} = \sqrt{1+w}$。
将 $w$ 代入目标不等式,左侧转化为 $\ln 2 \cdot \ln(1+w)$,右侧转化为 $(\ln 2)^2 w$。两边同除以 $\ln 2 > 0$,等价于证明对于 $w \in (0, 1]$,恒有 $\ln(1+w) \geq w\ln 2$。

构造函数 $g(w) = \ln(1+w) - w\ln 2$。对其求二阶导数,得到 $g''(w) = -\frac{1}{(1+w)^2} < 0$ 恒成立。
因此 $g(w)$ 在闭区间 $[0, 1]$ 上是严格的凹函数。计算端点值可知 $g(0) = 0$ 且 $g(1) = \ln 2 - \ln 2 = 0$。根据凹函数的割线性质,函数图像在两端点连线之上,故对于 $w \in (0, 1]$,恒有 $g(w) \geq 0$。
等号成立当且仅当 $w = 1$,即 $x = 1/2$。左侧不等式得证。

证明右半部分:$f(x) \geq R(x)$:由原不等式关于 $x = 1/2$ 对称,作平移换元,令 $x = \frac{1-y}{2}$,其中 $y \in (-1, 1)$。由对称性,仅需严格证明当 $y \in [0, 1)$ 时不等式成立即可。代入变量后,$1-x = \frac{1+y}{2}$ 且 $\sqrt{x(1-x)} = \frac{1}{2}\sqrt{1-y^2}$。不等式转化为:
$$ (\ln 2)^2 \sqrt{1-y^2} \geq (\ln(1-y)-\ln 2)(\ln(1+y)-\ln 2) $$
移项并构造差值函数 $F(y)$:
$$ F(y) = (\ln 2)^2(1 - \sqrt{1-y^2}) - \ln 2 \ln(1-y^2) + \ln(1-y)\ln(1+y) $$
目标为证明在 $y \in [0, 1)$ 上,$F(y) \leq 0$ 恒成立。

分析端点极限。显然 $F(0) = 0$。当 $y \to 1^-$ 时,利用极限运算律与洛必达法则($\lim_{z \to 0^+} z\ln z = 0$),可得 $\lim_{y \to 1^-} \ln(1-y)[\ln(1+y) - \ln 2] = 0$,从而 $\lim_{y \to 1^-} F(y) = (\ln 2)^2 - (\ln 2)^2 = 0$。

对 $F(y)$ 求导,并提取正因子 $\frac{1}{1-y^2}$,定义符号判定函数 $K(y) = (1-y^2)F'(y)$:
$$ K(y) = (\ln 2)^2 y \sqrt{1-y^2} + 2y \ln 2 - (1+y)\ln(1+y) + (1-y)\ln(1-y) $$
验证端点得 $K(0) = 0$ 且 $\lim_{y \to 1^-}K(y) = 0$。
对 $K(y)$ 求导:
$$ K'(y) = (\ln 2)^2 \frac{1-2y^2}{\sqrt{1-y^2}} + 2\ln 2 - 2 - \ln(1-y^2) $$
为分析 $K'(y)$ 的符号,令 $t = \sqrt{1-y^2}$,由于 $y \in (0, 1)$,则 $t \in (0, 1)$。将 $K'(y)$ 转化为关于 $t$ 的函数 $H(t)$:
$$ H(t) = (\ln 2)^2 \frac{2t^2-1}{t} + 2\ln 2 - 2 - 2\ln t $$

对 $H(t)$ 求导:
$$ H'(t) = \frac{2(\ln 2)^2 t^2 - 2t + (\ln 2)^2}{t^2} $$
分子为开口向上的二次函数,其判别式 $\Delta = 4 - 8(\ln 2)^4$。因 $\ln 2 \approx 0.693 < 0.70$,可知 $(\ln 2)^4 < 0.2401$,从而 $\Delta > 0$。这表明 $H'(t)$ 在实数域内存在两个不等实根,且由韦达定理可知存在一根 $t_1 \in (0, 1)$。因此 $H(t)$ 在 $(0, 1)$ 内先单调递增后单调递减。

由于 $K(y)$ 在闭区间连续且开区间可导,且 $K(0)=0$, $K(1)=0$,由罗尔定理可知,必然存在 $c \in (0, 1)$ 使得 $K'(c) = 0$。这等价于 $H(t)$ 在 $(0, 1)$ 内至少有一个根。
结合 $H(t)$ 先增后减的单调性,且 $\lim_{t \to 0^+} H(t) = -\infty$,$H(1) = (\ln 2)^2 + 2\ln 2 - 2 < 0$,可推断出极大值 $H(t_1)$ 必定严格大于 $0$。由此,连续函数 $H(t)$ 在 $(0, 1)$ 内有且仅有两个根。

将此结论逆推至原导函数。$H(t)$ 有两个根意味着 $K'(y)$ 在 $(0, 1)$ 内有且仅有两个根。由于 $K'(0) = H(1) < 0$,$K'(y)$ 的符号在 $(0, 1)$ 内的演变必然为:负 $\to$ 正 $\to$ 负。
这使得 $K(y)$ 的走势为:先减 $\to$ 后增 $\to$ 再减。结合出发点 $K(0) = 0$ 与终点 $\lim_{y \to 1^-}K(y) = 0$,$K(y)$ 必须在 $(0, 1)$ 内唯一一次穿透 $y$ 轴(从负值变为正值)。设此根为 $y_0$。

因为 $F'(y) = \frac{K(y)}{1-y^2}$,其符号与 $K(y)$ 完全一致。故 $F(y)$ 在 $(0, y_0)$ 上单调递减,在 $(y_0, 1)$ 上单调递增。
考虑到 $F(y)$ 从 $F(0)=0$ 开始下降,且最终极限为 $\lim_{y \to 1^-}F(y)=0$,因此在整个区间 $y \in (0, 1)$ 内,$F(y) \leq 0$ 严格成立。
等号成立当且仅当 $y = 0$,即 $x = 1/2$。右侧不等式得证。

综上所述,原不等式首尾取等的条件完全重合($x = 1/2$),充分必要条件一一对应,证明闭环成立。
\end{solution}


\section{分离+泰勒型}
\begin{example}{}{}
证明当 $x>1$ 时,下式成立:
$$x-\ln x-\frac{\ln(x-\ln x)}{x(x-\ln x-1)}+\ln\frac{\ln(x-\ln x)}{x(x-\ln x-1)}>0$$
\end{example}
\begin{solution}
\textbf{第一部分:深度解密中间函数的构造奥妙}

构造中间函数 $\frac{x-\ln x+1}{2}$ 的过程,并非基于直觉,而是基于“变量解耦”与“泰勒渐近匹配(阶数探测)”的严密推导。

\textbf{1. 动机:变量解耦(主元分离)}
原不等式中 $x$ 与 $\ln x$ 高度耦合,直接处理极易导致求导陷入极其复杂的运算。通过对数项的天然性质 $\ln\frac{\ln t}{x(t-1)} = \ln\frac{\ln t}{t-1} - \ln x$ (其中设 $t = x-\ln x$),可以将含有 $x$ 的独立项全部移至不等式右侧。
变形后的左侧变为 $L(t) = t + \ln\frac{\ln t}{t-1}$,成功将一个双变量纠缠的复杂系统降维成关于 $t$ 的单变量纯函数,为后续局部分析奠定了基础。

\textbf{2. 核心:极值点的切线放缩与泰勒阶数匹配}
不等式的最弱环节(即最易取等的临界点)在于 $x \to 1^+$,此时 $t \to 1^+$。
考察左侧纯函数 $L(t)$ 在 $t \to 1$ 处的局部性质:
极限值为 $\lim_{t \to 1} L(t) = 1 + \ln(1) = 1$。
一阶导数(即切线斜率)为 $L'(t) = 1 + \frac{t-1-t\ln t}{t(t-1)\ln t}$,利用洛必达法则或泰勒展开求极限得 $\lim_{t \to 1} L'(t) = \frac{1}{2}$。
因此,$L(t)$ 在临界点 $t=1$ 处的切线方程自然生成为 $y = 1 + \frac{1}{2}(t-1) = \frac{t+1}{2}$。将 $t=x-\ln x$ 代回,便得到了该中间函数。由于 $L(t)$ 是严格下凸函数,其图像始终位于切线上方,这保证了左侧必然大于此中间函数。

\textbf{3. 必然性:切线作为右侧上界的阶数分析}
在极值临界点进行多项式阶数探测。令 $x = 1+v$ ($v \to 0^+$),复合微元 $t-1 = x-\ln x-1 \approx \frac{1}{2}v^2 - \frac{1}{3}v^3$。
在 $v \to 0$ 处展开到第三阶($v^3$):
左侧展开:$L(x) \approx 1 + \frac{1}{4}v^2 - \frac{1}{6}v^3$
右侧展开:$R(x) \approx 1 + \frac{1}{4}v^2 - \frac{1}{4}v^3$
切线展开:$\frac{t+1}{2} \approx 1 + \frac{1}{4}v^2 - \frac{1}{6}v^3$
在微观极限处,左侧、右侧与切线的常数项和二次项系数完全一致(均为 $1+\frac{1}{4}v^2$)。真正的差距在三阶项 $v^3$ 才显露:由于切线的三阶系数 $-\frac{1}{6}$ 严格大于右侧的三阶系数 $-\frac{1}{4}$,切线从 $x$ 离开 $1$ 的最初瞬间起,就天然凌驾于右侧函数之上。这种通过阶数匹配直接锁定局部压制的方法,构成了放缩的底层逻辑。

\textbf{考场证明:}利用对数运算法则,将含有单变量 $x$ 的对数项移至不等式右侧,原命题等价于证明:
$$x-\ln x+\ln\frac{\ln(x-\ln x)}{x-\ln x-1} > \ln x+\frac{\ln(x-\ln x)}{x(x-\ln x-1)}$$
设 $t = x-\ln x$。由 $x>1$ 时 $(x-\ln x)' = 1-\frac{1}{x} > 0$ 可知,$t$ 在 $(1, +\infty)$ 上单调递增,故 $t > 1-\ln 1 = 1$。
上述不等式可转化为证明当 $x>1$ 且 $t>1$ 时:
$$t+\ln\frac{\ln t}{t-1} > \ln x+\frac{\ln t}{x(t-1)}$$
为证此不等式,引入中间过渡函数 $\frac{t+1}{2}$,将证明拆解为如下两个部分。

第一部分:证明当 $t>1$ 时,$t+\ln\frac{\ln t}{t-1} > \frac{t+1}{2}$。
移项后等价于证明 $\frac{t-1}{2} + \ln(\ln t) - \ln(t-1) > 0$。
构造函数 $g(t) = \frac{t-1}{2} + \ln(\ln t) - \ln(t-1)$。对其求导并通分得:
$$g'(t) = \frac{1}{2} + \frac{1}{t\ln t} - \frac{1}{t-1} = \frac{t(t-3)\ln t + 2(t-1)}{2t(t-1)\ln t}$$
因 $t>1$,分母 $2t(t-1)\ln t > 0$。考察分子,构造函数 $k(t) = t(t-3)\ln t + 2(t-1)$,对其连续求导:
$$k'(t) = (2t-3)\ln t + t - 1$$
$$k''(t) = 2\ln t + 3 - \frac{3}{t} = 2\ln t + 3\left(1 - \frac{1}{t}\right)$$
当 $t>1$ 时,$\ln t > 0$ 且 $1 - \frac{1}{t} > 0$,故 $k''(t) > 0$ 恒成立。
由此推知 $k'(t)$ 在 $(1, +\infty)$ 上单调递增,结合 $k'(1)=0$,得 $k'(t)>0$;进而 $k(t)$ 在 $(1, +\infty)$ 上单调递增,结合 $k(1)=0$,得 $k(t)>0$。
因此,当 $t>1$ 时 $g'(t) > 0$,$g(t)$ 在 $(1, +\infty)$ 上单调递增。由极限 $\lim_{t\to 1^+} \frac{\ln t}{t-1} = 1$ 可知 $\lim_{t\to 1^+} g(t) = 0$。从而当 $t>1$ 时 $g(t) > 0$,该部分不等式成立。

第二部分:证明当 $x>1$ 时,$\frac{t+1}{2} > \ln x+\frac{\ln t}{x(t-1)}$。
将其中的 $t$ 代回为 $x-\ln x$,整理得:
$$\frac{x-3\ln x+1}{2} > \frac{\ln(x-\ln x)}{x(x-\ln x-1)}$$
对不等式右端引入引理:对于任意 $u>1$,恒有 $\frac{\ln u}{u-1} < \frac{1}{\sqrt{u}}$。
证明引理:构造函数 $\varphi(u) = u - 1 - \sqrt{u}\ln u$。对其连续求导得 $\varphi'(u) = 1 - \frac{\ln u + 2}{2\sqrt{u}}$ 及 $\varphi''(u) = \frac{\ln u}{4u\sqrt{u}}$。当 $u>1$ 时,$\varphi''(u) > 0$,故 $\varphi'(u)$ 在 $(1, +\infty)$ 上单调递增,由 $\varphi'(1) = 0$ 得 $\varphi'(u) > 0$;进而 $\varphi(u)$ 单调递增,由 $\varphi(1) = 0$ 得 $\varphi(u) > 0$,引理得证。
应用引理,原不等式右端满足 $\frac{\ln t}{x(t-1)} < \frac{1}{x\sqrt{t}} = \frac{1}{x\sqrt{x-\ln x}}$。故只需证明充分条件:
$$\frac{x-3\ln x+1}{2} \ge \frac{1}{x\sqrt{x-\ln x}}$$
记 $L(x) = x-3\ln x+1$。由 $L'(x) = 1-\frac{3}{x}$ 可知,$L(x)$ 在 $x=3$ 处取得极小值 $4-3\ln 3$。因 $\ln 27 < \ln e^4 = 4$,即 $4-3\ln 3 > 0$,故当 $x>1$ 时 $L(x) > 0$ 恒成立。
将上述待证不等式两端平方并交叉相乘,等价于证明当 $x>1$ 时 $G(x) = x^2(x-\ln x)L^2(x) - 4 \ge 0$。
显然端点处 $G(1) = 0$。对 $G(x)$ 求导并提取公因式 $xL(x)$,得:
$$G'(x) = xL(x)\left[ (3x-2\ln x-1)L(x) + 2(x-\ln x)(x-3) \right]$$
将方括号内的表达式展开并合并同类项,设其为 $M(x)$:
$$M(x) = 5x^2 - 13x\ln x - 4x + 6\ln^2 x + 7\ln x - 1$$
对 $M(x)$ 连续求导:
$$M'(x) = 10x - 13\ln x - 17 + \frac{12\ln x + 7}{x}$$
$$M''(x) = \frac{10x^2 - 13x + 5 - 12\ln x}{x^2}$$
当 $x>1$ 时,利用基础放缩 $\ln x \le x-1$,可对 $M''(x)$ 的分子进一步放缩:
$$10x^2 - 13x + 5 - 12\ln x \ge 10x^2 - 13x + 5 - 12(x-1) = 10x^2 - 25x + 17$$
二次函数 $10x^2 - 25x + 17$ 的判别式 $\Delta = (-25)^2 - 4 \times 10 \times 17 = -55 < 0$。因开口向上且 $\Delta < 0$,该多项式恒大于 $0$,故 $M''(x) > 0$ 在 $x>1$ 时恒成立。
由此推知 $M'(x)$ 在 $(1, +\infty)$ 上单调递增,由 $M'(1) = 0$ 得 $M'(x) > 0$;进而 $M(x)$ 单调递增,由 $M(1) = 0$ 得 $M(x) > 0$。
因 $x>1$ 时 $xL(x) > 0$ 且 $M(x) > 0$,故 $G'(x) = xL(x)M(x) > 0$ 恒成立。这表明 $G(x)$ 在 $(1, +\infty)$ 上单调递增,从而 $G(x) \ge G(1) = 0$ 成立。
综上所述,不等式左端严格大于中间函数,中间函数严格大于右端,原不等式在 $x>1$ 时得证。
\end{solution}

