% ========== 基本包加载 ==========
% 1. 基础宏包
\usepackage{geometry}
\usepackage{fontspec}
\usepackage{amsmath, amsthm, amssymb}
\usepackage{mathrsfs}
\usepackage{enumitem}
\usepackage{graphicx}
\usepackage{array}
\usepackage{ulem}
\usepackage{caption}
\usepackage{tocloft}
% 2. 图形和颜色宏包
\usepackage[dvipsnames]{xcolor}
\usepackage{tikz}
\usetikzlibrary{shapes.geometric}
\usepackage[most]{tcolorbox}
\tcbuselibrary{theorems}
% 3. 页眉页脚宏包
\usepackage{fancyhdr}
\usepackage{lastpage}
% 4. 超链接和书签
\usepackage{hyperref}
\usepackage{bookmark}
% 5. 智能引用
\usepackage{cleveref}
% ========================
% 1. 页面布局设置(窄边距)
\geometry{
  a4paper,
  top=15mm,
  bottom=10mm,
  left=15mm,
  right=15mm,
  headheight=25pt,
  headsep=8mm,
  footskip=15mm,
  includehead,
  includefoot
}
% 2. 字体设置(修改部分)
% 设置全局英文字体
\setmainfont{Times New Roman}
\setsansfont{Arial}
\setmonofont{Consolas}
% 定义字体命令(保持不变)
\newcommand{\kt}{\kaishu} % 楷体
\newcommand{\st}{\songti} % 宋体
\newcommand{\htbf}{\heiti\bfseries} % 黑体加粗

\newcounter{theoremcounter}[section]
\newtcbtheorem[use counter=theoremcounter,number within=section]{word}{单词}{
  enhanced,  
  colback=SeaGreen!10!CornflowerBlue!10,
  colframe=RoyalPurple!55!Aquamarine!100!,
  before upper={\parindent 2em}, % 首行缩进
  arc=3pt, % 圆角
  boxrule=1pt, % 边框粗细
}{word}
\newtcbtheorem[number within=section]{expression}{搭配}{
  enhanced,
  colback=Salmon!20,
  colframe=Salmon!90!Black,
  before upper={\parindent 2em},
  arc=3pt,
  boxrule=1pt,
}{exp}
\newtcbtheorem[number within=section]{wordroot}{词根}{
  enhanced,
  colback=Peach!15,              % 浅橙色背景(比纯Orange更柔和)
  colframe=Orange!80!black,  % 深橙色边框带黑色加深
  before upper={\parindent 2em},
  arc=3pt,
  boxrule=1pt,        % 标题加粗与其他环境统一
}{wordroot}
\newtcbtheorem[number within=section]{affix}{词缀}{
  enhanced,
  colback=Thistle!15,            % 浅紫色背景(柔和)
  colframe=MediumPurple!80!black, % 深紫色边框带黑色加深
  before upper={\parindent 2em},
  arc=3pt,
  boxrule=1pt,
}{affix}
\newtheoremstyle{plain-chinese}% 名称
  {6pt}% 上方空白
  {6pt}% 下方空白
  {\st}% 正文字体(宋体)
  {}% 缩进
  {\heiti}% 标题字体(黑体加粗)
  {.}% 标题后标点
  { }% 标题后空白
  {}% 标题说明
\theoremstyle{plain-chinese}
\newtheorem{eg.}{例句}[section] % 使用与例题相同的计数器
\newtheorem{attention}{注意}[section]
\newtheorem{releventwords}{形近词}[section]

% 4. 智能引用设置(保持不变)
% ========================
\crefname{word}{单词}{单词}
\crefname{expression}{搭配}{搭配}
\crefname{eg.}{例句}{例句}
\crefname{attention}{注意}{注意}
\crefname{releventwords}{形近词}{形近词}
\crefname{wordroot}{词根}{词根}
\crefname{affix}{词缀}{词缀}


% 5. 数学字体设置(修改)
\usepackage{unicode-math} % 更好的数学字体支持
\setmathfont{Latin Modern Math} % 使用默认数学字体
% ========== 自定义命令(保持不变) ==========
\newcommand{\R}{\mathbb{R}} % 实数集
\newcommand{\C}{\mathbb{C}} % 复数集
\newcommand{\Z}{\mathbb{Z}} % 整数集
\newcommand{\N}{\mathbb{N}} % 自然数集
\newcommand{\dif}{\mathrm{d}} % 微分符号

% ========== 图形路径设置(保持不变) ==========
\graphicspath{{./flg/}} % 图片路径

% 6. 页眉页脚设置
% ========================
\usepackage{fancyhdr}
\usepackage{lastpage} % 获取总页数
\pagestyle{fancy}
\fancyhf{} % 清除所有页眉页脚设置

% 通用设置
\fancyhead[L]{\small\kt 半小时记住10000个单词} % 左边:书名(楷体)
\fancyhead[C]{\small\st 还在尬黑出品}
\fancyhead[R]{\small\st 版权所有,侵权必究} % 中间:声明(宋体)
\fancyfoot[C]{\thepage} % 居中页码(自定义样式)

% 正文部分设置
\fancyhead[R]{\small\st\rightmark} % 右边:章节名称(宋体)

% 前言部分设置(使用罗马数字页码)
\fancypagestyle{frontmatter}{
    \fancyhf{}
    \fancyhead[L]{\small\kt 半小时记住10000个单词}
    \fancyhead[C]{\small\st 还在尬黑出品}
    \fancyhead[R]{} % 前言部分无章节名称
    \fancyfoot[C]{\thepage}
    \renewcommand{\headrulewidth}{0.4pt} % 页眉线
    \renewcommand{\footrulewidth}{0pt} % 无页脚线
    \pagenumbering{Roman} % 罗马数字页码
}

% 目录部分设置(使用罗马数字页码,延续前言页码)          % 去掉标题下方的横线
\fancypagestyle{tocmatter}{
    \fancyhf{}
    \fancyhead[L]{\small\kt 半小时记住10000个单词}
    \fancyhead[C]{\small\st 居敬持志~守正出奇} % 居中显示"目录"
    \fancyhead[R]{\small\st 还在尬黑出品} 
    \fancyfoot[C]{\thepage}
    \renewcommand{\headrulewidth}{0.4pt}
    \renewcommand{\footrulewidth}{0pt}
    % 注意:这里不重置页码,延续前面的罗马数字
}
% 正文部分设置(使用阿拉伯数字页码)
\fancypagestyle{mainmatter}{
    \fancyhf{}
    \fancyhead[L]{\small\kt 半小时记住10000个单词}
    \fancyhead[C]{\small\st 居敬持志~守正出奇} 
    \fancyhead[R]{\small\st 还在尬黑出品} 
    \fancyfoot[C]{\thepage}
    \renewcommand{\headrulewidth}{0.4pt} % 页眉线
    \renewcommand{\footrulewidth}{0pt} % 无页脚线
    \pagenumbering{arabic} % 阿拉伯数字页码
}
% 设置章节标记格式
\renewcommand{\chaptermark}[1]{\markboth{#1}{}}
\renewcommand{\sectionmark}[1]{\markright{\thesection.\ #1}}

% 8. 章节标题字体设置
\ctexset{
    chapter = {
        format = \centering, % 整体居中
        nameformat = \kaishu\LARGE, % 编号部分黑体加粗
        titleformat = \songti\LARGE, % 标题部分宋体
        aftername = \quad, % 编号和标题之间的间距
        beforeskip = 30pt, % 标题前的垂直间距
        afterskip = 20pt, % 标题后的垂直间距
        name = {第,章}, % 中文章节编号格式
        number = \chinese{chapter}, % 使用中文数字
    },
    section = {
        format = \raggedright, % 左对齐
        nameformat = \heiti\bfseries\large, % 编号部分黑体加粗
        titleformat = \songti\large, % 标题部分宋体
        aftername = \quad, % 编号和标题之间的间距
        beforeskip = 15pt, % 标题前的垂直间距
        afterskip = 6pt, % 标题后的垂直间距
    },
    subsection = {
        format = \raggedright, % 左对齐
        nameformat = \heiti\bfseries\normalsize, % 编号部分黑体加粗
        titleformat = \songti\normalsize, % 标题部分宋体
        aftername = \quad, % 编号和标题之间的间距
        beforeskip = 6pt, % 标题前的垂直间距
        afterskip = 3pt, % 标题后的垂直间距
    }
}