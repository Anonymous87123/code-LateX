\chapter{构词法}
按照词根是否加词缀以及加何种词缀,可以将派生法分为三级:词根不加缀的零级派生可能产生简单词,原始复合词,转化词和合成词;粘附词根加原生词缀的一级派生产生原始派生词(原生词);自由词根或单词加派生词缀的二级派生产生二级派生词,即传统意义上的派生词。
\section{零级派生的无词缀构词(零派生构词)}
这类单词参加构词的只有词根(粘附词根或自由词根)再加上零派生词缀(就是不加东西,原封不动)
\subsection{粘附词根构词}
(1)单黏附词根:如act, firm, press

(2)双粘附词根:如manu(手)+ script(写), ped(男孩)+ agogue(引导), helic(旋转)+ o +pter(写)
\subsection{自由词根构词}
(1)单自由词根→转化词(词性改变),这也解释了为什么有些词有多个词性。

如act由名词直接变为动词,firm由形容词直接变为动词,man由名词直接变为动词(给……配备人员),full由形容词改元音变为动词fill,gold由m名词改元音变为动词gild给……镀金。

(2)双自由词根→合成词或复合词。如manpower, blackboard, honeymoon等,十分明显。
\section{词根词缀构词}
这种方法生成的词为派生词,又称复杂词或主从词。
\subsection{一级派生:粘附词根+原生词缀(原始派生词或原生词)}
(1)原生前缀 + 粘附词根→前缀原生词。如

ex-(出)+ pend(支付)= expend~~~im-(入)+ port(运)= import输入,进口~~~pro(向前)+ gress(步)= progress

(2)粘附词根 + 原生后缀→后缀原生词。如

equ(相等的)+ -ate(使…)= equate使相等~~~liter(文字)+ -al(…的)= literal文字的~~~vis(看)+ -ible(可…的)= visible可见的

(3)原生前缀 + 粘附词根 + 原生后缀→前后缀原生词。如

circum-(环绕)+ locut(说)+ -ion(表方式)= circumlocution迂回说法;

tele-(远)+ vis(看)+ -ion(表状态)= television~~~~~im-(不)+ pecc(犯罪)= impeccable无瑕疵的

(4)双原生前缀 + 粘附词根(+原生后缀)→双前缀原生词。如果去掉第一个前缀,剩余部分并非前缀派生词。

co-(共)+ in-(在…上)+ cid(降)+ e = coincide v.符合;~~~~~par-(旁)+ en-(进)+ thesis(放置)= parenthesis圆括号

pre-(事先)+ di-(出)+ lect(选)+ -ion(表行为)= predilection n.偏爱,偏好

\subsection{二级派生:自由词根加派生词缀}
(1)派生前缀 + 自由词根→前缀派生词。如en-(使)+ rich = enrich使富裕

(2)自由词根 + 派生后缀→后缀派生词。如real + -ize(使…)= realize使实现

\section{派生词在构词过程中的变化}
本节主要探讨词根词缀的形态特征与派生词词形(顾名思义)的关系。
\subsection{词根的形态特征}
在分析词根的形态特征时,要注意词根的音节数和词根结合端字母(辅音或元音)。常考词根都是单音节或双音节的,其中单音节占比显著高于双音节。从独立性来看,自由词根多是本族语的,粘附词根多是古典语的。

单音节词根大部分为“辅元辅”,即“辅音字母(或其组合)+ 元音字母(可能会包括r音节)+ 辅音字母(或其组合)”。如man, vis, log等,元音字母包括r的例子有work, child, fect, phon, therm等少部分为“元辅”,即“元音字母(可能会包括r音节) + 辅音字母(或其组合)”的,如act, ann, erg, un;极少部分为“辅元”,即“辅音字母(或其组合)+ 元音字母(可能会包括r音节)”的,如bi, ge, sci等。

双音节词根往往是在单音节词根的基础上在头尾加上元音字母和r,也就是说双音节词根往往以元音开头或收尾。但是不包括单音节词跟加上连接字母构成的“伪双音节词根”。如anim, imit, ocul, labor, liter, mater等。
\subsection{词根的形态变化}
在构词过程中,往往出于读者需要或习惯原因,使得派生词的词形并不简单的等于各构词成分形态的总和,往往会发生字母的增加,脱落,变更,以及连字符的使用现象。

一、字母的增加

(1)重读闭音节的词根在添加后缀时,尾部辅音字母发生双写,如:

hap→happen~~~~~~~run→runner~~~~~~~prefer→preferred

refer + -al = referral n. 对相关者的分配,安排

但是在构词过程中重音发生移动的不需要双写,如:

re'fer + -ee = refer'ee n.仲裁人~~~~~pre'fer + -able = 'preferable adj.更可取的

(2)当古典语词根与后缀或另一词根构成一级派生词时,可能增加连接字母。连接字母一般是元音字母,往往出现在接合部两边都是辅音字母的情形,但也有反例。如:

\begin{tabular}{@{}ll@{}}
helic(旋转)+ o + pter(翼)= helicopter&cert(确定)+ i + -fy = certify v.证实\\
anthrop(人)+ o + -logy = anthropology&sent(感觉)+ i + -ment = sentiment n.情感\\
bi(生命)+ o + -logy(研究)= biology&ge-(地球/土地)+ o + graphy(记录)= geography
\end{tabular}

(3)为保持某些位于词尾的重读词根中元音字母读长音,在该词根后添加无语义的默音字母e,使之为开音节。

 in(内)+ clud(关)+ e = include;
 
 ante(前)+ ced(走)+ e = antecede;
 
 pro- + mot(移动)= promote;

二、字母的脱落(主要是词素尾字母的脱落)

(1)在二级派生情况下,在添加后缀时,某些词根的默音字母e脱落,以保证词根原来的音值。

\begin{tabular}{@{}lll@{}}
write + er = writer&create + -ive = creative&persevere + -ance = perseverance\\
use + -able = usable&promote + -ion = promotion
\end{tabular}

(2)有些以辅音字母结尾的前缀与以相同的辅音字母开头的词根结合时,前缀尾字母脱落。最常见的是s。

trans-(穿越)+ spir(呼吸)+ e = transpire v.蒸发

trans- + ship(装运)= tranship v.转运(或transship)

(3)前缀尾字母和词根首字母均为元音,结合时前缀尾字母脱落。

    anti-(相反,对抗)+ arctic(北极)= Antarctic adj.南极的

(4)前缀尾字母为s, x等,在以s, l等辅音字母开头的词根前, 结合时前缀尾字母脱落

ex- (出)+ lect(选择)= elect 

三、字母的变更

(1)在零级派生情况下,通过词根内部元音字母的变化使单词的词性发生转化。

full adj.→fill v.~~~~~~~food n.→feed v.~~~~~~~~gold n. →gild v.

(2)在一级派生情况下,某些前缀的尾辅音字母受拉丁词根首字母的同化,一般变成与词根首字母相同的字母。

\begin{tabular}{@{}llll@{}}
ad- + fect = affect&com- + rect = correct&dis- + fer = differ&ex- + fect = effect n.结果\\
in- +rupt = irrupt&inter- + lect = intellect&sub- + gest = suggest&syn- + pathy = sympathy
\end{tabular}

(3)在二级派生情况下,某些拉丁语源的原始派生词的词根尾辅音字母在添加后缀-ion, -ive, -ible时发生变化。

\begin{tabular}{@{}lll@{}}
absorb + -ion = absorption&produce + -ive = productive&decide + -ion = decision\\
defend + -ive = defensive&permit+ -ible = permissible&respond + -ible = responsible
\end{tabular}

(4)在二级派生情况下,以辅音字母加y结尾的单词在添加派生后缀时,元音字母y变为i。

dry + -ly = drily~~~~~~~apply + -ance = appliance

(5)连字符只与自由词根连用,可以用在自由词根与自由词根之间,也可用在自由词根和派生词缀之间。

1.用在临时缀合的,尚不稳定的,或未获得公认的二次派生词和复合词中。

\begin{tabular}{@{}lll@{}}
pre-battle adj.战前的&post-liberation adj. 解放后的&semi-feudal adj. 半封建的\\
breast-feed adj. 母乳喂养&cross-board adj. 越界的&
\end{tabular}

2.与某些形似单词的派生词缀连用,以避免误解。

construction-wise adv.就建筑而言~~~~~~~ultra-reactionary adj.极其反动的

3.以元音字母结尾的派生前缀与以相同的元音字母开头的单词缀合时,使用连字符避免发音错误。

co-operation~~~~co-owner~~~~~re-edit~~~~~de-escalate~~~~~pre-existing~~~~~anti-inflammatory抗炎的

4.用在派生前缀后,避免该二次派生词与同形的外来借词(多为原生词)相混淆,其含义可以直接拆开分析。
\[
\begin{cases}\text{recover  发现}\\\text{re-cover 重新覆盖}\end{cases},
\begin{cases}\text{recollect 使回想}\\\text{re-collect  再集合}\end{cases},
\begin{cases}\text{reclaim 开垦}\\\text{re-claim  要求恢复}\end{cases},
\begin{cases}\text{resort 求助}\\\text{re-sort  重新分类}\end{cases}\\
\]

5.用在派生前缀和专有名词之前,隔开大小写字母。如pro-Germany

\section{拼读}
词根无论能否独立,均有相对固定的读音,在多数情况下,词根的读音在派生词中保持不变。对单词读音影响最大的是词根与派生词重读音节之间的关系。零级派生情况下默认重音在首字母。如blackboard,manuscript等。从理论上来说,构词成分所带的语言信息的主次与读音的重轻应该是一致的,由此可以归纳出“词根重读,词缀轻读”的经验规律。但是也有反例,有的词缀与词根一起重读。有的词缀能使重读音节转移,有的词缀则会喧宾夺主,成为重读音节。下面介绍几种例外情况:

(1)表示否定意义的派生前缀如un, non在派生词中与词根都要重读(可以类比我们在语文课上给句子提取主干时,否定词虽然不是主谓宾,但也是句子的主干成分,要提取出来)。如 'un'happy,'un'easy,'un'known,'un'close,'non'sense,'non'fiction,'non'intervention等

(2)双音节的前缀原生词由动词转化为名词或形容词时,重音从词根所在音节转移到词缀所在音节。如

\begin{tabular}{@{}lll@{}}
con'tract v.缔结→'contract n.契约&con'duct v.引导→'conduct n.品行\\
pro'gress v.进步→'progress n.前进&re'cord v.记录→'record n.记录\\
pre'sent v.赠送→'present n.礼物 adj.现在的&
\end{tabular}

(3)借用了派生前缀,尤其是拉丁语源的派生词缀,为了表示其有别于同源同形的原生词缀往往与充当词根的单词一起重读。在这种情况下,词缀重读还是轻读,往往可以作为区别派生词缀和原生词缀的标志,亦即区别二级派生词还是一级派生词的标志。如expand和ex-husband,intercept(拦截)和interclass(年级之间)的,prefer和prehuman,reflect和reform,submit和sublease,survive和surcharge等

(4)少数拥有很实际的含义的前缀,在单词中往往重读,以及历史或习惯使然的情况,阅读以下单词加以体会:

\begin{tabular}{@{}lll@{}}
   postman&postposition(后置)&postgraduate&selfcare\\
   self-dependence&semimonthly&semiliterate&differ\\
   common&comfort
\end{tabular}

(5)个别后缀在派生词中总是重读,最为典型的有-ee, -ese, -esque, -ette,查询以下单词读音加以体会:

\begin{tabular}{@{}lll@{}}
   employee&examinee&refugee&Chinese&Japanese\\
   Portuguese葡萄牙的&Viennese维也纳的&grotesque怪异的&picturesque风景如画的   statuesque(人)身材高大优美的;轮廓分明的
   cigarette           kitchenette小厨房        leatherette n. [皮革] 人造革;假皮
(6)有些后缀本身虽不重读,却会使派生词的重音发生有规则的转移,即总是使单词的重音移动到自己的前一个音节上。最有代表性的是 -ian, -ic, -ion, -ity
'grammar→gram'marian n. 语法学家     'period→peri'odic adj.周期性的
'celebrate→cele'bration                  'popular→popu'larity

