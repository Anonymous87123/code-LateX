\chapter{构词法}
按照词根是否加词缀以及加何种词缀,可以将派生法分为三级:词根不加缀的零级派生可能产生简单词,原始复合词,转化词和合成词;粘附词根加原生词缀的一级派生产生原始派生词(原生词);自由词根或单词加派生词缀的二级派生产生二级派生词,即传统意义上的派生词。
\section{零级派生的无词缀构词(零派生构词)}
这类单词参加构词的只有词根(粘附词根或自由词根)再加上零派生词缀(就是不加东西,原封不动)
\subsection{粘附词根构词}
(1)单黏附词根:如act, firm, press

(2)双粘附词根:如manu(手)+ script(写), ped(男孩)+ agogue(引导), helic(旋转)+ o +pter(写)
\subsection{自由词根构词}
(1)单自由词根→转化词(词性改变),这也解释了为什么有些词有多个词性。

如act由名词直接变为动词,firm由形容词直接变为动词,man由名词直接变为动词(给……配备人员),full由形容词改元音变为动词fill,gold由m名词改元音变为动词gild给……镀金。

(2)双自由词根→合成词或复合词。如manpower, blackboard, honeymoon等,十分明显。
\section{词根词缀构词}
这种方法生成的词为派生词,又称复杂词或主从词。
\subsection{一级派生:粘附词根+原生词缀(原始派生词或原生词)}
(1)原生前缀 + 粘附词根→前缀原生词。如
ex-(出)+ pend(支付)= expend

im-(入)+ port(运)= import输入,进口

pro(向前)+ gress(步)= progress

(2)粘附词根 + 原生后缀→后缀原生词。如

equ(相等的)+ -ate(使…)= equate使相等

liter(文字)+ -al(…的)= literal文字的;

vis(看)+ -ible(可…的)= visible可见的

(3)原生前缀 + 粘附词根 + 原生后缀→前后缀原生词。如

circum-(环绕)+ locut(说)+ -ion(表方式)= circumlocution迂回说法;

tele-(远)+ vis(看)+ -ion(表状态)= television;

im-(不)+ pecc(犯罪)= impeccable无瑕疵的

(4)双原生前缀 + 粘附词根(+原生后缀)→双前缀原生词。如果去掉第一个前缀,剩余部分并非前缀派生词。

co-(共)+ in-(在…上)+ cid(降)+ e = coincide v.符合;

par-(旁)+ en-(进)+ thesis(放置)= parenthesis圆括号

pre-(事先)+ di-(出)+ lect(选)+ -ion(表行为)= predilection n.偏爱,偏好

\subsection{二级派生:自由词根加派生词缀}
(1)派生前缀 + 自由词根→前缀派生词。如en-(使)+ rich = enrich使富裕

(2)自由词根 + 派生后缀→后缀派生词。如real + -ize(使…)= realize使实现

\section{派生词在构词过程中的变化}
本节主要探讨词根词缀的形态特征与派生词词形(顾名思义)的关系。
\subsection{词根的形态特征}
在分析词根的形态特征时,要注意词根的音节数和词根结合端字母(辅音或元音)。常考词根都是单音节或双音节的,其中单音节占比显著高于双音节。从独立性来看,自由词根多是本族语的,粘附词根多是古典语的。

单音节词根大部分为“辅元辅”,即“辅音字母(或其组合)+ 元音字母(可能会包括r音节)+ 辅音字母(或其组合)”。如man, vis, log等,元音字母包括r的例子有work, child, fect, phon, therm等少部分为“元辅”,即“元音字母(可能会包括r音节) + 辅音字母(或其组合)”的,如act, ann, erg, un;极少部分为“辅元”,即“辅音字母(或其组合)+ 元音字母(可能会包括r音节)”的,如bi, ge, sci等。

双音节词根往往是在单音节词根的基础上在头尾加上元音字母和r,也就是说双音节词根往往以元音开头或收尾。但是不包括单音节词跟加上连接字母构成的“伪双音节词根”。如anim, imit, ocul, labor, liter, mater等。
\subsection{词根的形态变化}
在构词过程中,往往出于读者需要或习惯原因,使得派生词的词形并不简单的等于各构词成分形态的总和,往往会发生字母的增加,脱落,变更,以及连字符的使用现象。

一、字母的增加

(1)重读闭音节的词根在添加后缀时,尾部辅音字母发生双写,如:
hap→happen

run→runner

prefer→preferred

refer + -al = referral n. 对相关者的分配,安排

但是在构词过程中重音发生移动的不需要双写,如:

re’fer + -ee = refer’ee n.仲裁人

pre’fer + -able = ’preferable adj.更可取的

(2)当古典语词根与后缀或另一词根构成一级派生词时,可能增加连接字母。连接字母一般是元音字母,往往出现在接合部两边都是辅音字母的情形,但也有反例。如:

helic(旋转)+ o + pter(翼)= helicopter

cert(确定)+ i + -fy = certify v.证实

anthrop(人)+ o + -logy(研究)= anthropology

sent(感觉)+ i + -ment(表结果)= sentiment n.情感

bi(生命)+ o + -logy(研究)= biology

ge-(地球,土地)+ o + graphy(记录)= geography

(3)为保持某些位于词尾的重读词根中元音字母读长音,在该词根后添加无语义的默音字母e,使之为开音节。
 in(内)+ clud(关)+ e = include;
 
 ante(前)+ ced(走)+ e = antecede;
 
 pro- + mot(移动)= promote;

二、字母的脱落(主要是词素尾字母的脱落)

(1)在二级派生情况下,在添加后缀时,某些词根的默音字母e脱落,以保证词根原来的音值。

\begin{tabular}{@{}lll@{}}
write + er = writer&create + -ive = creative&persevere + -ance = perseverance\\
use + -able = usable&promote + -ion = promotion
\end{tabular}

(2)有些以辅音字母结尾的前缀与以相同的辅音字母开头的词根结合时,前缀尾字母脱落。最常见的是s。

trans-(穿越)+ spir(呼吸)+ e = transpire v.蒸发

trans- + ship(装运)= tranship v.转运(或transship)

(3)前缀尾字母和词根首字母均为元音,结合时前缀尾字母脱落。

    anti-(相反,对抗)+ arctic(北极)= Antarctic adj.南极的

(4)前缀尾字母为s, x等,在以s, l等辅音字母开头的词根前, 结合时前缀尾字母脱落

ex- (出)+ lect(选择)= elect 

