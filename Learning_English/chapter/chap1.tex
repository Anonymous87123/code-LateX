\chapter{词素}
\section{引入词素、词根词缀和“基于词素的构词理论”}
基于词素的构词理论像分子论原子论一样,更能深刻地揭示英语单词的本质。

在英语中,绝大部分单词都是由词素组成,词素是最低一级的语法单位,词素按照其独立性可以分为两种:可以单独作为词使用的是自由词素(Free Morpheme),如英语中的friend、white、wash、you等;不能单独使用,即必须与其他词素结合在一起才能使用的(看起来像粘在单词上的词素)叫黏附词素(Bound Morpheme),如英语中的vis、manu、lect、-ible、-ly、-er、-un等。

同时,词素还可分为词根和词缀。英语单词往往由词根和词缀构成。它们都含有该词汇的各种信息,我们在通过解析词根词缀分析单词的构成及其语义时,要注意以词根的语义为主,以词缀的语义为次。词根分为单词词根和非单词词根,而词缀一般都长得不像一个单词。注意单词词根等价于自由词素,非单词词根和词缀一起构成黏附词素。词根和词缀是基本的构词成分,它们就像汉字的偏旁部首,例如,一个字可以拆分出一个小字和偏旁部首,比如“字”可以拆一个“子”出来,这可以类比自由词素或者单词词根,而“宀”则可以类比词缀。

综上,词素分为自由词素和黏附词素,前者等价于单词词根,后者包含非单词词根和词缀。

\section{词根的起源}
从语源上来说,语源词根可分为本族语词根和外来语词根(其中大部分是古典语词根)。

英语起源于盎格鲁---撒克逊语,是日耳曼语的一支,如今源自日耳曼语的词汇(以及词根词缀)被称为“日耳曼语源”,由于法语和拉丁语的涌入(中世纪后期和文艺复兴时期),这些来自拉丁语和希腊语的词汇(或改造来的词根词缀)被称为“古典语源”,这两种语源的占比大概九成。另外还有凯尔特语源,混合语源,东方美洲非洲语源等。日耳曼语源的单词一般很简单,意思浅近,是小学初中会大量接触到的词,如we、our、a、the、in、to、for、of、and、do、more、this等,这些单词转化出的词根叫本族语词根。而古典语源单词中含有大量但有限的粘附词根(古典语词根),它们意思较为单一,而且多词同根的情况非常普遍,因此如果能掌握古典语词根,就能有效提升词汇量。

只有了解词根的起源以及转化,我们才能更好的从单词中识别并拆出词根。

\section{英语从外来语中引入词根}
拉丁语词在变换词性时会经常变化词尾。而英语在吸收拉丁语词汇时,往往舍弃其复杂的词尾变化形式,保留稳定的词干部分,按照自己的构词规则造新词,常见的方式有两种:

(1)直接引入外来语词(整个单词),然后改造头尾,中间词干不变,或者直接不改造。如amo(爱)可以在amatory(恋爱的),amicable(友好的),amorous(色情的)和enamor(使迷恋的)见到;又如insul(岛)可以在insular(岛屿的),insulate(使孤立),peninsula(半岛)中见到。
(2)引入外来词干,再加入粘附词根或词缀,如insul加-in变成insulin(胰岛素)

英语从希腊语中吸收名词性的词根或其同根词。例如希腊名词logos(言)、demos(人民)的词干是log和dem,可以在logic、eulogy(颂词),democracy,epidemic中辨认出它们。

\section{词缀}
词缀和词根一样,有本族和外来的区别。 定义零派生词缀就是不加东西,即原封不动。
本族语词缀可以分为两类,一类是派生词缀,可以加在单词上构成派生词,如un-、-er、-ly、-ing等;一类是屈折词缀,如-ing表示进行,-ed表示完成,加s表示复数等。它们的共同特点是可以与自由词根(单词词素)直接结合。外来语词缀被《英语词根与单词的说文解字》命名为原生词缀,它们经过改造,多数被同化为可以与单词结合的派生词缀,有些甚至被赋予了新的含义,如原生词缀ex-再exceed(超出)中的意思是out(出),被同化后变成ex-(前)如ex-president;又如原生词缀pro-=向前(forward),比如progress,而同化后变为in favor of(亲),比如pro-Germany;再如re原本是“回”back的意思,如regress退回,改造后为again“再”的意思,如rewrite重写。
原生词缀按照常见的情况分为4类:

1.	介词或副词性的词缀,主要连接动词性词根,表示动作的时间,方式,方向等。

(1)	a-, ab-, abs-=from从,离;如ab-+solv(to loose)+e=absolve解除

(2)	ad-, ac-, af-, ag- ,al-, an-, ap-, ar-, as-, at-=towards向;如af+fect(to make)=affect影响

(3)	ante-=before在……之前;如ante-+ced(to go)+-e=antecede原先,先前

(4)	anti-, ant-=against相对,相反;如anti-+path(to feel)+y=antipathy反感

(5) circum-=around围绕;circum-+locut(to speak)+-ion(名词性后缀)=circumlocution委婉话

(6)	com-, co-, col-, con-, cor-=together with与,共;如con-+nect(to join)=connect连接,联系

(7)	de-=off, from离开,从,由;如de-+duc(to lead)+-e=deduce推论

2.	修饰,限定性原生词缀。多表示否定,数量,状态。

3.	名词性原生词缀,多表性质,状态,特征,等于“the state, act, condition or quality of(doing or being)”

4.	动词性原生词缀,连接名词性或形容词性的词根上使之动词化,主要是-ate(to make), -fy(to have), -ish(to cause to be)

(1)	anim-(life)+-ate=animate(to make or put life into)使活跃

(2)	un-(one)+i+-fy=unify使统一

(3)	ad-+mon(to remind)j+-ish=admonish(to cause to be reminded of a fault)

