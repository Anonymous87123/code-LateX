\chapter{LLMs 激活函数篇}

\section{FFN 块基础}

\subsection{FFN 块计算公式}
前馈神经网络(FFN)块的计算公式如下:
$$FFN(x)=f(x W_{1}+b_{1})W_{2}+b_{2}$$

\section{常见激活函数}

GeLU 激活函数:GeLU(高斯误差线性单元)激活函数的计算公式如下
$$ GeLU(x)\approx 0.5 x\left(1+\tanh\left(\sqrt{\frac{2}{\pi}}\left(x+0.044715 x^3\right)\right)\right)$$

Swish 激活函数:Swish 激活函数的计算公式如下
$$Swish_{\beta}(x)=x\cdot\sigma(\beta x)$$

\section{GLU 线性门控单元}

\subsection{什么是GLU?}

门控线性单元(Gated Linear Unit, GLU)是一种神经网络激活机制,由Dauphin等人在2017年提出。它的核心思想是使用一个\textbf{门控信号}来控制信息的流动,类似于LSTM中的门控机制,但应用于前馈网络。

\subsubsection{工作原理}
\begin{itemize}
\item \textbf{输入} $x$ 经过两个不同的线性变换:$xW$ 和 $xV$
\item 其中一个变换经过sigmoid函数,产生0到1之间的门控值
\item 另一个变换与门控值进行元素级相乘(Hadamard积,记为$\otimes$)
\item 门控值决定哪些信息应该被保留,哪些应该被抑制
\end{itemize}

\subsection{基础GLU计算公式}

传统的Transformer前馈网络(FFN)通常使用两层线性变换加激活函数:
$$ FFN_{ReLU}(x) = \text{ReLU}(xW_1)W_2 $$

而使用GLU的FFN块计算公式为:
\begin{align*}
GU(x) &= \sigma(x W + b) \otimes (x V + c) \\
FFN_{GLU}(x) &= \left(\sigma(x W_1) \otimes (x V)\right) W_2
\end{align*}

其中:
\begin{itemize}
\item $\sigma$ 是sigmoid函数:$\sigma(z) = \frac{1}{1+e^{-z}}$
\item $\otimes$ 表示元素级乘法(逐元素相乘)
\item $W, V, W_1, W_2$ 是可训练的权重矩阵
\item 通常省略偏置项$b, c$以简化表示
\end{itemize}

\subsubsection{GLU的作用机理}
\begin{figure}[htbp]
\centering
\begin{tikzpicture}[
    node distance=1.5cm,
    input/.style={rectangle, draw=blue!50, fill=blue!5, thick, minimum size=8mm},
    transform/.style={rectangle, draw=green!50, fill=green!5, thick, minimum size=8mm},
    gate/.style={ellipse, draw=red!50, fill=red!5, thick, minimum size=8mm},
    output/.style={rectangle, draw=orange!50, fill=orange!5, thick, minimum size=8mm},
    arrow/.style={-stealth, thick}
]

\node[input] (x) {$x$};
\node[transform, above right=0.5cm and 1.5cm of x] (wx) {$xW$};
\node[transform, below right=0.5cm and 1.5cm of x] (vx) {$xV$};
\node[gate, right=1.5cm of wx] (sigma) {$\sigma$};
\node[output, right=2.5cm of x] (mul) {$\otimes$};
\node[output, right=1.5cm of mul] (result) {$GU(x)$};

\draw[arrow] (x) -- ++(0:1) |- (wx);
\draw[arrow] (x) -- ++(0:1) |- (vx);
\draw[arrow] (wx) -- (sigma);
\draw[arrow] (sigma) -- node[above] {门控信号} (mul);
\draw[arrow] (vx) -- node[below] {原始信息} (mul);
\draw[arrow] (mul) -- (result);

\node[below=0.2cm of vx] {\small 线性变换$V$};
\node[above=0.2cm of wx] {\small 线性变换$W$};
\end{tikzpicture}
\caption{GLU计算流程图:通过门控机制控制信息流动}
\end{figure}

\subsection{GeGLU:基于GELU的GLU变体}

GeGLU将sigmoid函数替换为GELU(Gaussian Error Linear Unit)激活函数:

$$ \text{GeGLU}(x) = \text{GELU}(xW) \otimes (xV) $$

其中GELU函数的定义为:
$$ \text{GELU}(x) = x \cdot \Phi(x) = x \cdot \frac{1}{2}\left[1 + \text{erf}\left(\frac{x}{\sqrt{2}}\right)\right] $$

\subsubsection{GELU函数特点}
\begin{itemize}
\item \textbf{平滑性}:GELU处处可导,比ReLU更平滑
\item \textbf{概率解释}:基于高斯分布的累积分布函数
\item \textbf{性能优势}:在Transformer中通常比ReLU表现更好
\item \textbf{计算复杂度}:需要计算误差函数erf,但现代硬件有优化实现
\end{itemize}

\subsection{SwiGLU:基于Swish的GLU变体}

SwiGLU使用Swish激活函数(也称为SiLU)替换sigmoid:

$$ \text{SwiGLU}(x) = \text{Swish}_\beta(xW) \otimes (xV) $$

其中Swish函数定义为:
$$ \text{Swish}_\beta(x) = x \cdot \sigma(\beta x) $$

当$\beta=1$时,就是标准的Swish/SiLU函数:
$$ \text{SiLU}(x) = x \cdot \sigma(x) = \frac{x}{1 + e^{-x}} $$

\subsubsection{Swish函数特点}
\begin{itemize}
\item \textbf{自门控}:Swish本身就是$x$与sigmoid的乘积,具有自门控特性
\item \textbf{无界性}:输出可以大于1,这有助于避免梯度消失
\item \textbf{平滑性}:处处连续可导
\item \textbf{实证优势}:在多个任务上优于ReLU和GELU
\end{itemize}

\subsection{参数规模分析}

\subsubsection{传统FFN的参数规模}
传统的Transformer FFN层包含两个线性变换:
\begin{align*}
\text{中间表示} &= \text{ReLU}(xW_1) \\
\text{输出} &= (\text{中间表示})W_2
\end{align*}

其中:
\begin{itemize}
\item 输入维度:$d_{\text{model}} = h$
\item 中间维度:通常为 $4h$(放大4倍)
\item $W_1$ 形状:$h \times 4h$,参数量:$4h^2$
\item $W_2$ 形状:$4h \times h$,参数量:$4h^2$
\item 总参数量:$8h^2$
\end{itemize}

\subsubsection{SwiGLU FFN的参数规模}
SwiGLU FFN包含三个线性变换:
\begin{align*}
\text{SwiGLU}(x) &= \text{Swish}(xW) \otimes (xV) \\
\text{输出} &= (\text{SwiGLU}(x))U
\end{align*}

为了保持与原始FFN相似的参数总量,需要调整中间维度:
\begin{itemize}
\item 设原始中间维度为 $4h$,有2个矩阵
\item SwiGLU有3个矩阵,为保持总参数量相近,每个矩阵应更小
\item 设SwiGLU中间维度为 $d_{\text{ffn}}$
\item 参数总量:$h \times d_{\text{ffn}} \times 3 = 3h \cdot d_{\text{ffn}}$
\item 令其等于传统FFN参数量:$3h \cdot d_{\text{ffn}} \approx 8h^2$
\item 解得:$d_{\text{ffn}} \approx \frac{8}{3}h = 2.666...h$
\item 实际中取 $\frac{8}{3}h$ 的近似值
\end{itemize}

\subsubsection{维度计算示例}
以LLaMA-7B为例,$h = 4096$:

传统FFN中间维度:
$$ 4h = 4 \times 4096 = 16384 $$

SwiGLU的中间维度计算:
\begin{align*}
\frac{2}{3} \times 4h &= \frac{2}{3} \times 16384 = 10922.\overline{6} \\
\text{取整到128的倍数} & \rightarrow 11008
\end{align*}

为什么是$\frac{2}{3}$?因为:
\begin{itemize}
\item 传统FFN:2个矩阵,每个维度$h \times 4h$
\item 总参数:$2 \times (h \times 4h) = 8h^2$
\item SwiGLU FFN:3个矩阵,每个维度$h \times d$
\item 总参数:$3 \times (h \times d) = 3hd$
\item 令$3hd = 8h^2$,则$d = \frac{8}{3}h = \frac{2}{3} \times 4h$
\end{itemize}

所以SwiGLU的中间维度是传统FFN中间维度$4h$的$\frac{2}{3}$倍。

\subsection{为什么GLU有效?}

\subsubsection{理论解释}
\begin{enumerate}
\item \textbf{梯度流改善}:门控机制提供了更丰富的梯度信号
\item \textbf{信息控制}:允许网络学习哪些信息应该被传递
\item \textbf{表达能力强}:GLU可以看作是条件仿射变换
\item \textbf{避免梯度消失}:相比sigmoid/tanh,GLU的梯度更稳定
\end{enumerate}

\subsubsection{实证结果}
\begin{itemize}
\item 在多种自然语言处理任务中,GLU变体通常优于标准ReLU
\item SwiGLU在PaLM、LLaMA等大型模型中被广泛采用
\item 在相似的参数量下,SwiGLU通常能获得更好的性能
\end{itemize}

\subsection{不同GLU变体对比}

\begin{table}[h]
\centering
\caption{GLU变体对比}
\label{tab:glu_comparison}
\begin{tabular}{lllll}
\hline
\textbf{类型} & \textbf{激活函数} & \textbf{公式} & \textbf{特点} & \textbf{应用模型} \\ 
\textbf{GLU} & Sigmoid & $\sigma(xW) \otimes xV$ & 原始版本,门控值在[0,1] & 早期Transformer变体 \\ 
\textbf{GeGLU} & GELU & $\text{GELU}(xW) \otimes xV$ & 更平滑,概率解释 & T5、部分BERT变体 \\ 
\textbf{SwiGLU} & Swish/SiLU & $\text{Swish}(xW) \otimes xV$ & 自门控,无界输出 & LLaMA \\ 
\textbf{ReGLU} & ReLU & $\text{ReLU}(xW) \otimes xV$ & 简单,稀疏激活 & 部分研究模型\\ 
\end{tabular}
\end{table}


\section{各 LLMs 激活函数使用情况}

\begin{table}[h]
\centering
\caption{各 LLMs 模型使用的激活函数对比}
\begin{tabular}{@{}ll@{}}
\toprule
\textbf{模型} & \textbf{激活函数} \\
\midrule
GPT3 & GeLU \\
LLaMA & SwiGLU \\
LLaMA2 & SwiGLU \\
baichuan & SwiGLU \\
ChatGLM-6B & GeLU \\
ChatGLM2-6B & SwiGLU \\
Bloom & GeLU \\
Falcon & GeLU \\
\bottomrule
\end{tabular}
\end{table}