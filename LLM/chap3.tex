\chapter{LLMs 激活函数篇}

\section{FFN 块基础}

\subsection{FFN 块计算公式}
前馈神经网络(FFN)块的计算公式如下:
$$FFN(x)=f(x W_{1}+b_{1})W_{2}+b_{2}$$

\section{常见激活函数}

\subsection{GeLU 激活函数}
GeLU(高斯误差线性单元)激活函数的计算公式如下:
$$ GeLU(x)\approx 0.5 x\left(1+\tanh\left(\sqrt{\frac{2}{\pi}}\left(x+0.044715 x^3\right)\right)\right)$$

\subsection{Swish 激活函数}
Swish 激活函数的计算公式如下:
$$Swish_{\beta}(x)=x\cdot\sigma(\beta x)$$

\section{GLU 线性门控单元}

\subsection{基础 GLU 计算公式}
使用 GLU(门控线性单元)的 FFN 块计算公式:
\begin{align*}
GU(x)&=\sigma(x W+b)\otimes x V\\
FFN_{GLU}&=(f(x W_{1})\otimes x V) W_{2}
\end{align*}

\subsection{GeGLU 计算公式}
使用 GeLU 的 GLU 块计算公式:
$$GeGLU(x)=GeLU(xW)\otimes xV$$

\subsection{SwiGLU 计算公式}
使用 Swish 的 GLU 块计算公式:
$$SwiGLU=Swish_{\beta}(xW)\otimes xV$$

\subsection{参数规模说明}
\begin{itemize}
\item 传统 FFN:2个可训练权重矩阵,中间维度为 $4h$
\item SwiGLU FFN:3个可训练权重矩阵,中间维度为 $4h \times 2/3$
\end{itemize}

维度计算示例:
\begin{align*}
4h &= 4 \times 4096 = 16384 \\
\frac{2}{3} \times 4h &= 10022 \rightarrow 11008
\end{align*}

\section{各 LLMs 激活函数使用情况}

\begin{table}[h]
\centering
\caption{各 LLMs 模型使用的激活函数对比}
\begin{tabular}{@{}ll@{}}
\toprule
\textbf{模型} & \textbf{激活函数} \\
\midrule
GPT3 & GeLU \\
LLaMA & SwiGLU \\
LLaMA2 & SwiGLU \\
baichuan & SwiGLU \\
ChatGLM-6B & GeLU \\
ChatGLM2-6B & SwiGLU \\
Bloom & GeLU \\
Falcon & GeLU \\
\bottomrule
\end{tabular}
\end{table}

\section{模型参数结构示例}

\subsection{LLaMA 模型参数结构}
\begin{table}[h]
\centering
\caption{LLaMA 模型参数结构示例}
\begin{tabular}{@{}llll@{}}
\toprule
\makecell[l]{模块类型} & \makecell[l]{模块名称} & \makecell[l]{参数形状} & \makecell[l]{参数数量} \\
\midrule
\multirow{9}{*}{Embedding \& Layers} 
& model.embed\_tokens.weight & [32000, 4096] & 131072000 \\
& model.layers.0.self\_attn.q\_proj.weight & [4096, 4096] & 16777216 \\
& model.layers.0.self\_attn.k\_proj.weight & [4096, 4096] & 16777216 \\
& model.layers.0.self\_attn.v\_proj.weight & [4096, 4096] & 16777216 \\
& model.layers.0.self\_attn.o\_proj.weight & [4096, 4096] & 16777216 \\
& model.layers.0.mlp.gate\_proj.weight & [11008, 4096] & 45088768 \\
& model.layers.0.mlp.down\_proj.weight & [4096, 11008] & 45088768 \\
& model.layers.0.mlp.up\_proj.weight & [11008, 4096] & 45088768 \\
& model.layers.0.input\_layernorm.weight & [4096] & 4096 \\
& model.layers.0.post\_attention\_layernorm.weight & [4096] & 4096 \\
\bottomrule
\end{tabular}
\end{table}

\subsection{Bloom 模型参数结构}
\begin{table}[h]
\centering
\caption{Bloom 模型参数结构示例}
\begin{tabular}{@{}llll@{}}
\toprule
\makecell[l]{模块类型} & \makecell[l]{模块名称} & \makecell[l]{参数形状} & \makecell[l]{参数数量} \\
\midrule
\multirow{12}{*}{Embedding} 
& transformer.word\_embeddings.weight & [250880, 4096] & 1027604480 \\
& transformer.word\_embeddings\_layernorm.weight & [4096] & 4096 \\
& transformer.word\_embeddings\_layernorm.bias & [4096] & 4096 \\
& transformer.h.0.input\_layernorm.weight & [4096] & 4096 \\
& transformer.h.0.input\_layernorm.bias & [4096] & 4096 \\
& transformer.h.0.self\_attention.query\_key\_value.weight & [12288, 4096] & 50331648 \\
& transformer.h.0.self\_attention.query\_key\_value.bias & [12288] & 12288 \\
& transformer.h.0.self\_attention.dense.weight & [4096, 4096] & 16777216 \\
& transformer.h.0.self\_attention.dense.bias & [4096] & 4096 \\
& transformer.h.0.post\_attention\_layernorm.weight & [4096] & 4096 \\
& transformer.h.0.post\_attention\_layernorm.bias & [4096] & 4096 \\
& transformer.h.0.mlp.dense\_h\_to\_4h.weight & [16384, 4096] & 67108864 \\
& transformer.h.0.mlp.dense\_h\_to\_4h.bias & [16384] & 16384 \\
& transformer.h.0.mlp.dense\_4h\_to\_h.weight & [4096, 16384] & 67108864 \\
& transformer.h.0.mlp.dense\_4h\_to\_h.bias & [4096] & 4096 \\
\bottomrule
\end{tabular}
\end{table}
\subsection{特殊说明}
BLOOM 在 embedding 层后添加 layer normalization,有利于提升训练稳定性,但可能会带来很大的性能损失。