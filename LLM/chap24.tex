\chapter{PEFT库中LoRA使用详解}

\section{前言}

本文主要介绍使用PEFT库中的LoRA模块对大模型进行高效参数微调,涉及以下内容:

\begin{itemize}
\item PEFT库中LoRA模块的使用方法
\item PEFT库中LoRA模块的代码实现原理
\item 推理时权重合并与模型加载策略
\end{itemize}

\subsection{环境依赖配置}
\begin{lstlisting}
# 以下配置可能会随时间变化,建议根据实际情况调整
accelerate
appdirs
loralib
bitsandbytes
black
black[jupyter]
datasets
fire
transformers>=4.28.0
git+https://github.com/huggingface/peft.git
sentencepiece
gradio
wandb
cpm-kernel
\end{lstlisting}

\section{LoraConfig配置详解}

\subsection{基本配置示例}
\begin{lstlisting}[language=Python]
# 设置超参数及配置
LORA_R = 8
LORA_ALPHA = 16
LORA_DROPOUT = 0.05
TARGET_MODULES = [
    "q_proj",
    "v_proj",
]

config = LoraConfig(
    r=LORA_R,
    lora_alpha=LORA_ALPHA,
    target_modules=TARGET_MODULES,
    lora_dropout=LORA_DROPOUT,
    bias="none",
    task_type="CAUSAL_LM",
)
\end{lstlisting}

\subsection{参数说明}
\begin{itemize}
\item \textbf{r}:LoRA的秩,矩阵A和矩阵B相连的宽度,满足$r \ll d$
\item \textbf{lora\_alpha}:归一化超参数,LoRA参数$\Delta Wx$被以$\alpha/r$归一化
\item \textbf{target\_modules}:LoRA的目标应用位置
\item \textbf{merge\_weights}:eval模式中是否将LoRA矩阵值加到原有$W_0$上
\item \textbf{lora\_dropout}:LoRA层的dropout比率
\item \textbf{fan\_in\_fan\_out}:仅应用于Conv1D层时设为True
\item \textbf{bias}:偏置训练设置(none:均不训练;all:全训练;lora\_only:仅训练LoRA部分偏置)
\item \textbf{task\_type}:任务类型设置
\item \textbf{modules\_to\_save}:除LoRA外其他可训练并需要保存的层
\end{itemize}

注意:target\_modules中的目标名称在不同模型中可能不同,如ChatGLM中使用query\_key\_value。

\section{模型加入PEFT策略}

\subsection{模型加载策略}
模型加载涉及两个重要的显存优化技巧:load\_in\_8bit和prepare\_model\_for\_int8\_training。

\subsection{模型显存占用分析}
\begin{itemize}
\item \textbf{静态显存}:由模型参数量级决定
\item \textbf{动态显存}:前向传播过程中每个样本的每个神经元计算的激活值存储,用于反向传播的梯度计算
\end{itemize}

\subsection{显存优化策略}

\subsubsection{8bit量化优化}
from\_pretrained中的load\_in\_8bit参数由bitsandbytes库提供,将加载的模型转换为混合8bit量化模型。

\begin{lstlisting}[language=Python]
# 8bit量化示例
model = AutoModel.from_pretrained(
    "THUDM/chatglm3-6b", 
    load_in_8bit=True,
    device_map='auto'
)
\end{lstlisting}

量化原理:将FP32(4字节)压缩到INT8(1字节),实现1/4的显存占用。采用absolute-maximum或zero-point量化方案,其中absmax方案的基本原理:

\begin{align*}
\text{FP16 vector} &: [1.2, 0.5, -4.3, 1.2, -3.1, 0.8, 2.4, 5.4] \\
\text{缩放因子} &= \frac{127}{\max(|\text{vector}|)} \\
\text{量化结果} &= \text{round}\left(\text{vector} \times \text{缩放因子}\right)
\end{align*}

LLM.int8()实现对outlier进行优化,将outlier和非outlier矩阵分开计算再合并,降低精度损失。

prepare\_model\_for\_int8\_training函数适配LLM.int8()以提高训练稳定性,主要包括:
\begin{itemize}
\item Layer norm层保持FP32精度
\item 输出层保持FP32精度保证解码时随机采样的差异性
\end{itemize}

\subsubsection{梯度检查优化}
设置gradient\_checkpointing=True实现时间换空间的优化:
\begin{itemize}
\item 前向传播使用torch.no\_grad()不存储中间激活值
\item 反向传播时重新计算激活值用于梯度计算
\item 前向传播计算两遍,增加训练时间但减少显存占用
\end{itemize}

\subsection{PEFT策略集成}
\begin{lstlisting}[language=Python]
# 加入PEFT策略
model = get_peft_model(model, config)
model = model.to(device)
model.config.use_cache = False
\end{lstlisting}

注意:use\_cache设置为False是因为与gradient checkpoint存在冲突。use\_cache优化解码速度,在解码时存储每一步的hidden-state用于下一步输入,而gradient checkpoint不存储中间激活值。

\section{PEFT库中LoRA模块代码实现}

\subsection{整体架构设计}
PEFT库中,peft\_model.py中的PeftModel类是总控类,继承transformers中的Mixin类,负责模型读取保存等功能。

\begin{lstlisting}[language=Python]
class LoraModel(torch.nn.Module):
    def __init__(self, config, model):
        super().__init__()
        self.peft_config = config
        self.model = model
        self._find_and_replace()
        mark_only_lora_as_trainable(self.model, self.peft_config.bias)
        self.forward = self.model.forward
\end{lstlisting}

构造方法主要完成两个步骤:
\begin{enumerate}
\item \_find\_and\_replace():找到需要加入LoRA策略的层并替换
\item mark\_only\_lora\_as\_trainable():固定非LoRA参数,仅训练LoRA部分
\end{enumerate}

\subsection{\_find\_and\_replace()实现}
\begin{lstlisting}[language=Python]
def _find_and_replace(self):
    # 1. 找到目标层
    target_module_found = re.fullmatch(self.peft_config.target_modules, key)
    
    # 2. 创建新的LoRA层
    new_module = Linear(target.in_features, target.out_features, bias=bias, **kwargs)
    
    # 3. 替换原层
    self._replace_module(parent, target_name, new_module, target)
\end{lstlisting}

替换方法实现:
\begin{lstlisting}[language=Python]
def _replace_module(self, parent_module, child_name, new_module, old_module):
    setattr(parent_module, child_name, new_module)
    new_module.weight = old_module.weight
    if old_module.bias is not None:
        new_module.bias = old_module.bias
    if getattr(old_module, "state", None) is not None:
        new_module.state = old_module.state
    new_module.to(old_module.weight.device)
    
    # 分配到正确设备
    for name, module in new_module.named_modules():
        if "lora" in name:
            module.to(old_module.weight.device)
\end{lstlisting}

\subsection{LoRA层实现细节}

\subsubsection{基类LoraLayer}
\begin{lstlisting}[language=Python]
class LoraLayer:
    def __init__(
        self,
        r: int,
        lora_alpha: int,
        lora_dropout: float,
        merge_weights: bool,
    ):
        self.r = r
        self.lora_alpha = lora_alpha
        # 可选的dropout
        if lora_dropout > 0.0:
            self.lora_dropout = nn.Dropout(p=lora_dropout)
        else:
            self.lora_dropout = lambda x: x
        # 标记权重未合并
        self.merged = False
        self.merge_weights = merge_weights
        self.disable_adapters = False
\end{lstlisting}

\subsubsection{Linear层实现}
Linear类同时继承nn.Linear和LoraLayer:

\begin{lstlisting}[language=Python]
class Linear(nn.Linear, LoraLayer):
    # LoRA在dense层的实现
    def __init__(
        self,
        in_features: int,
        out_features: int,
        r: int = 0,
        lora_alpha: int = 1,
        lora_dropout: float = 0.0,
        fan_in_fan_out: bool = False,  # 替换层存储权重如(fan_in, fan_out)时设为True
        **kwargs,
    ):
        nn.Linear.__init__(self, in_features, out_features, **kwargs)
        LoraLayer.__init__(self, r=r, lora_alpha=lora_alpha,
                          lora_dropout=lora_dropout, merge_weights=merge_weights)
        self.fan_in_fan_out = fan_in_fan_out
        
        if self.r > 0:
            self.weight.data -= (
                transpose(self.lora_B.weight @ self.lora_A.weight,
                         self.fan_in_fan_out) * self.scaling
            )
            self.merged = False
\end{lstlisting}

前向传播实现:
\begin{lstlisting}[language=Python]
def forward(self, x: torch.Tensor):
    if self.disable_adapters:
        if self.r > 0 and self.merged:
            self.weight.data -= (
                transpose(self.lora_B.weight @ self.lora_A.weight,
                         self.fan_in_fan_out) * self.scaling
            )
            self.merged = False
        return F.linear(x, transpose(self.weight, self.fan_in_fan_out), 
                       bias=self.bias)
    elif self.r > 0 and not self.merged:
        result = F.linear(x, transpose(self.weight, self.fan_in_fan_out),
                         bias=self.bias)
        if self.r > 0:
            result += self.lora_B(self.lora_A(self.lora_dropout(x))) * self.scaling
        return result
    else:
        return F.linear(x, transpose(self.weight, self.fan_in_fan_out),
                       bias=self.bias)
\end{lstlisting}

\section{LoRA微调存储策略}

\subsection{存储实现}
PeftModel重写了save\_pretrained函数,只存储LoRA层权重:

\begin{lstlisting}[language=Python]
# 重写Trainer的save_model,在checkpoint时只存LoRA权重
class ModifiedTrainer(Trainer):
    def save_model(self, output_dir=None, _internal_call=False):
        from transformers.trainer import TRAINING_ARGS_NAME
        os.makedirs(output_dir, exist_ok=True)
        torch.save(self.args, os.path.join(output_dir, TRAINING_ARGS_NAME))
        
        saved_params = {
            k: v.to("cpu") for k, v in self.model.named_parameters() 
            if v.requires_grad
        }
        torch.save(saved_params, os.path.join(output_dir, "adapter_model.bin"))

# 使用示例
trainer = ModifiedTrainer(
    model=model,
    train_dataset=train_data,
    args=transformers.TrainingArguments(
        per_device_train_batch_size=8,
        gradient_accumulation_steps=16,
        num_train_epochs=10,
        learning_rate=3e-4,
        fp16=True,
        logging_steps=10,
        save_steps=200,
        output_dir=output_dir
    ),
    data_collator=DataCollatorForSeq2Seq(
        tokenizer, pad_to_multiple_of=8, return_tensors="pt", padding=True
    ),
)
trainer.train()
model.save_pretrained(train_args.output_dir)
\end{lstlisting}

\section{LoRA推理加载策略}

\subsection{方案一:直接加载LoRA层}
\begin{lstlisting}[language=Python]
from peft import PeftModel
from transformers import AutoModel, AutoTokenizer

model = AutoModel.from_pretrained(
    "THUDM/chatglm3-6b", 
    trust_remote_code=True, 
    load_in_8bit=True, 
    device_map='auto'
)
tokenizer = AutoTokenizer.from_pretrained("THUDM/chatglm3-6b", 
                                         trust_remote_code=True)
model = PeftModel.from_pretrained(model, "./lora_ckpt")
model.half().to(device)
model.eval()
\end{lstlisting}
\textbf{缺点}:增加推理延迟,适合线下测评。

\subsection{方案二:权重合并后加载}
\begin{lstlisting}[language=Python]
tokenizer = AutoTokenizer.from_pretrained("THUDM/chatglm3-6b", 
                                         trust_remote_code=True)

# 合并时禁用int8
model = AutoModel.from_pretrained(
    "THUDM/chatglm3-6b", 
    load_in_8bit=False, 
    torch_dtype=torch.float16,
    trust_remote_code=True, 
    device_map={"": "cpu"},
)

# 检查权重是否合并成功
first_weight = model.base_model.layers[0].attention.query_key_value.weight
first_weight_old = first_weight.clone()

# 加载LoRA模型
lora_model = PeftModel.from_pretrained(
    model,
    "./lora_ckpt",
    device_map={"": "cpu"},
    torch_dtype=torch.float16,
)

# 合并权重
lora_model = lora_model.merge_and_unload()
lora_model.train(False)

# 验证合并
assert not torch.allclose(first_weight_old, first_weight), 
       'Weight Should Changes after Lora Merge'

# 恢复原始key格式
deloreanized_sd = {
    k.replace("base_model.model.", ""): v
    for k, v in lora_model.state_dict().items()
    if "lora" not in k
}

# 保存合并后的权重
lora_model.save_pretrained(output_dir, state_dict=deloreanized_sd)
\end{lstlisting}

\section{多LoRA适配器切换}

\subsection{环境要求}
\begin{lstlisting}
peft >= 0.3.0
\end{lstlisting}

\subsection{使用方法}
\begin{enumerate}
\item \textbf{加载第一个适配器}:指定adapter\_name参数
\begin{lstlisting}[language=Python]
model = PeftModel.from_pretrained(model, "tloen/alpaca-lora-7b", 
                                 adapter_name="eng_alpaca")
\end{lstlisting}

\item \textbf{加载其他适配器}:使用load\_adapter方法
\begin{lstlisting}[language=Python]
model.load_adapter(peft_model_path, adapter_name)
\end{lstlisting}

\item \textbf{切换适配器}:使用set\_adapter方法
\begin{lstlisting}[language=Python]
model.set_adapter(adapter_name)
\end{lstlisting}

\item \textbf{禁用适配器}:使用disable\_adapter上下文管理器
\begin{lstlisting}[language=Python]
with model.disable_adapter():
    # 在此代码块内禁用适配器
\end{lstlisting}

\item \textbf{合并卸载适配器}:使用merge\_and\_unload方法
\begin{lstlisting}[language=Python]
model = model.merge_and_unload()
\end{lstlisting}
\end{enumerate}

\subsection{实战案例}
\begin{lstlisting}[language=Python]
from peft import PeftModel
from transformers import LlamaTokenizer, LlamaForCausalLM, GenerationConfig

model_name = "decapoda-research/llama-7b-hf"
tokenizer = LlamaTokenizer.from_pretrained(model_name)
model = LlamaForCausalLM.from_pretrained(
    model_name,
    load_in_8bit=True,
    device_map="auto",
    use_auth_token=True
)

# 加载多个适配器
model = PeftModel.from_pretrained(model, "tloen/alpaca-lora-7b",
                                adapter_name="eng_alpaca")
model.load_adapter("22h/cabrita-lora-v0-1", adapter_name="portuguese_alpaca")

# 切换适配器
model.set_adapter("eng_alpaca")
instruction = "Tell me about alpacas."
print(evaluate(instruction))

model.set_adapter("portuguese_alpaca")
instruction = "Invente uma desculpa criativa pra dizer que nao preciso ir a festa."
print(evaluate(instruction))

# 禁用适配器
with model.disable_adapter():
    instruction = "Invente uma desculpa criativa pra dizer que nao preciso ir festa."
    print(evaluate(instruction))
\end{lstlisting}

\section{总结}

本文详细介绍了PEFT库中LoRA模块的完整使用流程,从基础配置到高级功能,包括:

\begin{itemize}
\item \textbf{配置管理}:LoraConfig的参数详解和最佳实践
\item \textbf{显存优化}:8bit量化和梯度检查等关键技术
\item \textbf{代码实现}:PEFT库中LoRA的核心实现原理
\item \textbf{存储策略}:训练和推理时的权重管理方案
\item \textbf{多适配器}:支持多个LoRA适配器的动态加载和切换
\end{itemize}

通过这些技术,可以在资源受限的环境下高效地对大语言模型进行微调,并在推理时灵活地应用不同的适配器来适应多种任务场景。
