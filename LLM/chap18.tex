
\chapter{大模型(LLMs)RAG关键痛点及解决方案}

\section{前言}

受到Barnett等人的论文《Seven Failure Points When Engineering a Retrieval Augmented Generation System》的启发,本文将探讨论文中提到的七个痛点,以及在开发检索增强型生成(RAG)流程中常见的五个额外痛点。更为关键的是,我们将深入讨论这些RAG痛点的解决策略,使我们在日常RAG开发中能更好地应对这些挑战。

\section{问题一:内容缺失问题}

\subsection{内容缺失问题介绍}
当实际答案不在知识库中时,RAG系统往往给出一个貌似合理却错误的答案,而不是承认无法给出答案。这导致用户接收到误导性信息,造成错误的引导。

\subsection{内容缺失问题解决方案}
\begin{enumerate}
\item \textbf{优化数据源}:"输入什么,输出什么。"如果源数据质量差,比如充斥着冲突信息,那么无论如何构建RAG流程,都不可能从杂乱无章的数据中得到有价值的结果。

\item \textbf{改进提示方式}:在知识库缺乏信息,系统可能给出错误答案的情况下,改进提示方式可以起到显著帮助。例如,通过设置提示"如果你无法确定答案,请表明你不知道"可以鼓励模型认识到自己的局限并更透明地表达不确定性。虽然无法保证百分百准确,但在优化数据源之后,改进提示方式是我们能做的最好努力之一。
\end{enumerate}

\section{问题二:错过排名靠前的文档}

\subsection{错过排名靠前的文档问题介绍}
有时候系统在检索资料时,最关键的文件可能并没有出现在返回结果的最前面。这就导致了正确答案被忽略,系统因此无法给出精准的回答。即:"问题的答案其实在某个文档里面,只是它没有获得足够高的排名以致于没能呈现给用户"。

\subsection{错过排名靠前的文档问题解决方案}
\begin{enumerate}
\item \textbf{重新排名检索结果}:在将检索到的结果发送给大型语言模型(LLM)之前,对结果进行重新排名可以显著提升RAG的性能。

\item \textbf{调整超参数}:chunk\_size和similarity\_top\_k都是用来调控RAG模型数据检索过程中效率和效果的参数。改动这些参数能够影响计算效率与信息检索质量之间的平衡。

\begin{lstlisting}[language=Python]
param_tuner = ParamTuner(
    param_fn=objective_function_semantic_similarity,
    param_dict=param_dict,
    fixed_param_dict=fixed_param_dict,
    show_progress=True,
)

# 包含需要调优的参数
param_dict = {"chunk_size": [256, 512, 1024], "top_k": [1, 2, 5]}

# 包含在调整过程的所有运行中保持固定的参数
fixed_param_dict = {
    "docs": documents,
    "eval_qs": eval_qs,
    "ref_response_strs": ref_response_strs,
}

def objective_function_semantic_similarity(params_dict):
    chunk_size = params_dict["chunk_size"]
    docs = params_dict["docs"]
    top_k = params_dict["top_k"]
    eval_qs = params_dict["eval_qs"]
    ref_response_strs = params_dict["ref_response_strs"]
    
    # 建立索引
    index = build_index(chunk_size, docs)
    # 查询引擎
    query_engine = index.as_query_engine(similarity_top_k=top_k)
    # 获得预测响应
    pred_response_objs = get_responses(eval_qs, query_engine, show_progress=True)
    # 运行评估程序
    eval_batch_runner = _get_eval_batch_runner_semantic_similarity()
    eval_results = eval_batch_runner.evaluate_responses(
        eval_qs, responses=pred_response_objs, reference=ref_response_strs
    )
    # 获取语义相似度度量
    mean_score = np.array(
        [r.score for r in eval_results["semantic_similarity"]]
    ).mean()
    return RunResult(score=mean_score, params=params_dict)
\end{lstlisting}
\end{enumerate}

\section{问题三:脱离上下文—整合策略的限制}

\subsection{脱离上下文问题介绍}
论文中提到了这样一个问题:"虽然数据库检索到了含有答案的文档,但这些文档并没有被用来生成答案。这种情况往往出现在数据库返回大量文档后,需要通过一个整合过程来找出答案"。

\subsection{脱离上下文问题解决方案}
\begin{enumerate}
\item \textbf{优化检索策略}:提供从基础到高级的检索策略,包括从每个索引进行基础检索、高级检索和搜索、自动检索、知识图谱检索器、组合/分层检索器等。

\item \textbf{微调嵌入模型}:如果使用开源嵌入模型,对其进行微调是提高检索准确性的有效方法。

\begin{lstlisting}[language=Python]
finetune_engine = SentenceTransformersFinetuneEngine(
    train_dataset,
    model_id="BAAI/bge-small-en",
    model_output_path="test_model",
    val_dataset=val_dataset,
)

finetune_engine.finetune()
embed_model = finetune_engine.get_finetuned_model()
\end{lstlisting}
\end{enumerate}

\section{问题四:未能提取答案}

\subsection{未能提取答案问题介绍}
当系统需要从提供的上下文中提取正确答案时,尤其是在信息量巨大时,系统往往会遇到困难。关键信息被遗漏,从而影响了回答的质量。论文中提到:"这种情况通常是由于上下文中存在太多干扰信息或相互矛盾的信息"。

\subsection{未能提取答案问题解决方案}
\begin{enumerate}
\item \textbf{清理数据}:必须再次强调,干净整洁的数据至关重要!在质疑RAG流程之前,务必先要清理数据。

\item \textbf{提示压缩}:通过LongLLMLingua研究项目提出的提示压缩技术,在检索步骤之后压缩上下文,再将其输入大语言模型。

\begin{lstlisting}[language=Python]
from llama_index.query_engine import RetrieverQueryEngine
from llama_index.response_synthesizers import CompactAndRefine
from llama_index.postprocessor import LongLLMLinguaPostprocessor
from llama_index.schema import QueryBundle

node_postprocessor = LongLLMLinguaPostprocessor(
    instruction_str="鉴于上下文,请回答最后一个问题",
    target_token=300,
    rank_method="longllmlingua",
    additional_compress_kwargs={
        "condition_compare": True,
        "condition_in_question": "after",
        "context_budget": "+100",
        "reorder_context": "sort",  # 启用文档重新排序
    },
)

retrieved_nodes = retriever.retrieve(query_str)
synthesizer = CompactAndRefine()
# 处理(压缩)、合成
new_retrieved_nodes = node_postprocessor.postprocess_nodes(
    retrieved_nodes, query_bundle=QueryBundle(query_str=query_str)
)
print("\n\n".join([n.get_content() for n in new_retrieved_nodes]))
response = synthesizer.synthesize(query_str, new_retrieved_nodes)
\end{lstlisting}

\item \textbf{LongContextReorder}:研究发现当关键信息位于输入上下文的开始或结尾时,通常能得到最好的性能。LongContextReorder被设计用来重新排序检索到的节点,在需要大量top-k结果时特别有效。

\begin{lstlisting}[language=Python]
from llama_index.postprocessor import LongContextReorder
reorder = LongContextReorder()
reorder_engine = index.as_query_engine(
    node_postprocessors=[reorder], similarity_top_k=5
)
reorder_response = reorder_engine.query("作者见过山姆·奥尔特曼吗?")
\end{lstlisting}
\end{enumerate}

\section{问题五:格式错误}

\subsection{格式错误问题介绍}
当告诉计算机以某种特定格式(比如表格或清单)来整理信息,但大型语言模型(LLM)没能注意到时,就会出现格式错误。

\subsection{格式错误问题解决方案}
\begin{enumerate}
\item \textbf{更精准的提示}:让指令更加明确、简化问题并突出关键词、提供示例、循环提问不断细化问题。

\item \textbf{输出解析}:为任何查询提供格式化指南,对计算机的回答进行"解析"。支持与Guardrails和LangChain提供的输出解析模块集成。

\begin{lstlisting}[language=Python]
from llama_index import VectorStoreIndex, SimpleDirectoryReader
from llama_index.output_parsers import LangchainOutputParser
from llama_index.llms import OpenAI
from langchain.output_parsers import StructuredOutputParser, ResponseSchema

# 加载文档,构建索引
documents = SimpleDirectoryReader("../paul_graham_essay/data").load_data()
index = VectorStoreIndex.from_documents(documents)

# 定义输出模式
response_schemas = [
    ResponseSchema(
        name="Education",
        description="描述作者的教育经历/背景。",
    ),
    ResponseSchema(
        name="Work", 
        description="描述作者的工作经验/背景。",
    ),
]

# 定义输出解析器
lc_output_parser = StructuredOutputParser.from_response_schemas(response_schemas)
output_parser = LangchainOutputParser(lc_output_parser)

# 将输出解析器附加到LLM
llm = OpenAI(output_parser=output_parser)

# 获得结构化响应
from llama_index import ServiceContext
ctx = ServiceContext.from_defaults(llm=llm)
query_engine = index.as_query_engine(service_context=ctx)
response = query_engine.query("作者成长过程中做了哪些事情?")
print(str(response))
\end{lstlisting}

\item \textbf{Pydantic程序}:将输入文字串转换成结构化的Pydantic对象,包括LLM文本完成Pydantic程序、LLM函数调用Pydantic程序、预设的Pydantic程序。

\item \textbf{OpenAI JSON模式}:设置response\_format为{"type": "json\_object"},激活响应的JSON模式,确保生成能够被解析为有效JSON对象的字符串。
\end{enumerate}

\section{问题六:特异性错误}

\subsection{特异性错误问题介绍}
有时候得到的回答可能缺少必要的细节或特定性,需要进一步提问来获取清晰的信息。有些答案可能过于含糊或泛泛,不能有效地满足用户的实际需求。

\subsection{特异性错误问题解决方案}
采用更高级的检索技巧来改善答案的详细程度:
\begin{itemize}
\item 从细节到全局的检索
\item 围绕特定句子进行的检索  
\item 逐步深入的检索
\end{itemize}

\section{问题七:回答不全面}

\subsection{回答不全面问题介绍}
有时候得到的是部分答案,并不是错误的,但没有提供所有必要的细节,即便这些信息实际上是存在并且可以获取的。

\subsection{回答不全面问题解决方案}
\textbf{查询优化}:在简单的RAG模型中,比较性问题往往处理得不够好。加入查询理解层—在实际进行向量存储查询之前进行查询优化。四种查询优化方式:
\begin{itemize}
\item \textbf{路由优化}:保留原始查询内容,明确涉及的特定工具子集
\item \textbf{查询改写}:保持选定工具不变,重新构思多种查询方式
\item \textbf{细分问题}:将大问题拆分成几个小问题
\item \textbf{ReAct Agent工具选择}:根据原始查询确定使用哪个工具
\end{itemize}

\begin{lstlisting}[language=Python]
# 使用HyDE查询转换运行查询
query_str = "what did paul graham do after going to RISD"
hyde = HyDEQueryTransform(include_original=True)
query_engine = index.as_query_engine()
query_engine = TransformQueryEngine(query_engine, query_transform=hyde)
response = query_engine.query(query_str)
print(response)
\end{lstlisting}

\section{问题八:数据处理能力的挑战}

\subsection{数据处理能力挑战介绍}
在RAG技术流程中,处理大量数据时常会遇到系统若无法高效地管理和加工这些数据,就可能导致性能瓶颈甚至系统崩溃。这种处理能力上的挑战可能会让数据处理的时间大幅拉长,系统超负荷运转,数据质量下降,以及服务的可用性降低。

\subsection{数据处理能力挑战解决方案}
\textbf{并行技术}:推出数据处理的并行技术,能够使文档处理速度最多提升15倍。

\begin{lstlisting}[language=Python]
# 加载数据
documents = SimpleDirectoryReader(input_dir="./data/source_files").load_data()

# 创建带有转换的管道
pipeline = IngestionPipeline(
    transformations=[
        SentenceSplitter(chunk_size=1024, chunk_overlap=20),
        TitleExtractor(),
        OpenAIEmbedding(),
    ]
)

# 将num_workers设置为大于1的值将调用并行执行
nodes = pipeline.run(documents=documents, num_workers=4)
\end{lstlisting}

\section{问题九:结构化数据查询的难题}

\subsection{结构化数据查询难题介绍}
用户在查询结构化数据时,精准地获取想要的信息是一项挑战,尤其是遇到复杂或含糊的查询条件时。当前的大语言模型在这方面还存在局限,例如无法灵活地将自然语言转换为SQL查询语句。

\subsection{结构化数据查询难题解决方案}
\begin{enumerate}
\item \textbf{Chain-of-table Pack}:将链式思考与表格的转换和表述相结合,通过一系列规定的操作逐步变换表格,并在每一步向大语言模型展示新变化的表格。

\item \textbf{Mix-Self-Consistency Pack}:通过自治机制(多数投票)聚合文本和符号推理的结果。

\begin{lstlisting}[language=Python]
download_llama_pack(
    "MixSelfConsistencyPack",
    "./mix_self_consistency_pack", 
    skip_load=True,
)

query_engine = MixSelfConsistencyQueryEngine(
    df=table,
    llm=llm,
    text_paths=5,      # 抽样5条文本推理路径
    symbolic_paths=5,  # 抽样5个符号推理路径
    aggregation_mode="self-consistency",  # 通过自治聚合结果
    verbose=True,
)

response = await query_engine.aquery(example["utterance"])
\end{lstlisting}
\end{enumerate}

\section{问题十:从复杂PDF文件中提取数据}

\subsection{复杂PDF数据提取问题介绍}
处理PDF文件时,需要从里面复杂的表格中提取出数据来回答问题,但简单的检索方法做不到这一点,需要更高效的技术。

\subsection{复杂PDF数据提取问题解决方案}
\textbf{嵌入式表格检索}:提供EmbeddedTablesUnstructuredRetrieverPack工具包,从HTML文档中解析出嵌入的表格,把它们组织成清晰的结构图,然后根据用户问题找出并获取相关表格的数据。

\begin{lstlisting}[language=Python]
# 下载和安装依赖项
EmbeddedTablesUnstructuredRetrieverPack = download_llama_pack(
    "EmbeddedTablesUnstructuredRetrieverPack",
    "./embedded_tables_unstructured_pack",
)

# 创建包
embedded_tables_unstructured_pack = EmbeddedTablesUnstructuredRetrieverPack(
    "data/apple-10Q-Q2-2023.html",  # 接收html文件
    nodes_save_path="apple-10-q.pkl"
)

# 运行包
response = embedded_tables_unstructured_pack.run("总运营费用是多少?").response
display(Markdown(f"{response}"))
\end{lstlisting}

\section{问题十一:备用模型}

\subsection{备用模型问题介绍}
在使用大型语言模型时,可能会担心如果模型出了问题怎么办,比如遇到了OpenAI模型的使用频率限制。这时候就需要一个或多个备用模型以防万一主模型出现故障。

\subsection{备用模型问题解决方案}
\begin{enumerate}
\item \textbf{Neutrino路由器}:一个大语言模型的集合,把问题发送到这里,用预测模型判断哪个大语言模型最适合处理问题。

\begin{lstlisting}[language=Python]
from llama_index.llms import Neutrino
from llama_index.llms import ChatMessage

llm = Neutrino(
    api_key="<your-Neutrino-api-key>",
    router="test"  # 在Neutrino仪表板中配置的"测试"路由器
)

response = llm.complete("什么是大语言模型?")
print(f"Optimal model: {response.raw['model']}")
\end{lstlisting}

\item \textbf{OpenRouter}:统一的接口,可以访问任何大语言模型,自动找到最便宜的模型,并在主服务器出现问题时提供备选方案。

\begin{lstlisting}[language=Python]
from llama_index.llms import OpenRouter
from llama_index.llms import ChatMessage

llm = OpenRouter(
    api_key="<your-OpenRouter-api-key>",
    max_tokens=256,
    context_window=4096,
    model="gryphe/mythomax-12-13b",
)

message = ChatMessage(role="user", content="Tell me a joke")
resp = llm.chat([message])
print(resp)
\end{lstlisting}
\end{enumerate}

\section{问题十二:大语言模型(LLM)的安全挑战}

\subsection{LLM安全挑战介绍}
面对如何防止恶意输入操控、处理潜在的不安全输出和避免敏感信息泄露等问题,每位AI架构师和工程师都需要找到解决方案。

\subsection{LLM安全挑战解决方案}
\textbf{Llama Guard}:以7-B Llama 2为基础,旨在对大语言模型进行内容分类,通过对输入的提示进行分类和对输出的响应进行分类来工作。能够产生文本结果,判断特定的输入提示或输出响应是否安全。如果根据规则识别出内容不安全,还会指出违反的具体规则子类别。

\section{总结}

\begin{table}[h]
\centering
\caption{RAG痛点及解决方案总结}
\begin{tabular}{@{}p{0.08\textwidth}p{0.3\textwidth}p{0.5\textwidth}@{}}
\toprule
\textbf{序号} & \textbf{痛点} & \textbf{解决方案} \\
\midrule
1 & 内容缺失 & 清洗数据\&改善提示方法 \\
2 & 错过排名靠前的文档 & 超参数调优\&文档重新排序 \\
3 & 上下文不连贯-综合策略限制 & 调整检索策略\&微调嵌入向量 \\
4 & 未提取到信息 & 清洗数据,压缩提示,\&长上下文重排序 \\
5 & 格式错误 & 改善提示方法,输出解析,pydantic程序设计,\& OpenAI的JSON模式 \\
6 & 特异性错误 & 高级检索策略 \\
7 & 响应不完整 & 查询优化 \\
8 & 数据摄取的可扩展性 & 并行化摄取流程 \\
9 & 结构化数据问答 & Chain-of-table Pack\& Mix-Self-Consistency Pack \\
10 & 从复杂PDF文件中提取数据 & 嵌入式表格检索 \\
11 & 后备模型 & Neutrino路由器\& OpenRouter \\
12 & 大语言模型(LLM)安全性 & Llama Guard守护程序 \\
\bottomrule
\end{tabular}
\end{table}

我们讨论了开发RAG应用时的12个痛点(论文中的7个加上另外5个),并为它们每一个都提供了相应的解决方案。这些解决方案涵盖了从数据预处理到查询优化,从模型选择到安全防护的各个方面,为构建高质量的RAG系统提供了全面的指导。

