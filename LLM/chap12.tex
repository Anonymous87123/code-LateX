
\chapter{大模型 RAG 经验面}

\section{LLMs 的不足与挑战}

\subsection{LLMs 存在的不足点}
在LLM已经具备了较强能力的基础上,仍然存在以下问题:

\begin{itemize}
\item \textbf{幻觉问题}:LLM文本生成的底层原理是基于概率的token by token的形式,因此会不可避免地产生"一本正经的胡说八道"的情况

\item \textbf{时效性问题}:LLM的规模越大,大模型训练的成本越高,周期也就越长。那么具有时效性的数据也就无法参与训练,所以也就无法直接回答时效性相关的问题,例如"帮我推荐几部热映的电影?"

\item \textbf{数据安全问题}:通用的LLM没有企业内部数据和用户数据,那么企业想要在保证安全的前提下使用LLM,最好的方式就是把数据全部放在本地,企业数据的业务计算全部在本地完成。而在线的大模型仅仅完成一个归纳的功能
\end{itemize}

\section{RAG 技术概述}

\subsection{什么是 RAG?}
RAG(Retrieval Augmented Generation,检索增强生成),即LLM在回答问题或生成文本时,先会从大量文档中检索出相关的信息,然后基于这些信息生成回答或文本,从而提高预测质量。

\subsection{RAG 核心组件}

\subsubsection{检索器模块(R)}
在RAG技术中,"R"代表检索,其作用是从大量知识库中检索出最相关的前k个文档。构建高质量的检索器面临三个关键挑战:

\textbf{2.1.1 如何获得准确的语义表示?}
在RAG中,语义空间指的是查询和文档被映射的多维空间。构建准确语义空间的方法:

\begin{itemize}
\item \textbf{块优化}:处理外部文档的第一步是分块,以获得更细致的特征。选择分块策略时需要考虑被索引内容的特点、使用的嵌入模型及其最适块大小、用户查询的预期长度和复杂度

\item \textbf{微调嵌入模型}:在确定Chunk的适当大小后,通过嵌入模型将Chunk和查询嵌入。优秀的嵌入模型如UAE、Voyage、BGE等,它们在大规模语料库上预训练过
\end{itemize}

\textbf{2.1.2 如何协调查询和文档的语义空间?}
协调用户的查询与文档的语义空间的技术:

\begin{itemize}
\item \textbf{查询重写}:利用大语言模型的能力生成指导性伪文档,或将原始查询与伪文档结合形成新查询。多查询检索方法让大语言模型能够同时产生多个搜索查询

\item \textbf{嵌入变换}:通过在查询编码器后加入特殊适配器并微调,优化查询的嵌入表示。SANTA方法让检索系统能够理解并处理结构化的信息
\end{itemize}

\textbf{2.1.3 如何对齐检索模型的输出和大语言模型的偏好?}
对齐方法:

\begin{itemize}
\item \textbf{大语言模型的监督训练}:REPLUG使用检索模型和大语言模型计算检索到的文档的概率分布,然后通过计算KL散度进行监督训练

\item \textbf{适配器附加}:在检索模型上外部附加适配器来实现对齐,避免微调嵌入模型的挑战

\item \textbf{指令微调}:PKG通过指令微调将知识注入到白盒模型中,直接替换检索模块
\end{itemize}

\subsubsection{生成器模块(G)}
\textbf{2.2.1 生成器介绍}
\begin{itemize}
\item \textbf{作用}:将检索到的信息转化为自然流畅的文本。输入不仅包括传统的上下文信息,还有通过检索器得到的相关文本片段

\item \textbf{特点}:能够更深入地理解问题背后的上下文,并产生更加信息丰富的回答。根据检索到的文本来指导内容的生成,确保一致性
\end{itemize}

\textbf{2.2.2 后检索处理提升策略}
\begin{itemize}
\item \textbf{目的}:提高检索结果的质量,更好地满足用户需求或为后续任务做准备

\item \textbf{策略}:包括信息压缩和结果的重新排序
\end{itemize}

\textbf{2.2.3 生成器优化方法}
\begin{itemize}
\item \textbf{优化目的}:确保生成文本既流畅又能有效利用检索文档,更好地回应用户的查询

\item \textbf{方法}:对检索器找到的文档进行后续处理,微调方式与大语言模型的普通微调方法大体相同
\end{itemize}

\section{RAG 的优势}

使用RAG的好处包括:

\begin{itemize}
\item \textbf{可扩展性}:减少模型大小和训练成本,允许轻松扩展知识

\item \textbf{准确性}:通过引用信息来源,用户可以核实答案的准确性,增强对模型输出结果的信任

\item \textbf{可控性}:允许更新或定制知识

\item \textbf{可解释性}:检索到的项目作为模型预测中来源的参考

\item \textbf{多功能性}:可以针对多种任务进行微调和定制,包括QA、文本摘要、对话系统等

\item \textbf{及时性}:使用检索技术能识别到最新的信息,保持回答的及时性和准确性

\item \textbf{定制性}:通过索引与特定领域相关的文本语料库,为不同领域提供专业的知识支持

\item \textbf{安全性}:通过数据库中设置的角色和安全控制,实现对数据使用的更好控制
\end{itemize}

\section{RAG 与 SFT 对比}

\begin{table}[h]
\centering
\caption{RAG与SFT对比分析}
\begin{tabular}{@{}p{0.25\textwidth}p{0.35\textwidth}p{0.35\textwidth}@{}}
\toprule
\textbf{维度} & \textbf{RAG} & \textbf{SFT} \\
\midrule
数据 & 动态数据。RAG不断查询外部源,确保信息保持最新,而无需频繁的模型重新训练 & (相对)静态数据,并且在动态数据场景中可能很快就会过时。SFT也不能保证记住这些知识 \\
外部知识 & RAG擅长利用外部资源。通过在生成响应之前从知识源检索相关信息来增强LLM能力。它非常适合文档或其他结构化/非结构化数据库 & SFT可以对LLM进行微调以对齐预训练学到的外部知识,但对于频繁更改的数据源来说可能不太实用 \\
模型定制 & RAG主要关注信息检索,擅长整合外部知识,但可能无法完全定制模型的行为或写作风格 & SFT允许根据特定的语气或术语调整LLM的行为、写作风格或特定领域的知识 \\
减少幻觉 & RAG本质上不太容易产生幻觉,因为每个回答都建立在检索到的证据上 & SFT可以通过将模型基于特定领域的训练数据来帮助减少幻觉。但当面对不熟悉的输入时,它仍然可能产生幻觉 \\
透明度 & RAG系统通过将响应生成分解为不同的阶段来提供透明度,提供对数据检索的匹配度以提高对输出的信任 & SFT就像一个黑匣子,使得响应背后的推理更加不透明 \\
技术专长 & RAG需要高效的检索策略和大型数据库相关技术。另外还需要保持外部数据源集成以及数据更新 & SFT需要准备和整理高质量的训练数据集、定义微调目标以及相应的计算资源 \\
\bottomrule
\end{tabular}
\end{table}

两种方法并非非此即彼,合理的方式是结合业务需要与两种方法的优点,合理使用两种方法。

\section{RAG 典型实现方法}

RAG的实现主要包括三个主要步骤:数据索引、检索和生成。

\subsection{数据索引构建}

数据索引是一个离线的过程,主要是将私域数据向量化后构建索引并存入数据库的过程。

\textbf{Step1:数据提取}
\begin{itemize}
\item \textbf{数据获取}:包括多格式数据(PDF、word、markdown以及数据库和API等)加载、不同数据源获取等

\item \textbf{Doc类文档}:直接解析得到文本元素及其属性,用于后续切分的依据

\item \textbf{PDF类文档}:使用多个开源模型进行协同分析,如版面分析使用百度的PP-StructureV2

\item \textbf{PPT类文档}:将PPT转换成PDF形式,然后用处理PDF的方式来进行解析

\item \textbf{数据清洗}:对源数据进行去重、过滤、压缩和格式化等处理

\item \textbf{信息提取}:提取数据中关键信息,包括文件名、时间、章节title、图片等信息
\end{itemize}

\textbf{Step2:文本分割(Chunking)}
\begin{itemize}
\item \textbf{动机}:由于文本可能较长,或者仅有部分内容相关的情况下,需要对文本进行分块切分

\item \textbf{考虑因素}:embedding模型的Tokens限制情况;语义完整性对整体的检索效果的影响

\item \textbf{分块方式}:
\begin{itemize}
\item 句分割:以"句"的粒度进行切分,保留一个句子的完整语义
\item 固定大小的分块方式:根据embedding模型的token长度限制,将文本分割为固定长度
\item 基于意图的分块方式:句分割、递归分割、特殊分割
\end{itemize}

\item \textbf{常用工具}:langchain.text\_splitter库中的CharacterTextSplitter类
\end{itemize}

\textbf{Step3:向量化及创建索引}
\begin{itemize}
\item \textbf{向量化}:将文本、图像、音频和视频等转化为向量矩阵的过程

\item \textbf{常见embedding模型}:ChatGPT-Embedding、ERNIE-Embedding V1、M3E、BGE

\item \textbf{创建索引}:数据向量化后构建索引,并写入数据库的过程

\item \textbf{常用工具}:FAISS、Chromadb、ES、milvus等

\item \textbf{选择考虑}:根据业务场景、硬件、性能需求等多因素综合考虑
\end{itemize}

\subsection{数据检索策略}

\textbf{检索思路}:
\begin{itemize}
\item \textbf{元数据过滤}:通过元数据先进行过滤,提升效率和相关度

\item \textbf{图关系检索}:引入知识图谱,利用知识之间的关系做更准确的回答

\item \textbf{检索技术}:
\begin{itemize}
\item 向量化相似度检索:使用欧氏距离、曼哈顿距离、余弦等计算方式
\item 关键词检索:传统检索方式,元数据过滤也是一种
\item 全文检索、SQL检索:传统检索算法
\end{itemize}

\item \textbf{重排序}:根据相关度、匹配度等因素重新调整,得到更符合业务场景的排序

\item \textbf{查询轮换}:
\begin{itemize}
\item 子查询:使用各种查询策略,如树查询、向量查询、顺序查询chunks等
\item HyDE:生成相似的或更标准的prompt模板
\end{itemize}
\end{itemize}

\subsection{文本生成与回复}

文本生成就是将原始query和检索得到的文本组合起来输入模型得到结果的过程,本质上就是prompt engineering过程。

\begin{lstlisting}[language=Python]
from langchain.chat_models import ChatOpenAI
from langchain.schema.runnable import RunnablePassthrough

llm = ChatOpenAI(model_name="gpt-3.5-turbo", temperature=0)
rag_chain = {"context": retriever, "question": RunnablePassthrough()} | rag_prompt | llm
rag_chain.invoke("What is Task Decomposition?")
\end{lstlisting}

全流程框架如Langchain和LlamaIndex,都非常简单易用。

\section{RAG 典型案例}

\subsection{ChatPDF 及其复刻版}
ChatPDF的实现流程:
\begin{enumerate}
\item 读取PDF文件,转换为可处理的文本格式(如txt格式)
\item 对提取出来的文本进行清理和标准化(去除特殊字符、分段、分句等)
\item 使用OpenAI的Embeddings API将每个分段转换为向量
\item 将用户问题转换为向量,并与每个分段的向量进行比较,找到最相似的分段
\item 将最相似的分段与问题作为prompt,调用OpenAI的Completion API
\item 将ChatGPT生成的答案返回给用户
\end{enumerate}

\subsection{Baichuan 搜索增强系统}
百川大模型的搜索增强系统融合模块:
\begin{itemize}
\item \textbf{指令意图理解}:深入理解用户指令
\item \textbf{智能搜索}:精确驱动查询词的搜索
\item \textbf{结果增强}:结合大语言模型技术来优化模型结果生成的可靠性
\end{itemize}

通过这一系列协同作用,实现更精确、智能的模型结果回答,减少模型的幻觉。

\subsection{多模态检索增强模型}
RA-CM3是一个检索增强的多模态模型:
\begin{itemize}
\item 使用预训练的CLIP模型实现检索器(retriever)
\item 使用CM3 Transformer架构构成生成器(generator)
\item 检索器辅助模型从外部存储库中搜索有关提示文本的精确信息
\item 将该信息连同文本送入生成器中进行图像合成
\item 设计的模型的准确性大大提高
\end{itemize}

\section{RAG 存在的问题与挑战}

RAG技术目前存在以下问题:

\begin{itemize}
\item \textbf{检索效果依赖}:检索效果依赖embedding和检索算法。目前可能检索到无关信息,反而对输出有负面影响

\item \textbf{黑盒利用}:大模型如何利用检索到的信息仍是黑盒的。可能仍存在不准确(甚至生成的文本与检索信息相冲突)

\item \textbf{效率问题}:对所有任务都无差别检索k个文本片段,效率不高,同时会大大增加模型输入的长度

\item \textbf{引用和验证困难}:无法引用来源,也因此无法精准地查证事实,检索的真实性取决于数据源及检索算法
\end{itemize}

