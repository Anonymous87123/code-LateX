
\chapter{基于知识图谱的大模型检索增强实现策略:Graph RAG}

\section{引言:为什么需要Graph RAG?}

虽然LlamaIndex能够利用摘要索引进行增强的方案,但这些方案都是利用非结构化文本来实现的。对于知识图谱,是否可以将其作为一种检索路径来提高检索的相关性,这就引出了Graph RAG的概念。

知识图谱可以减少基于嵌入的语义搜索所导致的不准确性。例如:"保温大棚"与"保温杯",尽管在语义上两者存在相关性,但在大多数场景下,这种通用语义(Embedding)下的相关性很高,可能作为错误的上下文而引入"幻觉"。这时候,可以利用领域知识的知识图谱来缓解这种幻觉问题。

\section{Graph RAG基本概念}

\subsection{什么是Graph RAG?}
Graph RAG(Retrieval-Augmented Generation)是一种基于知识图谱的检索增强技术。它通过构建图模型的知识表达,将实体和关系之间的联系用图的形式进行展示,然后利用大语言模型(LLM)进行检索增强。

\subsection{Graph RAG的核心思路}
Graph RAG将知识图谱等价于一个超大规模的词汇表,而实体和关系则对应于单词。通过这种方式,Graph RAG在检索时能够将实体和关系作为单元进行联合建模。

Graph RAG的基本思想是:对用户输入的query提取实体,然后构造子图形成上下文,最后送入大模型完成生成。

\section{Graph RAG技术架构}

\subsection{整体工作流程}
Graph RAG的工作流程包含三个主要步骤:

\begin{enumerate}
\item \textbf{实体提取}:使用LLM(或其他)模型从问题中提取关键实体
\item \textbf{子图检索}:根据提取的实体检索相关子图,深入到一定的深度(如2度或更多)
\item \textbf{答案生成}:利用获得的上下文利用LLM产生答案
\end{enumerate}

通过这种方式,知识图谱召回可以作为一路检索路径与传统的向量检索进行融合。

\subsection{代码实现框架}

\begin{lstlisting}[language=Python]
class GraphRAGSystem:
    def __init__(self, kg_store, llm, entity_extractor):
        self.kg_store = kg_store  # 知识图谱存储
        self.llm = llm            # 大语言模型
        self.entity_extractor = entity_extractor  # 实体提取器
    
    def extract_entities(self, query):
        """从查询中提取实体"""
        # 使用LLM或实体识别模型提取实体
        entities = self.entity_extractor.extract(query)
        return entities
    
    def retrieve_subgraph(self, entities, max_depth=2):
        """根据实体检索子图"""
        subgraph = self.kg_store.get_subgraph(
            entities=entities, 
            max_depth=max_depth
        )
        return subgraph
    
    def generate_answer(self, query, subgraph):
        """基于子图生成答案"""
        # 将子图转换为文本上下文
        context = self.subgraph_to_text(subgraph)
        
        # 构建提示
        prompt = f"""
        基于以下知识图谱信息回答问题:
        
        知识图谱上下文:
        {context}
        
        问题:{query}
        
        请根据提供的知识图谱信息给出准确的回答。
        """
        
        # 使用LLM生成答案
        answer = self.llm.generate(prompt)
        return answer
    
    def query(self, query, max_depth=2):
        """完整的Graph RAG查询流程"""
        # 1. 实体提取
        entities = self.extract_entities(query)
        print(f"提取的实体: {entities}")
        
        # 2. 子图检索
        subgraph = self.retrieve_subgraph(entities, max_depth)
        print(f"检索到的子图包含 {len(subgraph.nodes)} 个节点和 {len(subgraph.edges)} 条边")
        
        # 3. 答案生成
        answer = self.generate_answer(query, subgraph)
        return answer
\end{lstlisting}

\section{Graph RAG具体实现}

\subsection{实体提取技术}

实体提取是Graph RAG的第一步,需要从用户查询中准确识别出关键实体。常用的实体提取方法包括:

\begin{lstlisting}[language=Python]
import spacy
import re
from typing import List

class EntityExtractor:
    def __init__(self, model_name="zh_core_web_sm"):
        self.nlp = spacy.load(model_name)
    
    def extract_with_llm(self, query: str) -> List[str]:
        """使用LLM进行实体提取"""
        prompt = f"""
        从以下问题中提取关键实体(人名、地名、机构名、专业术语等):
        问题:"{query}"
        
        请以JSON格式返回实体列表:
        {{"entities": ["实体1", "实体2", ...]}}
        """
        
        response = llm.generate(prompt)
        # 解析JSON响应
        entities = self._parse_llm_response(response)
        return entities
    
    def extract_with_ner(self, query: str) -> List[str]:
        """使用命名实体识别提取实体"""
        doc = self.nlp(query)
        entities = []
        for ent in doc.ents:
            if ent.label_ in ["PERSON", "ORG", "GPE", "PRODUCT", "EVENT"]:
                entities.append(ent.text)
        return entities
    
    def hybrid_extract(self, query: str) -> List[str]:
        """混合实体提取方法"""
        # 先用NER提取
        ner_entities = self.extract_with_ner(query)
        
        # 如果NER提取结果不足,使用LLM补充
        if len(ner_entities) < 2:
            llm_entities = self.extract_with_llm(query)
            # 去重合并
            all_entities = list(set(ner_entities + llm_entities))
            return all_entities
        
        return ner_entities
\end{lstlisting}

\subsection{子图检索策略}

子图检索需要根据提取的实体在知识图谱中检索相关的子图结构:

\begin{lstlisting}[language=Python]
class KnowledgeGraphStore:
    def __init__(self, graph_db_connection):
        self.conn = graph_db_connection
    
    def get_subgraph(self, entities: List[str], max_depth: int = 2):
        """检索包含实体的子图"""
        # 构建Cypher查询
        query = self._build_cypher_query(entities, max_depth)
        
        # 执行查询
        result = self.conn.run(query)
        subgraph = self._parse_cypher_result(result)
        return subgraph
    
    def _build_cypher_query(self, entities: List[str], max_depth: int) -> str:
        """构建Cypher查询语句"""
        entity_conditions = " OR ".join([f"n.name = '{e}'" for e in entities])
        
        cypher_query = f"""
        MATCH path = (start)-[*1..{max_depth}]-(end)
        WHERE {entity_conditions}
        WITH nodes(path) as nodes, relationships(path) as rels
        RETURN nodes, rels
        LIMIT 100
        """
        return cypher_query
    
    def _parse_cypher_result(self, result):
        """解析Cypher查询结果"""
        subgraph = {
            "nodes": set(),
            "edges": set()
        }
        
        for record in result:
            # 处理节点
            for node in record["nodes"]:
                node_id = node.id
                node_props = dict(node)
                subgraph["nodes"].add((node_id, node_props))
            
            # 处理关系
            for rel in record["rels"]:
                rel_id = rel.id
                rel_type = rel.type
                rel_props = dict(rel)
                subgraph["edges"].add((
                    rel.start_node.id, 
                    rel.end_node.id, 
                    rel_type, 
                    rel_props
                ))
        
        return subgraph
\end{lstlisting}

\section{Graph RAG应用示例}

\subsection{示例1:人物信息查询}

当用户输入"tell me about Peter Quill"时,Graph RAG的处理流程如下:

\begin{enumerate}
\item \textbf{实体提取}:识别关键词["Peter", "Quill"]
\item \textbf{子图检索}:编写Cypher语句获得二跳结果
\item \textbf{答案生成}:基于检索到的子图信息生成回答
\end{enumerate}

\subsection{示例2:机构事件查询}

用户输入:"Tell me events about NASA"

\begin{lstlisting}
> 提取的关键词:['NASA', 'events']
> 召回的2度关系子图:
Extracted relationships: The following are knowledge triplets in max depth 2 in the form of subject[predicate, object, predicate_next_hop, object_next_hop...]

nasa['public release date', 'mid-2023']
nasa['announces', 'future space telescope programs']
nasa['publishes images of', 'debris disk'] 
nasa['discovers', 'exoplanet lhs 475 b']
\end{lstlisting}

\subsection{完整示例代码}

\begin{lstlisting}[language=Python]
def graph_rag_demo():
    """Graph RAG完整示例"""
    # 初始化组件
    kg_store = Neo4jGraphStore(uri="bolt://localhost:7687", 
                              user="neo4j", 
                              password="password")
    llm = OpenAI(model="gpt-4")
    entity_extractor = EntityExtractor()
    
    # 创建Graph RAG系统
    graph_rag = GraphRAGSystem(kg_store, llm, entity_extractor)
    
    # 示例查询
    queries = [
        "Tell me about Peter Quill",
        "Tell me events about NASA",
        "莫妮卡·贝鲁奇的代表作是什么?"
    ]
    
    for query in queries:
        print(f"\n=== 查询: {query} ===")
        
        # 执行Graph RAG查询
        result = graph_rag.query(query, max_depth=2)
        
        print(f"提取的实体: {graph_rag.entities}")
        print(f"生成的答案: {result}")
        
        # 记录token使用情况
        print("Token使用统计:")
        print(f"LLM总token使用: 159 tokens")
        print(f"Embedding总token使用: 0 tokens")

# 运行示例
if __name__ == "__main__":
    graph_rag_demo()
\end{lstlisting}

\section{Graph RAG排序优化策略}

基于知识图谱召回的方法可以和其他召回方法融合,但在图谱规模很大时,单纯的子图检索可能存在优化空间。

\subsection{现有方法的局限性}

\begin{itemize}
\item \textbf{子图召回的多条路径中可能会出现不相关的路径}
\item \textbf{实体识别阶段的精度有限},采用关键词提取方法比较直接,效果有待提升
\item \textbf{依赖于基础知识图谱库的质量},如果数据量或覆盖范围不足,可能引入噪声
\end{itemize}

\subsection{两阶段排序优化}

为了提升Graph RAG的效果,可以采用先粗排后精排的两阶段排序策略:

\subsubsection{粗排阶段(LightGBM模型)}

在粗排阶段,根据问题query和候选路径path的特征,使用LightGBM机器学习模型对候选路径进行初步排序,保留top N条路径。

使用的特征包括:
\begin{itemize}
\item 字符重合数
\item 词重合数  
\item 编辑距离
\item path跳数
\item path长度
\item 字符的Jaccard相似度
\item 词语的Jaccard相似度
\item path中的关系数
\item path中的实体个数
\item path中的答案个数
\item 判断path的字符是否全在query中
\item 判断query和path中是否都包含数字
\item 获取数字的Jaccard相似度
\end{itemize}

\subsubsection{精排阶段(预训练语言模型)}

在精排阶段,采用预训练语言模型计算query和粗排阶段path的语义匹配度,选择得分top 2-3的答案路径作为最终结果。

\begin{lstlisting}[language=Python]
class GraphRAGRanker:
    def __init__(self, lightgbm_model, sentence_transformer):
        self.lgb_model = lightgbm_model
        self.st_model = sentence_transformer
    
    def coarse_ranking(self, candidate_paths, query, top_k=50):
        """粗排阶段"""
        features = self.extract_features(candidate_paths, query)
        scores = self.lgb_model.predict(features)
        
        # 按分数排序并返回top K
        ranked_indices = np.argsort(scores)[::-1][:top_k]
        return [candidate_paths[i] for i in ranked_indices]
    
    def fine_ranking(self, coarse_paths, query, top_k=3):
        """精排阶段"""
        # 计算语义相似度
        query_embedding = self.st_model.encode([query])
        path_embeddings = self.st_model.encode([str(path) for path in coarse_paths])
        
        similarities = cosine_similarity(query_embedding, path_embeddings)[0]
        
        # 选择最相似的路径
        top_indices = np.argsort(similarities)[::-1][:top_k]
        return [coarse_paths[i] for i in top_indices]
    
    def extract_features(self, paths, query):
        """提取排序特征"""
        features = []
        for path in paths:
            feature_vector = []
            
            # 字符级别特征
            path_str = str(path)
            feature_vector.append(len(set(query) & set(path_str)))  # 字符重合数
            feature_vector.append(editdistance.eval(query, path_str))  # 编辑距离
            
            # 词级别特征
            query_words = set(query.split())
            path_words = set(path_str.split())
            feature_vector.append(len(query_words & path_words))  # 词重合数
            feature_vector.append(jaccard_similarity(query_words, path_words))  # Jaccard相似度
            
            # 图结构特征
            feature_vector.append(len(path.nodes))  # 实体个数
            feature_vector.append(len(path.edges))  # 关系数
            feature_vector.append(self.get_path_depth(path))  # 路径深度
            
            features.append(feature_vector)
        
        return np.array(features)
\end{lstlisting}

\section{Graph RAG的优势与挑战}

\subsection{技术优势}
\begin{itemize}
\item \textbf{结构化知识利用}:能够充分利用知识图谱中的结构化信息
\item \textbf{减少语义歧义}:通过实体关系明确语义,减少Embedding相似性带来的歧义
\item \textbf{可解释性强}:检索路径清晰,结果可追溯
\item \textbf{领域适应性好}:特别适合领域知识丰富的场景
\end{itemize}

\subsection{面临挑战}
\begin{itemize}
\item \textbf{知识图谱构建成本高}:需要高质量的知识图谱数据
\item \textbf{实体识别精度依赖}:实体提取的准确性直接影响后续效果
\item \textbf{子图检索效率}:大规模图谱上的子图检索需要优化
\item \textbf{多源信息融合}:如何与文本检索等其他检索方式有效融合
\end{itemize}

\subsection{未来发展方向}
\begin{itemize}
\item \textbf{自动化知识图谱构建}:结合大语言模型自动构建和更新知识图谱
\item \textbf{多模态Graph RAG}:支持文本、图像等多模态知识的图谱检索
\item \textbf{动态图谱学习}:支持实时更新和增量学习的图谱系统
\item \textbf{跨语言Graph RAG}:支持多语言知识图谱的检索增强
\end{itemize}

\section{总结}

Graph RAG作为一种基于知识图谱的检索增强生成技术,通过将结构化的图谱知识与大语言模型相结合,为RAG系统提供了新的技术路径。它特别适合需要精确结构化知识支持的场景,能够有效减少单纯基于Embedding相似性检索带来的语义歧义问题。

尽管Graph RAG在实体识别精度、图谱构建成本等方面仍面临挑战,但通过两阶段排序优化、多源检索融合等技术手段,可以显著提升其效果。随着知识图谱技术和大语言模型的不断发展,Graph RAG在专业领域问答、知识推理等场景中将发挥越来越重要的作用。

未来的研究方向包括自动化知识图谱构建、多模态Graph RAG、动态图谱学习等,这些技术的发展将进一步拓展Graph RAG的应用边界和实用价值。

