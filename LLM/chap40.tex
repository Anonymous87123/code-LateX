\chapter{SQLENS:文本到SQL语义错误检测与纠正的端到端框架}

文本到SQL系统(text-to-SQL system)将自然语言问题转换为SQL查询,使得非技术用户能够与结构化数据交互。虽然大语言模型在文本到SQL任务上显示出有希望的结果,但它们经常产生语义错误但语法有效的查询,且对其可靠性缺乏深入理解。

该论文提出SQLENS,一个用于细粒度检测和纠正LLM生成SQL中语义错误的端到端框架。SQLENS集成来自底层数据库和LLM的错误信号来识别SQL子句中的潜在语义错误,并进一步利用这些信号指导查询纠正。在两个公开基准测试上的实证结果表明,SQLENS在错误检测的F1分数上比最佳基于LLM的自评估方法高出25.78\%,并将现成文本到SQL系统的执行准确率提升高达20\%。

\subsection{核心概念定义}
\subsubsection{细粒度检测}
在文本到SQL任务的上下文中,\textbf{细粒度检测} 特指对大型语言模型生成的SQL查询进行深入的结构化分析,旨在精准地定位和分类语义错误的精确位置与具体类型。与仅判断查询整体正确性的粗粒度评估不同,细粒度检测具备以下特征:

\begin{itemize}
    \item \textbf{分析粒度:} 以SQL查询的\textbf{子句}为基本分析单元(如\texttt{SELECT}、\texttt{FROM}、\texttt{WHERE}、\texttt{GROUP BY}、\texttt{JOIN ON}等),而非整个查询语句。
    \item \textbf{错误类型:} 专门针对\textbf{语义错误},即语法有效但逻辑与用户意图或数据库现实状态不符的查询。常见错误类型包括:
    \begin{itemize}
        \item \textbf{模式引用错误:} 引用不存在的表、列,或错误拼写。
        \item \textbf{值域不匹配错误:} 谓词条件中的值不在目标列的值分布内(例如,查询“city = 北京”但该列仅有“上海”、“广州”)。
        \item \textbf{聚合误用错误:} 错误地使用(或遗漏)聚合函数(如\texttt{COUNT}, \texttt{SUM})。
        \item \textbf{连接条件错误:} 多表连接时关联条件不正确,导致笛卡尔积或丢失关联。
    \end{itemize}
    \item \textbf{输出形式:} 产生结构化的错误信号,明确指出错误所在的子句及其错误类别,而非简单的二分类(正确/错误)结果。
\end{itemize}

\textbf{比喻:} 细粒度检测如同一位经验丰富的数据库管理员进行代码审查,不仅指出SQL无法运行,更精确标注出“\texttt{WHERE}子句中的列名拼写错误”或“\texttt{JOIN}条件遗漏了关键关联字段”。

\subsubsection{端到端框架}
在SQLENS的上下文中,\textbf{端到端框架} 指一个将\textbf{细粒度错误检测}与\textbf{基于反馈的查询纠正}两个核心阶段无缝集成的自动化系统。其“端到端”特性体现在:

\begin{itemize}
    \item \textbf{输入与输出:}
    \begin{itemize}
        \item \textbf{输入端:} 原始自然语言问题 $Q$ 和初始LLM生成的、可能包含错误的SQL查询 $S_{initial}$。
        \item \textbf{输出端:} 经过检测与纠正后,语义正确性显著提升的最终SQL查询 $S_{corrected}$。
    \end{itemize}
    \item \textbf{内部工作流:} 框架内部形成一个自动化、信息闭环的处理流水线,无需人工干预:
    \begin{enumerate}
        \item \textbf{统一信号采集:} 同时利用\textbf{底层数据库}的元信息、值分布、约束,以及\textbf{LLM}自身的推理能力,对 $S_{initial}$ 进行细粒度分析,生成结构化错误诊断报告 $\mathcal{E}$。
        \item \textbf{闭环引导式纠正:} 将错误报告 $\mathcal{E}$ 作为明确的、结构化的指导信号,输入到\textbf{纠正模块}(通常仍为LLM),引导其针对性地修正 $S_{initial}$ 中已识别的具体错误,生成 $S_{corrected}$。
    \end{enumerate}
    \item \textbf{关键特征:} 两个阶段共享内部状态与信号,纠正阶段直接利用检测阶段输出的精细化诊断信息,而非独立运作。
\end{itemize}

\textbf{比喻:} 该框架如同一条智能装配线,视觉检测单元(细粒度检测)发现特定螺丝未拧紧,立即将“位置A,螺丝松动”的指令发送给机械臂(纠正模块),机械臂精准地拧紧该螺丝,完成产品修正。整个过程自动连续,信息无损传递。

\subsection{总结}
在SQLENS框架中:
\begin{itemize}
    \item \textbf{细粒度检测} 代表了其\textbf{分析的深度与精度},使其具备“精准诊断”的能力。
    \item \textbf{端到端框架} 代表了其\textbf{系统架构的集成度与自动化水平},实现了从“诊断”到“治疗”的连贯、自主闭环。
\end{itemize}
二者的结合,使SQLENS不仅能更准确地定位LLM生成SQL中的深层语义错误,还能高效地利用这些诊断信息驱动查询的自动纠正,从而系统性提升文本到SQL系统的可靠性。

\section{引入}
文本到SQL系统通过将自然语言问题转换为SQL查询,使非技术用户能够查询关系数据库并提取见解。大语言模型显著推进了这一任务,但LLM生成的查询仍然容易出错。文本到SQL系统的主要差距是缺乏细粒度、可解释的错误检测,特别是对于语义错误(查询执行但返回错误结果)。

\subsection{目标}
\begin{itemize}
\item \textbf{挑战1}:在子句级别识别语义错误
\item \textbf{挑战2}:使用噪声错误信号预测SQL查询的正确性
\item \textbf{挑战3}:在没有正确性预言的情况下修复SQL查询
\end{itemize}

\subsection{相关工作}
\subsubsection{文本到SQL}
基于LLM的文本到SQL系统(如DIN-SQL、MAC-SQL)使用多智能体框架将任务分解为子任务。其他方法如MCS-SQL、CHESS和Chase-SQL遵循类似的工作流程。最近还探索了基于强化学习的文本到SQL推理模型。

\subsubsection{文本到SQL中的错误检测与纠正}
早期方法主要关注二元正确性分类,提供有限的错误性质洞察。基于LLM的文本到SQL系统通常依赖SQL执行反馈或LLM自反思来判断正确性。相比之下,SQLENS产生可解释的子句级错误信号,可以指导用户调试和下游学习框架。

\section{方法论}
\subsection{问题定义}
\begin{itemize}
\item \textbf{定义1(文本到SQL算法)}:算法$f$接受自然语言问题$Q$、数据库实例$D$和可选的外部知识$K$,生成SQL查询$q = f(Q, D, K)$
\item \textbf{定义2(语义错误)}:语义错误$e$导致SQL查询$q$无法正确回答自然语言查询$Q$,形式化为$do(e) \Rightarrow O(q, D) \neq O(Q, D)$
\end{itemize}

\subsection{SQLENS错误检测}
SQLENS的错误检测器结合来自数据库和LLM的信号来检测语义错误。常见的语义错误类别包括:

\subsubsection{数据库错误信号}
\begin{itemize}
\item 异常结果信号:检测SQL输出是否异常
\item 空谓词信号:检测SQL查询中是否产生空结果的谓词
\item 子查询中错误过滤信号:检测子查询中的问题过滤条件
\item 错误GROUP BY信号:检测错误使用的GROUP BY子句
\item 错误连接谓词信号:检测无效的连接谓词
\item 次优连接树信号:检测是否使用最优连接树
\item 表相似性信号:检测潜在的表选择错误
\item 不必要子查询信号:检测过度使用子查询
\item 值歧义信号:检测值使用中的歧义
\end{itemize}

\subsubsection{LLM错误信号}
\begin{itemize}
\item 列歧义信号:识别数据库中与SQL查询中使用列相似的列
\item 证据违反信号:识别生成的SQL查询是否违反问题中指定的证据
\item 证据不足信号:评估可用证据是否足以确认SQL查询正确回答问题
\item 错误问题子句链接信号:评估LLM对生成SQL子句的信心
\item LLM自检信号:LLM对自身输出的整体评估
\end{itemize}

\subsection{信号聚合}
SQLENS采用弱监督框架聚合这些噪声信号,将每个信号视为提供部分和噪声监督的标记函数。使用生成模型估计联合分布$p(\Lambda, Y)$,其中$Y$是未观察到的真实标签。

\subsection{SQLENS错误纠正}
SQLENS的错误纠正策略采用分解方法,逐步指导LLM修复查询:

\subsubsection{错误报告}
每个错误信号与包含以下内容的详细错误报告关联:
\begin{itemize}
\item 信号描述
\item 修正指令
\item 识别为潜在错误源的问题子句
\item 基于错误检测权重的置信度分级
\end{itemize}

\subsubsection{纠正组件}
\begin{itemize}
\item \textbf{错误选择器}:使用LLM确定首先修复哪个错误
\item \textbf{错误修复器}:应用针对性修正并使用SQL解析器验证修订后的查询
\item \textbf{SQL审计器}:在原始版本和修订版本之间选择最佳查询
\end{itemize}

\begin{itemize}
    \item \textbf{初始化步骤}:
    \begin{itemize}
        \item 调用 ErrorDetector 检测初始 SQL 查询 $q$ 中的所有语义错误,得到错误集合 $\mathcal{E}$
        \item 初始化迭代计数器 $i \leftarrow 0$
        \item 初始化修正后的查询 $q' \leftarrow q$
    \end{itemize}
    \item \textbf{迭代修正循环}:
    \begin{itemize}
        \item 循环条件:错误集合 $\mathcal{E}$ 非空且迭代次数 $i$ 小于最大迭代次数 $max\_iter$
        \item 每次迭代执行:
        \begin{itemize}
            \item 调用 ErrorSelector 从当前错误集合 $\mathcal{E}$ 中选择优先级最高的错误 $e$
            \item 调用 ErrorFixer 修复错误 $e$,生成修正后的查询 $q'$
            \item 从错误信号集合 $\mathcal{S}$ 中移除已处理的错误信号 $e$
            \item 调用 ErrorDetector 重新检测修正后查询 $q'$ 中的错误,更新错误集合 $\mathcal{E}$
        \end{itemize}
    \end{itemize}
    \item \textbf{护栏信号检查}:
    \begin{itemize}
        \item 检查护栏信号 $e_g$ 是否触发(返回集合大小为 1)
        \item 如果触发,调用 ErrorFixer 修复护栏信号对应的错误
    \end{itemize}
    \item \textbf{最终审计与返回}:
    \begin{itemize}
        \item 调用 SQLAuditor 审计修正后的查询 $q'$ 和原始查询 $q$
        \item 返回最优的修正查询 $q'$
    \end{itemize}
\end{itemize}

\section{实验评估}
\subsection{实验设置}
在BIRD和Spider两个标准文本到SQL基准测试的开发集上进行评估。使用四种最先进的文本到SQL系统生成SQL查询进行评估。

\subsection{主要结果}
SQLENS在所有现成的文本到SQL系统上一致提高了端到端准确率,并在BIRD上优于自反思和Fix-ALL。在现实环境中,SQLENS实现了最高的净改进,修复了更多查询并引入了更少的回归。

\begin{table}[htbp]
\centering
\caption{BIRD上的端到端准确率提升}
\begin{tabular}{lcccccc}
\toprule
方法 & $\triangle$Acc.(Nnet) & $\triangle$Acc.(Nfix) & Nnet & Nfix & Nbreak \\
\midrule
\multicolumn{6}{l}{Vanilla (初始准确率: 59.07\%)} \\
Self-Reflection & 59.07(+0.00\%) & 60.6(+1.53\%) & 1 & 22 & 21 \\
SQLENS w. Fix-ALL & 61.15(+2.08\%) & 62.34(+3.27\%) & 30 & 47 & 17 \\
SQLENS & 62.54(+3.47\%) & 63.66(+4.59\%) & 50 & 66 & 16 \\
\midrule
\multicolumn{6}{l}{DIN-SQL (初始准确率: 39.49\%)} \\
Self-Reflection & 54.63(+15.14\%) & 56.37(+16.88\%) & 209 & 233 & 24 \\
SQLENS w. Fix-ALL & 58.11(+18.62\%) & 59.71(+20.22\%) & 257 & 279 & 22 \\
SQLENS & 59.99(+20.50\%) & 61.45(+21.96\%) & 283 & 303 & 20 \\
\bottomrule
\end{tabular}
\end{table}

\subsection{消融研究}
\subsubsection{错误检测性能分析}
SQLENS在预测语义正确性方面优于所有基线方法。对于DIN-SQL,SQLENS实现了78.88的F1分数,比最佳LLM自评估基线高出21.45点。

\begin{table}[htbp]
\centering
\caption{BIRD上SQLENS错误检测效果}
\begin{tabular}{lccccc}
\toprule
方法 & 准确率 & AUC & 精确率 & 召回率 & F1 \\
\midrule
\multicolumn{6}{l}{Vanilla} \\
LLM Self-Eval.(Bool) & 60.53(±2.30) & X & 57.70(±18.90) & 10.02(±4.54) & 16.97(±7.35) \\
SQLENS & 64.63(±1.97) & 61.90(±2.18) & 58.11(±2.80) & 48.74(±4.31) & 52.94(±3.24) \\
\midrule
\multicolumn{6}{l}{DIN-SQL} \\
LLM Self-Eval.(Bool) & 61.52(±1.20) & X & 86.83(±2.18) & 42.99(±2.72) & 57.43(±2.20) \\
SQLENS & 75.29(±2.33) & 81.49(±1.74) & 81.64(±2.17) & 76.41(±3.50) & 78.88(±2.19) \\
\bottomrule
\end{tabular}
\end{table}

\subsubsection{SQL审计器和护栏信号的影响}
SQL审计器有助于减少修正过程中破坏的SQL查询数量,但以修复较少查询为代价。护栏信号的使用通过增加修复总数同时减少回归总数来提高净改进。

\subsection{个体信号的有效性}
\subsubsection{错误检测中的个体信号}
对于MAC-SQL,14个信号中有13个超过60\%的精确率。异常结果信号在识别40个错误时达到100\%精确率,次优连接树信号检测到最多错误(62\%精确率)。

\subsubsection{错误纠正中的个体信号}
在MAC-SQL生成的SQL查询上评估个体错误信号的纠正效果。在数据库信号中,次优连接树信号影响最强,空谓词信号也表现良好。在LLM信号中,证据违反信号产生最高的净增益。

\section{结论}
我们提出了SQLENS,一个利用数据库和基于LLM的错误信号进行文本到SQL中子句级语义错误检测和纠正的新框架。SQLENS使用弱监督聚合噪声信号,预测查询正确性,并应用LLM指导的迭代纠正。它在错误检测方面优于LLM自评估,并在不依赖正确性预言的情况下将现成文本到SQL系统的执行准确率提升高达20\%。

\section{局限性}
SQLENS在SQL错误检测和纠正中利用了LLM,因此继承了底层LLM的某些限制。SQLENS还产生了聚合多个信号和执行迭代纠正的计算开销,可能在实时环境中引入高延迟。然而,SQLENS旨在作为离线的通用SQL调试工具,设计为异步细化和验证生成的查询。
