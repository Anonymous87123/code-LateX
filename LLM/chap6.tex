\chapter{LLMs 损失函数篇}

\section{KL 散度}
KL(Kullback-Leibler)散度衡量了两个概率分布之间的差异。其公式为:
\begin{align*}
D_{KL}(P \parallel Q) = -\sum_{x \in X} P(x) \log \frac{1}{P(x)} + \sum_{x \in X} P(x) \log \frac{1}{Q(x)}
\end{align*}

\section{交叉熵损失函数}
交叉熵损失函数(Cross-Entropy Loss Function)是用于度量两个概率分布之间的差异的一种损失函数。在分类问题中,它通常用于衡量模型的预测分布与实际标签分布之间的差异。

$$H(p,q) = -\sum_{i=1}^{N} p_i \log(q_i) - (1-p_i) \log(1-q_i)$$

注:其中,p表示真实标签,q表示模型预测的标签,N表示样本数量。该公式可以看作是一个基于概率分布的比较方式,即将真实标签看做一个概率分布,将模型预测的标签也看做一个概率分布,然后计算它们之间的交叉熵。

物理意义:交叉熵损失函数可以用来衡量实际标签分布与模型预测分布之间的"信息差"。当两个分布完全一致时,交叉熵损失为0,表示模型的预测与实际情况完全吻合。当两个分布之间存在差异时,损失函数的值会增加,表示预测错误程度的大小。

\section{KL 散度与交叉熵的区别}
KL散度指的是相对熵,KL散度是两个概率分布P和Q差别的非对称性的度量。KL散度越小表示两个分布越接近。也就是说KL散度是不对称的,且KL散度的值是非负数。(也就是熵和交叉熵的差)

\begin{itemize}
\item 交叉熵损失函数是二分类问题中最常用的损失函数,由于其定义出于信息学的角度,可以泛化到多分类问题中。
\item KL散度是一种用于衡量两个分布之间差异的指标,交叉熵损失函数是KL散度的一种特殊形式。在二分类问题中,交叉熵函数只有一项,而在多分类问题中有多项。
\end{itemize}

\section{多任务学习各loss差异过大处理}
多任务学习中,如果各任务的损失差异过大,可以通过动态调整损失权重、使用任务特定的损失函数、改变模型架构或引入正则化等方法来处理。目标是平衡各任务的贡献,以便更好地训练模型。

\section{分类问题为什么用交叉熵损失函数不用均方误差(MSE)?}
交叉熵损失函数通常在分类问题中使用,而均方误差(MSE)损失函数通常用于回归问题。这是因为分类问题和回归问题具有不同的特点和需求。

分类问题的目标是将输入样本分到不同的类别中,输出为类别的概率分布。交叉熵损失函数可以度量两个概率分布之间的差异,使得模型更好地拟合真实的类别分布。它对概率的细微差异更敏感,可以更好地区分不同的类别。此外,交叉熵损失函数在梯度计算时具有较好的数学性质,有助于更稳定地进行模型优化。

相比之下,均方误差(MSE)损失函数更适用于回归问题,其中目标是预测连续数值而不是类别。MSE损失函数度量预测值与真实值之间的差异的平方,适用于连续数值的回归问题。在分类问题中使用MSE损失函数可能不太合适,因为它对概率的微小差异不够敏感,而且在分类问题中通常需要使用激活函数(如sigmoid或softmax)将输出映射到概率空间,使得MSE的数学性质不再适用。

综上所述,交叉熵损失函数更适合分类问题,而MSE损失函数更适合回归问题。

\section{信息增益}
信息增益是在决策树算法中用于选择最佳特征的一种评价指标。在决策树的生成过程中,选择最佳特征来进行节点的分裂是关键步骤之一,信息增益可以帮助确定最佳特征。

信息增益衡量了在特征已知的情况下,将样本集合划分成不同类别的纯度提升程度。它基于信息论的概念,使用熵来度量样本集合的不确定性。具体而言,信息增益是原始集合的熵与特定特征下的条件熵之间的差异。

在决策树的生成过程中,选择具有最大信息增益的特征作为当前节点的分裂标准,可以将样本划分为更加纯净的子节点。信息增益越大,意味着使用该特征进行划分可以更好地减少样本集合的不确定性,提高分类的准确性。

\section{多分类的分类损失函数(Softmax)}
多分类的分类损失函数采用Softmax交叉熵(Softmax Cross Entropy)损失函数。Softmax函数可以将输出值归一化为概率分布,用于多分类问题的输出层。Softmax交叉熵损失函数可以写成:
$-\sum_{i=1}^{n} y_i \log(p_i)$

注:其中,$n$是类别数,$y_i$是第$i$类的真实标签,$p_i$是第$i$类的预测概率。

\section{Softmax和交叉熵损失计算}
softmax计算公式如下:
$$y = \frac{e^{f_i}}{\sum_j e^{f_j}}$$

多分类交叉熵:
$$L = \frac{1}{N} \sum_i L_i = -\frac{1}{N} \sum_i \sum_{c=1}^{M} y_{ic} \log(p_{ic})$$

其中:
\begin{itemize}
\item M——类别的数量
\item $y_{ic}$——符号函数(0或1),如果样本i的真实类别等于c取1,否则取0
\item $p_{ic}$——观测样本i属于类别c的预测概率
\end{itemize}

二分类交叉熵:
\begin{align*}
L &= \frac{1}{N} \sum_i L_i = \frac{1}{N} \sum_i - \left[ y_i \cdot \log(p_i) + (1-y_i) \cdot \log(1-p_i) \right] \\
&\text{其中:} \\
&- y_i - \text{表示样本i的label,正类为1,负类为0} \\
&- p_i - \text{表示样本i预测为正类的概率}
\end{align*}

\section{Softmax数值稳定性问题}
如果softmax的e次方超过float的值了怎么办?

将分子分母同时除以x中的最大值,可以解决。
$$\tilde{x}_k = \frac{e^{x_k - \max(x)}}{e^{x_1 - \max(x)} + e^{x_2 - \max(x)} + \ldots + e^{x_k - \max(x)} + \ldots + e^{x_n - \max(x)}}$$

\chapter{相似度函数篇}

\section{相似度计算方法}
\subsection{除了余弦相似度还有哪些方法}
除了余弦相似度(cosine similarity)之外,常见的相似度计算方法还包括欧氏距离、曼哈顿距离、Jaccard相似度、皮尔逊相关系数等。

\section{对比学习}
\subsection{对比学习概述}
对比学习是一种无监督学习方法,通过训练模型使得相同样本的表示更接近,不同样本的表示更远离,从而学习到更好的表示。对比学习通常使用对比损失函数,例如Siamese网络、Triplet网络等,用于学习数据之间的相似性和差异性。

\section{对比学习中的负样本问题}
\subsection{负样本的重要性}
对比学习中负样本的重要性取决于具体的任务和数据。负样本可以帮助模型学习到样本之间的区分度,从而提高模型的性能和泛化能力。然而,负样本的构造成本可能会较高,特别是在一些领域和任务中。

\subsection{负样本构造成本过高的解决方案}
为了解决负样本构造成本过高的问题,可以考虑以下方法:

\begin{itemize}
\item \textbf{降低负样本的构造成本}:通过设计更高效的负样本生成算法或采样策略,减少负样本的构造成本。例如,可以利用数据增强技术生成合成的负样本,或者使用近似采样方法选择与正样本相似但不相同的负样本。

\item \textbf{确定关键负样本}:根据具体任务的特点,可以重点关注一些关键的负样本,而不是对所有负样本进行详细的构造。这样可以降低构造成本,同时仍然能够有效训练模型。

\item \textbf{迁移学习和预训练模型}:利用预训练模型或迁移学习的方法,可以在其他领域或任务中利用已有的负样本构造成果,减少重复的负样本构造工作。
\end{itemize}