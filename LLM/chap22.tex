
\chapter{提示学习(Prompting)技术详解}

\section{引言:为什么需要提示学习?}

\subsection{全量微调的挑战}
在面对特定的下游任务时,如果进行全量微调(Full Fine-Tuning),即对预训练模型中的所有参数都进行微调,这种方法太过低效。而如果采用固定预训练模型的某些层,只微调接近下游任务的那几层参数,又难以达到较好的效果。

\subsection{提示学习的优势}
提示学习(Prompting)旨在通过最小化微调参数的数量和计算复杂度,来提高预训练模型在新任务上的性能,从而缓解大型预训练模型的训练成本。这样一来,即使计算资源受限,也可以利用预训练模型的知识来迅速适应新任务,实现高效的迁移学习。

\section{提示学习基本概念}

\subsection{什么是提示学习?}
提示学习通过提供上下文和任务相关信息(即Prompt),帮助模型更好地理解要求并生成正确的输出。Prompt作为模型输入的引导信息,能够将下游任务重新表述为预训练任务的形式。

\subsection{提示学习应用实例}
\begin{itemize}
\item \textbf{问答任务}:Prompt可能包含问题或话题的描述,以帮助模型生成正确的答案
\item \textbf{情感分析任务}:在句子前面加入前缀"该句子的情感是",将情感分类任务转换为"填空"任务
\end{itemize}

在训练过程中,模型可以学习到Prompt与任务输出之间的关联。例如,可以学习到"该句子的情感是积极的"和"该句子的情感是消极的"之间的差异。

\section{提示学习的优点}

\begin{itemize}
\item \textbf{参数高效}:仅需微调少量参数,大幅降低计算资源需求
\item \textbf{训练快速}:收敛速度快,适合资源受限的场景
\item \textbf{知识保留}:保持预训练模型参数不变,避免灾难性遗忘
\item \textbf{多任务适配}:同一模型可适配多个下游任务
\item \textbf{迁移性强}:易于实现跨领域知识迁移
\end{itemize}

\section{提示学习方法综述}

\subsection{方法分类}
主要的提示学习方法包括:
\begin{itemize}
\item 前缀微调(Prefix-tuning)
\item 指示微调(Prompt-tuning) 
\item P-tuning
\item P-tuning v2
\end{itemize}

\section{前缀微调(Prefix-tuning)}

\subsection{为什么需要前缀微调?}
\begin{itemize}
\item \textbf{人工设计离散Prompts的缺点}:Prompts的变化对模型性能特别敏感,加一个词、少一个词或者变动位置都会造成较大变化
\item \textbf{自动化搜索离散Prompts的缺点}:成本较高,且离散化的token搜索结果可能不是最优的
\item \textbf{传统微调的问题}:对每个任务都要保存微调后的模型权重,耗时长且占用存储空间大
\end{itemize}

\subsection{前缀微调思路}
\begin{enumerate}
\item \textbf{Prefix构建}:在输入token之前构造一段任务相关的virtual tokens作为Prefix
\item \textbf{参数冻结}:训练时只更新Prefix部分的参数,而Transformer中的其他参数固定
\item \textbf{稳定化设计}:在Prefix层前面添加MLP结构,将Prefix分解为更小维度的Input与MLP的组合输出,防止直接更新Prefix参数导致训练不稳定
\end{enumerate}

\subsection{前缀微调优点}
\begin{itemize}
\item \textbf{vs 人工设计Prompts}:可以学习"隐式"的Prompts,而非依赖人工设计
\item \textbf{批量处理能力}:可以在一个批次中处理来自多个用户/任务的样本
\item \textbf{vs 全量微调}:只更新Prefix部分参数,大幅减少训练参数量
\end{itemize}

\subsection{前缀微调缺点}
\begin{itemize}
\item \textbf{序列长度占用}:占用序列长度,带来额外计算开销
\item \textbf{架构改动较大}:在每层都添加prompt参数,模型结构改动较大
\end{itemize}

\section{指示微调(Prompt-tuning)}

\subsection{为什么需要指示微调?}
\begin{itemize}
\item \textbf{全量微调问题}:对每个任务训练一个模型,开销和部署成本高
\item \textbf{离散prompts问题}:人工设计prompts成本高,效果不稳定
\item \textbf{前缀微调局限性}:占用序列长度,计算开销大,架构改动大
\end{itemize}

\subsection{指示微调思路}
\begin{enumerate}
\item \textbf{连续空间扩展}:将prompt扩展到连续空间,仅在输入层添加prompt连续向量
\item \textbf{参数优化}:通过反向传播更新参数学习prompts,而非人工设计
\item \textbf{模型冻结}:冻结模型原始权重,只训练prompts参数
\item \textbf{关联建模}:使用LSTM建模prompt向量间关联性
\end{enumerate}

\subsection{指示微调优点}
\begin{itemize}
\item \textbf{架构简洁}:只在输入层加入prompt tokens,无需添加MLP调整
\item \textbf{规模效应}:随着预训练模型参数量的增加,效果逼近全参数微调
\item \textbf{集成效率}:支持prompt ensembling,在同一批次训练不同prompt
\end{itemize}

\subsection{指示微调缺点}
\begin{itemize}
\item \textbf{训练难度}:训练不稳定,省显存但不一定省时间
\item \textbf{独立性假设}:多个prompt token之间相互独立,可能影响效果
\item \textbf{小模型局限}:在正常大小的预训练模型上表现不佳
\item \textbf{任务限制}:难以处理复杂的序列标注任务
\end{itemize}

\subsection{指示微调 vs 前缀微调}
\begin{table}[h]
\centering
\caption{指示微调与前缀微调对比}
\begin{tabular}{@{}lp{0.4\textwidth}p{0.4\textwidth}@{}}
\toprule
\textbf{特性} & \textbf{前缀微调(Prefix-tuning)} & \textbf{指示微调(Prompt-tuning)} \\
\midrule
适用任务 & 仅针对NLG任务有效,服务于GPT架构 & 考虑所有类型的语言模型 \\
添加方式 & 限定在输入前面添加 & 可以在任意位置添加 \\
参数添加 & 每一层都添加,保证效果 & 可以只在输入层添加 \\
架构复杂度 & 相对复杂,每层都修改 & 相对简单,主要在输入层 \\
\bottomrule
\end{tabular}
\end{table}

\subsection{指示微调 vs 全量微调}
\begin{table}[h]
\centering
\caption{指示微调与全量微调对比}
\begin{tabular}{@{}lp{0.45\textwidth}p{0.45\textwidth}@{}}
\toprule
\textbf{特性} & \textbf{全量微调(Fine-tuning)} & \textbf{指示微调(Prompt-tuning)} \\
\midrule
参数更新 & 需要改变预训练模型参数 & 不改变预训练模型参数 \\
遗忘问题 & 可能带来灾难性遗忘 & 保持预训练知识完整性 \\
训练成本 & 计算资源需求高 & 参数高效,资源需求低 \\
多任务支持 & 需要为每个任务存储完整模型 & 共享基础模型,存储效率高 \\
\bottomrule
\end{tabular}
\end{table}

\section{P-tuning方法}

\subsection{为什么需要P-tuning?}
\begin{itemize}
\item \textbf{Prompt敏感性}:大模型的Prompt构造方式严重影响下游任务效果
\item \textbf{GPT系列局限}:自回归建模在自然语言理解任务上效果不如BERT双向模型
\item \textbf{人工设计问题}:人工设计的模板对变化特别敏感
\item \textbf{自动化成本}:离散化token搜索成本高,结果可能非最优
\end{itemize}

\subsection{P-tuning思路}
\begin{enumerate}
\item \textbf{可学习Embedding层}:将Prompt转换为可学习的Embedding层
\item \textbf{Prompt编码器}:使用Prompt编码器(双向LSTM+两层MLP)对Prompt Embedding进行处理
\item \textbf{依赖建模}:建模伪token的相互依赖关系,提供更好的初始化
\end{enumerate}

\subsection{P-tuning优点}
\begin{itemize}
\item \textbf{依赖建模}:通过Prompt编码器建模token间依赖关系
\item \textbf{更好初始化}:提供更合理的参数初始化策略
\item \textbf{连续优化}:在连续空间优化prompt表示
\end{itemize}

\subsection{P-tuning缺点}
\begin{itemize}
\item \textbf{复杂性增加}:架构相对复杂,不太像传统prompt
\item \textbf{连续性不一致}:伪token编码连续但与输入结合时可能不连续
\item \textbf{插入问题}:中间可能会插入输入,影响连贯性
\end{itemize}

\subsection{P-tuning vs 传统微调}
\begin{lstlisting}[language=Python]
# 传统全量微调
class TraditionalFineTuning:
    def __init__(self, model):
        self.model = model
        # 所有参数可训练
        for param in self.model.parameters():
            param.requires_grad = True
    
    def train(self, data):
        # 更新所有参数
        return self.model.update_all_parameters(data)

# P-tuning微调
class PTuning:
    def __init__(self, model, prompt_length=10):
        self.model = model
        self.prompt_length = prompt_length
        # 只添加prompt相关参数
        self.prompt_embeddings = nn.Parameter(
            torch.randn(prompt_length, model.hidden_size)
        )
        # 冻结原模型参数
        for param in self.model.parameters():
            param.requires_grad = False
    
    def train(self, data):
        # 只更新prompt参数
        return self.update_prompt_parameters(data)
\end{lstlisting}

\section{P-tuning v2方法}

\subsection{为什么需要P-tuning v2?}
主要解决如何让Prompt Tuning在不同参数规模的预训练模型、针对不同下游任务的结果上都达到匹敌全量微调的效果。

\subsection{P-tuning v2思路}
\begin{enumerate}
\item \textbf{深度提示编码}:采用Prefix-tuning做法,在输入前面的每层加入可微调的Prompts tokens
\item \textbf{简化架构}:移除重参数化编码器(MLP/LSTM),提高训练速度和鲁棒性
\item \textbf{动态提示长度}:针对不同任务采用不同的提示长度
\item \textbf{多任务预训练}:先在多任务prompt上预训练,再适配下游任务
\item \textbf{传统分类范式}:抛弃verbalizer,回归到CLS和token label分类范式
\end{enumerate}

\subsection{P-tuning v2优点}
\begin{itemize}
\item \textbf{参数规模适中}:可学习参数从0.01\%增加到0.1\%-3\%,平衡效率与效果
\item \textbf{深度集成}:深层Prompt对预测产生更直接影响
\item \textbf{小模型适配}:解决Prompt Tuning在小模型上效果差的问题
\item \textbf{任务扩展}:可应用于NER等序列标注任务
\item \textbf{训练稳定}:通过多任务预训练缓解优化困难
\end{itemize}

\subsection{P-tuning v2缺点}
\begin{itemize}
\item \textbf{Prompt特性弱化}:抛弃verbalizer一定程度上弱化了prompt的原始特性
\item \textbf{架构复杂性}:深度集成增加了模型复杂度
\item \textbf{超参数敏感}:提示长度等超参数需要仔细调优
\end{itemize}

\subsection{P-tuning v2架构实现}
\begin{lstlisting}[language=Python]
class PTuningV2(nn.Module):
    def __init__(self, model, prompt_length=20, num_layers=12):
        super().__init__()
        self.model = model
        self.prompt_length = prompt_length
        self.num_layers = num_layers
        
        # 每层都添加prompt参数
        self.layer_prompts = nn.ParameterList([
            nn.Parameter(torch.randn(prompt_length, model.hidden_size))
            for _ in range(num_layers)
        ])
        
        # 冻结原始模型参数
        for param in self.model.parameters():
            param.requires_grad = False
    
    def forward(self, input_ids, attention_mask=None):
        batch_size = input_ids.shape[0]
        
        # 原始输入embedding
        input_embeds = self.model.embeddings(input_ids)
        
        # 为每层添加prompt
        hidden_states = input_embeds
        for i, layer in enumerate(self.model.encoder.layer):
            # 添加该层的prompt
            layer_prompt = self.layer_prompts[i].unsqueeze(0).expand(batch_size, -1, -1)
            prompt_length = layer_prompt.shape[1]
            
            # 拼接prompt和输入
            combined_embeds = torch.cat([layer_prompt, hidden_states], dim=1)
            
            # 调整attention mask
            if attention_mask is not None:
                prompt_mask = torch.ones(batch_size, prompt_length, 
                                        device=attention_mask.device)
                combined_mask = torch.cat([prompt_mask, attention_mask], dim=1)
            else:
                combined_mask = None
            
            # 通过transformer层
            layer_output = layer(combined_embeds, attention_mask=combined_mask)
            hidden_states = layer_output[0]
        
        return hidden_states
\end{lstlisting}

\section{方法对比与分析}

\subsection{技术演进路径}
\begin{table}[h]
\centering
\caption{提示学习技术演进对比}
\begin{tabular}{@{}lp{0.25\textwidth}p{0.25\textwidth}p{0.25\textwidth}p{0.25\textwidth}@{}}
\toprule
\textbf{特性} & \textbf{Prefix-tuning} & \textbf{Prompt-tuning} & \textbf{P-tuning} & \textbf{P-tuning v2} \\
\midrule
提出时间 & 2021 & 2021 & 2021 & 2022 \\
核心思想 & 每层添加可学习prefix & 输入层添加连续prompt & 引入prompt编码器 & 深度提示编码 \\
参数位置 & 每层Transformer & 仅输入层 & 输入层+编码器 & 每层Transformer \\
训练参数 & 0.1\%-1\% & 0.01\%-0.1\% & 0.1\%-1\% & 0.1\%-3\% \\
主要优势 & 深度集成效果 & 架构简单高效 & 依赖建模能力强 & 小模型效果好 \\
主要局限 & 计算开销大 & 小模型效果差 & 架构复杂 & 超参数敏感 \\
\bottomrule
\end{tabular}
\end{table}

\subsection{适用场景建议}
\begin{itemize}
\item \textbf{资源极度受限}:优先考虑Prompt-tuning,架构最简单
\item \textbf{效果优先}:选择P-tuning v2,效果最接近全量微调
\item \textbf{序列生成任务}:Prefix-tuning在NLG任务上表现优异
\item \textbf{快速实验}:Prompt-tuning实现快速,适合原型验证
\item \textbf{生产部署}:P-tuning v2平衡效果与效率,适合生产环境
\end{itemize}

\section{实践建议与最佳实践}

\subsection{参数配置建议}
\begin{lstlisting}[language=Python]
# 提示学习超参数配置示例
prompt_config = {
    'prompt_length': 20,      # 提示长度:10-50
    'learning_rate': 1e-4,    # 学习率:通常1e-5到1e-3
    'batch_size': 16,         # 批大小:根据显存调整
    'num_epochs': 10,         # 训练轮数:通常5-20
    'warmup_ratio': 0.1,      # 热身比例:0.1-0.2
}

# 不同模型规模的提示长度建议
model_size_prompt_config = {
    'small': {'prompt_length': 10, 'learning_rate': 1e-3},
    'base': {'prompt_length': 20, 'learning_rate': 5e-4},
    'large': {'prompt_length': 30, 'learning_rate': 1e-4},
    'xl': {'prompt_length': 50, 'learning_rate': 5e-5},
}
\end{lstlisting}

\subsection{训练技巧}
\begin{enumerate}
\item \textbf{渐进式训练}:先在小批量数据上训练,再扩展到全量数据
\item \textbf{学习率调度}:使用warmup和cosine衰减策略
\item \textbf{早停策略}:监控验证集性能,防止过拟合
\item \textbf{提示初始化}:使用任务相关词汇初始化prompt embedding
\item \textbf{多任务预训练}:先在相关任务上预训练prompt,再微调
\end{enumerate}

\section{挑战与未来方向}

\subsection{当前挑战}
\begin{itemize}
\item \textbf{训练稳定性}:提示学习方法训练过程可能不稳定
\item \textbf{超参数敏感}:对提示长度、学习率等超参数敏感
\item \textbf{理论理解}:缺乏对提示学习工作原理的深入理论分析
\item \textbf{多模态扩展}:如何扩展到视觉、语音等多模态任务
\item \textbf{推理效率}:提示添加可能增加推理延迟
\end{itemize}

\subsection{未来研究方向}
\begin{itemize}
\item \textbf{自动化提示学习}:自动学习最优提示结构和内容
\item \textbf{理论分析}:深入理解提示学习的表示学习机制
\item \textbf{多模态提示}:开发跨模态的提示学习方法
\item \textbf{动态提示}:根据输入动态调整提示内容
\item \textbf{提示压缩}:研究更紧凑的提示表示方法
\end{itemize}

\section{总结}

提示学习作为一种参数高效的微调方法,通过引入可学习的提示参数来引导预训练模型适应下游任务,在保持预训练知识的同时大幅减少训练成本。从Prefix-tuning到P-tuning v2,提示学习技术不断演进,在效果和效率之间寻求更好的平衡。

随着大模型技术的快速发展,提示学习将在资源受限的应用场景中发挥越来越重要的作用,为更广泛的人群提供使用大模型的能力。未来的研究将着重于提高方法的稳定性、扩展性和理论可解释性,推动提示学习技术在更多领域的应用。
