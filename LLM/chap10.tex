\chapter{多轮对话中让AI保持长期记忆的8种优化方式篇}

\section{前言}
在基于大模型的Agent中,长期记忆的状态维护至关重要。在OpenAI AI应用研究主管博客《基于大模型的Agent构成》中,将记忆视为关键的组件之一。下面将结合LangChain中的代码,介绍8种不同的记忆维护方式在不同场景中的应用。

\section{Agent获取上下文对话信息的8种方式}

\subsection{获取全量历史对话}
\textbf{应用场景:}以一般客服场景为例

在电信公司的客服聊天机器人场景中,如果用户在对话中先是询问了账单问题,接着又谈到了网络连接问题,ConversationBufferMemory可以用来记住整个与用户的对话历史,可以帮助AI在回答网络问题时还记得账单问题的相关细节,从而提供更连贯的服务。

\begin{lstlisting}[language=Python]
from langchain.memory import ConversationBufferMemory
memory = ConversationBufferMemory()
memory.save_context({"input": "你好"}, {"output": "怎么了"})
variables = memory.load_memory_variables({})
\end{lstlisting}

\subsection{滑动窗口获取最近部分对话内容}
\textbf{应用场景:}以商品咨询场景为例

在一个电商平台上,如果用户询问关于特定产品的问题(如手机的电池续航时间),然后又问到了配送方式,ConversationBufferWindowMemory可以帮助AI只专注于最近的一两个问题(如配送方式),而不是整个对话历史,以提供更快速和专注的答复。

\begin{lstlisting}[language=Python]
from langchain.memory import ConversationBufferWindowMemory
# 只保留最后1次互动的记忆
memory = ConversationBufferWindowMemory(k=1)
\end{lstlisting}

\subsection{获取历史对话中实体信息}
\textbf{应用场景:}以法律咨询场景为例

在法律咨询的场景中,客户可能会提到特定的案件名称、相关法律条款或个人信息(如"我在去年的交通事故中受了伤,想了解关于赔偿的法律建议")。ConversationEntityMemory可以帮助AI记住这些关键实体和实体关系细节,从而在整个对话过程中提供更准确、更个性化的法律建议。

\begin{lstlisting}[language=Python]
llm = ChatOpenAI(model="gpt-3.5-turbo", temperature=0)
memory = ConversationEntityMemory(llm=llm)
_input = {"input": "公众号《LLM应用全栈开发》的作者是莫尔索"}
memory.load_memory_variables(_input)
memory.save_context(
    _input,
    {"output": "是吗,这个公众号是干嘛的"}
)
print(memory.load_memory_variables({"input": "莫尔索是谁?"}))
# 输出,可以看到提取了实体关系
# {'history': 'Human: 公众号《LLM应用全栈开发》的作者是莫尔索\nAI: 是吗,这个公众号是干嘛的',
#  'entities': {'莫尔索': '《LLM应用全栈开发》的作者。'}}
\end{lstlisting}

\subsection{利用知识图谱获取历史对话中的实体及其联系}
\textbf{应用场景:}以医疗咨询场景为例

在医疗咨询中,一个病人可能会描述多个症状和过去的医疗历史(如"我有糖尿病史,最近觉得经常口渴和疲劳")。ConversationKGMemory可以构建一个包含病人症状、疾病历史和可能的健康关联的知识图谱,从而帮助AI提供更全面和深入的医疗建议。

\begin{lstlisting}[language=Python]
from langchain.memory import ConversationKGMemory
from langchain.llms import OpenAI
llm = OpenAI(temperature=0)
memory = ConversationKGMemory(llm=llm)
memory.save_context({"input": "小李是程序员"}, {"output": "知道了,小李是程序员"})
memory.save_context({"input": "莫尔索是小李的笔名"}, {"output": "明白,莫尔索是小李的笔名"})
variables = memory.load_memory_variables({"input": "告诉我关于小李的信息"})
print(variables)
# 输出
# {'history': 'On 小李: 小李 is 程序员. 小李 的笔名 莫尔索.'}
\end{lstlisting}

\subsection{对历史对话进行阶段性总结摘要}
\textbf{应用场景:}以教育辅导场景为例

在一系列的教育辅导对话中,学生可能会提出不同的数学问题或理解难题(如"我不太理解二次方程的求解方法")。ConversationSummaryMemory可以帮助AI总结之前的辅导内容和学生的疑问点,以便在随后的辅导中提供更针对性的解释和练习。

\subsection{需要获取最新对话,又要兼顾较早历史对话}
\textbf{应用场景:}以技术支持场景为例

在处理一个长期的技术问题时(如软件故障排查),用户可能会在多次对话中提供不同的错误信息和反馈。ConversationSummaryBufferMemory可以帮助AI保留最近几次交互的详细信息,同时提供历史问题处理的摘要,以便于更有效地识别和解决问题。

\subsection{回溯最近和最关键的对话信息}
\textbf{应用场景:}以金融咨询场景为例

在金融咨询聊天机器人中,客户可能会提出多个问题,涉及投资、市场动态或个人财务规划(如"我想了解股市最近的趋势以及如何分配我的投资组合")。ConversationTokenBufferMemory可以帮助AI聚焦于最近和最关键的几个问题,同时避免由于记忆过多而导致的信息混淆。

\subsection{基于向量检索对话信息}
\textbf{应用场景:}以了解最新新闻事件为例

用户可能会对特定新闻事件提出问题,如"最近的经济峰会有什么重要决策?" VectorStoreRetrieverMemory能够快速从大量历史新闻数据中检索出与当前问题最相关的信息,即使这些信息在整个对话历史中不是最新的,也能提供及时准确的背景信息和详细报道。

\begin{lstlisting}[language=Python]
vectorstore = Chroma(embedding_function=OpenAIEmbeddings())
retriever = vectorstore.as_retriever(search_kwargs=dict(k=1))
memory = VectorStoreRetrieverMemory(retriever=retriever)

memory.save_context({"input": "我喜欢吃火锅"}, {"output": "听起来很好吃"})
memory.save_context({"input": "我不喜欢看摔跤比赛"}, {"output": "我也是"})

PROMPT_TEMPLATE = """
以下是人类和AI之间的友好对话。AI话语多且提供了许多来自其上下文的具体细节。如果AI不知道问题的答案,它会诚实地说不知道。

以前对话的相关片段:
{history}
(如果不相关, 你不需要使用这些信息)

当前对话:
人类: {input}
AI:
"""

prompt = PromptTemplate(input_variables=["history", "input"], template=PROMPT_TEMPLATE)
conversation_with_summary = ConversationChain(
    llm=llm,
    prompt=prompt,
    memory=memory,
    verbose=True
)

print(conversation_with_summary.predict(input="你好,我是莫尔索,你叫什么"))
print(conversation_with_summary.predict(input="我喜欢的食物是什么?"))
print(conversation_with_summary.predict(input="我提到了哪些运动?"))
\end{lstlisting}

\section{总结}
这8种记忆优化方式各有其适用的场景和特点:

\begin{itemize}
\item \textbf{全量历史对话}:适用于需要完整上下文记忆的客服场景
\item \textbf{滑动窗口}:适用于关注最近对话的电商咨询场景
\item \textbf{实体信息提取}:适用于需要记忆关键实体的法律咨询场景
\item \textbf{知识图谱}:适用于复杂关系建模的医疗咨询场景
\item \textbf{阶段性总结}:适用于长期教育辅导场景
\item \textbf{摘要缓冲区}:适用于技术支持类长期问题跟踪
\item \textbf{令牌缓冲区}:适用于金融咨询等需要关注关键信息的场景
\item \textbf{向量检索}:适用于需要从大量历史数据中检索相关信息的新间查询场景
\end{itemize}

在实际应用中,可以根据具体的业务需求和对话特点选择合适的记忆策略,或者组合使用多种策略来达到最佳的记忆效果。


\chapter{基于LangChain RAG问答应用实战}

\section{前言}

\subsection{项目介绍}
本次选用百度百科藜藜麦数据(https://baike.baidu.com/item/藜藜麦/5843874)模拟人或企业私域数据,并基于LangChain开发框架,实现一种简单的RAG问答应用示例。

\subsection{软件资源}
\begin{itemize}
\item CUDA 11.7
\item Python 3.10
\item PyTorch 1.13.1+cu117
\item LangChain
\end{itemize}

\section{环境搭建}

\subsection{环境配置}
\begin{lstlisting}[language=bash]
# 创建新环境
$ conda create -n py310_chat python=3.10

# 激活环境
$ source activate py310_chat
\end{lstlisting}

\subsection{安装依赖}
\begin{lstlisting}[language=bash]
$ pip install datasets langchain sentence_transformers tqdm chromadb langchain_wenxin
\end{lstlisting}

\section{RAG问答应用实战}

\subsection{数据构建}
藜藜麦数据从百度百科获取并保存到藜藜.txt文件中。

\subsection{本地数据加载}
\begin{lstlisting}[language=Python]
from langchain.document_loaders import TextLoader

loader = TextLoader("./藜藜.txt")
documents = loader.load()
documents
\end{lstlisting}

\subsection{文档分割}
采用固定字符长度分割,chunk\_size=128

\begin{lstlisting}[language=Python]
# 文档分割
from langchain.text_splitter import CharacterTextSplitter

# 创建拆分器
text_splitter = CharacterTextSplitter(chunk_size=128, chunk_overlap=0)

# 拆分文档
documents = text_splitter.split_documents(documents)
documents
\end{lstlisting}

分割后的文档示例:
\begin{lstlisting}
[Document(page_content='藜藜(读音li)麦(Chenopodium quinoa Willd.)是藜藜科藜藜属植物...', 
          metadata={'source': './藜藜.txt'}),
 Document(page_content='藜藜麦是印第安人的传统主食,几乎和水稻同时被驯服有着6000多年的种植和食用历史...', 
          metadata={'source': './藜藜.txt'}),
 Document(page_content='繁殖\n地块选择:应选择地势较高、阳光充足、通风条件好及肥力较好的地块种植...', 
          metadata={'source': './藜藜.txt'})]
\end{lstlisting}

\subsection{向量化与数据入库}
选用m3e-base作为embedding模型,向量数据库选用Chroma

\begin{lstlisting}[language=Python]
from langchain.embeddings import HuggingFaceBgeEmbeddings
from langchain.vectorstores import Chroma

# embedding model: m3e-base
model_name = "moka-ai/m3e-base"
model_kwargs = {'device': 'cpu'}
encode_kwargs = {'normalize_embeddings': True}
embedding = HuggingFaceBgeEmbeddings(
    model_name=model_name,
    model_kwargs=model_kwargs,
    encode_kwargs=encode_kwargs,
    query_instruction="为文本生成向量表示用于文本检索"
)

# load data to Chroma db
db = Chroma.from_documents(documents, embedding)

# similarity search
db.similarity_search("藜藜一般在几月播种?")
\end{lstlisting}

\subsection{Prompt设计}
\begin{lstlisting}[language=Python]
template = '''
[任务描述]
请根据用户输入的上下文回答问题,并遵守回答要求。

[背景知识]
{{context}}

[回答要求]
- 你需要严格根据背景知识的内容回答,禁止根据常识和已知信息回答问题。
- 对于不知道的信息,直接回答"未找到相关答案"

{question}
'''
\end{lstlisting}

\subsection{RetrievalQAChain构建}
采用ConversationalRetrievalChain,提供历史聊天记录组件

\begin{lstlisting}[language=Python]
from langchain import LLMChain
from langchain_wenxin.llms import Wenxin
from langchain.prompts import PromptTemplate
from langchain.memory import ConversationBufferMemory
from langchain.chains import ConversationalRetrievalChain

# LLM选型
llm = Wenxin(model="ernie-bot", 
             baidu_api_key="baidu_api_key",
             baidu_secret_key="baidu_secret_key")

retriever = db.as_retriever()
memory = ConversationBufferMemory(memory_key="chat_history", return_messages=True)

qa = ConversationalRetrievalChain.from_llm(llm, retriever, memory=memory)
qa({"question": "藜藜怎么防治虫害?"})
\end{lstlisting}

运行结果:
\begin{lstlisting}
{'question': '藜藜怎么防治虫害?',
 'chat_history': [HumanMessage(content='藜藜怎么防治虫害?'),
                 AIMessage(content='藜藜麦常见虫害有象甲虫、金针虫、蝼蝼蛄蛄、黄条跳甲...')],
 'answer': '藜藜麦常见虫害有象甲虫、金针虫、蝼蝼蛄蛄、黄条跳甲、横纹菜蝽蝽...'}
\end{lstlisting}

\subsection{高级用法}
针对多轮对话场景,增加question\_generator对历史对话记录进行压缩生成新的question,增加combine\_docs\_chain对检索得到的文本进一步融合

\begin{lstlisting}[language=Python]
from langchain import LLMChain
from langchain.prompts import PromptTemplate
from langchain.memory import ConversationBufferMemory
from langchain.chains import ConversationalRetrievalChain, StuffDocumentsChain
from langchain.chains.qa_with_sources import load_qa_with_sources_chain
from langchain.prompts.chat import ChatPromptTemplate, SystemMessagePromptTemplate, HumanMessagePromptTemplate

# 构建初始messages列表
messages = [
    SystemMessagePromptTemplate.from_template(qa_template),
    HumanMessagePromptTemplate.from_template('{question}')
]

# 初始化prompt对象
prompt = ChatPromptTemplate.from_messages(messages)

llm_chain = LLMChain(llm=llm, prompt=prompt)

combine_docs_chain = StuffDocumentsChain(
    llm_chain=llm_chain,
    document_separator="\n\n",
    document_variable_name="context"
)

q_gen_chain = LLMChain(llm=llm)

qa = ConversationalRetrievalChain(
    combine_docs_chain=combine_docs_chain,
    question_generator=q_gen_chain,
    return_source_documents=True,
    return_generated_question=True,
    retriever=retriever
)

print(qa({'question': "藜藜麦怎么防治虫害?", "chat_history": []}))
\end{lstlisting}

高级用法运行结果:
\begin{lstlisting}
{'question': '藜藜怎么防治虫害?',
 'chat_history': [],
 'answer': '根据背景知识,藜藜麦常见虫害有象甲虫、金针虫、蝼蝼蛄蛄、黄条跳甲...',
 'source_documents': [Document(page_content='病害:主要防治叶斑病...', 
                              metadata={'source': './藜藜.txt'})],
 'generated_question': '藜藜怎么防治虫害?'}
\end{lstlisting}

\section{技术要点总结}

\subsection{核心组件}
\begin{itemize}
\item \textbf{文档加载器}:TextLoader用于加载本地文本文件
\item \textbf{文本分割器}:CharacterTextSplitter用于将长文本分割为小块
\item \textbf{嵌入模型}:HuggingFaceBgeEmbeddings用于生成文本向量表示
\item \textbf{向量数据库}:Chroma用于存储和检索向量数据
\item \textbf{对话链}:ConversationalRetrievalChain用于处理多轮对话
\end{itemize}

\subsection{优化建议}
\begin{itemize}
\item 根据具体业务场景调整chunk\_size和chunk\_overlap参数
\item 选择合适的embedding模型以获得更好的检索效果
\item 针对具体场景优化prompt模板
\item 考虑使用更复杂的内存管理策略处理长对话历史
\end{itemize}

\subsection{扩展应用}
\begin{itemize}
\item 可以扩展到处理PDF、Word等格式的文档
\item 可以集成多种向量数据库(如Pinecone、Weaviate等)
\item 可以结合多种LLM提供商(如OpenAI、Claude等)
\item 可以添加更复杂的检索策略(如混合检索、重排序等)
\end{itemize}

