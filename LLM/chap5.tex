\chapter{Transformers 操作篇}

\section{Transformers 库基础操作}

\subsection{如何利用 transformers 加载 Bert 模型?}

\begin{verbatim}
import torch 
from transformers import BertModel, BertTokenizer

# 这里我们调用bert-base模型,同时模型的词典经过小写处理
model_name = 'bert-base-uncased'

# 读取模型对应的tokenizer
tokenizer = BertTokenizer.from_pretrained(model_name)

# 载入模型
model = BertModel.from_pretrained(model_name)

# 输入文本
input_text = "Here is some text to encode"

# 通过tokenizer把文本变成token_id
input_ids = tokenizer.encode(input_text, add_special_tokens=True)
# input_ids: [101, 2182, 2003, 2070, 3793, 2000, 4372, 16044, 102]

input_ids = torch.tensor([input_ids])

# 获得BERT模型最后一个隐层结果
with torch.no_grad():
    last_hidden_states = model(input_ids)[0]
    # Models outputs are now tuples

"""
tensor([[[-0.0549,  0.1053, -0.1065, ..., -0.3550,  0.0686,  0.6506],
         [-0.5759, -0.3650, -0.1383, ..., -0.6782,  0.2092, -0.1639],
         [-0.1641, -0.5597,  0.0150, ..., -0.1603, -0.1346,  0.6216],
         [ 0.2448,  0.1254,  0.1587, ..., -0.2749, -0.1163,  0.8809],
         [ 0.0481,  0.4950, -0.2827, ..., -0.6097, -0.1212,  0.2527],
         [ 0.9046,  0.2137, -0.5897, ...,  0.3040, -0.6172, -0.1950]]])
shape: (1, 9, 768)
"""
\end{verbatim}

可以看到,包括import在内的不到十行代码,我们就实现了读取一个预训练过的BERT模型,来encode我们指定的一个文本,对文本的每一个token生成768维的向量。如果是二分类任务,我们接下来就可以把第一个token也就是[CLS]的768维向量,接一个linear层,预测出分类的logits,或者根据标签进行训练。

\subsection{如何利用 transformers 输出 Bert 指定 hidden\_state?}

Bert默认是十二层,但是有时候预训练时并不需要利用全部利用,而只需要预训练前面几层即可,此时该怎么做呢?

下载到bert-base-uncased的模型目录里面包含配置文件config.json,该文件中包含output\_hidden\_states,可以利用该参数来设置编码器内隐藏层层数。

\section{BERT 输出向量获取}

\subsection{BERT 模型输出结构}

BERT模型输出包含以下几个部分:

\begin{itemize}
\item \textbf{last\_hidden\_state}: shape是(batch\_size, sequence\_length, hidden\_size), hidden\_size=768, 它是模型最后一层输出的隐藏状态

\item \textbf{pooler\_output}: shape是(batch\_size, hidden\_size),这是序列的第一个token(classification token)的最后一层的隐藏状态,它是由线性层和Tanh激活函数进一步处理的,这个输出不是对输入的语义内容的一个很好的总结,对于整个输入序列的隐藏状态序列的平均化或池化通常更好。

\item \textbf{hidden\_states}: 这是输出的一个可选项,如果输出,需要指定config.output\_hidden\_states=True, 它也是一个元组,它的第一个元素是embedding,其余元素是各层的输出,每个元素的形状是(batch\_size, sequence\_length, hidden\_size)

\item \textbf{attentions}: 这也是输出的一个可选项,如果输出,需要指定config.output\_attentions=True, 它也是一个元组,它的元素是每一层的注意力权重,用于计算self-attention heads的加权平均值
\end{itemize}

\subsection{获取每一层网络的向量输出}

\begin{verbatim}
## 最后一层的所有token向量
outputs.last_hidden_state

## cls向量
outputs.pooler_output

## hidden_states, 包括13层, 第一层即索引0是输入embedding向量, 
## 后面1-12索引是每层的输出向量
hidden_states = outputs.hidden_states
embedding_output = hidden_states[0]
attention_hidden_states = hidden_states[1:]
\end{verbatim}