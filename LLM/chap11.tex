
\chapter{基于LLM+向量库的文档对话经验面}

\section{基础理论}

\subsection{为什么大模型需要外挂(向量)知识库?}
如何将外部知识注入大模型,最直接的方法:利用外部知识对大模型进行微调

\textbf{思路:}构建几十万量级的数据,然后利用这些数据对大模型进行微调,以将额外知识注入大模型

\textbf{优点:}简单粗暴

\textbf{缺点:}
\begin{itemize}
\item 这几十万量级的数据并不能很好的将额外知识注入大模型
\item 训练成本昂贵。不仅需要多卡并行,还需要训练很多天
\end{itemize}

既然大模型微调不是将外部知识注入大模型的最优方案,那是否有其它可行方案?

\subsection{基于LLM+向量库的文档对话思路}
\begin{enumerate}
\item 加载文件
\item 读取文本
\item 文本分割
\item 文本向量化
\item 问句向量化
\item 在文本向量中匹配出与问句向量最相似的topk个
\item 匹配出的文本作为上下文和问题一起添加到prompt中
\item 提交给LLM生成回答
\end{enumerate}

\subsection{核心技术:Embedding}
基于LLM+向量库的文档对话核心技术:embedding

\textbf{思路:}将用户知识库内容经过embedding存入向量知识库,然后用户每一次提问也会经过embedding,利用向量相关性算法(例如余弦算法)找到最匹配的几个知识库片段,将这些知识库片段作为上下文,与用户问题一起作为prompt提交给LLM回答

\subsection{Prompt模板构建}
\begin{verbatim}
已知信息:
{context}

根据上述已知信息,简洁和专业的来回答用户的问题。如果无法从中得到答案,请说
"根据已知信息无法回答该问题"或"没有提供足够的相关信息",不允许在答案中
添加编造成分,答案请使用中文。

问题是:{question}
\end{verbatim}

\section{优化问题与解决方案}

\subsection{痛点1:文档切分粒度不好把控}
\textbf{问题描述:}既担心噪声太多又担心语义信息丢失

\textbf{问题1:}如何让LLM简要、准确回答细粒度知识?

\textbf{问题2:}如何让LLM回答出全面的粗粒度(跨段落)知识?

\textbf{解决方案思想:}
基于LLM的文档对话架构分为两部分,先检索,后推理。重心在检索(推荐系统),推理交给LLM整合即可。

检索部分要满足三点:
\begin{itemize}
\item 尽可能提高召回率
\item 尽可能减少无关信息
\item 速度快
\end{itemize}

将所有的文本组织成二级索引,第一级索引是[关键信息],第二级是[原始文本],二者一一映射。

检索部分只对关键信息做embedding,参与相似度计算,把召回结果映射的原始文本交给LLM。

\subsubsection{如何构建关键信息?}
\textbf{文章的切分及关键信息抽取:}
\begin{itemize}
\item \textbf{关键信息:}为各语义段的关键信息集合,或者是各个子标题语义扩充之后的集合
\item \textbf{语义切分方法1:}利用NLP的篇章分析工具,提取出段落之间的主要关系
\item \textbf{语义切分方法2:}利用BERT等模型判断相邻段落相似度
\end{itemize}

\begin{lstlisting}[language=Python]
def is_nextsent(sent, next_sent):
    encoding = tokenizer(sent, next_sent, return_tensors="pt", 
                        truncation=True, padding=False)
    with torch.no_grad():
        outputs = model(**encoding, labels=torch.LongTensor([1]))
        logits = outputs.logits
        probs = torch.softmax(logits/TEMPERATURE, dim=1)
        next_sentence_prob = probs[:, 0].item()
        if next_sentence_prob <= MERGE_RATIO:
            return False
        else:
            return True
\end{lstlisting}

\textbf{语义段的切分及段落关键信息抽取:}
\begin{itemize}
\item \textbf{方法1:}利用成分句法分析工具和命名实体识别工具提取
\item \textbf{方法2:}用语义角色标注分析句子的谓词论元结构
\item \textbf{方法3:}关键词提取工具(HanLP、KeyBERT)
\item \textbf{方法4:}训练生成关键词的模型(如ChatLaw的KeyLLM)
\end{itemize}

\subsection{痛点2:在垂直领域表现不佳}
\textbf{解决方案:}模型微调
\begin{itemize}
\item 对embedding模型基于垂直领域的数据进行微调
\item 对LLM模型基于垂直领域的数据进行微调
\end{itemize}

\subsection{痛点3:LangChain内置问答分句效果不佳}
\textbf{文档加工方案:}
\begin{itemize}
\item 使用更好的文档拆分方式(如达摩院语义识别模型)
\item 改进填充方式,仅添加相关度高的上下文句子
\item 对每段分别进行总结,基于总结内容进行语义匹配
\end{itemize}

\subsection{痛点4:如何尽可能召回与query相关的Document}
\textbf{解决方法:}
\begin{itemize}
\item 优化Document的长度、embedding质量和召回数量之间的平衡
\item 使用Faiss搜索,基于本地知识对文本向量化工具进行Finetune
\item 将ES搜索结果与Faiss结果相结合
\end{itemize}

\subsection{痛点5:如何让LLM基于query和context得到高质量的response}
\textbf{解决方法:}
\begin{itemize}
\item 尝试多个prompt模板,选择最合适的
\item 用与本地知识问答相关的语料对LLM进行Finetune
\end{itemize}

\subsection{痛点6:Embedding模型在表示text chunks时偏差太大}
\textbf{问题描述:}
\begin{itemize}
\item 开源embedding模型效果一般,text chunk大时表示不准确
\item 多语言对齐问题(英文内容,中文query)
\end{itemize}

\textbf{解决方法:}
\begin{itemize}
\item 使用更小的text chunk配合更大的topk
\item 寻找更适合多语言的embedding模型
\end{itemize}

\subsection{痛点7:不同的prompt产生完全不同的效果}
\textbf{问题描述:}prompt的提法不同会产生完全不同的效果,特别是输出格式要求

\subsection{痛点8:LLM生成效果问题}
\textbf{问题描述:}不同LLM在理解context和生成环节表现差异大

\textbf{解决思路:}选择开源模型(如llama2、baichuan2),构造domain dataset进行微调

\subsection{痛点9:如何更高质量地召回context喂给LLM}
\textbf{问题描述:}召回内容与query相关性差

\textbf{解决思路:}更细颗粒度的recall,针对性的pdf解析

\section{工程实践与避坑指南}

\subsection{本地知识库问答系统(Langchain-chatGLM)}

\subsubsection{环境配置问题解决}
\begin{lstlisting}[language=bash]
# 解决持续网页loading问题
$ pip install gradio==3.21.0

# 解决detectron2安装问题
$ cd detectron2
$ pip install -e .
$ pip install torch==2.0.0
$ pip install protobuf==3.20.0
\end{lstlisting}

\subsubsection{PDF加载问题解决}
\begin{itemize}
\item 更新apt包:sudo apt update
\item 安装依赖:sudo apt install libmagic-dev poppler-utils tesseract-ocr
\item 配置中文识别包
\end{itemize}

\subsubsection{NLTK数据包问题解决}
\begin{itemize}
\item 手动解压punkt和tagger到指定目录
\item 通过nltk.data.path查询存储路径
\end{itemize}

\subsubsection{PaddleOCR错误解决}
\textbf{错误:}ModuleNotFoundError: No module named 'tools.infer'

\textbf{解决:}将所有from tools.infer import改为from paddleocr.tools.infer import

\subsubsection{MOSS模型加载错误解决}
\textbf{错误:}get\_class\_from\_dynamic\_module() missing 2 required positional arguments

\textbf{修改方案:}
\begin{lstlisting}[language=Python]
def auto_configure_device_map() -> Dict[str, int]:
    cls = get_class_from_dynamic_module(
        pretrained_model_name_or_path="fnlp/moss-moon-003-sft",
        module_file="modeling_moss.py", 
        class_name="MossForCausalLM"
    )
\end{lstlisting}

\subsubsection{MOSS提问错误解决}
\textbf{错误:}RuntimeError: probability tensor contains either inf, nan or element < 0

\textbf{解决:}移除do\_sample=True参数

\section{技术要点总结}

\subsection{核心架构设计}
\begin{itemize}
\item \textbf{二级索引系统:}关键信息索引 + 原始文本映射
\item \textbf{语义分割策略:}基于篇章分析和BERT相似度的混合方法
\item \textbf{检索优化:}平衡召回率、准确性和效率
\end{itemize}

\subsection{关键优化建议}
\begin{itemize}
\item \textbf{文档预处理:}采用语义级别的分割而非简单的格式分割
\item \textbf{Embedding选择:}根据语言和领域特点选择合适的模型
\item \textbf{Prompt工程:}针对具体任务设计合适的模板
\item \textbf{模型微调:}在垂直领域进行针对性的模型优化
\end{itemize}

\subsection{工程实践建议}
\begin{itemize}
\item \textbf{版本兼容性:}注意各组件版本匹配问题
\item \textbf{错误处理:}建立完善的错误监控和处理机制
\item \textbf{性能优化:}针对大规模文档建立分级索引系统
\item \textbf{多语言支持:}考虑跨语言检索和生成的需求
\end{itemize}

