\chapter{基础知识}
\section{大模型基础概念}

\textbf{大模型}:一般指1亿以上参数模型,但标准一直升级,目前已有万亿参数以上的模型。

\textbf{大语言模型(Large Language Model, LLM)}:针对语言的大模型。

参数规模:\textbf{175B、60B、540B等},这些一般指参数的个数,B是Billion/十亿的意思,175B是1750亿参数,这是ChatGPT大约的参数规模。

\section{单双向注意力}
\subsection{核心概念:“阅读”和“写作”}
理解单双向注意力是掌握大模型架构差异的关键。我们可以用两个生动的比喻来理解:

\textbf{双向注意力:像阅读侦探小说}:想象你在阅读一本悬疑小说。为了理解复杂的情节,你会\textbf{随意地前后翻看}。当看到最后一章揭示凶手时,你可能会翻回前面的章节,查看某个角色的不在场证明是否有漏洞。这就是"双向"的精髓——任何一个部分的信息都可以参考全文的任何其他部分,获得全局的、最充分的理解。

\textbf{单向注意力:像写作小说续集}:现在你要为这本小说写续集。你只能\textbf{从左到右一个一个词地写}。在写下"侦探"这个词时,你只能基于前面已经写好的"突然,门开了,走进来一位…"来构思,你\textbf{不能提前知道或使用}后面将要写出的"掏出了手枪"这个词。这就是"单向"或"因果"的本质——每个新词只能基于它之前的所有词来生成。

\subsection{技术深度解析}
双向注意力机制:
\begin{itemize}
\item \textbf{目标}:深度\textbf{理解}和\textbf{编码}输入信息
\item \textbf{工作原理}:模型同时处理整个句子的所有词。当理解某个词(如代词"它")时,可以\textbf{同时关注}该词左右两侧的所有上下文,从而准确判断指代关系
\item \textbf{优势}:文本理解能力极强,能把握复杂语义关系和指代消解
\item \textbf{局限}:不适合直接用于生成任务,否则会"作弊"(提前看到答案)
\item \textbf{典型代表}:BERT模型,主要用于文本分类、情感分析等理解型任务
\end{itemize}

单向注意力机制:
\begin{itemize}
\item \textbf{目标}:序列\textbf{生成}
\item \textbf{工作原理}:模型以自回归方式工作,每次只能基于当前和之前的词预测下一个词,严格遵循因果律
\item \textbf{优势}:天然适合文本生成任务,训练目标与实际应用完全一致,Zero-shot能力强,容易涌现新能力
\item \textbf{局限}:在纯理解任务上可能不如双向模型深入
\item \textbf{典型代表}:GPT系列、LLaMA系列
\end{itemize}

\subsection{单双向注意力对比总结}
\begin{table}[h]
\centering
\caption{单双向注意力机制对比}
\begin{tabular}{p{0.3\textwidth}p{0.3\textwidth}p{0.3\textwidth}}
\toprule
\textbf{特性} & \textbf{双向注意力} & \textbf{单向注意力(因果注意力)} \\
\midrule
\textbf{核心目标} & 理解与分析 & 生成与创作 \\
\textbf{信息流动} & 全局、无方向限制 & 从左到右、严格因果 \\
\textbf{形象比喻} & 阅读分析文章 & 写作口述文章 \\
\textbf{主要优势} & 深层语义理解、分类任务强 & 文本生成、零样本能力、涌现能力 \\
\textbf{主要局限} & 不直接适用于生成 & 理解任务可能缺少全局上下文 \\
\textbf{典型架构} & Encoder(编码器) & Decoder(解码器) \\
\textbf{代表模型} & BERT & GPT、LLaMA系列 \\
\bottomrule
\end{tabular}
\end{table}


\section{主流开源模型体系}
\subsection{三种主流体系}
\begin{itemize}
\item \textbf{Prefix Decoder系}
    \begin{itemize}
    \item 介绍:输入双向注意力,输出单向注意力
    \item 代表模型:ChatGLM、ChatGLM2、U-PaLM
    \end{itemize}

\item \textbf{Causal Decoder系}
    \begin{itemize}
    \item 介绍:从左到右的单向注意力
    \item 代表模型:LLaMA-7B、LLaMa衍生物
    \end{itemize}

\item \textbf{Encoder-Decoder系}
    \begin{itemize}
    \item 介绍:输入双向注意力,输出单向注意力
    \item 代表模型:T5、Flan-T5、BART y1y2
    \end{itemize}
\end{itemize}

\section{三种Decoder架构区别}
\subsection{核心区别}
主要区别在于attention mask不同:

\subsection{Encoder-Decoder架构}
\begin{itemize}
\item 在输入上采用双向注意力,对问题的编码理解更充分
\item 适用任务:在偏理解的NLP任务上效果好
\item 缺点:在长文本生成任务上效果差,训练效率低
\end{itemize}

\subsection{Causal Decoder架构}
\begin{itemize}
\item 自回归语言模型,预训练和下游应用是完全一致的,严格遵守只有后面的token才能看到前面的token的规则
\item 适用任务:文本生成任务效果好
\item 优点:训练效率高,zero-shot能力更强,具有涌现能力
\end{itemize}

\subsection{Prefix Decoder架构}
\begin{itemize}
\item 特点:prefix部分的token互相能看到,是Causal Decoder和Encoder-Decoder的折中
\item 缺点:训练效率低
\end{itemize}

\section{大模型训练目标}
\subsection{语言模型}
根据已有词预测下一个词,训练目标为最大似然函数:
\[
\mathcal{L}_{LM}(x)=\sum_{i=1}^{n}\log P(x_{i}|x_{<i})
\]
训练效率:Prefix Decoder $<$ Causal Decoder\\
Causal Decoder结构会在所有token上计算损失,而Prefix Decoder只会在输出上计算损失。

\subsection{去噪自编码器}
随机替换掉一些文本段,训练语言模型去恢复被打乱的文本段。目标函数为:
\[
\mathcal{L}_{DAE}(x)=\log P(\tilde{x}|x_{/\tilde{x}})
\]
去噪自编码器的实现难度更高。采用去噪自编码器作为训练目标的任务有GLM-130B、T5。

\section{涌现能力分析}
根据前人分析和论文总结,大致是2个猜想:
\begin{itemize}
\item 任务的评价指标不够平滑
\item 复杂任务vs子任务:假设某个任务T有5个子任务Sub-T构成,每个sub-T随着模型增长,指标从40\%提升到60\%,但是最终任务的指标只从1.1\%提升到了7\%,也就是说宏观上看到了涌现现象,但是子任务效果其实是平滑增长的
\end{itemize}

\section{Decoder Only架构优势}
\begin{itemize}
\item decoder-only结构模型在没有任何微调数据的情况下,zero-shot的表现能力最好
\item 而encoder-decoder则需要在一定量的标注数据上做multitask-finetuning才能够激发最佳性能
\item 目前的Large LM的训练范式还是在大规模语料上做自监督学习,zero-shot性能更好的decoder-only架构才能更好的利用这些无标注的数据
\item 大模型使用decoder-only架构除了训练效率和工程实现上的优势外,在理论上因为Encoder的双向注意力会存在低秩的问题,这可能会削弱模型的表达能力
\item 就生成任务而言,引入双向注意力并无实质的好处
\item Encoder-decoder模型架构之所以能够在某些场景下表现更好,大概是因为它多了一倍参数
\item 在同等参数量、同等推理成本下,Decoder-only架构就是最优的选择
\end{itemize}

\section{大模型优缺点分析}
\subsection{优点}
\begin{itemize}
\item 可以利用大量的无标注数据来训练一个通用的模型,然后再用少量的有标注数据来微调模型,以适应特定的任务。这种预训练和微调的方法可以减少数据标注的成本和时间,提高模型的泛化能力
\item 可以利用生成式人工智能技术来产生新颖和有价值的内容,例如图像、文本、音乐等。这种生成能力可以帮助用户在创意、娱乐、教育等领域获得更好的体验和效果
\item 可以利用涌现能力(Emergent Capabilities)来完成一些之前无法完成或者很难完成的任务,例如数学应用题、常识推理、符号操作等。这种涌现能力可以反映模型的智能水平和推理能力
\end{itemize}

\subsection{缺点}
\begin{itemize}
\item 需要消耗大量的计算资源和存储资源来训练和运行,这会增加经济和环境的负担。据估计,训练一个GPT-3模型需要消耗约30万美元,并产生约284吨二氧化碳排放
\item 需要面对数据质量和安全性的问题,例如数据偏见、数据泄露、数据滥用等。这些问题可能会导致模型产生不准确或不道德的输出,并影响用户或社会的利益
\item 需要考虑可解释性、可靠性、可持续性等方面的挑战,例如如何理解和控制模型的行为、如何保证模型的正确性和稳定性、如何平衡模型的效益和风险等。这些挑战需要多方面的研究和合作,以确保大模型能够健康地发展
\end{itemize}