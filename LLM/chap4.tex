\chapter{Attention 升级面}

传统 Attention 的问题:存在上下文长度约束问题,速度慢,内存占用大。

Attention 优化方向:提升上下文长度,加速、减少内存占用

\section{Attention 变体概述}
\begin{itemize}
\item \textbf{稀疏 Attention}:将稀疏偏差引入 attention 机制可以降低复杂性
\item \textbf{线性化 Attention}:解开 attention 矩阵与内核特征图,然后以相反的顺序计算 attention 以实现线性复杂度
\item \textbf{原型和内存压缩}:这类方法减少了查询或键值记忆对的数量,以减少注意力矩阵的大小
\item \textbf{低阶 self-Attention}:这一系列工作捕获了 self-Attention 的低阶属性
\item \textbf{Attention 与先验}:该研究探索了用先验 attention 分布来补充或替代标准 attention
\item \textbf{改进多头机制}:该系列研究探索了不同的替代多头机制
\end{itemize}

\section{Multi-Query Attention}

Multi-head Attention 存在的问题:
\begin{itemize}
\item \textbf{训练过程}:不会显著影响训练过程,训练速度不变,会引起非常细微的模型效果损失
\item \textbf{推理过程}:反复加载巨大的 KV cache,导致内存开销大,性能是内存受限
\end{itemize}

Multi-head Attention 与 Multi-Query Attention 对比:
\begin{itemize}
\item \textbf{Multi-head Attention}:每个注意力头都有各自的 query、key 和 value
\item \textbf{Multi-query Attention}:在所有的注意力头上共享 key 和 value
\end{itemize}

\begin{table}[h]
\centering
\caption{各模型注意力机制参数配置对比}
\begin{tabular}{@{}lllll@{}}
\toprule
\textbf{模型} & \textbf{n\_heads} & \textbf{head\_dim} & \textbf{FFN中间维度} & \textbf{维度 h} \\
\midrule
LLaMA & 32 & 128 & 11008 & 4096 \\
baichuan & 32 & 128 & 11008 & 4096 \\
ChatGLM-6B & 32 & 128 & 4h, 16384 & 4096 \\
ChatGLM2-6B & 32 & 128 & 13696 & 4096 \\
Bloom & 32 & 128 & 4h, 16384 & 4096 \\
Falcon & 71 & 64 & 4h, 18176 & 4544 \\
\bottomrule
\end{tabular}
\end{table}

\subsection{Multi-Query Attention 的模型实现差异}
Falcon、PaLM、ChatGLM2-6B 都使用了 Multi-query Attention,但有细微差别:
\begin{itemize}
\item \textbf{为了保持参数量一致}:
\item \textbf{Falcon}:把隐藏维度从 4096 增大到了 4544。多余的参数量分给了 Attention 块和 FFN 块
\item \textbf{ChatGLM2}:把 FFN 中间维度从 11008 增大到了 13696。多余的参数分给了 FFN 块
\end{itemize}

Multi-Query Attention 的优势:减少 KV cache 的大小,减少显存占用,提升推理速度。使用 Multi-Query Attention 的模型:PaLM、ChatGLM2、Falcon 等。

\section{Grouped-query Attention}

Grouped query attention:介于 multi head 和 multi query 之间,多个 key 和 value。ChatGLM2,LLaMA2-34B/70B 使用了 Grouped query attention。

\section{FlashAttention}

FlashAttention 核心技术:
\begin{itemize}
\item \textbf{核心}:用分块 softmax 等价替代传统 softmax
\item \textbf{关键词}:HBM、SRAM、分块 Softmax、重计算、Kernel 融合
\end{itemize}

FlashAttention 优点:节约 HBM,高效利用 SRAM,省显存,提速度

使用 FlashAttention 的模型:Meta 推出的开源大模型 LLaMA,阿联酋推出的开源大模型 Falcon 都使用了 Flash Attention 来加速计算和节省显存

\section{并行 Transformer Block}

原理:用并行公式替换了串行,提升了 15\% 的训练速度。Falcon、PaLM 都使用了该技术来加速训练。
\begin{itemize}
\item 在 8B 参数量规模,会有轻微的模型效果损失
\item 在 62B 参数量规模,就不会损失模型效果
\end{itemize}