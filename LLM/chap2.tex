\chapter{Layer Normalization 篇}

\section{Layer Norm 基础}

\subsection{Layer Norm 计算公式}
Layer Normalization 的计算公式如下:

\begin{align*}
\mu &= E(X) = \frac{1}{H}\sum_{i=1}^{H}x_{i} \\
\sigma &= \sqrt{Var(x)} = \sqrt{\frac{1}{H}\sum_{i=1}^{H}(x_{i}-\mu)^{2}+\epsilon} \\
y &= \frac{x-E(x)}{\sqrt{Var(X)+\epsilon}}\cdot\gamma+\beta
\end{align*}

其中:
\begin{itemize}
\item $\gamma$: 可训练的再缩放参数
\item $\beta$: 可训练的再偏移参数
\end{itemize}

\section{RMS Norm(均方根 Norm)}

\subsection{RMS Norm 计算公式}
RMS Norm 的计算公式如下:

\begin{align*}
RMS(x) &= \sqrt{\frac{1}{H}\sum_{i=1}^{H}x_{i}^{2}} \\
x &= \frac{x}{RMS(x)}\cdot\gamma
\end{align*}

\subsection{RMS Norm 的特点}
\begin{itemize}
\item RMS Norm 简化了 Layer Norm,去除掉计算均值进行平移的部分
\item 对比 LN,RMS Norm 的计算速度更快
\item 效果基本相当,甚至略有提升
\end{itemize}

\section{Deep Norm}

\subsection{Deep Norm 思路}
Deep Norm 方法在执行 Layer Norm 之前,up-scale 了残差连接 ($\alpha>1$);另外,在初始化阶段 down-scale 了模型参数 ($\beta<1$)。

\subsection{Deep Norm 代码实现}
\begin{verbatim}
def deepnorm(x):
    return LayerNorm(x*α + f(x))

def deepnorm_init(w):
    if w in ['ffn', 'v_proj', 'out_proj']:
        nn.init.xavier_normal_(w, gain=β)
    elif w in ['q_proj', 'k_proj']:
        nn.init.xavier_normal_(w, gain=1)
\end{verbatim}

\subsection{Deep Norm 的优点}
Deep Norm 可以缓解爆炸式模型更新的问题,把模型更新限制在常数,使得模型训练过程更稳定。

\section{Layer Normalization 的位置设计}

\subsection{LN 在 LLMs 中的不同位置}

\subsubsection{Post LN}
\begin{itemize}
\item \textbf{位置}: Layer Norm 在残差链接之后
\item \textbf{缺点}: Post LN 在深层的梯度范式逐渐增大,导致使用 post-LN 的深层 transformer 容易出现训练不稳定的问题
\end{itemize}

\subsubsection{Pre-LN}
\begin{itemize}
\item \textbf{位置}: Layer Norm 在残差链接中
\item \textbf{优点}: 相比于 Post-LN,Pre LN 在深层的梯度范式近似相等,所以使用 Pre-LN 的深层 transformer 训练更稳定,可以缓解训练不稳定问题
\item \textbf{缺点}: 相比于 Post-LN,Pre-LN 的模型效果略差
\end{itemize}

\subsubsection{Sandwich-LN}
\begin{itemize}
\item \textbf{位置}: 在 pre-LN 的基础上,额外插入了一个 layer norm
\item \textbf{优点}: Cogview 用来避免值爆炸的问题
\item \textbf{缺点}: 训练不稳定,可能会导致训练崩溃
\end{itemize}

\section{Layer Normalization 对比分析}

\subsection{各模型使用的 Normalization 方法}

\begin{table}[h]
\centering
\caption{LLMs 各模型使用的 Layer Normalization 方法对比}
\begin{tabular}{@{}ll@{}}
\toprule
\textbf{模型} & \textbf{Normalization 方法} \\
\midrule
GPT3 & Pre Layer Norm \\
LLaMA & Pre RMS Norm \\
baichuan & Pre RMS Norm \\
ChatGLM-6B & Post Deep Norm \\
ChatGLM2-6B & Post RMS Norm \\
Bloom & Pre Layer Norm \\
Falcon & Pre Layer Norm \\
\bottomrule
\end{tabular}
\end{table}

\subsection{特殊说明}
BLOOM 在 embedding 层后添加 layer normalization,有利于提升训练稳定性,但可能会带来很大的性能损失。