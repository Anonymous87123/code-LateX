
\chapter{大模型(LLMs) RAG优化策略RAG-Fusion篇}

\section{RAG技术概述}

\subsection{RAG的优点}
\begin{enumerate}
\item \textbf{向量搜索融合}:RAG通过将向量搜索功能与生成模型相结合,引入了一种新颖的范式。这种融合使大型语言模型(LLM)能够生成更丰富、更具上下文意识的输出。

\item \textbf{减少幻觉现象}:RAG显著降低了LLM产生幻觉的倾向,使生成的文本更加基于数据。

\item \textbf{个人和专业效用}:从个人应用(如浏览笔记)到更专业的集成,RAG在提高生产力和内容质量方面展示了其多功能性,同时基于可信的数据来源。
\end{enumerate}

\subsection{RAG的局限性}
\begin{enumerate}
\item \textbf{当前搜索技术的限制}:RAG受到限制的方面与我们的检索式基于词汇和向量的搜索技术相同。

\item \textbf{人类搜索效率低下}:人类在向搜索系统输入他们想要的内容时并不擅长,如打字错误、含糊的查询或词汇有限,这常常导致错过那些超出显而易见的顶部搜索结果的大量信息。虽然RAG有所帮助,但它并没有完全解决这个问题。

\item \textbf{搜索的过度简化}:我们普遍的搜索范式是将查询线性映射到答案,缺乏理解人类查询的多维性。这种线性模型通常无法捕捉更复杂用户查询的细微差别和上下文,导致结果相关性较低。
\end{enumerate}

\section{RAG-Fusion技术概述}

\subsection{为什么需要RAG-Fusion?}
RAG-Fusion旨在解决RAG固有的限制,通过生成多个用户查询并重新排序结果。利用逆向排名融合和自定义向量评分加权进行综合、准确的搜索。

RAG-Fusion旨在弥合用户明确询问与他们意图询问之间的差距,更接近于发现通常隐藏的变革性知识。

\subsection{RAG-Fusion核心技术}
RAG-Fusion的基础三元组与RAG相似,核心技术包括:
\begin{itemize}
\item \textbf{通用编程语言}:通常是Python
\item \textbf{专用向量搜索数据库}:如Elasticsearch或Pinecone,用于驱动文档检索
\item \textbf{强大的大型语言模型}:如ChatGPT,用于创造文本
\end{itemize}

与RAG不同的是,RAG-Fusion通过几个额外的步骤区分自己:查询生成和结果重新排序。

\section{RAG-Fusion工作流程}

\subsection{多查询生成}

\subsubsection{为什么要生成多个查询?}
在传统的搜索系统中,用户通常输入一个查询来查找信息。虽然这种方法直接简单,但它有局限性。单一查询可能无法完全捕捉用户感兴趣的全部范围,或者可能过于狭窄而无法产生全面的结果。因此,从不同角度生成多个查询就显得尤为重要。

\subsubsection{多查询生成技术实现(提示工程)}
利用提示工程和自然语言模型拓宽搜索视野,提升结果质量。利用提示工程生成多个查询至关重要,这些查询不仅与原始查询相似,还提供不同的视角或角度。

\begin{lstlisting}[language=Python]
def generate_queries(original_query, num_queries=5):
    """
    基于原始查询生成多个相关查询
    """
    system_message = """
    你是一个AI助手,负责从不同角度生成搜索查询。
    请基于用户提供的原始查询,生成{num_queries}个相关的搜索查询。
    这些查询应该:
    1. 与原始查询语义相关但角度不同
    2. 涵盖原始查询的不同方面
    3. 具有一定的多样性和覆盖面
    """
    
    prompt = f"""
    原始查询:{original_query}
    请生成{num_queries}个相关的搜索查询:
    """
    
    # 调用LLM生成多个查询
    response = llm.chat_complete(
        system=system_message,
        messages=[{"role": "user", "content": prompt}]
    )
    
    # 解析生成的查询
    generated_queries = parse_generated_queries(response)
    return [original_query] + generated_queries  # 包含原始查询
\end{lstlisting}

\subsubsection{多查询生成工作原理}
\begin{enumerate}
\item \textbf{调用语言模型}:该函数调用一个语言模型(如chatGPT)。该方法期望一个特定的指令集,通常描述为"系统消息",以指导模型。例如,系统消息指导模型充当"AI助手"。

\item \textbf{自然语言查询}:模型基于原始查询生成多个查询。

\item \textbf{多样性和覆盖范围}:这些查询不是随机变化,而是经过精心生成的,以提供原始问题的不同视角。例如,如果原始查询是关于"气候变化的影响",那么生成的查询可能包括"气候变化的经济后果"、"气候变化与公共健康"等角度。这种方法确保了搜索过程考虑了更广泛的信息范围,从而提高生成总结的质量和深度。
\end{enumerate}

\subsection{逆向排名融合(RRF)}

\subsubsection{为什么选择RRF?}
逆向排名融合(RRF)是一种将多个搜索结果列表的排名结合起来产生单一统一排名的技术。该技术由滑铁卢大学(加拿大)和谷歌合作开发,根据其作者的说法,"产生的结果比任何单个系统更好,也比标准重新排名方法更好"。

RRF算法的数学表达式为:
$$RRFscore(d \in D) = \sum_{r \in R} \frac{1}{k + r(d)}$$

其中$k=60$是一个常数,$r(d)$是文档$d$在排名$r$中的位置。

通过结合不同查询的排名,我们增加了最相关文档出现在最终列表顶部的机会。RRF特别有效,因为它不依赖于搜索引擎分配的绝对分数,而是依赖于相对排名,使其非常适合结合可能具有不同规模或分数分布的查询结果。

通常情况下,RRF被用于混合词汇和向量结果。虽然这种方法有助于弥补向量搜索在查找特定术语(例如缩写)时的不足,但对结果并不印象深刻,这些结果往往更像是多个结果集的拼凑,因为同一个查询的词汇和向量搜索很少出现相同的结果。

可以将RRF看作是那种坚持在做决定前听取每个人意见的人。只不过在这种情况下,它不仅不烦人,而且有帮助。众多观点越多,结果越准确。

\subsubsection{RRF技术实现}
运用RRF根据多组搜索结果的位置重新排序文档。逆向排名融合位置重新排序系统:

\begin{enumerate}
\item 函数reciprocal\_rank\_fusion接收一个搜索结果的字典,其中每个键是一个查询,相应的值是根据该查询的相关性排名的文档ID列表。

\item RRF算法然后基于其在不同列表中的排名为每个文档计算一个新分数,并根据这些分数排序以创建最终的重新排名列表。

\item 计算完融合分数后,函数按照这些分数的降序对文档进行排序,以获得最终的重新排名列表,然后返回该列表。
\end{enumerate}

\begin{lstlisting}[language=Python]
def reciprocal_rank_fusion(search_results_dict, k=60):
    """
    实现逆向排名融合算法
    search_results_dict: 字典,键为查询,值为文档ID排名列表
    k: 平滑常数,通常设为60
    """
    # 初始化文档得分字典
    doc_scores = {}
    
    # 对每个查询的排名结果进行处理
    for query, ranked_docs in search_results_dict.items():
        for rank, doc_id in enumerate(ranked_docs):
            if doc_id not in doc_scores:
                doc_scores[doc_id] = 0
            # 计算RRF分数:1/(k + rank)
            doc_scores[doc_id] += 1.0 / (k + rank + 1)
    
    # 按分数降序排序
    fused_ranking = sorted(doc_scores.items(), 
                          key=lambda x: x[1], reverse=True)
    
    return fused_ranking

# 示例使用
search_results = {
    "query1": ["doc1", "doc3", "doc2", "doc5"],
    "query2": ["doc2", "doc1", "doc4", "doc3"], 
    "query3": ["doc3", "doc1", "doc5", "doc2"]
}

final_ranking = reciprocal_rank_fusion(search_results)
print("融合后的排名:", final_ranking)
\end{lstlisting}

\subsubsection{生成性输出用户意图保留}
使用多个查询的一个挑战是可能稀释用户的原始意图。为了缓解这一点,我们指示模型在提示工程中更重视原始查询。

\subsubsection{生成性输出用户意图保留技术实现}
最后,将重新排名的文档和所有查询输入到LLM提示中,以生成典型的RAG方式的生成性输出,如请求回应或摘要。

\begin{lstlisting}[language=Python]
def generate_final_response(original_query, all_queries, reranked_docs):
    """
    基于重新排名的文档生成最终响应
    """
    # 构建包含原始查询优先的提示
    prompt = f"""
    基于以下文档内容,回答用户的原始问题。
    
    原始问题:{original_query}
    相关查询:{', '.join(all_queries[1:])}
    
    相关文档内容:
    {chr(10).join([doc['content'] for doc in reranked_docs[:5]])}
    
    请优先考虑原始问题的意图,提供准确、全面的回答。
    """
    
    response = llm.generate(prompt)
    return response
\end{lstlisting}

通过将这些技术和技巧层叠起来,RAG-Fusion提供了一种强大而细腻的文本生成方法。它利用搜索技术和生成性人工智能的最佳特性,产生高质量、可靠的输出。

\section{RAG-Fusion的优势和挑战}

\subsection{RAG-Fusion优势}
\begin{enumerate}
\item \textbf{更优质的源材料}:使用RAG-Fusion时,搜索深度不仅仅是"增强"而是被放大。重新排名的相关文档列表意味着不只是在信息表面刮刮而已,而是潜入观点的海洋。结构化输出易于阅读,直观上可信赖,这在对人工智能生成内容持怀疑态度的世界中至关重要。

\item \textbf{增强用户意图对齐}:RAG-Fusion的核心设计是作为一个富有同情心的人工智能,揭示用户努力表达但可能无法清晰表述的内容。采用多查询策略捕捉用户信息需求的多面性表现,因此提供全面的输出,并与用户意图产生共鸣。

\item \textbf{结构化、富有洞见的输出}:通过汲取多样化的信息源,模型制作出组织良好且富有洞见的答案,预测后续问题并主动解答。

\item \textbf{自动纠正用户查询}:该系统不仅解释,还优化用户查询。通过生成多个查询变体,RAG-Fusion执行隐含的拼写和语法检查,从而提高搜索结果的准确性。

\item \textbf{处理复杂查询}:人类语言在表达复杂或专业思想时常常出现障碍。该系统作为语言催化剂,生成可能包含所需专业术语或术语的变体,用于更集中和相关的搜索结果。它还可以将更长、更复杂的查询分解成向量搜索可以处理的更小、更易管理的部分。

\item \textbf{搜索中的意外发现}:考虑"未知的未知"直到遇到才知道需要的信息。通过采用更广泛的查询范围,系统促进了发现意外信息的可能性。虽然这些信息并非明确寻求,但对用户来说却可能是一个"恍然大悟"的时刻。这使RAG-Fusion区别于其他传统搜索模型。
\end{enumerate}

\subsection{RAG-Fusion挑战}
\begin{enumerate}
\item \textbf{过于冗长的风险}:RAG-Fusion的深度有时可能导致信息泛滥。输出可能过于详细,令人不堪重负。可以将RAG-Fusion比作那个解释过多的朋友——信息丰富,但有时可能需要更直接了当。

\item \textbf{平衡上下文窗口}:多查询输入和多样化文档集的引入可能会使语言模型的上下文窗口受到压力。想象一个舞台上挤满了演员,使得剧情难以跟进。对于上下文限制较紧的模型,这可能导致输出不连贯甚至被截断。

\item \textbf{伦理和用户体验考虑}:拥有巨大力量的同时也伴随着巨大的责任。对于RAG-Fusion来说,操作用户查询以改善结果的能力似乎正踏入某种道德灰区。在改善搜索结果的同时平衡用户意图的完整性至关重要。
\end{enumerate}

\subsubsection{伦理和用户体验具体考虑}
\begin{itemize}
\item \textbf{伦理顾虑}:
\begin{itemize}
\item \textbf{用户自主性}:操作用户查询有时可能偏离原始意图。考虑向人工智能让渡多少控制权以及代价是什么非常重要。
\item \textbf{透明度}:不仅仅是关于更好的结果;如果用户的查询被调整,他们应当意识到这一点。这种透明度对于维护信任和尊重用户意图至关重要。
\end{itemize}

\item \textbf{用户体验(UX)增强}:
\begin{itemize}
\item \textbf{保留原始查询}:RAG-Fusion优先考虑初始用户查询,确保其在生成过程中的重要性。这作为防止误解的保障。
\item \textbf{过程可见性}:展示生成的查询以及最终结果,为用户提供搜索范围和深度的透明视图。这有助于建立信任和理解。
\end{itemize}

\item \textbf{UX/UI实施建议}:
\begin{itemize}
\item \textbf{用户控制}:提供用户切换RAG-Fusion的选项,允许他们在手动控制和增强的人工智能辅助之间选择。
\item \textbf{指导和清晰度}:关于RAG-Fusion工作方式的工具提示或简要说明可以帮助设定明确的用户期望。
\end{itemize}
\end{itemize}

\section{完整RAG-Fusion实现示例}

\begin{lstlisting}[language=Python]
class RAGFusionSystem:
    def __init__(self, vector_db, llm, k=60):
        self.vector_db = vector_db
        self.llm = llm
        self.k = k  # RRF常数
    
    def generate_queries(self, original_query, num_queries=5):
        """生成多个相关查询"""
        prompt = f"""
        基于以下原始查询,生成{num_queries}个相关的搜索查询:
        原始查询:{original_query}
        
        要求:
        1. 每个查询应该从不同角度探讨原始主题
        2. 保持语义相关性但提供多样性
        3. 每个查询不超过15个词
        """
        
        response = self.llm.generate(prompt)
        queries = self._parse_generated_queries(response)
        return [original_query] + queries
    
    def search_all_queries(self, queries, top_k=10):
        """对每个查询执行搜索"""
        all_results = {}
        
        for query in queries:
            results = self.vector_db.similarity_search(query, k=top_k)
            all_results[query] = [doc.doc_id for doc in results]
        
        return all_results
    
    def reciprocal_rank_fusion(self, search_results):
        """执行逆向排名融合"""
        doc_scores = {}
        
        for query, ranked_docs in search_results.items():
            for rank, doc_id in enumerate(ranked_docs):
                if doc_id not in doc_scores:
                    doc_scores[doc_id] = 0
                doc_scores[doc_id] += 1.0 / (self.k + rank + 1)
        
        # 按分数排序
        fused_ranking = sorted(doc_scores.items(), 
                              key=lambda x: x[1], reverse=True)
        return fused_ranking
    
    def retrieve_documents(self, fused_ranking, top_n=5):
        """根据融合排名检索实际文档内容"""
        doc_ids = [doc_id for doc_id, score in fused_ranking[:top_n]]
        documents = self.vector_db.get_documents(doc_ids)
        return documents
    
    def generate_final_response(self, original_query, queries, documents):
        """生成最终响应"""
        context = "\n\n".join([doc.content for doc in documents])
        
        prompt = f"""
        基于以下上下文信息,回答用户的原始问题。
        
        原始问题:{original_query}
        生成的辅助查询:{', '.join(queries[1:])}
        
        上下文信息:
        {context}
        
        请提供:
        1. 直接回答原始问题
        2. 涵盖相关的重要方面
        3. 保持回答的准确性和全面性
        4. 如果信息不足,请明确指出
        """
        
        response = self.llm.generate(prompt)
        return response
    
    def query(self, original_query, num_queries=5, top_k=10, top_n=5):
        """完整的RAG-Fusion查询流程"""
        # 1. 生成多个查询
        queries = self.generate_queries(original_query, num_queries)
        
        # 2. 对每个查询执行搜索
        search_results = self.search_all_queries(queries, top_k)
        
        # 3. 逆向排名融合
        fused_ranking = self.reciprocal_rank_fusion(search_results)
        
        # 4. 检索文档内容
        documents = self.retrieve_documents(fused_ranking, top_n)
        
        # 5. 生成最终响应
        response = self.generate_final_response(original_query, queries, documents)
        
        return {
            'original_query': original_query,
            'generated_queries': queries,
            'fused_ranking': fused_ranking,
            'documents': documents,
            'response': response
        }

# 使用示例
rag_fusion = RAGFusionSystem(vector_db=pinecone_index, llm=chatgpt)
result = rag_fusion.query("气候变化对全球经济的影响")
print("最终回答:", result['response'])
print("生成的查询:", result['generated_queries'])
print("使用的文档数量:", len(result['documents']))
\end{lstlisting}

\section{总结}

RAG-Fusion通过引入多查询生成和逆向排名融合技术,显著提升了传统RAG系统的性能。其主要优势在于能够从多个角度理解用户查询意图,通过融合不同视角的搜索结果提供更全面、准确的回答。

关键技术特点包括:
\begin{itemize}
\item \textbf{多视角查询生成}:通过LLM生成多个相关但角度不同的查询
\item \textbf{逆向排名融合}:智能融合不同查询的搜索结果
\item \textbf{用户意图保留}:在增强搜索的同时保持对原始查询的忠实
\item \textbf{意外发现能力}:通过广泛搜索范围促进意外有价值信息的发现
\end{itemize}

尽管存在计算复杂度增加和伦理考量等挑战,但通过合理的工程实现和用户体验设计,RAG-Fusion为构建更智能、更全面的检索增强生成系统提供了有力的技术路径。
