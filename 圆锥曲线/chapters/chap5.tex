\chapter{抹茶题集}
\begin{example}{抹茶T1}{}
    椭圆 $\frac{x^2}{4}+y^2=m(m>1)$ 上3点 $A,B,P(0,1)$满足$\overrightarrow{AP}=2\overrightarrow{PB}$,求 $|x_B|$ 的最大值
\end{example}
\begin{solution}
由于我们将证明:一旦 $m$ 固定,符合题意的弦最多只有一对对称解,$|x_B|$ 此时是一个确定的定值。因此“求最大值”必然是指 随着参数 $m$ 在允许范围 $(m>1)$ 内变化时,绝对值 $|x_B|$ 能达到的全局上界。设点 $A(x_A, y_A)$,点 $B(x_B, y_B)$。已知定点 $P(0, 1)$。有
$$ \overrightarrow{AP} = (0 - x_A, 1 - y_A) = (-x_A, 1 - y_A), \overrightarrow{PB} = (x_B - 0, y_B - 1) = (x_B, y_B - 1) $$
由于 $\overrightarrow{AP} = 2\overrightarrow{PB}$,代入条件:$$ \begin{cases} -x_A = 2x_B \\ 1 - y_A = 2(y_B - 1) \end{cases} \implies \begin{cases} \mathbf{x_A = -2x_B} \\ \mathbf{y_A = 3 - 2y_B} \end{cases} $$
已知 $A$ 和 $B$ 都在椭圆 $\frac{x^2}{4} + y^2 = m$ 上。分别代入
$$ \frac{x_B^2}{4} + y_B^2 = m ,\frac{(-2x_B)^2}{4} + (3 - 2y_B)^2 = m $$
消去 $x_B$得到:$$ \frac{x_B^2}{4} + y_B^2 = x_B^2 + 4y_B^2 - 12y_B + 9 \Leftrightarrow\left(1 - \frac{1}{4}\right)x_B^2 + (4 - 1)y_B^2 - 12y_B + 9 = 0 $$
整理得到椭圆轨迹 $E_B: \frac{x^2}{4} + (y - 2)^2 = 1$ ,将 $x_B^2$ 反代,找出参数 $m$ 和 $y_B$ 的对应关系为$m = 4y_B - 3$,由条件 $m > 1$:$$ 4y_B - 3 > 1 \implies 4y_B > 4 \implies y_B > 1 $$,因此,满足条件的点 $B$ 的充分且必要集合为:椭圆 $\frac{x_B^2}{4} + (y_B - 2)^2 = 1$ 上满足 $y_B > 1$ 的所有点。由于点 $B$ 在 $\frac{x_B^2}{4} + (y_B - 2)^2 = 1$ 上,且任意实数的平方非负 $(y_B - 2)^2 \geq 0$:$$ \frac{x_B^2}{4} = 1 - (y_B - 2)^2 \le 1 \implies x_B^2 \le 4 \implies \mathbf{|x_B| \le 2} $$当且仅当 $y_B = 2$ 时取得等号,此时完美符合约束 $y_B > 1$,且对应的 $m = 4(2)-3 = 5$ 同样符合 $m>1$。
\end{solution}
\begin{example}{抹茶T2}{}
    椭圆 $\frac{x^{2}}{4}+y^{2}=1$ 的右焦点为 $F$,过点 $F$ 的直线交椭圆于 $A,B$ 两点,若 $|OA|^{2}=|AB|$,求点 $A$ 的坐标.
\end{example}
\begin{solution}
    点 $A$ 的焦半径为:$ r_A = 2 - \frac{\sqrt{3}}{2}x_A $,凡是过焦点的弦被焦点分成的两段 $r_A$ 和 $r_B$,必然满足经典的调和性质:$ \frac{1}{r_A} + \frac{1}{r_B} = \frac{2a}{b^2} $,将 $a=2, b=1$ 代入:$$ \frac{1}{r_A} + \frac{1}{r_B} = \frac{4}{1} = 4 $$消去$r_B$:$$ \frac{1}{r_B} = 4 - \frac{1}{r_A} = \frac{4r_A - 1}{r_A} \implies r_B = \frac{r_A}{4r_A - 1} $$
弦长 $|AB| = r_A + r_B = r_A + \frac{r_A}{4r_A - 1} = \frac{4r_A^2}{4r_A - 1}$将 $r_A = 2 - \frac{\sqrt{3}}{2}x_A$ 代入:$$ |AB| = \frac{4\left(2 - \frac{\sqrt{3}}{2}x_A\right)^2}{4\left(2 - \frac{\sqrt{3}}{2}x_A\right) - 1} = \frac{4\left(4 - 2\sqrt{3}x_A + \frac{3}{4}x_A^2\right)}{8 - 2\sqrt{3}x_A - 1} = \frac{3x_A^2 - 8\sqrt{3}x_A + 16}{7 - 2\sqrt{3}x_A} $$
然后将 $|OA|^2$ 同样转化为关于 $x_A$ 的表达式
$$|OA|^2 = x_A^2 + y_A^2 = x_A^2 + 1 - \frac{x_A^2}{4} = 1 + \frac{3}{4}x_A^2 = \frac{3x_A^2 + 4}{4}$$
联立$ \frac{3x_A^2 + 4}{4} = \frac{3x_A^2 - 8\sqrt{3}x_A + 16}{7 - 2\sqrt{3}x_A} $得到
$$ 2\sqrt{3}x_A^3 - 3x_A^2 - 8\sqrt{3}x_A + 12 = 0 $$分解得$$ (x_A^2 - 4)(2\sqrt{3}x_A - 3) = 0 $$
当$x_A^2 - 4 = 0$解得 $x_A = 2$ 或 $x_A = -2$。即$(2, 0)$和$(-2, 0)$。当$2\sqrt{3}x_A - 3 = 0$
解得 $x_A = \frac{3}{2\sqrt{3}} = \frac{\sqrt{3}}{2}$。即$\left(\frac{\sqrt{3}}{2}, \frac{\sqrt{13}}{4}\right)$ 和 $\left(\frac{\sqrt{3}}{2}, -\frac{\sqrt{13}}{4}\right)$。
\end{solution}

\begin{example}{抹茶T3}{}
    抛物线$y^2=4x$的焦点为$F$,过焦点的直线交抛物线于$A,B$两点,点$M$在抛物线上,满足$MA垂直 MB,FM垂直AB$,求 $\triangle ABM$ 的面积.
\end{example}
\begin{solution}
    假设 $AB \perp x$ 轴,则直线方程为 $x = 1$。解得 $y^2 = 4 \implies A(1, 2), B(1, -2)$,此时$\overrightarrow{MA} \cdot \overrightarrow{MB} = (1, 2) \cdot (1, -2) = 1 - 4 = -3 \neq 0$。矛盾!故设直线 $AB$ 的方程为 $x = ty + 1$ (其中 $t \neq 0$)。联立得$y^2 - 4ty - 4 = 0$,
设点 $M$ 的坐标为 $(x_0, y_0)$,且 $x_0 = \frac{y_0^2}{4}$。
$\overrightarrow{MA} = (x_1 - x_0, y_1 - y_0) = \left(\frac{y_1^2 - y_0^2}{4}, y_1 - y_0\right),\overrightarrow{MB} = (x_2 - x_0, y_2 - y_0) = \left(\frac{y_2^2 - y_0^2}{4}, y_2 - y_0\right)$
由 $MA \perp MB$ 得 $\overrightarrow{MA} \cdot \overrightarrow{MB} = 0$:
$$\frac{y_1^2 - y_0^2}{4} \cdot \frac{y_2^2 - y_0^2}{4} + (y_1 - y_0)(y_2 - y_0) = 0\Leftrightarrow(y_1 - y_0)(y_2 - y_0) \left[ \frac{(y_1 + y_0)(y_2 + y_0)}{16} + 1 \right] = 0$$化简得$(y_1 + y_0)(y_2 + y_0) + 16 = 0$,代入得$-4 + y_0(4t) + y_0^2 + 16 = 0 \implies y_0^2 + 4ty_0 + 12 = 0$,所以$t = \frac{-y_0^2 - 12}{4y_0}$。

由 $FM \perp AB$ 得$t(y_0^2 - 4) + 4y_0 = 0$,联立消去 $t$:$$\frac{-y_0^2 - 12}{4y_0} = \frac{-4y_0}{y_0^2 - 4}\Leftrightarrow (y_0^2 - 12)(y_0^2 + 4) = 0$$解得$M$ 点的横坐标:$x_0 = \frac{y_0^2}{4} = \frac{12}{4} = 3$。所以 $M(3, \pm 2\sqrt{3})$。$t^2 = \left( \frac{-4(\pm 2\sqrt{3})}{12 - 4} \right)^2 = \left(\mp \sqrt{3}\right)^2 = 3$。已知 $FM \perp AB$,且焦点 $F$ 在直线 $AB$ 上,因此点 $M$ 到底边 $AB$ 的距离$$|FM| = \sqrt{(x_0 - 1)^2 + (y_0 - 0)^2} = \sqrt{(3 - 1)^2 + 12} = \sqrt{4 + 12} = 4$$底边 $|AB|$:由抛物线焦点弦长公式 $|AB| = x_1 + x_2 + p$:由于 $x = ty+1$,可得 $x_1+x_2 = t(y_1+y_2)+2 = t(4t)+2 = 4t^2+2$。因此 $|AB| = 4t^2 + 2 + p = 4(3) + 2 + 2 = 16$。面积 $S$:$$S_{\triangle ABM} = \frac{1}{2} \times |AB| \times |FM| = \frac{1}{2} \times 16 \times 4 = 32$$

另解:由 $\triangle AMB$ 是以 $AB$ 为斜边的直角三角形且 $FM \perp AB$ 可知,根据射影定理得 $|FM|^2 = |FA| \cdot |FB|$。
建立以 $F$ 为极点,极轴沿 $x$ 轴正方向的极坐标系,抛物线方程为 $r = \frac{2}{1 - \cos\theta}$。设直线 $AB$ 的倾斜角为 $\alpha$,则 $A, B$ 的极角分别为 $\alpha$ 和 $\alpha + \pi$。由极坐标性质可知:
$$ |FA| = \frac{2}{1 - \cos\alpha}, \quad |FB| = \frac{2}{1 + \cos\alpha} $$
从而 $|FA| \cdot |FB| = \frac{4}{1 - \cos^2\alpha} = \frac{4}{\sin^2\alpha}$。又弦长 $|AB| = |FA| + |FB| = \frac{4}{\sin^2\alpha}$,故由此确立几何关系:$|FM|^2 = |AB|$。
由于 $FM \perp AB$,点 $M$ 的极角为 $\alpha \pm \frac{\pi}{2}$,代入极坐标方程得 $|FM| = \frac{2}{1 \pm \sin\alpha}$。结合前述关系式得:
$$ \left( \frac{2}{1 \pm \sin\alpha} \right)^2 = \frac{4}{\sin^2\alpha} \implies \frac{1}{1 \pm \sin\alpha} = \frac{1}{|\sin\alpha|} $$
解得 $\sin^2\alpha = \frac{1}{4}$。此时,弦长 $|AB| = \frac{4}{1/4} = 16$,高 $|FM| = \sqrt{16} = 4$。
因此,$\triangle ABM$ 的面积为:
$$ S_{\triangle ABM} = \frac{1}{2} |AB| \cdot |FM| = \frac{1}{2} \times 16 \times 4 = 32 $$
\end{solution}

\begin{example}{抹茶T4}{}
    点 $A(0,-1)$,点 $P,Q$ 满足点 $A,P,Q$ 共线,且 $|AP|\cdot |AQ|=3$,若 $k_{OQ}=3k_{OP}$,求点 $P$ 的轨迹方程.
\end{example}
\begin{solution}
设动点 $P$ 的坐标为 $(x, y)$,点 $Q$ 的坐标为 $(x_Q, y_Q)$。由题设 $k_{OQ} = 3k_{OP}$ 可知,直线 $OP$ 与 $OQ$ 的斜率均存在且不为零,故点 $P$ 与点 $Q$ 的横坐标均满足 $x \neq 0$ 且 $x_Q \neq 0$。由斜率公式得 $\frac{y_Q}{x_Q} = \frac{3y}{x}$,即 $x y_Q = 3x_Q y$。

由于点 $A(0, -1)$ 与 $P, Q$ 三点共线,且 $P$ 不在 $y$ 轴上,故向量 $\overrightarrow{AP} \neq \vec{0}$。设 $\overrightarrow{AQ} = \lambda \overrightarrow{AP}$($\lambda \neq 0$),则有:
$$ (x_Q, y_Q + 1) = \lambda (x, y + 1) \implies x_Q = \lambda x, \quad y_Q = \lambda(y + 1) - 1 $$
将上述关系代入 $x y_Q = 3x_Q y$ 中,得 $x[\lambda(y + 1) - 1] = 3\lambda x y$。因 $x \neq 0$,等式两边同除以 $x$ 并整理得 $\lambda(y + 1) - 1 = 3\lambda y$,即 $\lambda(1 - 2y) = 1$。若 $y = \frac{1}{2}$,方程无解,故 $y \neq \frac{1}{2}$,从而解得 $\lambda = \frac{1}{1 - 2y}$。

根据长度条件 $|AP| \cdot |AQ| = 3$,结合 $\overrightarrow{AQ} = \lambda \overrightarrow{AP}$ 可得 $|\lambda| \cdot |AP|^2 = 3$。代入距离公式与 $\lambda$ 的表达式,得到关于 $P$ 点坐标的方程:
$$ \left| \frac{1}{1 - 2y} \right| \cdot [x^2 + (y + 1)^2] = 3 \implies x^2 + (y + 1)^2 = 3|1 - 2y| $$
针对 $1 - 2y$ 的正负号进行分类讨论:

当 $1 - 2y > 0$,即 $y < \frac{1}{2}$ 时,方程化为 $x^2 + y^2 + 2y + 1 = 3 - 6y$,整理得 $x^2 + y^2 + 8y - 2 = 0$,配方得标准方程 $x^2 + (y + 4)^2 = 18$。该圆上点的纵坐标最大值为 $y_{max} = -4 + \sqrt{18} = -4 + 3\sqrt{2}$。由于 $3\sqrt{2} = \sqrt{18} < 4.5$,故 $y_{max} < 0.5$,即该轨迹圆上的所有点均满足 $y < \frac{1}{2}$ 的前提条件。

当 $1 - 2y < 0$,即 $y > \frac{1}{2}$ 时,方程化为 $x^2 + y^2 + 2y + 1 = -(3 - 6y)$,整理得 $x^2 + y^2 - 4y + 4 = 0$,即 $x^2 + (y - 2)^2 = 0$。此方程的唯一实数解为点 $(0, 2)$,但该点横坐标 $x = 0$,不满足斜率存在的初始前提,故应予以舍弃。

综上所述,动点 $P$ 的轨迹是以 $(0, -4)$ 为圆心,$\sqrt{18}$ 为半径的圆,且需除去与 $y$ 轴的交点。其轨迹方程为:
$$ x^2 + (y + 4)^2 = 18 \quad (x \neq 0) $$
或写成一般式:$x^2 + y^2 + 8y - 2 = 0 \quad (x \neq 0)$。
\end{solution}
\begin{example}{抹茶T5}{}
    双曲线 $\frac{x^{2}}{a^{2}} - \frac{y^{2}}{b^{2}} = 1$ 的右顶点为 $A$, $\overrightarrow{OB} = 2\overrightarrow{OA}$, 若双曲线上存在点 $P$, 使得 $\angle APB = \frac{\pi}{2}$, 求双曲线的离心率的取值范围.
\end{example}
\begin{solution}
设双曲线的标准方程为 $\frac{x^2}{a^2} - \frac{y^2}{b^2} = 1$ ($a>0, b>0$),其右顶点为 $A(a, 0)$。由 $\overrightarrow{OB} = 2\overrightarrow{OA}$ 可得 $B$ 点坐标为 $(2a, 0)$。设双曲线上存在点 $P(x, y)$ 满足 $\angle APB = \frac{\pi}{2}$,则有 $\overrightarrow{PA} \cdot \overrightarrow{PB} = 0$ 且点 $P$ 不与 $A, B$ 重合,即 $y \neq 0$。

根据向量坐标运算,$\overrightarrow{PA} = (a - x, -y)$,$\overrightarrow{PB} = (2a - x, -y)$,由数量积为零得:
$$(x - a)(x - 2a) + y^2 = 0 \implies y^2 = -(x - a)(x - 2a)$$
由于 $y^2 > 0$,可得 $x$ 的取值范围满足 $a < x < 2a$。此条件同时限定了点 $P$ 必须位于双曲线的右支。

将点 $P$ 在双曲线上的关系 $y^2 = b^2 \left( \frac{x^2}{a^2} - 1 \right)$ 代入上述方程。利用离心率关系 $\frac{b^2}{a^2} = e^2 - 1$,得 $y^2 = (e^2 - 1)(x^2 - a^2)$。联立圆与双曲线的方程:
$$(e^2 - 1)(x - a)(x + a) = -(x - a)(x - 2a)$$
由于 $x > a$,等式两边可约去 $x - a$,得:
$$(e^2 - 1)(x + a) = -(x - 2a)$$
整理得 $e^2 x + e^2 a - x - a = -x + 2a$,化简可解得点 $P$ 的横坐标为:
$$x = \frac{3 - e^2}{e^2} a$$

要使满足题意的点 $P$ 存在,该横坐标 $x$ 必须满足范围 $a < x < 2a$。代入得:
$$a < \frac{3 - e^2}{e^2} a < 2a$$
由于 $a > 0$,不等式各边同除以 $a$ 得 $1 < \frac{3}{e^2} - 1 < 2$,即 $2 < \frac{3}{e^2} < 3$。由此解得:
$$1 < e^2 < \frac{3}{2}$$
考虑到双曲线离心率 $e > 1$,对不等式开平方得 $1 < e < \frac{\sqrt{6}}{2}$。

经过检验,当 $e \in (1, \frac{\sqrt{6}}{2})$ 时,所求横坐标 $x$ 对应的纵坐标 $y = \pm \sqrt{-(x-a)(x-2a)}$ 必为非零实数,保证了点 $P$ 存在且不与 $A, B$ 重合,满足 $\angle APB = \frac{\pi}{2}$。若 $e = \frac{\sqrt{6}}{2}$,则 $x = a$,此时点 $P$ 与顶点 $A$ 重合,角不存在。因此,双曲线离心率 $e$ 的取值范围是 $(1, \frac{\sqrt{6}}{2})$。
\end{solution}