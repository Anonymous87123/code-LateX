\chapter{杂题}
\section{数学问题}
\begin{example}{新曲线}{}
已知曲线 $\Gamma: 12x^3 + 12xy^2 + 3x^2 + 4y^2 = 0$。过原点 $O$ 作两条互相垂直的直线,分别交曲线 $\Gamma$ 于异于原点的两点 $A, B$。
求证:无论直线如何转动,$\triangle OAB$ 的外接圆恒过一个异于原点的定点,并求出该定点坐标。
\end{example}

\begin{example}{}{}
(多选)已知非零复数 $z_1, z_2$ 满足 $z\bar{z} = z + \bar{z}$。设复数 $z_3$ 满足$\frac{2}{z_3} = \frac{1}{z_1} + \frac{1}{z_2}$。下列关于 $z_1, z_2, z_3$ 的结论中,正确的有(~~~~~)\\
A. 复数 $z_3$ 必定满足方程 $z\bar{z} = z + \bar{z}$\\B. $\frac{|z_1 - z_3|}{|z_1|} = \frac{|z_2 - z_3|}{|z_2|}$\\C. $\frac{1}{|z_1|} + \frac{1}{|z_2|} \ge \frac{2}{|z_3|}$\\D. $\frac{1}{|z_1|^2} + \frac{1}{|z_2|^2} \le \frac{2}{|z_3|^2}$
\end{example}
证明 $z_3$ 满足 $z_3\bar{z_3} = z_3 + \bar{z_3}$已知条件是 $z_1\bar{z_1} = z_1 + \bar{z_1}$ 和 $z_2\bar{z_2} = z_2 + \bar{z_2}$。既然 $z_3$ 是 $z_1$ 和 $z_2$ 的调和平均数,即 $\frac{2}{z_3} = \frac{1}{z_1} + \frac{1}{z_2}$,我们先把 $z_3$ 解出来:$$z_3 = \frac{2z_1z_2}{z_1 + z_2}$$同样的,它的共轭复数就是:$$\bar{z_3} = \frac{2\bar{z_1}\bar{z_2}}{\bar{z_1} + \bar{z_2}}$$我们要证 $z_3\bar{z_3} = z_3 + \bar{z_3}$,那就把等号左右两边分别算出来,看能不能对得上!先算左边 (LHS):$$z_3\bar{z_3} = \left(\frac{2z_1z_2}{z_1 + z_2}\right) \left(\frac{2\bar{z_1}\bar{z_2}}{\bar{z_1} + \bar{z_2}}\right) = \frac{4(z_1\bar{z_1})(z_2\bar{z_2})}{(z_1 + z_2)(\bar{z_1} + \bar{z_2})}$$把已知条件 $z_1\bar{z_1} = z_1 + \bar{z_1}$ 和 $z_2\bar{z_2} = z_2 + \bar{z_2}$ 直接代入分子:$$\text{LHS} = \frac{4(z_1 + \bar{z_1})(z_2 + \bar{z_2})}{(z_1 + z_2)(\bar{z_1} + \bar{z_2})}$$再算右边 (RHS):$$z_3 + \bar{z_3} = \frac{2z_1z_2}{z_1 + z_2} + \frac{2\bar{z_1}\bar{z_2}}{\bar{z_1} + \bar{z_2}}$$面对两个分数相加,毫不犹豫直接通分!$$\text{RHS} = \frac{2z_1z_2(\bar{z_1} + \bar{z_2}) + 2\bar{z_1}\bar{z_2}(z_1 + z_2)}{(z_1 + z_2)(\bar{z_1} + \bar{z_2})}$$分母现在和左边一样了,我们专心拆开分子看一看:$$\text{分子} = 2z_1z_2\bar{z_1} + 2z_1z_2\bar{z_2} + 2\bar{z_1}\bar{z_2}z_1 + 2\bar{z_1}\bar{z_2}z_2$$把式子里的项重新组合一下,把 $z$ 和它的共轭 $\bar{z}$ 凑对:$$\text{分子} = 2(z_1\bar{z_1})z_2 + 2(z_2\bar{z_2})z_1 + 2(z_1\bar{z_1})\bar{z_2} + 2(z_2\bar{z_2})\bar{z_1}$$再次代入已知条件 $z\bar{z} = z + \bar{z}$ 替换掉乘积项:$$\text{分子} = 2(z_1 + \bar{z_1})z_2 + 2(z_2 + \bar{z_2})z_1 + 2(z_1 + \bar{z_1})\bar{z_2} + 2(z_2 + \bar{z_2})\bar{z_1}$$稍微提取一下公因式:$$\text{分子} = 2(z_1 + \bar{z_1})(z_2 + \bar{z_2}) + 2(z_2 + \bar{z_2})(z_1 + \bar{z_1})$$合并同类项,搞定!$$\text{分子} = 4(z_1 + \bar{z_1})(z_2 + \bar{z_2})$$放回分母上一看,$\text{RHS} = \frac{4(z_1 + \bar{z_1})(z_2 + \bar{z_2})}{(z_1 + z_2)(\bar{z_1} + \bar{z_2})}$。完美!左边完全等于右边,第一问爆算成功!第二问:证明 $\frac{|z_1 - z_3|}{|z_1|} = \frac{|z_2 - z_3|}{|z_2|}$这一问看着要处理绝对值很麻烦,但别急,我们先把绝对值里面的代数式整理干净。先看等式左边的内部 $z_1 - z_3$:把 $z_3 = \frac{2z_1z_2}{z_1 + z_2}$ 代进去硬减:$$z_1 - z_3 = z_1 - \frac{2z_1z_2}{z_1 + z_2} = \frac{z_1(z_1 + z_2) - 2z_1z_2}{z_1 + z_2}$$分子展开化简:$$z_1^2 + z_1z_2 - 2z_1z_2 = z_1^2 - z_1z_2 = z_1(z_1 - z_2)$$所以:$$z_1 - z_3 = \frac{z_1(z_1 - z_2)}{z_1 + z_2}$$把它塞回我们要算的左边式子里(利用复数模的性质 $|\frac{a}{b}| = \frac{|a|}{|b|}$):$$\frac{|z_1 - z_3|}{|z_1|} = \left| \frac{\frac{z_1(z_1 - z_2)}{z_1 + z_2}}{z_1} \right|$$看到没有,$z_1$ 被完美约掉了!$$\text{左边} = \left| \frac{z_1 - z_2}{z_1 + z_2} \right| = \frac{|z_1 - z_2|}{|z_1 + z_2|}$$再用一模一样的套路处理等式右边的 $z_2 - z_3$:$$z_2 - z_3 = z_2 - \frac{2z_1z_2}{z_1 + z_2} = \frac{z_2(z_1 + z_2) - 2z_1z_2}{z_1 + z_2}$$分子展开:$$z_2z_1 + z_2^2 - 2z_1z_2 = z_2^2 - z_1z_2 = z_2(z_2 - z_1)$$所以:$$z_2 - z_3 = \frac{z_2(z_2 - z_1)}{z_1 + z_2}$$代入右边的式子:$$\frac{|z_2 - z_3|}{|z_2|} = \left| \frac{\frac{z_2(z_2 - z_1)}{z_1 + z_2}}{z_2} \right| = \left| \frac{z_2 - z_1}{z_1 + z_2} \right|$$因为复数相减的模长满足 $|z_2 - z_1| = |z_1 - z_2|$(也就是两点之间的距离),所以:$$\text{右边} = \frac{|z_1 - z_2|}{|z_1 + z_2|}$$左边 = 右边,第二问也就这么被纯代数步骤“爆”出来了!