\chapter{反演变换在解析几何中的应用}

反演变换作为一种高级几何变换,常被应用于解析几何的命题与解题中。通过反演映射,基础的二次曲线问题可转化为高次曲线(如三次、四次曲线)的交轨问题。本章将系统阐述反演变换的原理,并通过具体实例展示其在圆锥曲线定理推广中的应用。

\section{反演变换的基本原理}

定义:假设反演中心为$O$,反演半径为$R$。平面上任意一点$P$(异于点$O$)被变换至反演像上的点$P^{\prime}$,满足$\overrightarrow{OP}$与$\overrightarrow{OP}^{\prime}$同向共线,且$OP\cdot OP^\prime=R^2$。

反演变换具有以下基本性质:
\begin{itemize}
    \item 过反演中心的直线变为过反演中心的直线。
    \item 不过反演中心的直线变为过反演中心的圆。
    \item 过反演中心的圆变为不过反演中心的直线。
    \item 不过反演中心的圆变为不过反演中心的圆。
    \item 反演是可逆的双射变换。若曲线$\Gamma$的反演像为$\Gamma^{\prime}$,则$\Gamma^{\prime}$作相同参数的反演变换可还原为$\Gamma$。
    \item 反演中心$O$与无穷远点$\infty$互为反演点。
    \item 反演具有保角性,两曲线交角在反演前后保持不变,正交关系在变换中被严格保持。
\end{itemize}

在平面直角坐标系中,设反演中心为原点$O(0,0)$。对于任意一点$(x_0, y_0)$,其反演点坐标为$\left(\frac{R^{2}x_{0}}{x_{0}^{2}+y_{0}^{2}}, \frac{R^{2}y_{0}}{x_{0}^{2}+y_{0}^{2}}\right)$。根据反演的可逆性,对于一般二次曲线方程
$$Ax^{2}+Bxy+Cy^{2}+Dx+Ey+F=0$$
将其中的$x$替换为$\frac{R^{2}x}{x^{2}+y^{2}}$,将$y$替换为$\frac{R^{2}y}{x^{2}+y^{2}}$,即可得到反演像的代数方程:
$$R^{4}(Ax^{2}+Bxy+Cy^{2})+R^{2}(Dx+Ey)(x^{2}+y^{2})+F(x^{2}+y^{2})^{2}=0$$
其中$x, y$不同时为$0$。通常情况下,该反演像为一条四次曲线。若反演中心在原二次曲线上,则退化为三次曲线;直线与圆的情况则分别退化为对应的直线或圆。

例如,以原点$O$为反演中心,$R=2$为反演半径,对椭圆$\frac{x^{2}}{4}+y^{2}=1$进行反演。代入坐标变换公式并化简,可得反演像方程为$(x^{2}+y^{2})^{2}=4x^{2}+16y^{2}$。

\subsection{四次曲线的垂足曲线生成方式}
上述四次曲线也可通过垂足曲线的方式生成。考虑椭圆$\frac{x^2}{a^2}+\frac{y^2}{b^2}=1$,$O$是坐标原点,$l$是椭圆的动切线,$P$是点$O$在直线$l$上的垂足。设$P(x_0, y_0)$,则切线$l$的方程为$y - y_0 = -\frac{x_0}{y_0}(x - x_0)$,整理得$\frac{x_0}{x_0^2 + y_0^2}x + \frac{y_0}{x_0^2 + y_0^2}y = 1$。由相切条件可知$a^2\left(\frac{x_0}{x_0^2 + y_0^2}\right)^2 + b^2\left(\frac{y_0}{x_0^2 + y_0^2}\right)^2 = 1$。整理得$P$的轨迹方程为$(x^2 + y^2)^2 = ax^2 + by^2$。此轨迹即为椭圆$ax^2 + by^2 = 1$以$O$为反演中心、$R=1$为反演半径的反演像。同理,对于双曲线$xy = 1$,点$O$到其切线的垂足轨迹方程为$(x^2 + y^2)^2 = 4xy$,该轨迹恰为双曲线$xy = 1$以$O$为中心、$\sqrt{2}$为半径的反演像。

\section{基础圆锥曲线定理的反演命题}

利用反演变换的保角性与拓扑不变性,可将基础的解析几何题目转化为高次曲线交轨问题。

\begin{example}{}{}
已知曲线$\Gamma: (x^2 + y^2)^2 = 4x^2 + 16y^2$,$A(-2,0)$。过原点$O$和定点$E\left(-\frac{10}{3},0\right)$的动圆与$\Gamma$交于$B,C$两点,$\triangle AOB$与$\triangle AOC$的外接圆半径分别为$R_1,R_2$。证明:$(R_1^2 - 1)(R_2^2 - 1)$为定值。
\end{example}
\begin{solution}
设过点$O(0,0)$和$E\left(-\frac{10}{3},0\right)$的圆方程为
$$x^{2}+y^{2}+\frac{10}{3}x+Fy=0$$
联立曲线$\Gamma$的方程$(x^{2}+y^{2})^{2}=4x^{2}+16y^{2}$,消去二次项得$\left(-\frac{10}{3}x-Fy\right)^{2}=4x^{2}+16y^{2}$,整理得
$$64x^{2}+60Fxy+9(F^{2}-16)y^{2}=0$$
设$B(x_{1},y_{1})$,$C(x_{2},y_{2})$,且$y_{1},y_{2}\neq 0$。令$t_{1}=\frac{x_{1}}{y_{1}}$,$t_{2}=\frac{x_{2}}{y_{2}}$,则$t_{1},t_{2}$为方程$64t^{2}+60Ft+9(F^{2}-16)=0$的两根。由韦达定理可得
$$t_{1}+t_{2}=-\frac{15F}{16},\quad t_{1}t_{2}=\frac{9(F^{2}-16)}{64}$$
由于$\triangle ABO$的外接圆过原点$O$及定点$A(-2,0)$,可设其方程为$x^{2}+y^{2}+2x+E_{1}y=0$。将点$B$满足的圆方程$x_{1}^{2}+y_{1}^{2}=-\frac{10}{3}x_{1}-Fy_{1}$代入,得$-\frac{4}{3}x_{1}+(E_{1}-F)y_{1}=0$。由此解得$E_{1}=F+\frac{4}{3}t_{1}$。同理,$\triangle ACO$外接圆的参数满足$E_{2}=F+\frac{4}{3}t_{2}$。
分析两外接圆半径。其半径平方分别为$R_{1}^{2}=1+\frac{E_{1}^{2}}{4}$与$R_{2}^{2}=1+\frac{E_{2}^{2}}{4}$。因此
$$(R_{1}^{2}-1)(R_{2}^{2}-1)=\frac{E_{1}^{2}E_{2}^{2}}{16}$$
计算$E_{1}E_{2}$的乘积:
$$E_{1}E_{2}=F^{2}+\frac{4}{3}F(t_{1}+t_{2})+\frac{16}{9}t_{1}t_{2}$$
代入韦达定理结论可得$E_{1}E_{2}=F^{2}-\frac{5}{4}F^{2}+\frac{1}{4}(F^{2}-16)=-4$。于是$(R_{1}^{2}-1)(R_{2}^{2}-1)=\frac{(-4)^{2}}{16}=1$。得证。
(注:此题本质为证明椭圆$\frac{x^2}{4}+y^2=1$中,过点$(-\frac{6}{5},0)$的弦对左顶点张角为直角)。
\end{solution}

\begin{example}{}{}
已知曲线$\Gamma: (x^2 + y^2)^2 = 4xy$。过坐标原点$O$的动圆$\odot Q$与$\Gamma$相切,$\odot Q$分别交$x$轴、$y$轴于$B, C$两点。求证:$\triangle OBC$的面积为定值。
\end{example}
\begin{solution}
设$\odot Q$方程为$x^2+y^2=Dx+Ey$。联立$\Gamma$方程,得$(Dx+Ey)^2=4xy$。整理得关于$\frac{x}{y}$的二次方程:
$$D^2\left(\frac{x}{y}\right)^2+(2DE-4)\left(\frac{x}{y}\right)+E^2=0$$
由相切条件,该方程的判别式必定为零:
$$\Delta=(2DE-4)^2-4D^2E^2=0$$
化简得$-16DE+16=0$,即$DE=1$。
圆$\odot Q$交坐标轴于$B(D,0)$及$C(0,E)$。$\triangle OBC$的面积为$\frac{1}{2}|D||E|=\frac{1}{2}$。得证。
\end{solution}

\section{进阶命题:蒙日圆与垂直弦性质的反演}

\begin{example}{}{}
已知曲线$\Gamma: (x^2 + y^2)^2 = 4x^2 + 16y^2$,$O$为坐标原点。有两个过原点$O$的圆$\odot C_1$与$\odot C_2$均与曲线$\Gamma$相切,且$\odot C_1$与$\odot C_2$在原点正交。设$\odot C_1$与$\odot C_2$除原点$O$外的交点为$P$。求证:线段$OP$的长度为定值。
\end{example}
\begin{solution}
设过原点的圆方程为$x^2 + y^2 = Dx + Ey$。将其与$\Gamma$的方程联立,考察相切条件,得$(Dx + Ey)^2 = 4x^2 + 16y^2$。整理得关于$x,y$的齐次二次方程:
$$(D^2 - 4)x^2 + 2DExy + (E^2 - 16)y^2 = 0$$
相切要求此二次齐次方程的判别式为零:
$$\Delta = (2DE)^2 - 4(D^2 - 4)(E^2 - 16) = 0$$
化简得$16D^2 + 4E^2 = 64$,即$4D^2 + E^2 = 16$。
设$\odot C_1, \odot C_2$的方程分别为$x^2 + y^2 = D_1x + E_1y$与$x^2 + y^2 = D_2x + E_2y$。两圆相切的条件为$4D_1^2 + E_1^2 = 16$且$4D_2^2 + E_2^2 = 16$。两圆在原点正交的条件为$D_1D_2 + E_1E_2 = 0$。
联立两圆方程,求交点$P(x,y)$。令$|OP|^2 = x^2 + y^2$。由两圆方程得$D_1x + E_1y = |OP|^2$且$D_2x + E_2y = |OP|^2$。运用克莱姆法则解得$x, y$,并代入$x^2 + y^2 = |OP|^2$中,化简得:
$$|OP|^2 = \frac{(D_1E_2 - D_2E_1)^2}{(D_1 - D_2)^2 + (E_1 - E_2)^2}$$
利用恒等式及正交条件$D_1D_2 + E_1E_2 = 0$,将分子和分母展开:
$$|OP|^2 = \frac{(D_1^2 + E_1^2)(D_2^2 + E_2^2)}{D_1^2 + E_1^2 + D_2^2 + E_2^2}$$
代入条件$E_1^2 = 16 - 4D_1^2$与$E_2^2 = 16 - 4D_2^2$,以及正交导出的关系$D_1^2 D_2^2 = E_1^2 E_2^2$,进一步化简可得分子与分母均含有公因式$32 - 3(D_1^2 + D_2^2)$。约去该公因式后得到$|OP|^2 = \frac{16}{5}$。因此线段$OP$的长为$\frac{4\sqrt{5}}{5}$。
\end{solution}

\begin{example}{}{}
已知曲线$\Gamma: x^3 + xy^2 - y^2 = 0$(蔓叶线)。设$A, B$是$\Gamma$上异于原点$O$的两个动点,且满足$OA \perp OB$。求证:$\triangle OAB$的外接圆恒过一个异于原点的定点。
\end{example}
\begin{solution}
设直线$OA$的方程为$y = kx$,则$OB$的方程为$y = -\frac{1}{k}x$。代入曲线$\Gamma$求点$A, B$坐标。对点$A$,得$x^3 + k^2x^3 - k^2x^2 = 0$,解得$x_A = \frac{k^2}{1+k^2}$,$y_A = \frac{k^3}{1+k^2}$。同理,将$k$替换为$-\frac{1}{k}$得$x_B = \frac{1}{1+k^2}$,$y_B = -\frac{1}{k(1+k^2)}$。
设$\triangle OAB$外接圆方程为$x^2 + y^2 + Dx + Ey = 0$。代入点$A$坐标并除以$\frac{k^2}{1+k^2}$得$k^2 + D + Ek = 0$。代入点$B$坐标得$-\frac{1}{k^2} + D - \frac{E}{k} = 0$,即$Dk^2 - Ek = -1$。两式相加消去$E$得$D(1+k^2) = -(1+k^2)$,由此解得$D=-1$及$E=\frac{1-k^2}{k}$。外接圆方程为$x^2 + y^2 - x + \frac{1-k^2}{k}y = 0$。令含有参数的项$y=0$,得$x^2 - x = 0$。解得异于原点的定点为$(1, 0)$。
\end{solution}

\section{离心率、极点与极线的反演表示}

\begin{example}{}{}
已知曲线$\Gamma: (x^2+y^2+2x)^2 = 4(x^2+y^2)$(心脏线),$O$为坐标原点。过原点$O$作任意一条不与$y$轴重合的直线,交曲线$\Gamma$于$A, B$两点。求证:线段$AB$的长度为定值$4$。
\end{example}
\begin{solution}
设直线方程为$y = kx$。代入曲线$\Gamma$方程,得$(x^2 + k^2x^2 + 2x)^2 = 4(x^2 + k^2x^2)$。由于交点异于原点,提取公因式$x^2$并化简得$[x(1+k^2) + 2]^2 = 4(1+k^2)$。开平方后解出两交点的横坐标:
$$x_A = \frac{-2 + 2\sqrt{1+k^2}}{1+k^2}, \quad x_B = \frac{-2 - 2\sqrt{1+k^2}}{1+k^2}$$
线段长度$|AB| = \sqrt{1+k^2} |x_A - x_B| = \sqrt{1+k^2} \left| \frac{4\sqrt{1+k^2}}{1+k^2} \right| = 4$。
\end{solution}

极性变换与反演能够反映极点与极线中深刻的调和共轭性质。

\begin{example}{}{}
已知曲线$\Gamma_1: (x^2+y^2)^2 - 2x(x^2+y^2) + 2y^2 = 0$以及圆$\Gamma_2: x^2+y^2 - x = 0$。过原点$O$作任意一条射线,交$\Gamma_1$于$A, B$两点,交$\Gamma_2$于$Q$点。求证:点$Q$恒为线段$AB$的中点。
\end{example}
\begin{solution}
设射线方程为$y = kx$ ($x > 0$)。将射线代入圆$\Gamma_2$方程得$x(1+k^2) = 1$,解得$x_Q = \frac{1}{1+k^2}$。将其代入$\Gamma_1$方程,提取非零因子后得关于横坐标的方程$x^2(1+k^2)^2 - 2x(1+k^2) + 2k^2 = 0$。由韦达定理,交点$A, B$的横坐标之和为$x_A + x_B = \frac{2}{1+k^2}$。显然满足$x_A + x_B = 2x_Q$,结合三点共射线的条件,可得结论成立。此性质实为圆锥曲线极点、极线与截线所构成的调和共轭线段经反演后的同构表现。
\end{solution}

对于任意离心率$e > 0$,帕斯卡蜗牛线族$\Gamma_e: (x^2+y^2+ex)^2 = x^2+y^2$中过极点的弦长恒为常数。设截线$y = kx$,代入化简后得$[x(1+k^2) + e]^2 = 1+k^2$,解出两交点横坐标之差,并结合弦长公式计算可得$|AB|=2$,此结果不受离心率参数$e$的影响。

\section{射影几何定理的反演表示}

反演不仅适用于度量性质的推广,对于射影几何中的结合关系(如共点、共线)也能建立对应结构。

以帕斯卡定理为例:圆锥曲线上内接六边形对边延长线的三个交点共线。通过选定特定反演中心,双曲线反演生成双纽线$\Gamma: (x^2+y^2)^2 = 4xy$。原图形中的割线反演为过原点的圆,三点共线反演为四点共圆。由此得出结论:在$\Gamma$上取$6$个异于原点$O$的点构造$6$个过原点的圆,其对位交点$P, Q, R$必与点$O$共圆。

同理,结合极性变换(对偶原则),将“点共线”转化为“线共点”,可严格证明布里昂雄定理。针对彭赛列闭合定理中偏心两圆的交错折线问题,通过选取特定的极限点作为反演中心进行莫比乌斯变换,可将偏心圆映射为同心圆,将复杂的边长动态变化转化为静态的旋转对称性证明。

\section{隐蔽的反演命题:弗雷歇定理与中点轨迹}

为增加问题的代数复杂度与隐蔽性,命题常采用弗雷歇定理或弦中点轨迹进行包装。

\begin{example}{}{}
已知曲线$\Gamma: 12x^3 + 12xy^2 + 3x^2 + 4y^2 = 0$。过原点$O$作两条互相垂直的直线,分别交$\Gamma$于异于原点的两点$A, B$。求证:$\triangle OAB$的外接圆恒过一定点。
\end{example}
\begin{solution}
设$OA$方程为$y = kx$,$OB$方程为$y = -\frac{1}{k}x$。代入$\Gamma$并降次,解得$x_A = -\frac{3+4k^2}{12(1+k^2)}$,$y_A = -\frac{k(3+4k^2)}{12(1+k^2)}$。利用对称性得$x_B = -\frac{3k^2+4}{12(k^2+1)}$,$y_B = -\frac{3k^2+4}{12k(k^2+1)}$。设外接圆方程为$x^2 + y^2 + Dx + Ey = 0$。代入$A, B$坐标并消元,化简可解得关于$D, E$的线性方程组。其中$D + Ek = \frac{3+4k^2}{12}$与$Dk^2 - Ek = \frac{3k^2+4}{12}$。两式相加消去$E$项,得$D(1+k^2) = \frac{7(1+k^2)}{12}$,即$D = \frac{7}{12}$。令含参项$y=0$,解得外接圆恒过的定点为$\left(-\frac{7}{12}, 0\right)$。
\end{solution}

\begin{example}{}{}
已知曲线$\Gamma: (x^2+y^2)^2 + 2x(x^2+y^2) - 2y^2 = 0$。过原点$O$的直线$l$交$\Gamma$于两点$A, B$,设中点为$M$。求$M$的轨迹。
\end{example}
\begin{solution}
设直线$l$方程为$y = kx$。代入$\Gamma$方程,化简上述四次方程可得一元二次方程:$(1+k^2)^2 x^2 + 2(1+k^2) x - 2k^2 = 0$。由韦达定理,中点横坐标$x_M = \frac{-1}{1+k^2}$,纵坐标$y_M = \frac{-k}{1+k^2}$。消除参数$k$,计算$x_M^2 + y_M^2 = \frac{1}{1+k^2} = -x_M$。整理得轨迹方程为$x^2 + y^2 + x = 0$(除去原点)。该结果本质上体现了反演空间中极线与圆的同构对应。
\end{solution}

\section{复平面上的反演变换}

在复平面$\mathbb{C}$中,取单位圆为反演基准,坐标$(x,y)$在复反演映射下可表达为极简的代数形式:
$$w = \frac{1}{\bar{z}}$$
广义圆与直线的统一方程$A z \bar{z} + \bar{B} z + B \bar{z} + C = 0$在经过上述映射后,化为$C w \bar{w} + B \bar{w} + \bar{B} w + A = 0$。二次项系数$A$与常数项$C$发生了位置互换,充分展现了复数域中反演对称的代数结构特征。

\begin{example}{}{}
已知非零复数$z_1, z_2$满足方程$z\bar{z} = z + \bar{z}$。设复数$z_3$满足$\frac{2}{z_3} = \frac{1}{z_1} + \frac{1}{z_2}$。关于$z_1, z_2, z_3$的论断如下:
A. 复数$z_3$必定满足方程$z\bar{z} = z + \bar{z}$ \\
B. $\frac{|z_1 - z_3|}{|z_1|} = \frac{|z_2 - z_3|}{|z_2|}$ \\
C. $\frac{1}{|z_1|} + \frac{1}{|z_2|} \ge \frac{2}{|z_3|}$ \\
D. $\frac{1}{|z_1|^2} + \frac{1}{|z_2|^2} \ge \frac{2}{|z_3|^2}$ \\
判断上述论断的正确性。
\end{example}
\begin{solution}
将原方程$z\bar{z} = z + \bar{z}$两边同除以$z\bar{z}$,得$\frac{1}{\bar{z}} + \frac{1}{z} = 1$。转化至反演平面中,令$w = \frac{1}{z}$,方程化为$w + \bar{w} = 1$,即$\text{Re}(w) = \frac{1}{2}$。该方程在$w$平面上表示一条垂直于实轴的直线。由调和平均条件$\frac{2}{z_3} = \frac{1}{z_1} + \frac{1}{z_2}$可知$w_3 = \frac{w_1 + w_2}{2}$,即$w_3$为线段$w_1 w_2$的中点。

对于论断A:由于$w_1, w_2$均在直线$\text{Re}(w) = \frac{1}{2}$上,其所在线段的中点$w_3$必满足$\text{Re}(w_3) = \frac{1}{2}$。映射回原复平面,$z_3$满足方程$z\bar{z} = z + \bar{z}$。A正确。

对于论断B:将距离公式转化至$w$平面。$\frac{|z_1 - z_3|}{|z_1|} = \frac{|w_3 - w_1|/|w_1 w_3|}{1/|w_1|} = \frac{|w_3 - w_1|}{|w_3|}$。同理等式右侧为$\frac{|w_3 - w_2|}{|w_3|}$。由中点性质可知$|w_3 - w_1| = |w_3 - w_2|$,故等式成立。B正确。

对于论断C:在$w$平面应用三角不等式。由于$w_1 + w_2 = 2w_3$,取模长得$|w_1| + |w_2| \ge |w_1 + w_2| = 2|w_3|$。将其还原为原坐标表达即为$\frac{1}{|z_1|} + \frac{1}{|z_2|} \ge \frac{2}{|z_3|}$。C正确。

对于论断D:由复数运算的平行四边形法则,$|w_1 - w_2|^2 + |w_1 + w_2|^2 = 2(|w_1|^2 + |w_2|^2)$。代入$w_1 + w_2 = 2w_3$并利用$|w_1 - w_2|^2 \ge 0$放缩可得$4|w_3|^2 \le 2(|w_1|^2 + |w_2|^2)$。还原后不等式成立。D正确。
\end{solution}

\begin{example}{}{}
已知曲线$\Gamma: (x^2+y^2)^2 = x^2-y^2$。设过坐标原点$O$的动圆$C$与$\Gamma$相切。若圆$C$分别交直线$y=x$与$y=-x$于异于原点的$M, N$两点,求证:$\triangle OMN$的面积为定值。
\end{example}
\begin{solution}
设圆$C$的方程为$x^2+y^2+Dx+Ey=0$。将该方程与直线$y=x$联立,可得交点纵横坐标满足$2x^2+(D+E)x=0$。若交点$M$异于原点,则$x_M=y_M=-\frac{D+E}{2}$。由此可得线段长度$|OM|=\frac{|D+E|}{\sqrt{2}}$。
同理,将其与直线$y=-x$联立,解得交点$N$满足$x_N=-\frac{D-E}{2}, y_N=\frac{D-E}{2}$,线段长度$|ON|=\frac{|D-E|}{\sqrt{2}}$。
因直线$y=x$与$y=-x$垂直,$\triangle OMN$面积表达为$S = \frac{1}{2}|OM||ON| = \frac{1}{4}|D^2-E^2|$。
探讨相切条件:将圆$C$的代数式代入$\Gamma$中消去高次项得$(-Dx-Ey)^2 = x^2-y^2$。重新排列为齐次形式$(D^2-1)x^2 + 2DExy + (E^2+1)y^2 = 0$。
若相切则此式判别式等于零,即$(2DE)^2 - 4(D^2-1)(E^2+1) = 0$。化简可得方程约束$D^2 - E^2 = 1$。
将该约束代入面积公式中,得到$S = \frac{1}{4}$,该面积恒为定值。(本题几何背景为等轴双曲线切线交其渐近线所成三角形面积定值性质的反演)。
\end{solution}

\begin{example}{}{}
已知曲线$\Gamma: x^3+xy^2-y^2=0$。设有两个过原点$O$的动圆$C_1, C_2$均与$\Gamma$相切,且$C_1$与$C_2$在原点处正交。若$P$为两圆除原点外的交点,求证:点$P$恒在一个定圆上。
\end{example}
\begin{solution}
设切圆一般方程为$x^2+y^2+Dx+Ey=0$。代入$\Gamma$提取共因式后化为$Dx^2+Exy+y^2=0$。
相切要求判别式等于零,即满足$4D=E^2$。
设圆$C_1, C_2$的对应系数为$(D_1, E_1)$与$(D_2, E_2)$。两圆在原点正交等效于一次项系数内积$D_1D_2+E_1E_2=0$。代入切线约束条件后得$\frac{E_1^2E_2^2}{16}+E_1E_2=0$。排除退化情形,恒得$E_1E_2 = -16$。
由方程作差联立解交点$P(x,y)$,得到直线约束$y = -\frac{D_1-D_2}{E_1-E_2}x = -\frac{E_1+E_2}{4}x$。
利用$E_1E_2=-16$及和式构建一元二次方程,可知$E_1^2 = -\frac{4y}{x}E_1 + 16$。
将$D_1=\frac{E_1^2}{4}$回代至$C_1$方程中并将二次形式带入替换,方程变为$x^2+y^2+\frac{1}{4}(-\frac{4y}{x}E_1 + 16)x+E_1y=0$。展开并合并同类项,含参项$E_1$相互抵消,恒等化简为$x^2+y^2+4x=0$。该定圆即为点$P$之轨迹。
\end{solution}

\begin{example}{}{}
已知曲线$\Gamma: 27x^2+36y^2+6x(x^2+y^2)-(x^2+y^2)^2=0$。设过坐标原点$O$的动圆$C$与$\Gamma$相切,且线段$OM$为圆$C$的一条直径。求证:点$M$的轨迹是一个圆。
\end{example}
\begin{solution}
设动圆$C$方程为$x^2+y^2-Dx-Ey=0$。因$OM$为其直径且过原点,端点$M$坐标恰为$(D, E)$。
代入曲线$\Gamma$可得关于$x,y$的二次齐次形式:$(27+6D-D^2)x^2 + (6E-2DE)xy + (36-E^2)y^2 = 0$。
结合相切条件判别式设为零,即$(6E-2DE)^2 - 4(27+6D-D^2)(36-E^2) = 0$。
提取公因式并逐步展开,可消去$D^2E^2$与$DE^2$等高次相互干扰项。化简后余项构成$36E^2 + 36D^2 - 216D - 972 = 0$。
对其同除以$36$,即得变量结构关系$D^2+E^2-6D-27=0$。
将坐标$(x_M, y_M)$替代$(D, E)$,可知点$M$的轨迹为圆$x^2+y^2-6x-27=0$。(几何背景:椭圆焦点对切线垂足的轨迹为对应辅圆的反演等价表示)。
\end{solution}

\begin{example}{}{}
定圆$C_1: x^2+y^2+4x=0$。对于$C_1$上任一异于原点$O$的点$P(x_0, y_0)$,作经过$O$与$P$且同曲线$\Gamma: x^3+xy^2-y^2=0$相切的两圆,设切点为$A, B$。求证:$\triangle OAB$的外接圆恒过一定点。
\end{example}
\begin{solution}
构建经过$O, P$并同$\Gamma$相切之圆方程$x^2+y^2+Dx+Ey=0$。根据例题已有结论,其相切约束仍为$4D=E^2$。
由经过点$P(x_0, y_0)$,将已知定圆约束$x_0^2+y_0^2=-4x_0$带入切圆方程化简得$(D-4)x_0+Ey_0=0$。
代入$D$表达式构成二次方程$(E^2-16)x_0+4Ey_0=0$。其两根$E_1, E_2$由韦达定理给定$E_1E_2=-16$及$E_1+E_2=-\frac{4y_0}{x_0}$。
求解具体切点坐标,原齐次方程$Dx^2+Exy+y^2=0$等效于直线$y=-\frac{E}{2}x$。将其与切圆方程联立解得切点横纵坐标分别为$x = \frac{E^2}{E^2+4}, y = \frac{-2E}{E^2+4}$。
计算可知平方项满足$x_A^2+y_A^2 = \frac{E_1^2}{E_1^2+4} = x_A$。由此可知点$A$的坐标无条件满足方程$x^2+y^2-x=0$。
同理切点$B$坐标也满足该方程。由于原点$O$显然同置其上,三点唯一的共圆外接方程即为$x^2+y^2-x=0$。该外接圆恒常穿过定点$(1, 0)$。
\end{solution}



\begin{example}{}{}
曲线$\Gamma: x^3+xy^2-y^2=0$。圆$C$过原点$O$与定点$F(4,0)$,并且同$\Gamma$在另两点$A, B$相交。求证:直线$OA$与$OB$的斜率乘积为常数。
\end{example}
\begin{solution}
由动圆$C$经过点$O(0,0)$与$F(4,0)$两点,其圆心必位于直线$x=2$上。设圆$C$的一般方程为$x^2+y^2-4x+Ey=0$。
将圆$C$与曲线$\Gamma$联立以求相交弦属性。对于$\Gamma$方程可化为$x(x^2+y^2) = y^2$。
将圆$C$之二次结构形式代入,得到$x(4x-Ey) = y^2$,展开得$4x^2-Exy-y^2=0$。
两边同除以$x^2$转化为关于直线斜率$k=y/x$的方程,得到$k^2+Ek-4=0$。
该一元二次方程的两根$k_A, k_B$即分别代表交点直线$OA$与$OB$之斜率。由韦达定理直观获取两根之积$k_Ak_B = -4$。
因所得之积无涉变动参数$E$,证明斜率乘积恒等于常值$-4$。该题运用了抛物线焦弦端点对顶点的反演投射属性构造。
\end{solution}

综上,复平面反演为许多经典几何问题提供了降维工具。例如托勒密定理,通过将反演中心置于圆内接四边形的一个顶点,圆被反演变换为一条直线,定理在反演平面中直接退化为共线三点之间的距离线段相加定理$|B'D'| = |B'C'| + |C'D'|$,代入距离反演公式$|w_X - w_Y| = \frac{|z_X - z_Y|}{|z_X||z_Y|}$并整理即可得证。上述证明避免了传统几何方法中繁复的辅助线构造与推导,体现了变换群视角下的结构优势。

题目:曲线 $\Gamma: x^3 + xy^2 - 3x^2 + 2y^2 = 0$。动圆 $C$ 过原点 $O(0,0)$ 与定点 $F(4,0)$,并且与曲线 $\Gamma$ 在另两点 $A, B$ 相交。求证:直线 $OA$ 与 $OB$ 的斜率乘积为常数。

证明:由于动圆 $C$ 经过 $O(0,0)$ 和 $F(4,0)$,其圆心必然在直线 $x=2$ 上。设动圆 $C$ 的一般方程为 $x^2 + y^2 - 4x + Ey = 0$。将其改写为 $x^2 + y^2 = 4x - Ey$,并代入曲线 $\Gamma$ 的方程中。将 $\Gamma$ 变形为 $x(x^2 + y^2) - 3x^2 + 2y^2 = 0$,代入后得到:$$x(4x - Ey) - 3x^2 + 2y^2 = 0$$展开并合并同类项:$$4x^2 - Exy - 3x^2 + 2y^2 = 0 \implies x^2 - Exy + 2y^2 = 0$$两边同除以 $x^2$,化为关于直线斜率 $k = \frac{y}{x}$ 的一元二次方程:$$2k^2 - Ek + 1 = 0$$设直线 $OA, OB$ 的斜率分别为 $k_A, k_B$,它们即为该方程的两个实根。由韦达定理可知,两根之积为:$$k_A k_B = \frac{1}{2}$$因为该乘积与动参数 $E$ 无关,故证明直线 $OA$ 与 $OB$ 的斜率乘积恒为常数 $\frac{1}{2}$。

题目二:证明斜率和为定值(椭圆弦性质的转化)题目:曲线 $\Gamma: x^3 + xy^2 - x^2 - 2y^2 = 0$。动圆 $C$ 过原点 $O(0,0)$ 与定点 $P(0,2)$,并且与曲线 $\Gamma$ 相交于另外两点 $A, B$。求证:直线 $OA$ 与 $OB$ 的斜率之和为定值。

证明:动圆 $C$ 经过 $O(0,0)$ 和 $P(0,2)$,说明其圆心在直线 $y=1$ 上。设动圆 $C$ 的方程为 $x^2 + y^2 + Dx - 2y = 0$。将其改写为 $x^2 + y^2 = 2y - Dx$。把 $\Gamma$ 变形为 $x(x^2 + y^2) - x^2 - 2y^2 = 0$,并将上式代入:$$x(2y - Dx) - x^2 - 2y^2 = 0$$展开得到:$$2xy - Dx^2 - x^2 - 2y^2 = 0 \implies -2y^2 + 2xy - (D + 1)x^2 = 0$$两边同除以 $-x^2$,转化为关于斜率 $k = \frac{y}{x}$ 的方程:$$2k^2 - 2k + (D + 1) = 0$$该方程的两根 $k_A, k_B$ 对应直线 $OA$ 与 $OB$ 的斜率。由韦达定理可知,两根之和为:$$k_A + k_B = -\frac{-2}{2} = 1$$定值与动参数 $D$ 无关,故直线 $OA$ 与 $OB$ 的斜率之和恒为 $1$。

\begin{example}{}{}
    已知曲线 $\Gamma: x^3 + xy^2 + 2x^2 + xy + y^2 = 0$。动圆 $C$ 过原点 $O(0,0)$ 与定点 $P(-1, 1)$,并且与曲线 $\Gamma$ 在另外两点 $A, B$ 相交。求证:直线 $OA$ 与 $OB$ 的斜率满足关系式 $k_A k_B + k_A + k_B$ 为常数,并求出该常数的值。
\end{example}
\begin{solution}
    设经过原点 $O(0,0)$ 的动圆 $C$ 的一般方程为 $x^2 + y^2 + Dx + Ey = 0$。因为动圆 $C$ 经过定点 $P(-1, 1)$,将其坐标代入圆的方程可得:$$(-1)^2 + 1^2 - D + E = 0$$化简得到参数 $D$ 与 $E$ 的约束关系:$$E - D = -2$$现将动圆方程改写为 $x^2 + y^2 = -Dx - Ey$。同时,将曲线 $\Gamma$ 的方程提取公因式变形为:$$x(x^2 + y^2) + 2x^2 + xy + y^2 = 0$$将圆的二次形式代入 $\Gamma$ 中以求交点性质:$$x(-Dx - Ey) + 2x^2 + xy + y^2 = 0$$展开并合并同类项:$$-Dx^2 - Exy + 2x^2 + xy + y^2 = 0$$$$(2 - D)x^2 + (1 - E)xy + y^2 = 0$$在 $x \neq 0$ 的前提下(即交点不在 $y$ 轴上),等式两边同除以 $x^2$,转化为关于交点与原点连线斜率 $k = \frac{y}{x}$ 的一元二次方程:$$k^2 + (1 - E)k + (2 - D) = 0$$该方程的两个根 $k_A, k_B$ 即为直线 $OA$ 与 $OB$ 的斜率。由韦达定理可得:$$k_A + k_B = -(1 - E) = E - 1$$$$k_A k_B = 2 - D$$我们需要求证的代数式为 $k_A k_B + k_A + k_B$,代入上述结果:$$k_A k_B + k_A + k_B = (2 - D) + (E - 1) = 1 + (E - D)$$将前面得到的圆系约束条件 $E - D = -2$ 代入上式:$$k_A k_B + k_A + k_B = 1 + (-2) = -1$$由于最终结果与动圆的参数 $D, E$ 无关,故证明该组合式恒为常数 $-1$。\end{solution}

\begin{example}{}{}
    已知曲线 $\Gamma: x^3 + xy^2 - y^2 = 0$。动直线 $L$ 不经过原点 $O(0,0)$,且与曲线 $\Gamma$ 交于 $A, B, C$ 三点。记直线 $OA, OB, OC$ 的斜率分别为 $k_A, k_B, k_C$。若这三条直线的斜率满足条件:$$k_A + k_B + k_C + k_A k_B k_C = 2$$求证:动直线 $L$ 恒过一个定点,并求出该定点坐标。\end{example}
\begin{solution}设不过原点的动直线 $L$ 的方程为截距一般式 $ux + vy = 1$。将曲线 $\Gamma$ 的方程变形提取公因式,得到 $x(x^2 + y^2) = y^2$。为了寻找交点 $A, B, C$ 与原点连线的斜率属性,我们将直线 $L$ 的方程代入 $\Gamma$ 中进行齐次化(Homogenization)构造。将右侧的 $y^2$ 乘以 $1$(即 $ux + vy$):$$x(x^2 + y^2) = y^2(ux + vy)$$展开并整理,得到一个关于 $x, y$ 的三次齐次方程:$$x^3 + x y^2 - u x y^2 - v y^3 = 0$$$$x^3 + (1 - u)x y^2 - v y^3 = 0$$因为直线 $L$ 不经过原点,且交点显然不全在 $y$ 轴上(可设 $x \neq 0$)。我们在等式两边同除以 $x^3$,并令斜率 $k = \frac{y}{x}$,即可得到一个关于斜率 $k$ 的一元三次方程:$$1 + (1 - u)k^2 - v k^3 = 0$$改写为标准形式:$$v k^3 + (u - 1)k^2 - 1 = 0$$由题意,该方程的三个根即为直线 $OA, OB, OC$ 的斜率 $k_A, k_B, k_C$。请注意,该方程中一次项 $k$ 的系数为 $0$。根据一元三次方程的韦达定理(Vieta's formulas),我们可以提取出根与系数的关系:$$k_A + k_B + k_C = -\frac{u - 1}{v} = \frac{1 - u}{v}$$$$k_A k_B k_C = -\frac{-1}{v} = \frac{1}{v}$$已知题目给定的斜率约束条件为:$$k_A + k_B + k_C + k_A k_B k_C = 2$$将韦达定理的结果代入上式:$$\frac{1 - u}{v} + \frac{1}{v} = 2$$合并同分母的分子:$$\frac{2 - u}{v} = 2 \implies 2 - u = 2v$$移项整理,得到参数 $u$ 和 $v$ 之间的线性约束关系:$$u + 2v = 2$$等式两边同除以 $2$,化为:$$u\left(\frac{1}{2}\right) + v(1) = 1$$将此时的 $u, v$ 结构与我们最初设定的动直线方程 $ux + vy = 1$ 进行对比。这说明,无论参数 $u$ 和 $v$ 如何变动,只要它们满足上述约束,当 $x = \frac{1}{2}$ 且 $y = 1$ 时,等式 $ux + vy = 1$ 将恒等成立。因此,证明动直线 $L$ 恒过定点 $\left(\frac{1}{2}, 1\right)$。
\end{solution}
题目背后的“反演透视”这道题的设计完全贯彻了你提到的**“圆过定点 $\rightarrow$ 直线过定点”**的跨维度映射:在原像平面(Conic Plane)中:存在一条抛物线 $y^2 = x$。我们让一个动圆 $C$ 经过原点 $O(0,0)$,并且该圆与抛物线相交于原点及另外三点。由于某种极其对称的几何约束(在这里表现为斜率方程),这个动圆必定会经过另一个定点 $(2, 1)$。反演操作(Inversion):我们以原点 $O$ 为反演中心进行反演变换。在目标平面(Problem Plane)中:抛物线反演成了题目中的三次曲线 $\Gamma: x^3 + xy^2 - y^2 = 0$。过原点的动圆,瞬间被“拉直”,反演成了题目中不过原点的动直线 $L$。原像中圆所经过的定点 $(2, 1)$,经过反演坐标变换 $X = \frac{x}{x^2+y^2}, Y = \frac{y}{x^2+y^2}$ 后,完美地变成了直线 $L$ 所经过的定点 $\left(\frac{1}{2}, 1\right)$。你是否觉得,这种利用代数里的“齐次化”来处理几何上的“反演变换”,有一种将复杂降维打击的数学暴力美学?