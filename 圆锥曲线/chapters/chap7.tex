\chapter{杂题}
\begin{example}{浙江风味圆锥曲线}{}
椭圆$C_1:\frac{x^2}4+\frac{y^2}3=1$,抛物线$C_2:y^2=2px(p>0).$斜率为正的直线$l$与椭圆$C_1$和抛物线$C_2$均恰有一个公共点,分别为点$A,B$,椭圆$C_1$交抛物线$C_2$ 于点 $C,D.$\\(1)若直线$CD$经过椭圆$C_1$的右焦点$F$,求$p$的值;\\(2)设直线$CD$交直线$l$于点$E$,求$\triangle AOE$面积的最小值.
\end{example}
\begin{solution}
已知椭圆 $C_1:\frac{x^2}{4}+\frac{y^2}{3}=1$ 的右焦点为 $F(c, 0)$,其中 $c = \sqrt{4-3} = 1$,即 $F(1,0)$。由对称性可知,椭圆 $C_1$ 与抛物线 $C_2: y^2=2px$ 的交点 $C,D$ 所在的直线 $CD$ 垂直于 $x$ 轴。若直线 $CD$ 经过右焦点 $F(1,0)$,则点 $C,D$ 的横坐标必定为 $x=1$。将 $x=1$ 代入椭圆 $C_1$ 的方程中,解得 $y^2 = \frac{9}{4}$。由于点 $C, D$ 亦在抛物线 $C_2$ 上,将 $x=1, y^2=\frac{9}{4}$ 代入抛物线方程 $y^2=2px$ 中,得 $\frac{9}{4} = 2p$,解得 $p = \frac{9}{8}$。

对于 $\triangle AOE$ 面积最小值的求解,首先对坐标系作伸缩变换,令 $x = 2X, y = \sqrt{3}Y$。在此仿射变换下,椭圆 $C_1$ 转化为单位圆 $C_1': X^2+Y^2=1$,抛物线 $C_2$ 转化为 $3Y^2=4pX$。令参量 $p' = \frac{2p}{3}$,则抛物线方程变为 $C_2': Y^2=2p'X$。直线 $l$ 依然是 $C_1'$ 和 $C_2'$ 的公切线,设其交 $X$ 轴于点 $T(t, 0)$。由伸缩变换的性质,原面积与新面积的关系满足 $S_{\triangle AOE} = 2\sqrt{3} S_{\triangle A'O'E'}$。

为建立公切线的统一方程,在此引入并证明二次曲线的切线对引理:设二次曲线方程为 $S(x,y)=0$,曲线外一点 $P_0(x_0,y_0)$。记将点 $P_0$ 代入曲线方程所得的常数值为 $S_{11}$,点 $P_0$ 对应的极线方程为 $T(x,y)=0$。则从 $P_0$ 引向该二次曲线的两条切线所构成的整体方程为 $T^2 = S \cdot S_{11}$。

证法一(齐次坐标与参数方程法):
将二次曲线方程写为齐次矩阵形式 $S = \mathbf{X}^T M \mathbf{X} = 0$,其中 $\mathbf{X} = (x, y, 1)^T$,$M$ 为对称矩阵。设已知点坐标对应的向量为 $\mathbf{X}_0 = (x_0, y_0, 1)^T$,则 $S_{11} = \mathbf{X}_0^T M \mathbf{X}_0$,极线 $T = \mathbf{X}_0^T M \mathbf{X}$。设切线上任意一点为 $\mathbf{X}$,过 $\mathbf{X}_0$ 与 $\mathbf{X}$ 的直线上任意一点可表示为 $\mathbf{X}_Q = \mathbf{X} + \lambda \mathbf{X}_0$。因该直线与二次曲线相切,代入曲线方程得 $(\mathbf{X} + \lambda \mathbf{X}_0)^T M (\mathbf{X} + \lambda \mathbf{X}_0) = 0$。展开并利用矩阵对称性化简,可得关于 $\lambda$ 的一元二次方程 $\lambda^2 S_{11} + 2\lambda T + S = 0$。由相切条件可知该直线与曲线仅有一个交点,即该方程具有唯一实数解,故其判别式 $\Delta = (2T)^2 - 4 S_{11} S = 0$。整理化简即得 $T^2 = S \cdot S_{11}$。

证法二(过交点的曲线系法):
设从 $P_0$ 引出的两条切线与曲线 $S=0$ 切于点 $A, B$,则直线 $AB$ 即为切点弦(极线),其方程为 $T(x,y)=0$。这两条切线在代数上构成一退化的二次曲线,且与原曲线 $S=0$ 共用交点 $A, B$ 并在此处相切。根据曲线系理论,与 $S=0$ 在其与 $T=0$ 交点处相切的二次曲线系可表示为 $S(x,y) + \lambda T(x,y)^2 = 0$。由于切线对图形必然经过公共端点 $P_0(x_0, y_0)$,将 $P_0$ 的坐标代入该曲线系方程。已知将极点坐标代入极线方程所得的值等于极点代入原曲线方程的值,即 $T(x_0, y_0) = S_{11}$,代入后方程化为 $S_{11} + \lambda S_{11}^2 = 0$。因 $P_0$ 在曲线外,有 $S_{11} \neq 0$,解得 $\lambda = -\frac{1}{S_{11}}$。将参量 $\lambda$ 代回曲线系方程,得 $S(x,y) - \frac{1}{S_{11}} T(x,y)^2 = 0$,同乘 $S_{11}$ 并移项,即证得 $T^2 = S \cdot S_{11}$。

由上述引理,设点 $T(t, 0)$,由于 $l$ 为公切线,从 $T$ 点引出的两条切线满足如下方程:
对于单位圆 $C_1'$,代入公式可得方程为 $(tX-1)^2 = (t^2-1)(X^2+Y^2-1)$。
对于抛物线 $C_2'$,其极线为 $-p'(X+t)=0$,代入点 $T$ 得 $S_{11} = -2p't$,其切线对方程为 $(-p'(X+t))^2 = (-2p't)(Y^2-2p'X)$,化简得 $p'(X+t)^2 = -2t(Y^2-2p'X)$。
因直线 $l$ 是公切线且 $T$ 位于 $X$ 轴上,两方程必然表示同一对直线。提取两式中 $X^2$ 和 $Y^2$ 的系数并令其成比例,圆切线对方程化简后系数比为 $\frac{1}{1-t^2}$,抛物线切线对方程化简后系数比为 $\frac{p'}{-2t}$。令两式相等,解得 $p' = \frac{2t}{1-t^2}$。

直线 $CD$ 为交点弦,联立圆 $X^2+Y^2=1$ 与抛物线 $Y^2=2p'X$,消去 $Y$ 可得交点横坐标 $X_0$ 满足 $X_0^2 + 2p'X_0 - 1 = 0$,解得 $p' = \frac{1-X_0^2}{2X_0}$。将两个关于 $p'$ 的表达式联立,得 $\frac{1-X_0^2}{2X_0} = \frac{2t}{1-t^2}$,交叉相乘并展开重组得 $(X_0+t)^2 = (1-X_0t)^2$。由几何关系可知交点 $T$ 在抛物线左侧,即 $t<0$,且交点 $X_0 \in (0,1)$,故开方取合理符号解得 $t = \frac{X_0+1}{X_0-1}$。

在变换后的图形中,设圆心为 $O'(0,0)$,$A'$ 为切点,则 $O'A' \perp A'E'$ 且半径 $|O'A'|=1$。在直角三角形 $\triangle O'A'E'$ 中,由勾股定理可得其面积的平方为 $(S_{\triangle A'O'E'})^2 = \frac{1}{4}|O'A'|^2|A'E'|^2 = \frac{1}{4} \times 1^2 \times (|O'E'|^2 - |O'A'|^2) = \frac{1}{4}(X_0^2+Y_0^2-1)$。因交点 $E'(X_0, Y_0)$ 位于公切线上,必然满足圆的切线对方程,代入整理得 $X_0^2+Y_0^2-1 = \frac{(tX_0-1)^2}{t^2-1}$。将 $t = \frac{X_0+1}{X_0-1}$ 代入该式,计算分子得 $tX_0-1 = \frac{X_0^2+1}{X_0-1}$,计算分母得 $t^2-1 = \frac{4X_0}{(X_0-1)^2}$。代入面积公式化简得 $(S_{\triangle A'O'E'})^2 = \frac{(X_0^2+1)^2}{16X_0}$。

构造函数 $f(X_0) = \frac{(X_0^2+1)^2}{16X_0}$,其中 $X_0 \in (0, 1)$。对其求导得 $f'(X_0) = \frac{16(X_0^2+1)(3X_0^2-1)}{256X_0^2}$。令 $f'(X_0) = 0$,在区间 $(0,1)$ 内解得唯一的驻点 $X_0 = \frac{\sqrt{3}}{3}$。当 $X_0 \in (0, \frac{\sqrt{3}}{3})$ 时,$f'(X_0) < 0$,$f(X_0)$ 单调递减;当 $X_0 \in (\frac{\sqrt{3}}{3}, 1)$ 时,$f'(X_0) > 0$,$f(X_0)$ 单调递增。由此可知 $f(X_0)$ 在 $X_0 = \frac{\sqrt{3}}{3}$ 处取得最小值,计算得 $f\left(\frac{\sqrt{3}}{3}\right) = \frac{\sqrt{3}}{9}$,即 $S_{\triangle A'O'E'}$ 的最小值为 $\frac{\sqrt[4]{3}}{3}$。

根据仿射变换还原面积的几何倍数关系,可得原三角形面积的最小值为 $S_{min} = 2\sqrt{3} \times S_{\triangle A'O'E'} = 2\sqrt{3} \times \frac{\sqrt[4]{3}}{3} = \frac{2}{\sqrt[4]{3}}$。故 $\triangle AOE$ 面积的最小值为 $\frac{2}{\sqrt[4]{3}}$。
\end{solution}