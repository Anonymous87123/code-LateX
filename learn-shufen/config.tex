% ========== 基本包加载 ==========
% 1. 基础宏包
\usepackage{geometry}
\usepackage{fontspec}
\usepackage{amsmath, amsthm, amssymb}
\usepackage{extarrows}
\allowdisplaybreaks
\usepackage{mathrsfs}
\usepackage{enumitem}
\usepackage{graphicx}
\usepackage{array}
\usepackage{ulem}
\usepackage{caption}
\usepackage{tocloft}
% 2. 图形和颜色宏包
\usepackage[dvipsnames]{xcolor}
\usepackage{tikz}
\usetikzlibrary{shapes.geometric}
\usepackage[most]{tcolorbox}
\tcbuselibrary{theorems}
% 3. 页眉页脚宏包
\usepackage{fancyhdr}
\usepackage{lastpage}
% 4. 超链接和书签
\usepackage{hyperref}
\usepackage{bookmark}
% 5. 智能引用
\usepackage{cleveref}
% ========================
% 1. 页面布局设置(窄边距)
\geometry{
  a4paper,
  top=15mm,
  bottom=10mm,
  left=15mm,
  right=15mm,
  headheight=25pt,
  headsep=8mm,
  footskip=15mm,
  includehead,
  includefoot
}
% 2. 字体设置(修改部分)
% 设置全局英文字体
\setmainfont{Times New Roman}
\setsansfont{Arial}
\setmonofont{Consolas}
% 定义字体命令(保持不变)
\newcommand{\kt}{\kaishu} % 楷体
\newcommand{\st}{\songti} % 宋体
\newcommand{\htbf}{\heiti\bfseries} % 黑体加粗
\newcommand{\e}{\text{e}}
\newcommand{\dd}{\text{d}}
\definecolor{deepblue}{HTML}{003366}
\definecolor{lightblue}{HTML}{E5F6FF}
\definecolor{deepgreen}{HTML}{006633}
\definecolor{lightgreen}{HTML}{E5F6E5}
\definecolor{deeppurple}{HTML}{660066}
\definecolor{lightpurple}{HTML}{F6E5F6}
\definecolor{deepred}{HTML}{990000}
\definecolor{lightred}{HTML}{FFE5E5}
\definecolor{deeporange}{HTML}{CC6600}
\definecolor{lightorange}{HTML}{FFF6E5}
\definecolor{deepgray}{HTML}{333333}
\definecolor{lightgray}{HTML}{F6F6F6}
\definecolor{deepbrown}{HTML}{8B4513}
\definecolor{lightbrown}{HTML}{FFE4B5}
\definecolor{lightyellow}{HTML}{FFFFE0}
\definecolor{darkyellow}{HTML}{CCCC00}
\definecolor{lightcyan}{HTML}{E0FFFF}
\definecolor{darkcyan}{HTML}{008B8B}
\definecolor{lightpink}{HTML}{FFB6C1}
\definecolor{darkpink}{HTML}{FF1493}
\definecolor{lavender}{HTML}{E6E6FA}
\definecolor{thistle}{HTML}{D8BFD8}
\definecolor{lightblue}{HTML}{ADD8E6}
\definecolor{darkblue}{HTML}{00008B}
\definecolor{lightgray}{HTML}{D3D3D3}
\definecolor{darkgray}{HTML}{A9A9A9}
\newcounter{theoremcounter}[section]
% 定理环境
\newtcbtheorem[use counter=theoremcounter,number within=section]{theorem}{定理}{
    colback=lightgreen, %背景颜色
    colframe=deepgreen, %边框颜色
    colbacktitle=deepgreen, %标题框颜色
    coltitle=white,  %标题颜色
    fonttitle=\upshape\color{white}, %标题字体
    fontupper=\rmfamily, %标题内容字体
    separator sign none,
    top=8pt,
    attach boxed title to top left={yshift=-2mm,xshift=5mm},
    description delimiters={ }{ },
    enhanced % 边框粗细
}{thm}
% 引理环境
\newtcbtheorem[use counter=theoremcounter,number within=section]{lemma}{引理}{
    colback=lightpurple,
    colframe=deeppurple,
    colbacktitle=deeppurple,
    coltitle=white,
    fonttitle=\upshape\color{white},
    fontupper=\rmfamily,
    separator sign none,
    top=8pt,
    attach boxed title to top left={yshift=-2mm,xshift=5mm},
    description delimiters={ }{ },
    enhanced
}{lem}
% 定义环境
\newtcbtheorem[number within=section,number within=section]{definition}{定义}{
    colback=lightred,
    colframe=deepred,
    colbacktitle=deepred,
    coltitle=white,
    fonttitle=\upshape\color{white},
    fontupper=\rmfamily,
    separator sign none,
    top=8pt,
    attach boxed title to top left={yshift=-2mm,xshift=5mm},
    description delimiters={ }{ },
    enhanced
}{def}

% 例题和解共享计数器
\newcounter{examplecounter}[section]
% 例题环境
\newtcbtheorem[use counter=examplecounter,number within=section]{example}{例题}{
    colback=lightblue,
    colframe=deepblue,
    colbacktitle=deepblue,
    coltitle=white,
    fonttitle=\upshape\color{white},
    fontupper=\rmfamily,
    separator sign none,
    top=8pt,
    attach boxed title to top left={yshift=-2mm,xshift=5mm},
    description delimiters={ }{ },
    enhanced  
}{ex}
% 结论环境(宋体)
\newtcbtheorem[number within=section]{conclusion}{结论}{
    colback=lightorange,
    colframe=deeporange,
    colbacktitle=deeporange,
    coltitle=white,
    fonttitle=\upshape\color{white},
    fontupper=\rmfamily,
    separator sign none,
    top=8pt,
    attach boxed title to top left={yshift=-2mm,xshift=5mm},
    description delimiters={ }{ },
    enhanced
}{con}
\newtcbtheorem[number within=section]{property}{性质}{
    colback=lightgray,
    colframe=deepgray,
    colbacktitle=deepgray,
    coltitle=white,
    fonttitle=\upshape\color{white},
    fontupper=\rmfamily,
    separator sign none,
    top=8pt,
    attach boxed title to top left={yshift=-2mm,xshift=5mm},
    description delimiters={ }{ },
    enhanced       % 标题加粗与其他环境统一
}{prop}
\newtcbtheorem[number within=section]{criterion}{准则}{
    colback=lightbrown,
    colframe=deepbrown,
    colbacktitle=deepbrown,
    coltitle=white,
    fonttitle=\upshape\color{white},
    fontupper=\rmfamily,
    separator sign none,
    top=8pt,
    attach boxed title to top left={yshift=-2mm,xshift=5mm},
    description delimiters={ }{ },
    enhanced
}{crit}
\newtcbtheorem[use counter=theoremcounter,number within=section]{corollary}{推论}{
    colback=lightcyan,
    colframe=darkcyan,
    colbacktitle=darkcyan,
    coltitle=white,
    fonttitle=\upshape\color{white},
    fontupper=\rmfamily,
    separator sign none,
    top=8pt,
    attach boxed title to top left={yshift=-2mm,xshift=5mm},
    description delimiters={ }{ },
    enhanced       % 标题加粗与其他环境统一
}{cor}
% 传统样式环境(无背景色)
\newtheoremstyle{plain-chinese}% 名称
  {6pt}% 上方空白
  {6pt}% 下方空白
  {\st}% 正文字体(宋体)
  {}% 缩进
  {\heiti}% 标题字体(黑体加粗)
  {.}% 标题后标点
  { }% 标题后空白
  {}% 标题说明
\theoremstyle{plain-chinese}
\newtheorem{solution}{解}[section] % 使用与例题相同的计数器
\newtheorem{remark}{注}[section]
% 4. 智能引用设置(保持不变)
% ========================
\crefname{theorem}{定理}{定理}
\crefname{lemma}{引理}{引理}
\crefname{definition}{定义}{定义}
\crefname{example}{例题}{例题}
\crefname{conclusion}{结论}{结论}
\crefname{solution}{解}{解}
\crefname{remark}{注}{注}
\crefname{property}{性质}{性质}
\crefname{criterion}{准则}{准则}
\crefname{proof}{证明}{证明}

% 5. 数学字体设置(修改)
\usepackage{unicode-math} % 更好的数学字体支持
\setmathfont{Latin Modern Math} % 使用默认数学字体
% ========== 自定义命令(保持不变) ==========
\newcommand{\R}{\mathbb{R}} % 实数集
\newcommand{\C}{\mathbb{C}} % 复数集
\newcommand{\Z}{\mathbb{Z}} % 整数集
\newcommand{\N}{\mathbb{N}} % 自然数集

% ========== 图形路径设置(保持不变) ==========
\graphicspath{{./flg/}} % 图片路径

% 6. 页眉页脚设置
% ========================
\usepackage{fancyhdr}
\usepackage{lastpage} % 获取总页数
\pagestyle{fancy}
\fancyhf{} % 清除所有页眉页脚设置

% 通用设置
\fancyhead[L]{\small\kt 工科数学分析作业} % 左边:书名(楷体)
\fancyhead[C]{\small\st 华南理工大学}
\fancyhead[R]{\small\st 夏同202530451676} % 中间:声明(宋体)
\fancyfoot[C]{\thepage} % 居中页码(自定义样式)

% 正文部分设置
\fancyhead[R]{\small\st\rightmark} % 右边:章节名称(宋体)

% 前言部分设置(使用罗马数字页码)
\fancypagestyle{frontmatter}{
    \fancyhf{}
    \fancyhead[L]{\small\kt 工科数学分析作业}
    \fancyhead[C]{\small\st 前言}
    \fancyhead[R]{} % 前言部分无章节名称
    \fancyfoot[C]{\thepage}
    \renewcommand{\headrulewidth}{0.4pt} % 页眉线
    \renewcommand{\footrulewidth}{0pt} % 无页脚线
    \pagenumbering{Roman} % 罗马数字页码
}

% 目录部分设置(使用罗马数字页码,延续前言页码)          % 去掉标题下方的横线
\fancypagestyle{tocmatter}{
    \fancyhf{}
    \fancyhead[L]{\small\kt 工科数学分析作业}
    \fancyhead[C]{\small\st 目录} % 居中显示"目录"
    \fancyhead[R]{\small\st 夏同} 
    \fancyfoot[C]{\thepage}
    \renewcommand{\headrulewidth}{0.4pt}
    \renewcommand{\footrulewidth}{0pt}
    % 注意:这里不重置页码,延续前面的罗马数字
}
% 正文部分设置(使用阿拉伯数字页码)
\fancypagestyle{mainmatter}{
    \fancyhf{}
    \fancyhead[L]{\small\kt 工科数学分析作业}
    \fancyhead[C]{\small\st 计算机科学与工程学院~计类1班~夏同}
    \fancyhead[R]{\small\st\rightmark}
    \fancyfoot[C]{\thepage}
    \renewcommand{\headrulewidth}{0.4pt} % 页眉线
    \renewcommand{\footrulewidth}{0pt} % 无页脚线
    \pagenumbering{arabic} % 阿拉伯数字页码
}
% 设置章节标记格式
\renewcommand{\chaptermark}[1]{\markboth{#1}{}}
\renewcommand{\sectionmark}[1]{\markright{\thesection.\ #1}}

% 8. 章节标题字体设置
\ctexset{
    chapter = {
        format = \centering, % 整体居中
        nameformat = \kaishu\LARGE, % 编号部分黑体加粗
        titleformat = \songti\LARGE, % 标题部分宋体
        aftername = \quad, % 编号和标题之间的间距
        beforeskip = 30pt, % 标题前的垂直间距
        afterskip = 20pt, % 标题后的垂直间距
        name = {第,章}, % 中文章节编号格式
        number = \chinese{chapter}, % 使用中文数字
    },
    section = {
        format = \raggedright, % 左对齐
        nameformat = \heiti\bfseries\large, % 编号部分黑体加粗
        titleformat = \songti\large, % 标题部分宋体
        aftername = \quad, % 编号和标题之间的间距
        beforeskip = 15pt, % 标题前的垂直间距
        afterskip = 6pt, % 标题后的垂直间距
    },
    subsection = {
        format = \raggedright, % 左对齐
        nameformat = \heiti\bfseries\normalsize, % 编号部分黑体加粗
        titleformat = \songti\normalsize, % 标题部分宋体
        aftername = \quad, % 编号和标题之间的间距
        beforeskip = 6pt, % 标题前的垂直间距
        afterskip = 3pt, % 标题后的垂直间距
    }
}