\chapter{集合,映射与函数}
\section{第1周作业}
\begin{example}{讨论下列函数的奇偶性}{}\vspace{-10pt}
\[\begin{array}{l@{\quad\quad}l@{\quad\quad}l}
    (1)y=3x-x^3&(2)2+3x-x^3&(3)y=\sqrt[3]{(1+x)^2}+\sqrt[3]{(1-x)^2}\\
    (4)y=\dfrac{e^x+e^{-x}}{2}&(5)y=\sqrt{x(2-x)}&(6)y=2^{-x}\\
    (7)f(x)=\begin{cases}x-1,&x<0\\0,&x=0\\x+1,&x>0\end{cases}&(8)y=\ln (x+\sqrt{x^2+1})
\end{array}\]\end{example}
\begin{solution}
    (1)由于$f(x)+f(-x)=3x+3(-x)-x^3-(-x)^3=0$,故为奇函数

    (2)由于$f(x)+f(-x)=2+3x+3(-x)+2-x^3-(-x)^3=4$,不为奇函数;而$4\ne 2f(x)\Rightarrow f(x)\ne f(-x)$,故为非奇非偶函数

    (3)由于$f(x)=\sqrt[3]{(1+x)^2}+\sqrt[3]{(1-x)^2}=\sqrt[3]{(1-x)^2}+\sqrt[3]{(1+x)^2}=f(-x)$,故为偶函数

    (4)由于$f(x)=\frac{e^x+e^{-x}}{2}=\frac{e^{-x}+e^x}{2}=f(-x)$,故为偶函数

    (5)由于$f(x)=\sqrt{x(2-x)}\neq\sqrt{-x(2+x)}=f(-x)$,故不为偶函数,由于$f(x)+f(-x)=\sqrt{x(2-x)}+\sqrt{-x(2+x)}\neq 0$,故为非奇非偶函数

    (6)由于$\begin{cases}f(x)+f(-x)=2^{-x}+2^{-(-x)}=2^{-x}+2^{x}\ne 0\\f(x)=2^{-x}\neq2^{-(-x)}=f(-x)\end{cases}$,故为非奇非偶函数

    (7)由于$\begin{cases}f(0)=0\\f(x)+f(-x)=x-1+(-x)+1=0\\f(x)\neq f(-x)\end{cases}$,故为奇函数

    (8)由于$f(x)+f(-x)=\ln (x+\sqrt{x^2+1})+\ln (-x+\sqrt{x^2+1})=\ln (-x^2+(x^2+1))=0$,故为奇函数
\end{solution}
\begin{example}{研究函数的单调性}{}
    (1)$y=ax+b\quad(2)y=ax^2+bx+c\quad(3)y=x^3\quad(4)y=a^x$
\end{example}
\begin{solution}
    (1)若$a\ge0$,则$y$单调递增;若$a<0$,则$y$单调递减;若$a>0$,则$y$严格单调递增

    (2)若$a>0$,则$y$先严格单调减后严格单调增,若$a<0$,则$y$先严格单调增后严格单调减,若$a=0$,则当$b>0$时,$y$单调递增,当$b<0$时,$y$单调递减;若$a=b=0$,则$y$非严格单调递增

    (3)若$x_1>x_2$,则$f(x_1)-f(x_2)=x_1^3-x_2^3=(x_1-x_2)(x_1^2+x_2^2-x_1x_2)>x_1-x_2>0$故单调递增

    (4)需限定$a>0$,则当$a>1$时,$y$单调递增,当$a<1$时,$y$单调递减;若$a=1$,则$y=1$非严格单调递增;
\end{solution}
\begin{example}{哪些是周期函数?如果是说明其周期,并说明有无最小周期,有就求出来}{}
    \[\begin{array}{l@{\quad\quad}l@{\quad\quad}l}
    (1)y=\sin^2x&(2)y=\sin x^2&(3)y=\cos(x-2)\\
    (4)y=A\cos\lambda x+B\sin\lambda x&(5)y=x-[x]&(6)y=\tan|x|
    \end{array}\]
\end{example}
\begin{solution}
    (1)是周期函数,周期为$k\pi,(k\in\mathbb{Z})$,最小正周期为$\pi$

    (2)不是周期函数,因为\[\sin(x+T)^2-\sin x^2=2\cos\dfrac{(x+T)^2+x^2}{2}\sin\dfrac{(x+T)^2-x^2}{2}=2\cos\dfrac{(x+T)^2+x^2}{2}\sin\dfrac{2x+T^2}{2}\ne 0\]则这样的$T$不存在.

    (3)是周期函数,周期为$2k\pi,(k\in\mathbb{Z})$,最小正周期为$2\pi$.

    (4)$y=A\cos\lambda x+B\sin\lambda x=\sqrt{A^2+B^2}\sin(\lambda x+\arctan\dfrac{B}{A})$是周期函数,周期为$\frac{2k\pi}{\lambda},(k\in\mathbb{Z},\lambda>0)$,最小正周期为$\frac{2\pi}{\lambda},(\lambda>0)$

    (5)是周期函数,因为$[x]+1 = [x+1]$,则$x+1-[x+1]=x+1-[x]-1=x-[x]$,所以$y=x-[x]$是周期函数,周期为$\mathbb{Z}$,最小正周期为$1$.

    (6)不是周期函数。证明:由于正切函数的一个周期是$\pi$,假设$\tan|x|$也是周期函数,则存在$T>0$使得对于定义域内的任意实数$x$都有$|x|+\pi=|x+T|$,代入$x=-\pi$得到$T=3\pi$,代入$x=0$得到$T=\pi$,矛盾!所以$y=\tan|x|$不是周期函数.
\end{solution}
\begin{example}{证明}{}
    两个奇函数之积为偶函数,奇函数和偶函数之积仍然是奇函数。
\end{example}
\begin{solution}
    (1)设$f(x),g(x)$为两个奇函数,则\[
    f(x)g(x)=(-f(-x))(-g(-x))=f(-x)g(-x)
    \]故两个奇函数之积为偶函数

    (2)设$f(x)$是奇函数,$g(x)$为偶函数,则\[
    f(x)g(x)=(-f(-x))(g(-x))=-f(-x)g(-x)\Leftrightarrow f(x)g(x)+f(-x)g(-x)=0\]故奇函数和偶函数之积仍然是奇函数.
\end{solution}
\begin{example}{证明}{}
    若函数$f(x)$周期为$T(T>0)$,则函数$f(-x)$的周期也是$T$.
\end{example}
\begin{solution}
    设$f(x)$周期为$T$,则$f(x+T)=f(x)\Rightarrow f(-x+T)=f(-x)$,故$f(-x)$的周期也是$T$.
\end{solution}
\begin{example}{证明}{}
    设$f(x)$和$g(x)$都是定义域为$R$的单调函数,求证:$f(g(x))$也是定义域为$R$的单调函数.
\end{example}
\begin{solution}
    由于$f(x),g(x)$是定义域为$R$的单调函数,则:\[
    \forall x_1,x_2\in R,~(x_1-x_2)(f(x_1)-f(x_2))\geq 0,~~
    \forall x_3,x_4\in R,~(x_3-x_4)(g(x_3)-g(x_4))\geq 0,~~\]
    那么一定存在$x_1=g(x_3),x_2=g(x_4)$,则相乘\[
    (g(x_3)-g(x_4))(f(g(x_3))-f(g(x_4)))\ge 0\]
    结合$x_3,x_4\in R,(x_3-x_4)(g(x_3)-g(x_4))\geq 0$就有:\[
    (x_3-x_4)(g(x_3)-g(x_4))^2(f(g(x_3))-f(g(x_4)))\ge 0\Rightarrow 
    (x_3-x_4)(f(g(x_3))-f(g(x_4)))\ge 0\]
    故$f(g(x))$也是定义域为$R$的单调函数.
\end{solution}
\begin{example}{证明}{}
    \[(1)\arctan\dfrac12+\arctan\dfrac13=\dfrac{\pi}{4}.\quad(2)\cos(\arcsin x)=\sqrt{1-x^2}\]
\end{example}
\begin{solution}
    (1)构造复数$z_1=2+i,z_2=3+i\Rightarrow z_1z_2=5+5i$,则:\[
    \arg(z_1)+\arg(z_2)=\arg(z_1z_2)=\dfrac{\pi}{4}    \]
    (2)由于$\sin^2x+\cos^2x=1$,代入$x=\arcsin x$即可得到:
    \[\cos(\arcsin x)=\sqrt{1-x^2}\]
\end{solution}