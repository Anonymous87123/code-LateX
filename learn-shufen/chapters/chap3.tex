\chapter{一元函数微分学}
\section{习题3.1}
\begin{example}{3.1-A-3}{}
下列各式可否成为$f(x)$在$x_{0}$点的导数的定义?请说明理由.\\
(1) $y=f(x)$在$(a,b)$内定义,$x_{0}\in(a,b)$,若极限$\displaystyle\lim_{\Delta x \to 0}\frac{f(x_{0})-f(x_{0}-\Delta x)}{\Delta x}$存在,则称该极限为$f(x)$在$x_{0}$点的导数.\\
(2) $y=f(x)$在$(a,b)$内定义,$x_{0}\in(a,b)$,若极限$\displaystyle\lim_{x \to x_{0}}\frac{f(x)-f(x_{0})}{x-x_{0}}$存在,则称该极限为$f(x)$在$x_{0}$点的导数.\\
(3) $y=f(x)$在$(a,b)$内定义,$x_{0}\in(a,b)$,若极限$\displaystyle\lim_{\Delta x \to 0}\frac{f(x_{0}+\Delta x)-f(x_{0}-\Delta x)}{2\Delta x}$存在,则称该极限为$f(x)$在$x_{0}$点的导数.                                                   
\end{example}
\begin{solution}
(1)可以,因为\[\lim_{\Delta x \to 0}\dfrac{f(x_0)-f(x_0-\Delta x)}{\Delta x}=\lim_{\Delta x \to 0}\dfrac{f(x_0-\Delta x)-f(x_0)}{-\Delta x}=f'(x_0)\]
(2)可以,因为\[\lim_{x \to x_{0}}\dfrac{f(x)-f(x_{0})}{x-x_{0}}=\lim_{x-x_0\to 0}\dfrac{f(x)-f(x_{0})}{x-x_0}=\lim_{x-x_0\to 0}\dfrac{f(x_0+x-x_0)-f(x_0)}{x-x_0}=f'(x_0)\]
(3)不可以,当$f'(x_0)$存在时,该极限等于$f'(x_0)$,这是因为
\begin{align*}
    \lim_{\Delta x \to 0}\frac{f(x_{0}+\Delta x)-f(x_{0}-\Delta x)}{2\Delta x}&=\lim_{\Delta x \to 0}\frac{f(x_{0}+\Delta x)-f(x_{0})+f(x_{0})-f'(x_{0}-\Delta x)}{2\Delta x}\\
    &=\lim_{\Delta x \to 0}\frac{f(x_{0}+\Delta x)-f(x_{0})}{2\Delta x}+\lim_{\Delta x \to 0}\frac{f(x_{0})-f'(x_{0}-\Delta x)}{2\Delta x}\\
    &=\dfrac12 f'(x_0)+\dfrac12 f'(x_0)=f'(x_0)
\end{align*}
但是当$f'(x_0)$不存在,而左右导数存在时,该极限推出的是左导数和右导数的平均值,并非$f'(x_0)$,比如当$f(x)=|x|,x=0$导数不存在,但是\[\lim_{\Delta x \to 0}\frac{f(x_{0}+\Delta x)-f(x_{0}-\Delta x)}{2\Delta x}=\lim_{\Delta\to 0}\dfrac{|\Delta x|-|-\Delta x|}{2\Delta x}=0\]矛盾,所以不可以用作导数的定义.
\end{solution}
\begin{example}{3.1-A-9}{}
    利用定义求函数在$x=0$处的导数$f(x)=\begin{cases}x^2\sin \dfrac1{x},x\neq 0\\0,x=0\end{cases}$.
\end{example}
\begin{solution}
    利用定义,有\[f'(0)=\lim_{h\to 0}\dfrac{f(0+h)-f(0)}{h}=\lim_{h\to 0}h\sin \dfrac1{h}=0\]
\end{solution}
\begin{example}{3.1-B-3}{}
    求证:偶函数的导数是奇函数,奇函数的导数是偶函数;
\end{example}
\begin{solution}
    设$f(x)$是偶函数,所以$f(x)-f(-x)=0$,两边求导得到$f'(x)+f'(-x)=0$,所以偶函数的导数是奇函数;设$f(x)$是奇函数,所以$f(x)+f(-x)=0$,两边求导得到$f'(x)-f'(-x)=0$,所以奇函数的导数是偶函数\\我们也可以考虑定义:设$f(x)$是偶函数,所以\[f'(x)=\lim_{h\to 0}\dfrac{f(x+h)-f(x)}{h}=\lim_{h\to 0}\dfrac{f(-x-h)-f(-x)}{-(-h)}=-f'(-x)\]所以偶函数的导数是奇函数;设$f(x)$是奇函数,所以\[f'(x)=\lim_{h\to 0}\dfrac{f(x+h)-f(x)}{h}=\lim_{h\to 0}\dfrac{f(-x)-f(-x-h)}{-(-h)}=f'(-x)\]所以奇函数的导数是偶函数。
\end{solution}
\begin{example}{3.1-B-4}{}
    求证:周期函数的导数仍然是周期函数
\end{example}
\begin{solution}
    设$f(x)$是周期函数,$T$是周期,$f(x+T)=f(x)$,则两边求导得到$f'(x+T)=f'(x)$,所以$f'(x)$也是周期函数,也可以从定义考虑:\[f'(x)=\lim_{h\to 0}\dfrac{f(x+h)-f(x)}{h}=\lim_{h\to 0}\dfrac{f(x+T+h)-f(x+T)}{h}=f'(x+T)\]所以$f'(x)$是周期函数.
\end{solution}
\section{习题 3.2}
\begin{example}{3.2-A-2}{}
    求导(1)$y=1-\dfrac1x+\dfrac1{x^2}$;(2)$y=\dfrac1{x^3+2x+1}$;(3)$y=\dfrac1{x+\sqrt{x}}$.
\end{example}
\begin{solution}
    (1)$y'=\dfrac1{x^2}-\dfrac2{x^3}$;(2)$y'=\dfrac{-3x^2-2}{(x^3+2x+1)^2}$;(3)$y'=-\dfrac{1+2\sqrt{x}}{2\sqrt{x}(x+\sqrt{x})^2}$
\end{solution}
\begin{example}{3.2-A-4}{}
    求导(1)$y=\ln(x+\sqrt{1+x^2})$;(2)$y=\dfrac{x}{\sqrt{a^2-x^2}}$;(3)$y=\sqrt{x+\sqrt{x+\sqrt{x}}}$.
\end{example}
\begin{solution}
    (1)$y'=\dfrac{1+\frac{2x}{2\sqrt{1+x^2}}}{x+\sqrt{1+x^2}}=\dfrac1{\sqrt{1+x^2}}$;\\
    (2)$y'=\dfrac{\sqrt{a^2-x^2}-x\frac{-2x}{2\sqrt{a^2-x^2}}}{(\sqrt{a^2-x^2})^2}=\dfrac{a^2}{(\sqrt{a^2-x^2})^3}$;\\
    (3)$y'=\dfrac{1+\frac{1+\frac1{2\sqrt{x}}}{2\sqrt{x+\sqrt{x}}}}{2\sqrt{x+\sqrt{x+\sqrt{x}}}}=\dfrac{1+2\sqrt{x}+4\sqrt{x}\sqrt{x+\sqrt{x}}}{8\sqrt{x}\sqrt{x+\sqrt{x}}\sqrt{x+\sqrt{x+\sqrt{x}}}}$.
\end{solution}
\begin{example}{3.2-B-1}{}
    设 $y=f(x)$为严格递增的可导函数,$x=\varphi(y)$是它的反函数.证明:\\
(1) 当 $h\neq 0$时 ,$ f( x+ h) - f( x) = k\neq 0$,若 记 $f( x+ h) = y+ k$, 则$\varphi ( y+ k) = x+ h.$\\
(2)当$k\to0$时,$\dfrac{\varphi(y+k)-\varphi(y)}{k}=\dfrac{h}{k}=\dfrac{h}{y+k-y}=\dfrac{h}{f(x+h)-f(x)}$趋于$\dfrac1{f^{\prime}(x)}.$
\end{example}
\begin{solution}
    (1)由$x=\varphi(f(x))=f(\varphi(x))$,以及 $y=f(x)$为严格递增的可导函数,所以$\varphi'(x)=\dfrac{1}{f'(\varphi(x))}>0$,所以$x=\varphi(y)$为严格递增的可导函数,所以\[f( x+ h) = y+ k\Rightarrow \varphi(f(x+h))=x+h=\varphi ( y+ k)\]\\
    (2)$\displaystyle\lim_{k\to 0}\dfrac{\varphi(y+k)-\varphi(y)}{k}=\lim_{h\to 0}\dfrac{1}{\dfrac{f(x+h)-f(x)}{h}}=\dfrac1{f'(x)}$,这里$h\to 0\Leftrightarrow k\to 0$.
\end{solution}
\begin{example}{3.2-B-2}{}
    $\text{设 }f(x)=x^3+2x^2+3x+1,\text{用 }\varphi\text{表示 }f\text{ 的反函数}.\text{求证}:f(1)=7,\varphi(7)=1.\text{并计算}\varphi^{\prime}(7).$
\end{example}
\begin{solution}
代入得到$f(1)=1+2+3+1=7$,所以$\varphi(7)=1$,由$f(x)=x^3+2x^2+3x+1$可得$f'(x)=3x^2+4x+3$,所以$\varphi^{\prime}(7)=\dfrac{1}{f'(\varphi(7))}=\dfrac{1}{f'(1)}=\dfrac1{10}$.
\end{solution}
\begin{example}{3.2-B-3}{}
    $\text{设 }y=(\arcsin x)^{2},\text{证明}(1-x^{2})y''-xy'=2.$
\end{example}
\begin{solution}
    两边求导数得到$y'=2\arcsin x\cdot \dfrac1{\cos\arcsin x}=2\dfrac{\arcsin x}{\sqrt{1-x^2}}$,所以$\sqrt{1-x^2}y'=2\arcsin x$,再次两边求导得到$\sqrt{1-x^2}y''+\dfrac{-2x}{2\sqrt{1-x^2}}y'=\dfrac2{\sqrt{1-x^2}}$,等价变形就有$(1-x^{2})y''-xy'=2$.
\end{solution}
\begin{example}{3.2-B-4}{}
    求下列函数的$n$阶导数$y^{(n)}$:(1) $y= \frac 1{1- x^{2}}$ ;(2) $y= \sin ^{2}x.$
\end{example}
\begin{solution}
    (1)裂项有$y=\dfrac12\left(\dfrac1{1+x}+\dfrac1{1-x}\right)$,所以$y^{(n)}=\dfrac12\left[\left(\dfrac1{1+x}\right)^{(n)}+\left(\dfrac1{1-x}\right)^{(n)}\right]$,求$n$阶导数有$y^{(n)}=\dfrac12\left[\dfrac{(-1)^nn!}{(1+x)^{n+1}}+\dfrac{n!}{(1-x)^{n+1}}\right]$\\
    (2)$y=\sin^x=\dfrac12(1-\cos 2x),y^{(n)}=\left[\dfrac12-\dfrac12\cos 2x\right]^{(n)}=-2^{n-1}\cos\left(2x+\dfrac{n\pi}2\right)$,诱导公式得到$y^{(n)}=2^{n-1}\sin\left(2x+\dfrac{(n-1)\pi}2\right)$
\end{solution}
\newpage
\section{习题 3.3}
\begin{example}{3.3-A-2}{}
    方程$\e^y+xy+y=2$确定隐函数$y=y(x)$,求$y'(x)'$.
\end{example}
\begin{solution}
    两边求导得到$y'\e^y+y+xy'+y'=0$解得$y'=-\dfrac{y}{\e^y+x+1}$
\end{solution}
\begin{example}{3.3-A-3}{求下列方程所确定的隐函数$y=y(x)$,求$y'(x)'$}
    \[\begin{array}{l@{\quad}l@{\quad}l}
    \text{(1)}\e^x-\e^y+xy=0&\text{(2)}x^2+y^2-\arcsin y=0&\text{(3)}x^y=y^x\\
    \text{(4)}\arctan\dfrac{y}{x}=\ln\sqrt{x^2+y^2}&\text{(5)}x^2-2xy+y^2=2x&\text{(6)}\sqrt{x}+\sqrt{y}=1\\
    \text{(7)}xy^2+\e^y=\cos(x+y^2)&\text{(8)}\ln y=\sqrt{\dfrac{{1-x}}{1+x}}
    \end{array}\]
\end{example}
\begin{solution}
    (1)两边求导得$\e^x-y'\e^y+y+xy'=0$解得$y'=\dfrac{\e^x+y}{\e^y-x}$\\
    (2)两边求导得$2x+2yy'-\dfrac{y'}{\sqrt{1-y^2}}=0$解得$y'=\dfrac{2x\sqrt{1-y^2}}{1-2y\sqrt{1-y^2}}$\\
    (3)变换得到$\dfrac{\ln x}{x}=\dfrac{\ln y}{y}$,两边求导得到$y'=\dfrac{y(x\ln y-y)}{x(y\ln x-x)}$\\
    (4)两边求导得$\dfrac{1}{1+\frac{y^2}{x^2}}\dfrac{y'x-y}{x^2}=\dfrac12\dfrac{2x+2yy'}{x^2+y^2}$解得$y'=\dfrac{x+y}{x-y}$\\
    (5)两边求导得$2x-2y-2xy'+2yy'=2$解得$y'=\dfrac{x-y-1}{x-y}$\\
    (6)两边求导得$\dfrac1{2\sqrt{x}}+\dfrac{y'}{2\sqrt{y}}=0$,解得$y'=-\sqrt{\dfrac{y}{x}}$\\
    (7)两边求导得$y^2+2xyy'+\e^yy'=-(1+2yy')\sin(x+y^2)$,解得$y'=-\dfrac{y^2+\sin(x+y^2)}{\e^y+2xy+2y\sin(x+y^2)}$.\\
    (8)两边求导得$\dfrac{y'}{y}=\dfrac{\frac{-2}{(1+x)^2}}{2\sqrt{\frac{{1-x}}{1+x}}}$,解得$y'=-\dfrac{y}{(1+x)^2}\sqrt{\dfrac{{1-x}}{1+x}}$.
\end{solution}
\begin{example}{3.3-A-4~~求下列由参数方程表示的函数的导数:}
(1) $x=\sqrt[3]{1-\sqrt{t}}, y=\sqrt{1-\sqrt[3]{t}}$, 求 $\dfrac{\dd y}{\dd x}$;\quad
(2) $x=\sin^{2}t, y=\cos^{2}t$, 求 $\dfrac{\dd y}{\dd x}$;\\
(3) $x=1+t^{3}, y=e^{2t}$, 求 $\left.\dfrac{\dd y}{\dd x}\right|_{x=2}$;\quad
(4) $x=1+t^{2}, y=\cos t$, 求 $\dfrac{\dd y}{\dd x}$;\\
(5) $x=e^{t}\sin t, y=e^{-t}\cos t$, 求 $\dfrac{\dd y}{\dd x}$.
\end{example}
\begin{solution}
    使用链式法则,分别求导得:\\
    (1)$\dfrac{\dd y}{\dd x}=\dfrac{\dd y}{\dd t}\dfrac{\dd t}{\dd x}=\dfrac{-\frac32\sqrt{t}}{2\sqrt{1-\sqrt[3]{t}}}\dfrac{\frac23(1-\sqrt{t})^{\frac23}}{-\frac1{2\sqrt{t}}}=\dfrac{\sqrt{t}\sqrt[3]{(1-\sqrt{t})^2}}{\sqrt[3]{t^2}\sqrt{1-\sqrt[3]{t}}}$\\
    (2)$x+y=1\Rightarrow \dfrac{\dd y}{\dd x}=-1$.\quad(3)$\dfrac{\dd y}{\dd x}=\dfrac{\dd y}{\dd t}\dfrac{\dd t}{\dd x}=\dfrac{2\e^{2t}}{3t^2}$,代入$x=2,t=1$得到$\left.\dfrac{\dd y}{\dd x}\right|_{x=2}=\dfrac23\e^2$.\\
    (4)$\dfrac{\dd y}{\dd x}=\dfrac{\dd y}{\dd t}\dfrac{\dd t}{\dd x}=\dfrac{-\sin t}{2t}$.\quad (5)$\dfrac{\dd y}{\dd x}=\dfrac{\dd y}{\dd t}\dfrac{\dd t}{\dd x}=\dfrac{\e^{-t}(-\sin t-\cos t)}{\e^t(\sin t+\cos t)}=-\e^{2t}$.
\end{solution}
\begin{example}{3.3-A-5:用对数求导法求导数}{}
    (1)$y=x^{\sin x},(x>0)$;(2)$y=(\sqrt{x})^{\ln x},(x>0)$;(3)$y=a^{\sin x},(a>0)$;\\
    (4)$y=(1+x)^{\frac1x},(x>0)$;(5)$y=\dfrac{(x+5)^2(x-4)^{\frac13}}{(x+2)^5(x+4)^{\frac12}}$;(6)$y=x\sqrt{\dfrac{1-x}{1+x}}$
\end{example}
\begin{solution}
(1)$\ln y=(\sin x)\ln x\Rightarrow \dfrac{y'}{y}=\dfrac{\sin x}{x}+(\cos x)\ln x\Rightarrow y'=x^{\sin x}\left((\cos x)\ln x+\dfrac{\sin x}{x}\right)$\\
(2)$\ln y=\dfrac12\ln^2 x\Rightarrow \dfrac{\dd\ln y}{\dd y}\dfrac{\dd y}{\dd x}=\dfrac122\ln x\dfrac1x\Rightarrow \dfrac{\dd y}{\dd x}=\dfrac{y\ln x}{x}$\\
(3)$\ln y=\sin x\ln a\Rightarrow \dfrac{\dd\ln y}{\dd y}\dfrac{\dd y}{\dd x}=\cos x\ln a\Rightarrow \dfrac{\dd y}{\dd x}=a^{\sin x}\cos x\ln a$\\
(4)$\ln y=\dfrac{\ln(x+1)}{x}\Rightarrow \dfrac{\dd\ln y}{\dd y}\dfrac{\dd y}{\dd x}=\dfrac{1}{x(x+1)}-\dfrac{\ln(x+1)}{x^2}$解得:
\[\Rightarrow \dfrac{\dd y}{\dd x}=(1+x)^{\frac1x}\left(\dfrac{1}{x(x+1)}-\dfrac{\ln(x+1)}{x^2}\right)\]
(5)定义域为$(-4,-2)\cup(-2,+\infty)$,取绝对值,然后取对数
\[\ln|y|=\ln\left(\frac{(x+5)^2|x-4|^{\frac{1}{3}}}{|x+2|^5(x+4)^{\frac{1}{2}}}\right)=2\ln|x+5|+\frac{1}{3}\ln|x-4|-5\ln|x+2|-\frac{1}{2}\ln(x+4)\]
两边对$x$微分:\[\frac{\dd}{\dd x}\ln|y|=\frac{1}{y}\frac{\dd y}{\dd x}=2\cdot\frac{1}{x+5}+\frac{1}{3}\cdot\frac{1}{x-4}-5\cdot\frac{1}{x+2}-\frac{1}{2}\cdot\frac{1}{x+4}\]
解得:\[\frac{\dd y}{\dd x}=\frac{(x+5)^2(x-4)^{\frac{1}{3}}}{(x+2)^5(x+4)^{\frac{1}{2}}}\left(\frac{2}{x+5}+\frac{1}{3(x-4)}-\frac{5}{x+2}-\frac{1}{2(x+4)}\right)\]
(6)定义域为$(-1,1]$,$|y|=|x|\sqrt{\dfrac{1-x}{1+x}}\Rightarrow \ln|y|=\ln|x|+\dfrac12\ln|1-x|-\dfrac12\ln|1+x|$,两边对$x$微分:
\[\frac{\dd}{\dd x}\ln|y|=\frac{1}{y}\frac{\dd y}{\dd x}=\dfrac1x+\dfrac1{2(1-x)}-\dfrac1{2(1+x)}\Rightarrow \frac{\dd y}{\dd x}=y\left(\dfrac1x+\dfrac1{2(1-x)}-\dfrac1{2(1+x)}\right)\]
\end{solution}
\begin{example}{3.3-A-6}{}
    下列参数方程给出函数 $y=y(x),\text{求}\dfrac{\mathrm{d}^2y}{\mathrm{d}x^2}$\\
    $(1)x=a\cos t,y=a\sin t;\quad(2)x=2t-t^2,y=3t-t^3;\\(3)x=\ln(1+t^2),y=\arctan t;\quad(4)x=\ln(t+\sqrt{t^2+1}),y=t^2.$
\end{example}
\begin{solution}
    (1)$\dfrac{\dd^2y}{\dd x^2}=\dfrac{\dd}{\dd x}\left(\dfrac{\dd y}{\dd x}\right)=\dfrac{\dd}{\dd t}\left(\dfrac{\dd y}{\dd t}\dfrac{\dd t}{\dd x}\right)\dfrac{\dd t}{\dd x}=\dfrac{\dd}{\dd t}\left(-\cot t\right)\dfrac{\dd t}{\dd x}=\dfrac{\csc^2 t}{-a\sin t}=-\dfrac1{a\sin^3t}$\\
    (2)$\dfrac{\dd^2y}{\dd x^2}=\dfrac{\dd}{\dd x}\left(\dfrac{\dd y}{\dd x}\right)=\dfrac{\dd}{\dd t}\left(\dfrac{\dd y}{\dd t}\dfrac{\dd t}{\dd x}\right)\dfrac{\dd t}{\dd x}=\dfrac{\dd}{\dd t}\left(\dfrac32(1+t)\right)\dfrac{\dd t}{\dd x}=\dfrac{3}{4(1-t)}$\\
    (3)$\dfrac{\dd^2y}{\dd x^2}=\dfrac{\dd}{\dd x}\left(\dfrac{\dd y}{\dd x}\right)=\dfrac{\dd}{\dd t}\left(\dfrac{\dd y}{\dd t}\dfrac{\dd t}{\dd x}\right)\dfrac{\dd t}{\dd x}=\dfrac{\dd}{\dd t}\left(\dfrac1{1+t^2}\dfrac{1+t^2}{2t}\right)\dfrac{\dd t}{\dd x}=-\dfrac{(1+t^2)}{4t^3}$\\
    (4)$\dfrac{\dd^2y}{\dd x^2}=\dfrac{\dd}{\dd x}\left(\dfrac{\dd y}{\dd x}\right)=\dfrac{\dd}{\dd t}\left(\dfrac{\dd y}{\dd t}\dfrac{\dd t}{\dd x}\right)\dfrac{\dd t}{\dd x}=\dfrac{\dd}{\dd t}\left(2t\sqrt{t^2+1}\right)\dfrac{\dd t}{\dd x}=\dfrac{4t^2+2}{\sqrt{t^2+1}}\sqrt{t^2+1}=4t^2+2$.
\end{solution}
\begin{example}{3.3-A-7~求下列隐函数的二阶导数$y''$}{}
    (1)$x^3+y^3-3axy=0,(a>0)$;(2)$y^2+2\ln y=x^4$;(3)$xy=\e^{x+y}$;(4)$y=1-x\e^y$
\end{example}
\begin{solution}
    (1)两边求导数得\[3x^2+3y^2y'=3a(y+xy')\Rightarrow (ax-y^2)y'=x^2-ay\Rightarrow (a-2yy')y'+(ax-y^2)y''=2x-ay'\]解得
    \[y''=\dfrac1{y^2-ax}\left[\dfrac{2a(ay-x^2)}{y^2-ax}-2y\left(\dfrac{ay-x^2}{y^2-ax}\right)^2-2x\right]\]
    (2)两边求导数得\[2yy'+\dfrac{2y'}{y}=4x^3\Rightarrow (y^2+1)y'=2x^3y\Rightarrow (2yy')y'+(y^2+1)y''=2(3x^2y+x^3y')\]
    解得\[y''=\dfrac{2x^2y}{(1+y^2)^2}\left[3(1+y^2)^2+2x^4(1-y^2)\right]\]
    (3)两边求导数得$xy'+y=\e^{x+y}(1+y')\Rightarrow (\e^{x+y}-x)y'=y-\e^{x+y}\Leftrightarrow (xy-x)y'=y-xy$,再次在两边求导
    \[(xy'+y-1)y'+(xy-x)y''+y=0\Rightarrow y''=\dfrac{y}{x-xy}+\dfrac{(x+y-2)(xy-y)}{(x-xy)^2}+\dfrac{x(xy-y^2)}{(x-xy)^3}\]
    (4)两边求导数得$y'=-x\e^yy'-\e^y\Rightarrow y'(1+x\e^y)=-\e^y\Rightarrow y'(y-2)=\e^y$,再次求导有
    \[y''=\e^{2y}[\dfrac1{(y-2)^2}-\dfrac1{(y-2)^3}]\Leftrightarrow y''=\dfrac{2\e^{2y}}{(1+x\e^y)^2}-\dfrac{x\e^{3y}}{(1+x\e^y)^3}\]
\end{solution}
\begin{example}{3.3-A-9}{}
    求$\dfrac{\dd y}{\dd x}$:(1)$r^2=2a^2\cos 2\theta$在$\theta=\dfrac{\pi}6$处\\
    (2)$r=a\e^{m\theta}$,其中$r=\sqrt{x^2+y^2}$以及$\theta=\arctan\dfrac{y}{x}$为极坐标.
\end{example}
\begin{solution}
    \[\dfrac{\dd y}{\dd x}=\dfrac{\frac{\dd y}{\dd \theta}}{\frac{\dd x}{\dd \theta}}=\dfrac{\frac{\dd r(\theta)\sin\theta}{\dd \theta}}{\frac{\dd r(\theta)\cos\theta}{\dd \theta}}=\dfrac{r'(\theta)\sin\theta+r(\theta)\cos\theta}{r'(\theta)\cos\theta-r(\theta)\sin\theta}\]
    (1)参数方程为$r^2(\theta)=2a^2\cos 2\theta,r'(\theta)r(\theta)=2a^2(-\sin2\theta),r(\dfrac{\pi}6)=a^2,r(\theta)=a,r'(\theta)=-\sqrt{3}a$,代入
    \[\dfrac{\dd y}{\dd x}=\dfrac{r'(\theta)\sin\theta+r(\theta)\cos\theta}{r'(\theta)\cos\theta-r(\theta)\sin\theta}=0\]
    (2)参数方程为$\begin{cases}x=a\e^{m\theta}\cos\theta\\y=a\e^{m\theta}\sin\theta\end{cases},r(\theta)=a\e^{m\theta},r'(\theta)=am\e^{m\theta}$,代入就有
    \[\dfrac{\dd y}{\dd x}=\dfrac{r'(\theta)\sin\theta+r(\theta)\cos\theta}{r'(\theta)\cos\theta-r(\theta)\sin\theta}=\dfrac{\cos\theta+m\sin\theta}{-\sin\theta+m\cos\theta}=\tan\left(\theta+\arctan\dfrac1m\right)\]
\end{solution}
\begin{example}{3.3-B-1}{}
    求导:$y=\e^x+\e^{\e^x}\quad y=\left(\dfrac{a}{b}\right)^x\left(\dfrac{b}{x}\right)^a\left(\dfrac{x}{a}\right)^b\quad y=3^x\ln x$
\end{example}
\begin{solution}
(1)$y'=\e^x+\e^{\e^x}\e^x$;\\
(2)$\ln y=x\ln\left(\dfrac{a}{b}\right)+a\ln\left(\dfrac{b}{x}\right)+b\ln\left(\dfrac{x}{a}\right)=x\ln\left(\dfrac{a}{b}\right)+(b-a)\ln x$,两边求导得\[\dfrac{y'}{y}=\ln\left(\dfrac{a}{b}\right)+\dfrac{b-a}{x},y'=\left(\dfrac{a}{b}\right)^x\left(\dfrac{b}{x}\right)^a\left(\dfrac{x}{a}\right)^b\left(\ln\left(\dfrac{a}{b}\right)+\dfrac{b-a}{x}\right)\]
(3)$y'=3^x\ln 3\ln x+\dfrac{3^x}{x}$
\end{solution}
\section{习题3.4}
\begin{example}{3.4-A-5}{}
    利用一阶微分的形式不变性求微分:$(1)y=\arctan\e^x\quad(2)y=\e^{\sin x}$
\end{example}
\begin{solution}
    (1)$\dd y=\dfrac{\dd \e^x}{1+\e^{2x}}=\dfrac{\e^x}{1+\e^{2x}}\dd x$;
    (2)$\dd y=\e^{\sin x}\dd\sin x=\e^{\sin x}\cos x\dd x$
\end{solution}
\begin{example}{3.4-B-1}{}
    求下列函数的二阶微分$\dd^2y:(1)y=\sqrt{1+x^2}~~~(2)y=\dfrac{\ln x}{x}$
\end{example}
\begin{solution}
    (1)$\dd^2y=\dd(\dd y)=\dd(\dfrac{x}{\sqrt{1+x^2}}\dd x)=\dfrac{\sqrt{1+x^2}-x\frac{x}{\sqrt{1+x^2}}}{1+x^2}\dd x^2=\dfrac{1}{(1+x^2)^\frac32}\dd x^2$\\
    (2)$\dd^2y=\dd(\dd y)=\dd\left(\dfrac{1-\ln x}{x^2}\dd x\right)=\dfrac{-x-2x(1-\ln x)}{x^4}\dd x^2=\dfrac{2\ln x-3}{x^3}\dd x^2$
\end{solution}
\section{习题3.5}
\begin{example}{3.5-A-5}{}
    拉格朗日中值定理证明的关键是构造辅助函数,试利用下列辅助函数来证明这个定理:\\
(1) $\Phi(x) = [f(x)-f(a)] (b-a) -(x-a) [f(b)-f(a)] ;$\\
(2)$\Phi(x)=f(x)(b-a)-x[f(b)-f(a)].$
\end{example}
\begin{solution}
(1)过$(a,f(a)),(b,f(b))$的直线方程对应的一次函数为$g(x)=\dfrac{f(b)-f(a)}{b-a}(x-a)+f(a)$,所以构造
\begin{align*}
h(x)&=f(x)-g(x)=f(x)-\dfrac{f(b)-f(a)}{b-a}(x-a)-f(a)\\
&=\dfrac{(b-a)f(x)-(f(b)-f(a))(x-a)-(b-a)f(a)}{b-a}\\
&=\dfrac{(b-a)(f(x)-f(a))-(f(b)-f(a))(x-a)}{b-a}=\dfrac{\Phi(x)}{b-a}
\end{align*}
且$h(a)=h(b)=0$,所以根据罗尔定理,存在$\xi\in(a,b)$使得$h'(\xi)=0$,即$f'(\xi)=g'(\xi)$,即
\[\dfrac{f(b)-f(a)}{b-a}=f'(\xi)\]
(2)设$\varphi(x)=f(x)-\dfrac{f(b)-f(a)}{b-a}x$,容易发现$\varphi'(x)=h'(x)$,这表明$\varphi(x)$是$h(x)$向上平移得到的,所以尽管此时没有$h(a)=h(b)=0$,但是$h(a)=h(b)$却仍然成立,仍可利用罗尔定理得到存在$\xi\in(a,b)$使得$\varphi'(\xi)=0$,即$f'(\xi)=g'(\xi)$,即
\[\dfrac{f(b)-f(a)}{b-a}=f'(\xi)\]
\end{solution}
\begin{example}{3-5-A-6}{}
(1)证明:如果$\forall x\in[a.b]$,有 $f^\prime(x)\geqslant m,m$ 是某常数,则有 $f(b)\geqslant f(a)+m(b-a);$\\
(2)证明:如果$\forall x\in[a,b]$,有 $f^\prime(x)\leqslant M,M$ 是某常数,则有 $f(b)\leqslant f(a)+M(b-a);$\\
(3)如果$\forall x\in[a,b]$,有$|f^\prime(x)|\leqslant M$,试写出一个类似的定理.
\end{example}
\begin{solution}
    (1)根据导数存在,得知$f(x)$在$[a,b]$上连续可微,符合拉格朗日定理使用条件,于是存在$\xi\in(a,b)$使得$f'(\xi)=\dfrac{f(b)-f(a)}{b-a}$,又因为$f'(\xi)\geq m$,代入就有$f(b)\geqslant f(a)+m(b-a)$\\
(2)根据导数存在,得知$f(x)$在$[a,b]$上连续可微,符合拉格朗日定理使用条件,于是存在$\xi\in(a,b)$使得$f'(\xi)=\dfrac{f(b)-f(a)}{b-a}$,又因为$f'(\xi)\leqslant M$,代入就有$f(b)\leqslant f(a)+M(b-a)$\\
(3)根据导数存在,得知$f(x)$在$[a,b]$上连续可微,符合拉格朗日定理使用条件,于是存在$\xi\in(a,b)$使得$f'(\xi)=\dfrac{f(b)-f(a)}{b-a}$,即$|f'(\xi)|=\left|\dfrac{f(b)-f(a)}{b-a}\right|$,又因为$|f'(\xi)|\leqslant M$,代入就有$|f(b)-f(a)|\leqslant M(b-a)$,即$|f(b)|\leqslant |f(a)|+M(b-a)$
\end{solution}
\begin{example}{3-5-A-7}{}
    $\text{证明:无论}m\text{是什么数,多项式函数}f(x)=x^3-3x+m\text{在}[0,1]\text{内决不会有两个零点}.$
\end{example}
\begin{solution}
    用反证法,假设$f(x)$在$[0,1]$存在两个零点$x_1$和$x_2$,则由于罗尔定理,存在$\xi\in(0,1)$使得$f'(\xi)=0$,但是$f'(x)=3x^2-3$在$(0,1)$上小于$0$,矛盾,所以$f(x)=x^3-3x+m$在$[0,1]$内决不会有两个零点.
\end{solution}
\begin{example}{3-5-A-8}{}
    设$f(x)\in\mathbb{C}[0,1]$且可微;对于每个 $x,f(x)$的值都在(0,1)内;并且$\forall x\in ( 0, 1) , f^\prime ( x) \neq 1. \textbf{求 }$ 证:存在唯一的一个数$x_0\in(0,1)$,使得$f(x_0)=x_0.$
\end{example}
\begin{solution}
    构造函数$g(x)=f(x)-x$,则$g(x)$在$(0,1)$上连续可微,且$g(0)=f(0)-0>0,g(1)=f(1)-1<0$,所以$g(x)$在$(0,1)$上必有零点,假设有两个及以上个零点,则根据罗尔定理得到必然存在$\xi\in(0,1)$使得$g'(\xi)=0$,即$f'(\xi)=1$,矛盾,所以$f(x)$在$(0,1)$上只有一个零点,即存在唯一的一个数$x_0\in(0,1)$,使得$f(x_0)=x_0.$
\end{solution}
\begin{example}{3-5-A-10}{}
    证明:$(1)\mid\mathrm{sin}b-\mathrm{sin}a\mid\leqslant\mid b-a\mid;\quad(2)\frac{a-b}{a}<\ln\frac{a}{b}<\frac{a-b}{b}\quad(a>b>0).$
\end{example}
\begin{solution}
    (1)由Lagrange中值定理,存在$\xi\in(a,b)$使得$f'(\xi)=\dfrac{\sin b-\sin a}{b-a}$\\
    且$|f'(\xi)|=\left|\dfrac{\sin b-\sin a}{b-a}\right|\leq 1$,变形即可得到$|\sin b-\sin a|\leq |b-a|$\\
    (2)等价于证明$\dfrac{1}{a}<\dfrac{\ln a-\ln b}{b-a}<\dfrac{1}{b}$,根据Lagrange中值定理,对于函数$f(x)=\ln x$,存在$\xi\in(a,b)$使得$f'(\xi)=\dfrac{\ln a-\ln b}{b-a}$且由于$\ln x$的导函数在$(a,b)$上单调递减,所以$f'(a)<f'(\xi)<f(b)$,即$\dfrac{1}{a}<\dfrac{\ln a-\ln b}{b-a}<\dfrac{1}{b}$.
\end{solution}
\begin{example}{3-5-A-11}{}
证明:若 $f(x){\in}C[a,b]$,在$(a,b)$内可导,则必存在一点 $\xi\in(a,b)$,使得
\[2\xi[f(b)-f(a)]=(b^2-a^2)f^{\prime}(\xi)\]
\end{example}
\begin{solution}
等价于证明$\dfrac{f(a)-f(b)}{a^2-b^2}=\dfrac{f'(\xi)}{2\xi}$,根据柯西中值定理,对于$f(x)$和$g(x)=x^2$,存在$\xi\in(a,b)$使得$\dfrac{f(a)-f(b)}{g(a)-g(b)}=\dfrac{f'(\xi)}{g'(\xi)}$,即$\dfrac{f(a)-f(b)}{a^2-b^2}=\dfrac{f'(\xi)}{2\xi}$.
\end{solution}
\begin{example}{3-5-A-12}{}
    设 $f(x)\in C[a,b].$在$(a,b)$内可导,$0<a< b$. 求 证 : 存 在 一 点 $\xi \in ( a, b)$,使得
$$\frac{af(b)-bf(a)}{a-b}=f(\xi)-\xi f^{\prime}(\xi).$$
\end{example}
\begin{solution}
    等价于证明$\dfrac{\frac{f(a)}{a}-\frac{f(b)}{b}}{\frac1a-\frac1b}=f(\xi)-\xi f'(\xi)$,对于$f(x)$和$g(x)=\dfrac1x$,存在$\xi\in(a,b)$使得
    \[\dfrac{f(a)-f(b)}{g(a)-g(b)}=\dfrac{f'(\xi)}{g'(\xi)}=\dfrac{\frac{\xi f'(\xi)-f(\xi)}{\xi^2}}{-\frac1{\xi^2}}=f(\xi)-\xi f'(\xi)\]
\end{solution}
\begin{example}{3-5-B-2}{}
    $\text{设函数 }f(x)\text{满足}|f(x)-f(y)|\leqslant|x-y|^n,n>1,\text{通过考虑 }f^{\prime}\text{证明 }f\text{ 是常数}.$
\end{example}
\begin{solution}
变形得到$\left|\dfrac{f(x)-f(y)}{x-y}\right|\leq |x-y|^{n-1}$,由拉格朗日中值定理知存在$\xi\in(x,y)$使得$|f'(\xi)|=\left|\dfrac{f(x)-f(y)}{x-y}\right|$,则$|f'(\xi)|<|x-y|^{n-1}$,由于$x,y$的任意性,$|f'(\xi)|=0$,所以$f$是常数.
\end{solution}
\begin{example}{3-5-B-10}{}
    设 $h{>}0,f^{\prime}(x)$在$(a-h,a+h)$内存在.求证:\\
$(1)\dfrac{f(a+h)-f(a-h)}h=f^{\prime}(a+\theta h)+f^{\prime}(a-\theta h)\quad(0<\theta<1);$\\
$(2)\dfrac{f(a+h)-2f(a)+f(a-h)}{h}=f^{\prime}(a+\theta h)-f^{\prime}(a-\theta h)\quad(0<\theta<1).$
\end{example}
\begin{solution}
(1)由Lagrange中值定理得到,若$f^{\prime}(x)$在$(x,x+h)$内存在,那么存在$\xi\in(x,x+h)$使得$f'(\xi)=\dfrac{f(x+h)-f(x)}{h}$,这里可以换元,令$\xi=x+\theta h$,其中$\theta\in(0,1)$,则存在$\theta\in(0,1)$使得$f'(x+\theta h)=\dfrac{f(x+h)-f(x)}{h}$,使用这个定理就可以得到:
\begin{align*}
    \dfrac{f(a+h)-f(a-h)}h&=\dfrac{f(a+h)-f(a)}{h}+\dfrac{f(a)-f(a-h)}{h}\\
    &=f^{\prime}(a+\theta h)+f^{\prime}(a-\theta h)
\end{align*}
(2)同理,若$f^{\prime}(x)$在$(x,x+h)$内存在,那么存在$\xi\in(x,x+h)$使得$f'(\xi)=\dfrac{f(x+h)-f(x)}{h}$,这里可以换元,令$\xi=x+\theta h$,其中$\theta\in(0,1)$,则存在$\theta\in(0,1)$使得$f'(x+\theta h)=\dfrac{f(x+h)-f(x)}{h}$,使用这个定理就可以得到:
\begin{align*}
    \dfrac{f(a+h)-2f(a)+f(a-h)}{h}&=\dfrac{f(a+h)-f(a)}{h}-\dfrac{f(a)-f(a-h)}{h}\\
    &=f^{\prime}(a+\theta h)-f^{\prime}(a-\theta h)
\end{align*}
\end{solution}
\begin{example}{3-5-B-11}{}
$\text{由拉格朗日中值定理知,}\ln(1+x)-0=x\cdot\frac{1}{1+\theta x}\mathrm{(0<}\theta<1\text{),证明}\quad\underset{x\to0}{\operatorname*{\operatorname*{lim}}}\theta{=}\frac{1}{2}.$
\end{example}
\begin{solution}
    解得$\theta=\dfrac{x-\ln(x+1)}{x\ln(x+1)}$,转化为求$\displaystyle\lim_{x\to 0}\dfrac{x-\ln(x+1)}{x\ln(x+1)}$,由
    \[\dfrac{(x-\ln(x+1))'}{(x\ln(x+1))'}=\dfrac{1-\frac1{x+1}}{\frac{x}{x+1}+\ln(x+1)}=\dfrac{x}{x+(x+1)\ln(x+1)}\]
    则由于导函数之比在$0$处的极限存在,分母不为0,当$x\to 0$时,分子分母趋近于0,所以洛必达法则有效,则
    \[\lim_{x\to 0}\dfrac{x-\ln(x+1)}{x\ln(x+1)}=\lim_{x\to 0}\dfrac{x}{x+(x+1)\ln(x+1)}\]
    发现分子分母趋近于0,对$\dfrac{x}{x+(x+1)\ln(x+1)}$分子分母上下分别求导得到$\dfrac{1}{2+\ln(x+1)}$,由于导函数之比在$0$处的极限存在,分母不为0,则洛必达法则有效,则\[\lim_{x\to 0}\dfrac{x}{x+(x+1)\ln(x+1)}=\lim_{x\to 0}\dfrac{1}{2+\ln(x+1)}=\dfrac12\]
    所以$\displaystyle\lim_{x\to 0}\theta=\dfrac{1}{2}$
\end{solution}
\begin{example}{3-5-B-12}{}
$\text{设 }f(x)=\begin{cases}\dfrac{g(x)}{x},x\neq0,\\0,x=0.\end{cases},\text{并设 }g(0)=g^{\prime}(0)=0,g''(0)=17.\text{求 }f'(0).$
\end{example}
\begin{solution}
    利用导数的定义:已知$f(0)=0$
    \[f'(0)=\lim_{h\to 0}\dfrac{f(h)-f(0)}{h}=\lim_{h\to 0}\dfrac{g(h)}{h^2}\]
    又因为$g'(x),g''(x)$在$x=0$的邻域内有定义,所以使用洛必达法则得到
    \[\lim_{h\to 0}\dfrac{g(h)}{h^2}=\lim_{h\to 0}\dfrac{g'(h)}{2h}=\dfrac12\lim_{h\to 0}\dfrac{g'(h)-g'(0)}{h}=\dfrac12g''(0)=\dfrac{17}{2}\]
    可导一定连续,所以$f'(0)=\lim_{h\to 0}\dfrac{g''(h)}{2}=\dfrac{17}{2}$
\end{solution}
\begin{example}{3-5-B-14}{}
设$f(x)$一阶可导,且$f''( x_0)$存 在,求 证 :
$$\lim_{h\to0}\frac{f(x_0+2h)-2f(x_0+h)+f(x_0)}{h^2}=f''(x_0).$$
\end{example}
\begin{solution}
    发现分子分母都趋近于0,先分子分母上下分别求导得到\[\lim_{h\to 0}\dfrac{2f'(x_0+2h)-2f'(x_0+h)}{2h}=\lim_{h\to 0}\dfrac{f'(x_0+2h)-f'(x_0+h)}{h}\],发现此时极限仍然存在,分母仍不是0,所以洛必达法则有效,于是有\[\lim_{h\to0}\frac{f(x_0+2h)-2f(x_0+h)+f(x_0)}{h^2}=\lim_{h\to 0}\dfrac{f'(x_0+2h)-f'(x_0+h)}{h}\]
    再利用$f''(x_0)$的存在性,以及导数的定义,得到
    \begin{align*}\lim_{h\to 0}\dfrac{f'(x_0+2h)-f'(x_0+h)}{h}&=\lim_{x\to 0}\dfrac{f'(x_0+2h)-f'(x_0)-(f'(x_0+h)-f'(x_0))}{h}\\&=2f''(x_0)-f''(x_0)=f''(x_0)\end{align*}即证.
\end{solution}
\newpage
\section{泰勒公式}
\begin{example}{3.6-A-2}{}
    按$x$ 的 正 整 数 幂 , 写 出 下 列 函 数 的 展 开 式 至 含 有 指 定 阶 数 的 项 ( 带 皮 亚 诺 余 项 ) : \\
$(1)\frac1{1-x}$到含 $x^7$ 的项;$(2)\arctan x$到含$x^4$的项\\
$(3)\frac1{\sqrt{1+x}}$到含 $x^4$ 的项;$(4)\tan x$到含$x^4$的项.
\end{example}
\begin{solution}
(1)先求$f(x)=\dfrac{1}{1-x}$的在$x=0$处的$n$阶导数$f^{(n)}(0)$,由于$(1-x)f(x)=1$两边求$n$阶导数,并利用莱布尼兹公式,得到\[(1-x)f^{(n)}(x)+(-1)C_n^1f^{(n-1)}(x)=0\Leftrightarrow f^{(n)}(0)=nf^{(n-1)}(0)\]又$f'(0)=1$,所以$f^{(n)}(0)=n!$,于是展开到含 $x^7$ 的项就是:\[f(x)=1+x+x^2+x^3+x^4+x^5+x^6+x^7+o(x^7)\]
(2)令$f(x)=\dfrac1{1+x^2}=\dfrac1{1-(-x^2)}$,展开得到:\[f'(x)=1+(-x^2)+(-x^2)^2+(-x^2)^3+o(x^4)=1-x^2+x^4-x^6+o(x^6)\]
保留到到含$x^4$的项就是
\[\arctan x=x-\dfrac13x^3+o(x^4)\]
(3)直接使用广义二项式展开:
\begin{align*}
    f(x)&=(1+x)^{-\frac12}=1+C_{-\frac12}^1x+C_{-\frac12}^2x^2+C_{-\frac12}^3x^3+C_{-\frac12}^4x^4+o(x^4)\\
    &=1-\dfrac12x+\dfrac{-\frac12(-\frac12-1)}{2}x^2+\dfrac{\frac12(\frac12-1)(\frac12-2)}{6}x^3+\dfrac{-\frac12(\frac12-1)(\frac12-2)(\frac12-3)}{24}x^4+o(x^4)\\
    &=1-\frac{1}{2}x+\frac{3}{8}x^2-\frac{5}{16}x^3+\frac{35}{128}x^4+o(x^4)
\end{align*}
(4)反复求导过程复杂,可以先建立微分方程:
\[y'=(\tan x)'=1+\tan^2x=1+y^2\]
两边求导得到:
\begin{align*}
y''&=2yy^{\prime}=2y(1+y^2)=2y+2y^3\\
y'''&=2y^{\prime}+6y^2y^{\prime}=2(1+y^2)+6y^2(1+y^2)=2+8y^2+6y^4\\
y^{(4)}&=16yy'+24y^3y'=16y(1+y^2)+24y^3(1+y^2)=16y+40y^3+24y^5
\end{align*}
代入$y=0$,$f(x)=f(0)+f'(0)x+\frac{f''(0)}{2!}x^2+\cdots+\frac{f^{(n)}(0)}{n!}x^n+o(x^n)\quad(x\to0)$于是得到展开式:
\[\tan x=x+\dfrac13x^3+o(x^4)\]
\end{solution}
\begin{example}{3.6-A-5}{}
$\text{求函数 }f(x)=x\mathrm{e}^x\text{ 的 }n\text{ 阶麦克劳林公式,带拉格朗日余项}.$
\end{example}
\begin{solution}
设$f(x)=x\e^x$,其泰勒展开式展开到含$x^n$的项的式子,都可以由$x,\e^x$展开式的乘积确定:
\begin{align*}
    f(x)&=x\mathrm{e}^x=x\left(1+x+\frac{x^2}{2!}+\frac{x^3}{3!}+\cdots+\frac{x^{n-1}}{(n-1)!}\right)\\
&=x+x^2+\frac{x^3}{2!}+\frac{x^4}{3!}+\cdots+\frac{x^{n}}{(n-1)!}\end{align*}
再加上$\dfrac{f^{(n+1)}(\theta x)}{(n+1)!},\theta\in(0,1)$,即可得到
\[f(x)=x\e^x=x+x^2+\frac{x^3}{2!}+\frac{x^4}{3!}+\cdots+\frac{x^{n}}{(n-1)!}+\dfrac{\e^{\theta x}(\theta x+n+1)}{(n+1)!}x^{n+1}\]
\end{solution}
\begin{example}{3.6-A-6}{}
\begin{tabular}{@{}l@{}l@{}l@{}}
(1) $\lim\limits_{x\to0}\dfrac{\mathrm{e}^x-1-x}{x^2}$; & 
(2) $\lim\limits_{x\to0}\dfrac{\mathrm{e}^x-1-x-\frac{x^2}{2}}{x^3}$; & 
(3) $\lim\limits_{x\to0}\dfrac{\mathrm{e}^{x^3}-1-x^3}{(\sin2x)^6}$; \\
(4) $\lim\limits_{x\to+\infty}\left(x+\frac{1}{2}\right)\ln\left(1+\frac{1}{x}\right)$; & 
(5) $\lim\limits_{x\to0}\left(\frac{1}{x}-\frac{1}{\mathrm{e}^x-1}\right)$; & 
(6) $\lim\limits_{x\to0^+}\dfrac{e^x-1-x}{\sqrt{1-x}-\cos\sqrt{x}}$ \\
\end{tabular}
\end{example}
\begin{solution}
(1)$\displaystyle\lim_{x\to0}\frac{\e^x-1-x}{x^2}=\lim_{x\to0}\frac{1+x+\frac12x^2+o(x^2)-1-x}{x^2}=\dfrac12$\\
(2)$\displaystyle\lim_{x\to0}\frac{\e^x-1-x-\frac{x^2}{2}}{x^3}=\lim_{x\to0}\frac{1+x+\frac12x^2+\frac16x^3+o(x^3)-1-x-\frac12x^2}{x^3}=\dfrac16$\\
(3)$\displaystyle\lim_{x\to0}\frac{\e^{x^3}-1-x^3}{(\sin2x)^6}=\lim_{x\to0}\frac{1+x^3+\frac12x^6+o(x^6)-1-x^3}{64x^6}=\dfrac1{128}$\\
(4)$\displaystyle\lim_{x\to+\infty}\left(x+\frac{1}{2}\right)\ln\left(1+\frac{1}{x}\right)=\lim_{x\to0^+}\left(\dfrac1x+\frac{1}{2}\right)\ln\left(1+x\right)=\lim_{x\to0^+}\dfrac{x+2}{2}\dfrac{\ln(x+1)}{x}=1$\\
(5)$\displaystyle\lim_{x\to0}\left(\frac{1}{x}-\frac{1}{\e^x-1}\right)=\lim_{x\to0}\dfrac{\e^x-x-1}{x(\e^x-1)}=\frac{1-x+\frac12x^2+o(x^2)-1-x}{x^2}=\dfrac12$\\
(6)$\displaystyle\lim_{x\to0^+}\frac{\e^x-1-x}{\sqrt{1-x}-\cos\sqrt{x}}=\lim_{x\to0^+}\frac{1+x+\frac12x^2+o(x^2)-1-x}{1-\frac12x-\frac18x^2+o(x^2)-(1-\frac12x+\frac1{24}x^2)}=\dfrac{\frac12}{-\frac16}=-3$
\end{solution}
\section{函数性态的研究}
\begin{example}{3.7-A-3}{}
    $\text{证明函数 }y=x+\sin x\text{严格上升}.$
\end{example}
\begin{solution}
    $y'=1-\cos x\geq0$,故$y$严格上升.
\end{solution}
\begin{example}{3.7-A-7}{}
    $\text{设}f(x)=ax\mathrm{e}^{bx},\text{试确定常数 }a,b,\text{使得 }f(\frac{1}{3})=1,\text{且函数在 }x=\frac{1}{3}\text{处有极大值}.$
\end{example}
\begin{solution}
    $f'(x)=a\e^{bx}(bx+1),f'(\frac13)=a\e^{b\frac13}(1+\frac13b)=0,f(\frac13)=a\dfrac13\e^{b\frac13}=1$,得到由必要条件引出的方程组$a\left(b+\dfrac13\right)=0,a\e^{b\frac13}=3$,由第二个方程得到$a\ne 0$,所以$b=-3,a=3\e$\\
    下面证明充分性,当$b=-3,a=3\e$时,$f(x)=3x\e\cdot\e^{-3x}=3x\e^{1-3x},f(\dfrac13)=1,f'(x)=\e\dfrac{1-3x}{\e^{3x}}=0$,导函数的正负性取决于一次函数$y=1-3x$,当$x<\dfrac13$时,$f'(x)>0$,当$x=\dfrac13$时,$f'(x)=0$,当$x>\dfrac13$时,$f'(x)<0$,所以$f(x)$在$\frac{1}{3}$处的极大值是$f(\frac{1}{3})=1$.
\end{solution}
\begin{example}{3.7-A-8}{}
    \begin{tabular}{@{}l@{}l@{}}
(1) $\sin x>\frac{2}{\pi}x\quad(0<x<\frac{\pi}{2})$; & 
(2) $\cos x>1-\frac{x^{2}}{2}\quad(x\neq0)$; \\
(3) $x>\ln(1+x)>x-\frac{x^{2}}{2}\quad(x>0)$; & 
(4) $\ln(1+x)\geqslant\frac{\arctan x}{1+x}\quad(x\geqslant0)$.
\end{tabular}
\end{example}
\begin{solution}
    (1)对于$y=\sin x$,$y'=\cos x$,$y''=-\sin x$,$y$在$(0,\frac{\pi}{2})$上为凹函数。根据凹函数的定义,设$x_1=0$,$x_2=\frac{\pi}{2}$,$f(x)=\sin x$,有:
    \[f(tx_1+(1-t)x_2)=f\left(\frac{\pi}{2}(1-t)\right)\geq tf(x_1)+(1-t)f(x_2)=1-t\]
    令$x=\frac{\pi}{2}(1-t)\in(0,\frac{\pi}{2})$,则有$\sin x>\frac{2}{\pi}x\quad(0<x<\frac{\pi}{2})$.\\
    (2)考虑函数$f(x)=\cos x-\left(1-\frac{x^2}{2}\right)$,则$f(0)=0$。计算导数:
    \[f'(x)=-\sin x+x,\quad f''(x)=-\cos x+1\geq0\]
    因此$f'(x)$单调递增,又$f'(0)=0$,故当$x>0$时$f'(x)>0$,当$x<0$时$f'(x)<0$。所以$f(x)$在$x=0$处取得最小值$f(0)=0$,且当$x\neq0$时$f(x)>0$,即$\cos x>1-\frac{x^2}{2}\quad(x\neq0)$.\\
    (3)先证左边不等式:令$f(x)=x-\ln(1+x)$,则$f(0)=0$,$f'(x)=1-\frac{1}{1+x}=\frac{x}{1+x}>0\quad(x>0)$,故$f'(x)>0$,又因为$f(x)>f(0)=0$,故$f(x)>0$,即$x>\ln(1+x)$\\
    再证右边不等式:令$g(x)=\ln(1+x)-\left(x-\frac{x^2}{2}\right)$,则$g(0)=0$,$g'(x)=\frac{1}{1+x}-1+x=\frac{x^2}{1+x}>0\quad(x>0)$,故$g'(x)>0$,$g(x)$单调递增,又因为$g(x)>g(0)=0$,故$g(x)>0$,即$\ln(1+x)>x-\frac{x^2}{2}$\\
(4)令$h(x)=(1+x)\ln(1+x)-\arctan x$,则$h(0)=0$。计算导数:
    \[h'(x)=\ln(1+x)+1-\frac{1}{1+x^2}\]
    当$x\geq0$时,$\ln(1+x)\geq0$,且$1-\frac{1}{1+x^2}\geq0$,故$h'(x)\geq0$,$h(x)$单调递增,因此$h(x)\geq h(0)=0$,即$(1+x)\ln(1+x)\geq\arctan x$,亦即$\ln(1+x)\geq\frac{\arctan x}{1+x}\quad(x\geq0)$.
\end{solution}
\begin{example}{3.7-B-3}
    利用函数的凹凸性证明下列不等式:
(1) $\ln x \leqslant x - 1 \ (x > 0)$;\\
(2) $2\arctan\frac{a+b}{2} \geqslant \arctan a + \arctan b \ (a, b \geqslant 0)$;\\
(3) $1 + x^2 \leqslant 2^x \ (0 \leqslant x \leqslant 1)$;\\
(4) $\frac{x^n + y^n}{2} > \left(\frac{x+y}{2}\right)^n \ (x > 0, y > 0, x \neq y, n > 1)$.
\end{example}
\begin{solution}
    (1)考虑函数 $f(x) = \ln x$,其定义域为 $(0, +\infty)$。计算导数:
    \[f'(x) = \frac{1}{x}, \quad f''(x) = -\frac{1}{x^2} < 0\]
    因此 $f(x)$ 在 $(0, +\infty)$ 上是凹函数。根据凹函数的性质,对于任意 $x > 0$,有:
    \[f(x) \leq f(1) + f'(1)(x - 1)\]
    代入 $f(1) = 0$,$f'(1) = 1$,得:
    \[\ln x \leq 0 + 1 \cdot (x - 1) = x - 1\]
    即 $\ln x \leqslant x - 1 \ (x > 0)$.\\
    (2)考虑函数 $f(x) = \arctan x$,其定义域为 $[0, +\infty)$。计算导数:
    \[f'(x) = \frac{1}{1 + x^2}, \quad f''(x) = -\frac{2x}{(1 + x^2)^2} \leq 0\]
    因此 $f(x)$ 在 $[0, +\infty)$ 上是凹函数。根据凹函数的性质,对于任意 $a, b \geq 0$,有:
    \[f\left(\frac{a + b}{2}\right) \geq \frac{f(a) + f(b)}{2}\]
    即:
    \[2\arctan\frac{a + b}{2} \geq \arctan a + \arctan b \quad (a, b \geq 0)\]
    (3)考虑函数 $g(x) = 2^x - 1 - x^2$,我们需要证明在 $[0, 1]$ 上 $g(x) \geq 0$。计算导数:
    \[g'(x) = 2^x \ln 2 - 2x, \quad g''(x) = 2^x (\ln 2)^2 - 2\]
    在 $[0, 1]$ 上,由于 $2^x \leq 2$ 且 $(\ln 2)^2 <1$,所以 $g''(x) < 2 - 2 = 0$,即 $g(x)$ 是凹函数。
    由凹函数的性质,对于任意 $x \in [0, 1]$,有:
    \[g(x) \geq (1 - x)g(0) + x g(1)\]
    代入 $g(0) = 0$,$g(1) = 0$,得:
    \[g(x) \geq 0\]
    即 $2^x - 1 - x^2 \geq 0$,所以 $1 + x^2 \leqslant 2^x \ (0 \leqslant x \leqslant 1)$。\\
    (4)考虑函数 $f(x) = x^n \ (n > 1)$,其定义域为 $(0, +\infty)$。计算导数:
    \[f'(x) = n x^{n-1}, \quad f''(x) = n(n-1) x^{n-2} > 0\]
    因此 $f(x)$ 是凸函数。根据凸函数的性质,对于 $x > 0, y > 0, x \neq y$,有:
    \[\frac{f(x) + f(y)}{2} > f\left(\frac{x + y}{2}\right)\]
    即:
    \[\frac{x^n + y^n}{2} > \left(\frac{x + y}{2}\right)^n \quad (x > 0, y > 0, x \neq y, n > 1)\]
\end{solution}
\begin{example}{3.8-A-6}{}
    $\text{求从点 }M(p,p)\text{到抛物线 }y^2=2px\text{ 的最短距离}.$
\end{example}
\begin{solution}
    设函数$f(y)=\left(\dfrac{y^2}{2p}-p\right)^2+(y-p)^2,f'(y)=2\left(\dfrac{y^2}{2p}-p\right)\dfrac{y}{p}+2(y-p)=0$,导函数可以化为
    \[\dfrac{f'(y)}{p}=2\left(\dfrac12\left(\dfrac{y}{p}\right)^2-1\right)\dfrac{y}{p}+2\dfrac{y}{p}-2=\left(\dfrac{y}{p}\right)^3-2\]单调递增,那么$f'(y)=0\Leftrightarrow y=\sqrt[3]{2}p$,则$f(y)$在$(-\infty,\sqrt[3]{2}p)$单调递减,在$(\sqrt[3]{2}p,+\infty)$上单调递增,最小值为$p\sqrt{\left((2^{-\frac13}-1)^2+(2^{\frac13}-1)^2\right)}=p(\sqrt[3]{2}-1)\sqrt{\frac{\sqrt[3]{2}+2}{2}}.$
\end{solution}
\begin{example}{3.8-A-7}{}
    从面积为常数 $S$ 的一切矩形中,求其周长为最小者
\end{example}
\begin{solution}
    设$ab=S$,则$a+b\geq 2\sqrt{ab}=2\sqrt{S}$,当且仅当$a=b$,即正方形时取等号,所以最小者为边为$\sqrt{S}$的正方形
\end{solution}
\begin{example}{3.8-A-8}{}
    在椭圆$\frac{x^2}{a^2}+\frac{y^2}{b^2}=1$中,嵌入有最大面积而边平行于椭圆轴的矩形,求此矩形的边长
\end{example}
\begin{solution}
    设第一象限的点$P(x,y)$,使用基本不等式,矩形面积为$S=4xy\leq 2ab\left(\frac{x^2}{a^2}+\frac{y^2}{b^2}\right)=2ab$,解方程$\begin{cases}\frac{x^2}{a^2}+\frac{y^2}{b^2}=1\\\dfrac{y}{x}=\dfrac{b}{a}\end{cases}$当且仅当矩形的边长为$\sqrt{2}a,\sqrt{2}b$时取等号
\end{solution}
\begin{example}{3.8-B-5}{}
    求由 $y$ 轴上的一个给定点 $(0,b)$ 到抛物线 $x^{2}=4y$ 上的点的最短距离.
\end{example}
\begin{solution}
    设距离的平方为$f(x)=x^2+\left(\dfrac{x^2}{4}-b\right)^2$,则
    \[f'(x)=2x+2\left(\dfrac{x^2}{4}-b\right)\dfrac{x}2=x\left(\dfrac{x^2}{4}-(b-2)\right)\]
    当$b\leq 2$时,$f'(x)$在$x\leq 0$时小于等于0,在$x>0$时大于0,则$f(x)$在$(-\infty,0)$单调递减,在$[0,+\infty)$上单调递增,最小值为$f(0)=b^2$;\\
    当$b>2$时,$f'(x)$的根为$x_1=-2\sqrt{b-2},x_2=0,x_3=2\sqrt{b-2}$,$f'(x)$在$(-\infty,x_1)$小于0,在$[x_1,0]$上大于等于0,在$(0,x_2)$上小于0,在$[x_2,+\infty)$上大于等于0\\
    所以$f(x)$在$(-\infty,x_1)$单调递减,在$[x_1,0]$上单调递增,在$(0,x_2)$上单调递减,在$[x_2,+\infty)$上单调递增,又由于$f(x)$为偶函数,所以最小值为$f(-2\sqrt{b-2})=f(2\sqrt{b-2})=4(b-1)$,则距离的最小值为$\begin{cases}|b|,b\leq 2\\2\sqrt{b-1},b>2\end{cases}$
\end{solution}
\begin{example}{3.8-B-6}{}
    在椭圆$\frac{x^2}{a^2}+\frac{y^2}{b^2}=1$ 的第一象限部分求一点 $P$,使该点处的切线、椭圆及两坐标轴所围图形的面积为最小(其中$a>0,b>0).$
\end{example}
\begin{solution}
    取$P(x,y)$,切线斜率为$k=-\dfrac{b^2x^2}{a^2y^2}$,切线方程为$Y-y=-\dfrac{b^2x}{a^2y}(X-x)$,横截距和纵截距为$\dfrac{a^2}x,\dfrac{b^2}{y}$,三角形面积为$\dfrac{a^2b^2}{2xy}=\dfrac{a^3b}{2x\sqrt{a^2-x^2}}$,则设$f(x)=x^2(a-x^2),f'(x)=2a^2x-4x^3=2x(a^2-2x^2)=0$,又因为$x>0$,所以$f'(x)$在$(0,\dfrac{a}{\sqrt2})$大于0,在$[\dfrac{a}{\sqrt2},+\infty)$小于等于0,所以最小值为$f(\dfrac{a}{\sqrt2})$,所以所求的$P$坐标为$(\dfrac{a}{\sqrt2},\dfrac{b}{\sqrt2})$,三角形的面积是$S=\dfrac{a^2b^2}{2xy}=ab$
\end{solution}
