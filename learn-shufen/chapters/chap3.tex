\chapter{一元函数微分学}
\section{习题3.1}
\begin{example}{3.1-A-3}{}
下列各式可否成为$f(x)$在$x_{0}$点的导数的定义?请说明理由.\\
(1) $y=f(x)$在$(a,b)$内定义,$x_{0}\in(a,b)$,若极限$\displaystyle\lim_{\Delta x \to 0}\frac{f(x_{0})-f(x_{0}-\Delta x)}{\Delta x}$存在,则称该极限为$f(x)$在$x_{0}$点的导数.\\
(2) $y=f(x)$在$(a,b)$内定义,$x_{0}\in(a,b)$,若极限$\displaystyle\lim_{x \to x_{0}}\frac{f(x)-f(x_{0})}{x-x_{0}}$存在,则称该极限为$f(x)$在$x_{0}$点的导数.\\
(3) $y=f(x)$在$(a,b)$内定义,$x_{0}\in(a,b)$,若极限$\displaystyle\lim_{\Delta x \to 0}\frac{f(x_{0}+\Delta x)-f(x_{0}-\Delta x)}{2\Delta x}$存在,则称该极限为$f(x)$在$x_{0}$点的导数.                                                   
\end{example}
\begin{solution}
(1)可以,因为\[\lim_{\Delta x \to 0}\dfrac{f(x_0)-f(x_0-\Delta x)}{\Delta x}=\lim_{\Delta x \to 0}\dfrac{f(x_0-\Delta x)-f(x_0)}{-\Delta x}=f'(x_0)\]
(2)可以,因为\[\lim_{x \to x_{0}}\dfrac{f(x)-f(x_{0})}{x-x_{0}}=\lim_{x-x_0\to 0}\dfrac{f(x)-f(x_{0})}{x-x_0}=\lim_{x-x_0\to 0}\dfrac{f(x_0+x-x_0)-f(x_0)}{x-x_0}=f'(x_0)\]
(3)不可以,当$f'(x_0)$存在时,该极限等于$f'(x_0)$,这是因为
\begin{align*}
    \lim_{\Delta x \to 0}\frac{f(x_{0}+\Delta x)-f(x_{0}-\Delta x)}{2\Delta x}&=\lim_{\Delta x \to 0}\frac{f(x_{0}+\Delta x)-f(x_{0})+f(x_{0})-f'(x_{0}-\Delta x)}{2\Delta x}\\
    &=\lim_{\Delta x \to 0}\frac{f(x_{0}+\Delta x)-f(x_{0})}{2\Delta x}+\lim_{\Delta x \to 0}\frac{f(x_{0})-f'(x_{0}-\Delta x)}{2\Delta x}\\
    &=\dfrac12 f'(x_0)+\dfrac12 f'(x_0)=f'(x_0)
\end{align*}
但是当$f'(x_0)$不存在,而左右导数存在时,该极限推出的是左导数和右导数的平均值,并非$f'(x_0)$,比如当$f(x)=|x|,x=0$导数不存在,但是\[\lim_{\Delta x \to 0}\frac{f(x_{0}+\Delta x)-f(x_{0}-\Delta x)}{2\Delta x}=\lim_{\Delta\to 0}\dfrac{|\Delta x|-|-\Delta x|}{2\Delta x}=0\]矛盾,所以不可以用作导数的定义.
\end{solution}
\begin{example}{3.1-A-9}{}
    利用定义求函数在$x=0$处的导数$f(x)=\begin{cases}x^2\sin \dfrac1{x},x\neq 0\\0,x=0\end{cases}$.
\end{example}
\begin{solution}
    利用定义,有\[f'(0)=\lim_{h\to 0}\dfrac{f(0+h)-f(0)}{h}=\lim_{h\to 0}h\sin \dfrac1{h}=0\]
\end{solution}
\begin{example}{3.1-B-3}{}
    求证:偶函数的导数是奇函数,奇函数的导数是偶函数;
\end{example}
\begin{solution}
    设$f(x)$是偶函数,所以$f(x)-f(-x)=0$,两边求导得到$f'(x)+f'(-x)=0$,所以偶函数的导数是奇函数;设$f(x)$是奇函数,所以$f(x)+f(-x)=0$,两边求导得到$f'(x)-f'(-x)=0$,所以奇函数的导数是偶函数\\我们也可以考虑定义:设$f(x)$是偶函数,所以\[f'(x)=\lim_{h\to 0}\dfrac{f(x+h)-f(x)}{h}=\lim_{h\to 0}\dfrac{f(-x-h)-f(-x)}{-(-h)}=-f'(-x)\]所以偶函数的导数是奇函数;设$f(x)$是奇函数,所以\[f'(x)=\lim_{h\to 0}\dfrac{f(x+h)-f(x)}{h}=\lim_{h\to 0}\dfrac{f(-x)-f(-x-h)}{-(-h)}=f'(-x)\]所以奇函数的导数是偶函数。
\end{solution}
\begin{example}{3.1-B-4}{}
    求证:周期函数的导数仍然是周期函数
\end{example}
\begin{solution}
    设$f(x)$是周期函数,$T$是周期,$f(x+T)=f(x)$,则两边求导得到$f'(x+T)=f'(x)$,所以$f'(x)$也是周期函数,也可以从定义考虑:\[f'(x)=\lim_{h\to 0}\dfrac{f(x+h)-f(x)}{h}=\lim_{h\to 0}\dfrac{f(x+T+h)-f(x+T)}{h}=f'(x+T)\]所以$f'(x)$是周期函数.
\end{solution}
\section{习题 3.2}
\begin{example}{3.2-A-2}{}
    求导(1)$y=1-\dfrac1x+\dfrac1{x^2}$;(2)$y=\dfrac1{x^3+2x+1}$;(3)$y=\dfrac1{x+\sqrt{x}}$.
\end{example}
\begin{solution}
    (1)$y'=\dfrac1{x^2}-\dfrac2{x^3}$;(2)$y'=\dfrac{-3x^2-2}{(x^3+2x+1)^2}$;(3)$y'=-\dfrac{1+2\sqrt{x}}{2\sqrt{x}(x+\sqrt{x})^2}$
\end{solution}
\begin{example}{3.2-A-4}{}
    求导(1)$y=\ln(x+\sqrt{1+x^2})$;(2)$y=\dfrac{x}{\sqrt{a^2-x^2}}$;(3)$y=\sqrt{x+\sqrt{x+\sqrt{x}}}$.
\end{example}
\begin{solution}
    (1)$y'=\dfrac{1+\frac{2x}{2\sqrt{1+x^2}}}{x+\sqrt{1+x^2}}=\dfrac1{\sqrt{1+x^2}}$;\\
    (2)$y'=\dfrac{\sqrt{a^2-x^2}-x\frac{-2x}{2\sqrt{a^2-x^2}}}{(\sqrt{a^2-x^2})^2}=\dfrac{a^2}{(\sqrt{a^2-x^2})^3}$;\\
    (3)$y'=\dfrac{1+\frac{1+\frac1{2\sqrt{x}}}{2\sqrt{x+\sqrt{x}}}}{2\sqrt{x+\sqrt{x+\sqrt{x}}}}=\dfrac{1+2\sqrt{x}+4\sqrt{x}\sqrt{x+\sqrt{x}}}{8\sqrt{x}\sqrt{x+\sqrt{x}}\sqrt{x+\sqrt{x+\sqrt{x}}}}$.
\end{solution}
\begin{example}{3.2-B-1}{}
    设 $y=f(x)$为严格递增的可导函数,$x=\varphi(y)$是它的反函数.证明:\\
(1) 当 $h\neq 0$时 ,$ f( x+ h) - f( x) = k\neq 0$,若 记 $f( x+ h) = y+ k$, 则$\varphi ( y+ k) = x+ h.$\\
(2)当$k\to0$时,$\dfrac{\varphi(y+k)-\varphi(y)}{k}=\dfrac{h}{k}=\dfrac{h}{y+k-y}=\dfrac{h}{f(x+h)-f(x)}$趋于$\dfrac1{f^{\prime}(x)}.$
\end{example}
\begin{solution}
    (1)由$x=\varphi(f(x))=f(\varphi(x))$,以及 $y=f(x)$为严格递增的可导函数,所以$\varphi'(x)=\dfrac{1}{f'(\varphi(x))}>0$,所以$x=\varphi(y)$为严格递增的可导函数,所以\[f( x+ h) = y+ k\Rightarrow \varphi(f(x+h))=x+h=\varphi ( y+ k)\]\\
    (2)$\displaystyle\lim_{k\to 0}\dfrac{\varphi(y+k)-\varphi(y)}{k}=\lim_{h\to 0}\dfrac{1}{\dfrac{f(x+h)-f(x)}{h}}=\dfrac1{f'(x)}$,这里$h\to 0\Leftrightarrow k\to 0$.
\end{solution}
\begin{example}{3.2-B-2}{}
    $\text{设 }f(x)=x^3+2x^2+3x+1,\text{用 }\varphi\text{表示 }f\text{ 的反函数}.\text{求证}:f(1)=7,\varphi(7)=1.\text{并计算}\varphi^{\prime}(7).$
\end{example}
\begin{solution}
代入得到$f(1)=1+2+3+1=7$,所以$\varphi(7)=1$,由$f(x)=x^3+2x^2+3x+1$可得$f'(x)=3x^2+4x+3$,所以$\varphi^{\prime}(7)=\dfrac{1}{f'(\varphi(7))}=\dfrac{1}{f'(1)}=\dfrac1{10}$.
\end{solution}
\begin{example}{3.2-B-3}{}
    $\text{设 }y=(\arcsin x)^{2},\text{证明}(1-x^{2})y''-xy'=2.$
\end{example}
\begin{solution}
    两边求导数得到$y'=2\arcsin x\cdot \dfrac1{\cos\arcsin x}=2\dfrac{\arcsin x}{\sqrt{1-x^2}}$,所以$\sqrt{1-x^2}y'=2\arcsin x$,再次两边求导得到$\sqrt{1-x^2}y''+\dfrac{-2x}{2\sqrt{1-x^2}}y'=\dfrac2{\sqrt{1-x^2}}$,等价变形就有$(1-x^{2})y''-xy'=2$.
\end{solution}
\begin{example}{3.2-B-4}{}
    求下列函数的$n$阶导数$y^{(n)}$:(1) $y= \frac 1{1- x^{2}}$ ;(2) $y= \sin ^{2}x.$
\end{example}
\begin{solution}
    (1)裂项有$y=\dfrac12\left(\dfrac1{1+x}+\dfrac1{1-x}\right)$,所以$y^{(n)}=\dfrac12\left[\left(\dfrac1{1+x}\right)^{(n)}+\left(\dfrac1{1-x}\right)^{(n)}\right]$,求$n$阶导数有$y^{(n)}=\dfrac12\left[\dfrac{(-1)^nn!}{(1+x)^{n+1}}+\dfrac{n!}{(1-x)^{n+1}}\right]$\\
    (2)$y=\sin^x=\dfrac12(1-\cos 2x),y^{(n)}=\left[\dfrac12-\dfrac12\cos 2x\right]^{(n)}=-2^{n-1}\cos\left(2x+\dfrac{n\pi}2\right)$,诱导公式得到$y^{(n)}=2^{n-1}\sin\left(2x+\dfrac{(n-1)\pi}2\right)$
\end{solution}
\newpage
\section{习题 3.3}
\begin{example}{3.3-A-2}{}
    方程$\e^y+xy+y=2$确定隐函数$y=y(x)$,求$y'(x)'$.
\end{example}
\begin{solution}
    两边求导得到$y'\e^y+y+xy'+y'=0$解得$y'=-\dfrac{y}{\e^y+x+1}$
\end{solution}
\begin{example}{3.3-A-3}{求下列方程所确定的隐函数$y=y(x)$,求$y'(x)'$}
    \[\begin{array}{l@{\quad}l@{\quad}l}
    \text{(1)}\e^x-\e^y+xy=0&\text{(2)}x^2+y^2-\arcsin y=0&\text{(3)}x^y=y^x\\
    \text{(4)}\arctan\dfrac{y}{x}=\ln\sqrt{x^2+y^2}&\text{(5)}x^2-2xy+y^2=2x&\text{(6)}\sqrt{x}+\sqrt{y}=1\\
    \text{(7)}xy^2+\e^y=\cos(x+y^2)&\text{(8)}\ln y=\sqrt{\dfrac{{1-x}}{1+x}}
    \end{array}\]
\end{example}
\begin{solution}
    (1)两边求导得$\e^x-y'\e^y+y+xy'=0$解得$y'=\dfrac{\e^x+y}{\e^y-x}$\\
    (2)两边求导得$2x+2yy'-\dfrac{y'}{\sqrt{1-y^2}}=0$解得$y'=\dfrac{2x\sqrt{1-y^2}}{1-2y\sqrt{1-y^2}}$\\
    (3)变换得到$\dfrac{\ln x}{x}=\dfrac{\ln y}{y}$,两边求导得到$y'=\dfrac{y^2(1-\ln x)}{x^2(1-\ln y)}$\\
    (4)两边求导得$\dfrac{1}{1+\frac{y^2}{x^2}}\dfrac{y'x-y}{x^2}=\dfrac12\dfrac{2x+2yy'}{x^2+y^2}$解得$y'=\dfrac{x+y}{x-y}$\\
    (5)两边求导得$2x-2y-2xy'+2yy'=2$解得$y'=\dfrac{x-y-1}{x-y}$\\
    (6)两边求导得$\dfrac1{2\sqrt{x}}+\dfrac{y'}{2\sqrt{y}}=0$,解得$y'=-\sqrt{\dfrac{y}{x}}$\\
    (7)两边求导得$y^2+2xyy'+\e^yy'=-(1+2yy')\sin(x+y^2)$,解得$y'=-\dfrac{y^2+\sin(x+y^2)}{\e^y+2xy+2y\sin(x+y^2)}$.\\
    (8)两边求导得$\dfrac{y'}{y}=\dfrac{\frac{-2}{(1+x)^2}}{2\sqrt{\frac{{1-x}}{1+x}}}$,解得$y'=-\dfrac{y}{(1+x)^2}\sqrt{\dfrac{{1-x}}{1+x}}$.
\end{solution}
\begin{example}{3.3-A-4~~求下列由参数方程表示的函数的导数:}
(1) $x=\sqrt[3]{1-\sqrt{t}}, y=\sqrt{1-\sqrt[3]{t}}$, 求 $\dfrac{\dd y}{\dd x}$;\quad
(2) $x=\sin^{2}t, y=\cos^{2}t$, 求 $\dfrac{\dd y}{\dd x}$;\\
(3) $x=1+t^{3}, y=e^{2t}$, 求 $\left.\dfrac{\dd y}{\dd x}\right|_{x=2}$;\quad
(4) $x=1+t^{2}, y=\cos t$, 求 $\dfrac{\dd y}{\dd x}$;\\
(5) $x=e^{t}\sin t, y=e^{-t}\cos t$, 求 $\dfrac{\dd y}{\dd x}$.
\end{example}
\begin{solution}
    使用链式法则,分别求导得:\\
    (1)$\dfrac{\dd y}{\dd x}=\dfrac{\dd y}{\dd t}\dfrac{\dd t}{\dd x}=\dfrac{-\frac32\sqrt{t}}{2\sqrt{1-\sqrt[3]{t}}}\dfrac{\frac23(1-\sqrt{t})^{\frac23}}{-\frac1{2\sqrt{t}}}=\dfrac{\sqrt{t}\sqrt[3]{(1-\sqrt{t})^2}}{\sqrt[3]{t^2}\sqrt{1-\sqrt[3]{t}}}$\\
    (2)$x+y=1\Rightarrow \dfrac{\dd y}{\dd x}=-1$.\quad(3)$\dfrac{\dd y}{\dd x}=\dfrac{\dd y}{\dd t}\dfrac{\dd t}{\dd x}=\dfrac{2\e^{2t}}{3t^2}$,代入$x=2,t=1$得到$\left.\dfrac{\dd y}{\dd x}\right|_{x=2}=\dfrac23\e^2$.\\
    (4)$\dfrac{\dd y}{\dd x}=\dfrac{\dd y}{\dd t}\dfrac{\dd t}{\dd x}=\dfrac{-\sin t}{2t}$.\quad (5)$\dfrac{\dd y}{\dd x}=\dfrac{\dd y}{\dd t}\dfrac{\dd t}{\dd x}=\dfrac{\e^{-t}(-\sin t-\cos t)}{\e^t(\sin t+\cos t)}=-\e^{2t}$.
\end{solution}
\begin{example}{用对数求导法求导数}{}
    (1)$y=x^{\sin x},(x>0)$;(2)$y=(\sqrt{x})^{\ln x},(x>0)$;(3)$y=a^{\sin x},(a>0)$;\\
    (4)$y=(1+x)^{\frac1x},(x>0)$;(5)$y=\dfrac{(x+5)^2(x-4)^{\frac13}}{(x+2)^5(x+4)^{\frac12}}$;(6)$y=x\sqrt{\dfrac{1-x}{1+x}}$
\end{example}
\begin{solution}
(1)$\ln y=(\sin x)\ln x\Rightarrow \dfrac{y'}{y}=\dfrac{\sin x}{x}+(\cos x)\ln x\Rightarrow y'=x^{\sin x}\left((\cos x)\ln x+\dfrac{\sin x}{x}\right)$\\
(2)$\ln y=\dfrac12\ln^2 x\Rightarrow \dfrac{\dd\ln y}{\dd y}\dfrac{\dd y}{\dd x}=\dfrac122\ln x\dfrac1x\Rightarrow \dfrac{\dd y}{\dd x}=\dfrac{y\ln x}{x}$\\
(3)$\ln y=\sin x\ln a\Rightarrow \dfrac{\dd\ln y}{\dd y}\dfrac{\dd y}{\dd x}=\cos x\ln a\Rightarrow \dfrac{\dd y}{\dd x}=a^{\sin x}\cos x\ln a$\\
(4)$\ln y=\dfrac{\ln(x+1)}{x}\Rightarrow \dfrac{\dd\ln y}{\dd y}\dfrac{\dd y}{\dd x}=\dfrac{1}{x(x+1)}-\dfrac{\ln(x+1)}{x^2}$解得:
\[\Rightarrow \dfrac{\dd y}{\dd x}=(1+x)^{\frac1x}\left(\dfrac{1}{x(x+1)}-\dfrac{\ln(x+1)}{x^2}\right)\]
(5)定义域为$(-4,-2)\cup(-2,+\infty)$,取绝对值,然后取对数
\[\ln|y|=\ln\left(\frac{(x+5)^2|x-4|^{\frac{1}{3}}}{|x+2|^5(x+4)^{\frac{1}{2}}}\right)=2\ln|x+5|+\frac{1}{3}\ln|x-4|-5\ln|x+2|-\frac{1}{2}\ln(x+4)\]
两边对$x$微分:\[\frac{\dd}{\dd x}\ln|y|=\frac{1}{y}\frac{\dd y}{\dd x}=2\cdot\frac{1}{x+5}+\frac{1}{3}\cdot\frac{1}{x-4}-5\cdot\frac{1}{x+2}-\frac{1}{2}\cdot\frac{1}{x+4}\]
解得:\[\frac{\dd y}{\dd x}=\frac{(x+5)^2(x-4)^{\frac{1}{3}}}{(x+2)^5(x+4)^{\frac{1}{2}}}\left(\frac{2}{x+5}+\frac{1}{3(x-4)}-\frac{5}{x+2}-\frac{1}{2(x+4)}\right)\]
(6)定义域为$(-1,1]$,$|y|=|x|\sqrt{\dfrac{1-x}{1+x}}\Rightarrow \ln|y|=\ln|x|+\dfrac12\ln|1-x|-\dfrac12\ln|1+x|$,两边对$x$微分:
\[\frac{\dd}{\dd x}\ln|y|=\frac{1}{y}\frac{\dd y}{\dd x}=\dfrac1x+\dfrac1{2(1-x)}-\dfrac1{2(1+x)}\Rightarrow \frac{\dd y}{\dd x}=y\left(\dfrac1x+\dfrac1{2(1-x)}-\dfrac1{2(1+x)}\right)\]
\end{solution}
\begin{example}{3.3-A-6}{}
    下列参数方程给出函数 $y=y(x),\text{求}\dfrac{\mathrm{d}^2y}{\mathrm{d}x^2}$\\
    $(1)x=a\cos t,y=a\sin t;\quad(2)x=2t-t^2,y=3t-t^3;\\(3)x=\ln(1+t^2),y=\arctan t;\quad(4)x=\ln(t+\sqrt{t^2+1}),y=t^2.$
\end{example}
\begin{solution}
    (1)$\dfrac{\dd^2y}{\dd x^2}=\dfrac{\dd}{\dd x}\left(\dfrac{\dd y}{\dd x}\right)=\dfrac{\dd}{\dd t}\left(\dfrac{\dd y}{\dd t}\dfrac{\dd t}{\dd x}\right)\dfrac{\dd t}{\dd x}=\dfrac{\dd}{\dd t}\left(-\cot t\right)\dfrac{\dd t}{\dd x}=\dfrac{\csc^2 t}{-a\sin t}=-\dfrac1{a\sin^3t}$\\
    (2)$\dfrac{\dd^2y}{\dd x^2}=\dfrac{\dd}{\dd x}\left(\dfrac{\dd y}{\dd x}\right)=\dfrac{\dd}{\dd t}\left(\dfrac{\dd y}{\dd t}\dfrac{\dd t}{\dd x}\right)\dfrac{\dd t}{\dd x}=\dfrac{\dd}{\dd t}\left(\dfrac32(1+t)\right)\dfrac{\dd t}{\dd x}=\dfrac{3}{4(1-t)}$\\
    (3)$\dfrac{\dd^2y}{\dd x^2}=\dfrac{\dd}{\dd x}\left(\dfrac{\dd y}{\dd x}\right)=\dfrac{\dd}{\dd t}\left(\dfrac{\dd y}{\dd t}\dfrac{\dd t}{\dd x}\right)\dfrac{\dd t}{\dd x}=\dfrac{\dd}{\dd t}\left(\dfrac1{1+t^2}\dfrac{1+t^2}{2t}\right)\dfrac{\dd t}{\dd x}=-\dfrac{(1+t^2)}{4t^3}$\\
    (4)$\dfrac{\dd^2y}{\dd x^2}=\dfrac{\dd}{\dd x}\left(\dfrac{\dd y}{\dd x}\right)=\dfrac{\dd}{\dd t}\left(\dfrac{\dd y}{\dd t}\dfrac{\dd t}{\dd x}\right)\dfrac{\dd t}{\dd x}=\dfrac{\dd}{\dd t}\left(2t\sqrt{t^2+1}\right)\dfrac{\dd t}{\dd x}=\dfrac{4t^2+2}{\sqrt{t^2+1}}\sqrt{t^2+1}=4t^2+2$.
\end{solution}
\begin{example}{3.3-A-7~求下列隐函数的二阶导数$y''$}{}
    (1)$x^3+y^3-3axy=0,(a>0)$;(2)$y^2+2\ln y=x^4$;(3)$xy=\e^{x+y}$;(4)$y=1-x\e^y$
\end{example}
\begin{solution}
    (1)两边求导数得\[3x^2+3y^2y'=3a(y+xy')\Rightarrow (ax-y^2)y'=x^2-ay\Rightarrow (a-2yy')y'+(ax-y^2)y''=2x-ay'\]解得
    \[y''=\dfrac1{y^2-ax}\left[\dfrac{2a(ay-x^2)}{y^2-ax}-2y\left(\dfrac{ay-x^2}{y^2-ax}\right)^2-2x\right]\]
    (2)两边求导数得\[2yy'+\dfrac{2y'}{y}=4x^3\Rightarrow (y^2+1)y'=2x^3y\Rightarrow (2yy')y'+(y^2+1)y''=2(3x^2y+x^3y')\]
    解得\[y''=\dfrac{2x^2y}{(1+y^2)^2}\left[3(1+y^2)^2+2x^4(1-y^2)\right]\]
    (3)两边求导数得$xy'+y=\e^{x+y}(1+y')\Rightarrow (\e^{x+y}-x)y'=y-\e^{x+y}\Leftrightarrow (xy-x)y'=y-xy$,再次在两边求导
    \[(xy'+y-1)y'+(xy-x)y''+y=0\Rightarrow y''=\dfrac{y}{x-xy}+\dfrac{(x+y-2)(xy-y)}{(x-xy)^2}+\dfrac{x(xy-y^2)}{(x-xy)^3}\]
    (4)两边求导数得$y'=-x\e^yy'-\e^y\Rightarrow y'(1+x\e^y)=-\e^y\Rightarrow y'(y-2)=\e^y$,再次求导有
    \[y''=\e^{2y}[\dfrac1{(y-2)^2}-\dfrac1{(y-2)^3}]\Leftrightarrow y''=\dfrac{2\e^{2y}}{(1+x\e^y)^2}-\dfrac{x\e^{3y}}{(1+x\e^y)^3}\]
\end{solution}
\begin{example}{3.3-A-9}{}
    求$\dfrac{\dd y}{\dd x}$:(1)$r^2=2a^2\cos 2\theta$在$\theta=\dfrac{\pi}6$处\\
    (2)$r=a\e^{m\theta}$,其中$r=\sqrt{x^2+y^2}$以及$\theta=\arctan\dfrac{y}{x}$为极坐标.
\end{example}
\begin{solution}
    \[\dfrac{\dd y}{\dd x}=\dfrac{\frac{\dd y}{\dd \theta}}{\frac{\dd x}{\dd \theta}}=\dfrac{\frac{\dd r(\theta)\sin\theta}{\dd \theta}}{\frac{\dd r(\theta)\cos\theta}{\dd \theta}}=\dfrac{r'(\theta)\sin\theta+r(\theta)\cos\theta}{r'(\theta)\cos\theta-r(\theta)\sin\theta}\]
    (1)参数方程为$r^2(\theta)=2a^2\cos 2\theta,r'(\theta)r(\theta)=2a^2(-\sin2\theta),r(\dfrac{\pi}6)=a^2,r(\theta)=a,r'(\theta)=-\sqrt{3}a$,代入
    \[\dfrac{\dd y}{\dd x}=\dfrac{r'(\theta)\sin\theta+r(\theta)\cos\theta}{r'(\theta)\cos\theta-r(\theta)\sin\theta}=0\]
    (2)参数方程为$\begin{cases}x=a\e^{m\theta}\cos\theta\\y=a\e^{m\theta}\sin\theta\end{cases},r(\theta)=a\e^{m\theta},r'(\theta)=am\e^{m\theta}$,代入就有
    \[\dfrac{\dd y}{\dd x}=\dfrac{r'(\theta)\sin\theta+r(\theta)\cos\theta}{r'(\theta)\cos\theta-r(\theta)\sin\theta}=\dfrac{a\cos\theta+m\sin\theta}{-a\sin\theta+m\cos\theta}=\tan\left(\theta+\arctan\dfrac1m\right)\]
\end{solution}
\begin{example}{3.3-B-1}{}
    求导:$y=\e^x+\e^{\e^x}\quad y=\left(\dfrac{a}{b}\right)^x\left(\dfrac{b}{x}\right)^a\left(\dfrac{x}{a}\right)^b\quad y=3^x\ln x$
\end{example}
\begin{solution}
(1)$y'=\e^x+\e^{\e^x}\e^x$;\\
(2)$\ln y=x\ln\left(\dfrac{a}{b}\right)+a\ln\left(\dfrac{b}{x}\right)+b\ln\left(\dfrac{x}{a}\right)=x\ln\left(\dfrac{a}{b}\right)+(b-a)\ln x$,两边求导得\[\dfrac{y'}{y}=\ln\left(\dfrac{a}{b}\right)+\dfrac{b-a}{x},y'=\left(\dfrac{a}{b}\right)^x\left(\dfrac{b}{x}\right)^a\left(\dfrac{x}{a}\right)^b\left(\ln\left(\dfrac{a}{b}\right)+\dfrac{b-a}{x}\right)\]
(3)$y'=3^x\ln 3\ln x+\dfrac{3^x}{x}$
\end{solution}
\section{习题3.4}
\begin{example}{3.4-A-5}{}
    利用一阶微分的形式不变性求微分:$(1)y=\arctan\e^x\quad(2)y=\e^{\sin x}$
\end{example}
\begin{solution}
    (1)$\dd y=\dfrac{\dd \e^x}{1+\e^{2x}}=\dfrac{\e^x}{1+\e^{2x}}\dd x$;
    (2)$\dd y=\e^{\sin x}\dd\sin x=\e^{\sin x}\cos x\dd x$
\end{solution}
\begin{example}{3.4-B-1}{}
    求下列函数的二阶微分$\dd^2y:(1)y=\sqrt{1+x^2}~~~(2)y=\dfrac{\ln x}{x}$
\end{example}
\begin{solution}
    (1)$\dd^2y=\dd(\dd y)=\dd(\dfrac{x}{\sqrt{1+x^2}}\dd x)=\dfrac{\sqrt{1+x^2}-x\frac{x}{\sqrt{1+x^2}}}{1+x^2}\dd x^2=\dfrac{1}{(1+x^2)^\frac32}\dd x^2$\\
    (2)$\dd^2y=\dd(\dd y)=\dd\left(\dfrac{1-\ln x}{x^2}\dd x\right)=\dfrac{-x-2x(1-\ln x)}{x^4}\dd x^2=\dfrac{2\ln x-3}{x^3}\dd x^2$
\end{solution}
\newpage
\begin{example}{}{}
    设函数$f(x)$在$(0,+\infty)$上三阶可导,且$f(x)>0,f'(x)>0,f''(x)>0,\displaystyle\lim_{x\to+\infty}\dfrac{f'(x)f'''(x)}{[f''(x)]^2}=a\ne1$,求极限$\displaystyle\lim_{x\to+\infty}\dfrac{f(x)f''(x)}{[f'(x)]^2}$.
\end{example}
\begin{solution}
    首先由$f(x)>0,f'(x)>0,f''(x)>0$得知$f'(x),f(x)$均单调递增,所以
    \[f(x+h)>f(x)+f'(\xi)(x+h-x)>f(x)+f'(x)h\]
    令$h\to+\infty$,得$f(x)\to+\infty$,现在对已知极限变形:
    \[a=\lim_{x\to+\infty}\dfrac{f'(x)f'''(x)}{[f''(x)]^2}=1-\lim_{x\to+\infty}\dfrac{[f''(x)]^2-f'(x)f'''(x)}{[f''(x)]^2}=1-\lim_{x\to+\infty}\dfrac{\dd}{\dd x}\left(\dfrac{f'(x)}{f''(x)}\right)\]
    所以$\displaystyle\lim_{x\to+\infty}\dfrac{\dd}{\dd x}\left(\dfrac{f'(x)}{f''(x)}\right)=1-a\ne 0$,同样可以对所求极限变形:
    \[\lim_{x\to+\infty}\dfrac{f(x)f''(x)}{[f'(x)]^2}=1-\lim_{x\to+\infty}\dfrac{[f'(x)]^2-f(x)f''(x)}{[f'(x)]^2}=1-\lim_{x\to+\infty}\dfrac{\dd}{\dd x}\left(\dfrac{f(x)}{f'(x)}\right)\]
    问题在于如何沟通$\displaystyle\lim_{x\to+\infty}\dfrac{\dd}{\dd x}\left(\dfrac{f'(x)}{f''(x)}\right)=1-a$和$\displaystyle\lim_{x\to+\infty}\dfrac{\dd}{\dd x}\left(\dfrac{f(x)}{f'(x)}\right)$,观察到它们是微分形式,所以不妨反方向利用洛必达法则。须知,洛必达法则在$\dfrac{*}{\infty}$的情形中是适用的,即若$f\to\infty,f,g$在$a$的某个去心邻域内可导,且$\displaystyle\lim_{x\to a}\dfrac{g'}{f'}$存在(或为无穷大,但分母不等于$0$),那么$\displaystyle\lim_{x\to a}\dfrac{g}{f}=\lim_{x\to a}\dfrac{g'}{f'}$;相关证明可以参阅数分教材(如陈纪修的,卓里奇的,等等).
    \begin{align*}\lim_{x\to+\infty}\dfrac{\frac{f(x)}{f'(x)}}{x}=\lim_{x\to+\infty}\dfrac{\dd}{\dd x}\left(\dfrac{f(x)}{f'(x)}\right)=\lim_{x\to+\infty}\dfrac{f(x)}{xf'(x)}\\
    \lim_{x\to+\infty}\dfrac{\frac{f'(x)}{f''(x)}}{x}=\lim_{x\to+\infty}\dfrac{\dd}{\dd x}\left(\dfrac{f'(x)}{f''(x)}\right)=\lim_{x\to+\infty}\dfrac{f'(x)}{xf''(x)}\end{align*}
    设$\displaystyle\lim_{x\to+\infty}\dfrac{f(x)f''(x)}{[f'(x)]^2}=A,\lim_{x\to+\infty}\dfrac{\dd}{\dd x}\left(\dfrac{f(x)}{f'(x)}\right)=1-A,\lim_{x\to+\infty}\dfrac{\dd}{\dd x}\left(\dfrac{f'(x)}{f''(x)}\right)=\lim_{x\to+\infty}\dfrac{f'(x)}{xf''(x)}=1-a$
    \[\displaystyle\lim_{x\to+\infty}\dfrac{xf''(x)}{f'(x)}\lim_{x\to+\infty}\dfrac{f(x)}{xf'(x)}=\lim_{x\to+\infty}\dfrac{f(x)f''(x)}{[f'(x)]^2}\]
    由于$\lim_{x\to+\infty}\dfrac{f'(x)}{xf''(x)}=1-a\Rightarrow\lim_{x\to+\infty}\dfrac{xf''(x)}{f'(x)}=\dfrac1{1-a}$即$\dfrac{1-A}{1-a}=A$,解得\[A=\displaystyle\lim_{x\to+\infty}\dfrac{f(x)f''(x)}{[f'(x)]^2}=\dfrac1{2-a}\]
\end{solution}
