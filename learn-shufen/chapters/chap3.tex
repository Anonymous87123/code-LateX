\chapter{一元函数微分学}
\section{习题3.1}
\begin{example}{3.1-A-3}{}
下列各式可否成为$f(x)$在$x_{0}$点的导数的定义?请说明理由.\\
(1) $y=f(x)$在$(a,b)$内定义,$x_{0}\in(a,b)$,若极限$\displaystyle\lim_{\Delta x \to 0}\frac{f(x_{0})-f(x_{0}-\Delta x)}{\Delta x}$存在,则称该极限为$f(x)$在$x_{0}$点的导数.\\
(2) $y=f(x)$在$(a,b)$内定义,$x_{0}\in(a,b)$,若极限$\displaystyle\lim_{x \to x_{0}}\frac{f(x)-f(x_{0})}{x-x_{0}}$存在,则称该极限为$f(x)$在$x_{0}$点的导数.\\
(3) $y=f(x)$在$(a,b)$内定义,$x_{0}\in(a,b)$,若极限$\displaystyle\lim_{\Delta x \to 0}\frac{f(x_{0}+\Delta x)-f(x_{0}-\Delta x)}{2\Delta x}$存在,则称该极限为$f(x)$在$x_{0}$点的导数.                                                   
\end{example}
\begin{solution}
(1)可以,因为\[\lim_{\Delta x \to 0}\dfrac{f(x_0)-f(x_0-\Delta x)}{\Delta x}=\lim_{\Delta x \to 0}\dfrac{f(x_0-\Delta x)-f(x_0)}{-\Delta x}=f'(x_0)\]
(2)可以,因为\[\lim_{x \to x_{0}}\dfrac{f(x)-f(x_{0})}{x-x_{0}}=\lim_{x-x_0\to 0}\dfrac{f(x)-f(x_{0})}{x-x_0}=\lim_{x-x_0\to 0}\dfrac{f(x_0+x-x_0)-f(x_0)}{x-x_0}=f'(x_0)\]
(3)不可以,当$f'(x_0)$存在时,该极限等于$f'(x_0)$,这是因为
\begin{align*}
    \lim_{\Delta x \to 0}\frac{f(x_{0}+\Delta x)-f(x_{0}-\Delta x)}{2\Delta x}&=\lim_{\Delta x \to 0}\frac{f(x_{0}+\Delta x)-f(x_{0})+f(x_{0})-f'(x_{0}-\Delta x)}{2\Delta x}\\
    &=\lim_{\Delta x \to 0}\frac{f(x_{0}+\Delta x)-f(x_{0})}{2\Delta x}+\lim_{\Delta x \to 0}\frac{f(x_{0})-f'(x_{0}-\Delta x)}{2\Delta x}\\
    &=\dfrac12 f'(x_0)+\dfrac12 f'(x_0)=f'(x_0)
\end{align*}
但是当$f'(x_0)$不存在,而左右导数存在时,该极限推出的是左导数和右导数的平均值,并非$f'(x_0)$,比如当$f(x)=|x|,x=0$导数不存在,但是\[\lim_{\Delta x \to 0}\frac{f(x_{0}+\Delta x)-f(x_{0}-\Delta x)}{2\Delta x}=\lim_{\Delta\to 0}\dfrac{|\Delta x|-|-\Delta x|}{2\Delta x}=0\]矛盾,所以不可以用作导数的定义.
\end{solution}
\begin{example}{3.1-A-9}{}
    利用定义求函数在$x=0$处的导数$f(x)=\begin{cases}x^2\sin \dfrac1{x},x\neq 0\\0,x=0\end{cases}$.
\end{example}
\begin{solution}
    利用定义,有\[f'(0)=\lim_{h\to 0}\dfrac{f(0+h)-f(0)}{h}=\lim_{h\to 0}h\sin \dfrac1{h}=0\]
\end{solution}
\begin{example}{3.1-B-3}{}
    求证:偶函数的导数是奇函数,奇函数的导数是偶函数;
\end{example}
\begin{solution}
    设$f(x)$是偶函数,所以$f(x)-f(-x)=0$,两边求导得到$f'(x)+f'(-x)=0$,所以偶函数的导数是奇函数;设$f(x)$是奇函数,所以$f(x)+f(-x)=0$,两边求导得到$f'(x)-f'(-x)=0$,所以奇函数的导数是偶函数\\我们也可以考虑定义:设$f(x)$是偶函数,所以\[f'(x)=\lim_{h\to 0}\dfrac{f(x+h)-f(x)}{h}=\lim_{h\to 0}\dfrac{f(-x-h)-f(-x)}{-(-h)}=-f'(-x)\]所以偶函数的导数是奇函数;设$f(x)$是奇函数,所以\[f'(x)=\lim_{h\to 0}\dfrac{f(x+h)-f(x)}{h}=\lim_{h\to 0}\dfrac{f(-x)-f(-x-h)}{-(-h)}=f'(-x)\]所以奇函数的导数是偶函数。
\end{solution}
\begin{example}{3.1-B-4}{}
    求证:周期函数的导数仍然是周期函数
\end{example}
\begin{solution}
    设$f(x)$是周期函数,$T$是周期,$f(x+T)=f(x)$,则两边求导得到$f'(x+T)=f'(x)$,所以$f'(x)$也是周期函数,也可以从定义考虑:\[f'(x)=\lim_{h\to 0}\dfrac{f(x+h)-f(x)}{h}=\lim_{h\to 0}\dfrac{f(x+T+h)-f(x+T)}{h}=f'(x+T)\]所以$f'(x)$是周期函数.
\end{solution}
\begin{example}{3.2-A-2}{}
    求导(1)$y=1-\dfrac1x+\dfrac1{x^2}$;(2)$y=\dfrac1{x^3+2x+1}$;(3)$y=\dfrac1{x+\sqrt{x}}$.
\end{example}
\begin{solution}
    (1)$y'=\dfrac1{x^2}-\dfrac2{x^3}$;(2)$y'=\dfrac{-3x^2-2}{(x^3+2x+1)^2}$;(3)$y'=-\dfrac{1+2\sqrt{x}}{2\sqrt{x}(x+\sqrt{x})^2}$
\end{solution}
\begin{example}{3.2-A-4}{}
    求导(1)$y=\ln(x+\sqrt{1+x^2})$;(2)$y=\dfrac{x}{\sqrt{a^2-x^2}}$;(3)$y=\sqrt{x+\sqrt{x+\sqrt{x}}}$.
\end{example}
\begin{solution}
    (1)$y'=\dfrac{1+\frac{2x}{2\sqrt{1+x^2}}}{x+\sqrt{1+x^2}}=\dfrac1{\sqrt{1+x^2}}$;\\
    (2)$y'=\dfrac{\sqrt{a^2-x^2}-x\frac{-2x}{2\sqrt{a^2-x^2}}}{(\sqrt{a^2-x^2})^2}=\dfrac{a^2}{(\sqrt{a^2-x^2})^3}$;\\
    (3)$y'=\dfrac{1+\frac{1+\frac1{2\sqrt{x}}}{2\sqrt{x+\sqrt{x}}}}{2\sqrt{x+\sqrt{x+\sqrt{x}}}}=\dfrac{1+2\sqrt{x}+4\sqrt{x}\sqrt{x+\sqrt{x}}}{8\sqrt{x}\sqrt{x+\sqrt{x}}\sqrt{x+\sqrt{x+\sqrt{x}}}}$.
\end{solution}
\begin{example}{3.2-B-1}{}
    设 $y=f(x)$为严格递增的可导函数,$x=\varphi(y)$是它的反函数.证明:\\
(1) 当 $h\neq 0$时 ,$ f( x+ h) - f( x) = k\neq 0$,若 记 $f( x+ h) = y+ k$, 则$\varphi ( y+ k) = x+ h.$\\
(2)当$k\to0$时,$\dfrac{\varphi(y+k)-\varphi(y)}{k}=\dfrac{h}{k}=\dfrac{h}{y+k-y}=\dfrac{h}{f(x+h)-f(x)}$趋于$\dfrac1{f^{\prime}(x)}.$
\end{example}
\begin{solution}
    (1)由$x=\varphi(f(x))=f(\varphi(x))$,以及 $y=f(x)$为严格递增的可导函数,所以$\varphi'(x)=\dfrac{1}{f'(\varphi(x))}>0$,所以$x=\varphi(y)$为严格递增的可导函数,所以\[f( x+ h) = y+ k\Rightarrow \varphi(f(x+h))=x+h=\varphi ( y+ k)\]\\
    (2)$\displaystyle\lim_{k\to 0}\dfrac{\varphi(y+k)-\varphi(y)}{k}=\lim_{h\to 0}\dfrac{1}{\dfrac{f(x+h)-f(x)}{h}}=\dfrac1{f'(x)}$,这里$h\to 0\Leftrightarrow k\to 0$.
\end{solution}
\begin{example}{3.2-B-2}{}
    $\text{设 }f(x)=x^3+2x^2+3x+1,\text{用 }\varphi\text{表示 }f\text{ 的反函数}.\text{求证}:f(1)=7,\varphi(7)=1.\text{并计算}\varphi^{\prime}(7).$
\end{example}
\begin{solution}
代入得到$f(1)=1+2+3+1=7$,所以$\varphi(7)=1$,由$f(x)=x^3+2x^2+3x+1$可得$f'(x)=3x^2+4x+3$,所以$\varphi^{\prime}(7)=\dfrac{1}{f'(\varphi(7))}=\dfrac{1}{f'(1)}=\dfrac1{10}$.
\end{solution}
\begin{example}{3.2-B-3}{}
    $\text{设 }y=(\arcsin x)^{2},\text{证明}(1-x^{2})y''-xy'=2.$
\end{example}
\begin{solution}
    两边求导数得到$y'=2\arcsin x\cdot \dfrac1{\cos\arcsin x}=2\dfrac{\arcsin x}{\sqrt{1-x^2}}$,所以$\sqrt{1-x^2}y'=2\arcsin x$,再次两边求导得到$\sqrt{1-x^2}y''+\dfrac{-2x}{2\sqrt{1-x^2}}y'=\dfrac2{\sqrt{1-x^2}}$,等价变形就有$(1-x^{2})y''-xy'=2$.
\end{solution}
\begin{example}{3.2-B-4}{}
    求下列函数的$n$阶导数$y^{(n)}$:(1) $y= \frac 1{1- x^{2}}$ ;(2) $y= \sin ^{2}x.$
\end{example}
\begin{solution}
    (1)裂项有$y=\dfrac12\left(\dfrac1{1+x}+\dfrac1{1-x}\right)$,所以$y^{(n)}=\dfrac12\left[\left(\dfrac1{1+x}\right)^{(n)}+\left(\dfrac1{1-x}\right)^{(n)}\right]$,求$n$阶导数有$y^{(n)}=\dfrac12\left[\dfrac{(-1)^nn!}{(1+x)^{n+1}}+\dfrac{n!}{(1-x)^{n+1}}\right]$\\
    (2)$y=\sin^x=\dfrac12(1-\cos 2x),y^{(n)}=\left[\dfrac12-\dfrac12\cos 2x\right]^{(n)}=-2^{n-1}\cos\left(2x+\dfrac{n\pi}2\right)$,诱导公式得到$y^{(n)}=2^{n-1}\sin\left(2x+\dfrac{(n-1)\pi}2\right)$
\end{solution}
