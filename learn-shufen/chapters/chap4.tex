\chapter{一元函数积分学}
\begin{example}{4.1-B-2}{}
利用定积分的几何意义,求下列定积分:\\
$\displaystyle(1)\int_a^bx\dd x;$\quad$(2)\displaystyle\int_a^b\sqrt{(x-a)(b-x)}\dd x;$\quad$(3)\displaystyle\int_{a}^{b}\left|x-\frac{a+b}2\right|\dd x.$
\end{example}
\begin{solution}
    (1)几何意义为梯形面积
    \[\displaystyle\int_a^bx\dd x=\dfrac{(a+b)(b-a)}2\]
    (2)由于函数$y=\sqrt{(x-a)(b-x)}$在区间$(a,b)$上图像为一条以$(a,0),(b,0)$为顶点的半圆,所以\[\displaystyle\int_a^b\sqrt{(x-a)(b-x)}\dd x=\dfrac{\pi}2\cdot\dfrac{b-a}2\cdot\dfrac{b-a}2=\dfrac{\pi(b-a)^2}8\]
    (3)函数$y=\left|x-\dfrac{a+b}2\right|$在区间$[a,b]$上图像是两个对称的等腰直角三角形,所以\[\displaystyle\int_{a}^{b}\left|x-\dfrac{a+b}2\right|\dd x=2\dfrac12\cdot\left(\dfrac{b-a}2\right)^2=\left(\dfrac{b-a}2\right)^2\]
\end{solution}
\begin{example}{4.1-B-3}{}
    设$f(x),g(x)\in C[a,b]$,求证:\\
(1) 若$f(x)\geqslant 0,\forall x\in[a,b]$,且$\int_{a}^{b}f(x)dx=0$,则在区间$[a,b]$上$f(x)\equiv 0$;\\
(2) 若$f(x)\leqslant g(x),\forall x\in[a,b]$,且$\int_{a}^{b}f(x)dx=\int_{a}^{b}g(x)dx$,则在区间$[a,b]$上$f(x)\equiv g(x)$.
\end{example}
\begin{solution}
    (1)由于$f(x)$连续非负,反设存在$x_0\in[a,b]$使得$f(x_0)>0$,则由连续性知,存在区间$[a_1,b_1]\subset[a,b]$,使得$f(x)\geq \dfrac12f(x_0)>0$,则由积分中值定理:
    \begin{align*}
    \int_{a}^{b}f(x)\dd x&\geq \int_{a}^{a_1}f(x)\dd x+\int_{a_1}^{b_1}f(x)\dd x +\int_{b_1}^bf(x)\dd x\\
    &\geq\int_{a_1}^{b_1}f(x)\dd x=\dfrac12f(x_0)(b_1-a_1)>0
    \end{align*}
    (2)设$h(x)=g(x)-f(x)$,则$h(x)\geqslant 0,\forall x\in[a,b]$,且$\displaystyle\int_{a}^{b}h(x)dx=0$,则由(1)知$h(x)\equiv 0$,即$f(x)\equiv g(x)$.
\end{solution}
\begin{example}{4.1-B-4}{}
应用柯西-许瓦兹不等式证明:$\left(\int_a^bf(x)\mathrm{d}x\right)^2\leqslant(b-a)\int_a^bf^2(x)\mathrm{d}x.$
\end{example}
\begin{solution}
根据柯西-施瓦兹不等式,对于任意在区间 $[a, b]$ 上可积的函数 $f(x)$ 和 $g(x)$,有:
\[
\left( \int_a^b f(x) g(x) \, dx \right)^2 \leq \int_a^b [f(x)]^2 \, dx \cdot \int_a^b [g(x)]^2 \, dx
\]

特别地,取 $g(x) = 1$,则有:
\[
\left( \int_a^b f(x) \cdot 1 \, dx \right)^2 \leq \int_a^b [f(x)]^2 \, dx \cdot \int_a^b 1^2 \, dx
\]

计算右侧积分:
\[
\int_a^b 1^2 \, dx = \int_a^b 1 \, dx = b - a
\]

代入上式得:
\[
\left( \int_a^b f(x) \, dx \right)^2 \leq (b - a) \int_a^b f^2(x) \, dx
\]
当且仅当 $f(x)$ 与 $g(x) = 1$ 线性相关,即 $f(x)$ 为常数函数时等号成立。
\end{solution}
\begin{example}{4.1-B-5}{}
    函数 $f(x) \in C[0,1]$, 在 $(0,1)$ 内可导, 且 $3 \int_{\frac{2}{3}}^{1} f(x) \mathrm{d}x = f(0)$. 证明: $\exists c \in (0,1)$, 使得 $f'(c) = 0$.
\end{example}
\begin{solution}
改写成$\dfrac{\int_{\frac23}^1f(x)\dd x}{1-\frac23}=f(0)$,又由于$f(x)$在区间$[0,1]$上连续,所以存在$\xi\in[0,1]$使得$f(\xi)=\dfrac{\int_{\frac23}^1f(x)\dd x}{1-\frac23}$,假如在区间$[0,1]$上有且仅有一个点$\xi=0$使得$f(\xi)=f(0)=\dfrac{\int_{\frac23}^1f(x)\dd x}{1-\frac23}$,则由于$f(x)$在区间$[0,1]$上连续,所以$f(x)$在$(0,1)$上恒大于$f(0)$或恒小于$f(0)$,不妨设$f(x)>f(0),\forall x\in(0,1)$,则$\int_{\frac23}^1f(x)\dd x>\int_{\frac23}^1f(0)\dd x=\dfrac{1}{3}f(0)$,矛盾,同理另一种情况也矛盾。所以存在$c\in(0,1)$使得$f(c)=f(0)$,由罗尔定理以及$f(x)$在$(0,1)$内可导,知存在$c\in(0,1)$使得$f'(c)=0$.
\end{solution}
\begin{example}{4.2-A-5}{}
    \[(1)\frac{\mathrm{d}}{\mathrm{d}x}\left(\int_0^{x^2}\sqrt{1+t^2}\mathrm{d}t\right);\quad(2)\frac{\mathrm{d}}{\mathrm{d}x}\left(\int_{x+a}^{x+b}(t+1)^2\mathrm{d}t\right)\]
\end{example}
\begin{solution}
    (1)设函数$\sqrt{1+t^2}$的原函数是$F(t)$,则$F'(t)=\sqrt{1+t^2}$,所以
    \begin{align*}
    \frac{\mathrm{d}}{\mathrm{d}x}\left(\int_0^{x^2}\sqrt{1+t^2}\mathrm{d}t\right)&=\frac{\mathrm{d}}{\mathrm{d}x}\left(F(x^2)-F(0)\right)=\frac{\mathrm{d}}{\mathrm{d}x}F(x^2)\\
    &=\dfrac{\dd F(x^2)}{x^2}\cdot\dfrac{\mathrm{d}x^2}{\mathrm{d}x}=\sqrt{1+(x^2)^2}\cdot 2x=2x\sqrt{1+x^4}
    \end{align*}
    (2)设函数$(t+1)^2$的原函数是$F(t)$,则$F'(t)=(t+1)^2$,所以
    \begin{align*}
    \frac{\mathrm{d}}{\mathrm{d}x}\left(\int_{x+a}^{x+b}(t+1)^2\mathrm{d}t\right)&=\frac{\mathrm{d}}{\mathrm{d}x}\left(F(x+b)-F(x+a)\right)=\frac{\mathrm{d}}{\mathrm{d}x}F(x+b)-\frac{\mathrm{d}}{\mathrm{d}x}F(x+a)\\
    &=\dfrac{\dd F(x+b)}{\dd (x+b)}\cdot\dfrac{\mathrm{d}(x+b)}{\mathrm{d}x}-\dfrac{\dd F(x+a)}{\dd (x+a)}\cdot\dfrac{\mathrm{d}(x+a)}{\mathrm{d}x}\\
    &=(x+b+1)^2-(x+a+1)^2
    \end{align*}
\end{solution}
\begin{example}{4.2-A-6}{}
    $\text{求}\frac{\mathrm{d}}{\mathrm{d}x}\int_a^b\mathrm{sin}x^2\mathrm{d}x;\quad\frac{\mathrm{d}}{\mathrm{d}a}\int_a^b\mathrm{sin}x^2\mathrm{d}x;\quad\frac{\mathrm{d}}{\mathrm{d}b}\int_a^b\mathrm{sin}x^2\mathrm{d}x.$
\end{example}
\begin{solution}
    (1)设函数$\mathrm{sin}x^2$的原函数是$F(x)$,则$F'(x)=\mathrm{sin}x^2$,所以
    \begin{align*}
    \frac{\mathrm{d}}{\mathrm{d}x}\int_a^b\mathrm{sin}x^2\mathrm{d}x&=\frac{\mathrm{d}}{\mathrm{d}x}\left(F(b)-F(a)\right)=\frac{\mathrm{d}}{\mathrm{d}x}F(b)-\frac{\mathrm{d}}{\mathrm{d}x}F(a)\\
    &=\dfrac{\dd F(b)}{\dd b}\cdot\dfrac{\mathrm{d}b}{\mathrm{d}x}-\dfrac{\dd F(a)}{\dd a}\cdot\dfrac{\mathrm{d}a}{\mathrm{d}x}=0
    \end{align*}
    (2)设函数$\mathrm{sin}x^2$的原函数是$F(x)$,则$F'(x)=\mathrm{sin}x^2$,所以
    \begin{align*}
        \frac{\mathrm{d}}{\mathrm{d}a}\int_a^b\mathrm{sin}x^2\mathrm{d}x&=\frac{\mathrm{d}}{\mathrm{d}a}\left(F(b)-F(a)\right)=\frac{\mathrm{d}}{\mathrm{d}a}F(b)-\frac{\mathrm{d}}{\mathrm{d}a}F(a)\\
    &=\dfrac{\dd F(b)}{\dd b}\cdot\dfrac{\mathrm{d}b}{\mathrm{d}a}-\dfrac{\dd F(a)}{\dd a}\cdot\dfrac{\mathrm{d}a}{\mathrm{d}a}=-\sin a^2
    \end{align*}
    (3)设函数$\mathrm{sin}x^2$的原函数是$F(x)$,则$F'(x)=\mathrm{sin}x^2$,所以
    \begin{align*}
        \frac{\mathrm{d}}{\mathrm{d}b}\int_a^b\mathrm{sin}x^2\mathrm{d}x&=\frac{\mathrm{d}}{\mathrm{d}b}\left(F(b)-F(a)\right)=\frac{\mathrm{d}}{\mathrm{d}b}F(b)-\frac{\mathrm{d}}{\mathrm{d}b}F(a)\\
    &=\dfrac{\dd F(b)}{\dd b}\cdot\dfrac{\mathrm{d}b}{\mathrm{d}b}-\dfrac{\dd F(a)}{\dd a}\cdot\dfrac{\mathrm{d}a}{\mathrm{d}b}=\sin b^2
    \end{align*}
\end{solution}
\begin{example}{4.2-A-7}{}
    \[(1)\lim_{x\to+\infty}\frac{\int_0^x(\arctan t)^2\mathrm{d}t}{\sqrt{x^2+1}};\quad(2)\lim_{x\to+\infty}\frac{\int_1^x\sqrt{t+\frac{1}{t}}\mathrm{d}t}{x\sqrt{x}}.\]
\end{example}
\begin{solution}
    (1)当$x\to+\infty$时,$\int_0^x(\arctan t)^2\mathrm{d}t$趋于无穷大,$\sqrt{x^2+1}$也趋于无穷大,两者均对$x$可导,所以可以使用洛必达法则:
    \[\lim_{x\to+\infty}\frac{\int_0^x(\arctan t)^2\mathrm{d}t}{\sqrt{x^2+1}}=\lim_{x\to+\infty}\frac{\frac{\dd}{\dd x}\int_0^x(\arctan t)^2\mathrm{d}t}{\frac{x}{\sqrt{1+x^2}}}=\lim_{x\to+\infty}\sqrt{1+\frac1{x^2}}\arctan^2x=\dfrac{\pi^2}4\]
    (2)当$x\to+\infty$时,$\int_1^x\sqrt{t+\frac{1}{t}}\mathrm{d}t$趋于无穷大,$x\sqrt{x}$也趋于无穷大,两者均对$x$可导,所以可以使用洛必达法则:
    \[\lim_{x\to+\infty}\frac{\int_1^x\sqrt{t+\frac{1}{t}}\mathrm{d}t}{x\sqrt{x}}=\lim_{x\to+\infty}\dfrac{\frac{\dd}{\dd x}\int_1^x\sqrt{t+\frac{1}{t}}\mathrm{d}t}{\frac32\sqrt{x}}=\lim_{x\to+\infty}\dfrac{2\sqrt{x+\frac1x}}{3\sqrt{x}}=\dfrac23\]
\end{solution}
\begin{example}{4.2-B-3}{}
    $\text{设 }f(x)\text{连续,}F(x)=\int_{1/x}^{\ln x}f(t)\mathrm{d}t,\text{求 }F^{\prime}(x).$
\end{example}
\begin{solution}
    由题意知$f(x)$在$(1/x,\ln x)$上连续,设$f(t)$的原函数是$G(t)$,则$G'(t)=f(t)$,所以
    \begin{align*}
        F'(x)&=\dfrac{\dd}{\dd x}(G(\ln x)-G(1/x))=\dfrac{G'(\ln x)}{x}+\dfrac{G'(1/x)}{x^2}\\
        &=\dfrac{f(\ln x)}{x}+\dfrac{f(1/x)}{x^2}
    \end{align*}
\end{solution}
\begin{example}{4.2-B-4}{}
    $\text{设 }f(x)\in C^{(1)}[0,1],\text{即 }f^{\prime}(x)\in C[0,1],\text{且 }f(1)-f(0)=1,\text{证明}\int_0^1[f^{\prime}(x)]^2\mathrm{d}x\geqslant1.$
\end{example}
\begin{solution}
    $\int_0^1f'(x)\mathrm{d}x=f(1)-f(0)=1$,所以由柯西施瓦兹不等式知:
    \[\int_0^1(f'(x))^2\dd x\int_0^1 1^2\dd x\geqslant\left(\int_0^1 f'(x)\dd x\right)^2=1\]
    又因为$\int_0^1 1^2\dd x=1$,所以$\int_0^1[f^{\prime}(x)]^2\mathrm{d}x\geqslant1$. 
\end{solution}
\begin{example}{4.2-B-5}{}
    设 $f(x)\in\mathbb{C}[0,+\infty)$,并且 $x\in[0,+\infty)$时$,f(x)>0.$证明函数$F(x)=\frac{\int_{0}^{x}tf(t)\mathrm{d}t}{\int_{0}^{x}f(t)\mathrm{d}t}$在$(0,+\infty)$内为单调增加的函数。
\end{example}
\begin{solution}
    只需证明$F'(x)\geq 0$,由商法则,有:
\begin{align*}
    F'(x) &= \frac{ \left( \frac{\mathrm{d}}{\mathrm{d}x} \int_{0}^{x} t f(t) \, \mathrm{d}t \right) \cdot \int_{0}^{x} f(t) \, \mathrm{d}t - \int_{0}^{x} t f(t) \, \mathrm{d}t \cdot \left( \frac{\mathrm{d}}{\mathrm{d}x} \int_{0}^{x} f(t) \, \mathrm{d}t \right) }{ \left( \int_{0}^{x} f(t) \, \mathrm{d}t \right)^2 } \\
    &= \frac{ x f(x) \cdot \int_{0}^{x} f(t) \, \mathrm{d}t - \int_{0}^{x} t f(t) \, \mathrm{d}t \cdot f(x) }{ \left( \int_{0}^{x} f(t) \, \mathrm{d}t \right)^2 } \\
    &= \frac{ f(x) \left[ x \int_{0}^{x} f(t) \, \mathrm{d}t - \int_{0}^{x} t f(t) \, \mathrm{d}t \right] }{ \left( \int_{0}^{x} f(t) \, \mathrm{d}t \right)^2 }.
\end{align*}
由于 $f(x) > 0$ 且 $\int_{0}^{x} f(t) \, \mathrm{d}t > 0$(因为 $f(t) > 0$),分母恒正,故 $F'(x)$ 的符号取决于分子中的表达式:
\[
x \int_{0}^{x} f(t) \, \mathrm{d}t - \int_{0}^{x} t f(t) \, \mathrm{d}t = \int_{0}^{x} f(t) (x - t) \, \mathrm{d}t.
\]
对于 $x > 0$ 和 $t \in [0, x]$,有 $x - t \geq 0$ 且 $f(t) > 0$,因此被积函数 $f(t)(x - t) \geq 0$。当 $t < x$ 时,$x - t > 0$,故:
\[
\int_{0}^{x} f(t) (x - t) \, \mathrm{d}t \geq 0.
\]
因此,$F'(x) \geq 0$ 对于 $x > 0$,这意味着 $F(x)$ 在 $(0, +\infty)$ 内单调增加。
\end{solution}
\begin{example}{4.3-B-1}{}
\begin{tabular}{@{}l l l@{}}
    (1) $\displaystyle\int\left(e^x-\frac{2}{\sqrt[3]{x}}\right)\mathrm{d}x$ & 
    (2) $\displaystyle\int\frac{1+x+x^{2}}{x(1+x^{2})}\mathrm{d}x$ & 
    (3) $\displaystyle\int\frac{x^4}{1+x^2}\mathrm{d}x$ \\
    
    (4) $\displaystyle\int\sqrt{x\sqrt{x}}\mathrm{d}x$ & 
    (5) $\displaystyle\int\frac{\mathrm{d}x}{x^2\sqrt{x}}$ & 
    (6) $\displaystyle\int\frac{2x^{2}}{\sqrt{x}}\mathrm{d}x$ \\
    
    (7) $\displaystyle\int(x^2-1)^2\mathrm{d}x$ & 
    (8) $\displaystyle\int\frac{x+1}{\sqrt{x}}\mathrm{d}x$ & 
    (9) $\displaystyle\int\frac{\mathrm{e}^{3x}+1}{\mathrm{e}^{x}+1}\mathrm{d}x$ \\
    
    (10) $\displaystyle\int\left(2^x+3^x\right)\mathrm{d}x$ & 
    (11) $\displaystyle\int\frac{3x^2}{1+x^2}\mathrm{d}x$ & 
    (12) $\displaystyle\int\frac{\mathrm{d}x}{x^4(1+x^2)}$
\end{tabular}
\end{example}
\begin{solution}
(1)$\displaystyle\int\left(e^x-\frac{2}{\sqrt[3]{x}}\right)\mathrm{d}x=\int\left(e^x-2x^{-\frac{1}{3}}\right)\mathrm{d}x=e^x-3x^{\frac23}+C$\\
(2)$\displaystyle\int\frac{1+x+x^{2}}{x(1+x^{2})}\mathrm{d}x=\int\left(\frac{1}{x}+\frac{1}{1+x^{2}}\right)\mathrm{d}x=\ln|x|+\arctan x+C$\\
(3)$\displaystyle\int\frac{x^4-1+1}{1+x^2}\mathrm{d}x=\int(x^2-1)\mathrm{d}x+\int\frac{1}{1+x^2}\mathrm{d}x=\frac{x^3}{3}-x+\arctan x+C$\\
(4)$\displaystyle\int\sqrt{x\sqrt{x}}\mathrm{d}x=\int x^{\frac34}\mathrm{d}x=\frac{4}{7}x^{\frac74}+C$\\
(5)$\displaystyle\int\frac{\mathrm{d}x}{x^2\sqrt{x}}=\int x^{-\frac52}\mathrm{d}x=-\frac{2}{3}x^{-\frac32}+C$\\
(6)$\displaystyle\int\frac{2x^{2}}{\sqrt{x}}\mathrm{d}x=\int2x^{\frac32}\mathrm{d}x=\frac{4}{5}x^{\frac52}+C$\\
(7)$\displaystyle\int(x^2-1)^2\mathrm{d}x=\int(x^4-2x^2+1)\mathrm{d}x=\frac{x^5}{5}-\frac{2x^3}{3}+x+C$\\
(8)$\displaystyle\int\frac{x+1}{\sqrt{x}}\mathrm{d}x=\int(x^{\frac12}+x^{-\frac12})\mathrm{d}x=\frac{2}{3}x^{\frac32}+2x^{\frac12}+C$\\
(9)$\displaystyle\int\frac{e^{3x}+1}{e^{x}+1}\mathrm{d}x=\int(e^{2x}-e^{x}+1)\mathrm{d}x=\frac{1}{2}e^{2x}-e^{x}+x+C$\\
(10)$\displaystyle\int(2^x+3^x)\mathrm{d}x=\int2^x\mathrm{d}x+\int3^x\mathrm{d}x=\frac{2^x}{\ln2}+\frac{3^x}{\ln3}+C$\\
(11)$\displaystyle\int\frac{3x^2}{1+x^2}\mathrm{d}x=\int\left(3-\frac{3}{1+x^2}\right)\mathrm{d}x=3x-3\arctan x+C$\\
(12)$\displaystyle\int\frac{\mathrm{d}x}{x^4(1+x^2)}=\int\left(\frac{1}{x^4}-\frac{1}{x^2}+\frac{1}{1+x^2}\right)\mathrm{d}x=-\frac{1}{3x^3}+\frac{1}{x}+\arctan x+C$
\end{solution}
\begin{example}{4.3-B-2}{}
\begin{tabular}{@{}l l l@{}}
    (1) $\displaystyle\int\cos(t+1)\,\mathrm{d}t$; & (2) $\displaystyle\int\left(2\sin\theta-3\cos\theta\right)\,\mathrm{d}\theta$; & (3) $\displaystyle\int\frac{\cos 2x}{\sin^{2}x \cos^{2}x}\,\mathrm{d}x$; \\
    (4) $\displaystyle\int\frac{\mathrm{d}t}{\sin^2\frac{t}{2}\cos^2\frac{t}{2}}$; & (5) $\displaystyle\int\sqrt{1-\sin 2\theta}\,\mathrm{d}\theta$; & (6) $\displaystyle\int\frac{3+\sin^2 x}{\cos^2 x}\,\mathrm{d}x$; \\
    (7) $\displaystyle\int\cos^2\frac{t}{2}\,\mathrm{d}t$; & (8) $\displaystyle\int\frac{\mathrm{d}x}{1+\cos 2x}$; & (9) $\displaystyle\int\sec x(\sec x-\tan x)\,\mathrm{d}x$; \\
    (10) $\displaystyle\int\frac{\cos 2x}{\cos x-\sin x}\,\mathrm{d}x$; & (11) $\displaystyle\int 3^x e^x\,\mathrm{d}x$; & (12) $\displaystyle\int\frac{2\cdot 3^x-5\cdot 2^x}{3^x}\,\mathrm{d}x$.
\end{tabular}
\end{example}
\begin{solution}
(1)$\displaystyle\int\cos(t+1)\,\mathrm{d}t=\sin(t+1)+C$\\
(2)$\displaystyle\int\left(2\sin\theta-3\cos\theta\right)\,\mathrm{d}\theta=-2\cos\theta-3\sin\theta+C$\\
(3)$\displaystyle\int\frac{\cos 2x}{\sin^{2}x \cos^{2}x}\,\mathrm{d}x=\int\frac{\cos^2 x-\sin^2 x}{\sin^{2}x \cos^{2}x}\,\mathrm{d}x=\int\left(\frac{1}{\sin^2 x}-\frac{1}{\cos^2 x}\right)\,\mathrm{d}x=-\cot x-\tan x+C$\\
(4)$\displaystyle\int\frac{\mathrm{d}t}{\sin^2\frac{t}{2}\cos^2\frac{t}{2}}=\int\frac{4}{\sin^2 t}\,\mathrm{d}t=-4\cot t+C$\\
(5)$\displaystyle\int\sqrt{1-\sin 2\theta}\,\mathrm{d}\theta=\int|\cos\theta-\sin\theta|\,\mathrm{d}\theta=\begin{cases}\sin\theta+\cos\theta+C, & \cos\theta\geq\sin\theta \\-(\sin\theta+\cos\theta)+C, & \cos\theta<\sin\theta\end{cases}$\\
(6)$\displaystyle\int\frac{3+\sin^2 x}{\cos^2 x}\,\mathrm{d}x=\int(3\sec^2 x+\tan^2 x)\,\mathrm{d}x=\int(4\sec^2 x-1)\,\mathrm{d}x=4\tan x-x+C$\\
(7)$\displaystyle\int\cos^2\frac{t}{2}\,\mathrm{d}t=\int\frac{1+\cos t}{2}\,\mathrm{d}t=\frac{t}{2}+\frac{\sin t}{2}+C$\\
(8)$\displaystyle\int\frac{\mathrm{d}x}{1+\cos 2x}=\int\frac{1}{2\cos^2 x}\,\mathrm{d}x=\frac{1}{2}\tan x+C$\\
(9)$\displaystyle\int\sec x(\sec x-\tan x)\,\mathrm{d}x=\int(\sec^2 x-\sec x\tan x)\,\mathrm{d}x=\tan x-\sec x+C$\\
(10)$\displaystyle\int\frac{\cos 2x}{\cos x-\sin x}\,\mathrm{d}x=\int\frac{\cos^2 x-\sin^2 x}{\cos x-\sin x}\,\mathrm{d}x=\int(\cos x+\sin x)\,\mathrm{d}x=\sin x-\cos x+C$\\
(11)$\displaystyle\int 3^x e^x\,\mathrm{d}x=\int (3e)^x\,\mathrm{d}x=\frac{(3e)^x}{\ln(3e)}+C=\frac{3^xe^x}{\ln 3+1}+C$\\
(12)$\displaystyle\int\frac{2\cdot 3^x-5\cdot 2^x}{3^x}\,\mathrm{d}x=\int\left(2-5\left(\frac{2}{3}\right)^x\right)\,\mathrm{d}x=2x-\frac{5}{\ln\frac{2}{3}}\left(\frac{2}{3}\right)^x+C$
\end{solution}