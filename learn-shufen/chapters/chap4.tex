\chapter{一元函数积分学}
\begin{example}{4.1-B-2}{}
利用定积分的几何意义,求下列定积分:\\
$\displaystyle(1)\int_a^bx\dd x;$\quad$(2)\displaystyle\int_a^b\sqrt{(x-a)(b-x)}\dd x;$\quad$(3)\displaystyle\int_{a}^{b}\left|x-\frac{a+b}2\right|\dd x.$
\end{example}
\begin{solution}
    (1)几何意义为梯形面积
    \[\displaystyle\int_a^bx\dd x=\dfrac{(a+b)(b-a)}2\]
    (2)由于函数$y=\sqrt{(x-a)(b-x)}$在区间$(a,b)$上图像为一条以$(a,0),(b,0)$为顶点的半圆,所以\[\displaystyle\int_a^b\sqrt{(x-a)(b-x)}\dd x=\dfrac{\pi}2\cdot\dfrac{b-a}2\cdot\dfrac{b-a}2=\dfrac{\pi(b-a)^2}8\]
    (3)函数$y=\left|x-\dfrac{a+b}2\right|$在区间$[a,b]$上图像是两个对称的等腰直角三角形,所以\[\displaystyle\int_{a}^{b}\left|x-\dfrac{a+b}2\right|\dd x=2\dfrac12\cdot\left(\dfrac{b-a}2\right)^2=\left(\dfrac{b-a}2\right)^2\]
\end{solution}
\begin{example}{4.1-B-3}{}
    设$f(x),g(x)\in C[a,b]$,求证:\\
(1) 若$f(x)\geqslant 0,\forall x\in[a,b]$,且$\int_{a}^{b}f(x)dx=0$,则在区间$[a,b]$上$f(x)\equiv 0$;\\
(2) 若$f(x)\leqslant g(x),\forall x\in[a,b]$,且$\int_{a}^{b}f(x)dx=\int_{a}^{b}g(x)dx$,则在区间$[a,b]$上$f(x)\equiv g(x)$.
\end{example}
\begin{solution}
    (1)由于$f(x)$连续非负,反设存在$x_0\in[a,b]$使得$f(x_0)>0$,则由连续性知,存在区间$[a_1,b_1]\subset[a,b]$,使得$f(x)\geq \dfrac12f(x_0)>0$,则由积分中值定理:
    \begin{align*}
    \int_{a}^{b}f(x)\dd x&\geq \int_{a}^{a_1}f(x)\dd x+\int_{a_1}^{b_1}f(x)\dd x +\int_{b_1}^bf(x)\dd x\\
    &\geq\int_{a_1}^{b_1}f(x)\dd x=\dfrac12f(x_0)(b_1-a_1)>0
    \end{align*}
    (2)设$h(x)=g(x)-f(x)$,则$h(x)\geqslant 0,\forall x\in[a,b]$,且$\displaystyle\int_{a}^{b}h(x)dx=0$,则由(1)知$h(x)\equiv 0$,即$f(x)\equiv g(x)$.
\end{solution}
\begin{example}{4.1-B-4}{}
应用柯西-许瓦兹不等式证明:$\left(\int_a^bf(x)\mathrm{d}x\right)^2\leqslant(b-a)\int_a^bf^2(x)\mathrm{d}x.$
\end{example}
\begin{solution}
根据柯西-施瓦兹不等式,对于任意在区间 $[a, b]$ 上可积的函数 $f(x)$ 和 $g(x)$,有:
\[
\left( \int_a^b f(x) g(x) \, dx \right)^2 \leq \int_a^b [f(x)]^2 \, dx \cdot \int_a^b [g(x)]^2 \, dx
\]

特别地,取 $g(x) = 1$,则有:
\[
\left( \int_a^b f(x) \cdot 1 \, dx \right)^2 \leq \int_a^b [f(x)]^2 \, dx \cdot \int_a^b 1^2 \, dx
\]

计算右侧积分:
\[
\int_a^b 1^2 \, dx = \int_a^b 1 \, dx = b - a
\]

代入上式得:
\[
\left( \int_a^b f(x) \, dx \right)^2 \leq (b - a) \int_a^b f^2(x) \, dx
\]
当且仅当 $f(x)$ 与 $g(x) = 1$ 线性相关,即 $f(x)$ 为常数函数时等号成立。
\end{solution}
\begin{example}{4.1-B-5}{}
    函数 $f(x) \in C[0,1]$, 在 $(0,1)$ 内可导, 且 $3 \int_{\frac{2}{3}}^{1} f(x) \mathrm{d}x = f(0)$. 证明: $\exists c \in (0,1)$, 使得 $f'(c) = 0$.
\end{example}
\begin{solution}
改写成$\dfrac{\int_{\frac23}^1f(x)\dd x}{1-\frac23}=f(0)$,又由于$f(x)$在区间$[0,1]$上连续,所以存在$\xi\in[0,1]$使得$f(\xi)=\dfrac{\int_{\frac23}^1f(x)\dd x}{1-\frac23}$,假如在区间$[0,1]$上有且仅有一个点$\xi=0$使得$f(\xi)=f(0)=\dfrac{\int_{\frac23}^1f(x)\dd x}{1-\frac23}$,则由于$f(x)$在区间$[0,1]$上连续,所以$f(x)$在$(0,1)$上恒大于$f(0)$或恒小于$f(0)$,不妨设$f(x)>f(0),\forall x\in(0,1)$,则$\int_{\frac23}^1f(x)\dd x>\int_{\frac23}^1f(0)\dd x=\dfrac{1}{3}f(0)$,矛盾,同理另一种情况也矛盾。所以存在$c\in(0,1)$使得$f(c)=f(0)$,由罗尔定理以及$f(x)$在$(0,1)$内可导,知存在$c\in(0,1)$使得$f'(c)=0$.
\end{solution}
\begin{example}{4.2-A-5}{}
    \[(1)\frac{\mathrm{d}}{\mathrm{d}x}\left(\int_0^{x^2}\sqrt{1+t^2}\mathrm{d}t\right);\quad(2)\frac{\mathrm{d}}{\mathrm{d}x}\left(\int_{x+a}^{x+b}(t+1)^2\mathrm{d}t\right)\]
\end{example}
\begin{solution}
    (1)设函数$\sqrt{1+t^2}$的原函数是$F(t)$,则$F'(t)=\sqrt{1+t^2}$,所以
    \begin{align*}
    \frac{\mathrm{d}}{\mathrm{d}x}\left(\int_0^{x^2}\sqrt{1+t^2}\mathrm{d}t\right)&=\frac{\mathrm{d}}{\mathrm{d}x}\left(F(x^2)-F(0)\right)=\frac{\mathrm{d}}{\mathrm{d}x}F(x^2)\\
    &=\dfrac{\dd F(x^2)}{x^2}\cdot\dfrac{\mathrm{d}x^2}{\mathrm{d}x}=\sqrt{1+(x^2)^2}\cdot 2x=2x\sqrt{1+x^4}
    \end{align*}
    (2)设函数$(t+1)^2$的原函数是$F(t)$,则$F'(t)=(t+1)^2$,所以
    \begin{align*}
    \frac{\mathrm{d}}{\mathrm{d}x}\left(\int_{x+a}^{x+b}(t+1)^2\mathrm{d}t\right)&=\frac{\mathrm{d}}{\mathrm{d}x}\left(F(x+b)-F(x+a)\right)=\frac{\mathrm{d}}{\mathrm{d}x}F(x+b)-\frac{\mathrm{d}}{\mathrm{d}x}F(x+a)\\
    &=\dfrac{\dd F(x+b)}{\dd (x+b)}\cdot\dfrac{\mathrm{d}(x+b)}{\mathrm{d}x}-\dfrac{\dd F(x+a)}{\dd (x+a)}\cdot\dfrac{\mathrm{d}(x+a)}{\mathrm{d}x}\\
    &=(x+b+1)^2-(x+a+1)^2
    \end{align*}
\end{solution}
\begin{example}{4.2-A-6}{}
    $\text{求}\frac{\mathrm{d}}{\mathrm{d}x}\int_a^b\mathrm{sin}x^2\mathrm{d}x;\quad\frac{\mathrm{d}}{\mathrm{d}a}\int_a^b\mathrm{sin}x^2\mathrm{d}x;\quad\frac{\mathrm{d}}{\mathrm{d}b}\int_a^b\mathrm{sin}x^2\mathrm{d}x.$
\end{example}
\begin{solution}
    (1)设函数$\mathrm{sin}x^2$的原函数是$F(x)$,则$F'(x)=\mathrm{sin}x^2$,所以
    \begin{align*}
    \frac{\mathrm{d}}{\mathrm{d}x}\int_a^b\mathrm{sin}x^2\mathrm{d}x&=\frac{\mathrm{d}}{\mathrm{d}x}\left(F(b)-F(a)\right)=\frac{\mathrm{d}}{\mathrm{d}x}F(b)-\frac{\mathrm{d}}{\mathrm{d}x}F(a)\\
    &=\dfrac{\dd F(b)}{\dd b}\cdot\dfrac{\mathrm{d}b}{\mathrm{d}x}-\dfrac{\dd F(a)}{\dd a}\cdot\dfrac{\mathrm{d}a}{\mathrm{d}x}=0
    \end{align*}
    (2)设函数$\mathrm{sin}x^2$的原函数是$F(x)$,则$F'(x)=\mathrm{sin}x^2$,所以
    \begin{align*}
        \frac{\mathrm{d}}{\mathrm{d}a}\int_a^b\mathrm{sin}x^2\mathrm{d}x&=\frac{\mathrm{d}}{\mathrm{d}a}\left(F(b)-F(a)\right)=\frac{\mathrm{d}}{\mathrm{d}a}F(b)-\frac{\mathrm{d}}{\mathrm{d}a}F(a)\\
    &=\dfrac{\dd F(b)}{\dd b}\cdot\dfrac{\mathrm{d}b}{\mathrm{d}a}-\dfrac{\dd F(a)}{\dd a}\cdot\dfrac{\mathrm{d}a}{\mathrm{d}a}=-\sin a^2
    \end{align*}
    (3)设函数$\mathrm{sin}x^2$的原函数是$F(x)$,则$F'(x)=\mathrm{sin}x^2$,所以
    \begin{align*}
        \frac{\mathrm{d}}{\mathrm{d}b}\int_a^b\mathrm{sin}x^2\mathrm{d}x&=\frac{\mathrm{d}}{\mathrm{d}b}\left(F(b)-F(a)\right)=\frac{\mathrm{d}}{\mathrm{d}b}F(b)-\frac{\mathrm{d}}{\mathrm{d}b}F(a)\\
    &=\dfrac{\dd F(b)}{\dd b}\cdot\dfrac{\mathrm{d}b}{\mathrm{d}b}-\dfrac{\dd F(a)}{\dd a}\cdot\dfrac{\mathrm{d}a}{\mathrm{d}b}=\sin b^2
    \end{align*}
\end{solution}
\begin{example}{4.2-A-7}{}
    \[(1)\lim_{x\to+\infty}\frac{\int_0^x(\arctan t)^2\mathrm{d}t}{\sqrt{x^2+1}};\quad(2)\lim_{x\to+\infty}\frac{\int_1^x\sqrt{t+\frac{1}{t}}\mathrm{d}t}{x\sqrt{x}}.\]
\end{example}
\begin{solution}
    (1)当$x\to+\infty$时,$\int_0^x(\arctan t)^2\mathrm{d}t$趋于无穷大,$\sqrt{x^2+1}$也趋于无穷大,两者均对$x$可导,所以可以使用洛必达法则:
    \[\lim_{x\to+\infty}\frac{\int_0^x(\arctan t)^2\mathrm{d}t}{\sqrt{x^2+1}}=\lim_{x\to+\infty}\frac{\frac{\dd}{\dd x}\int_0^x(\arctan t)^2\mathrm{d}t}{\frac{x}{\sqrt{1+x^2}}}=\lim_{x\to+\infty}\sqrt{1+\frac1{x^2}}\arctan^2x=\dfrac{\pi^2}4\]
    (2)当$x\to+\infty$时,$\int_1^x\sqrt{t+\frac{1}{t}}\mathrm{d}t$趋于无穷大,$x\sqrt{x}$也趋于无穷大,两者均对$x$可导,所以可以使用洛必达法则:
    \[\lim_{x\to+\infty}\frac{\int_1^x\sqrt{t+\frac{1}{t}}\mathrm{d}t}{x\sqrt{x}}=\lim_{x\to+\infty}\dfrac{\frac{\dd}{\dd x}\int_1^x\sqrt{t+\frac{1}{t}}\mathrm{d}t}{\frac32\sqrt{x}}=\lim_{x\to+\infty}\dfrac{2\sqrt{x+\frac1x}}{3\sqrt{x}}=\dfrac23\]
\end{solution}
\begin{example}{4.2-B-3}{}
    $\text{设 }f(x)\text{连续,}F(x)=\int_{1/x}^{\ln x}f(t)\mathrm{d}t,\text{求 }F^{\prime}(x).$
\end{example}
\begin{solution}
    由题意知$f(x)$在$(1/x,\ln x)$上连续,设$f(t)$的原函数是$G(t)$,则$G'(t)=f(t)$,所以
    \begin{align*}
        F'(x)&=\dfrac{\dd}{\dd x}(G(\ln x)-G(1/x))=\dfrac{G'(\ln x)}{x}+\dfrac{G'(1/x)}{x^2}\\
        &=\dfrac{f(\ln x)}{x}+\dfrac{f(1/x)}{x^2}
    \end{align*}
\end{solution}
\begin{example}{4.2-B-4}{}
    $\text{设 }f(x)\in C^{(1)}[0,1],\text{即 }f^{\prime}(x)\in C[0,1],\text{且 }f(1)-f(0)=1,\text{证明}\int_0^1[f^{\prime}(x)]^2\mathrm{d}x\geqslant1.$
\end{example}
\begin{solution}
    $\int_0^1f'(x)\mathrm{d}x=f(1)-f(0)=1$,所以由柯西施瓦兹不等式知:
    \[\int_0^1(f'(x))^2\dd x\int_0^1 1^2\dd x\geqslant\left(\int_0^1 f'(x)\dd x\right)^2=1\]
    又因为$\int_0^1 1^2\dd x=1$,所以$\int_0^1[f^{\prime}(x)]^2\mathrm{d}x\geqslant1$. 
\end{solution}
\begin{example}{4.2-B-5}{}
    设 $f(x)\in\mathbb{C}[0,+\infty)$,并且 $x\in[0,+\infty)$时$,f(x)>0.$证明函数$F(x)=\frac{\int_{0}^{x}tf(t)\mathrm{d}t}{\int_{0}^{x}f(t)\mathrm{d}t}$在$(0,+\infty)$内为单调增加的函数。
\end{example}
\begin{solution}
    只需证明$F'(x)\geq 0$,由商法则,有:
\begin{align*}
    F'(x) &= \frac{ \left( \frac{\mathrm{d}}{\mathrm{d}x} \int_{0}^{x} t f(t) \, \mathrm{d}t \right) \cdot \int_{0}^{x} f(t) \, \mathrm{d}t - \int_{0}^{x} t f(t) \, \mathrm{d}t \cdot \left( \frac{\mathrm{d}}{\mathrm{d}x} \int_{0}^{x} f(t) \, \mathrm{d}t \right) }{ \left( \int_{0}^{x} f(t) \, \mathrm{d}t \right)^2 } \\
    &= \frac{ x f(x) \cdot \int_{0}^{x} f(t) \, \mathrm{d}t - \int_{0}^{x} t f(t) \, \mathrm{d}t \cdot f(x) }{ \left( \int_{0}^{x} f(t) \, \mathrm{d}t \right)^2 } \\
    &= \frac{ f(x) \left[ x \int_{0}^{x} f(t) \, \mathrm{d}t - \int_{0}^{x} t f(t) \, \mathrm{d}t \right] }{ \left( \int_{0}^{x} f(t) \, \mathrm{d}t \right)^2 }.
\end{align*}
由于 $f(x) > 0$ 且 $\int_{0}^{x} f(t) \, \mathrm{d}t > 0$(因为 $f(t) > 0$),分母恒正,故 $F'(x)$ 的符号取决于分子中的表达式:
\[
x \int_{0}^{x} f(t) \, \mathrm{d}t - \int_{0}^{x} t f(t) \, \mathrm{d}t = \int_{0}^{x} f(t) (x - t) \, \mathrm{d}t.
\]
对于 $x > 0$ 和 $t \in [0, x]$,有 $x - t \geq 0$ 且 $f(t) > 0$,因此被积函数 $f(t)(x - t) \geq 0$。当 $t < x$ 时,$x - t > 0$,故:
\[
\int_{0}^{x} f(t) (x - t) \, \mathrm{d}t \geq 0.
\]
因此,$F'(x) \geq 0$ 对于 $x > 0$,这意味着 $F(x)$ 在 $(0, +\infty)$ 内单调增加。
\end{solution}
\begin{example}{4.3-B-1}{}
\begin{tabular}{@{}l l l@{}}
    (1) $\displaystyle\int\left(e^x-\frac{2}{\sqrt[3]{x}}\right)\mathrm{d}x$ & 
    (2) $\displaystyle\int\frac{1+x+x^{2}}{x(1+x^{2})}\mathrm{d}x$ & 
    (3) $\displaystyle\int\frac{x^4}{1+x^2}\mathrm{d}x$ \\
    
    (4) $\displaystyle\int\sqrt{x\sqrt{x}}\mathrm{d}x$ & 
    (5) $\displaystyle\int\frac{\mathrm{d}x}{x^2\sqrt{x}}$ & 
    (6) $\displaystyle\int\frac{2x^{2}}{\sqrt{x}}\mathrm{d}x$ \\
    
    (7) $\displaystyle\int(x^2-1)^2\mathrm{d}x$ & 
    (8) $\displaystyle\int\frac{x+1}{\sqrt{x}}\mathrm{d}x$ & 
    (9) $\displaystyle\int\frac{\mathrm{e}^{3x}+1}{\mathrm{e}^{x}+1}\mathrm{d}x$ \\
    
    (10) $\displaystyle\int\left(2^x+3^x\right)\mathrm{d}x$ & 
    (11) $\displaystyle\int\frac{3x^2}{1+x^2}\mathrm{d}x$ & 
    (12) $\displaystyle\int\frac{\mathrm{d}x}{x^4(1+x^2)}$
\end{tabular}
\end{example}
\begin{solution}
(1)$\displaystyle\int\left(e^x-\frac{2}{\sqrt[3]{x}}\right)\mathrm{d}x=\int\left(e^x-2x^{-\frac{1}{3}}\right)\mathrm{d}x=e^x-3x^{\frac23}+C$\\
(2)$\displaystyle\int\frac{1+x+x^{2}}{x(1+x^{2})}\mathrm{d}x=\int\left(\frac{1}{x}+\frac{1}{1+x^{2}}\right)\mathrm{d}x=\ln|x|+\arctan x+C$\\
(3)$\displaystyle\int\frac{x^4-1+1}{1+x^2}\mathrm{d}x=\int(x^2-1)\mathrm{d}x+\int\frac{1}{1+x^2}\mathrm{d}x=\frac{x^3}{3}-x+\arctan x+C$\\
(4)$\displaystyle\int\sqrt{x\sqrt{x}}\mathrm{d}x=\int x^{\frac34}\mathrm{d}x=\frac{4}{7}x^{\frac74}+C$\\
(5)$\displaystyle\int\frac{\mathrm{d}x}{x^2\sqrt{x}}=\int x^{-\frac52}\mathrm{d}x=-\frac{2}{3}x^{-\frac32}+C$\\
(6)$\displaystyle\int\frac{2x^{2}}{\sqrt{x}}\mathrm{d}x=\int2x^{\frac32}\mathrm{d}x=\frac{4}{5}x^{\frac52}+C$\\
(7)$\displaystyle\int(x^2-1)^2\mathrm{d}x=\int(x^4-2x^2+1)\mathrm{d}x=\frac{x^5}{5}-\frac{2x^3}{3}+x+C$\\
(8)$\displaystyle\int\frac{x+1}{\sqrt{x}}\mathrm{d}x=\int(x^{\frac12}+x^{-\frac12})\mathrm{d}x=\frac{2}{3}x^{\frac32}+2x^{\frac12}+C$\\
(9)$\displaystyle\int\frac{e^{3x}+1}{e^{x}+1}\mathrm{d}x=\int(e^{2x}-e^{x}+1)\mathrm{d}x=\frac{1}{2}e^{2x}-e^{x}+x+C$\\
(10)$\displaystyle\int(2^x+3^x)\mathrm{d}x=\int2^x\mathrm{d}x+\int3^x\mathrm{d}x=\frac{2^x}{\ln2}+\frac{3^x}{\ln3}+C$\\
(11)$\displaystyle\int\frac{3x^2}{1+x^2}\mathrm{d}x=\int\left(3-\frac{3}{1+x^2}\right)\mathrm{d}x=3x-3\arctan x+C$\\
(12)$\displaystyle\int\frac{\mathrm{d}x}{x^4(1+x^2)}=\int\left(\frac{1}{x^4}-\frac{1}{x^2}+\frac{1}{1+x^2}\right)\mathrm{d}x=-\frac{1}{3x^3}+\frac{1}{x}+\arctan x+C$
\end{solution}
\begin{example}{4.3-B-2}{}
\begin{tabular}{@{}l l l@{}}
    (1) $\displaystyle\int\cos(t+1)\,\mathrm{d}t$; & (2) $\displaystyle\int\left(2\sin\theta-3\cos\theta\right)\,\mathrm{d}\theta$; & (3) $\displaystyle\int\frac{\cos 2x}{\sin^{2}x \cos^{2}x}\,\mathrm{d}x$; \\
    (4) $\displaystyle\int\frac{\mathrm{d}t}{\sin^2\frac{t}{2}\cos^2\frac{t}{2}}$; & (5) $\displaystyle\int\sqrt{1-\sin 2\theta}\,\mathrm{d}\theta$; & (6) $\displaystyle\int\frac{3+\sin^2 x}{\cos^2 x}\,\mathrm{d}x$; \\
    (7) $\displaystyle\int\cos^2\frac{t}{2}\,\mathrm{d}t$; & (8) $\displaystyle\int\frac{\mathrm{d}x}{1+\cos 2x}$; & (9) $\displaystyle\int\sec x(\sec x-\tan x)\,\mathrm{d}x$; \\
    (10) $\displaystyle\int\frac{\cos 2x}{\cos x-\sin x}\,\mathrm{d}x$; & (11) $\displaystyle\int 3^x e^x\,\mathrm{d}x$; & (12) $\displaystyle\int\frac{2\cdot 3^x-5\cdot 2^x}{3^x}\,\mathrm{d}x$.
\end{tabular}
\end{example}
\begin{solution}
(1)$\displaystyle\int\cos(t+1)\,\mathrm{d}t=\sin(t+1)+C$\\
(2)$\displaystyle\int\left(2\sin\theta-3\cos\theta\right)\,\mathrm{d}\theta=-2\cos\theta-3\sin\theta+C$\\
(3)$\displaystyle\int\frac{\cos 2x}{\sin^{2}x \cos^{2}x}\,\mathrm{d}x=\int\frac{\cos^2 x-\sin^2 x}{\sin^{2}x \cos^{2}x}\,\mathrm{d}x=\int\left(\frac{1}{\sin^2 x}-\frac{1}{\cos^2 x}\right)\,\mathrm{d}x=-\cot x-\tan x+C$\\
(4)$\displaystyle\int\frac{\mathrm{d}t}{\sin^2\frac{t}{2}\cos^2\frac{t}{2}}=\int\frac{4}{\sin^2 t}\,\mathrm{d}t=-4\cot t+C$\\
(5)$\displaystyle\int\sqrt{1-\sin 2\theta}\,\mathrm{d}\theta=\int|\cos\theta-\sin\theta|\,\mathrm{d}\theta=\begin{cases}\sin\theta+\cos\theta+C, & \cos\theta\geq\sin\theta \\-(\sin\theta+\cos\theta)+C, & \cos\theta<\sin\theta\end{cases}$\\
(6)$\displaystyle\int\frac{3+\sin^2 x}{\cos^2 x}\,\mathrm{d}x=\int(3\sec^2 x+\tan^2 x)\,\mathrm{d}x=\int(4\sec^2 x-1)\,\mathrm{d}x=4\tan x-x+C$\\
(7)$\displaystyle\int\cos^2\frac{t}{2}\,\mathrm{d}t=\int\frac{1+\cos t}{2}\,\mathrm{d}t=\frac{t}{2}+\frac{\sin t}{2}+C$\\
(8)$\displaystyle\int\frac{\mathrm{d}x}{1+\cos 2x}=\int\frac{1}{2\cos^2 x}\,\mathrm{d}x=\frac{1}{2}\tan x+C$\\
(9)$\displaystyle\int\sec x(\sec x-\tan x)\,\mathrm{d}x=\int(\sec^2 x-\sec x\tan x)\,\mathrm{d}x=\tan x-\sec x+C$\\
(10)$\displaystyle\int\frac{\cos 2x}{\cos x-\sin x}\,\mathrm{d}x=\int\frac{\cos^2 x-\sin^2 x}{\cos x-\sin x}\,\mathrm{d}x=\int(\cos x+\sin x)\,\mathrm{d}x=\sin x-\cos x+C$\\
(11)$\displaystyle\int 3^x e^x\,\mathrm{d}x=\int (3e)^x\,\mathrm{d}x=\frac{(3e)^x}{\ln(3e)}+C=\frac{3^xe^x}{\ln 3+1}+C$\\
(12)$\displaystyle\int\frac{2\cdot 3^x-5\cdot 2^x}{3^x}\,\mathrm{d}x=\int\left(2-5\left(\frac{2}{3}\right)^x\right)\,\mathrm{d}x=2x-\frac{5}{\ln\frac{2}{3}}\left(\frac{2}{3}\right)^x+C$
\end{solution}

\begin{example}{4.4-A-3}{}
    \begin{tabular}{@{} l l l @{}}
(1) $\displaystyle \int_1^3\frac{\mathrm{d}x}{x}$; & (2) $\displaystyle \int_{-1}^{3}(x^{3}+5x)\mathrm{d}x$; & (3) $\displaystyle \int_{0}^{\pi}\sin\theta(\cos\theta+5)^{7}\mathrm{d}\theta$; \\
(4) $\displaystyle \int_{-1}^1\frac{\mathrm{d}y}{1+y^2}$; & (5) $\displaystyle \int_0^1\frac{x}{1+5x^2}\mathrm{d}x$; & (6) $\displaystyle \int_0^{\pi/12}\sin3t\mathrm{d}t$; \\
(7) $\displaystyle \int_1^2\frac{x^2+1}{x}\mathrm{d}x$; & (8) $\displaystyle \int_{1}^{4}x\sqrt{x^{2}+4}\mathrm{d}x$; & (9) $\displaystyle \int_{0}^{1}\frac{\mathrm{d}x}{x^{2}+2x+1}$; \\
(10) $\displaystyle \int_0^{1/\sqrt{2}}\frac{x\mathrm{d}x}{\sqrt{1-x^4}}$; & (11) $\displaystyle \int_{-2}^{0}\frac{2x+4}{x^{2}+4x+5}\mathrm{d}x$; & (12) $\displaystyle \int_1^9x\sqrt[3]{1-x}\mathrm{d}x$; \\
(13) $\displaystyle \int_1^{\mathrm{e}^2}\frac{\mathrm{d}x}{x\sqrt{1+\ln x}}$; & (14) $\displaystyle \int_{-\frac{\pi}{2}}^{\frac{\pi}{2}}\sqrt{\cos\theta-\cos^{3}\theta}\mathrm{d}\theta$; & (15) $\displaystyle \int_1^2\mathrm{e}^{x^3}x^2\mathrm{d}x$; \\
(16) $\displaystyle \int_2^3\frac{\mathrm{e}^{1/x}}{x^2}\mathrm{d}x$; & (17) $\displaystyle \int_{\pi/6}^{\pi/4}\tan\theta\sec^2\theta\mathrm{d}\theta$; & (18) $\displaystyle \int_0^{\pi/2}\cos^5\theta\sin\theta\mathrm{d}\theta$.
\end{tabular}
\end{example}
\begin{solution}
(1)\[
\int_1^3\frac{\mathrm{d}x}{x}=\left.\ln x\right|_1^3=\ln 3-\ln 1=\ln 3
\]
(2)\[
\int_{-1}^3(x^3+5x)\mathrm{d}x=\left.\left(\frac{x^4}{4}+\frac{5x^2}{2}\right)\right|_{-1}^3=\frac{81-1}{4}+5\cdot\frac{8}{2}=40
\]
(3)令$u=\cos\theta+5$,则$\mathrm{d}u=-\sin\theta\mathrm{d}\theta$,当$\theta=0$时$u=6$,$\theta=\pi$时$u=4$,故\[
\int_{0}^{\pi}\sin\theta(\cos\theta+5)^{7}\mathrm{d}\theta = \int_{6}^{4} u^7 (-\mathrm{d}u) = \int_{4}^{6} u^7 \mathrm{d}u = \left.\frac{u^8}{8}\right|_{4}^{6} = \frac{6^8-4^8}{8}
\]
(4)\[
\int_{-1}^1\frac{\mathrm{d}y}{1+y^2}=\left.\arctan y\right|_{-1}^1 = \arctan1-\arctan(-1)=\frac{\pi}{4}-\left(-\frac{\pi}{4}\right)=\frac{\pi}{2}
\]
(5)令$u=1+5x^2$,则$\mathrm{d}u=10x\mathrm{d}x$,当$x=0$时$u=1$,$x=1$时$u=6$,故\[
\int_0^1\frac{x}{1+5x^2}\mathrm{d}x = \int_1^6\frac{1}{u}\cdot\frac{1}{10}\mathrm{d}u = \frac{1}{10}\left.\ln|u|\right|_1^6 = \frac{\ln6}{10}
\]
(6)\[
\int_0^{\pi/12}\sin3t\mathrm{d}t = \left.-\frac{1}{3}\cos3t\right|_0^{\pi/12} = -\frac{1}{3}\left(\cos\frac{\pi}{4}-\cos0\right) = -\frac{1}{3}\left(\frac{\sqrt{2}}{2}-1\right) = \frac{2-\sqrt{2}}{6}
\]
(7)\[
\int_1^2\frac{x^2+1}{x}\mathrm{d}x = \int_1^2\left(x+\frac{1}{x}\right)\mathrm{d}x = \left.\left(\frac{x^2}{2}+\ln|x|\right)\right|_1^2 = \left(2+\ln2\right)-\left(\frac{1}{2}+\ln1\right)=\frac{3}{2}+\ln2
\]
(8)令$u=x^2+4$,则$\mathrm{d}u=2x\mathrm{d}x$,当$x=1$时$u=5$,$x=4$时$u=20$,故\[
\int_{1}^{4}x\sqrt{x^{2}+4}\mathrm{d}x = \int_5^{20}\sqrt{u}\cdot\frac{1}{2}\mathrm{d}u = \frac{1}{2}\cdot\left.\frac{2}{3} u^{3/2}\right|_5^{20} = \frac{1}{3}\left(20^{3/2}-5^{3/2}\right) = \frac{35\sqrt{5}}{3}
\]
(9)\[
\int_{0}^{1}\frac{\mathrm{d}x}{x^{2}+2x+1} = \int_0^1 (x+1)^{-2}\mathrm{d}x = \left.-(x+1)^{-1}\right|_0^1 = -\frac{1}{2} - (-1) = \frac{1}{2}
\]
(10)令$x^2=\sin\theta$,则$2x\mathrm{d}x=\cos\theta\mathrm{d}\theta$,当$x=0$时$\theta=0$,$x=1/\sqrt{2}$时$\theta=\pi/6$,故\[
\int_0^{1/\sqrt{2}}\frac{x\mathrm{d}x}{\sqrt{1-x^4}} = \int_0^{\pi/6}\frac{\frac{1}{2}\cos\theta\mathrm{d}\theta}{\cos\theta} = \frac{1}{2}\int_0^{\pi/6}\mathrm{d}\theta = \frac{1}{2}\cdot\frac{\pi}{6} = \frac{\pi}{12}
\]
(11)令$u=x^2+4x+5$,则$\mathrm{d}u=(2x+4)\mathrm{d}x$,当$x=-2$时$u=1$,$x=0$时$u=5$,故\[
\int_{-2}^{0}\frac{2x+4}{x^{2}+4x+5}\mathrm{d}x = \int_1^5\frac{\mathrm{d}u}{u} = \left.\ln|u|\right|_1^5 = \ln5
\]
(12)令$u=1-x$,则$\mathrm{d}u=-\mathrm{d}x$,$x=1-u$,当$x=1$时$u=0$,$x=9$时$u=-8$,故\[
\int_1^9x\sqrt[3]{1-x}\mathrm{d}x = \int_0^{-8}(1-u)u^{1/3}(-\mathrm{d}u) = \int_{-8}^0(u^{1/3}-u^{4/3})\mathrm{d}u = \left.\left(\frac{3}{4} u^{4/3}-\frac{3}{7} u^{7/3}\right)\right|_{-8}^0 = -\frac{468}{7}
\]
(13)令$u=1+\ln x$,则$\mathrm{d}u=\frac{1}{x}\mathrm{d}x$,当$x=1$时$u=1$,$x=\mathrm{e}^2$时$u=3$,故\[
\int_1^{\mathrm{e}^2}\frac{\mathrm{d}x}{x\sqrt{1+\ln x}} = \int_1^3 u^{-1/2}\mathrm{d}u = \left.2u^{1/2}\right|_1^3 = 2(\sqrt{3}-1)
\]
(14)\[
\int_{-\frac{\pi}{2}}^{\frac{\pi}{2}}\sqrt{\cos\theta-\cos^{3}\theta}\mathrm{d}\theta = 2\int_0^{\frac{\pi}{2}}\sin\theta\sqrt{\cos\theta}\mathrm{d}\theta
\]
令$u=\cos\theta$,则$\mathrm{d}u=-\sin\theta\mathrm{d}\theta$,故\[
2\int_0^{\frac{\pi}{2}}\sin\theta\sqrt{\cos\theta}\mathrm{d}\theta = 2\int_1^0 \sqrt{u}(-\mathrm{d}u) = 2\int_0^1 u^{1/2}\mathrm{d}u = 2\cdot\left.\frac{2}{3} u^{3/2}\right|_0^1 = \frac{4}{3}
\]
(15)令$u=x^3$,则$\mathrm{d}u=3x^2\mathrm{d}x$,当$x=1$时$u=1$,$x=2$时$u=8$,故\[
\int_1^2\mathrm{e}^{x^3}x^2\mathrm{d}x = \int_1^8 \mathrm{e}^u\cdot\frac{1}{3}\mathrm{d}u = \frac{1}{3}\left.\mathrm{e}^u\right|_1^8 = \frac{1}{3}(\mathrm{e}^8-\mathrm{e})
\]
(16)令$u=1/x$,则$\mathrm{d}u=-\frac{1}{x^2}\mathrm{d}x$,当$x=2$时$u=1/2$,$x=3$时$u=1/3$,故\[
\int_2^3\frac{\mathrm{e}^{1/x}}{x^2}\mathrm{d}x = \int_{1/2}^{1/3} \mathrm{e}^u(-\mathrm{d}u) = \int_{1/3}^{1/2} \mathrm{e}^u\mathrm{d}u = \left.\mathrm{e}^u\right|_{1/3}^{1/2} = \mathrm{e}^{1/2}-\mathrm{e}^{1/3}
\]
(17)令$u=\tan\theta$,则$\mathrm{d}u=\sec^2\theta\mathrm{d}\theta$,当$\theta=\pi/6$时$u=1/\sqrt{3}$,$\theta=\pi/4$时$u=1$,故\[
\int_{\pi/6}^{\pi/4}\tan\theta\sec^2\theta\mathrm{d}\theta = \int_{1/\sqrt{3}}^{1} u\mathrm{d}u = \left.\frac{u^2}{2}\right|_{1/\sqrt{3}}^{1} = \frac{1}{2} - \frac{1}{2}\cdot\frac{1}{3} = \frac{1}{3}
\]
(18)令$u=\cos\theta$,则$\mathrm{d}u=-\sin\theta\mathrm{d}\theta$,当$\theta=0$时$u=1$,$\theta=\pi/2$时$u=0$,故\[
\int_0^{\pi/2}\cos^5\theta\sin\theta\mathrm{d}\theta = \int_1^0 u^5(-\mathrm{d}u) = \int_0^1 u^5\mathrm{d}u = \left.\frac{u^6}{6}\right|_0^1 = \frac{1}{6}
\]
\end{solution}
\begin{example}{4.4-A-4}{}
    \begin{tabular}{@{} l l l @{}}
(1) $\displaystyle \int\frac{x^2\mathrm{d}x}{\sqrt{a^2-x^2}}$; & (2) $\displaystyle \int_0^{a/\sqrt{2}}\frac{\mathrm{d}x}{(a^2-x^2)^{3/2}}\mathrm{(a>0)}$; & (3) $\displaystyle \int\sqrt{\frac{x}{1+x\sqrt{x}}}\mathrm{d}x$; \\
(4) $\displaystyle \int\frac{\mathrm{d}x}{1+\sqrt{x}}$; & (5) $\displaystyle \int\frac{\mathrm{e}^{x}}{\sqrt{1-\mathrm{e}^{2x}}}\mathrm{d}x$; & (6) $\displaystyle \int\frac{\mathrm{d}x}{1+\sqrt{1-x^{2}}}$; \\
(7) $\displaystyle \int_1^{\sqrt{3}}\frac{\sqrt{1+x^2}}{x}\mathrm{d}x$; & (8) $\displaystyle \int_0^ax^2\sqrt{a^2-x^2}\mathrm{d}x$; & (9) $\displaystyle \int\frac{\mathrm{d}x}{x\sqrt{x^{2}-1}}$; \\
(10) $\displaystyle \int\frac{\arctan\sqrt{x}}{\sqrt{x}(1+x)}\mathrm{d}x$. & & \\
\end{tabular}
\end{example}
\begin{solution}
(1)\begin{align*}
 \int\frac{x^2\mathrm{d}x}{\sqrt{a^2-x^2}}&=\int\dfrac{a^2\sin^2\theta \dd a\sin\theta}{a\sqrt{1-\sin^2\theta}}=a^2\int\sin^2\theta\dd \theta=a^2\int\frac12\dd\theta-\dfrac{a^2}2\int\cos 2\theta\dd\theta\\
&=\dfrac{a^2}{2}\theta-\dfrac{a^2}{4}\sin 2\theta+C=\dfrac{a^2}{2}\arcsin\dfrac{x}{a}-\dfrac{a^2}2\dfrac{x}{a}\sqrt{1-\dfrac{x^2}{a^2}}+C\\
&=\dfrac{a^2}{2}\arcsin\dfrac{x}{a}-\dfrac{x}{2}\sqrt{a^2-x^2}+C
\end{align*}
(2)\begin{align*}
\int_0^{a/\sqrt{2}}\frac{\mathrm{d}x}{(a^2-x^2)^{3/2}}&=\int_0^\frac{\pi}{4}\dfrac{\dd a\sin\theta}{(a^2-a^2\sin^2\theta)^{\frac32}}=\dfrac1{a^2}\int_0^\frac{\pi}{4}\sec^2\theta\dd\theta=\dfrac1{a^2}\left.\tan\theta\right|_0^\frac{\pi}{4}=\dfrac1{a^2}\\
\end{align*}
(3)\begin{align*}
    \int\sqrt{\frac{x}{1+x\sqrt{x}}}\mathrm{d}x&=\int\dfrac{t\dd t^2}{\sqrt{1+t^3}}=\int\dfrac{2t^2\dd t}{\sqrt{1+t^3}}=\dfrac23\int\dfrac{3t^2\dd t}{\sqrt{1+t^3}}=\dfrac23\int\dfrac{\dd t^3}{\sqrt{1+t^3}}\\
    &=\dfrac23\dfrac{(1+t^3)^{\frac12}}{1-\frac12}+C=\dfrac{4}3\sqrt{1+t^3}+C=\dfrac43\sqrt{1+x\sqrt{x}}+C
\end{align*}
(4)\begin{align*}
\int\frac{\mathrm{d}x}{1+\sqrt{x}} &=\int\frac{2t}{1+t}\mathrm{d}t = 2\int\frac{t}{1+t}\mathrm{d}t = 2\int\left(1-\frac{1}{1+t}\right)\mathrm{d}t \\
&= 2\left(t - \ln|1+t|\right)+C = 2\sqrt{x} - 2\ln(1+\sqrt{x}) + C
\end{align*}
(5)\begin{align*}
\int\frac{\mathrm{e}^{x}}{\sqrt{1-\mathrm{e}^{2x}}}\mathrm{d}x &=\int\frac{\mathrm{d}u}{\sqrt{1-u^2}} = \arcsin u + C = \arcsin(\mathrm{e}^x) + C
\end{align*}
(6)\begin{align*}
\int\frac{\mathrm{d}x}{1+\sqrt{1-x^{2}}} &= \int\frac{1-\sqrt{1-x^2}}{x^2}\mathrm{d}x = \int\frac{1}{x^2}\mathrm{d}x - \int\frac{\sqrt{1-x^2}}{x^2}\mathrm{d}x \\
&= -\frac{1}{x} - \int\frac{\sqrt{1-x^2}}{x^2}\mathrm{d}x
\end{align*}
对后一项积分,令$x=\sin t$,$\mathrm{d}x=\cos t\mathrm{d}t$,则
\begin{align*}
\int\frac{\sqrt{1-x^2}}{x^2}\mathrm{d}x &= \int\frac{\cos t}{\sin^2 t}\cdot\cos t\mathrm{d}t = \int\cot^2 t\mathrm{d}t = \int(\csc^2 t -1)\mathrm{d}t \\
&= -\cot t - t + C = -\frac{\sqrt{1-x^2}}{x} - \arcsin x + C
\end{align*}
代回得
\begin{align*}
\int\frac{\mathrm{d}x}{1+\sqrt{1-x^{2}}} &= -\frac{1}{x} - \left(-\frac{\sqrt{1-x^2}}{x} - \arcsin x\right) + C \\
&= \frac{\sqrt{1-x^2}-1}{x} + \arcsin x + C
\end{align*}
(7)\begin{align*}
\int_1^{\sqrt{3}}\frac{\sqrt{1+x^2}}{x}\mathrm{d}x &\xlongequal{x=\tan t} \int_{\frac{\pi}{4}}^{\frac{\pi}{3}} \frac{\sec t}{\tan t}\cdot\sec^2 t\mathrm{d}t = \int_{\frac{\pi}{4}}^{\frac{\pi}{3}} \frac{\sec^3 t}{\tan t}\mathrm{d}t \\
&= \int_{\frac{\pi}{4}}^{\frac{\pi}{3}} \left(\frac{\sin t}{\cos^2 t} + \csc t\right)\mathrm{d}t \\
&= \int_{\frac{\pi}{4}}^{\frac{\pi}{3}} \frac{\sin t}{\cos^2 t}\mathrm{d}t + \int_{\frac{\pi}{4}}^{\frac{\pi}{3}} \csc t\mathrm{d}t \\
&= \left[ \sec t + \ln|\csc t - \cot t| \right]_{\frac{\pi}{4}}^{\frac{\pi}{3}} \\
&= \left(2 + \ln\frac{1}{\sqrt{3}}\right) - \left(\sqrt{2} + \ln(\sqrt{2}-1)\right) \\
&= 2 - \sqrt{2} - \ln\left(\sqrt{3}(\sqrt{2}-1)\right)
\end{align*}
(8)\begin{align*}
\int_0^a x^2\sqrt{a^2-x^2}\mathrm{d}x &\xlongequal{x=a\sin t} \int_0^{\frac{\pi}{2}} a^2\sin^2 t \cdot a\cos t \cdot a\cos t\mathrm{d}t = a^4\int_0^{\frac{\pi}{2}} \sin^2 t \cos^2 t\mathrm{d}t \\
&= a^4\int_0^{\frac{\pi}{2}} \frac{1}{4}\sin^2 2t\mathrm{d}t = \frac{a^4}{4}\int_0^{\frac{\pi}{2}} \frac{1-\cos4t}{2}\mathrm{d}t \\
&= \frac{a^4}{8} \left[t - \frac{1}{4}\sin4t\right]_0^{\frac{\pi}{2}} = \frac{a^4}{8}\cdot\frac{\pi}{2} = \frac{\pi a^4}{16}
\end{align*}
(9)\begin{align*}
\int\frac{\mathrm{d}x}{x\sqrt{x^{2}-1}} &\xlongequal{x=\sec t} \int\frac{\sec t\tan t}{\sec t\cdot\tan t}\mathrm{d}t = \int\mathrm{d}t = t + C = \arccos\left(\frac{1}{x}\right) + C
\end{align*}
(10)\begin{align*}
\int\frac{\arctan\sqrt{x}}{\sqrt{x}(1+x)}\mathrm{d}x &\xlongequal{t=\sqrt{x}} \int\frac{\arctan t}{t(1+t^2)}\cdot 2t\mathrm{d}t = 2\int\frac{\arctan t}{1+t^2}\mathrm{d}t \\
&\xlongequal{u=\arctan t} 2\int u\mathrm{d}u = u^2 + C = (\arctan t)^2 + C = (\arctan\sqrt{x})^2 + C
\end{align*}
\end{solution}
\begin{example}{4.4-B-1}{}
    \begin{tabular}{@{} l l l @{}}
(1) $\displaystyle \int x\sqrt[3]{1-5x^2}\,\mathrm{d}x$; & (2) $\displaystyle \int\frac{\mathrm{d}x}{1-x}$; & (3) $\displaystyle \int\frac{1+\mathrm{e}^x}{\sqrt{x+\mathrm{e}^x}}\,\mathrm{d}x$; \\
(4) $\displaystyle \int\frac{\mathrm{d}x}{\mathrm{e}^x+\mathrm{e}^{-x}}$; & (5) $\displaystyle \int\frac{\mathrm{e}^{\sqrt{y}}}{\sqrt{y}}\,\mathrm{d}y$; & (6) $\displaystyle \int\frac{\mathrm{e}^x-\mathrm{e}^{-x}}{\mathrm{e}^x+\mathrm{e}^{-x}}\,\mathrm{d}x$; \\
(7) $\displaystyle \int\frac{x^3}{9+x^2}\,\mathrm{d}x$; & (8) $\displaystyle \int\frac{\mathrm{d}x}{x(x^6+4)}$; & (9) $\displaystyle \int\frac{x+1}{x^2+2x+19}\,\mathrm{d}x$; \\
(10) $\displaystyle \int\frac{\sin x+\cos x}{\sqrt[3]{\sin x-\cos x}}\,\mathrm{d}x$; & (11) $\displaystyle \int\frac{\mathrm{d}x}{1+\cos x}$; & (12) $\displaystyle \int\frac{\mathrm{d}x}{1+\sin x}$.
\end{tabular}
\end{example}
\begin{solution}
(1) \begin{align*}
\int x\sqrt[3]{1-5x^2}\,\mathrm{d}x &\overset{\text{令 }t=1-5x^2}{=} \int t^{1/3} \cdot \left(-\frac{1}{10}\right)\mathrm{d}t = -\frac{1}{10} \cdot \frac{3}{4} t^{4/3} + C \\
&= -\frac{3}{40} (1-5x^2)^{4/3} + C
\end{align*}
(2) \begin{align*}
\int\frac{x\mathrm{d}x}{1-x} =\int\left(-1+\dfrac1{1-x}\right)\dd x= -x-\ln|1-x| + C
\end{align*}
(3) \begin{align*}
\int\frac{1+\mathrm{e}^x}{\sqrt{x+\mathrm{e}^x}}\,\mathrm{d}x &\overset{\text{令 }u=x+\mathrm{e}^x}{=} \int \frac{1}{\sqrt{u}} \,\mathrm{d}u = 2\sqrt{u} + C \\
&= 2\sqrt{x+\mathrm{e}^x} + C
\end{align*}
(4) \begin{align*}
\int\frac{\mathrm{d}x}{\mathrm{e}^x+\mathrm{e}^{-x}} &= \int \frac{\mathrm{e}^x}{\mathrm{e}^{2x}+1} \,\mathrm{d}x \overset{\text{令 }t=\mathrm{e}^x}{=} \int \frac{1}{t^2+1} \,\mathrm{d}t = \arctan t + C \\
&= \arctan(\mathrm{e}^x) + C
\end{align*}
(5) \begin{align*}
\int\frac{\mathrm{e}^{\sqrt{y}}}{\sqrt{y}}\,\mathrm{d}y &\overset{\text{令 }t=\sqrt{y}}{=} \int \frac{\mathrm{e}^t}{t} \cdot 2t \,\mathrm{d}t = 2\int \mathrm{e}^t \,\mathrm{d}t = 2\mathrm{e}^t + C \\
&= 2\mathrm{e}^{\sqrt{y}} + C
\end{align*}
(6) \begin{align*}
\int\frac{\mathrm{e}^x-\mathrm{e}^{-x}}{\mathrm{e}^x+\mathrm{e}^{-x}}\,\mathrm{d}x &\overset{\text{令 }u=\mathrm{e}^x+\mathrm{e}^{-x}}{=} \int \frac{1}{u} \,\mathrm{d}u = \ln|u| + C \\
&= \ln(\mathrm{e}^x+\mathrm{e}^{-x}) + C
\end{align*}
(7) \begin{align*}
\int\frac{x^3}{9+x^2}\,\mathrm{d}x &= \int \left( x - \frac{9x}{9+x^2} \right) \,\mathrm{d}x = \int x \,\mathrm{d}x - 9\int \frac{x}{9+x^2} \,\mathrm{d}x \\
&= \frac{x^2}{2} - \frac{9}{2} \ln(9+x^2) + C
\end{align*}
(8) \begin{align*}
\int\frac{\mathrm{d}x}{x(x^6+4)} &\overset{\text{令 }t=x^3}{=} \int \frac{1}{t(t^2+4)} \cdot \frac{1}{3} \,\mathrm{d}t = \frac{1}{3} \int \left( \frac{1}{4t} - \frac{t}{4(t^2+4)} \right) \,\mathrm{d}t \\
&= \frac{1}{12} \ln|t| - \frac{1}{24} \ln(t^2+4) + C \\
&= \frac{1}{12} \ln|x^3| - \frac{1}{24} \ln(x^6+4) + C \\
&= \frac{1}{4} \ln|x| - \frac{1}{24} \ln(x^6+4) + C
\end{align*}
(9) \begin{align*}
\int\frac{x+1}{x^2+2x+19}\,\mathrm{d}x &\overset{\text{令 }u=x^2+2x+19}{=} \frac{1}{2} \int \frac{1}{u} \,\mathrm{d}u = \frac{1}{2} \ln|u| + C \\
&= \frac{1}{2} \ln(x^2+2x+19) + C
\end{align*}
(10) \begin{align*}
\int\frac{\sin x+\cos x}{\sqrt[3]{\sin x-\cos x}}\,\mathrm{d}x &\overset{\text{令 }t=\sin x-\cos x}{=} \int t^{-1/3} \,\mathrm{d}t = \frac{3}{2} t^{2/3} + C \\
&= \frac{3}{2} (\sin x-\cos x)^{2/3} + C
\end{align*}
(11) \begin{align*}
\int\frac{\mathrm{d}x}{1+\cos x} &= \int \frac{1}{2\cos^2(x/2)} \,\mathrm{d}x = \int \sec^2\left(\frac{x}{2}\right) \cdot \frac{1}{2} \,\mathrm{d}x \\
&\overset{\text{令 }t=x/2}{=} \int \sec^2 t \,\mathrm{d}t = \tan t + C \\
&= \tan\left(\frac{x}{2}\right) + C
\end{align*}
(12) \begin{align*}
\int\frac{\mathrm{d}x}{1+\sin x} &= \int \frac{1-\sin x}{\cos^2 x} \,\mathrm{d}x = \int \left( \sec^2 x - \sec x \tan x \right) \,\mathrm{d}x \\
&= \tan x - \sec x + C
\end{align*}
\end{solution}
\begin{example}{4.4-B-2}{}
    \begin{tabular}{@{} l l l @{}}
(1) $\displaystyle \int \frac{\mathrm{d}x}{\sqrt{1+e^x}}$; & (2) $\displaystyle \int \frac{\mathrm{d}x}{x+\sqrt{1-x^2}}$; & (3) $\displaystyle \int \frac{\sqrt{a^2-x^2}}{x^4}\mathrm{d}x$ (令 $x=\frac{1}{t}$); \\
(4) $\displaystyle \int \frac{\mathrm{d}x}{\sqrt{(x^2+1)^3}}$; & (5) $\displaystyle \int \frac{1}{4} \frac{\arcsin \sqrt{x}}{\sqrt{x(1-x)}}\mathrm{d}x$; & (6) $\displaystyle \int \frac{1+\ln x}{(x\ln x)^2}\mathrm{d}x$; \\
(7) $\displaystyle \int \frac{\ln \tan x}{\cos x \sin x}\mathrm{d}x$; & (8) $\displaystyle \int_a^0 \sqrt{\frac{a-x}{a+x}}\mathrm{d}x$ (令 $x=a\sin t$). & \\
\end{tabular}
\end{example}
\begin{solution}
(1) \begin{align*}
\int \frac{\mathrm{d}x}{\sqrt{1+e^x}} &\overset{\text{令 }t=\sqrt{1+e^x}}{=} \int \frac{1}{t} \cdot \frac{2t}{t^2-1} \,\mathrm{d}t = 2\int \frac{\mathrm{d}t}{t^2-1} \\
&= 2 \cdot \frac{1}{2} \ln \left| \frac{t-1}{t+1} \right| + C = \ln \left| \frac{\sqrt{1+e^x}-1}{\sqrt{1+e^x}+1} \right| + C\\
&=x-2\ln(1+\sqrt{1+\e^x})+C
\end{align*}
(2) \begin{align*}
\int \frac{\mathrm{d}x}{x+\sqrt{1-x^2}} &\overset{\text{令 }x=\sin \theta}{=} \int \frac{\cos \theta}{\sin \theta + \cos \theta} \,\mathrm{d}\theta = \frac{1}{2} \int \left( 1 + \frac{\cos \theta - \sin \theta}{\sin \theta + \cos \theta} \right) \,\mathrm{d}\theta \\
&= \frac{1}{2} \left( \theta + \ln |\sin \theta + \cos \theta| \right) + C \\
&= \frac{1}{2} \arcsin x + \frac{1}{2} \ln \left( x + \sqrt{1-x^2} \right) + C.
\end{align*}
(3) \begin{align*}
\int \frac{\sqrt{a^2-x^2}}{x^4} \,\mathrm{d}x &\overset{\text{令 }x=\frac{1}{t}}{=} \int \frac{\sqrt{a^2 - \frac{1}{t^2}}}{\frac{1}{t^4}} \left( -\frac{1}{t^2} \right) \,\mathrm{d}t = -\int t \sqrt{a^2 t^2 - 1} \,\mathrm{d}t \\
&\overset{\text{令 }u=a^2 t^2 - 1}{=} -\frac{1}{2a^2} \int \sqrt{u} \,\mathrm{d}u = -\frac{1}{3a^2} u^{3/2} + C \\
&= -\frac{1}{3a^2} \left( a^2 t^2 - 1 \right)^{3/2} + C = -\frac{1}{3a^2} \left( \frac{a^2}{x^2} - 1 \right)^{3/2} + C \\
&= -\frac{(a^2 - x^2)^{3/2}}{3a^2 x^3} + C.
\end{align*}
(4) \begin{align*}
\int \frac{\mathrm{d}x}{\sqrt{(x^2+1)^3}} &\overset{\text{令 }x=\tan \theta}{=} \int \frac{\sec^2 \theta}{\sec^3 \theta} \,\mathrm{d}\theta = \int \cos \theta \,\mathrm{d}\theta = \sin \theta + C = \frac{x}{\sqrt{x^2+1}} + C.
\end{align*}
(5) \begin{align*}
\int_{\frac{1}{4}}^{\frac12} \frac{\arcsin \sqrt{x}}{\sqrt{x(1-x)}} \,\mathrm{d}x &\overset{\text{令 }t=\arcsin \sqrt{x}}{=} \int_{\frac{\pi}{6}}^{\frac{\pi}4}\frac{t}{\sin t \cos t} \cdot 2\sin t \cos t \,\mathrm{d}t = 2\int_{\frac{\pi}{6}}^{\frac{\pi}4} t \,\mathrm{d}t \\
&= \left.t^2\right|_{\frac{\pi}{6}}^{\frac{\pi}4} = \dfrac{5}{144}\pi^2
\end{align*}
(6) \begin{align*}
\int \frac{1+\ln x}{(x\ln x)^2} \,\mathrm{d}x &= \int \frac{1}{ (x\ln x)^2} \,\mathrm{d}x\ln x = -\frac{1}{x\ln x} + C.
\end{align*}
(7) \begin{align*}
\int \frac{\ln \tan x}{\cos x \sin x} \,\mathrm{d}x &\overset{\text{令 }t=\ln \tan x}{=}\int\dfrac{t\dd\arctan\e^t}{\frac{\e^t}{\sqrt{1+\e^{2t}}}\frac{1}{\sqrt{1+\e^{2t}}}}\\
&= \int t \,\mathrm{d}u = \frac{1}{2} t^2 + C = \frac{1}{2} \left( \ln \tan x \right)^2 + C.
\end{align*}
(8) \begin{align*}
\int_0^a \sqrt{\frac{a-x}{a+x}} \,\mathrm{d}x &\overset{\text{令 }x=a\sin t}{=} \int_0^{\pi/2}\sqrt{\frac{1-\sin t}{1+\sin t}} \cdot a\cos t \,\mathrm{d}t = a \int_0^{\pi/2} \frac{1-\sin t}{\cos t} \cdot \cos t \,\mathrm{d}t \\
&= a \int_0^{\pi/2}(1-\sin t) \,\mathrm{d}t = a \left[ t + \cos t \right]_0^{\pi/2} = a \left(  \frac{\pi}{2}-1 \right).
\end{align*}
\end{solution}
\begin{example}{4.4-B-3}{}
设 $f(x) \in C[0, \pi]$,证明:

(1) $\int_{0}^{\pi/2} f(\sin x) \, dx = \int_{0}^{\pi/2} f(\cos x) \, dx$;

(2) $\int_{0}^{\pi} x f(\sin x) \dd x = \frac{\pi}{2} \int_{0}^{\pi} f(\sin x) \dd x$,并由此计算 $\int_{0}^{\pi} \frac{x \sin x}{1 + \cos^2 x} \dd x$.
\end{example}
\begin{solution}
(1)\begin{align*}
\int_{0}^{\pi/2} f(\sin x) \dd x=\int_{-\frac{\pi}2}^0f\left(\sin(x+\frac{\pi}2)\right)\dd x=\int_{-\frac{\pi}2}^0f(\cos x)\dd x=\int_{0}^{\pi/2}f(\cos x)\dd x
\end{align*}
(2)\begin{align*}
    \int_{0}^{\pi} \frac{x \sin x}{1 + \cos^2 x} \dd x &=\int_{-\frac{\pi}{2}}^{\frac{\pi}{2}} \frac{(x+\frac{\pi}2) \sin (x+\frac{\pi}2) }{1 + \cos^2 (x+\frac{\pi}2) } \dd x=\int_{-\frac{\pi}{2}}^{\frac{\pi}{2}} \frac{(x+\frac{\pi}2) \cos x}{1 + \sin^2x}\dd x\\
    &=\int_{-\frac{\pi}{2}}^{\frac{\pi}{2}} \frac{x\cos x}{1 + \sin^2x}\dd x+\dfrac{\pi}{2}\int_{-\frac{\pi}{2}}^{\frac{\pi}{2}} \frac{\cos x}{1 + \sin^2x}\dd x\\
    &=\pi\int_0^{\frac{\pi}{2}}\frac{\cos x}{1+\sin^2x}\dd x=\pi\int_0^{\frac{\pi}{2}}\frac{\dd\sin x}{1+\sin^2x}\\
    &=\pi\times \left.\arctan (\sin x)\right|_0^{\frac{\pi}{2}}=\frac{\pi^2}{4}
\end{align*}
\end{solution}
\begin{example}{4.4-B-4}{}
    设$f(x)$是以$T$为周期的周期函数,证明积分$\displaystyle\int_{a}^{a+T}f(x)\dd x$的值与$a$无关。
\end{example}
\begin{solution}
\begin{align*}
    \int_{a}^{a+T} f(x)\,\mathrm{d}x &= \int_{a}^{T} f(x)\,\mathrm{d}x + \int_{T}^{a+T} f(x)\,\mathrm{d}x\\
    &= \int_{a}^{T} f(x)\dd x+\int_{0}^af(x+T)\dd x\\
    &= \int_{a}^{T} f(x)\dd x+\int_{0}^a f(x)\dd x\\
    &= \int_{0}^{T} f(x)\dd x
\end{align*}
\end{solution}
\begin{example}{4.4-B-5}{}
若$f( t) $连 续 且 为 奇 函 数 , 证 明 $\int _0^{x}f(t)\dd t$是偶函数;若$f( t) $连 续 且 为 偶 函 数 , 证 明 $\int _0^xf( t)dt$是奇函数
\end{example}
\begin{solution}
(1)若$f( t) $连 续 且 为 奇 函 数 :由$\displaystyle\int _0^{x}f(t)\dd t + \int_{-x}^0f(t)\dd t=0$:
\begin{align*}
    \int _0^{x}f(t)\dd t &= -\int_{-x}^0f(t)\dd t=\int_0^{-x}f(t)\dd t\\
\end{align*}
所以$\int _0^{x}f(t)\dd t$是偶函数。

(2)若$f( t) $连 续 且 为 偶 函 数 :由$\displaystyle\int_0^xf( t)dt = \int_{-x}^{0}f( t)\dd t$得到:
\begin{align*}
    \int_0^xf( t)\dd t &= \int_{-x}^{0}f( t)\dd t = -\int_{0}^{-x}f(t)\dd t\\
    \Rightarrow &\int_0^xf( t)\dd t+\int_{-x}^{0}f( t)\dd t = 0
\end{align*}
所以$\int _0^xf( t)dt$是奇函数。
\end{solution}
\begin{example}{4.4-B-6}{}
    设$f( x) $是 连 续 函 数 , 证 明 $\int _0^2f( x)$d$x=\int_0^1[f(x)+f(x+1)]\dd x.$
\end{example}
\begin{solution}
\begin{align*}
    \int _0^2f( x)\dd x &= \int_0^1f(x)\dd x + \int_1^2f(x)\dd x=\int_0^1f(x)\dd x + \int_0^1f(x+1)\dd x\\
    &=\int_0^1[f(x)+f(x+1)]\dd x
\end{align*}
\end{solution}
\begin{example}{4.5-A-5}{}
    求$I_n=\int x^n\e^{-x}\dd x$
\end{example}
\begin{solution}
\begin{align*}
    I_n=\int x^n\e^{-x}\dd x &= -\int x^n\dd\e^{-x}=-x^n\e^{-x}+\int \e^{-x}\dd x^n\\
    &= -x^n\e^{-x}+n\int \e^{-x}\dd x^{n-1}=-x^n\e^{-x}+nI_{n-1}\\
    &= -x^n\e^{-x}-nx^{n-1}\e^{-x}-n(n-1)x^{n-2}\e^{-x}-\cdots-n!x\e^{-x}+n!I_0\\
    &= -e^{-x} \left( x^n + n x^{n-1} + n(n-1) x^{n-2} + \cdots + n! x + n! \right) + C
\end{align*}
\end{solution}
\begin{example}{4.5-B-3}{}
    \begin{tabular}{@{} l l l @{}}
(1) $\displaystyle \int\ln(1+x^2)\,\mathrm{d}x$; & (2) $\displaystyle \int\arctan\sqrt{x}\,\mathrm{d}x$; & (3) $\displaystyle \int\frac{x+\sin x}{1+\cos x}\,\mathrm{d}x$; \\
(4) $\displaystyle \int\frac{\sin^2x}{\cos^3x}\,\mathrm{d}x$; & (5) $\displaystyle \int\frac{\mathrm{d}x}{(1+\mathrm{e}^x)^2}$; & (6) $\displaystyle \int\frac{x\mathrm{e}^x}{(\mathrm{e}^x+1)^2}\,\mathrm{d}x$; \\
(7) $\displaystyle \int\frac{x\mathrm{e}^x}{\sqrt{\mathrm{e}^x-1}}\,\mathrm{d}x$; & (8) $\displaystyle \int\frac{x^2}{(1+x^2)^2}\,\mathrm{d}x$; & (9) $\displaystyle \int_{0}^{1}\frac{\ln(1+x)}{(2-x)^{2}}\,\mathrm{d}x$; \\
(10) $\displaystyle \int\frac{x+\ln x}{\left(1+x\right)^{2}}\,\mathrm{d}x$; & (11) $\displaystyle \int_{0}^{1}\frac{x}{\mathrm{e}^{x}+\mathrm{e}^{1-x}}\,\mathrm{d}x$; & (12) $\displaystyle \int_0^{1/2}x\ln\frac{1+x}{1-x}\,\mathrm{d}x$; \\
(13) $\displaystyle \int_0^3\arcsin\sqrt{\frac{x}{1+x}}\,\mathrm{d}x$; & (14) $\displaystyle \int x\sqrt{x+3}\,\mathrm{d}x$; & (15) $\displaystyle \int\frac{x}{\sqrt{5-x}}\,\mathrm{d}x$; \\
(16) $\displaystyle \int(t+2)\sqrt{2+3t}\,\mathrm{d}t$; & (17) $\displaystyle \int\frac{t+7}{\sqrt{5-t}}\,\mathrm{d}t$. & \\
\end{tabular}
\end{example}
\begin{solution}
(1) 
\begin{align*}
\int \ln(1+x^2)\,\mathrm{d}x &= x\ln(1+x^2) - \int \frac{2x^2}{1+x^2}\,\mathrm{d}x \\
&= x\ln(1+x^2) - 2\int \left(1 - \frac{1}{1+x^2}\right)\mathrm{d}x \\
&= x\ln(1+x^2) - 2x + 2\arctan x + C.
\end{align*}

(2) 
\begin{align*}
\int \arctan\sqrt{x}\,\mathrm{d}x &\overset{u=\sqrt{x}}{=} 2\int u\arctan u\,\mathrm{d}u = u^2\arctan u - \int \frac{u^2}{1+u^2}\,\mathrm{d}u \\
&= u^2\arctan u - \int \left(1 - \frac{1}{1+u^2}\right)\mathrm{d}u \\
&= u^2\arctan u - u + \arctan u + C = x\arctan\sqrt{x} - \sqrt{x} + \arctan\sqrt{x} + C.
\end{align*}

(3) 
\begin{align*}
\int\frac{x+\sin x}{1+\cos x}\,\mathrm{d}x &= \int \frac{x}{2\cos^2(x/2)}\,\mathrm{d}x + \int \tan(x/2)\,\mathrm{d}x \\
&= \int x\,\mathrm{d}\tan(x/2) + 2\ln|\cos(x/2)| \\
&= x\tan(x/2) - \int \tan(x/2)\,\mathrm{d}x + 2\ln|\cos(x/2)| \\
&= x\tan(x/2) + 2\ln|\cos(x/2)| - 2\ln|\cos(x/2)| + C \\
&= x\tan(x/2) + C.
\end{align*}

(4) 
\begin{align*}
\int\frac{\sin^2x}{\cos^3x}\,\mathrm{d}x &= \int \tan^2x\sec x\,\mathrm{d}x = \int (\sec^2x-1)\sec x\,\mathrm{d}x \\
&= \int \sec^3x\,\mathrm{d}x - \int \sec x\,\mathrm{d}x \\
&= \frac{1}{2}\left(\sec x\tan x + \ln|\sec x+\tan x|\right) - \ln|\sec x+\tan x| + C \\
&= \frac{1}{2}\sec x\tan x - \frac{1}{2}\ln|\sec x+\tan x| + C.
\end{align*}

(5) 
\begin{align*}
\int\frac{\mathrm{d}x}{(1+\mathrm{e}^x)^2} &\overset{u=\mathrm{e}^x}{=} \int \frac{1}{u(1+u)^2}\,\mathrm{d}u = \int \left(\frac{1}{u} - \frac{1}{1+u} - \frac{1}{(1+u)^2}\right)\mathrm{d}u \\
&= \ln|u| - \ln|1+u| + \frac{1}{1+u} + C \\
&= x - \ln(1+\mathrm{e}^x) + \frac{1}{1+\mathrm{e}^x} + C.
\end{align*}

(6) 
\begin{align*}
\int\frac{x\mathrm{e}^x}{(\mathrm{e}^x+1)^2}\,\mathrm{d}x &= -\int x\,\mathrm{d}\left(\frac{1}{\mathrm{e}^x+1}\right) = -\frac{x}{\mathrm{e}^x+1} + \int \frac{\mathrm{d}x}{\mathrm{e}^x+1} \\
&= -\frac{x}{\mathrm{e}^x+1} + \int \left(1 - \frac{\mathrm{e}^x}{\mathrm{e}^x+1}\right)\mathrm{d}x \\
&= -\frac{x}{\mathrm{e}^x+1} + x - \ln(\mathrm{e}^x+1) + C \\
&= \frac{x\mathrm{e}^x}{\mathrm{e}^x+1} - \ln(\mathrm{e}^x+1) + C.
\end{align*}

(7) 
\begin{align*}
\int\frac{x\mathrm{e}^x}{\sqrt{\mathrm{e}^x-1}}\,\mathrm{d}x &\overset{u=\sqrt{\mathrm{e}^x-1}}{=} 2\int \ln(u^2+1)\,\mathrm{d}u \\
&= 2\left(u\ln(u^2+1) - \int \frac{2u^2}{u^2+1}\,\mathrm{d}u\right) \\
&= 2u\ln(u^2+1) - 4\int \left(1 - \frac{1}{u^2+1}\right)\mathrm{d}u \\
&= 2u\ln(u^2+1) - 4u + 4\arctan u + C \\
&= 2x\sqrt{\mathrm{e}^x-1} - 4\sqrt{\mathrm{e}^x-1} + 4\arctan\sqrt{\mathrm{e}^x-1} + C.
\end{align*}

(8) 
\begin{align*}
\int\frac{x^2}{(1+x^2)^2}\,\mathrm{d}x &\overset{x=\tan\theta}{=} \int \sin^2\theta\,\mathrm{d}\theta = \int \frac{1-\cos2\theta}{2}\,\mathrm{d}\theta \\
&= \frac{\theta}{2} - \frac{\sin2\theta}{4} + C \\
&= \frac{1}{2}\arctan x - \frac{x}{2(1+x^2)} + C.
\end{align*}

(9) 
\begin{align*}
\int_{0}^{1}\frac{\ln(1+x)}{(2-x)^{2}}\,\mathrm{d}x &= \left[\frac{\ln(1+x)}{2-x}\right]_0^1 - \int_0^1 \frac{1}{(2-x)(1+x)}\,\mathrm{d}x \\
&= \ln2 - \frac{1}{3}\int_0^1 \left(\frac{1}{2-x}+\frac{1}{1+x}\right)\mathrm{d}x \\
&= \ln2 - \frac{1}{3}\left[-\ln|2-x| + \ln|1+x|\right]_0^1 \\
&= \ln2 - \frac{1}{3}\left(\ln2 + \ln2\right) = \frac{1}{3}\ln2.
\end{align*}

(10) 
\begin{align*}
\int\frac{x+\ln x}{(1+x)^2}\,\mathrm{d}x &= \int \frac{x}{(1+x)^2}\,\mathrm{d}x + \int \frac{\ln x}{(1+x)^2}\,\mathrm{d}x \\
&= \int \left(\frac{1}{1+x} - \frac{1}{(1+x)^2}\right)\mathrm{d}x - \frac{\ln x}{1+x} + \int \frac{1}{x(1+x)}\,\mathrm{d}x \\
&= \ln|1+x| + \frac{1}{1+x} - \frac{\ln x}{1+x} + \int \left(\frac{1}{x} - \frac{1}{1+x}\right)\mathrm{d}x \\
&= \ln|1+x| + \frac{1}{1+x} - \frac{\ln x}{1+x} + \ln|x| - \ln|1+x| + C \\
&= \ln x + \frac{1-\ln x}{1+x} + C = \frac{1 + x\ln x}{1+x} + C.
\end{align*}

(11) 
\begin{align*}
\int_{0}^{1}\frac{x}{\mathrm{e}^{x}+\mathrm{e}^{1-x}}\,\mathrm{d}x &= \frac{1}{2}\int_0^1 \frac{1}{\mathrm{e}^{x}+\mathrm{e}^{1-x}}\,\mathrm{d}x \quad \text{(由对称性)} \\
&= \frac{1}{2}\int_0^1 \frac{\mathrm{e}^{-x}}{1+\mathrm{e}^{1-2x}}\,\mathrm{d}x \\
&\overset{u=\mathrm{e}^{-x}}{=} \frac{1}{2}\int_1^{e^{-1}} \frac{u}{1+\mathrm{e}u^2}\left(-\frac{\mathrm{d}u}{u}\right) \\
&= \frac{1}{2}\int_{e^{-1}}^1 \frac{\mathrm{d}u}{1+\mathrm{e}u^2} = \frac{1}{2\sqrt{\mathrm{e}}} \left[\arctan(\sqrt{\mathrm{e}}u)\right]_{e^{-1}}^1 \\
&= \frac{1}{2\sqrt{\mathrm{e}}} \left(\arctan\sqrt{\mathrm{e}} - \arctan\frac{1}{\sqrt{\mathrm{e}}}\right).
\end{align*}

(12) 
\begin{align*}
\int_0^{1/2} x\ln\frac{1+x}{1-x}\,\mathrm{d}x &= \int_0^{1/2} x\ln(1+x)\,\mathrm{d}x - \int_0^{1/2} x\ln(1-x)\,\mathrm{d}x \\
&= \left[\frac{x^2}{2}\ln(1+x) - \frac{x^2}{4} + \frac{x}{2} - \frac{1}{2}\ln(1+x)\right]_0^{1/2} \\
&\quad - \left[\frac{x^2}{2}\ln(1-x) + \frac{x^2}{4} + \frac{x}{2} + \frac{1}{2}\ln(1-x)\right]_0^{1/2} \\
&= \ln2 - \frac{3}{8}\ln3 - \frac{1}{8}.
\end{align*}

(13) 
\begin{align*}
\int_0^3\arcsin\sqrt{\frac{x}{1+x}}\,\mathrm{d}x &= \int_0^3 \arctan\sqrt{x}\,\mathrm{d}x \quad \text{(因为 } \arcsin\sqrt{\frac{x}{1+x}} = \arctan\sqrt{x} \text{)} \\
&\overset{u=\sqrt{x}}{=} 2\int_0^{\sqrt{3}} u\arctan u\,\mathrm{d}u \\
&= \left[u^2\arctan u\right]_0^{\sqrt{3}} - \int_0^{\sqrt{3}} \frac{u^2}{1+u^2}\,\mathrm{d}u \\
&= \pi - \int_0^{\sqrt{3}} \left(1 - \frac{1}{1+u^2}\right)\mathrm{d}u \\
&= \pi - \left[u - \arctan u\right]_0^{\sqrt{3}} \\
&= \pi - (\sqrt{3} - \frac{\pi}{3}) = \frac{4\pi}{3} - \sqrt{3}.
\end{align*}

(14) 
\begin{align*}
\int x\sqrt{x+3}\,\mathrm{d}x &\overset{u=x+3}{=} \int (u-3)\sqrt{u}\,\mathrm{d}u = \int (u^{3/2} - 3u^{1/2})\,\mathrm{d}u \\
&= \frac{2}{5}u^{5/2} - 2u^{3/2} + C = \frac{2}{5}(x+3)^{5/2} - 2(x+3)^{3/2} + C.
\end{align*}

(15) 
\begin{align*}
\int\frac{x}{\sqrt{5-x}}\,\mathrm{d}x &\overset{u=5-x}{=} \int \frac{5-u}{\sqrt{u}}(-\mathrm{d}u) = \int (u-5)u^{-1/2}\,\mathrm{d}u \\
&= \int (u^{1/2} - 5u^{-1/2})\,\mathrm{d}u = \frac{2}{3}u^{3/2} - 10u^{1/2} + C \\
&= \frac{2}{3}(5-x)^{3/2} - 10\sqrt{5-x} + C.
\end{align*}

(16) 
\begin{align*}
\int (t+2)\sqrt{2+3t}\,\mathrm{d}t &\overset{u=2+3t}{=} \int \left(\frac{u-2}{3}+2\right)\sqrt{u}\cdot\frac{1}{3}\,\mathrm{d}u = \frac{1}{9}\int (u+4)u^{1/2}\,\mathrm{d}u \\
&= \frac{1}{9}\int (u^{3/2} + 4u^{1/2})\,\mathrm{d}u = \frac{1}{9}\left(\frac{2}{5}u^{5/2} + \frac{8}{3}u^{3/2}\right) + C \\
&= \frac{2}{45}(2+3t)^{5/2} + \frac{8}{27}(2+3t)^{3/2} + C.
\end{align*}

(17) 
\begin{align*}
\int\frac{t+7}{\sqrt{5-t}}\,\mathrm{d}t &\overset{u=5-t}{=} \int \frac{5-u+7}{\sqrt{u}}(-\mathrm{d}u) = \int \frac{u-12}{\sqrt{u}}\,\mathrm{d}u \\
&= \int (u^{1/2} - 12u^{-1/2})\,\mathrm{d}u = \frac{2}{3}u^{3/2} - 24u^{1/2} + C \\
&= \frac{2}{3}(5-t)^{3/2} - 24\sqrt{5-t} + C.
\end{align*}
\end{solution}
\begin{example}{4.5-B-4}{}
    $\begin{aligned}&(1)\lim_{n\to\infty}\left(\frac{1}{n}+\frac{1}{n+1}+\cdots+\frac{1}{2n}\right);\quad(2)\lim_{n\to\infty}\left(\frac{1}{n^{2}}+\frac{2}{n^{2}}+\cdots+\frac{n-1}{n^{2}}\right);\\&(3)\lim_{n\to\infty}\frac{1}{n}\left(\sin\frac{\pi}{n}+\sin\frac{2\pi}{n}+\cdots+\sin\frac{(n-1)}{n}\pi\right).\end{aligned}$
\end{example}
\begin{solution}
    每道题利用定义,提取$\Delta x=\dfrac{1}{n},x_k=\dfrac{k}{n}$,然后找出来$f(x_k)$:\\
    (1)$\displaystyle\lim_{n\to\infty}\left(\dfrac{1}{n}+\sum_{k=1}^n\dfrac{1}{n+k}\right)=\lim_{n\to\infty}\dfrac1n\left(\dfrac1{1+\frac{k}{n}}\right)=\int_0^1\dfrac{1}{1+x}\dd x=\ln 2$\\
    (2)$\displaystyle\lim_{n\to\infty}\sum_{k=1}^{n-1}\dfrac{k}{n^{2}}=\lim_{n\to\infty}\sum_{k=1}^{n-1}\dfrac{1}{n}\dfrac{k}{n}=\int_0^1x\dd x=\frac{1}{2}$\\
    (3)$\displaystyle\lim_{n\to\infty}\dfrac{1}{n}\left(\sin\dfrac{\pi}{n}+\sin\dfrac{2\pi}{n}+\cdots+\sin\dfrac{(n-1)}{n}\pi\right)=\lim_{n\to\infty}\dfrac{1}{n}\left(\sum_{k=1}^{n-1}\sin\dfrac{k\pi}{n}\right)=\int_0^1\sin\pi x\dd x=\frac2{\pi}$
\end{solution}
\begin{example}{4.6-A-2}{}
(1)$\displaystyle\int\cos^4x\sin^3x\dd x$\quad(2)$\displaystyle\int\dfrac{\sin 2x}{1+\cos^2x}\dd x$\quad(3)$\displaystyle\int\dfrac{\dd x}{\sin 2x+2\sin x}$\quad(4)$\displaystyle\int\dfrac{\dd x}{2+\sin x}$
\end{example}
\begin{solution}
(1)\begin{align*}
\int\cos^4x\sin^3x\dd x&=-\int\sin^2x\cos^4x\dd\cos x=\int(\cos^2x-1)\cos^4x\dd\cos x\\
&=\int \cos^6x\dd\cos x-\int\cos^4 x\dd\cos x\\
&=\dfrac{\cos^7x}{7}-\dfrac{\cos^5x}{5}+C
\end{align*}
(2)\begin{align*}
    \int\dfrac{\sin 2x}{1+\cos^2x}\dd x&=\int\dfrac{2\sin x\cos x}{\sin^2 x+2\cos^2x}=\int\dfrac{2\tan x}{\tan^2 x+2}\dd x\\
    &=\int\dfrac{2t}{t^2+2}\dd\arctan t=\int\dfrac{\dd t^2}{(t^2+1)(t^2+2)}\\
    &=\int\left(\dfrac{1}{t^2+1}-\dfrac1{t^2+2}\right)\dd t^2\\
    &=\ln\dfrac{t^2+2}{t^2+1}+C=\ln\dfrac{\tan^2 x+2}{\tan^2 x+1}+C
\end{align*}
(3)
\begin{align*}
\int\dfrac{\dd x}{\sin 2x+2\sin x}&=\int\dfrac{\dd x}{2\sin x\cos x+2\sin x}=\int\dfrac{\dd x}{2\sin x(1+\cos x)}\\
&\overset{\text{令 }t=\tan\frac{x}{2}}{=}\int\dfrac{1}{2\cdot\frac{2t}{1+t^2}\left(1+\frac{1-t^2}{1+t^2}\right)}\cdot\dfrac{2}{1+t^2}\dd t\\
&=\int\dfrac{1}{2\cdot\frac{2t}{1+t^2}\cdot\frac{2}{1+t^2}}\cdot\dfrac{2}{1+t^2}\dd t=\int\dfrac{1+t^2}{4t}\dd t\\
&=\dfrac{1}{4}\int\left(\dfrac{1}{t}+t\right)\dd t=\dfrac{1}{4}\left(\ln|t|+\dfrac{t^2}{2}\right)+C\\
&=\dfrac{1}{4}\ln\left|\tan\dfrac{x}{2}\right|+\dfrac{1}{8}\tan^2\dfrac{x}{2}+C
\end{align*}
(4)
\begin{align*}
\int\dfrac{\dd x}{2+\sin x}&\overset{\text{令 }t=\tan\frac{x}{2}}{=}\int\dfrac{1}{2+\frac{2t}{1+t^2}}\cdot\dfrac{2}{1+t^2}\dd t\\
&=\int\dfrac{2}{2(1+t^2)+2t}\dd t=\int\dfrac{\dd t}{t^2+t+1}\\
&=\int\dfrac{\dd t}{\left(t+\frac{1}{2}\right)^2+\frac{3}{4}}=\dfrac{2}{\sqrt{3}}\arctan\left(\dfrac{t+\frac{1}{2}}{\frac{\sqrt{3}}{2}}\right)+C\\
&=\dfrac{2}{\sqrt{3}}\arctan\left(\dfrac{2t+1}{\sqrt{3}}\right)+C\\
&=\dfrac{2}{\sqrt{3}}\arctan\left(\dfrac{2\tan\frac{x}{2}+1}{\sqrt{3}}\right)+C
\end{align*}
\end{solution}
\begin{example}{4.6-A-3}{}
    (1)$\displaystyle\int\dfrac{\sqrt{x-1}}{\sqrt{x}}\dd x$\quad(2)$\displaystyle\int\dfrac{\dd x}{1+\sqrt[3]{x+2}}$
\end{example}
\begin{solution}
(1)
\begin{align*}
\int\dfrac{\sqrt{x-1}}{\sqrt{x}}\dd x
&\overset{t=\sqrt{x}}{=} \int\dfrac{\sqrt{t^2-1}}{t} \cdot 2t\dd t
= 2\int\sqrt{t^2-1}\dd t \\
&\overset{\sec\theta=t}{=} 2\int\tan\theta \cdot \sec\theta\tan\theta\dd\theta
= 2\int\tan^2\theta\sec\theta\dd\theta \\
&= 2\int(\sec^2\theta-1)\sec\theta\dd\theta
= 2\int(\sec^3\theta-\sec\theta)\dd\theta \\
&= 2\left( \frac{1}{2}\sec\theta\tan\theta + \frac{1}{2}\ln|\sec\theta+\tan\theta| - \ln|\sec\theta+\tan\theta| \right) + C \\
&= \sec\theta\tan\theta - \ln|\sec\theta+\tan\theta| + C \\
&= t\sqrt{t^2-1} - \ln|t+\sqrt{t^2-1}| + C \\
&= \sqrt{x}\sqrt{x-1} - \ln(\sqrt{x}+\sqrt{x-1}) + C
\end{align*}
(2)
\begin{align*}
\int\dfrac{\dd x}{1+\sqrt[3]{x+2}}
&\overset{t=\sqrt[3]{x+2}}{=} \int\dfrac{1}{1+t} \cdot 3t^2\dd t
= 3\int\dfrac{t^2}{1+t}\dd t \\
&= 3\int\left(t-1+\dfrac{1}{1+t}\right)\dd t \\
&= 3\left( \frac{1}{2}t^2 - t + \ln|1+t| \right) + C \\
&= \frac{3}{2}t^2 - 3t + 3\ln|1+t| + C \\
&= \frac{3}{2}(x+2)^{2/3} - 3(x+2)^{1/3} + 3\ln|1+(x+2)^{1/3}| + C
\end{align*}
\end{solution}
\begin{example}{4.6-B}{}
\begin{tabular}{@{} l l l @{}}
(1) $\displaystyle \int\frac{\mathrm{d}x}{x^4+1}$; & (2) $\displaystyle \int\frac{\mathrm{d}x}{x^4-1}$; & (3) $\displaystyle \int\frac{\mathrm{d}x}{x^4+x^2+1}$; \\[0.2ex]
(4) $\displaystyle \int\tan^3x\,\mathrm{d}x$; & (5) $\displaystyle \int\frac{\mathrm{d}x}{\sin^2x\cos x}$; & (6) $\displaystyle \int\frac{\mathrm{d}x}{3+\cos x}$; \\[0.2ex]
(7) $\displaystyle \int\frac{\mathrm{d}x}{(1+\sqrt[3]{x})\sqrt{x}}$; & (8) $\displaystyle \int\frac{\sqrt{x+1}-1}{\sqrt{x+1}+1}\mathrm{d}x$. & 
\end{tabular}
\end{example}
\begin{solution}
(1) 
\begin{align*}
\int \frac{\mathrm{d}x}{x^4+1} 
&= \int \frac{\mathrm{d}x}{(x^2+\sqrt{2}x+1)(x^2-\sqrt{2}x+1)} \\
&= \frac{1}{2\sqrt{2}} \int \left( \frac{x+\sqrt{2}}{x^2+\sqrt{2}x+1} - \frac{x-\sqrt{2}}{x^2-\sqrt{2}x+1} \right) \mathrm{d}x \\
&= \frac{1}{4\sqrt{2}} \int \frac{2x+\sqrt{2}}{x^2+\sqrt{2}x+1} \mathrm{d}x 
   + \frac{1}{4} \int \frac{1}{x^2+\sqrt{2}x+1} \mathrm{d}x \\
&\quad - \frac{1}{4\sqrt{2}} \int \frac{2x-\sqrt{2}}{x^2-\sqrt{2}x+1} \mathrm{d}x 
   + \frac{1}{4} \int \frac{1}{x^2-\sqrt{2}x+1} \mathrm{d}x \\
&= \frac{1}{4\sqrt{2}} \ln\left(x^2+\sqrt{2}x+1\right) 
   + \frac{1}{2\sqrt{2}} \arctan\left(\sqrt{2}x+1\right) \\
&\quad - \frac{1}{4\sqrt{2}} \ln\left(x^2-\sqrt{2}x+1\right) 
   + \frac{1}{2\sqrt{2}} \arctan\left(\sqrt{2}x-1\right) + C \\
&= \frac{1}{4\sqrt{2}} \ln \frac{x^2+\sqrt{2}x+1}{x^2-\sqrt{2}x+1} 
   + \frac{1}{2\sqrt{2}} \left[ \arctan(\sqrt{2}x+1) + \arctan(\sqrt{2}x-1) \right] + C.
\end{align*}

(2) 
\begin{align*}
\int \frac{\mathrm{d}x}{x^4-1} 
&= \int \frac{\mathrm{d}x}{(x-1)(x+1)(x^2+1)} \\
&= \frac{1}{4} \int \left( \frac{1}{x-1} - \frac{1}{x+1} - \frac{2}{x^2+1} \right) \mathrm{d}x \\
&= \frac{1}{4} \ln|x-1| - \frac{1}{4} \ln|x+1| - \frac{1}{2} \arctan x + C \\
&= \frac{1}{4} \ln \left| \frac{x-1}{x+1} \right| - \frac{1}{2} \arctan x + C.
\end{align*}

(3) 
\begin{align*}
\int \frac{\mathrm{d}x}{x^4+x^2+1} 
&= \int \frac{\mathrm{d}x}{(x^2+x+1)(x^2-x+1)} \\
&= \frac{1}{2} \int \left( \frac{1}{x^2+x+1} + \frac{1}{x^2-x+1} \right) \mathrm{d}x \\
&= \frac{1}{2} \int \frac{1}{\left(x+\frac{1}{2}\right)^2 + \left(\frac{\sqrt{3}}{2}\right)^2} \mathrm{d}x 
   + \frac{1}{2} \int \frac{1}{\left(x-\frac{1}{2}\right)^2 + \left(\frac{\sqrt{3}}{2}\right)^2} \mathrm{d}x \\
&= \frac{1}{\sqrt{3}} \left[ \arctan \left( \frac{2x+1}{\sqrt{3}} \right) + \arctan \left( \frac{2x-1}{\sqrt{3}} \right) \right] + C \\
&= \frac{1}{\sqrt{3}} \arctan \left( \frac{x\sqrt{3}}{1-x^2} \right) + C \quad (\text{利用恒等式 } \arctan u + \arctan v = \arctan \frac{u+v}{1-uv} ).
\end{align*}

(4) 
\begin{align*}
\int \tan^3 x \, \mathrm{d}x 
&= \int \tan x \cdot \tan^2 x \, \mathrm{d}x 
 = \int \tan x (\sec^2 x - 1) \, \mathrm{d}x \\
&= \int \tan x \sec^2 x \, \mathrm{d}x - \int \tan x \, \mathrm{d}x \\
&= \frac{1}{2} \tan^2 x - (-\ln|\cos x|) + C \\
&= \frac{1}{2} \tan^2 x + \ln|\cos x| + C.
\end{align*}

(5) 
\begin{align*}
\int \frac{\mathrm{d}x}{\sin^2 x \cos x} 
&= \int \frac{\sin^2 x + \cos^2 x}{\sin^2 x \cos x} \, \mathrm{d}x 
 = \int \frac{\sin^2 x}{\sin^2 x \cos x} \, \mathrm{d}x + \int \frac{\cos^2 x}{\sin^2 x \cos x} \, \mathrm{d}x \\
&= \int \frac{1}{\cos x} \, \mathrm{d}x + \int \frac{\cos x}{\sin^2 x} \, \mathrm{d}x \\
&= \int \sec x \, \mathrm{d}x + \int \cot x \csc x \, \mathrm{d}x \\
&= \ln|\sec x + \tan x| - \csc x + C.
\end{align*}

(6) 
\begin{align*}
\int \frac{\mathrm{d}x}{3+\cos x} 
&\overset{t = \tan \frac{x}{2}}{=} \int \frac{1}{3 + \frac{1-t^2}{1+t^2}} \cdot \frac{2}{1+t^2} \, \mathrm{d}t 
 = \int \frac{2}{3(1+t^2) + (1-t^2)} \, \mathrm{d}t \\
&= \int \frac{2}{2t^2+4} \, \mathrm{d}t 
 = \int \frac{1}{t^2+2} \, \mathrm{d}t \\
&= \frac{1}{\sqrt{2}} \arctan \left( \frac{t}{\sqrt{2}} \right) + C 
 = \frac{1}{\sqrt{2}} \arctan \left( \frac{\tan \frac{x}{2}}{\sqrt{2}} \right) + C.
\end{align*}

(7) 
\begin{align*}
\int \frac{\mathrm{d}x}{(1+\sqrt[3]{x})\sqrt{x}} 
&\overset{t = \sqrt[6]{x}}{=} \int \frac{1}{(1+t^2) \cdot t^3} \cdot 6t^5 \, \mathrm{d}t 
 = 6 \int \frac{t^2}{1+t^2} \, \mathrm{d}t \\
&= 6 \int \left( 1 - \frac{1}{1+t^2} \right) \mathrm{d}t 
 = 6 \left( t - \arctan t \right) + C \\
&= 6 \left( \sqrt[6]{x} - \arctan \sqrt[6]{x} \right) + C.
\end{align*}

(8) 
\begin{align*}
\int \frac{\sqrt{x+1}-1}{\sqrt{x+1}+1} \, \mathrm{d}x 
&= \int \left( 1 - \frac{2}{\sqrt{x+1}+1} \right) \mathrm{d}x 
 = \int 1 \, \mathrm{d}x - 2 \int \frac{1}{\sqrt{x+1}+1} \, \mathrm{d}x \\
&\overset{t = \sqrt{x+1}}{=} x - 2 \int \frac{1}{t+1} \cdot 2t \, \mathrm{d}t 
 = x - 4 \int \frac{t}{t+1} \, \mathrm{d}t \\
&= x - 4 \int \left( 1 - \frac{1}{t+1} \right) \mathrm{d}t \\
&= x - 4 \left( t - \ln|t+1| \right) + C \\
&= x - 4 \sqrt{x+1} + 4 \ln \left( \sqrt{x+1} + 1 \right) + C.
\end{align*}
\end{solution}
\newpage
\begin{example}{4.7-A}{}
    用定义判别下列反常积分的敛散性,如果积分收敛,则计算反常积分的值\\
\begin{tabular}{@{} l l l @{}}
(1) $\displaystyle \int_{1}^{+\infty} \mathrm{e}^{-2x} \, \mathrm{d}x$; & (2) $\displaystyle \int_{1}^{+\infty} \frac{x}{4+x^{2}} \, \mathrm{d}x$; & (3) $\displaystyle \int_{0}^{+\infty} \frac{x}{\mathrm{e}^{x}} \, \mathrm{d}x$; \\
(4) $\displaystyle \int_{-\infty}^{0} \frac{\mathrm{e}^x}{1+\mathrm{e}^x} \, \mathrm{d}x$; & (5) $\displaystyle \int_{x}^{+\infty} \sin y \, \mathrm{d}y$; & (6) $\displaystyle \int_{-\infty}^{+\infty} \frac{\mathrm{d}z}{z^{2}+25}$; \\
(7) $\displaystyle \int_{\pi/4}^{\pi/2} \frac{\sin x}{\sqrt{\cos x}} \, \mathrm{d}x$; & (8) $\displaystyle \int_{0}^{4} \frac{\mathrm{d}x}{\sqrt{16-x^{2}}}$; & (9) $\displaystyle \int_{-1}^{1} \frac{\mathrm{d}t}{t}$; \\
(10) $\displaystyle \int_1^{+\infty} \frac{\mathrm{d}x}{\sqrt{x^2+1}}$; & (11) $\displaystyle \int_0^1 \frac{x^4+1}{x} \, \mathrm{d}x$; & (12) $\displaystyle \int_4^{20} \frac{1}{y^2-16} \, \mathrm{d}y$; \\
(13) $\displaystyle \int_0^1 \frac{\ln x}{x} \, \mathrm{d}x$; & (14) $\displaystyle \int_{2}^{+\infty} \frac{\mathrm{d}x}{x\ln x}$; & (15) $\displaystyle \int_0^\pi \frac{1}{\sqrt{x}} \mathrm{e}^{-\sqrt{x}} \, \mathrm{d}x$; \\
(16) $\displaystyle \int_{3}^{+\infty} \frac{\mathrm{d}x}{x(\ln x)^2}$; & (17) $\displaystyle \int_1^2 \frac{\mathrm{d}x}{x\ln x}$; & (18) $\displaystyle \int_{1}^{+\infty} \frac{\ln\pi}{x^{2}} \, \mathrm{d}x$; \\
(19) $\displaystyle \int_{0}^{+\infty} \mathrm{e}^{-ax} \sin bx \, \mathrm{d}x, \; a>0$; & (20) $\displaystyle \int_{0}^{+\infty} \frac{x\ln x}{(1+x^{2})^{2}} \, \mathrm{d}x$. & \\
\end{tabular}
\end{example}
\begin{solution}
(1) 
\begin{align*}
\int_{1}^{+\infty} \mathrm{e}^{-2x} \, \mathrm{d}x &= \lim_{b \to +\infty} \int_{1}^{b} \mathrm{e}^{-2x} \, \mathrm{d}x = \lim_{b \to +\infty} \left( -\frac{1}{2} \mathrm{e}^{-2x} \Big|_{1}^{b} \right) \\
&= \lim_{b \to +\infty} \left( -\frac{1}{2} \mathrm{e}^{-2b} + \frac{1}{2} \mathrm{e}^{-2} \right) = \frac{1}{2} \mathrm{e}^{-2} \quad \text{(收敛)}.
\end{align*}

(2) 
\begin{align*}
\int_{1}^{+\infty} \frac{x}{4+x^{2}} \, \mathrm{d}x &= \lim_{b \to +\infty} \int_{1}^{b} \frac{x}{4+x^{2}} \, \mathrm{d}x = \lim_{b \to +\infty} \left( \frac{1}{2} \ln(4+x^{2}) \Big|_{1}^{b} \right) \\
&= \lim_{b \to +\infty} \left( \frac{1}{2} \ln(4+b^{2}) - \frac{1}{2} \ln 5 \right) = +\infty \quad \text{(发散)}.
\end{align*}

(3) 
\begin{align*}
\int_{0}^{+\infty} \frac{x}{\mathrm{e}^{x}} \, \mathrm{d}x &= \lim_{b \to +\infty} \int_{0}^{b} x \mathrm{e}^{-x} \, \mathrm{d}x = \lim_{b \to +\infty} \left( -x \mathrm{e}^{-x} - \mathrm{e}^{-x} \Big|_{0}^{b} \right) \\
&= \lim_{b \to +\infty} \left( -b \mathrm{e}^{-b} - \mathrm{e}^{-b} + 1 \right) = 1 \quad \text{(收敛)}.
\end{align*}

(4) 
\begin{align*}
\int_{-\infty}^{0} \frac{\mathrm{e}^x}{1+\mathrm{e}^x} \, \mathrm{d}x &= \lim_{a \to -\infty} \int_{a}^{0} \frac{\mathrm{e}^x}{1+\mathrm{e}^x} \, \mathrm{d}x = \lim_{a \to -\infty} \left( \ln(1+\mathrm{e}^x) \Big|_{a}^{0} \right) \\
&= \ln 2 - \lim_{a \to -\infty} \ln(1+\mathrm{e}^a) = \ln 2 \quad \text{(收敛)}.
\end{align*}

(5) 
\begin{align*}
\int_{x}^{+\infty} \sin y \, \mathrm{d}y &= \lim_{b \to +\infty} \int_{x}^{b} \sin y \, \mathrm{d}y = \lim_{b \to +\infty} \left( -\cos y \Big|_{x}^{b} \right) \\
&= \lim_{b \to +\infty} (-\cos b + \cos x) \quad \text{极限不存在,故发散}.
\end{align*}

(6) 
\begin{align*}
\int_{-\infty}^{+\infty} \frac{\mathrm{d}z}{z^{2}+25} &= \lim_{a \to -\infty} \int_{a}^{0} \frac{\mathrm{d}z}{z^{2}+25} + \lim_{b \to +\infty} \int_{0}^{b} \frac{\mathrm{d}z}{z^{2}+25} \\
&= \lim_{a \to -\infty} \left( \frac{1}{5} \arctan \frac{z}{5} \Big|_{a}^{0} \right) + \lim_{b \to +\infty} \left( \frac{1}{5} \arctan \frac{z}{5} \Big|_{0}^{b} \right) \\
&= \frac{1}{5} \cdot \frac{\pi}{2} - \lim_{a \to -\infty} \frac{1}{5} \arctan \frac{a}{5} + \lim_{b \to +\infty} \frac{1}{5} \arctan \frac{b}{5} - \frac{1}{5} \cdot 0 \\
&= \frac{\pi}{10} + \frac{\pi}{10} = \frac{\pi}{5} \quad \text{(收敛)}.
\end{align*}

(7) 
\begin{align*}
\int_{\pi/4}^{\pi/2} \frac{\sin x}{\sqrt{\cos x}} \, \mathrm{d}x &\xlongequal{u = \cos x} \int_{\sqrt{2}/2}^{0} \frac{-\mathrm{d}u}{\sqrt{u}} = \int_{0}^{\sqrt{2}/2} u^{-1/2} \, \mathrm{d}u \\
&= 2\sqrt{u} \Big|_{0}^{\sqrt{2}/2} = 2\sqrt{\frac{\sqrt{2}}{2}} = 2^{3/4} \quad \text{(收敛)}.
\end{align*}

(8) 
\begin{align*}
\int_{0}^{4} \frac{\mathrm{d}x}{\sqrt{16-x^{2}}} &= \lim_{b \to 4^{-}} \int_{0}^{b} \frac{\mathrm{d}x}{\sqrt{16-x^{2}}} = \lim_{b \to 4^{-}} \left( \arcsin \frac{x}{4} \Big|_{0}^{b} \right) \\
&= \lim_{b \to 4^{-}} \arcsin \frac{b}{4} = \arcsin 1 = \frac{\pi}{2} \quad \text{(收敛)}.
\end{align*}

(9) 
\begin{align*}
\int_{-1}^{1} \frac{\mathrm{d}t}{t} &= \lim_{a \to 0^{-}} \int_{-1}^{a} \frac{\mathrm{d}t}{t} + \lim_{b \to 0^{+}} \int_{b}^{1} \frac{\mathrm{d}t}{t} \\
&= \lim_{a \to 0^{-}} \left( \ln|t| \Big|_{-1}^{a} \right) + \lim_{b \to 0^{+}} \left( \ln|t| \Big|_{b}^{1} \right) \\
&= \lim_{a \to 0^{-}} (\ln|a| - \ln 1) + \lim_{b \to 0^{+}} (\ln 1 - \ln|b|) \\
&= \lim_{a \to 0^{-}} \ln|a| - \lim_{b \to 0^{+}} \ln|b| = -\infty - (-\infty) \quad \text{不存在,故发散}.
\end{align*}

(10) 
\begin{align*}
\int_1^{+\infty} \frac{\mathrm{d}x}{\sqrt{x^2+1}} &> \int_1^{+\infty} \frac{\mathrm{d}x}{\sqrt{x^2+x^2}} = \int_1^{+\infty} \frac{\mathrm{d}x}{\sqrt{2}x} = \frac{1}{\sqrt{2}} \int_1^{+\infty} \frac{\mathrm{d}x}{x} = +\infty \quad \text{(发散)}.
\end{align*}

(11) 
\begin{align*}
\int_0^1 \frac{x^4+1}{x} \, \mathrm{d}x &= \int_0^1 \left( x^3 + \frac{1}{x} \right) \mathrm{d}x = \lim_{a \to 0^{+}} \int_{a}^{1} \left( x^3 + \frac{1}{x} \right) \mathrm{d}x \\
&= \lim_{a \to 0^{+}} \left( \frac{x^4}{4} + \ln x \Big|_{a}^{1} \right) = \left( \frac{1}{4} + 0 \right) - \lim_{a \to 0^{+}} \left( \frac{a^4}{4} + \ln a \right) = +\infty \quad \text{(发散)}.
\end{align*}

(12) 
\begin{align*}
\int_4^{20} \frac{1}{y^2-16} \, \mathrm{d}y &= \lim_{a \to 4^{+}} \int_{a}^{20} \frac{1}{(y-4)(y+4)} \, \mathrm{d}y \\
&= \lim_{a \to 4^{+}} \frac{1}{8} \left( \ln\left| \frac{y-4}{y+4} \right| \Big|_{a}^{20} \right) \\
&= \frac{1}{8} \left( \ln \frac{16}{24} - \lim_{a \to 4^{+}} \ln \frac{a-4}{a+4} \right) = +\infty \quad \text{(发散)}.
\end{align*}

(13) 
\begin{align*}
\int_0^1 \frac{\ln x}{x} \, \mathrm{d}x &= \lim_{a \to 0^{+}} \int_{a}^{1} \frac{\ln x}{x} \, \mathrm{d}x = \lim_{a \to 0^{+}} \left( \frac{1}{2} (\ln x)^2 \Big|_{a}^{1} \right) \\
&= \lim_{a \to 0^{+}} \left( 0 - \frac{1}{2} (\ln a)^2 \right) = -\infty \quad \text{(发散)}.
\end{align*}

(14) 
\begin{align*}
\int_{2}^{+\infty} \frac{\mathrm{d}x}{x\ln x} &= \lim_{b \to +\infty} \int_{2}^{b} \frac{\mathrm{d}x}{x\ln x} = \lim_{b \to +\infty} \left( \ln(\ln x) \Big|_{2}^{b} \right) \\
&= \lim_{b \to +\infty} \left( \ln(\ln b) - \ln(\ln 2) \right) = +\infty \quad \text{(发散)}.
\end{align*}

(15) 
\begin{align*}
\int_0^\pi \frac{1}{\sqrt{x}} \mathrm{e}^{-\sqrt{x}} \, \mathrm{d}x &\xlongequal{t = \sqrt{x}} \int_{0}^{\sqrt{\pi}} \frac{1}{t} \mathrm{e}^{-t} \cdot 2t \, \mathrm{d}t = 2 \int_{0}^{\sqrt{\pi}} \mathrm{e}^{-t} \, \mathrm{d}t \\
&= 2 \left( -\mathrm{e}^{-t} \Big|_{0}^{\sqrt{\pi}} \right) = 2 \left( 1 - \mathrm{e}^{-\sqrt{\pi}} \right) \quad \text{(收敛)}.
\end{align*}

(16) 
\begin{align*}
\int_{3}^{+\infty} \frac{\mathrm{d}x}{x(\ln x)^2} &= \lim_{b \to +\infty} \int_{3}^{b} \frac{\mathrm{d}x}{x(\ln x)^2} = \lim_{b \to +\infty} \left( -\frac{1}{\ln x} \Big|_{3}^{b} \right) \\
&= \lim_{b \to +\infty} \left( -\frac{1}{\ln b} + \frac{1}{\ln 3} \right) = \frac{1}{\ln 3} \quad \text{(收敛)}.
\end{align*}

(17) 
\begin{align*}
\int_1^2 \frac{\mathrm{d}x}{x\ln x} &= \lim_{a \to 1^{+}} \int_{a}^{2} \frac{\mathrm{d}x}{x\ln x} = \lim_{a \to 1^{+}} \left( \ln|\ln x| \Big|_{a}^{2} \right) \\
&= \ln(\ln 2) - \lim_{a \to 1^{+}} \ln|\ln a| = -\infty \quad \text{(发散)}.
\end{align*}

(18) 
\begin{align*}
\int_{1}^{+\infty} \frac{\ln\pi}{x^{2}} \, \mathrm{d}x &= \ln\pi \lim_{b \to +\infty} \int_{1}^{b} x^{-2} \, \mathrm{d}x = \ln\pi \lim_{b \to +\infty} \left( -x^{-1} \Big|_{1}^{b} \right) \\
&= \ln\pi \lim_{b \to +\infty} \left( -\frac{1}{b} + 1 \right) = \ln\pi \quad \text{(收敛)}.
\end{align*}

(19) 
\begin{align*}
\int_{0}^{+\infty} \mathrm{e}^{-ax} \sin bx \, \mathrm{d}x &= \lim_{B \to +\infty} \int_{0}^{B} \mathrm{e}^{-ax} \sin bx \, \mathrm{d}x \\
&= \lim_{B \to +\infty} \left( \frac{\mathrm{e}^{-ax}(-a \sin bx - b \cos bx)}{a^2+b^2} \Big|_{0}^{B} \right) \\
&= 0 - \left( \frac{-b}{a^2+b^2} \right) = \frac{b}{a^2+b^2} \quad \text{(收敛)}.
\end{align*}

(20) 
\begin{align*}
\int_{0}^{+\infty} \frac{x\ln x}{(1+x^{2})^{2}} \, \mathrm{d}x &\xlongequal{u = \frac{1}{x}} \int_{+\infty}^{0} \frac{(1/u) \ln(1/u)}{(1+1/u^2)^2} \left( -\frac{1}{u^2} \right) \mathrm{d}u \\
&= \int_{0}^{+\infty} \frac{-\ln u}{u(1+1/u^2)^2} \cdot \frac{1}{u^2} \, \mathrm{d}u \\
&= \int_{0}^{+\infty} \frac{-\ln u}{u^3 \cdot \frac{(u^2+1)^2}{u^4}} \, \mathrm{d}u = \int_{0}^{+\infty} \frac{-u \ln u}{(u^2+1)^2} \, \mathrm{d}u \\
&= - \int_{0}^{+\infty} \frac{u \ln u}{(u^2+1)^2} \, \mathrm{d}u = -I.
\end{align*}
所以 $2I = 0$,即 $I = 0$,故积分收敛,值为0。
\end{solution}



\begin{example}{4.7-B-1}{}
已知$\displaystyle \int_{-\infty}^{\infty}\e^{-x^2}\dd x=\sqrt{\pi}$,若$\displaystyle \int_{-\infty}^{+\infty}A\e^{-x^2-x}\dd x=1$,求$A$。
\end{example}
\begin{solution}
\begin{align*}
    1=\int_{-\infty}^{+\infty}A\e^{-x^2-x}\dd x&=\int_{-\infty}^{+\infty}\e^{-(x+\frac12)^2+\frac14+\ln A}\dd x  \\
    &=\e^{\frac14+\ln A}\int_{-\infty}^{+\infty}\e^{-(x+\frac12)^2}\dd x  \\
    &=\e^{\frac14+\ln A}\int_{-\infty}^{+\infty}\e^{-(x+\frac12)^2}\dd (x+\dfrac12)=\e^{\frac14+\ln A}\sqrt{\pi} \\
\Rightarrow A&=\dfrac{1}{\e^{\frac14}\sqrt{\pi}}=\e^{-\frac14}\pi^{-\frac12}
\end{align*}
\end{solution}
\begin{example}{4.7-B-2}{}
\[\int_{\frac12}^{\frac32}\dfrac{\dd x}{\sqrt{|x-x^2|}}\]
\end{example}
\begin{solution}
\begin{align*}
\int_{\frac12}^{\frac32}\dfrac{\dd x}{\sqrt{|x-x^2|}}&=\int_{\frac12}^1\dfrac{\dd x}{\sqrt{x-x^2}}+\int_1^{\frac32}\dfrac{\dd x}{\sqrt{x^2-x}}\\
&=\int_{\frac{\pi}{4}}^{\frac{\pi}{2}}\dfrac{\dd\sin^2\theta}{\sqrt{\sin^2\theta-\sin^4\theta}}+\int_{0}^{\arctan{\sqrt{\frac12}}}\dfrac{\dd\sec^2\theta}{\sqrt{\sec^4\theta-\sec^2\theta}}\\
&=\int_{\frac{\pi}{4}}^{\frac{\pi}{2}}\dfrac{2\sin\theta\cos\theta\dd\theta}{\sin\theta\cos\theta}+\int_{0}^{\arctan{\sqrt{\frac12}}}\dfrac{2\tan\theta\sec^2\theta\dd\theta}{\tan\theta\sec\theta}\\
&=\dfrac{\pi}{2}+2\int_{0}^{\arctan{\sqrt{\frac12}}}\sec\theta\dd\theta\\
&=\dfrac{\pi}{2}+\left.\ln\left(\sec\theta+\tan\theta\right)^2\right|_{0}^{\arctan{\sqrt{\frac12}}}\\
&=\dfrac{\pi}2+\ln(2+\sqrt3)
\end{align*}
\end{solution}
\begin{example}{4.7-B-3}{}
    \[\int_0^{+\infty}\dfrac{x\e^{-x}\dd x}{(1+\e^{-x})^2}\]
\end{example}
\begin{solution}
\begin{align*}
    \int_0^{+\infty}\dfrac{x\e^{-x}\dd x}{(1+\e^{-x})^2}&=\int_{1}^{+\infty}\dfrac{\frac1{x}\ln x\dd\ln x}{(1+\frac1{x})^2}=\int_{1}^{+\infty}\dfrac{\ln x}{(x+1)^2}\dd x\\
    &=-\int_{1}^{+\infty}\ln x\dd\dfrac1{1+x}=\left.-\dfrac{\ln x}{x+1}\right|_{1}^{+\infty}+\int_{1}^{+\infty}\frac1{x+1}\dd \ln x\\
    &=\int_{1}^{+\infty}\frac1{x+1}\dd \ln x=\int_{1}^{+\infty}\frac{\dd x}{x(x+1)}=0-\ln\dfrac12=\ln2
\end{align*}
\end{solution}
\begin{example}{4.7-B-4}{}
    \[\int_0^{+\infty}\dfrac{x^n}{\e^{x}}\dd x,\quad n=1,2,3,...\]
\end{example}
\begin{solution}
\begin{align*}
    I_n=\int_0^{+\infty}\dfrac{x^n}{\e^{x}}\dd x&=-\int_0^{+\infty}x^n\dd\e^{-x}\\
    &=-x^n\e^{-x}\Big|_0^{+\infty}+\int_0^{+\infty}\e^{-x}\dd x^n\\
    &=x\int_0^{+\infty}\e^{-x}x^{n-1}\dd x=nI_{n-1}\\
    I_0&=-\e^{-x}\Big|_0^{+\infty}=\e^{-x}\Big|_{+\infty}^0=1,I_1=1I_0=1\\
    \Rightarrow I_n&=nI_{n-1}=n!
\end{align*}
\end{solution}
\begin{example}{4.8-A-1}{}
    求下列曲线所围图形的面积:
$(1)y=x^{2}$与$y=2x+3;$\quad$(2)y=\sqrt{x}$与 $y=x;$\quad $(3)y^2=2x$与$x=5$ ;\\
$(4)y=x$与$y=x+\sin^2x,(0\leqslant x\leqslant\pi)$\quad(5) $x^2+ 9y^2= 1$ ;\quad$(6)y^2=1+2x-x^2$与$x^2+y^2=1.$
\end{example}
\begin{solution}
(1)联立$y=x^2$和$y=2x+3$,得$x=-1,x=3$,则所围成的面积为
\begin{align*}
\int_{-1}^{3}(2x+3-x^2)\dd x&=x^2\Big|_{-1}^{3}+3x\Big|_{-1}^{3}-\dfrac{x^3}{3}\Big|_{-1}^{3}\\
&=9-1+(9-(-3))-\frac13(27+1)=\frac{32}{3}
\end{align*}
(2)联立$y=\sqrt{x}$和$y=x$,得$x=0,x=1$,则所围成的面积为
\begin{align*}
    \int_0^1(\sqrt{x}-x)\dd x&=\dfrac23x^{\frac23}\Big|_0^1-\dfrac12x^2\Big|_0^1\\
    &=\dfrac23-\dfrac12=\dfrac16
\end{align*}
(3)联立$y^2=2x$和$x=5$,得则所围成的面积为
\begin{align*}
2\int_0^5\sqrt{2x}\dd x&=2\sqrt2\int_0^5\sqrt{x}\dd x=2\sqrt2\cdot\dfrac23x^{\frac32}\Big|_0^5\\
&=\dfrac{4\sqrt2}{3}\cdot 5\sqrt5=\dfrac{20\sqrt{10}}{3}
\end{align*}
(4)则所围成的面积为
\begin{align*}
\int_0^{\pi}\sin^2 x\dd x&=\int_0^{\pi}\dfrac{1-\cos 2x}{2}\dd x=\dfrac12 x\Big|_0^{\pi}-\dfrac{\sin 2x}{4}\Big|_0^{\pi}\\
&=\dfrac{\pi}2
\end{align*}
(5)椭圆面积为
\begin{align*}
    2\int_{-1}^{1}\dfrac13\sqrt{1-x^2}\dd x&=\dfrac23\int_{-\frac{\pi}{2}}^{\frac{\pi}{2}}\sqrt{1-\sin^2\theta}\dd\sin\theta=\dfrac23\int_{-\frac{\pi}{2}}^{\frac{\pi}{2}}\cos^2 x\dd x\\
    &=\dfrac13\int_{-\frac{\pi}{2}}^{\frac{\pi}{2}}(1+\cos 2x)\dd x=\dfrac{\pi}3+ \dfrac{\sin 2x}{4}\Big|_{-\frac{\pi}{2}}^{\frac{\pi}{2}}=\dfrac{\pi}3\\
\end{align*}
(6)等效于一个半圆加一个拱形的面积
\begin{align*}
S=\dfrac{\pi}2+\dfrac{1}2\pi-1=\pi-1
\end{align*}
\end{solution}
\begin{example}{4.8-A-2}{}
    求下列极坐标表示的图形围成的面积:(1)$r=2a\cos\theta$;(2)$r=3\cos\theta$和$r=1+\cos\theta$围成的公共部分的面积
\end{example}
\begin{solution}
(1)曲线 $r=2a\cos\theta$ 表示一个圆,圆心在 $(a,0)$,半径为 $|a|$。其面积为
\begin{align*}
A &= \frac{1}{2} \int_{-\pi/2}^{\pi/2} \bigl(2a\cos\theta\bigr)^2 \,\mathrm{d}\theta 
    = 2a^2 \int_{-\pi/2}^{\pi/2} \cos^2\theta \,\mathrm{d}\theta 
    = 2a^2 \int_{-\pi/2}^{\pi/2} \frac{1+\cos2\theta}{2} \,\mathrm{d}\theta \\
  &= a^2 \int_{-\pi/2}^{\pi/2} (1+\cos2\theta) \,\mathrm{d}\theta 
    = a^2 \Bigl( \theta + \frac{1}{2}\sin2\theta \Bigr) \Big|_{-\pi/2}^{\pi/2} 
    = a^2 \bigl( \pi/2 - (-\pi/2) \bigr) 
    = \pi a^2.
\end{align*}

(2)先求两曲线的交点:由 $3\cos\theta = 1+\cos\theta$ 得 $\cos\theta = \frac{1}{2}$,故 $\theta = \pm\frac{\pi}{3}$。
由于图形关于极轴对称,只需计算上半部分再乘以 $2$。在上半平面,当 $\theta\in[0,\pi/3]$ 时,$1+\cos\theta \le 3\cos\theta$,
公共部分的边界为 $r=1+\cos\theta$;当 $\theta\in[\pi/3,\pi/2]$ 时,$3\cos\theta \le 1+\cos\theta$,
公共部分的边界为 $r=3\cos\theta$。于是上半部分的面积为
\begin{align*}
A_{\text{上半}} &= \frac{1}{2} \int_{0}^{\pi/3} \bigl(1+\cos\theta\bigr)^2 \,\mathrm{d}\theta 
                + \frac{1}{2} \int_{\pi/3}^{\pi/2} \bigl(3\cos\theta\bigr)^2 \,\mathrm{d}\theta.
\end{align*}
计算第一个积分:
\begin{align*}
\frac{1}{2} \int_{0}^{\pi/3} (1+\cos\theta)^2 \,\mathrm{d}\theta
&= \frac{1}{2} \int_{0}^{\pi/3} \bigl(1 + 2\cos\theta + \cos^2\theta\bigr) \,\mathrm{d}\theta \\
&= \frac{1}{2} \int_{0}^{\pi/3} \Bigl( \frac{3}{2} + 2\cos\theta + \frac{1}{2}\cos2\theta \Bigr) \,\mathrm{d}\theta \\
&= \frac{1}{2} \Bigl( \frac{3}{2}\theta + 2\sin\theta + \frac{1}{4}\sin2\theta \Bigr) \Big|_{0}^{\pi/3} \\
&= \frac{1}{2} \Bigl( \frac{\pi}{2} + \sqrt{3} + \frac{\sqrt{3}}{8} \Bigr)
 = \frac{\pi}{4} + \frac{9\sqrt{3}}{16}.
\end{align*}
计算第二个积分:
\begin{align*}
\frac{1}{2} \int_{\pi/3}^{\pi/2} 9\cos^2\theta \,\mathrm{d}\theta
&= \frac{9}{2} \int_{\pi/3}^{\pi/2} \frac{1+\cos2\theta}{2} \,\mathrm{d}\theta
 = \frac{9}{4} \int_{\pi/3}^{\pi/2} (1+\cos2\theta) \,\mathrm{d}\theta \\
&= \frac{9}{4} \Bigl( \theta + \frac{1}{2}\sin2\theta \Bigr) \Big|_{\pi/3}^{\pi/2} \\
&= \frac{9}{4} \Bigl( \frac{\pi}{2} - \frac{\pi}{3} - \frac{1}{2}\cdot\frac{\sqrt{3}}{2} \Bigr)
 = \frac{9}{4} \Bigl( \frac{\pi}{6} - \frac{\sqrt{3}}{4} \Bigr)
 = \frac{3\pi}{8} - \frac{9\sqrt{3}}{16}.
\end{align*}
所以
\[
A_{\text{上半}} = \Bigl( \frac{\pi}{4} + \frac{9\sqrt{3}}{16} \Bigr) + \Bigl( \frac{3\pi}{8} - \frac{9\sqrt{3}}{16} \Bigr) = \frac{5\pi}{8},
\]
总面积
\[
A = 2 A_{\text{上半}} = \frac{5\pi}{4}.
\]
\end{solution}
\begin{example}{4.8-A-3}{}
    曲线 $y=(x-1)(x-2)$ 和 $x$ 轴围成一平面图形,求此平面图形绕 $y$ 轴旋转一周所成旋转体的体积.
\end{example}
\begin{solution}
曲线 $y=(x-1)(x-2)$ 与 $x$ 轴的交点为 $x=1$ 和 $x=2$。在区间 $[1,2]$ 上,$y\le 0$,故所求平面图形为 $x$ 轴下方,由曲线 $y=(x-1)(x-2)$ 与 $x$ 轴围成的区域。

将此区域绕 $y$ 轴旋转,采用柱壳法。在 $x$ 处取厚度为 $\mathrm{d}x$ 的竖直窄条,其高度为 $|y| = -(x-1)(x-2)$,旋转生成的柱壳半径为 $x$,故体积微元为
\[
\mathrm{d}V = 2\pi x \cdot |y| \, \mathrm{d}x = 2\pi x \cdot [-(x-1)(x-2)] \, \mathrm{d}x.
\]
于是旋转体的体积为
\begin{align*}
V &= \int_{1}^{2} 2\pi x \cdot [-(x-1)(x-2)] \, \mathrm{d}x \\
  &= 2\pi \int_{1}^{2} \bigl( -x^3 + 3x^2 - 2x \bigr) \, \mathrm{d}x \\
  &= 2\pi \left[ -\frac{x^4}{4} + x^3 - x^2 \right]_{1}^{2} \\
  &= 2\pi \left( \left( -\frac{16}{4} + 8 - 4 \right) - \left( -\frac{1}{4} + 1 - 1 \right) \right) \\
  &= 2\pi \left( 0 + \frac{1}{4} \right) \\
  &= 2\pi \cdot \frac{1}{4} = \frac{\pi}{2}.
\end{align*}
故所求旋转体的体积为 $\dfrac{\pi}{2}$。
\end{solution}
\begin{example}{4.8-B-4}{}
    $\text{求由曲线 }y=\sqrt{x}\text{及 }y=x^2\text{ 所围平面图形绕 }x\text{ 轴旋转所得旋转体的体积}.$
\end{example}
\begin{solution}
两条曲线的交点为 $y=\sqrt{x}$ 与 $y=x^2$ 的解:
\[
\sqrt{x}=x^2 \quad\Longrightarrow\quad x=x^4 \quad\Longrightarrow\quad x(x^3-1)=0,
\]
得 $x=0$ 或 $x=1$,对应 $y=0$ 和 $y=1$,即交点 $(0,0)$ 和 $(1,1)$。

在区间 $[0,1]$ 上,有 $\sqrt{x} \ge x^2$。绕 $x$ 轴旋转,用圆盘法(或称为“切片法”),体积微元为
\[
\mathrm{d}V = \pi\left[ (\sqrt{x})^2 - (x^2)^2 \right] \mathrm{d}x = \pi (x - x^4) \,\mathrm{d}x.
\]
积分得
\begin{align*}
V &= \pi \int_0^1 (x - x^4) \,\mathrm{d}x 
   = \pi \left( \frac{x^2}{2} - \frac{x^5}{5} \right) \Big|_{0}^{1} \\
  &= \pi \left( \frac{1}{2} - \frac{1}{5} \right) 
   = \pi \cdot \frac{3}{10} = \frac{3\pi}{10}.
\end{align*}
故所求旋转体的体积为 $\dfrac{3\pi}{10}$。
\end{solution}
\begin{example}{4.8-A-5}{}
求曲线$y=\ln(-x^2+1)$上相应于$0\leqslant x\leqslant \dfrac12$的弧长
\end{example}
\begin{solution}
\begin{align*}
L=\int_0^{\frac12}\sqrt{1+\left(\dfrac{2x}{x^2-1}\right)^2}\dd x&=\int_0^{\frac12}\sqrt{\dfrac{(x^2+1)^2}{(x^2-1)^2}}\dd x=\int_0^{\frac12}\dfrac{1+x^2}{1-x^2}\dd x\\
&=\int_0^{\frac12}\left(\dfrac{-1+x^2+2}{1-x^2}\right)\dd x=\int_0^{\frac12}(-1+\dfrac{2}{1-x^2})\dd x\\
&=\int_0^{\frac12}\left(-1-\dfrac{1}{x-1}+\dfrac{1}{1+x}\right)\dd x\\
&=-x\Big|_0^{\frac12}-\ln|x-1|\Big|_0^{\frac12}+\ln|1+x|\Big|_0^{\frac12}\\
&=\ln 2-\frac12+\ln\frac32=\ln 3-\frac12
\end{align*}
\end{solution}
\begin{example}{4.8-A-6}{}
    求曲线$y=\int_0^{\frac{x}{n}}n\sqrt{\sin\theta}\dd\theta$的全长,其中$0\leqslant x\leqslant n\pi$
\end{example}
\begin{solution}
    \begin{align*}
    L&=\int_0^{n\pi}\sqrt{1+\left(\dfrac{\dd}{\dd x}\int_0^{\frac{x}{n}}n\sqrt{\sin\theta}\dd\theta\right)^2}=\int_0^{n\pi}\sqrt{1+\left(n\cdot\sqrt{\sin\frac{x}{n}}\frac1n\right)^2}\dd x\\
    &=\int_0^{n\pi}\sqrt{1+\sin\frac{x}{n}}\dd x=n\int_0^{n\pi}\sqrt{1+\sin\frac{x}{n}}\dd \frac{x}{n}=n\int_0^{\pi}\sqrt{1+\sin x}\dd x\\
    &=n\int_0^{\pi}\left(\sin\frac{x}{2}+\cos\frac{x}{2}\right)\dd x=n\left(-2\cos\frac{x}{2}\Big|_0^{\pi}+2\sin\frac{x}{2}\Big|_0^{\pi}\right)=4n\\
    \end{align*}
\end{solution}
\begin{example}{4.8-A-8}{}
    求心脏线$r=a(1+\cos\theta)$的全长n
\end{example}
\begin{solution}
    \begin{align*}
        S&=\int_0^{2\pi}r^2(\theta)\dd\theta=a^2\int_0^{2\pi}(1+\cos\theta)^2\dd\theta\\
        &=a^2\int_0^{2\pi}(1+2\cos\theta+\cos^2\theta)\dd\theta=a^2\int_0^{\pi}(\frac32+2\cos\theta+\frac12\cos 2\theta)\dd\theta\\
        &=a^2(\frac32\theta+2\sin\theta+\frac14\sin2\theta)\Big|_0^{2\pi}=3\pi a^2
    \end{align*}    
\end{solution}
\begin{example}{4.8-B-2}{}
    一立体的底面为一半径为 $R$ 的圆盘,其垂直于 $x$ 轴的截面是一等边三角形,求这个立体的体积.
\end{example}
\begin{solution}
\begin{align*}
V&=2\int_0^{R}\frac{\sqrt3}4 (\sqrt{R^2-x^2})^2\dd x=2\sqrt3\int_0^{R}(R^2-x^2)\dd x\\&=2\sqrt3(R^2x-\frac{x^3}{3})\Big|_0^{R}=\dfrac43\sqrt3R^3\\
\end{align*}
\end{solution}
\begin{example}{4.8-B-6}{}
    求下列平面曲线绕指定轴旋转所得旋转体的侧面积:\\
(1) $y= \sin x, 0\leqslant x\leqslant \pi , \textbf{绕  }x$轴;\\
(2) $x= a( t- \sin t)$ $, y= a( 1- \cos t)$ $, 0\leqslant t\leqslant 2\pi$,绕直线 $y=2a$\\
(3) $r= a( 1+ \cos \theta )$ $, 0\leq \theta \leq 2\pi$,绕极轴。
\end{example}
\begin{solution}
(1) 曲线 $y=\sin x$, $0\leqslant x\leqslant \pi$, 绕 $x$ 轴旋转所得旋转体的侧面积为
\begin{align*}
S &= 2\pi \int_{0}^{\pi} y \sqrt{1+\left(\frac{dy}{dx}\right)^2} \, dx \\
  &= 2\pi \int_{0}^{\pi} \sin x \sqrt{1+\cos^2 x} \, dx \\
  &= 2\pi \int_{1}^{-1} \sqrt{1+u^2} (-du) \quad (\text{令 } u=\cos x) \\
  &= 2\pi \int_{-1}^{1} \sqrt{1+u^2} \, du = 4\pi \int_{0}^{1} \sqrt{1+u^2} \, du \\
  &= 4\pi \cdot \frac{1}{2} \left[ u\sqrt{1+u^2} + \ln\left(u+\sqrt{1+u^2}\right) \right]_{0}^{1} \\
  &= 2\pi \left( \sqrt{2} + \ln(1+\sqrt{2}) \right).
\end{align*}
(2) 曲线为摆线 $x=a(t-\sin t)$, $y=a(1-\cos t)$, $0\leqslant t\leqslant 2\pi$, 绕直线 $y=2a$ 旋转. 旋转半径 $R = |y-2a| = a(1+\cos t)$, 弧微分为
\[
ds = \sqrt{\left(\frac{dx}{dt}\right)^2 + \left(\frac{dy}{dt}\right)^2} \, dt = 2a\sin\frac{t}{2} dt,
\]
故侧面积为
\begin{align*}
S &= \int_{0}^{2\pi} 2\pi R \, ds = 4\pi a^2 \int_{0}^{2\pi} (1+\cos t)\sin\frac{t}{2} \, dt \\
  &= 8\pi a^2 \int_{0}^{2\pi} \cos^2\frac{t}{2} \sin\frac{t}{2} \, dt \quad (\text{利用 } 1+\cos t = 2\cos^2\frac{t}{2}) \\
  &= 8\pi a^2 \cdot \frac{4}{3} = \frac{32\pi}{3} a^2.
\end{align*}
(3) 曲线为心形线 $r=a(1+\cos\theta)$, $0\leq \theta \leq 2\pi$, 绕极轴旋转. 由对称性, 仅需计算上半部分 ($0\leq \theta \leq \pi$) 的侧面积,上半部分上, $y = r\sin\theta$, $ds = 2a\cos\frac{\theta}{2} d\theta$, 故
\begin{align*}
S_{\text{上}} &= \int_{0}^{\pi} 2\pi y \, ds = 4\pi a^2 \int_{0}^{\pi} (1+\cos\theta)\sin\theta \cos\frac{\theta}{2} \, d\theta \\
  &= 16\pi a^2 \int_{0}^{\pi} \cos^4\frac{\theta}{2} \sin\frac{\theta}{2} \, d\theta \quad (\text{化简得}) \\
  &= 16\pi a^2 \cdot \frac{2}{5} = \frac{32\pi}{5} a^2,
\end{align*}
所以整个侧面积为
\[
S = S_{\text{上}} = \frac{32\pi}{5} a^2.
\]
\end{solution}