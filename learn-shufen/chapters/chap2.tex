\chapter{极限}
\newpage
\section{第2周作业}
\begin{example}{2.1-A-3:给出下列极限的精确定义}{}
    (1)$\displaystyle\lim_{x\to 0}f(x)=A$\quad (2)$\displaystyle\lim_{x\to 0}(1+x)^{\frac{1}{x}}=e$
\end{example}
\begin{solution}
    (1)对于任意$\varepsilon >0$,存在$\delta>0$使得当$0<|x|<\delta$时,$|f(x)|<\varepsilon$.

    (2)对于任意$\varepsilon>0$,存在$\delta>0$使得当$0<|x|<\delta$时,$|(1+x)^{\frac{1}{x}}-e|<\varepsilon$.
\end{solution}
\begin{example}{2.1-A-7}{}
    利用极限的精确定义证明下列函数的极限\\
    \noindent(1)$\displaystyle\lim_{x\to 3}(x^2+5x)=24$\quad (2)$\displaystyle\lim_{x\to 1}\frac{x^2-1}{x-1}=2$\\
    (3)$\displaystyle\lim_{x\to +\infty}\frac{x^2+2}{3x^2}=\frac13$\quad(4)$\displaystyle\lim_{x\to 2^{+}}\frac{2x}{x^2-4}=+\infty$.
\end{example}
\begin{solution}
    (1)要证对于任意$\varepsilon>0$,存在$\delta>0$使得当$0<|x-3|<\delta$时,$|(x^2+5x)-24|=|(x+8)(x-3)|<\varepsilon$。已经出现了$|x-3|$,所以现在只需限定$|x+8|$,先限定$|x-3|<1$,那么$|x+8|<12$,此时还需满足$|(x+8)(x-3)|<12|x-3|<\varepsilon$,得$|x-3|<\dfrac{\varepsilon}{12}$,故取$\delta=\min\left\{1,\dfrac{\varepsilon}{12}\right\}$,当$0<|x-3|<\delta$时,$|(x^2+5x)-24|=|(x+8)(x-3)|<\varepsilon$。\\
    (2)要证对于任意$\varepsilon>0$,存在$\delta>0$使得当$0<|x-1|<\delta$时,$\left|\dfrac{x^2-1}{x-1}-2\right|=|x-1|<\varepsilon$。取$\delta=\varepsilon$,当$0<|x-1|<\delta$时,$\left|\dfrac{x^2-1}{x-1}-2\right|<\varepsilon$。\\
    (3)要证对于任意$\varepsilon>0$,存在$\delta>0$使得当$|x|>\delta$时,$\left|\dfrac{x^2+2}{3x^2}-\dfrac{1}{3}\right|<\varepsilon$,因为$\left|\dfrac{x^2+2}{3x^2}-\dfrac{1}{3}\right|=\left|\dfrac{2}{3x^2}\right|$,取$\delta=\sqrt{\dfrac{2}{3\varepsilon}}$,则当$|x|>\delta$时,$\left|\dfrac{x^2+2}{3x^2}-\dfrac{1}{3}\right|<\varepsilon$。\\
    (4)要证对于任意$G>0$,存在$\delta>0$使得当$0<x-2<\delta$时,$\dfrac{2x}{x^2-4}>G$,不妨限定$x+2<5$,则$x-2<1$,则$\dfrac{2x}{x^2-4}>\dfrac{4}{(x+2)(x-2)}>\dfrac{4}{5(x-2)}>G$解得$x-2<\dfrac{4}{5G}$,所以取$\delta=\min\left\{1,\dfrac{4}{5G}\right\}$,当$0<x-2<\delta$时,$\dfrac{2x}{x^2-4}>G$。
\end{solution}
\begin{example}{2.1-A-10}{}
    证明:由$\displaystyle\lim_{x\to a}f(x)=A$能推出$\displaystyle\lim_{x\to a}|f(x)|=|A|$,但反之不然。
\end{example}
\begin{solution}
    对于任意$\varepsilon>0$,存在$\delta>0$使得当$0<|x-a|<\delta$时,$|f(x)-A|=<\varepsilon$,所以由绝对值不等式得到$\displaystyle|f(x)-A|>||f(x)-|-A||=||f(x)|-|A||>0$,故$||f(x)|-|A||<\varepsilon$,所以由$\displaystyle\lim_{x\to a}f(x)=A$能推出$\displaystyle\lim_{x\to a}|f(x)|=|A|$.然后反过来,考虑定义在实数域上的函数$f(x)=\begin{cases}1,x\in Q\\-1,x\notin Q\end{cases}$,其极限$\displaystyle\lim_{x\to a}|f(x)|=1$,但是$\displaystyle\lim_{x\to a}f(x)$不存在。
\end{solution}
\begin{example}{2.1-B-2(1):利用极限的精确证明}{}
   $\displaystyle\lim_{x\to a}\sin x=\sin a$
\end{example}
\begin{solution}
    要证$\displaystyle\lim_{x\to a}\sin x=\sin a$,只需证对于任意$\varepsilon>0$,存在$\delta>0$使得当$0<|x-a|<\delta$时,$\displaystyle\lim_{x\to a}|\sin x-\sin a|=2|\cos\dfrac{x+a}{2}||\sin\dfrac{x-a}{2}|<\varepsilon$。又因为$2|\cos\dfrac{x+a}{2}||\sin\dfrac{x-a}{2}|<2|\sin\dfrac{x-a}{2}|<2|\dfrac{x-a}{2}|=|x-a|$,所以取$\delta=\varepsilon$,当$0<|x-a|<\delta$时,$\displaystyle\lim_{x\to a}|\sin x-\sin a|<\varepsilon$.
\end{solution}
\begin{example}{2.1-B-4}{}
    利用极限的精确定义证明$\displaystyle\lim_{x\to +\infty}\frac{x}{x+1}=+\infty$是错误的.
\end{example}
\begin{solution}
    要证明存在$G>0,\forall \delta>0$使得当$x>\delta$时,$\dfrac{x}{x+1}\leq G$,则取$G=1$,便可以满足$\forall x>0,\dfrac{x}{x+1}\leq 1$,故存在$G>0,\forall \delta>0$使得当$x>\delta$时,$\dfrac{x}{x+1}\leq G$,$\displaystyle\lim_{x\to +\infty}\frac{x}{x+1}=+\infty$是错误的.(本题本质是找到一个够大的上界)
\end{solution}
\begin{example}{2.3-A-2}{}
   \vspace{-10pt}\[\begin{array}{l@{\quad}l@{\quad}l}
(1) \displaystyle \lim _{x \rightarrow 2}(3 x^{2}-5 x+2);&
(2) \displaystyle \lim _{x \rightarrow-1}(x^{2}+1)(1-2 x)^{2};&
(3) \displaystyle \lim _{x \rightarrow+\infty}(x^{5}-40 x^{4});\\
(4) \displaystyle \lim _{x \rightarrow-\infty}(6 x^{5}+21 x^{3});&
(5) \displaystyle \lim _{x \rightarrow 1^{-}} \frac{1}{x-1};&
(6) \displaystyle\lim _{x \rightarrow 1^{+}} \frac{1}{x-1};\\
(7) \displaystyle\lim _{x \rightarrow+\infty} \frac{x^{3}+1}{x^{4}+2};&
(8) \displaystyle \lim _{x \rightarrow-\infty} \frac{5 x^{3}+2 x}{x^{10}+x+7}.
    \end{array}\]
\end{example}
\begin{solution}
(1)$\displaystyle\lim_{x\to 2}(3 x^{2}-5 x+2)=\lim_{x\to 2}(3(2)^2-5(2)+2)=4$.\\
(2)$\displaystyle\lim_{x\to-1}(x^{2}+1)(1-2 x)^{2}=\lim_{x\to-1}(1+1)(1-2(-1))^{2}=18$.\\
(3)$\displaystyle\lim_{x\to+\infty}(x^{5}-40 x^{4})=\lim_{x\to+\infty}x^4(x-40)=\lim_{x\to+\infty}x^4\lim_{x\to+\infty}(x-40)=+\infty$.\\
(4)$\displaystyle\lim_{x\to-\infty}(6 x^{5}+21 x^{3})=\lim_{x\to-\infty}x^5(6+\frac{21}{x^2})=\lim_{x\to-\infty}x^5\lim_{x\to-\infty}(6+\frac{21}{x^2})=-\infty$.\\
(5)$\displaystyle\lim_{x\to 1^{-}} \frac{1}{x-1}=\lim_{x\to 0^{-}} \frac{1}{x}=-\infty$.\\
(6)$\displaystyle\lim_{x\to 1^{+}} \frac{1}{x-1}=\lim_{x\to 0^{+}} \frac{1}{x}=+\infty$.\\
(7)$\displaystyle\lim_{x\to+\infty} \frac{x^{3}+1}{x^{4}+2}=\lim_{x\to+\infty}\frac{1+\frac{1}{x^3}}{x+\frac{2}{x^3}}=0$.\\
(8)$\displaystyle\lim_{x\to-\infty} \frac{5 x^{3}+2 x}{x^{10}+x+7}=\lim_{x\to-\infty}\frac{\frac{5}{x^7}+\frac{2}{x^{10}}}{1+\frac{1}{x^9}+\frac{7}{x^{10}}}=0$;
\end{solution}
\begin{example}{2.3-A-4}{}
       \vspace{-5pt}\[\begin{array}{l@{\quad}l}
    (1)\displaystyle\lim_{x\rightarrow +\infty}(\sqrt{x^2+x}-x)&
    (2)\displaystyle\lim_{x\rightarrow -\infty}(\frac{\sqrt{3x^2+x}}{x})\\
    (3)\displaystyle\lim_{x\rightarrow +\infty}(\frac{\sqrt{4x^2+2x+1}}{3x})&
    (4)\displaystyle\lim_{x\rightarrow -\infty}(\frac{\sqrt{9x^2+x+3}}{6x})\\
       \end{array}\]
\end{example}
\begin{solution}
    (1)$\displaystyle\lim_{x\rightarrow +\infty}(\sqrt{x^2+x}-x)=\lim_{x\rightarrow +\infty}\frac{x}{\sqrt{x^2+x}+x}=\lim_{x\rightarrow +\infty}\frac{1}{\sqrt{1+\frac{1}{x}}+1}=\frac12$.\\
    (2)$\displaystyle\lim_{x\rightarrow -\infty}(\frac{\sqrt{3x^2+x}}{x})=-\lim_{x\rightarrow -\infty}\sqrt{3+\frac{1}{x^2}}=-\sqrt3$.\\
    (3)$\displaystyle\lim_{x\rightarrow +\infty}(\frac{\sqrt{4x^2+2x+1}}{3x})=\lim_{x\rightarrow +\infty}\sqrt{\frac{4}{9}+\frac{2}{9x}+\frac{1}{9x^2}}=\frac{2}{3}$.\\
    (4)$\displaystyle\lim_{x\rightarrow -\infty}(\frac{\sqrt{9x^2+x+3}}{6x})=-\lim_{x\rightarrow -\infty}\sqrt{\frac{9}{36}+\frac{1}{36x}+\frac{3}{6x^2}}=-\frac12$.
\end{solution}
\begin{example}{2.3-A-8}{}
    (1)$y=\dfrac{x^2-2x-2}{x-1}$\quad(2)$y=\dfrac{2x^2}{(1-x)^2}$\end{example}
\begin{solution}
    (1)$\displaystyle\lim_{x\to \infty}\dfrac{x^2-2x-2}{(x-1)x}=\lim_{x\to \infty}\dfrac{1-\frac{2}{x}-\frac{2}{x^2}}{1-\frac{1}{x}}=1$.
        \[\displaystyle\lim_{x\to \infty}\dfrac{x^2-2x-2}{x-1}-x=\lim_{x\to \infty}\dfrac{x^2-x^2-2x+x-2}{x-1}=\lim_{x\to \infty}\dfrac{-x-2}{x-1}=\lim_{x\to \infty}\dfrac{-1-\frac{2}{x}}{1-\frac{1}{x}}=-1\]
        故该函数在无穷远处的渐近线为$y=x-1$.
        \[\displaystyle\lim_{x\to 1^{-}}\dfrac{x^2-2x-2}{x-1}=\lim_{x\to 1^{-}}\dfrac{(x-1)^2-1}{x-1}=\lim_{x\to 0^{-}}(x-\dfrac{1}{x})=+\infty\]
        \[\displaystyle\lim_{x\to 1^{+}}\dfrac{x^2-2x-2}{x-1}=\lim_{x\to 1^{+}}\dfrac{(x-1)^2-1}{x-1}=\lim_{x\to 0^{+}}(x-\dfrac{1}{x})=-\infty\]
        故该函数在$x=1$处的渐近线为$x=1$.\\
    (2)\[\displaystyle\lim_{x\to \infty}\dfrac{2x^2}{(x-1)^2}=\lim_{x\to \infty}\dfrac{2}{(1-\frac{1}{x})^2}=2\]
        故该函数在无穷远处的渐近线为$y=2$.
        \[\displaystyle\lim_{x\to 2}\dfrac{2x^2}{(x-1)^2}=\lim_{x\to 1}\dfrac{2}{(1-\frac{1}{x})^2}=+\infty\]
        故该函数在$x=2$处的渐近线为$x=2$.
\end{solution}
\begin{example}{习题2.3-B组-1}{}
    已知$\displaystyle\lim_{x\to +\infty}f(x)=+\infty,\lim_{x\to +\infty}g(x)=+\infty$,讨论下列极限的状态:
    \vspace{-10pt}\[\begin{array}{l@{\quad}l}(1)\displaystyle\lim_{x\to +\infty}(f(x)+g(x))&(2)\displaystyle\lim_{x\to +\infty}(f(x)-g(x))\\
        (3)\displaystyle\lim_{x\to +\infty}f(x)g(x)&(4)\displaystyle\lim_{x\to +\infty}\frac{f(x)}{g(x)}\end{array}\]
\end{example}
\begin{solution}
    (1)$\displaystyle\lim_{x\to +\infty}(f(x)+g(x))=\lim_{x\to +\infty}f(x)+\lim_{x\to +\infty}g(x)=+\infty+\infty=+\infty$.\\
    (2)$\displaystyle\lim_{x\to +\infty}(f(x)-g(x))$不确定,比如当$f(x)=x$时,假如$g(x)=2x$,那么$\displaystyle\lim_{x\to +\infty}(f(x)-g(x))=-\infty$,但当$g(x)=\frac{x}{2}$时,$\displaystyle\lim_{x\to +\infty}(f(x)-g(x))=\lim_{x\to +\infty}\frac{x}{2}=+\infty$,当$f(x)=g(x)$时,$\displaystyle\lim_{x\to +\infty}(f(x)-g(x))=0$.\\
    (3)$\displaystyle\lim_{x\to +\infty}f(x)g(x)=\lim_{x\to +\infty}f(x)\lim_{x\to +\infty}g(x)=+\infty\cdot +\infty=+\infty$.\\
    (4)$\displaystyle\lim_{x\to +\infty}\frac{f(x)}{g(x)}$不确定,比如当$f(x)=x$时,假如$g(x)=2x$,那么$\displaystyle\lim_{x\to +\infty}\frac{f(x)}{g(x)}=\frac{1}{2}$,但当$g(x)=\sqrt{x}$时,$\displaystyle\lim_{x\to +\infty}\frac{f(x)}{g(x)}=\lim_{x\to +\infty}\sqrt{x}=\infty$,又当$g(x)=x^2$时,$\displaystyle\lim_{x\to +\infty}\frac{f(x)}{g(x)}=\lim_{x\to +\infty}\frac{1}{x}=0$.
\end{solution}
\begin{example}{习题2.3-B组-4}{}
    设$\displaystyle\lim_{x\to \infty}\bigg(\frac{x^2+1}{x+1}-ax-b\bigg)=0$,求$a$和$b$.
\end{example}
\begin{solution}
    \begin{align*}\displaystyle\lim_{x\to \infty}\bigg(\frac{x^2+1}{x+1}-ax-b\bigg)&=\lim_{x\to \infty}\bigg(\frac{x^2+1-ax^2-ax-bx-b}{x+1}\bigg)\\&=\lim_{x\to \infty}\dfrac{(1-a)x^2-(a+b)x+1-b}{x+1}=\lim_{x\to \infty}\dfrac{(1-a)x-(a+b)+\frac{1-b}{x}}{1+\frac{1}{x}}\\&=\dfrac{\lim_{x\to \infty}\bigg((1-a)x-(a+b)+\frac{1-b}{x}\bigg)}{1}\\&=\lim_{x\to \infty}(1-a)x-(a+b)=0\end{align*}
    必须有$a=1,b=-1$.
\end{solution}
\begin{example}{习题2.3-B组-5}{}
    $\text{设 }a,b,c\text{ 是常数 },a\neq0,\text{证明 }y=\dfrac{ax^{2}+bx+c}{x+1}\text{的图形有斜渐近线,并求出渐近线方程}.$
\end{example}
\begin{solution}
    \[\lim_{x\to \infty}\dfrac{ax^{2}+bx+c}{(x+1)x}=\lim_{x\to \infty}\dfrac{a+\frac{b}{x}+\frac{c}{x^2}}{1+\frac{1}{x}}=a\neq 0\]
    由此可知,$y=\dfrac{ax^{2}+bx+c}{x+1}$的图形有斜渐近线.
    \[\lim_{x\to \infty}\dfrac{ax^{2}+bx+c}{x+1}-ax=\lim_{x\to \infty}\dfrac{(b-a)x+c}{x+1}=\lim_{x\to \infty}\dfrac{b-a+\frac{c}{x}}{1+\frac{1}{x}}=b-a\]则渐近线方程为$y=ax+b-a$.
\end{solution}
\begin{example}{习题2.2-A-2}{}
    $\displaystyle (1)\lim_{n\to\infty}(\sqrt{n-1}-\sqrt{n})=0~~~~(2)\lim_{n\to \infty}\dfrac{n^2}{2^n}=0~~~(3)\lim_{n\to \infty}\dfrac{n}{n+1}=1~~~(4)\lim_{n\to\infty}\dfrac{3n}{5n+1}=\dfrac35$
\end{example}
\begin{solution}
(1)要证明对于任意$\varepsilon>0$,存在$N\in\mathbb{N}$使得当$n\geq N$时有$|\sqrt{n-1}-\sqrt{n}|<\varepsilon$.
\[|\sqrt{n-1}-\sqrt{n}|=\left|\dfrac{1}{\sqrt{n-1}+\sqrt{n}}\right|<\left|\dfrac{1}{2\sqrt{n-1}}\right|<\varepsilon\]
因此取$N=\left[2+\dfrac{1}{4\varepsilon^2}\right]$,则对于任意$\varepsilon>0$,当$n\geq N$时,有$\left|\sqrt{n-1}-\sqrt{n}\right|<\varepsilon$.\\
    (2)要证明对于任意$\varepsilon>0$,存在$N\in\mathbb{N}$使得$n\geq N$时,$\left|\dfrac{n^2}{2^n}\right|<\varepsilon$,限定$n>9$,则$2^n>n^3$,则有$\left|\dfrac{n^2}{2^n}\right|<\left|\dfrac{n^2}{n^3}\right|=2\left|\dfrac{1}{n}\right|<\varepsilon$,则取$N=\max\left\{9,\left[1+\dfrac{2}{\varepsilon}\right]\right\}$,任意$\varepsilon>0$,$\left|\dfrac{n^2}{2^n}\right|<\varepsilon$.\\
    (3)要证明对于任意$\varepsilon>0$,存在$N\in\mathbb{N}$使得当$n\geq N$时,$\left|\dfrac{n}{n+1}-1\right|=\left|\dfrac{1}{n+1}\right|<\varepsilon$,则取$N=\left[\dfrac{1}{\varepsilon}\right]$,任意$\varepsilon>0$,$\left|\dfrac{n}{n+1}-1\right|<\varepsilon$.\\
    (4)要证明对于任意$\varepsilon>0$,存在$N\in\mathbb{N}$使得当$n\geq N$时,$\left|\dfrac{3n}{5n+1}-\dfrac{3}{5}\right|=\dfrac{3}{5}\left|\dfrac{1}{5n+1}\right|<\left|\dfrac{1}{n}\right|<\varepsilon$,则取$N=\left[\dfrac{1}{\varepsilon}+1\right]$,任意$\varepsilon>0$,$\left|\dfrac{3n}{5n+1}-\dfrac{3}{5}\right|<\varepsilon$.
\end{solution}
\begin{example}{习题2.2-A-4}{}
    证明:由$\displaystyle\lim_{x\to\infty}x_n=a$能推出$\displaystyle\lim_{n\to\infty}|x_n|=|a|$,但反过来不可以.
\end{example}
\begin{solution}
    对于任意$\varepsilon>0$,存在$N\in\mathbb{N}$使得当$n\geq N$时,$||x_n|-|a||<|x_n-a|<\varepsilon$,因此$\displaystyle\lim_{n\to\infty}|x_n|=|a|$,但是考虑数列$a_n=(-1)^n$,则$\displaystyle\lim_{n\to\infty}|a_n|=1$,但是去掉绝对值后,$\displaystyle\lim_{n\to\infty}a_n$不存在,所以不能反过来.
\end{solution}
\begin{example}{习题2.2-B-1}{利用定义证明下列数列极限}
$(1)\displaystyle \lim_{n\to\infty}\dfrac{n^3-1}{n^2+1}=\infty~~(2)\lim_{n\to\infty}\arctan n=\dfrac{\pi}2~~(3)\displaystyle \lim_{n\to\infty}\dfrac{\sqrt{n}}{2\sqrt{n}+1}=\dfrac12~~(4)\lim_{n\to\infty}\dfrac{a^n}{n!}=0.(a>1)$
\end{example}
\begin{solution}
    (1)要证明任意$G>0$,存在$N\in\mathbb{N}$使得当$n\geq N$时有$\left|\dfrac{n^3-1}{n^2+1}\right|=\dfrac{n^3-1}{n^2+1}>G$,由
    \[\dfrac{n^3-1}{n^2+1}=\dfrac{(n-1)(n^2+n+1)}{n^2+1}>\dfrac{(n-1)(n^2+1)}{n^2+1}=n-1\]所以取$N=G+2$,则任意$G>0$,$\left|\dfrac{n^3-1}{n^2+1}\right|>G$.\newline
    (2)要证明对于任意$\varepsilon>0$,存在$N\in\mathbb{N}$使得当$n\geq N$时,$\left|\arctan n-\dfrac{\pi}{2}\right|=\dfrac{\pi}{2}-\arctan n<\varepsilon$,则取$N=[\tan\dfrac{\pi}{2}-\varepsilon]$,任意$\varepsilon>0$,$\left|\arctan n-\dfrac{\pi}{2}\right|<\varepsilon$.\newline
    (3)要证明对于任意$\varepsilon>0$,存在$N\in\mathbb{N}$使得当$n\geq N$时,$\left|\dfrac{\sqrt{n}}{2\sqrt{n}+1}-\dfrac{1}{2}\right|=\dfrac{1}{2(2\sqrt{n}+1)}$,取$N=\left[1+\bigg(\dfrac{1}{4\varepsilon}-\dfrac12\bigg)^2\right]$,任意$\varepsilon>0$,$\left|\dfrac{\sqrt{n}}{2\sqrt{n}+1}-\dfrac{1}{2}\right|<\varepsilon$.\newline
    (4)要证明对于任意$\varepsilon>0$,存在$N\in\mathbb{N}$使得当$n\geq N$时,$\left|\dfrac{a^n}{n!}\right|=\dfrac{a^n}{n!}<\varepsilon$,由
    \[\dfrac{a^n}{n!}=\dfrac{a}{1}\dfrac{a}{2}\dfrac{a}{3}\cdots\dfrac{a}{[a]}\dfrac{a}{[a]+1}\cdots\dfrac{a}{n}<\dfrac{a^{[a]}}{[a]!}\dfrac{a}{n}\]
    所以取$N=\left[\dfrac{a^{[a+1]}}{[a]!\varepsilon}+1\right]$,则任意$\varepsilon>0$,$\left|\dfrac{a^n}{n!}\right|<\varepsilon$.    
\end{solution}
\begin{example}{2.3-A-3}{求下列极限}
\vspace{-5pt}\[\begin{array}{l@{\quad}l}
    (1)\displaystyle \lim_{n\to\infty}\dfrac{(-2)^n+3^n}{(-2)^{n+1}+3^{n+1}}&(2)\displaystyle\lim_{n\to\infty}\bigg(\dfrac{1}{n^2}+\dfrac{2}{n^2}+\cdots+\dfrac{n-1}{n^2}\bigg)\\(3)\displaystyle \lim_{n\to\infty}\left[\dfrac1{1\cdot2}+\dfrac{1}{2\cdot3}+\dfrac{1}{3\cdot4}+\cdots+\dfrac{1}{n(n+1)}\right]&(4)\displaystyle\lim_{n\to\infty}\left[\dfrac{1}{2!}+\dfrac{2}{3!}+\cdots+\dfrac{n}{(n+1)!}\right]\end{array}\]
\end{example}
\begin{solution}
    (1)代数变形:
    \[\displaystyle\dfrac{(-2)^n+3^n}{(-2)^{n+1}+3^{n+1}}=\dfrac{(-2)^{-1}+\dfrac{3^n}{(-2)^{n+1}}}{1+(-\dfrac32)^{n+1}}=\dfrac{-\dfrac12+\dfrac13(-\dfrac32)^{n+1}}{1+(-\dfrac32)^{n+1}}=\dfrac13-\dfrac{5}{6\bigg(1+(-\dfrac32)^{n+1}\bigg)}\]
    故$\displaystyle\lim_{n\to\infty}\dfrac{(-2)^n+3^n}{(-2)^{n+1}+3^{n+1}}=\lim_{n\to\infty}\bigg(\dfrac13-\dfrac{5}{6\bigg(1+(-\dfrac32)^{n+1}\bigg)}\bigg)=\dfrac13$.\newline
    (2)\[\lim_{n\to\infty}\bigg(\dfrac{1}{n^2}+\dfrac{2}{n^2}+\cdots+\dfrac{n-1}{n^2}\bigg)=\lim_{n\to\infty}\dfrac{n(n-1)}{2n^2}=\dfrac12\]
    (3)\[\displaystyle\lim_{n\to\infty}\left[\dfrac{1}{1\cdot 2}+\dfrac{1}{2\cdot3}+\dfrac{1}{3\cdot4}+\cdots+\dfrac{1}{n(n+1)}\right]=\lim_{n\to\infty}\bigg(1-\dfrac{1}{n+1}\bigg)=1\]
    (4)用裂项$\dfrac{k}{(k+1)!}=\dfrac{k+1-1}{(k+1)!}=\dfrac{1}{k!}-\dfrac{1}{(k+1)!}$,那么
    \[\displaystyle\lim_{n\to\infty}\left[\dfrac{1}{2!}+\dfrac{2}{3!}+\cdots+\dfrac{n}{(n+1)!}\right]=\lim_{n\to\infty}\bigg(\dfrac12-\dfrac{1}{(n+1)!}\bigg)=\dfrac12\]
\end{solution}
\begin{example}{2.4-A-5}{求极限}
\vspace{-5pt}\[
\begin{array}{l@{\quad}l@{\quad}l}
(1) \displaystyle\lim_{x\to 0}\dfrac{\sin x}{2x}         & (2) \displaystyle\lim_{x\to 0}\dfrac{\sin 2x}{x}          & (3) \displaystyle\lim_{x\to 0}\dfrac{\sin 3x}{5x} \\
(4) \displaystyle\lim_{x\to 0}\dfrac{2x}{\sin 3x}        & (5) \displaystyle\lim_{\theta\to 0}\dfrac{\sin^2 \theta}{\theta} & (6) \displaystyle\lim_{h\to 0}\dfrac{\tan^2 h}{h} \\
(7) \displaystyle\lim_{\theta\to 0}\dfrac{1-\cos \theta}{\theta^2} & (8) \displaystyle\lim_{x\to 0}\dfrac{\sin^2 x}{x^2} & (9) \displaystyle\lim_{x\to 0}\dfrac{\tan x-\sin x}{\sin^3 x} \\
(10) \displaystyle\lim_{x\to 0}\dfrac{\sin 5x-\sin 3x}{\sin x} & (11) \displaystyle\lim_{x\to 0}\dfrac{\cos x-\cos 3x}{x^2} & (12) \displaystyle\lim_{x\to a}\dfrac{\sin x-\sin a}{x-a} \\
(13) \displaystyle\lim_{x\to a}\dfrac{\cos x-\cos a}{x-a} & (14) \displaystyle\lim_{x\to a}\dfrac{\tan x-\tan a}{x-a} & \\
\end{array}
\]
\end{example}
\begin{solution}
    这里只使用基本极限公式:
    
    \vspace{0.5em}
    (1)$\displaystyle\lim_{x\to 0}\frac{\sin x}{2x}=\lim_{x\to 0}\dfrac12\frac{\sin x}{x}=\dfrac12$.\quad
    (2)$\displaystyle\lim_{x\to 0}\frac{\sin 2x}{x}=\lim_{x\to 0}\dfrac{2\sin 2x}{2x}=2$.
    
    \vspace{0.5em}
    (3)$\displaystyle\lim_{x\to 0}\frac{\sin 3x}{5x}=\lim_{x\to 0}\dfrac{3}{5}\dfrac{\sin 3x}{3x}=\dfrac{3}{5}$.\quad
    (4)$\displaystyle\lim_{x\to 0}\frac{2x}{\sin 3x}=\lim_{x\to 0}\dfrac{2}{3}\dfrac{3x}{\sin 3x}=\dfrac{2}{3}$.
    
    \vspace{0.5em}
    (5)$\displaystyle\lim_{\theta\to 0}\frac{\sin^2 \theta}{\theta}=\lim_{\theta\to 0}\dfrac{\sin^2 \theta}{\theta^2}\theta=0$.\quad
    (6)$\displaystyle\lim_{h\to 0}\frac{\tan^2 h}{h}=\lim_{h\to 0}\dfrac{\tan^2 h}{h^2}h=0$.
    
    \vspace{0.5em}
    (7)$\displaystyle\lim_{\theta\to 0}\frac{1-\cos \theta}{\theta^2}=\lim_{\theta\to 0}\dfrac{\sin^2 \dfrac{\theta}{2}}{2\left(\dfrac{\theta}{2}\right)^2}=\dfrac12$.
    (8)$\displaystyle\lim_{x\to 0}\frac{\sin^2 x}{x^2}=\left(\lim_{x\to 0}\dfrac{\sin x}{x}\right)^2=1$.
    
    \vspace{0.5em}
    (9)$\displaystyle\lim_{x\to 0}\frac{\tan x-\sin x}{\sin^3 x}=\lim_{x\to 0}\dfrac{\sin x(1-\cos x)}{\sin^3 x\cos x}=\lim_{x\to 0}\dfrac{1-\cos x}{(1-\cos^2x)\cos x}=\lim_{x\to 0}\dfrac{\sec x}{(1+\cos x)}=\dfrac12$.
    
    \vspace{0.5em}
    (10)$\displaystyle\lim_{x\to 0}\frac{\sin 5x-\sin 3x}{\sin x}=\lim_{x\to 0}\dfrac{2\cos 4x\sin x}{\sin x}=\lim_{x\to 0}2\cos 4x=2$.
    
    \vspace{0.5em}
    (11)$\displaystyle\lim_{x\to 0}\frac{\cos x-\cos 3x}{x^2}=\lim_{x\to 0}\dfrac{2\sin 2x\sin x}{x^2}=4\lim_{x\to 0}\dfrac{\sin 2x}{2x}\dfrac{\sin x}{x}=4$.
    
    \vspace{0.5em}
    (12)$\displaystyle\lim_{x\to a}\frac{\sin x-\sin a}{x-a}=\lim_{x\to a}\dfrac{\cos\frac{x+a}{2}\sin\frac{x-a}{2}}{\frac{x-a}{2}}=\cos a$.
    
    \vspace{0.5em}
    (13)$\displaystyle\lim_{x\to a}\frac{\cos x-\cos a}{x-a}=\lim_{x\to a}\dfrac{-\sin\frac{x+a}{2}\sin\frac{x-a}{2}}{\frac{x-a}{2}}=-\sin a$.
    
    \vspace{0.5em}
    (14)$\displaystyle\lim_{x\to a}\frac{\tan x-\tan a}{x-a}=\lim_{x\to a}\dfrac{\frac{\sin x}{\cos x}-\frac{\sin a}{\cos a}}{x-a}=\lim_{x\to a}\dfrac{\sin(x-a)}{(x-a)\cos a\cos(x-a)}=\sec^2a$,其中 $a\neq (2k+1)\dfrac{\pi}{2}$,$k\in\mathbb{Z_+}$.
\end{solution}
\begin{example}{2.4-A-6}{}
    \vspace{-5pt}\[\begin{array}{l@{\quad}l@{\quad}l}
    (1)\lim_{x\to\infty}\left(1+\frac{3}{x}\right)^{x/2}\quad&(2)\lim_{x\to0}(1-x)^{1/x}\quad&(3)\lim_{\Delta x\to0}\left(1+\frac{\Delta x}{x}\right)^{x/\Delta x}(x\neq0)\\(4)\lim_{x\to0}(1+ax)^{1/x}\quad&(5)\lim_{x\to\infty}\left(1-\frac{1}{x}\right)^{4x}\end{array}\]
\end{example}
\begin{solution}
    (1)$\displaystyle\lim_{x\to \infty}\left(1+\dfrac{3}{x}\right)^{\frac{x}{2}}=\lim_{x\to \infty}\left(\left(1+\dfrac{3}{x}\right)^{x/3}\right)^{\frac{3}{2}}=\e^{\frac32}$.\\\vspace{0.5em}
    (2)$\displaystyle\lim_{x\to 0}(1-x)^{\frac1{x}}=\lim_{x\to 0}\left(\left(1+\frac{1}{\frac{-1}{x}}\right)^{\frac{-1}{x}}\right)^{-1}=\dfrac{1}{\e}$.\\\vspace{0.5em}
    (3)$\displaystyle\lim_{\Delta x\to 0}\left(1+\frac{\Delta x}{x}\right)^{x/\Delta x}(x\neq0)=\lim_{\frac{\Delta x}{x}\to 0}\left(\left(1+\frac{\Delta x}{x}\right)^{x/\Delta x}\right)=\e$\\vspace{0.5em}
    (4)$\displaystyle\lim_{x\to 0}(1+ax)^{\frac1{x}}=\lim_{x\to 0}\left[\left(1+\dfrac{1}{\frac{1}{ax}}\right)^{\frac{1}{ax}}\right]^{a}=\lim_{\frac1{ax}\to\infty}\left[\left(1+\dfrac{1}{\frac{1}{ax}}\right)^{\frac{1}{ax}}\right]^{a}=\e^a$\\\vspace{0.5em}
    (5)$\displaystyle\lim_{x\to\infty}\left(1-\frac{1}{x}\right)^{4x}=\lim_{x\to\infty}\left[\left(1+\frac{1}{-x}\right)^{-x}\right]^{-4}=\e^{-4}$.
\end{solution}
\begin{example}{2.4-B-4}{}
    \vspace{-5pt}\[\begin{array}{l@{\quad}l@{\quad}l}
    (1)\displaystyle\lim_{x\to\infty}\dfrac{3x-5}{x^3\sin\dfrac{1}{x^2}}\quad&
    (2)\displaystyle\lim_{x\rightarrow \frac{\pi}{6}} \sin \left(\frac{\pi}{6}-x\right) \tan 3 x\quad&
    (3)\displaystyle\lim_{x\rightarrow 0} \frac{2\sin x-\sin 2x}{x^{3}}\quad\\
    (4)\displaystyle\lim_{x\rightarrow \frac{\pi}{4}} \frac{\tan x-1}{x-\frac{\pi}{4}}\quad&
    (5)\displaystyle\lim_{x\rightarrow 1} x^{\frac{1}{1-x}}
\end{array}\]
\end{example}
\begin{solution}
    (1)$\displaystyle\lim_{x\to\infty}\dfrac{3x-5}{x^3\sin\dfrac{1}{x^2}}=\lim_{x\to 0}\dfrac{\dfrac{3}{x}-5}{\dfrac{1}{x^3}\sin x^2}=\lim_{x\to 0}\dfrac{3x^2-5x^3}{\sin x^2}=3\lim_{x\to 0}\dfrac{x^2}{\sin x^2}\dfrac{3x^2-5x}{3x^2}=3$.\\\vspace{0.5em}
    (2)$\displaystyle\lim _{x \rightarrow \frac{\pi}{6}} \sin \left(\frac{\pi}{6}-x\right) \tan 3 x=\lim _{x \rightarrow 0}-\sin x\tan(3x+\dfrac{\pi}{2})=\lim_{x\to 0}\dfrac{\sin x}{\tan 3x}=\dfrac13\lim_{x\to 0}\dfrac{\sin x}{x}\dfrac{x}{\tan x}=\dfrac13$.\\\vspace{0.5em}
    (3)$\displaystyle\lim _{x \rightarrow 0} \frac{2 \sin x-\sin 2 x}{x^{3}}=2\lim _{x \rightarrow 0}\dfrac{\sin x}{x}\dfrac{1-\cos x}{x^2}=2\dfrac12=1$.\\\vspace{0.5em}
    (4)$\displaystyle\lim _{x \rightarrow \frac{\pi}{4}} \frac{\tan x-1}{x-\frac{\pi}{4}}=\lim _{x \rightarrow 0}\dfrac{\dfrac{\tan x+1}{1-\tan x}-1}{x}=\lim _{x \rightarrow 0}\dfrac{2\tan x}{x(1-\tan x)}=\lim_{x\to 0}\dfrac{2\tan x}{x}=2$.\\\vspace{0.5em}
    (5)$\displaystyle\lim _{x \rightarrow 1} x^{\frac{1}{1-x}}=\lim _{x \rightarrow 0}(1+x)^{\frac{-1}{x}}=\lim_{x \rightarrow \infty}\bigg((1+\dfrac1x){x}\bigg)^{-1}=\dfrac{1}{\e}$.
\end{solution}
