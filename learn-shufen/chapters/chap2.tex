\chapter{极限}
\newpage
\section{第2周作业}
\begin{example}{给出下列极限的精确定义}{}
    (1)$\displaystyle\lim_{x\to 0}f(x)=A$\quad (2)$\displaystyle\lim_{x\to 0}(1+x)^{\frac{1}{x}}=e$
\end{example}
\begin{solution}
    (1)对于任意$\varepsilon >0$,存在$\delta>0$使得当$0<|x|<\delta$时,$|f(x)|<\varepsilon$.

    (2)对于任意$\varepsilon>0$,存在$\delta>0$使得当$0<|x|<\delta$时,$|(1+x)^{\frac{1}{x}}-e|<\varepsilon$.
\end{solution}
\begin{example}{利用极限的精确定义证明下列函数的极限}{}
    (1)$\displaystyle\lim_{x\to 3}(x^2+5x)=24$\quad (2)$\displaystyle\lim_{x\to 1}\frac{x^2-1}{x-1}=2$
\end{example}
\begin{solution}
    (1)要证对于任意$\varepsilon>0$,存在$\delta>0$使得当$0<|x-3|<\delta$时,$|(x^2+5x)-24|=|(x+8)(x-3)|<\varepsilon$。已经出现了$|x-3|$,所以现在只需限定$|x+8|$,先限定$|x-3|<1$,那么$|x+8|<12$,此时还需满足$|(x+8)(x-3)|<12|x-3|<\varepsilon$,得$|x-3|<\dfrac{\varepsilon}{12}$,故取$\delta=\min\{1,\dfrac{\varepsilon}{12}\}$,当$0<|x-3|<\delta$时,$|(x^2+5x)-24|=|(x+8)(x-3)|<\varepsilon$.

    (2)要证对于任意$\varepsilon>0$,存在$\delta>0$使得当$0<|x-1|<\delta$时,$|\frac{x^2-1}{x-1}-2|=|x-1|<\varepsilon$。取$\delta=\varepsilon$,当$0<|x-1|<\delta$时,$|\frac{x^2-1}{x-1}-2|<\varepsilon$.
\end{solution}
\begin{example}{证明}{}
    由$\displaystyle\lim_{x\to a}f(x)=A$能推出$\displaystyle\lim_{x\to a}|f(x)|=|A|$,但反之不然。
\end{example}
\begin{solution}
    对于任意$\varepsilon>0$,存在$\delta>0$使得当$0<|x-a|<\delta$时,$|f(x)-A|=<\varepsilon$,所以由绝对值不等式得到$\displaystyle|f(x)-A|>||f(x)-|-A||=||f(x)|-|A||>0$,故$||f(x)|-|A||<\varepsilon$,所以由$\displaystyle\lim_{x\to a}f(x)=A$能推出$\displaystyle\lim_{x\to a}|f(x)|=|A|$.然后反过来,考虑定义在实数域上的函数$f(x)=\begin{cases}1,x\in Q\\-1,x\notin Q\end{cases}$,其极限$\displaystyle\lim_{x\to a}|f(x)|=1$,但是$\displaystyle\lim_{x\to a}f(x)$不存在。
\end{solution}
\begin{example}{利用极限的精确证明}{}
   $\displaystyle\lim_{x\to a}\sin x=\sin a$
\end{example}
\begin{solution}
    要证$\displaystyle\lim_{x\to a}\sin x=\sin a$,只需证对于任意$\varepsilon>0$,存在$\delta>0$使得当$0<|x-a|<\delta$时,$\displaystyle\lim_{x\to a}|\sin x-\sin a|=2|\cos\dfrac{x+a}{2}||\sin\dfrac{x-a}{2}|<\varepsilon$。又因为$2|\cos\dfrac{x+a}{2}||\sin\dfrac{x-a}{2}|<2|\sin\dfrac{x-a}{2}|<2|\dfrac{x-a}{2}|=|x-a|$,所以取$\delta=\varepsilon$,当$0<|x-a|<\delta$时,$\displaystyle\lim_{x\to a}|\sin x-\sin a|<\varepsilon$.
\end{solution}
\begin{example}{证明}{}
   \vspace{-10pt}\[\begin{array}{l@{\quad}l@{\quad}l}
(1) \displaystyle \lim _{x \rightarrow 2}(3 x^{2}-5 x+2);&
(2) \displaystyle \lim _{x \rightarrow-1}(x^{2}+1)(1-2 x)^{2};&
(3) \displaystyle \lim _{x \rightarrow+\infty}(x^{5}-40 x^{4});\\
(4) \displaystyle \lim _{x \rightarrow-\infty}(6 x^{5}+21 x^{3});&
(5) \displaystyle \lim _{x \rightarrow 1^{-}} \frac{1}{x-1};&
(6) \displaystyle\lim _{x \rightarrow 1^{+}} \frac{1}{x-1};\\
(7) \displaystyle\lim _{x \rightarrow+\infty} \frac{x^{3}+1}{x^{4}+2};&
(8) \displaystyle \lim _{x \rightarrow-\infty} \frac{5 x^{3}+2 x}{x^{10}+x+7}.
    \end{array}\]
\end{example}
\begin{solution}
(1)$\displaystyle\lim_{x\to 2}(3 x^{2}-5 x+2)=\lim_{x\to 2}(3(2)^2-5(2)+2)=2$.\\
(2)$\displaystyle\lim_{x\to-1}(x^{2}+1)(1-2 x)^{2}=\lim_{x\to-1}(1+1)(1-2(-1))^{2}=9$.\\
(3)$\displaystyle\lim_{x\to+\infty}(x^{5}-40 x^{4})=\lim_{x\to+\infty}x^4(x-40)=\lim_{x\to+\infty}x^4\lim_{x\to+\infty}(x-40)=+\infty$.\\
(4)$\displaystyle\lim_{x\to-\infty}(6 x^{5}+21 x^{3})=\lim_{x\to-\infty}x^5(6-\frac{21}{x^2})=\lim_{x\to-\infty}x^5\lim_{x\to-\infty}(6-\frac{21}{x^2})=-\infty$.\\
(5)$\displaystyle\lim_{x\to 1^{-}} \frac{1}{x-1}=\lim_{x\to 0^{-}} \frac{1}{x}=-\infty$.\\
(6)$\displaystyle\lim_{x\to 1^{+}} \frac{1}{x-1}=\lim_{x\to 0^{+}} \frac{1}{x}=+\infty$.\\
(7)$\displaystyle\lim_{x\to+\infty} \frac{x^{3}+1}{x^{4}+2}=\lim_{x\to+\infty}\frac{1+\frac{1}{x^3}}{x+\frac{2}{x^3}}=0$.\\
(8)$\displaystyle\lim_{x\to-\infty} \frac{5 x^{3}+2 x}{x^{10}+x+7}=\lim_{x\to-\infty}\frac{\frac{5}{x^7}+\frac{2}{x^{10}}}{1+\frac{1}{x^9}+\frac{7}{x^{10}}}=0$;
\end{solution}
\begin{example}{证明}{}
       \vspace{-5pt}\[\begin{array}{l@{\quad}l}
    (1)\displaystyle\lim_{x\rightarrow +\infty}(\sqrt{x^2+x}-x)&
    (2)\displaystyle\lim_{x\rightarrow -\infty}(\frac{\sqrt{3x^2+x}}{x})\\
    (3)\displaystyle\lim_{x\rightarrow +\infty}(\frac{\sqrt{4x^2+2x+1}}{3x})&
    (4)\displaystyle\lim_{x\rightarrow -\infty}(\frac{\sqrt{9x^2+x+3}}{6x})\\
       \end{array}\]
\end{example}
\begin{solution}
    (1)$\displaystyle\lim_{x\rightarrow +\infty}(\sqrt{x^2+x}-x)=\lim_{x\rightarrow +\infty}\frac{x}{\sqrt{x^2+x}+x}=\lim_{x\rightarrow +\infty}\frac{1}{\sqrt{1+\frac{1}{x}}+1}=\frac12$.\\
    (2)$\displaystyle\lim_{x\rightarrow -\infty}(\frac{\sqrt{3x^2+x}}{x})=-\lim_{x\rightarrow -\infty}\sqrt{3+\frac{1}{x^2}}=-\sqrt3$.\\
\end{solution}