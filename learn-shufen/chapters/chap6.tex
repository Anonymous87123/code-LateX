\chapter{积分表}
\section{含有\texorpdfstring{$ax+b$}{ax+b}的积分}
\begin{example}{}{}
\[\int\dfrac{\mathrm{d}x}{ax+b}\]
\end{example}
\vspace{-15pt}
\begin{align*}
\int\dfrac{\mathrm{d}x}{ax+b} &= \frac{1}{a} \int \frac{\mathrm{d}(ax+b)}{ax+b} = \frac{1}{a} \ln|ax+b| + C.
\end{align*}

\begin{example}{}{}
\[\int(ax+b)^n\mathrm{d}x\]
\end{example}
\vspace{-15pt}
\begin{align*}
\int(ax+b)^n\mathrm{d}x &= \frac{1}{a} \int (ax+b)^n \mathrm{d}(ax+b) = \frac{1}{a} \cdot \frac{(ax+b)^{n+1}}{n+1} + C \quad (n \neq -1).
\end{align*}

\begin{example}{}{}
    \[\int\dfrac{x}{ax+b}\mathrm{d}x\]
\end{example}
\vspace{-15pt}
\begin{align*}
\int\dfrac{x}{ax+b}\mathrm{d}x &= \int \frac{1}{a} \left(1 - \frac{b}{ax+b}\right) \mathrm{d}x \\
&= \frac{1}{a} \int \mathrm{d}x - \frac{b}{a} \int \frac{\mathrm{d}x}{ax+b} \\
&= \frac{x}{a} - \frac{b}{a^2} \ln|ax+b| + C.
\end{align*}

\begin{example}{}{}
    \[\int\dfrac{x^2}{ax+b}\mathrm{d}x\]
\end{example}
\vspace{-15pt}
\begin{align*}
\int\dfrac{x^2}{ax+b}\mathrm{d}x &= \dfrac1a\int\dfrac{ax^2}{ax+b}\dd x=\dfrac1a\int\dfrac{ax^2+bx-bx}{ax+b}\dd x\\
&=\dfrac1a\int\left(x-\dfrac{bx}{ax+b}\right)\dd x=\dfrac1a\int\left(x-\dfrac{bx+\frac{b^2}{a}-\frac{b^2}{a}}{ax+b}\right)\dd x\\
&=\dfrac1a\int\left(x-\dfrac{b}{a}+\dfrac{b^2}{a}\dfrac1{ax+b}\right)\dd x\\
&= \frac{1}{a} \int x \mathrm{d}x - \frac{b}{a^2} \int \mathrm{d}x + \frac{b^2}{a^2} \int \frac{\mathrm{d}x}{ax+b} \\
&= \frac{x^2}{2a} - \frac{bx}{a^2} + \frac{b^2}{a^3} \ln|ax+b| + C.
\end{align*}

\begin{example}{}{}
    \[\int\dfrac{\mathrm{d}x}{x(ax+b)}\]
\end{example}
\vspace{-15pt}
\begin{align*}
\int\dfrac{\mathrm{d}x}{x(ax+b)} &= \frac{1}{b} \int \left( \frac{1}{x} - \frac{a}{ax+b} \right) \mathrm{d}x \\
&= \frac{1}{b} \left( \ln|x| - \ln|ax+b| \right) + C = \frac{1}{b} \ln\left| \frac{x}{ax+b} \right| + C.
\end{align*}

\begin{example}{}{}
    \[\int\dfrac{\mathrm{d}x}{x^2(ax+b)}\]
\end{example}
\vspace{-15pt}
\begin{align*}
\int\dfrac{\mathrm{d}x}{x^2(ax+b)} &=\int\left(\dfrac{-\frac{a}{b^2}x+\frac1b}{x^2}+\dfrac{\frac{a^2}{b^2}}{ax+b}\right)\dd x\\
&= \frac{1}{b} \int \left( \frac{a}{x} - \frac{a^2}{ax+b} - \frac{1}{x^2} \right) \mathrm{d}x \\
&= \frac{1}{b} \left( a \ln|x| - a \ln|ax+b| + \frac{1}{x} \right) + C \\
&= \frac{a}{b} \ln\left| \frac{x}{ax+b} \right| + \frac{1}{bx} + C.
\end{align*}

\begin{example}{}{}
    \[\int\dfrac{x\mathrm{d}x}{(ax+b)^2}\]
\end{example}
\vspace{-15pt}
\begin{align*}
\int\dfrac{x\mathrm{d}x}{(ax+b)^2} &=\int\dfrac{x+\frac{b}{a}-\frac{b}{a}}{(ax+b)^2}\dd x\\
&= \int \frac{1}{a} \left( \frac{1}{ax+b} - \frac{b}{(ax+b)^2} \right) \mathrm{d}x \\
&= \frac{1}{a} \int \frac{\mathrm{d}x}{ax+b} - \frac{b}{a} \int \frac{\mathrm{d}x}{(ax+b)^2} \\
&= \frac{1}{a^2} \ln|ax+b| + \frac{b}{a^2} \cdot \frac{1}{ax+b} + C.
\end{align*}

\begin{example}{}{}
    \[\int\dfrac{x^2\mathrm{d}x}{(ax+b)^2}\]
\end{example}
\vspace{-15pt}
\begin{align*}
\int\dfrac{x^2\mathrm{d}x}{(ax+b)^2} &= \dfrac1{a^2}\int\dfrac{a^2x^2\dd x}{(ax+b)^2}=\dfrac1{a^2}\int\dfrac{a^2x^2+2abx+b^2-2abx-b^2}{(ax+b)^2}\dd x\\
&=\dfrac1{a^2}\int\left(1-\dfrac{2abx+b^2}{(ax+b)^2}\right)\dd x=\dfrac1{a^2}\int\left(1-\dfrac{2abx+2b^2-b^2}{(ax+b)^2}\right)\dd x\\
&=\dfrac1{a^2}\int\left(1-\dfrac{2b}{ax+b}+\dfrac{b^2}{(ax+b)^2}\right)\dd x\\
&= \frac{x}{a^2} - \frac{2b}{a^3} \ln|ax+b| - \frac{b^2}{a^3} \cdot \frac{1}{ax+b} + C.
\end{align*}

\begin{example}{}{}
    \[\int\dfrac{\mathrm{d}x}{x(ax+b)^2}\]
\end{example}
\vspace{-15pt}
\begin{align*}
\int\dfrac{\mathrm{d}x}{x(ax+b)^2} &=\dfrac1{b^2}\int\dfrac{b^2}{x(ax+b)^2}\dd x=\dfrac1{b^2}\int\left(\dfrac1x-\dfrac{a^2x+2ab}{(ax+b)^2}\right)\dd x\\
&=\dfrac1{b^2}\int\left(\dfrac1x-\dfrac{a^2x+ab+ab}{(ax+b)^2}\right)\dd x= \frac{1}{b^2} \int \left( \frac{1}{x} - \frac{a}{ax+b} - \frac{ab}{(ax+b)^2} \right) \mathrm{d}x \\
&= \frac{1}{b^2} \left( \ln|x| - \ln|ax+b| + \frac{b}{ax+b} \right) + C \\
&= \frac{1}{b^2} \ln\left| \frac{x}{ax+b} \right| + \frac{1}{b(ax+b)} + C.
\end{align*}

\begin{example}{}{}
    \[\int\dfrac{\mathrm{d}x}{(ax+b)^2}\]
\end{example}
\vspace{-15pt}
\begin{align*}
\int\dfrac{\mathrm{d}x}{(ax+b)^2} &= \frac{1}{a} \int \frac{\mathrm{d}(ax+b)}{(ax+b)^2} = -\frac{1}{a} \cdot \frac{1}{ax+b} + C.
\end{align*}

\begin{example}{}{}
    \[\int\dfrac{\mathrm{d}x}{(ax+b)^3}\]
\end{example}
\vspace{-15pt}
\begin{align*}
\int\dfrac{\mathrm{d}x}{(ax+b)^3} &= \frac{1}{a} \int \frac{\mathrm{d}(ax+b)}{(ax+b)^3} = -\frac{1}{2a} \cdot \frac{1}{(ax+b)^2} + C.
\end{align*}

\section{含有\protect$\sqrt{ax+b}$的积分}
\begin{example}{}{}
    \[\int\sqrt{ax+b}\,\mathrm{d}x\]
\end{example}
\vspace{-15pt}
\begin{align*}
\int\sqrt{ax+b}\,\mathrm{d}x &= \frac{1}{a} \int (ax+b)^{1/2} \,\mathrm{d}(ax+b) = \frac{2}{3a} (ax+b)^{3/2} + C.
\end{align*}

\begin{example}{}{}
    \[\int x\sqrt{ax+b}\,\mathrm{d}x\]
\end{example}
\vspace{-15pt}
\begin{align*}
\int x\sqrt{ax+b}\,\mathrm{d}x &= \frac{1}{a^2} \int ax \sqrt{ax+b} \,\mathrm{d}(ax+b) \\
&= \dfrac{1}{a^2}\int(ax+b-b)\sqrt{ax+b}\dd(ax+b)\\
&= \frac{1}{a^2} \int \left( (ax+b)^{3/2} - b (ax+b)^{1/2} \right) \mathrm{d}(ax+b) \\
&= \frac{1}{a^2} \left( \frac{2}{5} (ax+b)^{5/2} - \frac{2b}{3} (ax+b)^{3/2} \right) + C \\
&= \frac{2(ax+b)^{3/2}(3ax-2b)}{15a^2} + C.
\end{align*}

\begin{example}{}{}
    \[\int x^2\sqrt{ax+b}\,\mathrm{d}x\]
\end{example}
\vspace{-15pt}
\begin{align*}
\int x^2\sqrt{ax+b}\,\mathrm{d}x &=\dfrac1a\int x^2\sqrt{ax+b}\,\mathrm{d}ax\\
&=\int \left(\dfrac{ax+b-b}{a}\right)^2 \sqrt{ax+b} \,\mathrm{d}(ax+b) \\
&= \frac{1}{a} \int \left( \frac{(ax+b)^2 - 2b(ax+b) + b^2}{a^2} \right) \sqrt{ax+b} \,\mathrm{d}(ax+b) \\
&= \frac{1}{a^3} \int \left( (ax+b)^{5/2} - 2b (ax+b)^{3/2} + b^2 (ax+b)^{1/2} \right) \mathrm{d}(ax+b) \\
&= \frac{1}{a^3} \left( \frac{2}{7} (ax+b)^{7/2} - \frac{4b}{5} (ax+b)^{5/2} + \frac{2b^2}{3} (ax+b)^{3/2} \right) + C \\
&= \frac{2(ax+b)^{3/2}(15a^2x^2 - 12abx + 8b^2)}{105a^3} + C.
\end{align*}

\begin{example}{}{}
    \[\int \dfrac{\sqrt{ax+b}}{x}\,\mathrm{d}x\]
\end{example}
\vspace{-15pt}
\begin{align*}
\int \dfrac{\sqrt{ax+b}}{x}\,\mathrm{d}x &=\int\dfrac{ax+b}{x\sqrt{ax+b}}\dd x=\int\dfrac{a\dd x}{\sqrt{ax+b}}+\int\dfrac{b\dd x}{x\sqrt{ax+b}}\\
&= 2\sqrt{ax+b} + b \int \frac{\mathrm{d}x}{x\sqrt{ax+b}}\\
&= 2\sqrt{ax+b} +b\int\left(\dfrac{\sqrt{ax+b}}{x}-\dfrac{a}{\sqrt{ax+b}}\right)\dd x
\end{align*}
对于后一项积分,需分情况讨论:\\
当 $b > 0$ 时,令 $\sqrt{ax+b} = t$,则 $x = \frac{t^2-b}{a}$,$\mathrm{d}x = \frac{2t}{a} \mathrm{d}t$,有
\begin{align*}
\int \frac{\mathrm{d}x}{x\sqrt{ax+b}} &= \int \frac{1}{\frac{t^2-b}{a}} \cdot \frac{2t}{a} \cdot \frac{1}{t} \mathrm{d}t = 2\int \frac{\mathrm{d}t}{t^2-b} \\
&= \begin{cases}
\displaystyle \frac{1}{\sqrt{b}} \ln \left| \frac{t-\sqrt{b}}{t+\sqrt{b}} \right| + C, & b > 0, \\
\displaystyle -\frac{2}{\sqrt{-b}} \arctan \frac{t}{\sqrt{-b}} + C, & b < 0.
\end{cases}
\end{align*}
因此原积分为
\[
\int \dfrac{\sqrt{ax+b}}{x}\,\mathrm{d}x = 2\sqrt{ax+b} + b \cdot \begin{cases}
\displaystyle \frac{1}{\sqrt{b}} \ln \left| \frac{\sqrt{ax+b}-\sqrt{b}}{\sqrt{ax+b}+\sqrt{b}} \right| + C_1, & b > 0, \\
\displaystyle -\frac{2}{\sqrt{-b}} \arctan \frac{\sqrt{ax+b}}{\sqrt{-b}} + C_1, & b < 0.
\end{cases}
\]
若 $b=0$,则积分简化为 $\int \frac{\sqrt{a}}{\sqrt{x}} \mathrm{d}x = 2\sqrt{a}\sqrt{x} + C$.

\begin{example}{}{}
    \[\int \dfrac{\sqrt{ax+b}}{x^2}\,\mathrm{d}x\]
\end{example}
\vspace{-15pt}
\begin{align*}
\int \dfrac{\sqrt{ax+b}}{x^2}\,\mathrm{d}x &= -\frac{\sqrt{ax+b}}{x} + \frac{a}{2} \int \frac{\mathrm{d}x}{x\sqrt{ax+b}} + C.
\end{align*}
后一项积分同上讨论。

\begin{example}{}{}
    \[\int\dfrac{\mathrm{d}x}{\sqrt{ax+b}}\]
\end{example}
\vspace{-15pt}
\begin{align*}
\int\dfrac{\mathrm{d}x}{\sqrt{ax+b}} &= \frac{1}{a} \int (ax+b)^{-1/2} \,\mathrm{d}(ax+b) = \frac{2}{a} \sqrt{ax+b} + C.
\end{align*}

\begin{example}{}{}
    \[\int\dfrac{x\,\mathrm{d}x}{\sqrt{ax+b}}\]
\end{example}
\vspace{-15pt}
\begin{align*}
\int\dfrac{x\,\mathrm{d}x}{\sqrt{ax+b}} &= \frac{1}{a} \int \frac{x}{\sqrt{ax+b}} \,\mathrm{d}(ax+b) \\
&= \frac{1}{a} \int \frac{\frac{ax+b-b}{a}}{\sqrt{ax+b}} \,\mathrm{d}(ax+b) \\
&= \frac{1}{a^2} \int \left( \sqrt{ax+b} - \frac{b}{\sqrt{ax+b}} \right) \mathrm{d}(ax+b) \\
&= \frac{1}{a^2} \left( \frac{2}{3} (ax+b)^{3/2} - 2b \sqrt{ax+b} \right) + C \\
&= \frac{2(ax-2b)\sqrt{ax+b}}{3a^2} + C.
\end{align*}

\begin{example}{}{}
    \[\int\dfrac{x^2\,\mathrm{d}x}{\sqrt{ax+b}}\]
\end{example}
\vspace{-15pt}
\begin{align*}
\int\dfrac{x^2\,\mathrm{d}x}{\sqrt{ax+b}} &= \frac{1}{a} \int \frac{x^2}{\sqrt{ax+b}} \,\mathrm{d}(ax+b) \\
&= \frac{1}{a} \int \frac{\frac{(ax+b)^2 - 2b(ax+b) + b^2}{a^2}}{\sqrt{ax+b}} \,\mathrm{d}(ax+b) \\
&= \frac{1}{a^3} \int \left( (ax+b)^{3/2} - 2b \sqrt{ax+b} + b^2 (ax+b)^{-1/2} \right) \mathrm{d}(ax+b) \\
&= \frac{1}{a^3} \left( \frac{2}{5} (ax+b)^{5/2} - \frac{4b}{3} (ax+b)^{3/2} + 2b^2 \sqrt{ax+b} \right) + C \\
&= \frac{2(3a^2x^2 - 4abx + 8b^2)\sqrt{ax+b}}{15a^3} + C.
\end{align*}

\begin{example}{}{}
    \[\int\dfrac{\mathrm{d}x}{x\sqrt{ax+b}}\]
\end{example}
\vspace{-15pt}
\begin{align*}
\int\dfrac{\mathrm{d}x}{x\sqrt{ax+b}} &= \begin{cases}
\displaystyle \frac{1}{\sqrt{b}} \ln \left| \frac{\sqrt{ax+b}-\sqrt{b}}{\sqrt{ax+b}+\sqrt{b}} \right| + C, & b > 0, \\
\displaystyle \frac{2}{\sqrt{-b}} \arctan \sqrt{\frac{ax+b}{-b}} + C, & b < 0.
\end{cases}
\end{align*}
注:当 $b > 0$ 时,令 $t = \sqrt{ax+b}$,则 $x = \frac{t^2-b}{a}$,$\mathrm{d}x = \frac{2t}{a} \mathrm{d}t$,代入得
\[
\int \frac{\mathrm{d}x}{x\sqrt{ax+b}} = \int \frac{1}{\frac{t^2-b}{a}} \cdot \frac{2t}{a} \cdot \frac{1}{t} \mathrm{d}t = 2\int \frac{\mathrm{d}t}{t^2-b} = \frac{1}{\sqrt{b}} \ln \left| \frac{t-\sqrt{b}}{t+\sqrt{b}} \right| + C.
\]
当 $b < 0$ 时,类似可得反正切形式。

\begin{example}{}{}
    \[\int\dfrac{\mathrm{d}x}{x^2\sqrt{ax+b}}\]
\end{example}
\vspace{-15pt}
\begin{align*}
\int\dfrac{\mathrm{d}x}{x^2\sqrt{ax+b}} &= -\frac{\sqrt{ax+b}}{bx} - \frac{a}{2b} \int \frac{\mathrm{d}x}{x\sqrt{ax+b}} + C.
\end{align*}
其中后一项积分如上所述。

\section{含有\texorpdfstring{$x^2+a^2$}{x^2+a^2}的积分}
\begin{example}{}{}
    \[\int\dfrac{\mathrm{d}x}{x^2+a^2}\]
\end{example}
\vspace{-15pt}
\begin{align*}
\int\dfrac{\mathrm{d}x}{x^2+a^2} = \frac{1}{a}\arctan\frac{x}{a} + C \quad (a>0).
\end{align*}

\begin{example}{}{}
    \[\int\dfrac{x\,\mathrm{d}x}{x^2+a^2}\]
\end{example}
\vspace{-15pt}
\begin{align*}
\int\dfrac{x\,\mathrm{d}x}{x^2+a^2} = \frac{1}{2}\ln(x^2+a^2) + C.
\end{align*}

\begin{example}{}{}
    \[\int\dfrac{\mathrm{d}x}{x(x^2+a^2)}\]
\end{example}
\vspace{-15pt}
\begin{align*}
\int\dfrac{\mathrm{d}x}{x(x^2+a^2)} &= \frac{1}{a^2}\int\left(\frac{1}{x} - \frac{x}{x^2+a^2}\right)\mathrm{d}x \\
&= \frac{1}{a^2}\left(\ln|x| - \frac{1}{2}\ln(x^2+a^2)\right) + C \\
&= \frac{1}{a^2}\ln\left|\frac{x}{\sqrt{x^2+a^2}}\right| + C.
\end{align*}

\begin{example}{}{}
    \[\int\dfrac{\mathrm{d}x}{x^2(x^2+a^2)}\]
\end{example}
\vspace{-15pt}
\begin{align*}
\int\dfrac{\mathrm{d}x}{x^2(x^2+a^2)} &= \frac{1}{a^2}\int\left(\frac{1}{x^2} - \frac{1}{x^2+a^2}\right)\mathrm{d}x \\
&= \frac{1}{a^2}\left(-\frac{1}{x} - \frac{1}{a}\arctan\frac{x}{a}\right) + C \\
&= -\frac{1}{a^2x} - \frac{1}{a^3}\arctan\frac{x}{a} + C.
\end{align*}

\begin{example}{}{}
    \[\int\dfrac{\mathrm{d}x}{x^2-a^2}\]
\end{example}
\vspace{-15pt}
\begin{align*}
\int\dfrac{\mathrm{d}x}{x^2-a^2} = \frac{1}{2a}\ln\left|\frac{x-a}{x+a}\right| + C \quad (a>0, \ |x| \neq a).
\end{align*}

\begin{example}{}{}
    \[\int\dfrac{x\,\mathrm{d}x}{x^2-a^2}\]
\end{example}
\vspace{-15pt}
\begin{align*}
\int\dfrac{x\,\mathrm{d}x}{x^2-a^2} = \frac{1}{2}\ln|x^2-a^2| + C.
\end{align*}

\begin{example}{}{}
    \[\int\dfrac{\mathrm{d}x}{x(x^2-a^2)}\]
\end{example}
\vspace{-15pt}
\begin{align*}
\int\dfrac{\mathrm{d}x}{x(x^2-a^2)} &= \frac{1}{a^2}\int\left(\frac{1}{x} - \frac{x}{x^2-a^2}\right)\mathrm{d}x \\
&= \frac{1}{a^2}\left(\ln|x| - \frac{1}{2}\ln|x^2-a^2|\right) + C \\
&= \frac{1}{a^2}\ln\left|\frac{x}{\sqrt{|x^2-a^2|}}\right| + C.
\end{align*}

\begin{example}{}{}
    \[\int\dfrac{\mathrm{d}x}{x^2(x^2-a^2)}\]
\end{example}
\vspace{-15pt}
\begin{align*}
\int\dfrac{\mathrm{d}x}{x^2(x^2-a^2)} &= \frac{1}{a^2}\int\left(-\frac{1}{x^2} - \frac{1}{x^2-a^2}\right)\mathrm{d}x \\
&= \frac{1}{a^2}\left(\frac{1}{x} - \frac{1}{2a}\ln\left|\frac{x-a}{x+a}\right|\right) + C \\
&= \frac{1}{a^2x} - \frac{1}{2a^3}\ln\left|\frac{x-a}{x+a}\right| + C.
\end{align*}

\begin{example}{}{}
    \[\int\dfrac{\mathrm{d}x}{x^3(x^2-a^2)}\]
\end{example}
\vspace{-15pt}
\begin{align*}
\int\dfrac{\mathrm{d}x}{x^3(x^2-a^2)} &= \frac{1}{a^2}\int\left(-\frac{1}{x^3} - \frac{1}{a^2x} + \frac{x}{a^2(x^2-a^2)}\right)\mathrm{d}x \\
&= \frac{1}{a^2}\left(\frac{1}{2x^2} - \frac{1}{a^2}\ln|x| + \frac{1}{2a^2}\ln|x^2-a^2|\right) + C \\
&= \frac{1}{2a^2x^2} - \frac{1}{a^4}\ln|x| + \frac{1}{2a^4}\ln|x^2-a^2| + C.
\end{align*}

\begin{example}{}{}
    \[\int\dfrac{\mathrm{d}x}{(x^2+a^2)^n}\]
\end{example}
\vspace{-15pt}
\begin{align*}
I_n = \int\dfrac{\mathrm{d}x}{(x^2+a^2)^n} \quad (n \in \mathbb{N}^*).
\end{align*}
利用递推公式:
\begin{align*}
I_n = \frac{x}{2a^2(n-1)(x^2+a^2)^{n-1}} + \frac{2n-3}{2a^2(n-1)} I_{n-1}, \quad n \geq 2,
\end{align*}
其中
\begin{align*}
I_1 = \int\dfrac{\mathrm{d}x}{x^2+a^2} = \frac{1}{a}\arctan\frac{x}{a} + C.
\end{align*}

\begin{example}{}{}
    \[\int\dfrac{\mathrm{d}x}{(x^2-a^2)^n}\]
\end{example}
\vspace{-15pt}
\begin{align*}
J_n = \int\dfrac{\mathrm{d}x}{(x^2-a^2)^n} \quad (n \in \mathbb{N}^*).
\end{align*}
利用递推公式:
\begin{align*}
J_n = -\frac{x}{2a^2(n-1)(x^2-a^2)^{n-1}} + \frac{2n-3}{2a^2(n-1)} J_{n-1}, \quad n \geq 2,
\end{align*}
其中
\begin{align*}
J_1 = \int\dfrac{\mathrm{d}x}{x^2-a^2} = \frac{1}{2a}\ln\left|\frac{x-a}{x+a}\right| + C.
\end{align*}


\section{含有\texorpdfstring{$ax^2+bx+c$}{ax^2+bx+c}的积分}
\begin{example}{}{}
\[\int\dfrac{\mathrm{d}x}{ax^2+bx+c}\]
\end{example}
\vspace{-15pt}
\begin{align*}
\int\dfrac{\mathrm{d}x}{ax^2+bx+c} &= \frac{1}{a} \int \frac{\mathrm{d}x}{x^2 + \frac{b}{a}x + \frac{c}{a}} \\
&= \frac{1}{a} \int \frac{\mathrm{d}x}{\left(x + \frac{b}{2a}\right)^2 + \frac{4ac - b^2}{4a^2}}.
\end{align*}
令 $t = x + \frac{b}{2a}$,$\Delta = b^2 - 4ac$,则
\[
\int\dfrac{\mathrm{d}x}{ax^2+bx+c} = \frac{1}{a} \int \frac{\mathrm{d}t}{t^2 + \frac{4ac - b^2}{4a^2}}.
\]
分情况讨论:
\begin{enumerate}
    \item 若 $\Delta < 0$,则 $4ac - b^2 > 0$,记 $k^2 = \frac{4ac - b^2}{4a^2}$,
    \[
    \int\dfrac{\mathrm{d}x}{ax^2+bx+c} = \frac{1}{a} \cdot \frac{1}{k} \arctan \frac{t}{k} + C = \frac{2}{\sqrt{4ac - b^2}} \arctan \frac{2ax + b}{\sqrt{4ac - b^2}} + C.
    \]
    \item 若 $\Delta = 0$,则
    \[
    \int\dfrac{\mathrm{d}x}{ax^2+bx+c} = \frac{1}{a} \int \frac{\mathrm{d}t}{t^2} = -\frac{1}{a t} + C = -\frac{2}{2ax + b} + C.
    \]
    \item 若 $\Delta > 0$,则分母可分解为两个不同实根 $x_1, x_2$,
    \[
    \int\dfrac{\mathrm{d}x}{ax^2+bx+c} = \frac{1}{a(x_1 - x_2)} \ln \left| \frac{x - x_1}{x - x_2} \right| + C = \frac{1}{\sqrt{\Delta}} \ln \left| \frac{2ax + b - \sqrt{\Delta}}{2ax + b + \sqrt{\Delta}} \right| + C.
    \]
\end{enumerate}

\begin{example}{}{}
\[\int\dfrac{x\,\mathrm{d}x}{ax^2+bx+c}\]
\end{example}
\vspace{-15pt}
\begin{align*}
\int\dfrac{x\,\mathrm{d}x}{ax^2+bx+c} &= \frac{1}{2a} \int \frac{2ax + b - b}{ax^2+bx+c} \,\mathrm{d}x \\
&= \frac{1}{2a} \int \frac{2ax + b}{ax^2+bx+c} \,\mathrm{d}x - \frac{b}{2a} \int \frac{\mathrm{d}x}{ax^2+bx+c}.
\end{align*}
其中
\[
\int \frac{2ax + b}{ax^2+bx+c} \,\mathrm{d}x = \ln|ax^2+bx+c| + C_1,
\]
而 $\displaystyle \int \frac{\mathrm{d}x}{ax^2+bx+c}$ 的结果如上所述。因此
\[
\int\dfrac{x\,\mathrm{d}x}{ax^2+bx+c} = \frac{1}{2a} \ln|ax^2+bx+c| - \frac{b}{2a} \int \frac{\mathrm{d}x}{ax^2+bx+c} + C.
\]
具体形式根据判别式代入即可。

\section{含有\texorpdfstring{$\sqrt{x^2+a^2}$}{\sqrt{x^2+a^2}}的积分}
\begin{example}{}{}
    \[\int\dfrac{\mathrm{d}x}{\sqrt{x^2+a^2}}\]
\end{example}
\vspace{-15pt}
\begin{align*}
\int\dfrac{\mathrm{d}x}{\sqrt{x^2+a^2}} &\xlongequal{x=a\tan t} \int \frac{a\sec^2 t}{\sqrt{a^2\tan^2 t+a^2}} \mathrm{d}t = \int \frac{a\sec^2 t}{a\sec t} \mathrm{d}t \\
&= \int \sec t \,\mathrm{d}t = \ln|\sec t + \tan t| + C \\
&\xlongequal{t=\arctan(x/a)} \ln\left| \frac{\sqrt{x^2+a^2}}{a} + \frac{x}{a} \right| + C \\
&= \ln\left| x + \sqrt{x^2+a^2} \right| + C' \quad (C' = C - \ln a).
\end{align*}

\begin{example}{}{}
    \[\int\dfrac{x\,\mathrm{d}x}{\sqrt{x^2+a^2}}\]
\end{example}
\vspace{-15pt}
\begin{align*}
\int\dfrac{x\,\mathrm{d}x}{\sqrt{x^2+a^2}} &\xlongequal{u=x^2+a^2} \frac{1}{2} \int \frac{\mathrm{d}u}{\sqrt{u}} = \sqrt{u} + C = \sqrt{x^2+a^2} + C.
\end{align*}

\begin{example}{}{}
    \[\int\dfrac{x^2\mathrm{d}x}{\sqrt{x^2+a^2}}\]
\end{example}
\vspace{-15pt}
\begin{align*}
\int\dfrac{x^2\mathrm{d}x}{\sqrt{x^2+a^2}} &\xlongequal{x=a\tan t} \int \frac{a^2\tan^2 t \cdot a\sec^2 t}{a\sec t} \mathrm{d}t = a^2 \int \tan^2 t \sec t \,\mathrm{d}t \\
&= a^2 \int (\sec^2 t - 1) \sec t \,\mathrm{d}t = a^2 \left( \int \sec^3 t \,\mathrm{d}t - \int \sec t \,\mathrm{d}t \right) \\
&= a^2 \left( \frac{1}{2} \sec t \tan t + \frac{1}{2} \ln|\sec t + \tan t| - \ln|\sec t + \tan t| \right) + C \\
&= \frac{a^2}{2} \left( \sec t \tan t - \ln|\sec t + \tan t| \right) + C \\
&\xlongequal{t=\arctan(x/a)} \frac{a^2}{2} \left( \frac{\sqrt{x^2+a^2}}{a} \cdot \frac{x}{a} - \ln\left| \frac{\sqrt{x^2+a^2}}{a} + \frac{x}{a} \right| \right) + C \\
&= \frac{x}{2} \sqrt{x^2+a^2} - \frac{a^2}{2} \ln\left| x + \sqrt{x^2+a^2} \right| + C'.
\end{align*}

\begin{example}{}{}
    \[\int\dfrac{\mathrm{d}x}{x\sqrt{x^2+a^2}}\]
\end{example}
\vspace{-15pt}
\begin{align*}
\int\dfrac{\mathrm{d}x}{x\sqrt{x^2+a^2}} &\xlongequal{x=a\tan t} \int \frac{a\sec^2 t}{a\tan t \cdot a\sec t} \mathrm{d}t = \frac{1}{a} \int \frac{\sec t}{\tan t} \mathrm{d}t \\
&= \frac{1}{a} \int \csc t \,\mathrm{d}t = \frac{1}{a} \ln|\csc t - \cot t| + C \\
&\xlongequal{t=\arctan(x/a)} \frac{1}{a} \ln\left| \frac{\sqrt{x^2+a^2}}{x} - \frac{a}{x} \right| + C \\
&= \frac{1}{a} \ln\left| \frac{\sqrt{x^2+a^2} - a}{x} \right| + C.
\end{align*}

\begin{example}{}{}
    \[\int\dfrac{\mathrm{d}x}{x^2\sqrt{x^2+a^2}}\]
\end{example}
\vspace{-15pt}
\begin{align*}
\int\dfrac{\mathrm{d}x}{x^2\sqrt{x^2+a^2}} &\xlongequal{x=a\tan t} \int \frac{a\sec^2 t}{a^2\tan^2 t \cdot a\sec t} \mathrm{d}t = \frac{1}{a^2} \int \frac{\sec t}{\tan^2 t} \mathrm{d}t \\
&= \frac{1}{a^2} \int \frac{\cos t}{\sin^2 t} \mathrm{d}t = \frac{1}{a^2} \int \csc t \cot t \,\mathrm{d}t = -\frac{1}{a^2} \csc t + C \\
&\xlongequal{t=\arctan(x/a)} -\frac{1}{a^2} \cdot \frac{\sqrt{x^2+a^2}}{x} + C = -\frac{\sqrt{x^2+a^2}}{a^2 x} + C.
\end{align*}

\begin{example}{}{}
    \[\int\dfrac{\mathrm{d}x}{\sqrt{(x^2+a^2)^3}}\]
\end{example}
\vspace{-15pt}
\begin{align*}
\int\dfrac{\mathrm{d}x}{\sqrt{(x^2+a^2)^3}} &\xlongequal{x=a\tan t} \int \frac{a\sec^2 t}{(a^3 \sec^3 t)} \mathrm{d}t = \frac{1}{a^2} \int \cos t \,\mathrm{d}t \\
&= \frac{1}{a^2} \sin t + C \xlongequal{t=\arctan(x/a)} \frac{1}{a^2} \cdot \frac{x}{\sqrt{x^2+a^2}} + C = \frac{x}{a^2\sqrt{x^2+a^2}} + C.
\end{align*}

\begin{example}{}{}
    \[\int\dfrac{x\,\mathrm{d}x}{\sqrt{(x^2+a^2)^3}}\]
\end{example}
\vspace{-15pt}
\begin{align*}
\int\dfrac{x\,\mathrm{d}x}{\sqrt{(x^2+a^2)^3}} &\xlongequal{u=x^2+a^2} \frac{1}{2} \int u^{-3/2} \mathrm{d}u = \frac{1}{2} \cdot (-2) u^{-1/2} + C = -\frac{1}{\sqrt{u}} + C \\
&= -\frac{1}{\sqrt{x^2+a^2}} + C.
\end{align*}

\begin{example}{}{}
    \[\int\dfrac{x^2\mathrm{d}x}{\sqrt{(x^2+a^2)^3}}\]
\end{example}
\vspace{-15pt}
\begin{align*}
\int\dfrac{x^2\mathrm{d}x}{\sqrt{(x^2+a^2)^3}} &\xlongequal{x=a\tan t} \int \frac{a^2\tan^2 t \cdot a\sec^2 t}{(a^3 \sec^3 t)} \mathrm{d}t = \frac{1}{a} \int \frac{\tan^2 t}{\sec t} \mathrm{d}t \\
&= \frac{1}{a} \int \sin t \tan t \,\mathrm{d}t = \frac{1}{a} \int \frac{\sin^2 t}{\cos t} \mathrm{d}t = \frac{1}{a} \int \frac{1-\cos^2 t}{\cos t} \mathrm{d}t \\
&= \frac{1}{a} \int (\sec t - \cos t) \mathrm{d}t = \frac{1}{a} (\ln|\sec t + \tan t| - \sin t) + C \\
&\xlongequal{t=\arctan(x/a)} \frac{1}{a} \left( \ln\left| \frac{\sqrt{x^2+a^2}}{a} + \frac{x}{a} \right| - \frac{x}{\sqrt{x^2+a^2}} \right) + C \\
&= \frac{1}{a} \ln\left| x + \sqrt{x^2+a^2} \right| - \frac{x}{a\sqrt{x^2+a^2}} + C'.
\end{align*}

\begin{example}{}{}
    \[\int\sqrt{x^2+a^2}\,\mathrm{d}x\]
\end{example}
\vspace{-15pt}
\begin{align*}
\int\sqrt{x^2+a^2}\,\mathrm{d}x &\xlongequal{x=a\tan t} \int a\sec t \cdot a\sec^2 t \,\mathrm{d}t = a^2 \int \sec^3 t \,\mathrm{d}t \\
&= \frac{a^2}{2} (\sec t \tan t + \ln|\sec t + \tan t|) + C \\
&\xlongequal{t=\arctan(x/a)} \frac{a^2}{2} \left( \frac{\sqrt{x^2+a^2}}{a} \cdot \frac{x}{a} + \ln\left| \frac{\sqrt{x^2+a^2}}{a} + \frac{x}{a} \right| \right) + C \\
&= \frac{x}{2} \sqrt{x^2+a^2} + \frac{a^2}{2} \ln\left| x + \sqrt{x^2+a^2} \right| + C'.
\end{align*}

\begin{example}{}{}
    \[\int x\sqrt{x^2+a^2}\,\mathrm{d}x\]
\end{example}
\vspace{-15pt}
\begin{align*}
\int x\sqrt{x^2+a^2}\,\mathrm{d}x &\xlongequal{u=x^2+a^2} \frac{1}{2} \int \sqrt{u} \,\mathrm{d}u = \frac{1}{2} \cdot \frac{2}{3} u^{3/2} + C = \frac{1}{3} (x^2+a^2)^{3/2} + C.
\end{align*}

\begin{example}{}{}
    \[\int x^2\sqrt{x^2+a^2}\,\mathrm{d}x\]
\end{example}
\vspace{-15pt}
\begin{align*}
\int x^2\sqrt{x^2+a^2}\,\mathrm{d}x &\xlongequal{x=a\tan t} \int a^2\tan^2 t \cdot a\sec t \cdot a\sec^2 t \,\mathrm{d}t = a^4 \int \tan^2 t \sec^3 t \,\mathrm{d}t \\
&= a^4 \int (\sec^2 t - 1) \sec^3 t \,\mathrm{d}t = a^4 \left( \int \sec^5 t \,\mathrm{d}t - \int \sec^3 t \,\mathrm{d}t \right).
\end{align*}
利用递推公式或分部积分,最终可得:
\begin{align*}
\int x^2\sqrt{x^2+a^2}\,\mathrm{d}x &= \frac{x}{8} (2x^2+a^2) \sqrt{x^2+a^2} - \frac{a^4}{8} \ln\left| x + \sqrt{x^2+a^2} \right| + C.
\end{align*}

\begin{example}{}{}
    \[\int \dfrac{\sqrt{x^2+a^2}}{x}\,\mathrm{d}x\]
\end{example}
\vspace{-15pt}
\begin{align*}
\int \dfrac{\sqrt{x^2+a^2}}{x}\,\mathrm{d}x &\xlongequal{x=a\tan t} \int \frac{a\sec t}{a\tan t} \cdot a\sec^2 t \,\mathrm{d}t = a \int \frac{\sec^3 t}{\tan t} \mathrm{d}t \\
&= a \int \frac{1}{\cos^3 t} \cdot \frac{\cos t}{\sin t} \mathrm{d}t = a \int \frac{1}{\cos^2 t \sin t} \mathrm{d}t \\
&= a \int \frac{\sin^2 t + \cos^2 t}{\cos^2 t \sin t} \mathrm{d}t = a \int \left( \frac{\sin t}{\cos^2 t} + \frac{\cos t}{\sin t} \right) \mathrm{d}t \\
&= a \left( \int \frac{\sin t}{\cos^2 t} \mathrm{d}t + \int \cot t \,\mathrm{d}t \right) \\
&= a \left( \sec t + \ln|\sin t| \right) + C \\
&\xlongequal{t=\arctan(x/a)} a \left( \frac{\sqrt{x^2+a^2}}{a} + \ln\left| \frac{x}{\sqrt{x^2+a^2}} \right| \right) + C \\
&= \sqrt{x^2+a^2} + a \ln\left| \frac{x}{\sqrt{x^2+a^2}} \right| + C.
\end{align*}

\begin{example}{}{}
    \[\int \dfrac{\sqrt{x^2+a^2}}{x^2}\,\mathrm{d}x\]
\end{example}
\vspace{-15pt}
\begin{align*}
\int \dfrac{\sqrt{x^2+a^2}}{x^2}\,\mathrm{d}x &\xlongequal{x=a\tan t} \int \frac{a\sec t}{a^2\tan^2 t} \cdot a\sec^2 t \,\mathrm{d}t = \frac{1}{a} \int \frac{\sec^3 t}{\tan^2 t} \mathrm{d}t \\
&= \frac{1}{a} \int \frac{1}{\cos^3 t} \cdot \frac{\cos^2 t}{\sin^2 t} \mathrm{d}t = \frac{1}{a} \int \frac{1}{\cos t \sin^2 t} \mathrm{d}t \\
&= \frac{1}{a} \int \frac{\sin^2 t + \cos^2 t}{\cos t \sin^2 t} \mathrm{d}t = \frac{1}{a} \int \left( \frac{\sin t}{\cos t} + \frac{\cos t}{\sin^2 t} \right) \mathrm{d}t \\
&= \frac{1}{a} \left( \int \tan t \,\mathrm{d}t + \int \cot t \csc t \,\mathrm{d}t \right) \\
&= \frac{1}{a} \left( -\ln|\cos t| - \csc t \right) + C \\
&\xlongequal{t=\arctan(x/a)} \frac{1}{a} \left( -\ln\left| \frac{a}{\sqrt{x^2+a^2}} \right| - \frac{\sqrt{x^2+a^2}}{x} \right) + C \\
&= -\frac{\sqrt{x^2+a^2}}{a x} + \frac{1}{a} \ln\left| \frac{\sqrt{x^2+a^2}}{a} \right| + C.
\end{align*}

\begin{example}{}{}
    \[\int\sqrt{(x^2+a^2)^3}\,\mathrm{d}x\]
\end{example}
\vspace{-15pt}
\begin{align*}
\int\sqrt{(x^2+a^2)^3}\,\mathrm{d}x &= \int (x^2+a^2)^{3/2} \mathrm{d}x \xlongequal{x=a\tan t} \int (a^2\sec^2 t)^{3/2} \cdot a\sec^2 t \,\mathrm{d}t \\
&= a^4 \int \sec^5 t \,\mathrm{d}t.
\end{align*}
利用递推公式:
\begin{align*}
\int \sec^n t \,\mathrm{d}t = \frac{\sec^{n-2} t \tan t}{n-1} + \frac{n-2}{n-1} \int \sec^{n-2} t \,\mathrm{d}t,
\end{align*}
可得:
\begin{align*}
\int \sec^5 t \,\mathrm{d}t &= \frac{\sec^3 t \tan t}{4} + \frac{3}{4} \int \sec^3 t \,\mathrm{d}t \\
&= \frac{\sec^3 t \tan t}{4} + \frac{3}{4} \left( \frac{\sec t \tan t}{2} + \frac{1}{2} \ln|\sec t + \tan t| \right) + C.
\end{align*}
代回并整理得:
\begin{align*}
\int\sqrt{(x^2+a^2)^3}\,\mathrm{d}x &= \frac{x}{8} (2x^2+5a^2) \sqrt{x^2+a^2} + \frac{3a^4}{8} \ln\left| x + \sqrt{x^2+a^2} \right| + C.
\end{align*}

\begin{example}{}{}
    \[\int\dfrac{\sqrt{(x^2+a^2)^3}}{x}\,\mathrm{d}x\]
\end{example}
\vspace{-15pt}
\begin{align*}
\int\dfrac{\sqrt{(x^2+a^2)^3}}{x}\,\mathrm{d}x &= \int \frac{(x^2+a^2)^{3/2}}{x} \mathrm{d}x \xlongequal{u=x^2+a^2} \frac{1}{2} \int \frac{u^{3/2}}{u-a^2} \mathrm{d}u.
\end{align*}
然后进行有理化代换或分部积分,最终可得:
\begin{align*}
\int\dfrac{\sqrt{(x^2+a^2)^3}}{x}\,\mathrm{d}x &= \frac{1}{3} (x^2+a^2)^{3/2} + a^2 \sqrt{x^2+a^2} - a^3 \ln\left| \frac{a+\sqrt{x^2+a^2}}{x} \right| + C.
\end{align*}

\begin{example}{}{}
    \[\int\dfrac{\sqrt{(x^2+a^2)^3}}{x^2}\,\mathrm{d}x\]
\end{example}
\vspace{-15pt}
\begin{align*}
\int\dfrac{\sqrt{(x^2+a^2)^3}}{x^2}\,\mathrm{d}x &\xlongequal{x=a\tan t} \int \frac{a^3\sec^3 t}{a^2\tan^2 t} \cdot a\sec^2 t \,\mathrm{d}t = a^2 \int \frac{\sec^5 t}{\tan^2 t} \mathrm{d}t \\
&= a^2 \int \frac{\sec^3 t}{\sin^2 t} \mathrm{d}t = a^2 \int \frac{1}{\cos^3 t \sin^2 t} \mathrm{d}t.
\end{align*}
该积分较为复杂,最终表达式为:
\begin{align*}
\int\dfrac{\sqrt{(x^2+a^2)^3}}{x^2}\,\mathrm{d}x &= -\frac{\sqrt{(x^2+a^2)^3}}{x} + \frac{3}{2} x \sqrt{x^2+a^2} + \frac{3}{2} a^2 \ln\left| x + \sqrt{x^2+a^2} \right| + C.
\end{align*}

\begin{example}{}{}
    \[\int x\sqrt{(x^2+a^2)^3}\,\mathrm{d}x\]
\end{example}
\vspace{-15pt}
\begin{align*}
\int x\sqrt{(x^2+a^2)^3}\,\mathrm{d}x &\xlongequal{u=x^2+a^2} \frac{1}{2} \int u^{3/2} \mathrm{d}u = \frac{1}{2} \cdot \frac{2}{5} u^{5/2} + C = \frac{1}{5} (x^2+a^2)^{5/2} + C.
\end{align*}

\begin{example}{}{}
    \[\int x^2\sqrt{(x^2+a^2)^3}\,\mathrm{d}x\]
\end{example}
\vspace{-15pt}
\begin{align*}
\int x^2\sqrt{(x^2+a^2)^3}\,\mathrm{d}x &= \int x^2 (x^2+a^2)^{3/2} \mathrm{d}x \xlongequal{x=a\tan t} \int a^2\tan^2 t \cdot a^3\sec^3 t \cdot a\sec^2 t \,\mathrm{d}t \\
&= a^6 \int \tan^2 t \sec^5 t \,\mathrm{d}t = a^6 \int (\sec^2 t - 1) \sec^5 t \,\mathrm{d}t \\
&= a^6 \left( \int \sec^7 t \,\mathrm{d}t - \int \sec^5 t \,\mathrm{d}t \right).
\end{align*}
利用递推公式,最终可得:
\begin{align*}
\int x^2\sqrt{(x^2+a^2)^3}\,\mathrm{d}x &= \frac{x}{48} (8x^4+26a^2x^2+33a^4) \sqrt{x^2+a^2} + \frac{5a^6}{16} \ln\left| x + \sqrt{x^2+a^2} \right| + C.
\end{align*}

\section{含有\texorpdfstring{$\sqrt{x^2-a^2}$}{\sqrt{x^2-a^2}}的积分}
\begin{example}{}{}
    \[\int\dfrac{\mathrm{d}x}{\sqrt{x^2-a^2}}\]
\end{example}
\vspace{-15pt}
\begin{align*}
\int\dfrac{\mathrm{d}x}{\sqrt{x^2-a^2}} &\xlongequal{x=a\sec t} \int \frac{a\sec t\tan t \,\mathrm{d}t}{\sqrt{a^2\sec^2 t - a^2}} = \int \frac{a\sec t\tan t}{a\tan t} \,\mathrm{d}t \\
&= \int \sec t \,\mathrm{d}t = \ln|\sec t + \tan t| + C \\
&\xlongequal{t=\arccos(a/x)} \ln\left| \frac{x}{a} + \frac{\sqrt{x^2-a^2}}{a} \right| + C \\
&= \ln\left| x + \sqrt{x^2-a^2} \right| + C' \quad (C' = C - \ln a).
\end{align*}
注意:通常写作 \(\displaystyle \int\dfrac{\mathrm{d}x}{\sqrt{x^2-a^2}} = \ln\left| x + \sqrt{x^2-a^2} \right| + C\).

\begin{example}{}{}
    \[\int\dfrac{x\,\mathrm{d}x}{\sqrt{x^2-a^2}}\]
\end{example}
\vspace{-15pt}
\begin{align*}
\int\dfrac{x\,\mathrm{d}x}{\sqrt{x^2-a^2}} &\xlongequal{u=x^2-a^2} \frac{1}{2} \int \frac{\mathrm{d}u}{\sqrt{u}} = \frac{1}{2} \cdot 2\sqrt{u} + C = \sqrt{x^2-a^2} + C.
\end{align*}

\begin{example}{}{}
    \[\int\dfrac{x^2\mathrm{d}x}{\sqrt{x^2-a^2}}\]
\end{example}
\vspace{-15pt}
\begin{align*}
\int\dfrac{x^2\mathrm{d}x}{\sqrt{x^2-a^2}} &\xlongequal{x=a\sec t} \int \frac{a^2\sec^2 t \cdot a\sec t\tan t \,\mathrm{d}t}{\sqrt{a^2\sec^2 t - a^2}} = \int \frac{a^3\sec^3 t\tan t}{a\tan t} \,\mathrm{d}t \\
&= a^2 \int \sec^3 t \,\mathrm{d}t = a^2 \left( \frac{1}{2} \sec t \tan t + \frac{1}{2} \ln|\sec t + \tan t| \right) + C \\
&\xlongequal{t=\arccos(a/x)} a^2 \left( \frac{1}{2} \cdot \frac{x}{a} \cdot \frac{\sqrt{x^2-a^2}}{a} + \frac{1}{2} \ln\left| \frac{x}{a} + \frac{\sqrt{x^2-a^2}}{a} \right| \right) + C \\
&= \frac{x}{2} \sqrt{x^2-a^2} + \frac{a^2}{2} \ln\left| x + \sqrt{x^2-a^2} \right| + C'.
\end{align*}

\begin{example}{}{}
    \[\int\dfrac{\mathrm{d}x}{x\sqrt{x^2-a^2}}\]
\end{example}
\vspace{-15pt}
\begin{align*}
\int\dfrac{\mathrm{d}x}{x\sqrt{x^2-a^2}} &\xlongequal{x=a\sec t} \int \frac{a\sec t\tan t \,\mathrm{d}t}{a\sec t \cdot a\tan t} = \frac{1}{a} \int \mathrm{d}t = \frac{t}{a} + C \\
&\xlongequal{t=\arccos(a/x)} \frac{1}{a} \arccos\frac{a}{x} + C.
\end{align*}
也可写作:\(\displaystyle \int\dfrac{\mathrm{d}x}{x\sqrt{x^2-a^2}} = \frac{1}{a} \sec^{-1}\left|\frac{x}{a}\right| + C\).

\begin{example}{}{}
    \[\int\dfrac{\mathrm{d}x}{x^2\sqrt{x^2-a^2}}\]
\end{example}
\vspace{-15pt}
\begin{align*}
\int\dfrac{\mathrm{d}x}{x^2\sqrt{x^2-a^2}} &\xlongequal{x=a\sec t} \int \frac{a\sec t\tan t \,\mathrm{d}t}{a^2\sec^2 t \cdot a\tan t} = \frac{1}{a^2} \int \cos t \,\mathrm{d}t = \frac{1}{a^2} \sin t + C \\
&\xlongequal{t=\arccos(a/x)} \frac{1}{a^2} \cdot \frac{\sqrt{x^2-a^2}}{x} + C = \frac{\sqrt{x^2-a^2}}{a^2 x} + C.
\end{align*}

\begin{example}{}{}
    \[\int\dfrac{\mathrm{d}x}{\sqrt{(x^2-a^2)^3}}\]
\end{example}
\vspace{-15pt}
\begin{align*}
\int\dfrac{\mathrm{d}x}{\sqrt{(x^2-a^2)^3}} &\xlongequal{x=a\sec t} \int \frac{a\sec t\tan t \,\mathrm{d}t}{(a^3 \tan^3 t)} = \frac{1}{a^2} \int \frac{\sec t}{\tan^2 t} \,\mathrm{d}t = \frac{1}{a^2} \int \frac{\cos t}{\sin^2 t} \,\mathrm{d}t \\
&= \frac{1}{a^2} \int \csc t \cot t \,\mathrm{d}t = -\frac{1}{a^2} \csc t + C \\
&\xlongequal{t=\arccos(a/x)} -\frac{1}{a^2} \cdot \frac{x}{\sqrt{x^2-a^2}} + C = -\frac{x}{a^2\sqrt{x^2-a^2}} + C.
\end{align*}

\begin{example}{}{}
    \[\int\dfrac{x\,\mathrm{d}x}{\sqrt{(x^2-a^2)^3}}\]
\end{example}
\vspace{-15pt}
\begin{align*}
\int\dfrac{x\,\mathrm{d}x}{\sqrt{(x^2-a^2)^3}} &\xlongequal{u=x^2-a^2} \frac{1}{2} \int u^{-3/2} \,\mathrm{d}u = \frac{1}{2} \cdot (-2) u^{-1/2} + C = -\frac{1}{\sqrt{u}} + C \\
&= -\frac{1}{\sqrt{x^2-a^2}} + C.
\end{align*}

\begin{example}{}{}
    \[\int\dfrac{x^2\mathrm{d}x}{\sqrt{(x^2-a^2)^3}}\]
\end{example}
\vspace{-15pt}
\begin{align*}
\int\dfrac{x^2\mathrm{d}x}{\sqrt{(x^2-a^2)^3}} &\xlongequal{x=a\sec t} \int \frac{a^2\sec^2 t \cdot a\sec t\tan t \,\mathrm{d}t}{(a^3 \tan^3 t)} = \frac{1}{a} \int \frac{\sec^3 t}{\tan^2 t} \,\mathrm{d}t = \frac{1}{a} \int \frac{1}{\cos^3 t} \cdot \frac{\cos^2 t}{\sin^2 t} \,\mathrm{d}t \\
&= \frac{1}{a} \int \frac{1}{\cos t \sin^2 t} \,\mathrm{d}t = \frac{1}{a} \int \frac{\sin^2 t + \cos^2 t}{\cos t \sin^2 t} \,\mathrm{d}t \\
&= \frac{1}{a} \int \left( \frac{\sin t}{\cos t} + \frac{\cos t}{\sin^2 t} \right) \mathrm{d}t = \frac{1}{a} \left( \int \tan t \,\mathrm{d}t + \int \cot t \csc t \,\mathrm{d}t \right) \\
&= \frac{1}{a} \left( -\ln|\cos t| - \csc t \right) + C \\
&\xlongequal{t=\arccos(a/x)} \frac{1}{a} \left( -\ln\left| \frac{a}{x} \right| - \frac{x}{\sqrt{x^2-a^2}} \right) + C \\
&= \frac{1}{a} \ln\left| \frac{x}{a} \right| - \frac{x}{a\sqrt{x^2-a^2}} + C \\
&= \frac{1}{a} \ln|x| - \frac{x}{a\sqrt{x^2-a^2}} + C'.
\end{align*}

\begin{example}{}{}
    \[\int\sqrt{x^2-a^2}\,\mathrm{d}x\]
\end{example}
\vspace{-15pt}
\begin{align*}
\int\sqrt{x^2-a^2}\,\mathrm{d}x &\xlongequal{x=a\sec t} \int \sqrt{a^2\sec^2 t - a^2} \cdot a\sec t\tan t \,\mathrm{d}t = \int a\tan t \cdot a\sec t\tan t \,\mathrm{d}t \\
&= a^2 \int \sec t \tan^2 t \,\mathrm{d}t = a^2 \int \sec t (\sec^2 t - 1) \,\mathrm{d}t \\
&= a^2 \left( \int \sec^3 t \,\mathrm{d}t - \int \sec t \,\mathrm{d}t \right).
\end{align*}
利用 \(\int \sec^3 t \,\mathrm{d}t = \frac{1}{2} (\sec t \tan t + \ln|\sec t + \tan t|) + C\) 和 \(\int \sec t \,\mathrm{d}t = \ln|\sec t + \tan t| + C\),得
\[
\int\sqrt{x^2-a^2}\,\mathrm{d}x = a^2 \left( \frac{1}{2} \sec t \tan t - \frac{1}{2} \ln|\sec t + \tan t| \right) + C.
\]
代回 \(t = \arccos(a/x)\),\(\sec t = x/a\),\(\tan t = \sqrt{x^2-a^2}/a\):
\begin{align*}
\int\sqrt{x^2-a^2}\,\mathrm{d}x &= a^2 \left( \frac{1}{2} \cdot \frac{x}{a} \cdot \frac{\sqrt{x^2-a^2}}{a} - \frac{1}{2} \ln\left| \frac{x}{a} + \frac{\sqrt{x^2-a^2}}{a} \right| \right) + C \\
&= \frac{x}{2} \sqrt{x^2-a^2} - \frac{a^2}{2} \ln\left| x + \sqrt{x^2-a^2} \right| + C'.
\end{align*}

\begin{example}{}{}
    \[\int x\sqrt{x^2-a^2}\,\mathrm{d}x\]
\end{example}
\vspace{-15pt}
\begin{align*}
\int x\sqrt{x^2-a^2}\,\mathrm{d}x &\xlongequal{u=x^2-a^2} \frac{1}{2} \int \sqrt{u} \,\mathrm{d}u = \frac{1}{2} \cdot \frac{2}{3} u^{3/2} + C = \frac{1}{3} (x^2-a^2)^{3/2} + C.
\end{align*}

\begin{example}{}{}
    \[\int x^2\sqrt{x^2-a^2}\,\mathrm{d}x\]
\end{example}
\vspace{-15pt}
\begin{align*}
\int x^2\sqrt{x^2-a^2}\,\mathrm{d}x &\xlongequal{x=a\sec t} \int a^2\sec^2 t \cdot a\tan t \cdot a\sec t\tan t \,\mathrm{d}t = a^4 \int \sec^3 t \tan^2 t \,\mathrm{d}t \\
&= a^4 \int \sec^3 t (\sec^2 t - 1) \,\mathrm{d}t = a^4 \left( \int \sec^5 t \,\mathrm{d}t - \int \sec^3 t \,\mathrm{d}t \right).
\end{align*}
利用递推公式计算,最终可得:
\[
\int x^2\sqrt{x^2-a^2}\,\mathrm{d}x = \frac{x}{8} (2x^2-a^2) \sqrt{x^2-a^2} - \frac{a^4}{8} \ln\left| x + \sqrt{x^2-a^2} \right| + C.
\]

\begin{example}{}{}
    \[\int \dfrac{\sqrt{x^2-a^2}}{x}\,\mathrm{d}x\]
\end{example}
\vspace{-15pt}
\begin{align*}
\int \dfrac{\sqrt{x^2-a^2}}{x}\,\mathrm{d}x &\xlongequal{x=a\sec t} \int \frac{a\tan t}{a\sec t} \cdot a\sec t\tan t \,\mathrm{d}t = a \int \tan^2 t \,\mathrm{d}t = a \int (\sec^2 t - 1) \,\mathrm{d}t \\
&= a (\tan t - t) + C \\
&\xlongequal{t=\arccos(a/x)} a \left( \frac{\sqrt{x^2-a^2}}{a} - \arccos\frac{a}{x} \right) + C \\
&= \sqrt{x^2-a^2} - a \arccos\frac{a}{x} + C.
\end{align*}
也可写作:\(\displaystyle \int \dfrac{\sqrt{x^2-a^2}}{x}\,\mathrm{d}x = \sqrt{x^2-a^2} - a \sec^{-1}\left|\frac{x}{a}\right| + C\).

\begin{example}{}{}
    \[\int \dfrac{\sqrt{x^2-a^2}}{x^2}\,\mathrm{d}x\]
\end{example}
\vspace{-15pt}
\begin{align*}
\int \dfrac{\sqrt{x^2-a^2}}{x^2}\,\mathrm{d}x &\xlongequal{x=a\sec t} \int \frac{a\tan t}{a^2\sec^2 t} \cdot a\sec t\tan t \,\mathrm{d}t = \frac{1}{a} \int \frac{\tan^2 t}{\sec t} \,\mathrm{d}t = \frac{1}{a} \int \frac{\sin^2 t}{\cos t} \,\mathrm{d}t \\
&= \frac{1}{a} \int \frac{1-\cos^2 t}{\cos t} \,\mathrm{d}t = \frac{1}{a} \int (\sec t - \cos t) \,\mathrm{d}t \\
&= \frac{1}{a} (\ln|\sec t + \tan t| - \sin t) + C \\
&\xlongequal{t=\arccos(a/x)} \frac{1}{a} \left( \ln\left| \frac{x}{a} + \frac{\sqrt{x^2-a^2}}{a} \right| - \frac{\sqrt{x^2-a^2}}{x} \right) + C \\
&= \frac{1}{a} \ln\left| x + \sqrt{x^2-a^2} \right| - \frac{\sqrt{x^2-a^2}}{a x} + C'.
\end{align*}

\begin{example}{}{}
    \[\int\sqrt{(x^2-a^2)^3}\,\mathrm{d}x\]
\end{example}
\vspace{-15pt}
\begin{align*}
\int\sqrt{(x^2-a^2)^3}\,\mathrm{d}x &= \int (x^2-a^2)^{3/2} \mathrm{d}x \xlongequal{x=a\sec t} \int (a^2\tan^2 t)^{3/2} \cdot a\sec t\tan t \,\mathrm{d}t \\
&= \int a^3|\tan^3 t| \cdot a\sec t\tan t \,\mathrm{d}t \quad (\text{假设 } x>a>0, \text{则 } \tan t>0) \\
&= a^4 \int \tan^4 t \sec t \,\mathrm{d}t = a^4 \int \tan^2 t \cdot \tan^2 t \sec t \,\mathrm{d}t \\
&= a^4 \int (\sec^2 t - 1) \sec t \tan^2 t \,\mathrm{d}t = a^4 \left( \int \sec^3 t \tan^2 t \,\mathrm{d}t - \int \sec t \tan^2 t \,\mathrm{d}t \right).
\end{align*}
利用递推,最终可得:
\[
\int\sqrt{(x^2-a^2)^3}\,\mathrm{d}x = \frac{x}{8} (2x^2-5a^2) \sqrt{x^2-a^2} + \frac{3a^4}{8} \ln\left| x + \sqrt{x^2-a^2} \right| + C.
\]

\begin{example}{}{}
    \[\int\dfrac{\sqrt{(x^2-a^2)^3}}{x}\,\mathrm{d}x\]
\end{example}
\vspace{-15pt}
\begin{align*}
\int\dfrac{\sqrt{(x^2-a^2)^3}}{x}\,\mathrm{d}x &= \int \frac{(x^2-a^2)^{3/2}}{x} \mathrm{d}x \xlongequal{x=a\sec t} \int \frac{a^3\tan^3 t}{a\sec t} \cdot a\sec t\tan t \,\mathrm{d}t \\
&= a^3 \int \tan^4 t \,\mathrm{d}t = a^3 \int (\sec^2 t - 1)^2 \,\mathrm{d}t = a^3 \int (\sec^4 t - 2\sec^2 t + 1) \,\mathrm{d}t.
\end{align*}
计算各项:
\[
\int \sec^4 t \,\mathrm{d}t = \int \sec^2 t \sec^2 t \,\mathrm{d}t = \int (1+\tan^2 t) \,\mathrm{d}(\tan t) = \tan t + \frac{1}{3}\tan^3 t + C,
\]
\[
\int \sec^2 t \,\mathrm{d}t = \tan t + C, \quad \int 1 \,\mathrm{d}t = t + C.
\]
所以
\[
\int\dfrac{\sqrt{(x^2-a^2)^3}}{x}\,\mathrm{d}x = a^3 \left( \tan t + \frac{1}{3}\tan^3 t - 2\tan t + t \right) + C = a^3 \left( t - \tan t + \frac{1}{3}\tan^3 t \right) + C.
\]
代回 \(t = \arccos(a/x)\),\(\tan t = \sqrt{x^2-a^2}/a\):
\begin{align*}
\int\dfrac{\sqrt{(x^2-a^2)^3}}{x}\,\mathrm{d}x &= a^3 \left( \arccos\frac{a}{x} - \frac{\sqrt{x^2-a^2}}{a} + \frac{1}{3} \left( \frac{\sqrt{x^2-a^2}}{a} \right)^3 \right) + C \\
&= a^3 \arccos\frac{a}{x} - a^2 \sqrt{x^2-a^2} + \frac{1}{3} (x^2-a^2)^{3/2} + C.
\end{align*}

\begin{example}{}{}
    \[\int\dfrac{\sqrt{(x^2-a^2)^3}}{x^2}\,\mathrm{d}x\]
\end{example}
\vspace{-15pt}
\[
\int\dfrac{\sqrt{(x^2-a^2)^3}}{x^2}\,\mathrm{d}x = -\frac{\sqrt{(x^2-a^2)^3}}{x} + \frac{3}{2} x \sqrt{x^2-a^2} - \frac{3}{2} a^2 \ln\left| x + \sqrt{x^2-a^2} \right| + C.
\]

\begin{example}{}{}
    \[\int x\sqrt{(x^2-a^2)^3}\,\mathrm{d}x\]
\end{example}
\vspace{-15pt}
\begin{align*}
\int x\sqrt{(x^2-a^2)^3}\,\mathrm{d}x &\xlongequal{u=x^2-a^2} \frac{1}{2} \int u^{3/2} \,\mathrm{d}u = \frac{1}{2} \cdot \frac{2}{5} u^{5/2} + C = \frac{1}{5} (x^2-a^2)^{5/2} + C.
\end{align*}

\begin{example}{}{}
    \[\int x^2\sqrt{(x^2-a^2)^3}\,\mathrm{d}x\]
\end{example}
\vspace{-15pt}
\begin{align*}
\int x^2\sqrt{(x^2-a^2)^3}\,\mathrm{d}x &= \int x^2 (x^2-a^2)^{3/2} \mathrm{d}x \\
&\xlongequal{x=a\sec t} \int a^2\sec^2 t \cdot a^3\tan^3 t \cdot a\sec t\tan t \,\mathrm{d}t \\
&= a^6 \int \sec^3 t \tan^4 t \,\mathrm{d}t = a^6 \int \sec^3 t (\sec^2 t - 1)^2 \,\mathrm{d}t \\
&= a^6 \int \sec^3 t (\sec^4 t - 2\sec^2 t + 1) \,\mathrm{d}t \\
&= a^6 \left( \int \sec^7 t \,\mathrm{d}t - 2\int \sec^5 t \,\mathrm{d}t + \int \sec^3 t \,\mathrm{d}t \right).
\end{align*}
利用递推公式,可得最终表达式,但较冗长。通常用分部积分或递推,此处略去详细步骤。

\begin{example}{}{}
    \[\int\dfrac{\mathrm{d}x}{\sqrt{a^2-x^2}}\]
\end{example}
\vspace{-15pt}
\begin{align*}
\int\dfrac{\mathrm{d}x}{\sqrt{a^2-x^2}} &\xlongequal{x=a\sin t} \int \frac{a\cos t \,\mathrm{d}t}{\sqrt{a^2 - a^2\sin^2 t}} = \int \frac{a\cos t}{a\cos t} \,\mathrm{d}t = \int \mathrm{d}t = t + C \\
&\xlongequal{t=\arcsin(x/a)} \arcsin\frac{x}{a} + C \quad (|x| < a).
\end{align*}
也可以表示为:\(\displaystyle \int\dfrac{\mathrm{d}x}{\sqrt{a^2-x^2}} = \arcsin\frac{x}{a} + C\).

\begin{example}{}{}
    \[\int\dfrac{x\,\mathrm{d}x}{\sqrt{a^2-x^2}}\]
\end{example}
\vspace{-15pt}
\begin{align*}
\int\dfrac{x\,\mathrm{d}x}{\sqrt{a^2-x^2}} &\xlongequal{u=a^2-x^2} \frac{1}{2} \int \frac{-\mathrm{d}u}{\sqrt{u}} = -\frac{1}{2} \int u^{-1/2} \,\mathrm{d}u = -\sqrt{u} + C = -\sqrt{a^2-x^2} + C.
\end{align*}

\begin{example}{}{}
    \[\int\dfrac{x^2\mathrm{d}x}{\sqrt{a^2-x^2}}\]
\end{example}
\vspace{-15pt}
\begin{align*}
\int\dfrac{x^2\mathrm{d}x}{\sqrt{a^2-x^2}} &\xlongequal{x=a\sin t} \int \frac{a^2\sin^2 t \cdot a\cos t \,\mathrm{d}t}{\sqrt{a^2 - a^2\sin^2 t}} = \int \frac{a^3\sin^2 t \cos t}{a\cos t} \,\mathrm{d}t \\
&= a^2 \int \sin^2 t \,\mathrm{d}t = a^2 \int \frac{1-\cos 2t}{2} \,\mathrm{d}t = \frac{a^2}{2} \left( t - \frac{\sin 2t}{2} \right) + C \\
&= \frac{a^2}{2} \left( t - \sin t \cos t \right) + C \\
&\xlongequal{t=\arcsin(x/a)} \frac{a^2}{2} \left( \arcsin\frac{x}{a} - \frac{x}{a} \cdot \frac{\sqrt{a^2-x^2}}{a} \right) + C \\
&= \frac{a^2}{2} \arcsin\frac{x}{a} - \frac{x}{2} \sqrt{a^2-x^2} + C.
\end{align*}

\begin{example}{}{}
    \[\int\dfrac{\mathrm{d}x}{x\sqrt{a^2-x^2}}\]
\end{example}
\vspace{-15pt}
\begin{align*}
\int\dfrac{\mathrm{d}x}{x\sqrt{a^2-x^2}} &\xlongequal{x=a\sin t} \int \frac{a\cos t \,\mathrm{d}t}{a\sin t \cdot a\cos t} = \frac{1}{a} \int \csc t \,\mathrm{d}t = \frac{1}{a} \ln|\csc t - \cot t| + C \\
&\xlongequal{t=\arcsin(x/a)} \frac{1}{a} \ln\left| \frac{a}{x} - \frac{\sqrt{a^2-x^2}}{x} \right| + C = \frac{1}{a} \ln\left| \frac{a - \sqrt{a^2-x^2}}{x} \right| + C.
\end{align*}
也可写作:\(\displaystyle \int\dfrac{\mathrm{d}x}{x\sqrt{a^2-x^2}} = -\frac{1}{a} \ln\left| \frac{a+\sqrt{a^2-x^2}}{x} \right| + C\).

\begin{example}{}{}
    \[\int\dfrac{\mathrm{d}x}{x^2\sqrt{a^2-x^2}}\]
\end{example}
\vspace{-15pt}
\begin{align*}
\int\dfrac{\mathrm{d}x}{x^2\sqrt{a^2-x^2}} &\xlongequal{x=a\sin t} \int \frac{a\cos t \,\mathrm{d}t}{a^2\sin^2 t \cdot a\cos t} = \frac{1}{a^2} \int \csc^2 t \,\mathrm{d}t = -\frac{1}{a^2} \cot t + C \\
&\xlongequal{t=\arcsin(x/a)} -\frac{1}{a^2} \cdot \frac{\sqrt{a^2-x^2}}{x} + C = -\frac{\sqrt{a^2-x^2}}{a^2 x} + C.
\end{align*}

\begin{example}{}{}
    \[\int\dfrac{\mathrm{d}x}{\sqrt{(a^2-x^2)^3}}\]
\end{example}
\vspace{-15pt}
\begin{align*}
\int\dfrac{\mathrm{d}x}{\sqrt{(a^2-x^2)^3}} &\xlongequal{x=a\sin t} \int \frac{a\cos t \,\mathrm{d}t}{(a^3 \cos^3 t)} = \frac{1}{a^2} \int \sec^2 t \,\mathrm{d}t = \frac{1}{a^2} \tan t + C \\
&\xlongequal{t=\arcsin(x/a)} \frac{1}{a^2} \cdot \frac{x}{\sqrt{a^2-x^2}} + C = \frac{x}{a^2\sqrt{a^2-x^2}} + C.
\end{align*}

\begin{example}{}{}
    \[\int\dfrac{x\,\mathrm{d}x}{\sqrt{(a^2-x^2)^3}}\]
\end{example}
\vspace{-15pt}
\begin{align*}
\int\dfrac{x\,\mathrm{d}x}{\sqrt{(a^2-x^2)^3}} &\xlongequal{u=a^2-x^2} \frac{1}{2} \int \frac{-\mathrm{d}u}{u^{3/2}} = -\frac{1}{2} \int u^{-3/2} \,\mathrm{d}u = -\frac{1}{2} \cdot (-2) u^{-1/2} + C \\
&= \frac{1}{\sqrt{u}} + C = \frac{1}{\sqrt{a^2-x^2}} + C.
\end{align*}

\begin{example}{}{}
    \[\int\dfrac{x^2\mathrm{d}x}{\sqrt{(a^2-x^2)^3}}\]
\end{example}
\vspace{-15pt}
\begin{align*}
\int\dfrac{x^2\mathrm{d}x}{\sqrt{(a^2-x^2)^3}} &\xlongequal{x=a\sin t} \int \frac{a^2\sin^2 t \cdot a\cos t \,\mathrm{d}t}{(a^3 \cos^3 t)} = \frac{1}{a} \int \frac{\sin^2 t}{\cos^2 t} \,\mathrm{d}t = \frac{1}{a} \int \tan^2 t \,\mathrm{d}t \\
&= \frac{1}{a} \int (\sec^2 t - 1) \,\mathrm{d}t = \frac{1}{a} (\tan t - t) + C \\
&\xlongequal{t=\arcsin(x/a)} \frac{1}{a} \left( \frac{x}{\sqrt{a^2-x^2}} - \arcsin\frac{x}{a} \right) + C.
\end{align*}

\begin{example}{}{}
    \[\int\sqrt{a^2-x^2}\,\mathrm{d}x\]
\end{example}
\vspace{-15pt}
\begin{align*}
\int\sqrt{a^2-x^2}\,\mathrm{d}x &\xlongequal{x=a\sin t} \int \sqrt{a^2 - a^2\sin^2 t} \cdot a\cos t \,\mathrm{d}t = \int a\cos t \cdot a\cos t \,\mathrm{d}t \\
&= a^2 \int \cos^2 t \,\mathrm{d}t = a^2 \int \frac{1+\cos 2t}{2} \,\mathrm{d}t = \frac{a^2}{2} \left( t + \frac{\sin 2t}{2} \right) + C \\
&= \frac{a^2}{2} \left( t + \sin t \cos t \right) + C \\
&\xlongequal{t=\arcsin(x/a)} \frac{a^2}{2} \left( \arcsin\frac{x}{a} + \frac{x}{a} \cdot \frac{\sqrt{a^2-x^2}}{a} \right) + C \\
&= \frac{a^2}{2} \arcsin\frac{x}{a} + \frac{x}{2} \sqrt{a^2-x^2} + C.
\end{align*}

\begin{example}{}{}
    \[\int x\sqrt{a^2-x^2}\,\mathrm{d}x\]
\end{example}
\vspace{-15pt}
\begin{align*}
\int x\sqrt{a^2-x^2}\,\mathrm{d}x &\xlongequal{u=a^2-x^2} \frac{1}{2} \int \sqrt{u} \, (-\mathrm{d}u) = -\frac{1}{2} \int u^{1/2} \,\mathrm{d}u = -\frac{1}{2} \cdot \frac{2}{3} u^{3/2} + C \\
&= -\frac{1}{3} (a^2-x^2)^{3/2} + C.
\end{align*}

\begin{example}{}{}
    \[\int x^2\sqrt{a^2-x^2}\,\mathrm{d}x\]
\end{example}
\vspace{-15pt}
\begin{align*}
\int x^2\sqrt{a^2-x^2}\,\mathrm{d}x &\xlongequal{x=a\sin t} \int a^2\sin^2 t \cdot a\cos t \cdot a\cos t \,\mathrm{d}t = a^4 \int \sin^2 t \cos^2 t \,\mathrm{d}t \\
&= a^4 \int \frac{1}{4} \sin^2 2t \,\mathrm{d}t = \frac{a^4}{4} \int \frac{1-\cos 4t}{2} \,\mathrm{d}t = \frac{a^4}{8} \left( t - \frac{\sin 4t}{4} \right) + C \\
&= \frac{a^4}{8} \left( t - \frac{1}{4} \cdot 2\sin 2t \cos 2t \right) + C = \frac{a^4}{8} \left( t - \frac{1}{2} \sin 2t \cos 2t \right) + C.
\end{align*}
利用 \(\sin 2t = 2\sin t \cos t = 2 \cdot \frac{x}{a} \cdot \frac{\sqrt{a^2-x^2}}{a} = \frac{2x\sqrt{a^2-x^2}}{a^2}\),\(\cos 2t = 1-2\sin^2 t = 1-2\frac{x^2}{a^2}\),代入化简得:
\begin{align*}
\int x^2\sqrt{a^2-x^2}\,\mathrm{d}x &= \frac{a^4}{8} \left( \arcsin\frac{x}{a} - \frac{1}{2} \cdot \frac{2x\sqrt{a^2-x^2}}{a^2} \cdot \left(1-2\frac{x^2}{a^2}\right) \right) + C \\
&= \frac{a^4}{8} \arcsin\frac{x}{a} - \frac{x}{8} \sqrt{a^2-x^2} \left( a^2 - 2x^2 \right) + C.
\end{align*}
也可整理为:
\begin{align*}
\int x^2\sqrt{a^2-x^2}\,\mathrm{d}x &= \frac{x}{8} (2x^2 - a^2) \sqrt{a^2-x^2} + \frac{a^4}{8} \arcsin\frac{x}{a} + C.
\end{align*}

\begin{example}{}{}
    \[\int \dfrac{\sqrt{a^2-x^2}}{x}\,\mathrm{d}x\]
\end{example}
\vspace{-15pt}
\begin{align*}
\int \dfrac{\sqrt{a^2-x^2}}{x}\,\mathrm{d}x &\xlongequal{x=a\sin t} \int \frac{a\cos t}{a\sin t} \cdot a\cos t \,\mathrm{d}t = a \int \frac{\cos^2 t}{\sin t} \,\mathrm{d}t \\
&= a \int \frac{1-\sin^2 t}{\sin t} \,\mathrm{d}t = a \int (\csc t - \sin t) \,\mathrm{d}t \\
&= a \left( \ln|\csc t - \cot t| + \cos t \right) + C \\
&\xlongequal{t=\arcsin(x/a)} a \left( \ln\left| \frac{a}{x} - \frac{\sqrt{a^2-x^2}}{x} \right| + \frac{\sqrt{a^2-x^2}}{a} \right) + C \\
&= a \ln\left| \frac{a - \sqrt{a^2-x^2}}{x} \right| + \sqrt{a^2-x^2} + C.
\end{align*}

\begin{example}{}{}
    \[\int \dfrac{\sqrt{a^2-x^2}}{x^2}\,\mathrm{d}x\]
\end{example}
\vspace{-15pt}
\begin{align*}
\int \dfrac{\sqrt{a^2-x^2}}{x^2}\,\mathrm{d}x &\xlongequal{x=a\sin t} \int \frac{a\cos t}{a^2\sin^2 t} \cdot a\cos t \,\mathrm{d}t = \frac{1}{a} \int \frac{\cos^2 t}{\sin^2 t} \,\mathrm{d}t \\
&= \frac{1}{a} \int \cot^2 t \,\mathrm{d}t = \frac{1}{a} \int (\csc^2 t - 1) \,\mathrm{d}t \\
&= \frac{1}{a} (-\cot t - t) + C \\
&\xlongequal{t=\arcsin(x/a)} -\frac{1}{a} \left( \frac{\sqrt{a^2-x^2}}{x} + \arcsin\frac{x}{a} \right) + C \\
&= -\frac{\sqrt{a^2-x^2}}{a x} - \frac{1}{a} \arcsin\frac{x}{a} + C.
\end{align*}

\begin{example}{}{}
    \[\int\sqrt{(a^2-x^2)^3}\,\mathrm{d}x\]
\end{example}
\vspace{-15pt}
\begin{align*}
\int\sqrt{(a^2-x^2)^3}\,\mathrm{d}x &= \int (a^2-x^2)^{3/2} \mathrm{d}x \xlongequal{x=a\sin t} \int (a^2\cos^2 t)^{3/2} \cdot a\cos t \,\mathrm{d}t \\
&= \int a^3\cos^3 t \cdot a\cos t \,\mathrm{d}t = a^4 \int \cos^4 t \,\mathrm{d}t.
\end{align*}
利用 \(\cos^4 t = \left(\frac{1+\cos 2t}{2}\right)^2 = \frac{1}{4}(1 + 2\cos 2t + \cos^2 2t) = \frac{1}{4}\left(1 + 2\cos 2t + \frac{1+\cos 4t}{2}\right) = \frac{1}{8}(3 + 4\cos 2t + \cos 4t)\),所以
\begin{align*}
\int \cos^4 t \,\mathrm{d}t &= \frac{1}{8} \int (3 + 4\cos 2t + \cos 4t) \,\mathrm{d}t = \frac{1}{8} \left( 3t + 2\sin 2t + \frac{1}{4}\sin 4t \right) + C.
\end{align*}
又 \(\sin 2t = 2\sin t \cos t = 2 \cdot \frac{x}{a} \cdot \frac{\sqrt{a^2-x^2}}{a} = \frac{2x\sqrt{a^2-x^2}}{a^2}\),\(\sin 4t = 2\sin 2t \cos 2t = 2 \cdot \frac{2x\sqrt{a^2-x^2}}{a^2} \cdot \left(1-2\frac{x^2}{a^2}\right)\),代入并整理得:
\begin{align*}
\int\sqrt{(a^2-x^2)^3}\,\mathrm{d}x &= \frac{a^4}{8} \left( 3\arcsin\frac{x}{a} + \frac{2x\sqrt{a^2-x^2}}{a^2} \left(5 - \frac{2x^2}{a^2}\right) \right) + C \\
&= \frac{3a^4}{8} \arcsin\frac{x}{a} + \frac{x}{8} \sqrt{a^2-x^2} (5a^2 - 2x^2) + C.
\end{align*}
也可写作:
\begin{align*}
\int\sqrt{(a^2-x^2)^3}\,\mathrm{d}x &= \frac{x}{8} (5a^2 - 2x^2) \sqrt{a^2-x^2} + \frac{3a^4}{8} \arcsin\frac{x}{a} + C.
\end{align*}

\begin{example}{}{}
    \[\int\dfrac{\sqrt{(a^2-x^2)^3}}{x}\,\mathrm{d}x\]
\end{example}
\vspace{-15pt}
\begin{align*}
\int\dfrac{\sqrt{(a^2-x^2)^3}}{x}\,\mathrm{d}x &= \int \frac{(a^2-x^2)^{3/2}}{x} \mathrm{d}x \xlongequal{x=a\sin t} \int \frac{a^3\cos^3 t}{a\sin t} \cdot a\cos t \,\mathrm{d}t \\
&= a^3 \int \frac{\cos^4 t}{\sin t} \,\mathrm{d}t = a^3 \int \frac{(1-\sin^2 t)^2}{\sin t} \,\mathrm{d}t \\
&= a^3 \int \left( \csc t - 2\sin t + \sin^3 t \right) \mathrm{d}t.
\end{align*}
计算各项:
\begin{align*}
\int \csc t \,\mathrm{d}t &= \ln|\csc t - \cot t|, \\
\int \sin t \,\mathrm{d}t &= -\cos t, \\
\int \sin^3 t \,\mathrm{d}t &= \int (1-\cos^2 t) \sin t \,\mathrm{d}t = -\cos t + \frac{\cos^3 t}{3}.
\end{align*}
所以
\begin{align*}
\int\dfrac{\sqrt{(a^2-x^2)^3}}{x}\,\mathrm{d}x &= a^3 \left( \ln|\csc t - \cot t| + 2\cos t - \cos t + \frac{\cos^3 t}{3} \right) + C \\
&= a^3 \left( \ln|\csc t - \cot t| + \cos t + \frac{\cos^3 t}{3} \right) + C.
\end{align*}
代回 \(t = \arcsin(x/a)\),\(\cos t = \sqrt{1-\frac{x^2}{a^2}} = \frac{\sqrt{a^2-x^2}}{a}\),\(\csc t = \frac{a}{x}\),\(\cot t = \frac{\sqrt{a^2-x^2}}{x}\):
\begin{align*}
\int\dfrac{\sqrt{(a^2-x^2)^3}}{x}\,\mathrm{d}x &= a^3 \left( \ln\left| \frac{a}{x} - \frac{\sqrt{a^2-x^2}}{x} \right| + \frac{\sqrt{a^2-x^2}}{a} + \frac{1}{3} \left( \frac{\sqrt{a^2-x^2}}{a} \right)^3 \right) + C \\
&= a^3 \ln\left| \frac{a - \sqrt{a^2-x^2}}{x} \right| + a^2 \sqrt{a^2-x^2} + \frac{1}{3} (a^2-x^2)^{3/2} + C.
\end{align*}
整理得:
\begin{align*}
\int\dfrac{\sqrt{(a^2-x^2)^3}}{x}\,\mathrm{d}x &= \frac{1}{3} (a^2-x^2)^{3/2} + a^2 \sqrt{a^2-x^2} + a^3 \ln\left| \frac{a - \sqrt{a^2-x^2}}{x} \right| + C.
\end{align*}

\begin{example}{}{}
    \[\int\dfrac{\sqrt{(a^2-x^2)^3}}{x^2}\,\mathrm{d}x\]
\end{example}
\vspace{-15pt}
\begin{align*}
\int\dfrac{\sqrt{(a^2-x^2)^3}}{x^2}\,\mathrm{d}x &\xlongequal{x=a\sin t} \int \frac{a^3\cos^3 t}{a^2\sin^2 t} \cdot a\cos t \,\mathrm{d}t = a^2 \int \frac{\cos^4 t}{\sin^2 t} \,\mathrm{d}t \\
&= a^2 \int \frac{(1-\sin^2 t)^2}{\sin^2 t} \,\mathrm{d}t = a^2 \int \left( \csc^2 t - 2 + \sin^2 t \right) \mathrm{d}t \\
&= a^2 \left( -\cot t - 2t + \int \sin^2 t \,\mathrm{d}t \right).
\end{align*}
而 \(\int \sin^2 t \,\mathrm{d}t = \frac{t}{2} - \frac{\sin 2t}{4} = \frac{t}{2} - \frac{\sin t \cos t}{2}\),所以
\begin{align*}
\int\dfrac{\sqrt{(a^2-x^2)^3}}{x^2}\,\mathrm{d}x &= a^2 \left( -\cot t - 2t + \frac{t}{2} - \frac{\sin t \cos t}{2} \right) + C \\
&= a^2 \left( -\cot t - \frac{3t}{2} - \frac{\sin t \cos t}{2} \right) + C.
\end{align*}
代回 \(t = \arcsin(x/a)\),\(\cot t = \frac{\sqrt{a^2-x^2}}{x}\),\(\sin t = \frac{x}{a}\),\(\cos t = \frac{\sqrt{a^2-x^2}}{a}\):
\begin{align*}
\int\dfrac{\sqrt{(a^2-x^2)^3}}{x^2}\,\mathrm{d}x &= a^2 \left( -\frac{\sqrt{a^2-x^2}}{x} - \frac{3}{2} \arcsin\frac{x}{a} - \frac{1}{2} \cdot \frac{x}{a} \cdot \frac{\sqrt{a^2-x^2}}{a} \right) + C \\
&= -\frac{a^2 \sqrt{a^2-x^2}}{x} - \frac{3a^2}{2} \arcsin\frac{x}{a} - \frac{x}{2} \sqrt{a^2-x^2} + C.
\end{align*}
整理得:
\begin{align*}
\int\dfrac{\sqrt{(a^2-x^2)^3}}{x^2}\,\mathrm{d}x &= -\frac{\sqrt{a^2-x^2}}{x} \left( a^2 + \frac{x^2}{2} \right) - \frac{3a^2}{2} \arcsin\frac{x}{a} + C.
\end{align*}

\begin{example}{}{}
    \[\int x\sqrt{(a^2-x^2)^3}\,\mathrm{d}x\]
\end{example}
\vspace{-15pt}
\begin{align*}
\int x\sqrt{(a^2-x^2)^3}\,\mathrm{d}x &\xlongequal{u=a^2-x^2} \frac{1}{2} \int u^{3/2} (-\mathrm{d}u) = -\frac{1}{2} \int u^{3/2} \,\mathrm{d}u = -\frac{1}{2} \cdot \frac{2}{5} u^{5/2} + C \\
&= -\frac{1}{5} (a^2-x^2)^{5/2} + C.
\end{align*}

\begin{example}{}{}
    \[\int x^2\sqrt{(a^2-x^2)^3}\,\mathrm{d}x\]
\end{example}
\vspace{-15pt}
\begin{align*}
\int x^2\sqrt{(a^2-x^2)^3}\,\mathrm{d}x &= \int x^2 (a^2-x^2)^{3/2} \mathrm{d}x \xlongequal{x=a\sin t} \int a^2\sin^2 t \cdot a^3\cos^3 t \cdot a\cos t \,\mathrm{d}t \\
&= a^6 \int \sin^2 t \cos^4 t \,\mathrm{d}t = a^6 \int \sin^2 t (1-\sin^2 t)^2 \,\mathrm{d}t \\
&= a^6 \int (\sin^2 t - 2\sin^4 t + \sin^6 t) \,\mathrm{d}t.
\end{align*}
这些积分可以通过递推公式或利用三角恒等式计算,但过程较长。最终结果为:
\begin{align*}
&\int x^2\sqrt{(a^2-x^2)^3}\,\mathrm{d}x \\
&= -\frac{x}{8} (2x^2 - 5a^2) \sqrt{(a^2-x^2)^3} + \frac{3a^6}{16} \left( \arcsin\frac{x}{a} - \frac{x}{a^2} \sqrt{a^2-x^2} (2x^2 - 3a^2) \right) + C.
\end{align*}
为简洁,通常用递推公式或分部积分得出,代入前面 \(\int\sqrt{(a^2-x^2)^3}\,\mathrm{d}x\) 的结果可得完整表达式。
\begin{align*}
\int x^2\sqrt{(a^2-x^2)^3}\,\mathrm{d}x &= -\frac{x}{6} (a^2-x^2)^{5/2} + \frac{a^2}{6} \int \sqrt{(a^2-x^2)^3} \,\mathrm{d}x + C.
\end{align*}


\section{含有\texorpdfstring{$\sqrt{ax^2+bx+c}$}{\sqrt{-ax^2+bx+c}}和的积分}
\begin{example}{}{}
    \[\int\sqrt{ax^2+bx+c}\,\mathrm{d}x\]
\end{example}
\vspace{-15pt}
\begin{align*}
\int\sqrt{ax^2+bx+c}\,\mathrm{d}x &= \int \sqrt{a\left( x^2+\frac{b}{a}x+\frac{c}{a} \right)} \,\mathrm{d}x \\
&= \sqrt{a} \int \sqrt{ \left( x+\frac{b}{2a} \right)^2 + \frac{4ac-b^2}{4a^2} } \,\mathrm{d}x \quad (\text{假设 } a>0) \\
&\xlongequal{u=x+\frac{b}{2a},\ \Delta=4ac-b^2} \sqrt{a} \int \sqrt{ u^2 + \frac{\Delta}{4a^2} } \,\mathrm{d}u.
\end{align*}
记 \( k = \frac{\sqrt{|\Delta|}}{2|a|} \)。分情况讨论:
\begin{enumerate}
    \item 若 \(\Delta > 0\),则 \( \sqrt{a} \int \sqrt{u^2 + k^2} \,\mathrm{d}u \),其中 \(k=\frac{\sqrt{\Delta}}{2a}\)。利用公式:
    \[
    \int \sqrt{u^2+k^2} \,\mathrm{d}u = \frac{u}{2}\sqrt{u^2+k^2} + \frac{k^2}{2} \ln\left| u+\sqrt{u^2+k^2} \right| + C.
    \]
    \item 若 \(\Delta = 0\),则积分简化为 \(\sqrt{a} \int |u| \,\mathrm{d}u = \frac{\sqrt{a}}{2} u |u| + C\),但注意 \(u=x+b/(2a)\),原被积函数为 \(\sqrt{a} |u|\)。
    \item 若 \(\Delta < 0\),则 \( \sqrt{a} \int \sqrt{u^2 - k^2} \,\mathrm{d}u \),其中 \(k=\frac{\sqrt{-\Delta}}{2a}\)。利用公式:
    \[
    \int \sqrt{u^2-k^2} \,\mathrm{d}u = \frac{u}{2}\sqrt{u^2-k^2} - \frac{k^2}{2} \ln\left| u+\sqrt{u^2-k^2} \right| + C.
    \]
\end{enumerate}
最终结果用 \(x\) 表示,这里不展开全部。若 \(a<0\),则被开方数为负,积分可能仅在定义域内实数,此时通常提取 \(-a\) 并处理为 \(\sqrt{-a}\sqrt{-\left( x^2+\frac{b}{a}x+\frac{c}{a} \right)}\),转化为第二种类型。

\begin{example}{}{}
    \[\int\sqrt{-ax^2+bx+c}\,\mathrm{d}x\]
\end{example}
\vspace{-15pt}
\begin{align*}
\int\sqrt{-ax^2+bx+c}\,\mathrm{d}x &= \int \sqrt{ -a\left( x^2 - \frac{b}{a}x - \frac{c}{a} \right) } \,\mathrm{d}x \quad (\text{假设 } a>0) \\
&= \sqrt{a} \int \sqrt{ - \left( x^2 - \frac{b}{a}x - \frac{c}{a} \right) } \,\mathrm{d}x \\
&= \sqrt{a} \int \sqrt{ k^2 - \left( x - \frac{b}{2a} \right)^2 } \,\mathrm{d}x,
\end{align*}
其中完成平方后,设 \( k^2 = \frac{b^2+4ac}{4a^2} \)(需保证被开方数非负)。令 \( u = x - \frac{b}{2a} \),则积分化为
\[
\sqrt{a} \int \sqrt{k^2 - u^2} \,\mathrm{d}u = \sqrt{a} \left( \frac{u}{2} \sqrt{k^2-u^2} + \frac{k^2}{2} \arcsin\frac{u}{k} \right) + C,
\]
其中 \(|u| \leq k\)。代回 \(u\) 和 \(k\) 即得结果。

\begin{example}{}{}
    \[\int x\sqrt{ax^2+bx+c}\,\mathrm{d}x\]
\end{example}
\vspace{-15pt}
\begin{align*}
\int x\sqrt{ax^2+bx+c}\,\mathrm{d}x &= \frac{1}{2a} \int (2ax+b-b) \sqrt{ax^2+bx+c} \,\mathrm{d}x \\
&= \frac{1}{2a} \int (2ax+b) \sqrt{ax^2+bx+c} \,\mathrm{d}x - \frac{b}{2a} \int \sqrt{ax^2+bx+c} \,\mathrm{d}x.
\end{align*}
对于第一项,令 \( u = ax^2+bx+c \),则 \( \mathrm{d}u = (2ax+b) \mathrm{d}x \),所以
\[
\int (2ax+b) \sqrt{ax^2+bx+c} \,\mathrm{d}x = \int \sqrt{u} \,\mathrm{d}u = \frac{2}{3} u^{3/2} = \frac{2}{3} (ax^2+bx+c)^{3/2}.
\]
第二项即为上一个积分的结果。因此
\[
\int x\sqrt{ax^2+bx+c}\,\mathrm{d}x = \frac{1}{3a} (ax^2+bx+c)^{3/2} - \frac{b}{2a} \int \sqrt{ax^2+bx+c} \,\mathrm{d}x + C.
\]

\begin{example}{}{}
    \[\int x\sqrt{-ax^2+bx+c}\,\mathrm{d}x\]
\end{example}
\vspace{-15pt}
\begin{align*}
\int x\sqrt{-ax^2+bx+c}\,\mathrm{d}x &= \frac{1}{-2a} \int (-2ax+b-b) \sqrt{-ax^2+bx+c} \,\mathrm{d}x \\
&= -\frac{1}{2a} \int (-2ax+b) \sqrt{-ax^2+bx+c} \,\mathrm{d}x + \frac{b}{2a} \int \sqrt{-ax^2+bx+c} \,\mathrm{d}x.
\end{align*}
对于第一项,令 \( u = -ax^2+bx+c \),则 \( \mathrm{d}u = (-2ax+b) \mathrm{d}x \),所以
\[
\int (-2ax+b) \sqrt{-ax^2+bx+c} \,\mathrm{d}x = \int \sqrt{u} \,\mathrm{d}u = \frac{2}{3} u^{3/2}.
\]
因此
\[
\int x\sqrt{-ax^2+bx+c}\,\mathrm{d}x = -\frac{1}{3a} (-ax^2+bx+c)^{3/2} + \frac{b}{2a} \int \sqrt{-ax^2+bx+c} \,\mathrm{d}x + C.
\]

\begin{example}{}{}
    \[\int\dfrac{\mathrm{d}x}{\sqrt{ax^2+bx+c}}\]
\end{example}
\vspace{-15pt}
\begin{align*}
\int\dfrac{\mathrm{d}x}{\sqrt{ax^2+bx+c}} &= \frac{1}{\sqrt{a}} \int \frac{\mathrm{d}x}{\sqrt{ \left( x+\frac{b}{2a} \right)^2 + \frac{4ac-b^2}{4a^2} }} \quad (\text{假设 } a>0) \\
&\xlongequal{u=x+\frac{b}{2a},\ \Delta=4ac-b^2} \frac{1}{\sqrt{a}} \int \frac{\mathrm{d}u}{\sqrt{u^2 + \frac{\Delta}{4a^2}}}.
\end{align*}
分情况:
\begin{enumerate}
    \item 若 \(\Delta > 0\),记 \(k=\frac{\sqrt{\Delta}}{2a}>0\),则
    \[
    \int \frac{\mathrm{d}u}{\sqrt{u^2+k^2}} = \ln\left| u + \sqrt{u^2+k^2} \right| + C.
    \]
    \item 若 \(\Delta = 0\),则积分简化为 \(\frac{1}{\sqrt{a}} \int \frac{\mathrm{d}u}{|u|} = \frac{1}{\sqrt{a}} \ln|u| + C\)。
    \item 若 \(\Delta < 0\),记 \(k=\frac{\sqrt{-\Delta}}{2a}>0\),则
    \[
    \int \frac{\mathrm{d}u}{\sqrt{u^2-k^2}} = \ln\left| u + \sqrt{u^2-k^2} \right| + C.
    \]
\end{enumerate}
统一形式:
\[
\int\dfrac{\mathrm{d}x}{\sqrt{ax^2+bx+c}} = \frac{1}{\sqrt{a}} \ln\left| 2a x + b + 2\sqrt{a} \sqrt{ax^2+bx+c} \right| + C \quad (a>0).
\]
若 \(a<0\),则需用反正弦函数等形式。

\begin{example}{}{}
    \[\int\dfrac{x\,\mathrm{d}x}{\sqrt{ax^2+bx+c}}\]
\end{example}
\vspace{-15pt}
\begin{align*}
\int\dfrac{x\,\mathrm{d}x}{\sqrt{ax^2+bx+c}} &= \frac{1}{2a} \int \frac{2ax+b-b}{\sqrt{ax^2+bx+c}} \,\mathrm{d}x \\
&= \frac{1}{2a} \int \frac{2ax+b}{\sqrt{ax^2+bx+c}} \,\mathrm{d}x - \frac{b}{2a} \int \frac{\mathrm{d}x}{\sqrt{ax^2+bx+c}}.
\end{align*}
第一项:令 \( u = ax^2+bx+c \),则 \( \mathrm{d}u = (2ax+b)\mathrm{d}x \),所以
\[
\int \frac{2ax+b}{\sqrt{ax^2+bx+c}} \,\mathrm{d}x = \int \frac{\mathrm{d}u}{\sqrt{u}} = 2\sqrt{u} = 2\sqrt{ax^2+bx+c}.
\]
因此
\[
\int\dfrac{x\,\mathrm{d}x}{\sqrt{ax^2+bx+c}} = \frac{1}{a} \sqrt{ax^2+bx+c} - \frac{b}{2a} \int \frac{\mathrm{d}x}{\sqrt{ax^2+bx+c}} + C.
\]

\begin{example}{}{}
    \[\int\dfrac{x^2\mathrm{d}x}{\sqrt{ax^2+bx+c}}\]
\end{example}
\vspace{-15pt}
\begin{align*}
\int\dfrac{x^2\mathrm{d}x}{\sqrt{ax^2+bx+c}} &= \frac{1}{a} \int \frac{ax^2}{\sqrt{ax^2+bx+c}} \,\mathrm{d}x \\
&= \frac{1}{a} \int \frac{ax^2+bx+c - bx - c}{\sqrt{ax^2+bx+c}} \,\mathrm{d}x \\
&= \frac{1}{a} \int \sqrt{ax^2+bx+c} \,\mathrm{d}x - \frac{b}{a} \int \frac{x\,\mathrm{d}x}{\sqrt{ax^2+bx+c}} - \frac{c}{a} \int \frac{\mathrm{d}x}{\sqrt{ax^2+bx+c}}.
\end{align*}
前两个积分已知,代入即可。

\begin{example}{}{}
    \[\int\dfrac{\mathrm{d}x}{\sqrt{-ax^2+bx+c}}\]
\end{example}
\vspace{-15pt}
\begin{align*}
\int\dfrac{\mathrm{d}x}{\sqrt{-ax^2+bx+c}} &= \frac{1}{\sqrt{a}} \int \frac{\mathrm{d}x}{\sqrt{ \frac{b^2+4ac}{4a^2} - \left( x-\frac{b}{2a} \right)^2 }} \quad (\text{假设 } a>0) \\
&\xlongequal{u=x-\frac{b}{2a},\ k^2=\frac{b^2+4ac}{4a^2}} \frac{1}{\sqrt{a}} \int \frac{\mathrm{d}u}{\sqrt{k^2-u^2}} \\
&= \frac{1}{\sqrt{a}} \arcsin\frac{u}{k} + C = \frac{1}{\sqrt{a}} \arcsin\frac{2a x - b}{\sqrt{b^2+4ac}} + C.
\end{align*}

\begin{example}{}{}
    \[\int\dfrac{x\,\mathrm{d}x}{\sqrt{-ax^2+bx+c}}\]
\end{example}
\vspace{-15pt}
\begin{align*}
\int\dfrac{x\,\mathrm{d}x}{\sqrt{-ax^2+bx+c}} &= \frac{1}{-2a} \int \frac{-2ax+b-b}{\sqrt{-ax^2+bx+c}} \,\mathrm{d}x \\
&= -\frac{1}{2a} \int \frac{-2ax+b}{\sqrt{-ax^2+bx+c}} \,\mathrm{d}x + \frac{b}{2a} \int \frac{\mathrm{d}x}{\sqrt{-ax^2+bx+c}}.
\end{align*}
第一项:令 \( u = -ax^2+bx+c \),则 \( \mathrm{d}u = (-2ax+b)\mathrm{d}x \),所以
\[
\int \frac{-2ax+b}{\sqrt{-ax^2+bx+c}} \,\mathrm{d}x = \int \frac{\mathrm{d}u}{\sqrt{u}} = 2\sqrt{u} = 2\sqrt{-ax^2+bx+c}.
\]
因此
\[
\int\dfrac{x\,\mathrm{d}x}{\sqrt{-ax^2+bx+c}} = -\frac{1}{a} \sqrt{-ax^2+bx+c} + \frac{b}{2a} \int \frac{\mathrm{d}x}{\sqrt{-ax^2+bx+c}} + C.
\]

\begin{example}{}{}
    \[\int\dfrac{x^2\mathrm{d}x}{\sqrt{-ax^2+bx+c}}\]
\end{example}
\vspace{-15pt}
\begin{align*}
\int\dfrac{x^2\mathrm{d}x}{\sqrt{-ax^2+bx+c}} &= \frac{1}{-a} \int \frac{-ax^2}{\sqrt{-ax^2+bx+c}} \,\mathrm{d}x \\
&= -\frac{1}{a} \int \frac{-ax^2+bx+c - bx - c}{\sqrt{-ax^2+bx+c}} \,\mathrm{d}x \\
&= -\frac{1}{a} \int \sqrt{-ax^2+bx+c} \,\mathrm{d}x + \frac{b}{a} \int \frac{x\,\mathrm{d}x}{\sqrt{-ax^2+bx+c}} + \frac{c}{a} \int \frac{\mathrm{d}x}{\sqrt{-ax^2+bx+c}}.
\end{align*}

\section{含有\texorpdfstring{$\sqrt{\pm \dfrac{x-a}{x-b}},\sqrt{(x-a)(b-x)}$}{}的积分}
\begin{example}{}{}
\[\int\sqrt{\dfrac{x-a}{x-b}}\,\mathrm{d}x\]
\end{example}
\begin{solution}
设 \( t = \sqrt{\dfrac{x-a}{x-b}} \),则 \( t^2 = \dfrac{x-a}{x-b} \),解得 \( x = \dfrac{a - b t^2}{1 - t^2} \)。微分得
\[
\mathrm{d}x = \frac{2(b-a)t}{(1-t^2)^2} \,\mathrm{d}t.
\]
于是
\begin{align*}
\int \sqrt{\frac{x-a}{x-b}} \,\mathrm{d}x &= \int t \cdot \frac{2(b-a)t}{(1-t^2)^2} \,\mathrm{d}t = 2(b-a) \int \frac{t^2}{(1-t^2)^2} \,\mathrm{d}t.
\end{align*}
将被积函数分解为部分分式:
\[
\frac{t^2}{(1-t^2)^2} = \frac{1}{4} \left( \frac{1}{1-t} + \frac{1}{1+t} \right) + \frac{1}{2} \left( \frac{1}{(1-t)^2} + \frac{1}{(1+t)^2} \right).
\]
积分得
\begin{align*}
\int \frac{t^2}{(1-t^2)^2} \,\mathrm{d}t &= \frac{1}{4} \left( -\ln|1-t| + \ln|1+t| \right) + \frac{1}{2} \left( \frac{1}{1-t} - \frac{1}{1+t} \right) + C \\
&= \frac{1}{4} \ln\left| \frac{1+t}{1-t} \right| + \frac{t}{1-t^2} + C.
\end{align*}
代回 \( t = \sqrt{\frac{x-a}{x-b}} \),并利用 \( 1-t^2 = \frac{b-a}{x-b} \),\( t/(1-t^2) = \frac{\sqrt{(x-a)(x-b)}}{b-a} \),以及
\[
\frac{1+t}{1-t} = \frac{\sqrt{x-b} + \sqrt{x-a}}{\sqrt{x-b} - \sqrt{x-a}},
\]
可得
\begin{align*}
\int \sqrt{\frac{x-a}{x-b}} \,\mathrm{d}x &= 2(b-a) \left( \frac{1}{4} \ln\left| \frac{\sqrt{x-b} + \sqrt{x-a}}{\sqrt{x-b} - \sqrt{x-a}} \right| + \frac{\sqrt{(x-a)(x-b)}}{b-a} \right) + C \\
&= \sqrt{(x-a)(x-b)} + \frac{b-a}{2} \ln\left| \frac{\sqrt{x-b} + \sqrt{x-a}}{\sqrt{x-b} - \sqrt{x-a}} \right| + C.
\end{align*}
利用恒等式 \(\ln\left| \frac{\sqrt{x-b} + \sqrt{x-a}}{\sqrt{x-b} - \sqrt{x-a}} \right| = 2 \ln\left( \sqrt{x-a} + \sqrt{x-b} \right) + \text{常数}\),可进一步写为
\[
\int \sqrt{\frac{x-a}{x-b}} \,\mathrm{d}x = \sqrt{(x-a)(x-b)} + (b-a) \ln\left( \sqrt{x-a} + \sqrt{x-b} \right) + C',
\]
其中 \(C'\) 为常数。通常假设 \(x > b > a\),故可略去绝对值。
\end{solution}

\begin{example}{}{}
\[\int\sqrt{\dfrac{x-a}{b-x}}\,\mathrm{d}x\]
\end{example}
\begin{solution}
设 \( t = \sqrt{\dfrac{x-a}{b-x}} \),则 \( t^2 = \dfrac{x-a}{b-x} \),解得 \( x = \dfrac{a + b t^2}{1 + t^2} \)。微分得
\[
\mathrm{d}x = \frac{2(b-a)t}{(1+t^2)^2} \,\mathrm{d}t.
\]
于是
\begin{align*}
\int \sqrt{\frac{x-a}{b-x}} \,\mathrm{d}x &= \int t \cdot \frac{2(b-a)t}{(1+t^2)^2} \,\mathrm{d}t = 2(b-a) \int \frac{t^2}{(1+t^2)^2} \,\mathrm{d}t.
\end{align*}
计算 \(\int \frac{t^2}{(1+t^2)^2} \,\mathrm{d}t\),令 \(t = \tan u\),则 \(\mathrm{d}t = \sec^2 u \,\mathrm{d}u\),且
\[
\int \frac{t^2}{(1+t^2)^2} \,\mathrm{d}t = \int \frac{\tan^2 u}{\sec^4 u} \sec^2 u \,\mathrm{d}u = \int \sin^2 u \,\mathrm{d}u = \frac{u}{2} - \frac{\sin 2u}{4} + C = \frac{1}{2} \arctan t - \frac{t}{2(1+t^2)} + C.
\]
因此
\[
\int \sqrt{\frac{x-a}{b-x}} \,\mathrm{d}x = 2(b-a) \left( \frac{1}{2} \arctan t - \frac{t}{2(1+t^2)} \right) + C = (b-a) \left( \arctan t - \frac{t}{1+t^2} \right) + C.
\]
代回 \(t = \sqrt{\frac{x-a}{b-x}}\),注意 \(\frac{t}{1+t^2} = \frac{\sqrt{(x-a)(b-x)}}{b-a}\),且 \(\arctan t = \arcsin \sqrt{\frac{x-a}{b-a}}\)(因为当 \(t \ge 0\) 时,\(\arctan t = \arcsin \frac{t}{\sqrt{1+t^2}} = \arcsin \sqrt{\frac{x-a}{b-a}}\))。所以
\[
\int \sqrt{\frac{x-a}{b-x}} \,\mathrm{d}x = (b-a) \arcsin \sqrt{\frac{x-a}{b-a}} - \sqrt{(x-a)(b-x)} + C.
\]
也可写为
\[
\int \sqrt{\frac{x-a}{b-x}} \,\mathrm{d}x = \sqrt{(x-a)(b-x)} + (b-a) \arcsin \sqrt{\frac{x-a}{b-a}} + C,
\]
其中符号差异可并入常数。通常采用后一种形式。
\end{solution}

\begin{example}{}{}
\[\int\dfrac{\mathrm{d}x}{\sqrt{(x-a)(b-x)}}\]
\end{example}
\begin{solution}
注意到 \((x-a)(b-x) = -(x-a)(x-b) = \left( \frac{b-a}{2} \right)^2 - \left( x - \frac{a+b}{2} \right)^2\)。令 \(u = x - \frac{a+b}{2}\),则
\[
\int \frac{\mathrm{d}x}{\sqrt{(x-a)(b-x)}} = \int \frac{\mathrm{d}u}{\sqrt{ \left( \frac{b-a}{2} \right)^2 - u^2 }} = \arcsin \frac{2u}{b-a} + C = \arcsin \frac{2x - a - b}{b-a} + C.
\]
另一种常见形式:令 \(t = \sqrt{\frac{x-a}{b-x}}\),可得
\[
\int \frac{\mathrm{d}x}{\sqrt{(x-a)(b-x)}} = 2 \arcsin \sqrt{\frac{x-a}{b-a}} + C.
\]
两种形式等价,因为 \(\arcsin \frac{2x - a - b}{b-a} = 2 \arcsin \sqrt{\frac{x-a}{b-a}} - \frac{\pi}{2}\),相差常数。
\end{solution}

\begin{example}{}{}
\[\int\sqrt{(x-a)(b-x)}\,\mathrm{d}x\]
\end{example}
\begin{solution}
完成平方:
\[
(x-a)(b-x) = -\left[ x^2 - (a+b)x + ab \right] = \left( \frac{b-a}{2} \right)^2 - \left( x - \frac{a+b}{2} \right)^2.
\]
令 \(u = x - \frac{a+b}{2}\),则
\[
\int \sqrt{(x-a)(b-x)} \,\mathrm{d}x = \int \sqrt{ \left( \frac{b-a}{2} \right)^2 - u^2 } \,\mathrm{d}u.
\]
利用公式 \(\int \sqrt{k^2 - u^2} \,\mathrm{d}u = \frac{u}{2} \sqrt{k^2 - u^2} + \frac{k^2}{2} \arcsin \frac{u}{k} + C\),其中 \(k = \frac{b-a}{2}\),得
\begin{align*}
\int \sqrt{(x-a)(b-x)} \,\mathrm{d}x &= \frac{u}{2} \sqrt{ \left( \frac{b-a}{2} \right)^2 - u^2 } + \frac{1}{2} \left( \frac{b-a}{2} \right)^2 \arcsin \frac{u}{(b-a)/2} + C \\
&= \frac{2x - a - b}{4} \sqrt{(x-a)(b-x)} + \frac{(b-a)^2}{8} \arcsin \frac{2x - a - b}{b-a} + C.
\end{align*}
也可写为
\[
\int \sqrt{(x-a)(b-x)} \,\mathrm{d}x = \frac{1}{2} \left( x - \frac{a+b}{2} \right) \sqrt{(x-a)(b-x)} + \frac{(b-a)^2}{8} \arcsin \frac{2x - a - b}{b-a} + C.
\]
\end{solution}

\section{含有三角函数的积分}
\begin{example}{}{}
\[\int\tan x\,\mathrm{d}x\]
\end{example}
\vspace{-15pt}
\begin{align*}
\int\tan x\,\mathrm{d}x = -\ln|\cos x| + C = \ln|\sec x| + C.
\end{align*}

\begin{example}{}{}
\[\int\cot x\,\mathrm{d}x\]
\end{example}
\vspace{-15pt}
\begin{align*}
\int\cot x\,\mathrm{d}x = \ln|\sin x| + C.
\end{align*}

\begin{example}{}{}
\[\int\sec x\,\mathrm{d}x\]
\end{example}
\vspace{-15pt}
\begin{align*}
\int\sec x\,\mathrm{d}x = \ln|\sec x + \tan x| + C.
\end{align*}

\begin{example}{}{}
\[\int\csc x\,\mathrm{d}x\]
\end{example}
\vspace{-15pt}
\begin{align*}
\int\csc x\,\mathrm{d}x = \ln|\csc x - \cot x| + C = \ln\left|\tan\frac{x}{2}\right| + C.
\end{align*}

\begin{example}{}{}
\[\int\tan^2 x\,\mathrm{d}x\]
\end{example}
\vspace{-15pt}
\begin{align*}
\int\tan^2 x\,\mathrm{d}x = \int (\sec^2 x - 1) \,\mathrm{d}x = \tan x - x + C.
\end{align*}

\begin{example}{}{}
\[\int\cot^2 x\,\mathrm{d}x\]
\end{example}
\vspace{-15pt}
\begin{align*}
\int\cot^2 x\,\mathrm{d}x = \int (\csc^2 x - 1) \,\mathrm{d}x = -\cot x - x + C.
\end{align*}

\begin{example}{}{}
\[\int\sec^2 x\,\mathrm{d}x\]
\end{example}
\vspace{-15pt}
\begin{align*}
\int\sec^2 x\,\mathrm{d}x = \tan x + C.
\end{align*}

\begin{example}{}{}
\[\int\csc^2 x\,\mathrm{d}x\]
\end{example}
\vspace{-15pt}
\begin{align*}
\int\csc^2 x\,\mathrm{d}x = -\cot x + C.
\end{align*}

\begin{example}{}{}
\[\int\sin ax\cos bx\,\mathrm{d}x\]
\end{example}
\vspace{-15pt}
\begin{align*}
\int\sin ax\cos bx\,\mathrm{d}x &= \frac{1}{2} \int \left( \sin((a+b)x) + \sin((a-b)x) \right) \,\mathrm{d}x \\
&= -\frac{1}{2} \left( \frac{\cos((a+b)x)}{a+b} + \frac{\cos((a-b)x)}{a-b} \right) + C \quad (a \neq b).
\end{align*}

\begin{example}{}{}
\[\int\sin ax\sin bx\,\mathrm{d}x\]
\end{example}
\vspace{-15pt}
\begin{align*}
\int\sin ax\sin bx\,\mathrm{d}x &= \frac{1}{2} \int \left( \cos((a-b)x) - \cos((a+b)x) \right) \,\mathrm{d}x \\
&= \frac{1}{2} \left( \frac{\sin((a-b)x)}{a-b} - \frac{\sin((a+b)x)}{a+b} \right) + C \quad (a \neq b).
\end{align*}

\begin{example}{}{}
\[\int\cos ax\cos bx\,\mathrm{d}x\]
\end{example}
\vspace{-15pt}
\begin{align*}
\int\cos ax\cos bx\,\mathrm{d}x &= \frac{1}{2} \int \left( \cos((a+b)x) + \cos((a-b)x) \right) \,\mathrm{d}x \\
&= \frac{1}{2} \left( \frac{\sin((a+b)x)}{a+b} + \frac{\sin((a-b)x)}{a-b} \right) + C \quad (a \neq b).
\end{align*}

\begin{example}{}{}
    \[\int\dfrac{\mathrm{d}x}{a+b\sin x}\]
\end{example}
\vspace{-15pt}
\begin{align*}
\int\dfrac{\mathrm{d}x}{a+b\sin x} &\xlongequal{t=\tan\frac{x}{2}} \int \frac{1}{a+b\cdot\frac{2t}{1+t^2}} \cdot \frac{2}{1+t^2} \,\mathrm{d}t = \int \frac{2}{a(1+t^2)+2bt} \,\mathrm{d}t \\
&= \int \frac{2}{a t^2 + 2b t + a} \,\mathrm{d}t = \frac{2}{a} \int \frac{\mathrm{d}t}{t^2 + \frac{2b}{a}t + 1} \\
&= \frac{2}{a} \int \frac{\mathrm{d}t}{(t+\frac{b}{a})^2 + \frac{a^2-b^2}{a^2}}.
\end{align*}
记 \(\Delta = a^2 - b^2\)。若 \(\Delta > 0\),则
\[
\int\dfrac{\mathrm{d}x}{a+b\sin x} = \frac{2}{\sqrt{\Delta}} \arctan\left( \frac{a t + b}{\sqrt{\Delta}} \right) + C = \frac{2}{\sqrt{a^2-b^2}} \arctan\left( \frac{a\tan\frac{x}{2}+b}{\sqrt{a^2-b^2}} \right) + C.
\]
若 \(\Delta < 0\),则用对数表示。当 \(a^2 > b^2\) 时,结果如上。

\begin{example}{}{}
    \[\int\dfrac{\mathrm{d}x}{a+b\cos x}\]
\end{example}
\vspace{-15pt}
\begin{align*}
\int\dfrac{\mathrm{d}x}{a+b\cos x} &\xlongequal{t=\tan\frac{x}{2}} \int \frac{1}{a+b\cdot\frac{1-t^2}{1+t^2}} \cdot \frac{2}{1+t^2} \,\mathrm{d}t = \int \frac{2}{(a+b)+(a-b)t^2} \,\mathrm{d}t.
\end{align*}
分情况:
\begin{enumerate}
    \item 若 \(a^2 > b^2\),则
    \[
    \int\dfrac{\mathrm{d}x}{a+b\cos x} = \frac{2}{\sqrt{a^2-b^2}} \arctan\left( \sqrt{\frac{a-b}{a+b}} \tan\frac{x}{2} \right) + C.
    \]
    \item 若 \(a^2 < b^2\),则
    \[
    \int\dfrac{\mathrm{d}x}{a+b\cos x} = \frac{1}{\sqrt{b^2-a^2}} \ln\left| \frac{\sqrt{b-a}\tan\frac{x}{2} + \sqrt{b+a}}{\sqrt{b-a}\tan\frac{x}{2} - \sqrt{b+a}} \right| + C.
    \]
\end{enumerate}

\begin{example}{}{}
    \[\int\dfrac{\mathrm{d}x}{a^2\sin^2x+b^2\cos^2x}\]
\end{example}
\vspace{-15pt}
\begin{align*}
\int\dfrac{\mathrm{d}x}{a^2\sin^2x+b^2\cos^2x} &= \int \frac{\mathrm{d}x}{\cos^2x (a^2\tan^2x + b^2)} \xlongequal{t=\tan x} \int \frac{\mathrm{d}t}{a^2 t^2 + b^2} \\
&= \frac{1}{ab} \arctan\left( \frac{a t}{b} \right) + C = \frac{1}{ab} \arctan\left( \frac{a \tan x}{b} \right) + C.
\end{align*}

\begin{example}{}{}
    \[\int\dfrac{\mathrm{d}x}{a^2\sin^2x-b^2\cos^2x}\]
\end{example}
\vspace{-15pt}
\begin{align*}
\int\dfrac{\mathrm{d}x}{a^2\sin^2x-b^2\cos^2x} &= \int \frac{\mathrm{d}x}{\cos^2x (a^2\tan^2x - b^2)} \xlongequal{t=\tan x} \int \frac{\mathrm{d}t}{a^2 t^2 - b^2} \\
&= \frac{1}{2ab} \ln\left| \frac{a t - b}{a t + b} \right| + C = \frac{1}{2ab} \ln\left| \frac{a \tan x - b}{a \tan x + b} \right| + C.
\end{align*}

\begin{example}{}{}
    \[\int x\sin ax\,\mathrm{d}x\]
\end{example}
\vspace{-15pt}
\begin{align*}
\int x\sin ax\,\mathrm{d}x &= -\frac{x}{a} \cos ax + \frac{1}{a} \int \cos ax \,\mathrm{d}x = -\frac{x}{a} \cos ax + \frac{1}{a^2} \sin ax + C.
\end{align*}

\begin{example}{}{}
    \[\int x\cos ax\,\mathrm{d}x\]
\end{example}
\vspace{-15pt}
\begin{align*}
\int x\cos ax\,\mathrm{d}x &= \frac{x}{a} \sin ax - \frac{1}{a} \int \sin ax \,\mathrm{d}x = \frac{x}{a} \sin ax + \frac{1}{a^2} \cos ax + C.
\end{align*}

\begin{example}{}{}
    \[\int x^2\sin ax\,\mathrm{d}x\]
\end{example}
\vspace{-15pt}
\begin{align*}
\int x^2\sin ax\,\mathrm{d}x &= -\frac{x^2}{a} \cos ax + \frac{2}{a} \int x \cos ax \,\mathrm{d}x \\
&= -\frac{x^2}{a} \cos ax + \frac{2}{a} \left( \frac{x}{a} \sin ax + \frac{1}{a^2} \cos ax \right) + C \\
&= -\frac{x^2}{a} \cos ax + \frac{2x}{a^2} \sin ax + \frac{2}{a^3} \cos ax + C.
\end{align*}

\begin{example}{}{}
    \[\int x^2\cos ax\,\mathrm{d}x\]
\end{example}
\vspace{-15pt}
\begin{align*}
\int x^2\cos ax\,\mathrm{d}x &= \frac{x^2}{a} \sin ax - \frac{2}{a} \int x \sin ax \,\mathrm{d}x \\
&= \frac{x^2}{a} \sin ax - \frac{2}{a} \left( -\frac{x}{a} \cos ax + \frac{1}{a^2} \sin ax \right) + C \\
&= \frac{x^2}{a} \sin ax + \frac{2x}{a^2} \cos ax - \frac{2}{a^3} \sin ax + C.
\end{align*}
