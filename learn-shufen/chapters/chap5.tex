\chapter{微分中值定理}
\begin{example}{}{}
    $x_1-\ln x_1=x_2-\ln x_2=m,x_1<x_2$,求证$x_1^{x_2}+x_2^{x_1}\sim 1+\dfrac{\ln m}{\e^m},m\to+\infty$.
\end{example}
\begin{solution}
    首先估计$x_1,x_2$在无穷远处的渐进,首先观察$f(x)=x-\ln x$,在$x\to 0$的时候函数值由$-\ln x$决定,在$x\to+\infty$的时候函数值由$x$决定,先估计大根$x_2$,设$x_2=m+\ln x_2\sim m, m\to+\infty$,于是设$x_2=m+\delta$将这个结果代入得到$\delta=m+\ln(m+\delta)$,泰勒展开得到
    \begin{align*}
    \delta = \ln(m+\ln\delta)=\ln m+\ln\left(1+\frac{\ln \delta}{m}\right)=\ln m+\dfrac{\delta}{\ln m}+o\left(\dfrac{\delta}{\ln m }\right)\sim \ln m
    \end{align*}
    得到$x_2\sim m+\ln m$,再设$x_2=m+\ln m+\delta$,代入得到
    \begin{align*}
       & x_2=m+\ln x_2=m+\ln(m+\ln m+\delta)=m+\ln m+\ln\left(1+\frac{\ln m}{m}+\frac{\delta}{m}\right)\\
       \Rightarrow & \delta = \ln\left(1+\frac{\ln m}{m}+\frac{\delta}{m}\right)=\frac{\ln m+\delta}{m}+o\left(\frac{\ln m+\delta}{m}\right)\sim \frac{\ln m+\delta}{m}
    \end{align*}
    解得$\delta\sim \dfrac{\ln m}{m-1}$,再设$x_2=m+\frac{m\ln m}{m-1}+\delta$,代入得到
    \begin{align*}
    & m+\frac{m\ln m}{m-1}+\delta=m+\ln\left(m+\frac{m\ln m}{m-1}+\delta\right)\\
    \Rightarrow &\frac{m\ln m}{m-1}+\delta=\ln m+\ln\left(1+\frac{\ln m}{m-1}+\frac{\delta}{m}\right)=\frac{\ln m}{m-1}+\delta=\ln\left(1+\frac{\ln m}{m-1}+\frac{\delta}{m}\right)\\
    \Rightarrow &\frac{m\ln m}{m-1}+\delta=\left(\frac{\ln m}{m-1}+\frac{\delta}{m}\right)-\frac12\left(\frac{\ln m}{m-1}+\frac{\delta}{m}\right)^2+o\left(\frac{\ln m}{m-1}+\frac{\delta}{m}\right)^2\\
    \Rightarrow &\ln m +\delta\sim\frac{\delta}{m}-\frac12\frac{\ln^2m}{(m-1)^2}-\delta\frac{\ln m}{m(m-1)}\quad\text{舍掉平方项}\\
    \Rightarrow &\delta\sim-\frac{\ln m+\frac12\frac{\ln^2m}{(m-1)^2}}{1-\frac1m+\frac{\ln m}{m(m-1)}}
    \end{align*}
\end{solution}
\newpage

\begin{example}{}{}
设$f\in C[0,2]\cap D(0,2)$,满足$f(0)=f(2)=0,|f'(x)|\leqslant M,\forall x\in(0,2)$,证明\[\left|\int_0^2f(x)\dd x\right|\leqslant M\]
\end{example}
\begin{solution}
    插$f(0),f(2)$,插值多项式1次(仅拉格朗日插值式,用不到导数修正), 余项2阶,导数不够, 属于靠近哪边对哪边插模型。所以在$[0,1]$上插$f(0)$,在$[1,2]$上插$f(2)$,则写出两段插值式。
    
    当$x\in[0,1]$时,对$f(x)$的插值为:\[f(x)=f(0)+\frac{f'(\xi(x))}{1!}(x-0)=f'(\xi(x))x\]这是拉格朗日中值定理的直接应用,所以不用考场翻译。得到\[|f(x)|\leqslant |f'(\theta(x))|x\leqslant Mx\]
    当$x\in[1,2]$时,对$f(x)$的插值为:\[f(x)=f(2)+\frac{f'(\eta(x))}{1!}(x-2)=f'(\eta(x))(x-2)\]同理,得到\[|f(x)|\leqslant |f'(\eta(x))||(x-2)|\leqslant M(2-x)\]
    所以
    \begin{align*}
    \left|\int_0^2f(x)\dd x\right|&\leqslant\left|\int_0^1f(x)\dd x\right|+\left|\int_1^2f(x)\dd x\right|\\
    &\leqslant \int_0^1|f(x)|\dd x+\int_1^2|f(x)|\dd x\\
    &\leqslant \int_0^1Mx\dd x+\int_1^2M(2-x)\dd x=M
    \end{align*}
\end{solution}
\newpage
\begin{example}{}{}
设$f\in D^2[0,1],f(0)=f(1)=0,|f''(x)|\leqslant M$,证明$\displaystyle \left|\int_0^1f(x)\dd x\right|\leqslant \dfrac{M}{12}$
\end{example}
\begin{solution}
要拟合$f(0),f(1)$一共两个插值条件,无导数条件需要拟合,所以插值多项式为一次,也不用待定多项式$r$,余项到了2阶导数,符合题目所给条件,所以根据埃米特插值,存在$\theta(x)\in(0,1)$使得:
\[f(x)=\frac{f''(\theta(x))}{2}x(x-1),\forall x\in[0,1]\]
这个结论可以用k值法来证明,由\[f(x)-Kx(x-1)=0\]的条件(其中$K$与$x$有关),设
\[F(y)=f(y)-Ky(y-1)\]则$F(0)=F(1)=F(x)=0$,罗尔得到$F'(\xi_1)=F'(\xi_2)=0,\xi_1,\xi_2\in(0,1)$,所以存在$\theta(x)\in(0,1)$使得$F''(\theta(x))=0$。对$F(x)$求二阶导数得到\[f''(\theta(x))-2K=0\Rightarrow K=\frac{f''(\theta(x))}{2}\]所以\[f(x)=\frac{f''(\theta(x))}{2}x(x-1)\]所以
\begin{align*}
    \left|\int_0^1f(x)\dd x\right|&=\left|\int_0^1\dfrac{f''(\theta(x))}{2}x(x-1)\dd x \right|\leqslant \int_0^1\left|\dfrac{f''(\theta(x))}{2}\right||x(x-1)|\dd x\\
&\leqslant \dfrac{M}{2}\int_0^1 x(1-x)\dd x=\frac{M}{12}
\end{align*}
\end{solution}
\newpage
\begin{example}{}{}
    设$f\in C^3[0,2]$满足\[f(0)=f'(0)=0,\int_0^2f(x)\dd x=8\int_0^1f(x)\dd x\]求证存在$\theta\in(0,2)$使得$f'''(\theta)=0$
\end{example}
\begin{solution}
套路地,设$\displaystyle F(x)=\int_0^x f(y)\dd y$,$F(x)\in C^4[0,2]$,$F(0)=F'(0)=F''(0)=0$,且$F(2)=8F(1)$,这里我们使用插值法,就是找一个多项式去尽可能的拟合$f$的行为。注意到$F(0)=F'(0)=F''(0)=0$,所以可以考虑插值多项式为$p(x)=x^3(ax+b)$,代入$F(2)=8F(1)$得到$a=0$,所以$p(x)=bx^3$,再由$p(1)=F(1)$得到$b=F(1)$,所以\[p(x)=F(1)x^3\]于是构造出\[G(x)=F(x)-F(1)x^3\]这个$G(x)$满足\[G(0)=G(1)=G(2)=0,G'(0)=0,G''(0)=0\]
所以由罗尔中值定理,存在$\xi_1\in(0,1)\xi_2\in(1,2)$,使得$G'(\xi_1)=G'(xi_2)=0$,搭配上$G'(0)=0$,所以又有罗尔中值定理,得知存在$\theta_1,\theta_2$,使得$G''(\theta_1)=G''(\theta_2)=0$,同样的,搭配上$G''(0)=0$,又有罗尔中值定理,得知存在$\eta_1\in(0,1),\eta_2\in(1,2)$使得\[G'''(\eta_1)=G'''(\eta_2)=0\]所以由罗尔中值定理,得到存在$\theta\in(0,2)$使得\[G''''(\theta)=0\]而求导易得\[G''''(x)=f'''(x)-p''''(x)=0\Rightarrow f'''(\theta)=0\]
\end{solution}

\begin{example}{}{}
设$f\in C^1[a,b]\cap D^3(a,b)$,证明存在$\xi\in(a,b)$,使得\[\frac{f(b)-f(a)}{b-a}=\frac{f'(a)+f'(b)}{2}-\frac{f'''(\xi)}{12}(b-a)^2\]
\end{example}
\begin{solution}
插$f(a),f(b),f'(a),f'(b)$,则插值多项式3次,余项4阶,但是条件只到3阶,属于$f'''=p'''$类型,插值:
\[p(x)=\frac{x-b}{a-b}f(a)+\frac{x-a}{b-a}f(b)+(kx+m)(x-a)(x-b)\]
此时自动保证$p(a)=f(a),p(b)=f(b)$,参数$k,m$待定,以期望$p'(a)=f'(a),p'(b)=f'(b)$,解出$k,m$.
\begin{align*}
p'(a)&=\frac{f(b)-f(a)}{b-a}+(ka+m)(a-b)=f'(a)\\
p'(b)&=\frac{f(b)-f(a)}{b-a}+(kb+m)(a-b)=f'(b)
\end{align*}
相减得到\[k=\frac{f'(b)+f'(a)}{(b-a)^2}-\frac{2(f(b)-f(a))}{(b-a)^3}\]
此时可以继续解方程得到$m$,但是没有必要了,我们假装$m$已知,设而不求就可以了
设$G(x)=f(x)-p(x)$,则$G(a)=G(b)=0,G'(a)=G'(b)=0$,所以由罗尔中值定理,存在$\xi_1\in(a,b)$使得$G'(\xi_1)=0$,再由罗尔中值定理,存在$\xi_2,\xi_3$使得$G''(\xi_2)=0,G''(\xi_3)=0$,最后由罗尔中值定理,存在$\xi\in(a,b)$使得$G'''(\xi)=0$,所以\[f'''(\xi)=p'''(\xi)=6k=\frac{6(f'(b)+f'(a))}{(b-a)^2}-\frac{12(f(b)-f(a))}{(b-a)^3}\]
\end{solution}
\begin{example}{}{}
    $f(x)$在$[0,1]$可导,且$\displaystyle \int_0^1f(x)\dd x=\dfrac52,\int_0^1 xf(x)\dd x=\dfrac32$,求证在$(0,1)$存在$\xi$使$f'(\xi)=3$.
\end{example}
\begin{solution}
套路式地,设\[F(x)=\displaystyle\int_0^xf(t)\dd t,G(x)=\displaystyle\int_0^xtf(t)\dd t\]则有\[F(0)=G(0)=0,F(1)=\dfrac52,G(1)=\dfrac32\]而问题要证明的结论似乎仅与$f(x)$(以及其导函数,原函数)有关,所以要从$G(x)$中分出$F(x)$或者$f(x)$来,考虑分部积分:
\[\int_0^1xf(x)\dd x=\int_0^1x\dd F(x)=\left.xF(x)\right|_0^1-\int_0^1F(x)\dd x\Rightarrow \int_0^1F(x)\dd x=\dfrac12\]
则问题转化为$F(x)$在$(0,1)$上二阶可导,且$F(0)=0,F(1)=\dfrac52,\int_0^1F(x)\dd x=\dfrac12$,求证存在$\xi\in(0,1)$使得$F''(\xi)=3$。又注意到$\displaystyle\int_0^1 x\dd x=\dfrac12$,所以考虑$\varphi(x)=F(x)-x$,则$\varphi(0)=0,\varphi(1)=\dfrac32,\int_0^1\varphi(x)\dd x=0$,求证存在$\xi\in(0,1)$使得$\varphi''(\xi)=3$
要证$\varphi''(\xi)-3=0$,如果我们可以证明$\varphi'(x)-3x+C=0$
\end{solution}

\begin{example}{}{}
$f(x)$在$[a,b]$上连续,在$(a,b)$上二次可微,证明存在$\xi\in(a,b)$使得
\[f(a)+f(b)-2f\left(\frac{a+b}{2}\right)=\frac14(b-a)^2f''(\xi)\]
\end{example}
\begin{solution}
拟合$f(a),f(b),f\left(\frac{a+b}{2}\right)$,设
\[G(x)=f(x)-\left[\frac{(x-\frac{a+b}{2})(x-b)}{(a-\frac{a+b}{2})(a-b)}f(a)+\frac{(x-a)(x-\frac{a+b}{2})}{(b-a)(b-\frac{a+b}{2})}f(b)+\frac{(x-a)(x-b)}{(\frac{a+b}{2}-a)(\frac{a+b}{2}-b)}f\left(\frac{a+b}{2}\right)\right]\]
\end{solution}


\begin{theorem}{双元第一公式}{}
    若$x^2+y^2=a^2$,则
    \[x\dd x+y\dd y=0\Rightarrow\int\frac{\dd x}{y}=\arctan\frac{x}{y}+C\]
    若$x^2-y^2=a^2$或$x^2-y^2=-a^2$,则
    \[x\dd x=y\dd y\Rightarrow\int\frac{\dd x}{y}=\ln (x+y)+C\]
\end{theorem}
若$x^2+y^2=a^2$,则$x\dd x+y\dd y=0\Rightarrow \dfrac{x\dd x+y\dd y}{xy}=\dfrac{\dd x}{y}+\dfrac{\dd y}{x}=0$,所以
\begin{align*}
    \int\frac{\dd x}{y}&=\int\frac{y\dd x}{y^2}=\int\frac{-\dd y}{x}=\int\frac{-x\dd y}{x^2}\\
    &=\int\dfrac{y\dd x-x\dd y}{x^2+y^2}=\int\dfrac{y^2}{x^2+y^2}\dfrac{y\dd x-x\dd y}{y^2}\\
    &=\int\dfrac{1}{1+(\frac{x}{y})^2}\dd\dfrac{x}{y}=\arctan\frac{x}{y}+C
\end{align*}
若$x^2-y^2=a^2$或$x^2-y^2=-a^2$,则$x\dd x=y\dd y\Rightarrow \dfrac{\dd x}{y}=\dfrac{\dd y}{x}$,所以
\begin{align*}
    \int\frac{\dd x}{y}&=\int\frac{\dd y}{x}=\int\dfrac{\dd x+\dd y}{x+y}=\int\dfrac{\dd(x+y)}{x+y}=\ln (x+y)+C\\
\end{align*}
\begin{theorem}{双元第二公式}{}
形如\[y\dd x-x\dd y\quad y\dd x+x\dd y\]
等都可以用$x\dd x+y\dd y=0$或$x\dd x=y\dd y$消元,化归成第一公式的适用情形,同样的,第一公式也可以向第二公式转化,若$x^2+y^2=a^2$:
\[\dfrac{\dd x}{y}=-\dfrac{\dd y}{x}=\dfrac{y\dd x}{y^2}=-\dfrac{x\dd y}{x^2}=\dfrac{y\dd x-x\dd y}{y^2+x^2}\]
若$x^2-y^2=a^2$或$x^2-y^2=-a^2$,则
\[\dfrac{\dd x}{y}=\dfrac{\dd y}{x}=\dfrac{y\dd x}{y^2}=\dfrac{x\dd y}{x^2}=\dfrac{y\dd x-x\dd y}{y^2-x^2}\]
\end{theorem}
若$x^2+y^2=a^2$,则
\begin{align*}
    y\dd x-x\dd y&=y\dd x-x\left(-\dfrac{x\dd x}{y}\right)=(x^2+y^2)\dfrac{\dd x}{y}\\
    y\dd x-x\dd y&=y\left(\dfrac{-y\dd y}{x}\right)-x\dd y=-(x^2+y^2)\dfrac{\dd y}{x}
\end{align*}
若$x^2-y^2=a^2$或$x^2-y^2=-a^2$,则
\begin{align*}
    y\dd x-x\dd y&=y\dd x-x\left(\dfrac{x\dd x}{y}\right)=(y^2-x^2)\dfrac{\dd x}{y}\\
    y\dd x-x\dd y&=y\left(\dfrac{y\dd y}{x}\right)-x\dd y=(y^2-x^2)\dfrac{\dd y}{x}
\end{align*}
\begin{theorem}{双元第三公式}{}
    单独的$y\dd x$和$x\dd y$都可以用“加一项,减一项”的方法化归成$\dd (xy)$和公式二的情形
\end{theorem}
这里只给出其中一种:
\begin{align*}
y\dd x&=\dfrac12\left(y\dd x+y\dd x\right)=\dfrac12\left(y\dd x+x\dd y+y\dd x-x\dd y\right)\\
&=\dfrac12\left[\dd(xy)+(y\dd x-x\dd y)\right]
\end{align*}
\begin{theorem}{双元第四公式}{}
    形如$\displaystyle \dfrac{\dd x}{y^3}$的式子可以通过第二公式化成含有$\displaystyle \dd\left(\dfrac{y}{x}\right)$的形式
\end{theorem}
若$x^2+y^2=a^2$,则
\begin{align*}
\dfrac{\dd x}{y^3}&=\dfrac1{y^2}\dfrac{\dd x}{y}=\dfrac1{y^2}\dfrac{y\dd x-x\dd y}{x^2+y^2}=\dfrac1{x^2+y^2}\dd\left(\dfrac{x}{y}\right)\\
\dfrac{\dd y}{x^3}&=\dfrac1{x^2}\dfrac{\dd y}{x}=\dfrac1{x^2}\dfrac{x\dd y-y\dd x}{x^2+y^2}=\dfrac1{x^2+y^2}\dd\left(\dfrac{y}{x}\right)
\end{align*}
若$x^2-y^2=a^2$或$x^2-y^2=-a^2$,则
\begin{align*}
\dfrac{\dd x}{y^3}&=\dfrac1{y^2}\dfrac{\dd x}{y}=\dfrac1{y^2}\dfrac{y\dd x-x\dd y}{x^2-y^2}=\dfrac1{x^2-y^2}\dd\left(\dfrac{x}{y}\right)\\
\dfrac{\dd y}{x^3}&=\dfrac1{x^2}\dfrac{\dd y}{x}=\dfrac1{x^2}\dfrac{x\dd y-y\dd x}{x^2-y^2}=\dfrac1{x^2-y^2}\dd\left(\dfrac{y}{x}\right)
\end{align*}

\newpage
\begin{example}{}{}
    设函数$f(x)$在$(0,+\infty)$上三阶可导,且$f(x)>0,f'(x)>0,f''(x)>0,\displaystyle\lim_{x\to+\infty}\dfrac{f'(x)f'''(x)}{[f''(x)]^2}=a\ne1$,求极限$\displaystyle\lim_{x\to+\infty}\dfrac{f(x)f''(x)}{[f'(x)]^2}$.
\end{example}
\begin{solution}
    首先由$f(x)>0,f'(x)>0,f''(x)>0$得知$f'(x),f(x)$均单调递增,所以
    \[f(x+h)>f(x)+f'(\xi)(x+h-x)>f(x)+f'(x)h\]
    令$h\to+\infty$,得$f(x)\to+\infty$,现在对已知极限变形:
    \[a=\lim_{x\to+\infty}\dfrac{f'(x)f'''(x)}{[f''(x)]^2}=1-\lim_{x\to+\infty}\dfrac{[f''(x)]^2-f'(x)f'''(x)}{[f''(x)]^2}=1-\lim_{x\to+\infty}\dfrac{\dd}{\dd x}\left(\dfrac{f'(x)}{f''(x)}\right)\]
    所以$\displaystyle\lim_{x\to+\infty}\dfrac{\dd}{\dd x}\left(\dfrac{f'(x)}{f''(x)}\right)=1-a\ne 0$,同样可以对所求极限变形:
    \[\lim_{x\to+\infty}\dfrac{f(x)f''(x)}{[f'(x)]^2}=1-\lim_{x\to+\infty}\dfrac{[f'(x)]^2-f(x)f''(x)}{[f'(x)]^2}=1-\lim_{x\to+\infty}\dfrac{\dd}{\dd x}\left(\dfrac{f(x)}{f'(x)}\right)\]
    问题在于如何沟通$\displaystyle\lim_{x\to+\infty}\dfrac{\dd}{\dd x}\left(\dfrac{f'(x)}{f''(x)}\right)=1-a$和$\displaystyle\lim_{x\to+\infty}\dfrac{\dd}{\dd x}\left(\dfrac{f(x)}{f'(x)}\right)$,观察到它们是微分形式,所以不妨反方向利用洛必达法则。须知,洛必达法则在$\dfrac{*}{\infty}$的情形中是适用的,即若$f\to\infty,f,g$在$a$的某个去心邻域内可导,且$\displaystyle\lim_{x\to a}\dfrac{g'}{f'}$存在(或为无穷大,但分母不等于$0$),那么$\displaystyle\lim_{x\to a}\dfrac{g}{f}=\lim_{x\to a}\dfrac{g'}{f'}$;相关证明可以参阅数分教材(如陈纪修的,卓里奇的,等等).
    \begin{align*}\lim_{x\to+\infty}\dfrac{\frac{f(x)}{f'(x)}}{x}=\lim_{x\to+\infty}\dfrac{\dd}{\dd x}\left(\dfrac{f(x)}{f'(x)}\right)=\lim_{x\to+\infty}\dfrac{f(x)}{xf'(x)}\\
    \lim_{x\to+\infty}\dfrac{\frac{f'(x)}{f''(x)}}{x}=\lim_{x\to+\infty}\dfrac{\dd}{\dd x}\left(\dfrac{f'(x)}{f''(x)}\right)=\lim_{x\to+\infty}\dfrac{f'(x)}{xf''(x)}\end{align*}
    设$\displaystyle\lim_{x\to+\infty}\dfrac{f(x)f''(x)}{[f'(x)]^2}=A,\lim_{x\to+\infty}\dfrac{\dd}{\dd x}\left(\dfrac{f(x)}{f'(x)}\right)=1-A,\lim_{x\to+\infty}\dfrac{\dd}{\dd x}\left(\dfrac{f'(x)}{f''(x)}\right)=\lim_{x\to+\infty}\dfrac{f'(x)}{xf''(x)}=1-a$
    \[\displaystyle\lim_{x\to+\infty}\dfrac{xf''(x)}{f'(x)}\lim_{x\to+\infty}\dfrac{f(x)}{xf'(x)}=\lim_{x\to+\infty}\dfrac{f(x)f''(x)}{[f'(x)]^2}\]
    由于$\lim_{x\to+\infty}\dfrac{f'(x)}{xf''(x)}=1-a\Rightarrow\lim_{x\to+\infty}\dfrac{xf''(x)}{f'(x)}=\dfrac1{1-a}$即$\dfrac{1-A}{1-a}=A$,解得\[A=\displaystyle\lim_{x\to+\infty}\dfrac{f(x)f''(x)}{[f'(x)]^2}=\dfrac1{2-a}\]
\end{solution}
\begin{example}{求极限}{}
    \[\lim_{n\to+\infty}n^x\left[\left(1+\dfrac1{n+1}\right)^{n+1}-\left(1+\dfrac1n\right)^n\right]\]
\end{example}
\begin{solution}
根据做差的形式不难想到拉格朗日中值定理,但是如果直接凑分母 $1=n+1-n$,那就会导致 $\xi\in(n,n+1)$ 的东西出现,而没有很好的利用上
\[
\left(1+\dfrac1{n+1}\right)^{n+1}-\left(1+\dfrac1n\right)^n\to 0,n\to+\infty
\]
所以先化为 $\mathrm{e}$ 指数,用取对数之后的差分形式作为分母$\mathrm{e}^{f}-\mathrm{e}^{g}=\mathrm{e}^{\xi}(f-g),\xi\in(f,g)$:
\begin{align*}
&\lim_{n\to+\infty}n^x\left[\left(1+\dfrac1{n+1}\right)^{n+1}-\left(1+\dfrac1n\right)^n\right]=\lim_{n\to+\infty}n^x\left[\mathrm{e}^{(n+1)\ln(1+\frac1{n+1})}-\mathrm{e}^{n\ln(1+\frac1n)}\right]\\
&=\lim_{n\to+\infty}n^x\left[\dfrac{\mathrm{e}^{(n+1)\ln(1+\frac1{n+1})}-\mathrm{e}^{n\ln(1+\frac1n)}}{(n+1)\ln(1+\frac1{n+1})-n\ln(1+\frac1n)}\right]\left[(n+1)\ln(1+\frac1{n+1})-n\ln(1+\frac1n)\right]\\
&=\lim_{n\to+\infty}n^x \mathrm{e}^{\xi}\left[(n+1)\ln(1+\frac1{n+1})-n\ln(1+\frac1n)\right]
\end{align*}
其中 $\mathrm{e}^{\xi}\in\left(\left(1+\dfrac1n\right)^n,\left(1+\dfrac1{n+1}\right)^{n+1}\right)$,当 $n\to+\infty$ 时, $\xi\to1$,所以:
\begin{align*}
&=\lim_{n\to+\infty}n^x\cdot \mathrm{e}\left[(n+1)\ln(1+\frac1{n+1})-n\ln(1+\frac1n)\right]\\
&=\lim_{n\to+\infty}n^x\cdot \mathrm{e}\left[ \left(1 - \frac{1}{2(n+1)} + \frac{1}{3(n+1)^2} + o\left(\frac{1}{(n+1)^2}\right)\right) - \left(1 - \frac{1}{2n} + \frac{1}{3n^2} + o\left(\frac{1}{n^2}\right)\right) \right] \\
&=\lim_{n\to+\infty}n^x\cdot \mathrm{e}\left[ -\frac{1}{2(n+1)} + \frac{1}{2n} + \frac{1}{3(n+1)^2} - \frac{1}{3n^2} + o\left(\frac{1}{n^2}\right) \right] \\
&=\lim_{n\to+\infty}n^x\cdot \mathrm{e}\left[ \frac{1}{2n(n+1)} + \frac{1}{3} \cdot \frac{n^2 - (n+1)^2}{n^2(n+1)^2} + o\left(\frac{1}{n^2}\right) \right] \\
&=\lim_{n\to+\infty}n^x\cdot \mathrm{e}\left[ \frac{1}{2n(n+1)} - \frac{2n+1}{3n^2(n+1)^2} + o\left(\frac{1}{n^2}\right) \right] \\
&=\lim_{n\to+\infty}n^x\cdot \mathrm{e}\left[ \frac{1}{2n^2} + o\left(\frac{1}{n^2}\right) \right] \\
&=\frac{\mathrm{e}}{2} \lim_{n\to+\infty} n^{x-2}
\end{align*}
因此,极限为:
\[
\lim_{n\to+\infty}n^x\left[\left(1+\dfrac1{n+1}\right)^{n+1}-\left(1+\dfrac1n\right)^n\right] = 
\begin{cases} 
0 & \text{如果 } x < 2 \\
\dfrac{\mathrm{e}}{2} & \text{如果 } x = 2 \\
+\infty\text{或不存在} & \text{如果 } x > 2 
\end{cases}
\]
\end{solution}
\newpage
\begin{example}{}{}
    设$f(x)=\left(\tan\dfrac{\pi x}4-1\right)\left(\tan\dfrac{\pi x^2}4-1\right)\left(\tan\dfrac{\pi x^3}4-1\right)\cdots\left(\tan\dfrac{\pi x^n}4-1\right)$,求$\displaystyle f^{(n)}(x)$
\end{example}
\begin{solution}
等价于给定函数 $\displaystyle f(x) = \prod_{k=1}^n \left( \tan \frac{\pi x^k}{4} - 1 \right)$,求 $f^{(n)}(1)$。由于当 $x=1$ 时,每个因子 $\tan \frac{\pi x^k}{4} - 1 = 0$,因此 $x=1$ 是 $f(x)$ 的 $n$ 重零点。由泰勒公式,在 $x=1$ 处展开:
\[
f(x) = \frac{f^{(n)}(1)}{n!} (x-1)^n + O((x-1)^{n+1}).
\]
计算 $h(1) = \displaystyle\lim_{x \to 1} \frac{f(x)}{(x-1)^n}$。由于 $f(x)$ 是乘积形式,我们将每个因子除以$x-1$有
\[
\frac{f(x)}{(x-1)^n} = \prod_{k=1}^n \frac{\tan \frac{\pi x^k}{4} - 1}{x-1}.
\]
取极限 $x \to 1$,每个因子的极限为导数:
\[
\lim_{x \to 1} \frac{\tan \frac{\pi x^k}{4} - 1}{x-1} = \frac{\dd}{\dd x} \left. \left( \tan \frac{\pi x^k}{4} \right) \right|_{x=1} = \frac{\pi}{4} \cdot k x^{k-1} \sec^2 \frac{\pi x^k}{4} \Big|_{x=1}=\frac{\pi}{4} \cdot k \cdot 1 \cdot 2 = \frac{\pi k}{2}.
\]
所以
\[
h(1) = \prod_{k=1}^n \frac{\pi k}{2} = \frac{\pi^n}{2^n} \prod_{k=1}^n k = \frac{\pi^n}{2^n} n!\quad f^{(n)}(1) = n! \cdot h(1) = \left( \frac{\pi}{2} \right)^n (n!)^2.
\]

当然本题目还可以先代入泰勒展开,再相乘,结果是一样的,对于第 $k$ 个因子 $\tan\left(\dfrac{\pi x^k}{4}\right) - 1$直接泰勒展开得:
\[
\tan\left(\frac{\pi x^k}{4}\right) - 1 = \frac{\pi k}{2} (x - 1) + O((x-1)^2)
\]
将各因子相乘,注意每个因子都包含 $(x-1)$ 的一阶项:
\[
f(x) = \prod_{k=1}^n \left( \tan\left(\frac{\pi x^k}{4}\right) - 1 \right) 
= \prod_{k=1}^n \left[ \frac{\pi k}{2} (x-1) + O((x-1)^2) \right]= \dfrac{\pi}{2}\prod_{k=1}^n \left[ k (x-1) + O((x-1)^2) \right]
\]
实际上我们只关心 $(x-1)^n$ 项的系数(因为高阶项对 $f^{(n)}(1)$ 无贡献),可得:
\[
f(x) = \left( \frac{\pi}{2} \right)^n \left( \prod_{k=1}^n k \right) (x-1)^n + O((x-1)^{n+1})
= \left( \frac{\pi}{2} \right)^n n! (x-1)^n + O((x-1)^{n+1})
\]
由泰勒公式:
\[
f(x) = \frac{f^{(n)}(1)}{n!} (x-1)^n + O((x-1)^{n+1})
\]
比较 $(x-1)^n$ 的系数:
\[
\frac{f^{(n)}(1)}{n!} = \left( \frac{\pi}{2} \right)^n n! \quad \Rightarrow \quad f^{(n)}(1) = \left( \frac{\pi}{2} \right)^n (n!)^2
\]
\end{solution}
\newpage
\begin{example}{}{}
    $f(x)=(x-1)(\sqrt x-1)(\sqrt[3]x-1)\cdots(\sqrt[n]x-1)$,求$\displaystyle f^{(n)}(1)$。
\end{example}
\begin{solution}
由于当 $x=1$ 时,每个因子 $x^{1/k} - 1 = 0$,因此 $x=1$ 是 $f(x)$ 的 $n$ 重零点。在 $x=1$ 处展开:
\[
f(x) = \frac{f^{(n)}(1)}{n!} (x-1)^n + O((x-1)^{n+1}).
\]
现在计算 $g(1) = \displaystyle\lim_{x \to 1} \frac{f(x)}{(x-1)^n}$。由于 $f(x)$ 是乘积,我们有
\[
\frac{f(x)}{(x-1)^n} = \prod_{k=1}^{n} \frac{x^{1/k} - 1}{x-1}.
\]
取极限 $x \to 1$,每个因子的极限为导数:
\[
\lim_{x \to 1} \frac{x^{1/k} - 1}{x-1} = \frac{\dd}{\dd x} \left. \left( x^{1/k} \right) \right|_{x=1} = \frac{1}{k} \cdot x^{1/k - 1} \Big|_{x=1} = \frac{1}{k}.
\]
因此
\[
g(1) = \prod_{k=1}^{n} \frac{1}{k} = \frac{1}{1 \cdot 2 \cdots n} = \frac{1}{n!}\quad f^{(n)}(1) = n! \cdot \frac{1}{n!} = 1.
\]
\end{solution}
\newpage
\begin{example}{}{}
    函数$f(x)$在区间$[0,1]$上有连续导数,且$\displaystyle \int_0^1f(x)\dd x=0$,证明对于任意的$\xi\in(0,1)$:\[\left|\int_0^{\xi}f(x)\dd x\right|\leqslant \dfrac18\max_{x\in[0,1]}|f'(x)|\]
\end{example}
\begin{solution}
由于 $f(x)$ 在 $[0,1]$ 上有连续导数,根据牛顿-莱布尼茨公式,对于任意 $\xi \in (0,1)$,考虑积分
\[
\int_0^\xi f(x) \dd x = \int_0^\xi \left( f(0) + \int_0^x f'(t) \dd t \right) \dd x = f(0) \xi + \int_0^\xi \int_0^x f'(t) \dd t \dd x.
\]
交换积分次序,
\[
\int_0^\xi \int_0^x f'(t) \dd t \dd x = \int_0^\xi \int_t^\xi f'(t) \dd x \dd t = \int_0^\xi (\xi - t) f'(t) \dd t.
\]
所以
\[
\int_0^\xi f(x) \dd x = f(0) \xi + \int_0^\xi (\xi - t) f'(t) \dd t.
\]
代入$\xi=1,\displaystyle  \int_0^1f(x)\dd x=0\Rightarrow f(0) = - \int_0^1 (1 - t) f'(t) \dd t$,所以代入并分区间讨论,利用绝对值不等式,有
\[
\left|\int_0^\xi f(x) \dd x \right|=\begin{cases}\displaystyle\left|\int_0^{\xi}(\xi-1)tf'(t)\dd t\right|\leqslant \max_{x \in [0,1]} |f'(x)|\left|\int_0^{\xi}(\xi-1)t\dd t\right|,&0\leqslant t\leqslant\xi\\[1.2ex]\displaystyle\left|\int_{\xi}^1\xi(t-1)f'(t)\dd t\right|\leqslant \max_{x \in [0,1]} |f'(x)|\left|\int_{\xi}^1\xi(t-1)\dd t\right|,&\xi\leqslant t\leqslant 1
\end{cases}
\]
去绝对值,只需分别计算:
\[
\int_0^\xi (1 - \xi) t \dd t = (1 - \xi) \cdot \frac{1}{2} \xi^2 = \frac{1}{2} (1 - \xi) \xi^2\quad
\int_\xi^1 \xi (1 - t) \dd t = \xi \cdot \frac{1}{2} (1 - \xi)^2 = \frac{1}{2} \xi (1 - \xi)^2.
\]
利用$\dfrac{\displaystyle \left|\int_0^\xi f(x) \dd x \right|}{\displaystyle \max_{x \in [0,1]} |f'(x)|}\leqslant \begin{cases}\displaystyle\left|\int_0^{\xi}(\xi-1)t\dd t\right|,&0\leqslant t\leqslant\xi\\\displaystyle\left|\int_{\xi}^1\xi(t-1)\dd t\right|,&\xi\leqslant t\leqslant 1\end{cases}$
相加得
\[
\dfrac{\displaystyle \left|\int_0^1 f(x) \dd x \right|}{\displaystyle \max_{x \in [0,1]} |f'(x)|}\leqslant \frac{1}{2} (1 - \xi) \xi^2 + \frac{1}{2} \xi (1 - \xi)^2 = \frac{1}{2} \xi (1 - \xi) \left( \xi + (1 - \xi) \right) = \frac{1}{2} \xi (1 - \xi)\leqslant\dfrac18.
\]
这就完成了证明。
\end{solution}
\begin{example}{}{}
$a_1=5,a_{n+1}=\dfrac{(1+a_n)^3-5}{3}$,计算$\displaystyle\sum_{n=1}^{\infty}\dfrac{a_n-1}{a_n^2+a_n+1}$
\end{example}
\begin{solution}
关键在于不动点方程的多重根,因为对于$a_{n+1}=f(a_n)$,若方程$f(x)-x=0$有二重根(甚至多重根)r,那么设$g(x)=f(x)-x$,则其可以被分解出一个因式$(x-r)^2$,此时不妨设
\[a_{n+1}-a_n=f(a_n)-a_n=(a_n-r)^2h(a_n)\]
然后就是等式两边减去$r$,此时:
\[a_{n+1}-r=f(a_n)-r=(a_n-r)^2h(a_n)+a_n-r=(a_n-r)(1+(a_n-r)h(a_n))\]
在尝试裂项求和时,常常考虑$\dfrac{1}{a_n-r}$的差分形式:
\begin{align*}
    \dfrac{1}{a_n-r}-\dfrac1{a_{n+1}-r}&=\dfrac{a_{n+1}-a_n}{(a_n-r)(a_{n+1}-r)}=\dfrac{(a_n-r)^2h(a_n)}{(a_n-r)(a_n-r)(1+(a_n-r)h(a_n))}\\&=\dfrac{h(a_n)}{1+(a_n-r)h(a_n)}
\end{align*}
此时分母的次数一定低于$f(a_n)$的次数,反观如果是一重零点,就变成
\begin{align*}
    \dfrac{1}{a_n-r}-\dfrac1{a_{n+1}-r}&=\dfrac{a_{n+1}-a_n}{(a_n-r)(a_{n+1}-r)}=\dfrac{(a_n-r)h(a_n)}{(a_n-r)^2(1+h(a_n))}\\&=\dfrac{h(a_n)}{(a_n-r)(1+h(a_n))}
\end{align*}
此时分母次数一定等于$f(a_n)$的次数,不符合题干结构。所以关注多重根,不妨通过做差来解:
\begin{align*}
    a_{n+1}-a_{n}&=\dfrac{(1+a_n)^3-3a_n-5}{3}=\dfrac{a_n^3+3a_n^2-4}{3}=\dfrac{(a_n+2)^2(a_n-1)}3\\
    \Leftrightarrow\dfrac{a_{n+1}-a_n}{(a_n+2)(a_{n+1}+2)}&=\dfrac{(a_n+2)^2(a_n-1)}{3(a_n+2)(a_{n+1}+2)}=\dfrac{a_n+2}{a_{n+1}+2}\dfrac{a_n-1}3\\
    \Leftrightarrow\dfrac{1}{a_n+2}-\dfrac1{a_{n+1}+2}&=\dfrac{a_n+2}{\dfrac{(1+a_n)^3-5}{3}+2}\dfrac{a_n-1}3=(a_n-1)\dfrac{a_n+2}{(1+a_n)^3+1}\\
    &=\dfrac{(a_n+2)(a_n-1)}{(a_n+1+1)((a_n+1)^2-(a_n+1)+1)}=\dfrac{a_n-1}{a_n^2+a_n+1}\\
    \Rightarrow\sum_{n=1}^{\infty}\dfrac{a_{n}-1}{a_n^2+a_n+1}&=\sum_{n=1}^{\infty}\left(\dfrac{1}{a_n+2}-\dfrac1{a_{n+1}+2}\right)=\dfrac{1}{a_1+2}=\dfrac17
\end{align*}
第一行找出来了$-2$是多重根,之后直接除以$(a_n+2)(a_{n+1}+2)$强行凑裂项,之后就出现了$\dfrac{a_n-r}{a_{n+1}-r}$结构,直接代入递推公式就可以,这就是通法了。
\end{solution}

\begin{theorem}{}{}
设$\lambda> 1,a_0\ne 0,a_1=\dfrac{\lambda+\lambda^{-1}}2a_0$,递推关系$a_{n+2}-(\lambda+\lambda^{-1})a_{n+1}+a_n=0$,则
\[a_n=\dfrac{a_0}{2}(\lambda^n+\lambda^{-n})\]进一步的
\[\sum_{n=0}^{\infty}\dfrac{4^n}{a_{2^n}^2}=\dfrac{4}{a_0^2}\dfrac{1}{(\lambda-\lambda^{-1})^2}\]
\end{theorem}
    注意到恒等式
    \begin{align*}
        \dfrac{1}{(a+a^{-1})^2}&=\dfrac{(a-a^{-1})^2}{(a^2-a^{-2})^2}=\dfrac{(a+a^{-1})^2-4}{(a^2-a^{-2})^2}\\
        &=\dfrac{(a+a^{-1})^2}{(a^2-a^{-2})^2}-\dfrac{4}{(a^2-a^{-2})^2}\\
        &=\dfrac1{(a-a^{-1})^2}-\dfrac{4}{(a^2-a^{-2})^2}
    \end{align*}
    代入$a=\lambda^{2^n}$得:
    \begin{align*}
        &\dfrac{1}{(\lambda^{2^n}+\lambda^{-2^n})^2}=\dfrac{1}{(\lambda^{2^n}-\lambda^{-2^n})^2}-\dfrac{4}{(\lambda^{2^{n+1}}-\lambda^{-2^{n+1}})^2}\\
        \Leftrightarrow &\dfrac{4^n}{(\lambda^{2^n}+\lambda^{-2^n})^2}=\dfrac{4^n}{(\lambda^{2^n}-\lambda^{-2^n})^2}-\dfrac{4^{n+1}}{(\lambda^{2^{n+1}}-\lambda^{-2^{n+1}})^2}\\
        \Leftrightarrow&\sum_{n=0}^{\infty}\dfrac{4^n}{a_{2^n}^2}=\dfrac{4}{a_0^2}\sum_{n=0}^{\infty}\left(\dfrac{4^n}{(\lambda^{2^n}-\lambda^{-2^n})^2}-\dfrac{4^{n+1}}{(\lambda^{2^{n+1}}-\lambda^{-2^{n+1}})^2}\right)
        =\dfrac{4}{a_0^2}\dfrac{1}{(\lambda-\lambda^{-1})^2}
    \end{align*}
\newpage
\begin{theorem}{}{}
    设$a_0$为已知量,数列$\{a_n\}$满足通项公式$a_n=C(\lambda^n-\lambda^{-n})$,则
    \[\sum_{n=0}^{\infty}\dfrac{1}{a_{2^n}}=\dfrac1{a_1}+\dfrac1{a_k}\left(\dfrac{1}{\lambda^k}-\dfrac{a_{1-k}}{a_1}\right)\]
\end{theorem}
注意到对任意整数$k$,成立恒等式:
\begin{align*}
    \dfrac{\lambda^k-\lambda^{-k}}{\lambda^{2n}-\lambda^{-2n}}&=\dfrac{\lambda^n\lambda^{-n}\lambda^k-\lambda^n\lambda^{-n}\lambda^{-k}}{\lambda^{2n}-\lambda^{-2n}}\\
    &=\dfrac{\lambda^n\lambda^{-n}\lambda^k-\lambda^n\lambda^{-n}\lambda^{-k}-(\lambda^{2n}\lambda^{-k}-\lambda^{-2n}\lambda^k)+(\lambda^{2n}\lambda^{-k}-\lambda^{-2n}\lambda^k)}{\lambda^{2n}-\lambda^{-2n}}\\
    &=\dfrac{\lambda^n\lambda^{-n}\lambda^k+\lambda^{-2n}\lambda^k-\lambda^n\lambda^{-n}\lambda^{-k}-\lambda^{2n}\lambda^{-k}+(\lambda^{2n}\lambda^{-k}-\lambda^{-2n}\lambda^k)}{\lambda^{2n}-\lambda^{-2n}}\\
    &=\dfrac{(\lambda^n+\lambda^{-n})(\lambda^{-n}\lambda^{k})-\lambda^n\lambda^{-k}(\lambda^n+\lambda^{-n})+(\lambda^{2n}\lambda^{-k}-\lambda^{-2n}\lambda^k)}{\lambda^{2n}-\lambda^{-2n}}\\
    &=\dfrac{(\lambda^n+\lambda^{-n})(\lambda^{-n}\lambda^{k}-\lambda^n\lambda^{-k})+(\lambda^{2n}\lambda^{-k}-\lambda^{-2n}\lambda^k)}{\lambda^{2n}-\lambda^{-2n}}\\
    &=-\dfrac{\lambda^n\lambda^{-k}-\lambda^{-n}\lambda^{k}}{\lambda^{n}-\lambda^{-n}}+\dfrac{\lambda^{2n}\lambda^{-k}-\lambda^{-2n}\lambda^k}{\lambda^{2n}-\lambda^{-2n}}\\
    &=\frac{\lambda^{2n-k} - \lambda^{-(2n-k)}}{\lambda^{2n} - \lambda^{-2n}}-\frac{\lambda^{n-k} - \lambda^{-(n-k)}}{\lambda^{n} - \lambda^{-n}} 
\end{align*}
赋值$n=2^{n-1}$,保留$k$,那么
\begin{align*}
\dfrac{\lambda^k-\lambda^{-k}}{\lambda^{2^{n}}-\lambda^{-2^{n}}}=\frac{\lambda^{2^n-k} - \lambda^{-(2^n-k)}}{\lambda^{2^n} - \lambda^{-2^n}}-\frac{\lambda^{2^{n-1}-k} - \lambda^{-(2^{n-1}-k)}}{\lambda^{2^{n-1}} - \lambda^{-2^{n-1}}} 
\end{align*}
所以当数列的通项是$a_n=C(\lambda^n-\lambda^{-n})$时,即:
\begin{align*}
\dfrac{a_k}{a_{2^n}}=\dfrac{a_{2^n-k}}{a_{2^{n}}}-\dfrac{a_{2^{n-1}-k}}{a_{2^{n-1}}}\Leftrightarrow& \dfrac{1}{a_{2^n}}=\dfrac1{a_k}\left(\dfrac{a_{2^n-k}}{a_{2^{n}}}-\dfrac{a_{2^{n-1}-k}}{a_{2^{n-1}}}\right)\\
\Rightarrow& \sum_{n=0}^{\infty}\dfrac{1}{a_{2^n}}=\dfrac1{a_1}+\dfrac1{a_k}\sum_{n=1}^{\infty}\left(\dfrac{a_{2^n-k}}{a_{2^{n}}}-\dfrac{a_{2^{n-1}-k}}{a_{2^{n-1}}}\right)\\
\Leftrightarrow& \sum_{n=0}^{\infty}\dfrac{1}{a_{2^n}}=\dfrac1{a_1}+\dfrac1{a_k}\left(\dfrac{1}{\lambda^k}-\dfrac{a_{1-k}}{a_1}\right)
\end{align*}
此时需将$a_n$的定义域延拓至整数集合,相应地,若$a_n=C(\lambda^n-(-\lambda^{-1})^{n})$,则只需取$k$为偶数$2k$,这样就回到了本题的情形:
\begin{theorem}{}{}
    设$a_0$为已知量,数列$\{a_n\}$满足通项公式$a_n=C(\lambda^n-(-\lambda^{-1})^{n})$,则
    \[\sum_{n=0}^{\infty}\dfrac{1}{a_{2^n}}=\dfrac1{a_1}+\dfrac1{a_{2k}}\left(\dfrac{1}{\lambda^{2k}}-\dfrac{a_{1-2k}}{a_1}\right)\]
\end{theorem}




\begin{example}{}{}
    $a_1=a_2=1,a_{n+2}=a_{n+1}+a_n$,求$\displaystyle $
\end{example}
\begin{example}{}{}
    $f(x)$在区间$[0,1]$上连续,证明存在$\xi\in(0,1)$使得\[\int_{\xi}^1f(x)\dd x=\xi f(\xi)\]
\end{example}
\begin{solution}
    设$f(x)$的原函数是$F(x)$,则要证明的结论等价为:
    \begin{align*}
        \int_{\xi}^1f(x)\dd x=\xi F(\xi)&\Leftrightarrow F(1)-F(\xi)=\xi F'(\xi)\\
&\Leftrightarrow \xi F'(\xi)+F(\xi)=F(1)\\
&\Leftrightarrow (\xi F(\xi))'=F(1)
    \end{align*}
    所以辅助函数就是$g(x)=xF(x)$,它在$[0,1]$内也连续,要证明存在$\xi\in(0,1)$使得$g'(\xi)=g(1)$,即:
    \[g'(\xi)=\dfrac{g(1)-0}{1-0}=\dfrac{g(1)-g(0)}{1-0}\]
    这正是拉格朗日中值定理的应用。
\end{solution}
\begin{example}{}{}
    设$f(x)$在区间$[0,1]$上可导,且$f(1)=0$,$\displaystyle \int_0^1xf'(x)\dd x=1$,证明:至少存在$\xi\in(0,1)$使得$f'(\xi)=2$。
\end{example}
\begin{solution}
    \[1=\displaystyle \int_0^1xf'(x)\dd x=\int_0^1x\dd f(x)=xf(x)\Big|_0^1-\int_0^1f(x)\dd x\Rightarrow \int_0^1f(x)\dd x=-1\]
    所以设$f(x)$的原函数为$F(x)$,则有$F(1)-F(0)=-1$,使用泰勒公式,存在$\xi\in(0,1)$使得:
    \[F(0)=F(1)+F'(1)(0-1)+\dfrac{1}{2}F''(\xi)\Rightarrow F''(\xi)=f'(\xi)=2\]
\end{solution}
\begin{example}{}{}
    \[\int_0^1\dfrac{x^2-1}{\ln x}\dd x\]
\end{example}
\begin{solution}
\begin{align*}
\int_0^1\dfrac{x^2-1}{\ln x}\dd x&=\int_0^1\dfrac{x^2-x^0}{\ln x}\dd x=\int_0^1\int_0^2x^y\dd y\dd x\\
&=\int_0^2\int_0^1x^y\dd x\dd y=\int_0^2\dfrac{\dd y}{y+1}=\ln|y+1|\Big|_0^2\\
&=\ln 3
\end{align*}
\end{solution}
\newpage
\begin{example}{}{}
    \[\int_0^{\frac{\pi}{2}}\dfrac{x\sin x}{1+\cos^2x}\dd x\]
\end{example}
\begin{solution}
\begin{align*}
\int_0^{\frac{\pi}{2}}\dfrac{x\sin x}{1+\cos^2x}\dd x&=-\int_0^{\frac{\pi}{2}}\dfrac{x\dd\cos x}{1+\cos^2 x}=\int_0^{\frac{\pi}{2}}x\dd\arctan(\cos x)\\
&=-x\arctan \cos x\Big|_0^{\frac{\pi}{2}}+\int_0^{\frac{\pi}{2}}\arctan\cos x\dd x\\
&=\int_0^{\frac{\pi}{2}}\arctan\cos x\dd x=\int_0^{\frac{\pi}{2}}\arctan\sin x\dd x
\end{align*}
令$I(a)=\displaystyle \int_0^{\frac{\pi}{2}}\arctan a\sin x\dd x$,则
\begin{align*}
    I'(a)=&\dfrac{\dd}{\dd a}\left(\int_0^{\frac{\pi}{2}}\arctan a\sin x\dd x\right)=\int_0^{\frac{\pi}{2}}\left(\dfrac{\dd}{\dd a}\arctan a\sin x\right)\dd x\\
    &=\int_0^{\frac{\pi}{2}}\dfrac{\sin x}{1+a^2\sin^2 x}\dd x=\int_0^{\frac{\pi}{2}}\frac{\sin x}{\left(a^2+1\right)-a^2\cos^2x}\mathrm{d}x\\&=-\int_0^{\frac{\pi}{2}}\frac{\mathrm{d}\cos x}{\left(a^2+1\right)-a^2\cos^2x}=\int_0^1\frac{\mathrm{d}t}{\left(a^2+1\right)-a^2t^2}=\frac{1}{a^2}\int_0^1\frac{\mathrm{d}t}{\frac{a^2+1}{a^2}-t^2}\\&=\frac{1}{2a\sqrt{a^2+1}}\mathrm{ln}\left|\frac{\frac{\sqrt{a^2+1}}{a}+t}{\frac{\sqrt{a^2+1}}{a}-t}\right||_0^1=\frac{\ln\left(\sqrt{a^2+1}+a\right)}{a\sqrt{a^2+1}}
\end{align*}
于是
\begin{align*}I(1)&=\int_0^1I^{\prime}(a)\mathrm{d}a\\&=\int_0^1\frac{\ln\left(\sqrt{a^2+1}+a\right)}{a\sqrt{a^2+1}}\mathrm{d}a\\&=-\ln(\sqrt{1+a^2}+a)\ln\left(\frac{\sqrt{1+a^2}+1}{a}\right)|_0^1+\int_0^1\frac{\ln\left(\frac{\sqrt{1+a^2}+1}{a}\right)}{\sqrt{1+a^2}}\mathrm{d}a\\&=-\ln^{2}(1+\sqrt{2})+\int_{0}^{1}\frac{\ln\left(\frac{\sqrt{1+a^{2}}+1}{a}\right)}{\sqrt{1+a^{2}}}\mathrm{d}a\\&=-\ln^2(1+\sqrt{2})+\int_1^\infty\frac{\ln\left(\sqrt{a^2+1}+a\right)}{a\sqrt{1+a^2}}\mathrm{d}a\\&=-\ln^2(1+\sqrt{2})+I(\infty)-I(1)\end{align*}
于是
\[I(1)=-\frac{1}{2}\mathrm{ln}^2(1+\sqrt{2})+\frac{I(\infty)}{2}=-\frac{1}{2}\mathrm{ln}^2(1+\sqrt{2})+\frac{\pi^2}{8}\]
\end{solution}
\begin{example}{}{}
    \[\int_0^{\frac{\pi}{2}}\dfrac{x\sin x}{1+\cos^2x}\dd x\]
\end{example}
\begin{solution}
    \begin{align*}
\int_0^{\frac{\pi}{2}}\dfrac{x\sin x}{1+\cos^2x}\dd x&=-\int_0^{\frac{\pi}{2}}\dfrac{x\dd\cos x}{1+\cos^2 x}=\int_0^{\frac{\pi}{2}}x\dd\arctan(\cos x)\\
&=-x\arctan \cos x\Big|_0^{\frac{\pi}{2}}+\int_0^{\frac{\pi}{2}}\arctan\cos x\dd x=\int_0^{\frac{\pi}{2}}\arctan\cos x\dd x\\
&=\int_0^{\frac{\pi}{2}}\int_0^1\dfrac{\cos x}{1+y^2\cos^2x}\dd y\dd x=\int_0^1\int_0^{\frac{\pi}{2}}\dfrac{\cos x}{1+y^2\cos^2x}\dd x\dd y\\
&=\int_0^{1}\int_0^1\dfrac{\dd\sin x}{1+y^2(1-\sin^2 x)}\dd y=\int_0^1\int_0^1\dfrac{\dd\sin x}{(\sqrt{1+y^2})^2-(y\sin x)^2}\dd y\\
&=\dfrac{1}{2y\sqrt{1+y^2}}\int_0^1\int_0^1\left(\dfrac1{\sqrt{1+y^2}+y\sin x}+\dfrac1{\sqrt{1+y^2}-y\sin x}\right)\dd\sin x\dd y\\
&=\int_0^1\dfrac{\ln(y+\sqrt{1+y^2})}{y\sqrt{1+y^2}}\dd y \xlongequal{y=\frac{t^2-1}{2t}\text{(万能代换)}} 2 \int_1^{1+\sqrt{2}} \frac{\ln t}{t^2 - 1}\dd t\\
    2 \int_1^{1+\sqrt{2}} \frac{\ln t}{t^2 - 1}\dd t&=\int_1^{1+\sqrt2}\ln t\left(\dfrac1{t-1}-\dfrac1{t+1}\right)\dd t=\int_1^{1+\sqrt2}\ln t~\dd\ln\dfrac{t-1}{t+1}\\
   &=-\ln^2(1+\sqrt2)- \int_1^{1+\sqrt{2}} \frac{1}{t} \ln \frac{t-1}{t+1}\dd t\quad\text{令}\frac{t-1}{t+1}=m\\
    &= -\ln^2(1+\sqrt2) - \int_0^{\sqrt{2}-1} \dfrac{1-m}{1+m}\ln\dfrac{\frac{1+m}{1-m}-1}{\frac{1+m}{1-m}+1}\dd\frac{1+m}{1-m}\\
    &=-\ln^2(1+\sqrt2) - 2\int_0^{\sqrt{2}-1}\dfrac{\ln m}{1-m^2}\dd m=-\ln^2(1+\sqrt2)-2\int_{+\infty}^{\sqrt2+1}\dfrac{\ln\frac1u}{1-\frac1{u^2}}\dd\frac1u\\
    &= -\ln^2(1+\sqrt2) - 2\int_{+\infty}^{\sqrt2+1}\dfrac{\ln u}{u^2-1}\dd u=-\ln^2(1+\sqrt2) + 2\int_{\sqrt2+1}^{+\infty}\dfrac{\ln u}{u^2-1}\dd u\\
    &=-\ln^2(1+\sqrt2) + 2\int_1^{+\infty}\dfrac{\ln t}{t^2-1}\dd t-2\int_1^{1+\sqrt2}\dfrac{\ln t}{t^2-1}\dd t\\
    2 \int_1^{1+\sqrt{2}} \frac{\ln t}{t^2 - 1}\dd t&=-\dfrac12\ln^2(1+\sqrt2) +\int_1^{+\infty}\dfrac{\ln t}{t^2-1}\dd t\xlongequal{t\leftarrow \frac1t}-\dfrac12\ln^2(1+\sqrt2) -\int_0^1\dfrac{\ln t}{1-t^2}\dd t\\
    &=-\dfrac12\ln^2(1+\sqrt2)-\int_0^1\sum_{n=0}^{\infty}t^{2n}\ln t\dd t=-\dfrac12\ln^2(1+\sqrt2)+\sum_{n=0}^{\infty}\dfrac{1}{(2n+1)^2}\\
    &=\dfrac{\pi^2}8-\dfrac12\ln^2(1+\sqrt2)
\end{align*}

\end{solution}