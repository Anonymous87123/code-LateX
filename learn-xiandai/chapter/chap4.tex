\chapter{第四章作业}
\begin{example}{}{}
    解线性方程组$\begin{cases}x_1+2x_2-5x_3+4x_4+x_5&=4,\\3x_1+7x_2-x_3-3x_4+2x_5&=10,\\-x_2-13x_3-2x_4+x_5&=-14,\\x_3-16x_4+2x_5&=-11,\\2x_4+5x_5&=12;\end{cases}$.
\end{example}
\begin{solution}
\end{solution}
\begin{example}{}{}
    设$\beta=( 1, 2, 1, 1)$, $\bv{\alpha }_1= ( 1, 1, 1, 1)$, $\bv{\alpha }_2= ( 1, 1, - 1, - 1)$, $\bv{\alpha }_3= ( 1, - 1, 1, - 1)$,$\bv{\alpha}_4=(1,-1,-1,1).$试将$\beta$表示成向量$\bv{\alpha_1},\bv{\alpha}_2,\bv{\alpha}_3,\bv{\alpha}_4$的线性组合.
\end{example}
\begin{solution}
\end{solution}
\begin{example}{}{}
    设$\bv{\alpha_1}=(3,-1,1),\bv{\alpha}_2=(1,1,2),\bv{\alpha}_3=(1,-3,-3),\bv{\alpha}_4=(4,0,5).$\\
(1)证明:$\alpha_1,\boldsymbol{\alpha}_2,\boldsymbol{\alpha}_3,\boldsymbol{\alpha}_4$线性相关;\\
(2) 证明:$\boldsymbol{\alpha}_1,\boldsymbol{\alpha}_2,\boldsymbol{\alpha}_4$ 线性无关.\end{exmaple}
\begin{solutiom}

\end{solutiom}
\begin{example}{}{}
    设向量组 $\alpha_{1}, \alpha_{2}, \alpha_{3}$ 线性无关. 证明: 向量组 $\alpha_{1}, \alpha_{1}+\alpha_{2}, \alpha_{1}+\alpha_{2}+\alpha_{3}$ 也线性无关.
\end{example}
\begin{solution}
\end{solution}
\begin{example}{}{}
    $\text{ 在 }\mathbb{R}^3\text{ 中},\text{ 求由基 }\bv{\alpha}_1=(1,2,-1),\bv{\alpha}_2=(1,-1,1),\bv{\alpha}_3=(-1,2,1)\text{到基 }\bv{\beta}_1=(2,0,1),\bv{\beta}_2=(0,1,1),\bv{\beta}_3=(1,-1,2)\text{的过渡矩阵}.$
\end{example}
\begin{solution}
\end{solution}
\begin{example}{}{}
    7.已知$\alpha_1=(1,0,0),\boldsymbol{\alpha}_2=(1,1,0),\boldsymbol{\alpha}_3=(1,1,1)$和$\beta_1=(1,2,3),\boldsymbol{\beta}_2=(2,3,1)$,
$\boldsymbol{\beta}_3=(3,1,2)$是$\mathbb{R}^3$的两组基.
(1)求从基$\alpha_1,\alpha_2,\alpha_3$到基$\beta_1,\beta_2,\beta_3$的过渡矩阵; (2)分别求向量$\alpha=(1,0,1)$在这两组基下的坐标.
\end{example}
\begin{solution}
\end{solution}