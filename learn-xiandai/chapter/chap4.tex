\chapter{线性方程组解的结构}
易错点:进行矩阵行变换时不能将上下两行同时加到中间的那一行上(通常是心血来潮),但是往往忽略了可能导致消掉了后面某一列的0,而前面的某一列出现非0的数字
\newpage
\begin{example}{}{}
    解线性方程组$\begin{cases}x_1+2x_2-5x_3+4x_4+x_5&=4,\\3x_1+7x_2-x_3-3x_4+2x_5&=10,\\-x_2-13x_3-2x_4+x_5&=-14,\\x_3-16x_4+2x_5&=-11,\\2x_4+5x_5&=12;\end{cases}$.
\end{example}
\begin{solution}
    构造增广矩阵
    \[\left(\begin{array}{cccccc}1&2&-5&4&1&4\\3&7&-1&-3&2&10\\0&-1&-13&-2&1&-14\\0&0&1&-16&2&-11\\0&0&0&2&5&12\end{array}\right)\rightarrow
    \left(\begin{array}{cccccc}1&2&-5&4&1&4\\0&1&14&-15&-1&-2\\0&-1&-13&-2&1&-14\\0&0&1&-16&2&-11\\0&0&0&2&5&12\end{array}\right)
    \]\[\rightarrow
    \left(\begin{array}{cccccc}1&2&-5&4&1&4\\0&1&14&-15&-1&-2\\0&0&1&-17&0&-16\\0&0&1&-16&2&-11\\0&0&0&2&5&12\end{array}\right)\rightarrow
     \left(\begin{array}{cccccc}1&2&-5&4&1&4\\0&1&14&-15&-1&-2\\0&0&1&-17&0&-16\\0&0&0&1&2&5\\0&0&0&2&5&12\end{array}\right)
    \]
    \[\rightarrow
    \left(\begin{array}{cccccc}1&2&-5&4&1&4\\0&1&14&-15&-1&-2\\0&0&1&-17&0&-16\\0&0&0&1&2&5\\0&0&0&0&1&2\end{array}\right)
    \Rightarrow \begin{cases}x_1=1\\x_2=1\\x_3=1\\x_4=1\\x_5=2\end{cases}
    \]
\end{solution}
\begin{example}{}{}
    设$\bv{\beta}=( 1, 2, 1, 1)$, $\bv{\alpha }_1= ( 1, 1, 1, 1)$, $\bv{\alpha }_2= ( 1, 1, - 1, - 1)$, $\bv{\alpha }_3= ( 1, - 1, 1, - 1)$,$\bv{\alpha}_4=(1,-1,-1,1).$试将$\bv{\beta}$表示成向量$\bv{\alpha_1},\bv{\alpha}_2,\bv{\alpha}_3,\bv{\alpha}_4$的线性组合.
\end{example}
\begin{solution}
    构造增广矩阵并进行行变换
    \[\left(\begin{array}{ccccc}1&1&1&1&1\\1&1&-1&-1&2\\1&-1&1&-1&1\\1&-1&-1&1&1\end{array}\right)\rightarrow
\left(\begin{array}{ccccc}1&1&1&1&1\\0&0&-2&-2&1\\0&-2&0&-2&0\\0&-2&-2&0&0\end{array}\right)\rightarrow
\left(\begin{array}{ccccc}1&1&1&1&1\\0&-2&0&-2&0\\0&0&-2&2&0\\0&0&0&-4&1\end{array}\right)
\]解得$k_1=\dfrac54,k_2=\dfrac14,k_3=-\dfrac14,k_4=-\dfrac14$,所以
\[\bv{\beta}=\dfrac54\bv{\alpha}_1+\dfrac14\bv{\alpha}_2-\dfrac14\bv{\alpha}_3-\dfrac14\bv{\alpha}_4.\]
\end{solution}
\begin{example}{}{}
    设$\bv{\alpha_1}=(3,-1,1),\bv{\alpha}_2=(1,1,2),\bv{\alpha}_3=(1,-3,-3),\bv{\alpha}_4=(4,0,5).$\\
(1)证明:$\bv{\alpha_1},\bv{\alpha}_2,\bv{\alpha}_3,\bv{\alpha}_4$线性相关;\\
(2) 证明:$\bv{\alpha}_1,\bv{\alpha}_2,\bv{\alpha}_4$ 线性无关.
\end{example}
\begin{solution}
(1)证明$k_1\bv{\alpha_1}+k_2\bv{\alpha}_2+k_3\bv{\alpha}_3+k_4\bv{\alpha}_4=\bv{0}$有非零解,系数矩阵$\bv{A}$与增广矩阵$\bv{\widetilde{A} }$:
\[\bv{A}=\left(\begin{array}{ccccc}3&1&1&4\\-1&1&-3&0\\1&2&-3&5\end{array}\right),\bv{\widetilde{A}}=\left(\begin{array}{ccccc}3&1&1&4&0\\-1&1&-3&0&0\\1&2&-3&5&0\end{array}\right)\]
取前三列写出行列式计算,易得$r(\bv{A})=r(\bv{\widetilde{A} })$所以有非零解,$\alpha_1,\alpha_2,\alpha_3,\alpha_4$线性相关.\\
(2)证明$\bv{\alpha}_1,\bv{\alpha}_2,\bv{\alpha}_4$线性无关,构造系数矩阵:
\[\left(\begin{array}{ccccc}3&1&4\\-1&1&0\\1&2&5\end{array}\right)\rightarrow \left(\begin{array}{ccccc}1&2&5\\-1&1&0\\3&1&4\end{array}\right)\rightarrow \left(\begin{array}{ccccc}1&2&5\\0&3&5\\0&-5&-11\end{array}\right)\]
计算行列式,得$|\bv{A}|\neq0$,所以$r(\bv{A})=3$,与列数相等,所以方程组无非零解,$\alpha_1,\alpha_2,\alpha_4$线性无关.
\end{solution}
\begin{example}{}{}
    设向量组 $\bv{\alpha_{1}, \alpha_{2}, \alpha_{3}}$ 线性无关. 证明: 向量组 $\bv{\alpha_{1}, \alpha_{1}+\alpha_{2}, \alpha_{1}+\alpha_{2}+\alpha_{3}}$ 也线性无关.
\end{example}
\begin{solution}
    即证明\[k_1\bv{\alpha_{1}}+k_2(\bv{\alpha_{1}+\alpha_{2}})+k_3(\bv{\alpha_{1}+\alpha_{2}+\alpha_{3}})=\bv{0}\]只有零解,即
    \[(k_1+k_2+k_3)\bv{\alpha_1}+(k_2+k_3)\bv{\alpha_2}+k_3\bv{\alpha_3}=\bv{0}\]
    由于向量组 $\bv{\alpha_{1}, \alpha_{2}, \alpha_{3}}$ 线性无关,所以
    \[\begin{cases}k_1+k_2+k_3=0\\k_2+k_3=0\\k_3=0\end{cases}\Leftrightarrow \begin{cases}k_1=0\\k_2=0\\k_3=0\end{cases}\]
    所以向量组 $\bv{\alpha_{1}, \alpha_{1}+\alpha_{2}, \alpha_{1}+\alpha_{2}+\alpha_{3}}$ 也线性无关.
\end{solution}
\begin{example}{}{}
    $\mathbb{R}^3$中,由基$\bv{\alpha}_1=(1,2,-1),\bv{\alpha}_2=(1,-1,1),\bv{\alpha}_3=(-1,2,1)$到基$\bv{\beta}_1=(2,0,1),\bv{\beta}_2=(0,1,1),\bv{\beta}_3=(1,-1,2)$的过渡矩阵?
\end{example}
\begin{solution}
设\[\bv{E}=\begin{pmatrix}1&0&0\\0&1&0\\0&0&1\end{pmatrix},\bv{A}=\begin{pmatrix}1&1&-1\\2&-1&2\\-1&1&1\end{pmatrix},\bv{B}=\begin{pmatrix}2&0&1\\0&1&-1\\1&1&2\end{pmatrix}\]则设过渡矩阵为$\bv{C},\bv{A}\bv{C}=\bv{B}$,则有$\bv{C}=\bv{A^{-1}}\bv{B}$
,所以由\[\bv{A^{-1}}(\bv{A}|\bv{B})=(\bv{A^{-1}}\bv{A}|\bv{A^{-1}}\bv{B})=(\bv{E}|\bv{A^{-1}}\bv{B})\]
且左乘一个可逆矩阵相当于作一系列初等行变换或列变换,所以我们可以写出矩阵$(\bv{A}|\bv{B})$,通过初等行变换得到$\bv{A^{-1}}\bv{B}$:
\[\left(\begin{array}{ccc|ccc}1&1&-1&2&0&1\\2&-1&2&0&1&-1\\-1&1&1&1&1&2\end{array}\right)\rightarrow \left(\begin{array}{ccc|ccc}1&1&-1&2&0&1\\0&-3&4&-4&1&-3\\-1&1&1&1&1&2\end{array}\right)\]\[
\rightarrow \left(\begin{array}{ccc|ccc}1&1&-1&2&0&1\\0&-3&4&-4&1&-3\\0&2&0&3&1&3\end{array}\right)\rightarrow
\left(\begin{array}{ccc|ccc}1&1&-1&2&0&1\\0&2&0&3&1&3\\0&-3&4&-4&1&-3\end{array}\right)
\]\[\rightarrow
\left(\begin{array}{ccc|ccc}1&0&-1&0.5&-0.5&-0.5\\0&1&0&1.5&0.5&1.5\\0&-3&4&-4&1&-3\end{array}\right)\rightarrow 
\left(\begin{array}{ccc|ccc}1&0&0&0.625&0.125&0.125\\0&1&0&1.5&0.5&1.5\\0&0&1&0.125&0.625&0.375\end{array}\right)
\]
过渡矩阵为:$\begin{pmatrix}\frac{5}{8}&\frac{1}{8}&\frac{1}{8}\\\frac{3}{2}&\frac{1}{2}&\frac{3}{2}\\\frac{1}{8}&\frac{5}{8}&\frac{3}{8}\end{pmatrix}$
\end{solution}
\begin{example}{}{}
    已知$\bv{\alpha_1}=(1,0,0),\bv{\alpha}_2=(1,1,0),\bv{\alpha}_3=(1,1,1)$和$\bv{\beta_1}=(1,2,3),\bv{\beta}_2=(2,3,1)$,
$\bv{\beta}_3=(3,1,2)$是$\mathbb{R}^3$的两组基.
(1)求从基$\bv{\alpha_1,\alpha_2,\alpha_3}$到基$\bv{\beta_1,\beta_2,\beta_3}$的过渡矩阵; (2)分别求向量$\bv{\alpha}=(1,0,1)$在这两组基下的坐标.
\end{example}
\begin{solution}
(1)设$\bv{E}=\begin{pmatrix}1&0&0\\0&1&0\\0&0&1\end{pmatrix},\bv{A}=\begin{pmatrix}1&1&1\\0&1&1\\0&0&1\end{pmatrix},\bv{B}=\begin{pmatrix}1&2&3\\2&3&1\\3&1&2\end{pmatrix}$,则有$\bv{A}\bv{C}=\bv{B}$,则有$\bv{C}=\bv{A^{-1}}\bv{B}$,所以由\[\bv{A^{-1}}(\bv{A}|\bv{B})=(\bv{A^{-1}}\bv{A}|\bv{A^{-1}}\bv{B})=(\bv{E}|\bv{A^{-1}}\bv{B})\]
且左乘一个可逆矩阵相当于作一系列初等行变换或列变换,所以我们可以写出矩阵$(\bv{A}|\bv{B})$,通过初等行变换得到$\bv{A^{-1}}\bv{B}$:
\[(\bv{A}|\bv{B})=\left(\begin{array}{ccc|ccc}1&1&1&1&2&3\\0&1&1&2&3&1\\0&0&1&3&1&2\end{array}\right)\rightarrow 
\left(\begin{array}{ccc|ccc}1&0&0&-1&-1&2\\0&1&0&-1&2&-1\\0&0&1&3&1&2\end{array}\right)\]
所以过渡矩阵为:$\bv{C}=\begin{pmatrix}-1&-1&2\\-1&2&-1\\3&1&2\end{pmatrix}$\\
(2)构造增广矩阵并进行行变换
\[\begin{pmatrix}1&1&1&1\\0&1&1&0\\0&0&1&1\end{pmatrix}\rightarrow \begin{pmatrix}1&0&0&1\\0&1&0&-1\\0&0&1&1\end{pmatrix}
\]所以向量$\bv{\alpha}=(1,0,1)$在基$\bv{\alpha_1}=(1,0,0),\bv{\alpha}_2=(1,1,0),\bv{\alpha}_3=(1,1,1)$的坐标为$(1,-1,1)$;\\
(3)构造增广矩阵并进行行变换
\[\begin{pmatrix}1&2&3&1\\2&3&1&0\\3&1&2&1\end{pmatrix}\rightarrow 
\begin{pmatrix}1&2&3&1\\0&-1&-5&-2\\3&1&2&1\end{pmatrix}\rightarrow 
\begin{pmatrix}1&2&3&1\\0&-1&-5&-2\\0&-5&-7&-2\end{pmatrix}\rightarrow
\begin{pmatrix}1&0&-7&-3\\0&1&5&2\\0&-5&-7&-2\end{pmatrix}\]
\[\rightarrow\begin{pmatrix}1&0&-7&-3\\0&1&5&2\\0&0&18&8\end{pmatrix}\rightarrow
\begin{pmatrix}1&0&-7&-3\\0&1&5&2\\0&0&1&\frac{4}{9}\end{pmatrix}\rightarrow
\begin{pmatrix}1&0&-7&-3\\0&1&0&-\frac{2}{9}\\0&0&1&\frac{4}{9}\end{pmatrix}
\rightarrow\begin{pmatrix}1&0&0&\frac{1}{9}\\0&1&0&-\frac{2}{9}\\0&0&1&\frac{4}{9}\end{pmatrix}\]
所以向量$\bv{\alpha}=(1,0,1)$在基$\bv{\beta_1}=(1,2,3),\bv{\beta}_2=(2,3,1)$,$\bv{\beta}_3=(3,1,2)$的坐标为
$\left(\frac{1}{9},-\frac{2}{9},\frac{4}{9}\right)$
\end{solution}
\begin{example}{}{}
    在$\mathbb{R}^3$中求向量$\alpha=(3,7,1)$在基$\bv{\alpha}_1=(1,3,5),\bv{\alpha}_2=(6,3,2),\bv{\alpha}_3=(3,1,0)$下的坐标
\end{example}
\begin{solution}
    构造矩阵进行初等行变换:
    \[\begin{pmatrix}1&6&3&3\\3&3&1&7\\5&2&0&1\end{pmatrix}\rightarrow\begin{pmatrix}1&6&3&3\\0&-15&-8&-2\\0&-28&-15&-14\end{pmatrix}\rightarrow\begin{pmatrix}1&6&3&3\\0&-15\times 28&-8\times 28&-2\times 28\\0&-28\times 15&-15\times 15&-14\times 15\end{pmatrix}\]\[
    \rightarrow\begin{pmatrix}1&6&3&3\\0&-15\times 28&-8\times 28&-2\times 28\\0&0&-1&-154\end{pmatrix}\rightarrow\begin{pmatrix}1&6&3&3\\0&-15&-8&-2\\0&0&-1&-154\end{pmatrix}\rightarrow\begin{pmatrix}1&6&3&3\\0&-15&0&1230\\0&0&1&154\end{pmatrix}
    \]\[
    \rightarrow\begin{pmatrix}1&6&0&-459\\0&-15&0&1230\\0&0&1&154\end{pmatrix}\rightarrow\begin{pmatrix}1&6&0&-459\\0&1&0&-82\\0&0&1&154\end{pmatrix}\rightarrow\begin{pmatrix}1&0&0&33\\0&1&0&-82\\0&0&1&154\end{pmatrix}\]
    坐标是$(33,-82,154)$
\end{solution}
\begin{example}{}{}
    在$\mathbb{R}^4$中找一个向量$\gamma$,使它在标准基$\bv{\varepsilon_1,\varepsilon_2,\varepsilon_3,\varepsilon_4}$和基$\bv{\beta}_1=(2,1,-1,1),\bv{\beta}_2=(0,3,1,0),\bv{\beta }_3= ( 5, 3, 2, 1) , \bv{\beta }_4= ( 6, 6, 1, 3)$下有相同的坐标.
\end{example}
\begin{solution}
    设$\gamma=(k_1,k_2,k_3,k_4)$则
    \[\begin{cases}2k_1+5k_3+6k_4=k_1\\k_1+3k_2+3k_3+6k_4=k_2\\-k_1+k_2+2k_3+k_4=k_3\\k_1+k_3+3k_4=k_4\end{cases}\Leftrightarrow \begin{cases}k_1+5k_3+6k_4=0\\k_1+2k_2+3k_3+6k_4=0\\-k_1+k_2+k_3+k_4=0\\k_1+k_3+2k_4=0\end{cases}\]
    构建增广矩阵\[\begin{pmatrix}1&0&5&6&0\\1&2&3&6&0\\-1&1&1&1&0\\1&0&1&2&0\end{pmatrix}
    \rightarrow\begin{pmatrix}1&0&5&6&0\\0&2&-2&0&0\\0&1&6&7&0\\0&0&-4&-4&0\end{pmatrix}
    \rightarrow\begin{pmatrix}1&0&5&6&0\\0&1&-1&0&0\\0&1&6&7&0\\0&0&-4&-4&0\end{pmatrix}\]
    \[\rightarrow \begin{pmatrix}1&0&5&6&0\\0&1&-1&0&0\\0&0&7&7&0\\0&0&-4&-4&0\end{pmatrix}\rightarrow
    \begin{pmatrix}1&0&5&6&0\\0&1&-1&0&0\\0&0&1&1&0\\0&0&0&0&0\end{pmatrix}\rightarrow
    \begin{pmatrix}1&0&0&1&0\\0&1&0&1&0\\0&0&1&1&0\\0&0&0&0&0\end{pmatrix}\]
    因为只需要找到一个向量,所以取$k_4=-1$,则$\gamma=(1,1,1,-1)$.
\end{solution}
\newpage
\begin{example}{}{}
    设向量组$\bv{\xi_1}=(1,-1,2,4),\bv{\xi_2}=(0,3,1,2),\bv{\xi_3}=(3,0,7,14),\bv{\xi_4}=(1,-1,2,0),\bv{\xi_5}=(2,1,5,6)$\\
    (1)证明$\bv{\xi_1},\bv{\xi_2}$线性无关.(2)求向量组中包含$\bv{\xi_1},\bv{\xi_2}$的极大线性无关组.
\end{example}
\begin{solution}
    (1)设$k_1\bv{\xi_1}+k_2\bv{\xi_2}=\bm{0}$,转化为方程组:\[
    \begin{cases}k_1+0k_2=0\\-k_1+3k_2=0\\2k_1+k_2=0\\4k_1+2k_2=0\end{cases},~~\text{解得}k_1=0,k_2=0
    \],因此$\bv{\xi_1},\bv{\xi_2}$线性无关.\\
    (2)将向量组构成矩阵 $A$,其中列向量为 $\bv{\xi_1}, \bv{\xi_2}, \bv{\xi_3}, \bv{\xi_4}, \bv{\xi_5}$,对 $A$ 进行行化简:
\[
\begin{aligned}
& \begin{pmatrix}
1 & 0 & 3 & 1 & 2 \\
-1 & 3 & 0 & -1 & 1 \\
2 & 1 & 7 & 2 & 5 \\
4 & 2 & 14 & 0 & 6
\end{pmatrix}
\xrightarrow[R_3 = R_3 - 2R_1]{R_2 = R_2 + R_1, \, R_4 = R_4 - 4R_1}
\begin{pmatrix}
1 & 0 & 3 & 1 & 2 \\
0 & 3 & 3 & 0 & 3 \\
0 & 1 & 1 & 0 & 1 \\
0 & 2 & 2 & -4 & -2
\end{pmatrix}
\xrightarrow{R_2 = \frac{1}{3} R_2}
\begin{pmatrix}
1 & 0 & 3 & 1 & 2 \\
0 & 1 & 1 & 0 & 1 \\
0 & 1 & 1 & 0 & 1 \\
0 & 2 & 2 & -4 & -2
\end{pmatrix} \\
& \xrightarrow[R_4 = R_4 - 2R_2]{R_3 = R_3 - R_2}
\begin{pmatrix}
1 & 0 & 3 & 1 & 2 \\
0 & 1 & 1 & 0 & 1 \\
0 & 0 & 0 & 0 & 0 \\
0 & 0 & 0 & -4 & -4
\end{pmatrix}
\xrightarrow{\text{交换 } R_3, R_4}
\begin{pmatrix}
1 & 0 & 3 & 1 & 2 \\
0 & 1 & 1 & 0 & 1 \\
0 & 0 & 0 & -4 & -4 \\
0 & 0 & 0 & 0 & 0
\end{pmatrix}
\xrightarrow{R_3 = -\frac{1}{4} R_3}
\begin{pmatrix}
1 & 0 & 3 & 1 & 2 \\
0 & 1 & 1 & 0 & 1 \\
0 & 0 & 0 & 1 & 1 \\
0 & 0 & 0 & 0 & 0
\end{pmatrix} 
\end{aligned}
\]
行简化阶梯形的主元列在第 1、2、4 列,对应向量 $\bv{\xi_1}, \bv{\xi_2}, \bv{\xi_4}$。向量$\bm{\xi_3}$可以由$\bv{\xi_1}$ 和 $\bv{\xi_2}$线性表示,$\bm{\xi_5}$可以由$\bv{\xi_1}, \bv{\xi_2}, \bv{\xi_4}$线性表示;因此,包含 $\bv{\xi_1}$ 和 $\bv{\xi_2}$ 的极大线性无关组为 $\bv{\xi_1}, \bv{\xi_2}, \bv{\xi_4}$.
\end{solution}
\newpage
\begin{example}{}{}
    设$\bm{\alpha_1}=(2,1,2,2,-4),\bm{\alpha_2}=(1,1,-1,0,2),\bm{\alpha_3}=(0,1,2,1,-1)$,\\$\bm{\alpha_4}=(-1,-1,-1,-1,1),\bm{\alpha_5}=(1,2,1,1,1)$,确定向量组$\bm{\alpha_1},\bm{\alpha_2},\bm{\alpha_3},\bm{\alpha_4},\bm{\alpha_5}$极大线性无关组.
\end{example}
\begin{solution}
    将向量组构成矩阵 $A$,其中列向量为 $\bv{\alpha_1}, \bv{\alpha_2}, \bv{\alpha_3}, \bv{\alpha_4}, \bv{\alpha_5}$,对 $A$ 进行行简化:
\[
\begin{aligned}
& A = \begin{pmatrix}
2 & 1 & 0 & -1 & 1 \\
1 & 1 & 1 & -1 & 2 \\
2 & -1 & 2 & -1 & 1 \\
2 & 0 & 1 & -1 & 1 \\
-4 & 2 & -1 & 1 & 1
\end{pmatrix}
\xrightarrow{\text{交换 } R_1, R_2}
\begin{pmatrix}
1 & 1 & 1 & -1 & 2 \\
2 & 1 & 0 & -1 & 1 \\
2 & -1 & 2 & -1 & 1 \\
2 & 0 & 1 & -1 & 1 \\
-4 & 2 & -1 & 1 & 1
\end{pmatrix} \\
& \xrightarrow[R_3 = R_3 - 2R_1]{R_2 = R_2 - 2R_1, \, R_4 = R_4 - 2R_1, \, R_5 = R_5 + 4R_1}
\begin{pmatrix}
1 & 1 & 1 & -1 & 2 \\
0 & -1 & -2 & 1 & -3 \\
0 & -3 & 0 & 1 & -3 \\
0 & -2 & -1 & 1 & -3 \\
0 & 6 & 3 & -3 & 9
\end{pmatrix}
\xrightarrow{R_2 = -R_2}
\begin{pmatrix}
1 & 1 & 1 & -1 & 2 \\
0 & 1 & 2 & -1 & 3 \\
0 & -3 & 0 & 1 & -3 \\
0 & -2 & -1 & 1 & -3 \\
0 & 6 & 3 & -3 & 9
\end{pmatrix} \\
& \xrightarrow[R_3 = R_3 + 3R_2]{R_1 = R_1 - R_2, \, R_4 = R_4 + 2R_2, \, R_5 = R_5 - 6R_2}
\begin{pmatrix}
1 & 0 & -1 & 0 & -1 \\
0 & 1 & 2 & -1 & 3 \\
0 & 0 & 6 & -2 & 6 \\
0 & 0 & 3 & -1 & 3 \\
0 & 0 & -9 & 3 & -9
\end{pmatrix}
\xrightarrow{R_3 = R_3 / 2}
\begin{pmatrix}
1 & 0 & -1 & 0 & -1 \\
0 & 1 & 2 & -1 & 3 \\
0 & 0 & 3 & -1 & 3 \\
0 & 0 & 3 & -1 & 3 \\
0 & 0 & -9 & 3 & -9
\end{pmatrix} \\
& \xrightarrow[R_4 = R_4 - R_3]{R_5 = R_5 + 3R_3}
\begin{pmatrix}
1 & 0 & -1 & 0 & -1 \\
0 & 1 & 2 & -1 & 3 \\
0 & 0 & 3 & -1 & 3 \\
0 & 0 & 0 & 0 & 0 \\
0 & 0 & 0 & 0 & 0
\end{pmatrix}
\xrightarrow{R_3 = R_3 / 3}
\begin{pmatrix}
1 & 0 & -1 & 0 & -1 \\
0 & 1 & 2 & -1 & 3 \\
0 & 0 & 1 & -1/3 & 1 \\
0 & 0 & 0 & 0 & 0 \\
0 & 0 & 0 & 0 & 0
\end{pmatrix} \\
& \xrightarrow[R_2 = R_2 - 2R_3]{R_1 = R_1 + R_3}
\begin{pmatrix}
1 & 0 & 0 & -1/3 & 0 \\
0 & 1 & 0 & -1/3 & 1 \\
0 & 0 & 1 & -1/3 & 1 \\
0 & 0 & 0 & 0 & 0 \\
0 & 0 & 0 & 0 & 0
\end{pmatrix}
\end{aligned}
\]行最简形式的主元列在第 1、2、3 列,对应向量 $\bv{\alpha_1}, \bv{\alpha_2}, \bv{\alpha_3}$。因此,向量组的秩为 3,且 $\bv{\alpha_1}, \bv{\alpha_2}, \bv{\alpha_3}$ 线性无关。其余向量可线性表示:$\bv{\alpha_4} = -\frac{1}{3} \bv{\alpha_1} - \frac{1}{3} \bv{\alpha_2} - \frac{1}{3} \bv{\alpha_3},
\bv{\alpha_5} = \bv{\alpha_2} + \bv{\alpha_3}$
故向量组的极大线性无关组为 $\bv{\alpha_1}, \bv{\alpha_2}, \bv{\alpha_3}$
\end{solution}
\begin{example}{}{}
    设$\bm{A}$为$n\times n$矩阵,且$\bm{A}^2=\bm{A}$,证明:$r(\bm{A})+r(\bm{A-E})=n$
\end{example}
\begin{solution}
假设$\bm{A}\bm{B}$为$n\times n$矩阵,$\bm{AB=O}$,首先,方程组$\bm{Ax=0}$的解向量组$\bm{x}$(显然都是列向量)的极大线性无关组的向量个数为原方程组自由变量的个数,即$n-r(\bm{A})$,而且矩阵$\bm{B}$恰好提供了一组列向量,满足要求,但是由于这组列向量一定可以被$\bm{x}$的极大线性无关组一个个地线性表示,所以说这组列向量的极大线性无关组的向量个数不一定等于(小于等于)解向量组$\bm{x}$(显然都是列向量)的极大线性无关组的向量个数;所以$r(\bm{A})+r(\bm{B})\leq n$\\
而且,我们知道,如果将$\bm{A+B,A,B}$分别视为向量组:
\[\begin{cases}\bm{A+B}=(\bm{\alpha_1+\beta_1,\alpha_2+\beta_2,\cdots,\alpha_n+\beta_n})\\\bm{A}=(\bm{\alpha_1,\alpha_2,\cdots,\alpha_n}),\bm{B}=(\bm{\beta_1,\beta_2,\cdots,\beta_n})\end{cases}\]
分别列出它们的极大线性无关组(等价于它们自身):
\[\begin{cases}\bm{A+B}\Leftrightarrow\bm{M_{12}}=(\bm{\alpha_{i_1}+\beta_{i_1},\bm{\alpha_{i_2}}+\beta_{i_2},\cdots,\bm{\alpha_{i_r}}+\beta_{i_r}})\\
\bm{A}\Leftrightarrow\bm{M_1}=(\bm{\alpha_{i_1},\bm{\alpha_{i_2}},\cdots,\bm{\alpha_{i_s}}})\\
\bm{B}\Leftrightarrow\bm{M_2}=(\bm{\beta_{i_1},\bm{\beta_{i_2}},\cdots,\bm{\beta_{i_t}}})
\end{cases}\]
则显然$\bm{M_{12}}$可以由$\bm{M_1,M_2}$线性表示,则$r(\bm{A})+r(\bm{B})=s+t\geq r=r(\bm{A}+\bm{B})$\\
对于本题,由于$r(\bm{E})=n=r(\bm{A}+\bm{E-A})\leq r(\bm{A})+r(\bm{A-E})$,且$\bm{A}(\bm{A}-\bm{E})=\bm{O}\Rightarrow r(\bm{A})+r(\bm{A-E})\leq n$,所以$r(\bm{A})+r(\bm{A-E})=n$成立。
\end{solution}
\begin{example}{}{}
求下列齐次线性方程组的一个基础解系, 并写出通解:
$\left\{
\begin{array}{l}
x_{1}+2 x_{2}+x_{3}-x_{4}=0, \\
3 x_{1}+6 x_{2}-x_{3}-3 x_{4}=0, \\
5 x_{1}+10 x_{2}+x_{3}-5 x_{4}=0 ;
\end{array}
\right.$
\end{example}
\begin{solution}
     写出系数矩阵并进行行变换:
\[\begin{pmatrix}
1 & 2 & 1 & -1 \\
3 & 6 & -1 & -3 \\
5 & 10 & 1 & -5
\end{pmatrix}\xrightarrow{R_2 \rightarrow R_2 - 3R_1}
\begin{pmatrix}
1 & 2 & 1 & -1 \\
0 & 0 & -4 & 0 \\
0 & 0 & -4 & 0
\end{pmatrix}
\xrightarrow{R_3 \rightarrow R_3 - R_2} 
\begin{pmatrix}
1 & 2 & 1 & -1 \\
0 & 0 & -4 & 0 \\
0 & 0 & 0 & 0
\end{pmatrix} 
\]
\[
\xrightarrow{R_2 \rightarrow -\frac{1}{4}R_2} 
\begin{pmatrix}
1 & 2 & 1 & -1 \\
0 & 0 & 1 & 0 \\
0 & 0 & 0 & 0
\end{pmatrix} \xrightarrow{R_1 \rightarrow R_1 - R_2} 
\begin{pmatrix}
1 & 2 & 0 & -1 \\
0 & 0 & 1 & 0 \\
0 & 0 & 0 & 0
\end{pmatrix}
\]
得到$x_3=0,x_1+2x_2-x_4=0$,将$x_2=s,x_4=t$看作自由未知量,则解向量为
\[\mathbf{x}=\begin{pmatrix}x_1\\x_2\\x_3\\x_4\end{pmatrix}=\begin{pmatrix}-2s+t\\s\\0\\t\end{pmatrix}=s\begin{pmatrix}-2\\1\\0\\0\end{pmatrix}+t\begin{pmatrix}1\\0\\0\\1\end{pmatrix}\Leftrightarrow \mathbf{x}=s\xi_1+t\xi_2,\quad s,t\in\mathbb{R}\]
其中$\xi_1,\xi_2$为基础解系:\[\xi_{1}=\begin{pmatrix}-2\\1\\0\\0\end{pmatrix},\quad\xi_{2}=\begin{pmatrix}1\\0\\0\\1\end{pmatrix}\]
\end{solution}
\begin{example}{}{}
设线性方程组为\[\begin{cases}2x_1-x_2+3x_3+2x_4=0,\\9x_1-x_2+14x_3+2x_4=1,\\3x_1+2x_2+5x_3-4x_4=1,\\4x_1+5x_2+7x_3-10x_4=2.\end{cases}\]
(1)求方程组导出组的一个基础解系(2)用特解和导出组的基础解系表示方程组的所有解.
\end{example}
\begin{solution}
(1)导出组为齐次线性方程组,以及系数矩阵:\[\begin{cases}
2x_1 - x_2 + 3x_3 + 2x_4 = 0 \\
9x_1 - x_2 + 14x_3 + 2x_4 = 0 \\
3x_1 + 2x_2 + 5x_3 - 4x_4 = 0 \\
4x_1 + 5x_2 + 7x_3 - 10x_4 = 0
\end{cases},
\bm{A} = \begin{pmatrix}
2 & -1 & 3 & 2 \\
9 & -1 & 14 & 2 \\
3 & 2 & 5 & -4 \\
4 & 5 & 7 & -10
\end{pmatrix}
\]
对系数矩阵进行行变换,得到行最简形式:
\[\bm{A}\xrightarrow{R_2 \rightarrow R_2 - 4R_1} 
\begin{pmatrix}
2 & -1 & 3 & 2 \\
1 & 3 & 2 & -6 \\
3 & 2 & 5 & -4 \\
4 & 5 & 7 & -10
\end{pmatrix}\xrightarrow{R_1 \leftrightarrow R_2} 
\begin{pmatrix}
1 & 3 & 2 & -6 \\
2 & -1 & 3 & 2 \\
3 & 2 & 5 & -4 \\
4 & 5 & 7 & -10
\end{pmatrix}
\rightarrow
\begin{pmatrix}
1 & 3 & 2 & -6 \\
0 & -7 & -1 & 14 \\
0 & -7 & -1 & 14 \\
0 & -7 & -1 & 14
\end{pmatrix}
\]\[
\xrightarrow{\text{保留前两行}} 
\begin{pmatrix}
1 & 3 & 2 & -6 \\
0 & -7 & -1 & 14
\end{pmatrix} 
 \xrightarrow{R_2 \rightarrow -\frac{1}{7}R_2} 
\begin{pmatrix}
1 & 3 & 2 & -6 \\
0 & 1 & \frac{1}{7} & -2
\end{pmatrix}  \xrightarrow{R_1 \rightarrow R_1 - 3R_2} 
\begin{pmatrix}
1 & 0 & \frac{11}{7} & 0 \\
0 & 1 & \frac{1}{7} & -2
\end{pmatrix}
\]自由变量 \(x_3 = 7s\), \(x_4 = t\)(其中 \(s, t \in \mathbb{R}\)),则:
\[x_1 = -11s, \quad x_2 = -s + 2t, \quad x_3 = 7s, \quad x_4 = t\]
解向量和导出组的基础解系为:
\[\mathbf{x} = s \begin{pmatrix} -11 \\ -1 \\ 7 \\ 0 \end{pmatrix} + t \begin{pmatrix} 0 \\ 2 \\ 0 \\ 1 \end{pmatrix},\xi_1 = \begin{pmatrix} -11 \\ -1 \\ 7 \\ 0 \end{pmatrix}, \quad \xi_2 = \begin{pmatrix} 0 \\ 2 \\ 0 \\ 1 \end{pmatrix}\]
(2)原方程组的增广矩阵为:
\[
\left(\begin{array}{cccc|c}
2 & -1 & 3 & 2 & 0 \\
9 & -1 & 14 & 2 & 1 \\
3 & 2 & 5 & -4 & 1 \\
4 & 5 & 7 & -10 & 2
\end{array}\right)\xrightarrow{R_2 \rightarrow R_2 - 4R_1} 
\left(\begin{array}{cccc|c}
2 & -1 & 3 & 2 & 0 \\
1 & 3 & 2 & -6 & 1 \\
3 & 2 & 5 & -4 & 1 \\
4 & 5 & 7 & -10 & 2
\end{array}\right)\xrightarrow{R_1 \leftrightarrow R_2} 
\left(\begin{array}{cccc|c}
1 & 3 & 2 & -6 & 1 \\
2 & -1 & 3 & 2 & 0 \\
3 & 2 & 5 & -4 & 1 \\
4 & 5 & 7 & -10 & 2
\end{array}\right)\]
\[\rightarrow\left(\begin{array}{cccc|c}
1 & 3 & 2 & -6 & 1 \\
0 & -7 & -1 & 14 & -2 \\
0 & -7 & -1 & 14 & -2 \\
0 & -7 & -1 & 14 & -2
\end{array}\right)
\xrightarrow{\text{保留前两行}} 
\left(\begin{array}{cccc|c}
1 & 3 & 2 & -6 & 1 \\
0 & -7 & -1 & 14 & -2
\end{array}\right) \\
\xrightarrow{R_2 \rightarrow -\frac{1}{7}R_2} 
\left(\begin{array}{cccc|c}
1 & 3 & 2 & -6 & 1 \\
0 & 1 & \frac{1}{7} & -2 & \frac{2}{7}
\end{array}\right) \]
\[
\xrightarrow{R_1 \rightarrow R_1 - 3R_2} 
\left(\begin{array}{cccc|c}
1 & 0 & \frac{11}{7} & 0 & \frac{1}{7} \\
0 & 1 & \frac{1}{7} & -2 & \frac{2}{7}
\end{array}\right)
\]令自由变量 \(x_3 = 7s\), \(x_4 = t\)(其中 \(s, t \in \mathbb{R}\)),则:
\[x_1 = \frac{1}{7} - 11s, \quad x_2 = \frac{2}{7} - s + 2t, \quad x_3 = 7s, \quad x_4 = t\]
特解为当 \(s = 0\), \(t = 0\) 时:
\[\eta = \begin{pmatrix} \frac{1}{7} \\ \frac{2}{7} \\ 0 \\ 0 \end{pmatrix}\]
所有解可表示为特解加上导出组基础解系的线性组合:\[
\mathbf{X} = \eta + c_1 \xi_1 + c_2 \xi_2 = \begin{pmatrix} \frac{1}{7} \\ \frac{2}{7} \\ 0 \\ 0 \end{pmatrix} + c_1 \begin{pmatrix} -11 \\ -1 \\ 7 \\ 0 \end{pmatrix} + c_2 \begin{pmatrix} 0 \\ 2 \\ 0 \\ 1 \end{pmatrix}, \quad c_1, c_2 \in \mathbb{R}
\]
\end{solution}
\begin{example}{}{}
    已知线性方程组:\[
    \begin{cases}x_1+x_2-2x_3+3x_4=0\\
    2x_1+x_2-6x_3+4x_4=-1\\
3x_1+2x_2+px_3+7x_4=-1\\
x_1-x_2-6x_3-x_4=t\end{cases}\]
讨论参数$s,t$取何值时,方程组有解,无解;有解时,试用导出组的基础解系表示通解.
\end{example}
\begin{solution}
    方程组的增广矩阵为:
    \[\left(\begin{array}{cccc|c}
    1 & 1 & -2 & 3 & 0 \\
    2 & 1 & -6 & 4 & -1 \\
    3 & 2 & p & 7 & -1 \\
    1 & -1 & -6 & -1 & t
    \end{array}\right)\xrightarrow[R_3 \rightarrow R_3 - 3R_1]{R_2 \rightarrow R_2 - 2R_1} 
\left(\begin{array}{cccc|c}
1 & 1 & -2 & 3 & 0 \\
0 & -1 & -2 & -2 & -1 \\
0 & -1 & p+6 & -2 & -1 \\
0 & -2 & -4 & -4 & t
\end{array}\right) 
\]\[
\xrightarrow{R_2 \rightarrow -R_2} 
\left(\begin{array}{cccc|c}
1 & 1 & -2 & 3 & 0 \\
0 & 1 & 2 & 2 & 1 \\
0 & -1 & p+6 & -2 & -1 \\
0 & -2 & -4 & -4 & t
\end{array}\right) 
\xrightarrow[R_4 \rightarrow R_4 + 2R_2]{R_1 \rightarrow R_1 - R_2,\; R_3 \rightarrow R_3 + R_2} 
\left(\begin{array}{cccc|c}
1 & 0 & -4 & 1 & -1 \\
0 & 1 & 2 & 2 & 1 \\
0 & 0 & p+8 & 0 & 0 \\
0 & 0 & 0 & 0 & t+2
\end{array}\right)\]
第四行对应方程 $0 = t+2$,因此当 $t \neq -2$ 时,方程组无解。当 $t = -2$ 时,方程组有解,且解的情况取决于参数 $p$.
情况1: $p \neq -8$,此时增广矩阵的秩为3,系数矩阵的秩也为3,自由变量个数为1。导出组的基础解系含一个向量。
从行最简形式:
\[
\left(\begin{array}{cccc|c}
1 & 0 & -4 & 1 & -1 \\
0 & 1 & 2 & 2 & 1 \\
0 & 0 & p+8 & 0 & 0 \\
0 & 0 & 0 & 0 & 0
\end{array}\right)
\]
由第三行得 $(p+8)x_3 = 0$,因 $p \neq -8$,故 $x_3 = 0$,进而解得:$x_1 = -1 - x_4,x_2 = 1 - 2x_4$
特解和导出组的基础解系为分别为:
\[\eta = \begin{pmatrix} -1 \\ 1 \\ 0 \\ 0 \end{pmatrix}\quad \xi = \begin{pmatrix} -1 \\ -2 \\ 0 \\ 1 \end{pmatrix}\]
通解为:
\[
\mathbf{x} = \eta + c \xi = \begin{pmatrix} -1 \\ 1 \\ 0 \\ 0 \end{pmatrix} + c \begin{pmatrix} -1 \\ -2 \\ 0 \\ 1 \end{pmatrix}, \quad c \in \mathbb{R}
\]
情况2: $p = -8$;此时增广矩阵的秩为2,系数矩阵的秩也为2,自由变量个数为2。导出组的基础解系含两个向量。从行最简形式:
\[
\left(\begin{array}{cccc|c}
1 & 0 & -4 & 1 & -1 \\
0 & 1 & 2 & 2 & 1 \\
0 & 0 & 0 & 0 & 0 \\
0 & 0 & 0 & 0 & 0
\end{array}\right)
\]
解得:$x_1 = -1 + 4x_3 - x_4, \quad x_2 = 1 - 2x_3 - 2x_4$
特解和导出组的基础解系为分别为:\[
\eta = \begin{pmatrix} -1 \\ 1 \\ 0 \\ 0 \end{pmatrix}\quad \xi_1 = \begin{pmatrix} 4 \\ -2 \\ 1 \\ 0 \end{pmatrix}, \quad \xi_2 = \begin{pmatrix} -1 \\ -2 \\ 0 \\ 1 \end{pmatrix}\]
通解为:
\[
\mathbf{x} = \eta + c_1 \xi_1 + c_2 \xi_2 = \begin{pmatrix} -1 \\ 1 \\ 0 \\ 0 \end{pmatrix} + c_1 \begin{pmatrix} 4 \\ -2 \\ 1 \\ 0 \end{pmatrix} + c_2 \begin{pmatrix} -1 \\ -2 \\ 0 \\ 1 \end{pmatrix}, \quad c_1, c_2 \in \mathbb{R}
\]
\end{solution}
\begin{example}{}{}
    设$\bm{A}$为$n$阶方阵,那么\[r(\bm{A^*})=\begin{cases}n,r(\bm{A})=n\\1,r(\bm{A})=n-1\\0,r(\bm{A})<n-1\end{cases}\]
\end{example}
\begin{solution}
(1)若$r(\bm{A})=n$,则$\bm{A}$可逆,所以由于$\bm{A^{-1}}=\dfrac1{|\bm{A}|}\bm{A^*}\Rightarrow \det{(\bm{A}^*)}=\det{(|\bm{A}|A^{-1})}\ne 0$,所以$\bm{A^*}$可逆,是满秩的;\\
(2)此时尽管$|\bm{A}|=0$,但是仍然有$\bm{A}\bm{A^*}=|\bm{A}|\bm{E}\Rightarrow \bm{A}\bm{A^*}=\bm{O}$,所以将$\bm{A}$视作方程组$\bm{A}\bm{x}=\bm{O}$的系数矩阵,由于$r(\bm{A})=n-1$,所以方程组自由未知量个数为$1$,所以解向量$\bm{x}$的极大线性无关组的向量个数就是$1$,而$\bm{A^*}$所提供的列向量都可以被$\bm{x}$的极大线性无关组线性表示,所以$r(\bm{A^*})\leq 1$,但是又由于矩阵$\bm{A}$存在$n-1$阶不为0的行列式,所以$r(\bm{A^*})=1$\\
(3)若$r(\bm{A})=n-2$,不存在$n-1$阶不为0的行列式,所以由(2)得到$r(\bm{A^*})=0$
\end{solution}