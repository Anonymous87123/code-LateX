\chapter{矩阵}
\newpage
\section{第1周作业}
\begin{example}{}{}
    求$\left( \begin{matrix} \cos \alpha&- \sin \alpha \\ \sin \alpha& \cos \alpha \end{matrix} \right)^{n}$
\end{example}
\begin{solution}
    因为\begin{align*}&\left( \begin{matrix} \cos n\alpha&- \sin n\alpha \\ \sin n\alpha& \cos n\alpha \end{matrix} \right)\left( \begin{matrix} \cos \alpha&- \sin \alpha \\ \sin \alpha& \cos \alpha \end{matrix} \right)\\&=\left( \begin{matrix} \cos\alpha\cos n\alpha-\sin \alpha\sin n\alpha &-\sin\alpha\cos n\alpha-\cos\alpha\sin n\alpha \\ \sin\alpha\cos n\alpha+\cos\alpha\sin n\alpha& \cos\alpha\cos n\alpha-\sin \alpha\sin n\alpha \end{matrix} \right)\\&=\left( \begin{matrix} \cos (n+1)\alpha&- \sin (n+1)\alpha \\ \sin (n+1)\alpha& \cos (n+1)\alpha \end{matrix} \right)\end{align*}所以答案是$\left( \begin{matrix} \cos n\alpha&- \sin n\alpha \\ \sin n\alpha& \cos n\alpha \end{matrix} \right)$
\end{solution}
\begin{example}{}{}
    $\begin{aligned}&\text{令}A=\begin{pmatrix}0&1&0\\0&0&1\\-2&0&1\end{pmatrix},\text{计算}\\&(1)A^2,A^3\text{ 和 }f(A),\text{ 其中 }f(x)=x^3-3x^2-2x+2;\end{aligned}$
\end{example}
\begin{solution}
    $A^2=\begin{pmatrix}0&0&1\\-2&0&1\\-2&-2&1\end{pmatrix},A^3=\begin{pmatrix}-2&0&1\\-2&-2&1\\-2&-2&-1\end{pmatrix},2=2I=\begin{pmatrix}2&0&0\\0&2&0\\0&0&2\end{pmatrix}$\\
    $f(A)=\begin{pmatrix}-2&0&1\\-2&-2&1\\-2&-2&-1\end{pmatrix}-3\begin{pmatrix}0&0&1\\-2&0&1\\-2&-2&1\end{pmatrix}-2\begin{pmatrix}0&1&0\\0&0&1\\-2&0&1\end{pmatrix}+\begin{pmatrix}2&0&0\\0&2&0\\0&0&2\end{pmatrix}=\begin{pmatrix}0&-2&-2\\4&0&-4\\8&4&-4\end{pmatrix}$
\end{solution}
\begin{example}{}{}
    求与$\begin{pmatrix}3&1\\-2&2\end{pmatrix}$可交换的所有矩阵.
\end{example}
\begin{solution}
    设所求矩阵为$\begin{pmatrix}a&b\\c&d\end{pmatrix}$,则
    \begin{align*}
    &\begin{pmatrix}a&b\\c&d\end{pmatrix}\begin{pmatrix}3&1\\-2&2\end{pmatrix}=\begin{pmatrix}3&1\\-2&2\end{pmatrix}\begin{pmatrix}a&b\\c&d\end{pmatrix}\\
    &\Rightarrow \begin{pmatrix}3a-2b&a+2b\\3c-2d&c+2d\end{pmatrix}=\begin{pmatrix}3a+c&3b+d\\-2a+2c&-2b+2d\end{pmatrix}\Rightarrow 
    \begin{cases}3a-2b=3a+c\\a+2b=-2a+2c\\3c-2d=3b+d\\c+2d=-2b+2d\end{cases}\\
    &\Rightarrow \begin{pmatrix}a&b\\-2b&a-b\end{pmatrix},a,b\in \mathbb{C}\end{align*}\end{solution}
\begin{example}{}{}
    证明:与任意$n$阶矩阵都可交换的矩阵只能是数量矩阵,即$\bv{A}=k\bv{E}$
\end{example}
\begin{solution}
    考虑一个特殊的矩阵$\bv{E_{ij}}$,这个矩阵的第$i$行$j$列的元素是$1$,其余元素全为$0$。并设矩阵$\bv{A}$与任意$n$阶矩阵都可交换,那么它显然为$n$阶矩阵。那么$n$阶矩阵$\bv{A}\bv{E_{ij}}$ 满足第$j$列的元素为$a_{i1},a_{i2},...,a_{in}$,其余元素全为0,而$n$阶矩阵$\bv{E_{ij}}\bv{A}$满足第$i$行的元素为$a_{1j},a_{2j},...,a_{nj}$,其余元素全为0。因此由于$\bv{E_{ij}}\bv{A}=\bv{A}\bv{E_{ij}}$,那么对于任意$i\ne j,a_{ij}=0$,所以矩阵$\bv{A}$只有对角线上的元素不是0,其余元素全为0。然后由于$\bv{E_{ij}}$中的1元素的位置任意,所以矩阵$\bv{A}$对角线上的元素必须互相相同,故与任意$n$阶矩阵都可交换的矩阵只能是数量矩阵,即$\bv{A}=k\bv{E}$
\end{solution}
\begin{example}{}{}
    证明:若$\bv{A}$为实对称矩阵且$\bv{A}^2=\bf{0}$,则$\bv{A}=\bf{0}$.
\end{example}
\begin{solution}
设 $\bv{A}$ 的行向量为 $\bv{\alpha}_1, \bv{\alpha}_2, \bv{\alpha}_3, \dots, \bv{\alpha}_n$,
列向量为 $\bv{\beta}_1, \bv{\beta}_2, \bv{\beta}_3, \dots, \bv{\beta}_n$。
考虑到 $\bv{0} = \bv{A}\bv{A}$,
则 $\bv{0}$ 的第 $i$ 行 $j$ 列的元素是 $\bv{\alpha}_i \cdot \bv{\beta}_j$,
故任意 $i,j$ 均有 $\bv{\alpha}_i \cdot \bv{\beta}_j = 0$。
又因为 $\bv{A}$ 为实对称矩阵,
所以 $\bv{\beta}_j = \bv{\alpha}_j$,
所以考虑当 $i=j$ 时,$\bv{\alpha}_i^2 = 0$,
则 $\bv{\alpha}_i$ 的元素均为 $0$,
故 $\bv{A}$ 为 $\bf{0}$。
\end{solution}
\begin{example}{}{}
    证明:任一方阵可以表示成一个对称矩阵和一个反对称矩阵的和
\end{example}
\begin{solution}
    设 \( A \) 为任意 \( n \times n \) 方阵。定义两个矩阵:
\begin{align}
S &= \frac{1}{2} (A + A^T) \\
K &= \frac{1}{2} (A - A^T)
\end{align}

\noindent 首先,验证 \( S \) 是对称矩阵:
\[
S^T = \left( \frac{1}{2} (A + A^T) \right)^T = \frac{1}{2} (A^T + A) = S
\]
因此,\( S \) 是对称矩阵。

\noindent 其次,验证 \( K \) 是反对称矩阵:
\[
K^T = \left( \frac{1}{2} (A - A^T) \right)^T = \frac{1}{2} (A^T - A) = -\frac{1}{2} (A - A^T) = -K
\]
因此,\( K \) 是反对称矩阵。

\noindent 最后,验证 \( A = S + K \):
\[
S + K = \frac{1}{2} (A + A^T) + \frac{1}{2} (A - A^T) = \frac{1}{2} (2A) = A
\]

故任一方阵 \( A \) 可分解为对称矩阵 \( S \) 和反对称矩阵 \( K \) 的和,即 \( A = S + K \)。证明完毕。
\end{solution}
\begin{example}{}{}
    将$\begin{pmatrix}1&3&5&-1\\2&-1&-3&4\\5&1&-1&7\\7&7&9&1\end{pmatrix}$化为阶梯形矩阵.
\end{example}
\begin{solution}
    第2行减去2倍第1行,第3行减去5倍第1行,第4行减去7倍第1行,得到:
\[\begin{pmatrix}1&3&5&-1\\2&-1&-3&4\\5&1&-1&7\\7&7&9&1\end{pmatrix}\rightarrow \begin{pmatrix}1&3&5&-1\\0&-7&-13&6\\0&-14&-26&12\\0&-14&-26&8\end{pmatrix}\]
第3行减去2倍第2行,第4行减去2倍第2行,交换第3行和第4行得到:
 \[\begin{pmatrix}1&3&5&-1\\0&-7&-13&6\\0&0&0&0\\0&0&0&-4\end{pmatrix}\rightarrow \begin{pmatrix}1&3&5&-1\\0&-7&-13&6\\0&0&0&-4\\0&0&0&0\end{pmatrix}\]
\end{solution}
\begin{example}{}{}
    $\text{已知}\bv{A}=\begin{pmatrix}1&2&-1&0&0&0\\3&4&-2&0&0&0\\5&-3&1&0&0&0\\0&0&0&3&2&0\\0&0&0&1&4&0\\0&0&0&0&0&3\end{pmatrix},\text{用分块矩阵的方法求}\bv{A}^2.$
\end{example}
\begin{solution}
    令$\bv{A}=\begin{pmatrix}\bv{B}&\bv{O}\\\bv{O}&\bv{C}\end{pmatrix}$,则$\bv{A}^2=\begin{pmatrix}\bv{B}^2&\bv{O}\\\bv{O}&\bv{C}^2\end{pmatrix}$,其中$\bv{O}$为零矩阵。
    \newline 计算得到$\bv{B^2}=\begin{pmatrix}2&13&-6\\5&28&-13\\1&5&-2\end{pmatrix},\bv{C^2}=\begin{pmatrix}10&-13&3\\-13&46&-13\\3&-13&10\end{pmatrix}$.故答案为\[\begin{pmatrix}2&13&-6&0&0&0\\5&28&-13&0&0&0\\1&-5&2&0&0&0\\0&0&0&11&14&0\\0&0&0&7&18&0\\0&0&0&0&0&9\end{pmatrix}\]
\end{solution}
\begin{example}{}{}
    计算下列矩阵的秩,如果矩阵为满秩,计算出矩阵的逆:\[\bv{A}=\begin{pmatrix}2&0&0\\0&2&-1\\0&3&5\end{pmatrix}\]
\end{example}
\begin{solution}
    \[\begin{pmatrix}2&0&0\\0&2&-1\\0&3&5\end{pmatrix}\rightarrow \begin{pmatrix}2&0&0\\0&2&-1\\0&0&6.5\end{pmatrix}\]
    则$\det{\bv{A}}\ne 0$矩阵的秩为3,矩阵为满秩,下求逆矩阵:
    \begin{align*}
        &\begin{pmatrix}2&0&0&1&0&0\\0&2&-1&0&1&0\\0&3&5&0&0&1\end{pmatrix}\rightarrow \begin{pmatrix}1&0&0&\frac{1}{2}&0&0\\0&1&-\frac{1}{2}&0&\frac{1}{2}&0\\0&3&5&0&0&1\end{pmatrix}\rightarrow \begin{pmatrix}1&0&0&\frac{1}{2}&0&0\\0&1&-\frac{1}{2}&0&\frac{1}{2}&0\\0&0&\frac{13}{2}&0&-\frac{3}{2}&1\end{pmatrix}\\&\rightarrow  \begin{pmatrix}1&0&0&\frac{1}{2}&0&0\\0&1&-\frac{1}{2}&0&\frac{1}{2}&0\\0&0&\frac{1}{2}&0&-\frac{3}{26}&\frac{1}{13}\end{pmatrix}\rightarrow \begin{pmatrix}1&0&0&\frac{1}{2}&0&0\\0&1&0&0&\frac{5}{13}&\frac{1}{13}\\0&0&\frac{1}{2}&0&-\frac{3}{26}&\frac{1}{13}\end{pmatrix}\rightarrow \begin{pmatrix}1&0&0&\frac{1}{2}&0&0\\0&1&0&0&\frac{5}{13}&\frac{1}{13}\\0&0&1&0&-\frac{3}{13}&\frac{2}{13}\end{pmatrix}\
    \end{align*}
所以\[\bv{A}^{-1}=\begin{pmatrix}
    \frac{1}{2}&0&0\\0&\frac{5}{13}&\frac{1}{13}\\0&-\frac{3}{13}&\frac{2}{13}
\end{pmatrix}\]
\end{solution}
\begin{example}{}{}
    求这个矩阵的逆,其中$a_i\neq 0,(i=1,2,\cdots,n)$\[\begin{pmatrix}0&a_1&0&\cdots&0&0\\0&0&a_2&\cdots&0&0\\\vdots&\vdots&\vdots&&\vdots&\vdots\\0&0&0&\cdots&a_{n-2}&0\\0&0&0&\cdots&0&a_{n-1}\\a_n&0&0&\cdots&0&0\end{pmatrix}\]
\end{example}
\begin{solution}
    考虑矩阵
将矩阵的第$1$行乘以$\frac{1}{a_n}$,第二行乘以$\frac{1}{a_1}$,依此类推,得到:
\[
\left(\begin{array}{cccccc|cccccc}
0 & a_1 & 0 & \cdots & 0 & 0 & 1 & 0 & 0 & \cdots & 0 & 0 \\
0 & 0 & a_2 & \cdots & 0 & 0 & 0 & 1 & 0 & \cdots & 0 & 0 \\
\vdots & \vdots & \vdots & \ddots & \vdots & \vdots & \vdots & \vdots & \vdots & \ddots & \vdots & \vdots \\
0 & 0 & 0 & \cdots & a_{n-2} & 0 & 0 & 0 & 0 & \cdots & 0 & 0 \\
0 & 0 & 0 & \cdots & 0 & a_{n-1} & 0 & 0 & 0 & \cdots & 1 & 0 \\
a_n & 0 & 0 & \cdots & 0 & 0 & 0 & 0 & 0 & \cdots & 0 & 1
\end{array}\right)
\]
将这个矩阵的第$n$行与第$n-1$行交换,再将第$n-1$行与第$n-2$行交换,依此类推,直到第$2$行与第$1$行交换,得到:
\[
\left(\begin{array}{cccccc|cccccc}
a_n & 0 & 0 & \cdots & 0 & 0 & 0 & 0 & 0 & \cdots & 0 & 1 \\
0 & a_1 & 0 & \cdots & 0 & 0 & 1 & 0 & 0 & \cdots & 0 & 0 \\
0 & 0 & a_2 & \cdots & 0 & 0 & 0 & 1 & 0 & \cdots & 0 & 0 \\
0 & 0 & 0 & \cdots & 0 & 0 & 0 & 0 & 1 & \cdots & 0 & 0 \\
\vdots & \vdots & \vdots & \ddots & \vdots & \vdots & \vdots & \vdots & \vdots & \ddots & \vdots & \vdots \\
0 & 0 & 0 & \cdots & a_{n-2} & 0 & 0 & 0 & 0 & \cdots & 1 & 0 \\
0 & 0 & 0 & \cdots & 0 & a_{n-1} & 0 & 0 & 0 & \cdots & 0 & 1 
\end{array}\right)
\]
将矩阵的第$1$行乘以$\frac{1}{a_n}$,第二行乘以$\frac{1}{a_1}$,依此类推,得到:
\[
\left(\begin{array}{cccccc|cccccc}
1 & 0 & 0 & \cdots & 0 & 0 & 0 & 0 & 0 & \cdots & 0 & \frac{1}{a_n} \\
0 & 1 & 0 & \cdots & 0 & 0 & \frac{1}{a_1} & 0 & 0 & \cdots & 0 & 0 \\
0 & 0 & 1 & \cdots & 0 & 0 & 0 & \frac{1}{a_2} & 0 & \cdots & 0 & 0 \\
0 & 0 & 0 & \cdots & 0 & 0 & 0 & 0 & \frac{1}{a_3} & \cdots & 0 & 0 \\
\vdots & \vdots & \vdots & \ddots & \vdots & \vdots & \vdots & \vdots & \vdots & \ddots & \vdots & \vdots \\
0 & 0 & 0 & \cdots & 1 & 0 & 0 & 0 & 0 & \cdots & \frac{1}{a_{n-2}} & 0 \\
0 & 0 & 0 & \cdots & 0 & 1 & 0 & 0 & 0 & \cdots & 0 & \frac{1}{a_{n-1}} 
\end{array}\right)
\Rightarrow\text{逆矩阵为}
\begin{pmatrix}
0 & 0 & 0 & \cdots & 0 & \frac{1}{a_n} \\
\frac{1}{a_1} & 0 & 0 & \cdots & 0 & 0 \\
0 & \frac{1}{a_2} & 0 & \cdots & 0 & 0 \\
0 & 0 & \frac{1}{a_3} & \cdots & 0 & 0 \\
\vdots & \vdots & \vdots & \ddots & \vdots & \vdots \\
0 & 0 & 0 & \cdots & \frac{1}{a_{n-2}} & 0 \\
0 & 0 & 0 & \cdots & 0 & \frac{1}{a_{n-1}}
\end{pmatrix}
\]
\end{solution}
\begin{example}{}{}
    求矩阵$\bv{X}$使得:
    \[\begin{pmatrix}1&1&1&\cdots&1&1\\0&1&1&\cdots&1&1\\0&0&1&\cdots&1&1\\\vdots&\vdots&\vdots&&\vdots&\vdots\\0&0&0&\cdots&1&1\\0&0&0&\cdots&0&1\end{pmatrix}\bv{X}=\begin{pmatrix}2&1&0&\cdots&0&0\\1&2&1&\cdots&0&0\\0&1&2&\cdots&0&0\\\vdots&\vdots&\vdots&&\vdots&\vdots\\0&0&0&\cdots&2&1\\0&0&0&\cdots&1&2\end{pmatrix}.\]
\end{example}
\begin{solution}
    转化为求这个矩阵的逆:\[\bv{A}=\begin{pmatrix}1&1&1&\cdots&1&1\\0&1&1&\cdots&1&1\\0&0&1&\cdots&1&1\\\vdots&\vdots&\vdots&&\vdots&\vdots\\0&0&0&\cdots&1&1\\0&0&0&\cdots&0&1\end{pmatrix}\Rightarrow\text{构造}[A|E]=\left(\begin{array}{cccccc|cccccc}1&1&1&\cdots&1&1&1&0&0&\cdots&0&0\\0&1&1&\cdots&1&1&0&1&0&\cdots&0&0\\0&0&1&\cdots&1&1&0&0&1&\cdots&0&0\\\vdots&\vdots&\vdots&\ddots&\vdots&\vdots&\vdots&\vdots&\vdots&\ddots&\vdots&\vdots\\0&0&0&\cdots&1&1&0&0&0&\cdots&1&0\\0&0&0&\cdots&0&1&0&0&0&\cdots&0&1\end{array}\right)\]
    第k行减去第k+1行,即可消去第k行中第k+1列及以后的所有1,得到:
    \[\left(
    \begin{array}{cccccc|cccccc}
1&0&0&\cdots&0&0&1&-1&0&\cdots&0&0\\0&1&0&\cdots&0&0&0&1&-1&\cdots&0&0\\0&0&1&\cdots&0&0&0&0&1&\cdots&0&0\\\vdots&\vdots&\vdots&\ddots&\vdots&\vdots&\vdots&\vdots&\vdots&\ddots&\vdots&\vdots\\0&0&0&\cdots&1&0&0&0&0&\cdots&1&-1\\0&0&0&\cdots&0&1&0&0&0&\cdots&0&1
    \end{array}\right)\Rightarrow\bv{A}^{-1}=\begin{pmatrix}1&-1&0&\cdots&0&0\\0&1&-1&\cdots&0&0\\0&0&1&\cdots&0&0\\\vdots&\vdots&\vdots&\ddots&\vdots&\vdots\\0&0&0&\cdots&1&-1\\0&0&0&\cdots&0&1\end{pmatrix}
    \]
    等式两边右乘以$\bv{A}^{-1}$,得:
    \[\bv{X}=\begin{pmatrix}1&-1&0&\cdots&0&0\\0&1&-1&\cdots&0&0\\0&0&1&\cdots&0&0\\\vdots&\vdots&\vdots&\ddots&\vdots&\vdots\\0&0&0&\cdots&1&-1\\0&0&0&\cdots&0&1\end{pmatrix}\begin{pmatrix}2&1&0&\cdots&0&0\\1&2&1&\cdots&0&0\\0&1&2&\cdots&0&0\\\vdots&\vdots&\vdots&&\vdots&\vdots\\0&0&0&\cdots&2&1\\0&0&0&\cdots&1&2\end{pmatrix}=\begin{pmatrix}1&-1&-1&0&\cdots&0&0&0\\1&1&-1&-1&\cdots&0&0&0\\0&1&1&-1&\cdots&0&0&0\\\vdots&\vdots&\vdots&\vdots&&\vdots&\vdots&\vdots\\0&0&0&0&\cdots&1&1&-1\\0&0&0&0&\cdots&0&1&2\end{pmatrix}\]
\end{solution}
\newpage
\begin{example}{}{}
    求解线性方程组:$\left\{\begin{aligned}2x_1+3x_3&=1,\\3x_2-5x_3&=-4,\\-2x_1+3x_2+2x_3&=4;\end{aligned}\right.$
\end{example}
\begin{solution}
    构造增广矩阵并进行矩阵的行变换:\[\left(\begin{array}{cccc}2&0&3&1\\0&3&-5&4\\-2&3&2&4\end{array}\right)\rightarrow \left(\begin{array}{cccc}2&0&3&1\\0&3&-5&-4\\0&3&5&5\end{array}\right)\rightarrow \left(\begin{array}{cccc}2&0&3&1\\0&3&-5&-4\\0&0&10&9\end{array}\right)\rightarrow \begin{pmatrix}1&0&\frac{3}{2}&\frac{1}{2}\\0&1&-\frac{5}{3}&-\frac{4}{3}\\0&0&1&\frac{9}{10}\end{pmatrix}\]\[\rightarrow \begin{pmatrix}1&0&0&-\frac{17}{20}\\0&1&0&\frac{1}{6}\\0&0&1&\frac{9}{10}\end{pmatrix} \Rightarrow x_1=-\dfrac{17}{20},x_2=\dfrac16,x_3=\dfrac{9}{10}.\]
最后一行是由于$\begin{pmatrix}2&0&3\\0&3&-5\\-2&3&2\end{pmatrix}\begin{pmatrix}x_1\\x_2\\x_3\end{pmatrix}=\bv{E}$,则$x_1=-\dfrac{17}{20},x_2=\dfrac16,x_3=\dfrac{9}{10}$
\end{solution}
\begin{example}{}{}
    设$\bv{A}=\left(\begin{array}{rrr}1&2&-1\\3&4&-2\\5&-3&1\end{array}\right)~~~~~~\bv{B}=\begin{pmatrix}3&4&-2\\5&-3&1\\1&2&-1\end{pmatrix}$,求$\bv{A}^{-1}\bv{B}$.
\end{example}
\begin{solution}
构造增广矩阵并进行矩阵的行变换
\[[A|E]=\left(\begin{array}{ccc|ccc}1&2&-1&1&0&0\\3&4&-2&0&1&0\\5&-3&1&0&0&1\end{array}\right)\rightarrow \left(\begin{array}{ccc|ccc}1&2&-1&1&0&0\\0&-2&-5&-3&1&0\\0&-13&-4&-5&0&1\end{array}\right)\rightarrow \left(\begin{array}{ccc|ccc}1&2&-1&1&0&0\\0&1&-\frac{1}{2}&\frac{3}{2}&-\frac{1}{2}&0\\0&-13&6&-5&0&1\end{array}\right)\]
\[\rightarrow \left(\begin{array}{ccc|ccc}1&0&0&-2&1&0\\0&1&-\frac{1}{2}&\frac{3}{2}&-\frac{1}{2}&0\\0&0&-\frac{1}{2}&\frac{29}{2}&-\frac{13}{2}&1\end{array}\right)
\rightarrow \left(\begin{array}{ccc|ccc}1&0&0&-2&1&0\\0&1&-\frac{1}{2}&\frac{3}{2}&-\frac{1}{2}&0\\0&0&1&-29&13&-2\end{array}\right)\rightarrow 
\left(\begin{array}{ccc|ccc}1&0&0&-2&1&0\\0&1&0&-13&6&-1\\0&0&1&-29&13&-2\end{array}\right)\]
\[\bv{A}^{-1}\bv{B}=\begin{pmatrix}-2&1&0\\-13&6&-1\\-29&13&-2\end{pmatrix}\begin{pmatrix}3&4&-2\\5&-3&1\\1&2&-1\end{pmatrix}=\begin{pmatrix}-1&-11&5\\-10&-72&33\\-24&-159&73\end{pmatrix}\]
\end{solution}
\begin{example}{}{}
    求$(k+l)\times(k+l)$矩阵$\begin{pmatrix}\bv{E}_k&\bv{B}\\\bv{O}&\bv{E}_l\end{pmatrix}$的逆,其中$\bv{E}_k$为$k\times k$单位矩阵,$\bv{E}_l$为$l\times l$单位矩阵.
\end{example}
\begin{solution}
设逆矩阵为:$\bv{M}^{-1} = \begin{pmatrix} \bv{X} & \bv{Y} \\ \bv{Z} & \bv{W} \end{pmatrix}$,其中 $\bv{X}$ 是 $k \times k$ 矩阵,$\bv{Y}$ 是 $k \times l$ 矩阵,$\bv{Z}$ 是 $l \times k$ 矩阵,$\bv{W}$ 是 $l \times l$ 矩阵。计算矩阵乘积:
\begin{align*}
\begin{pmatrix} \bv{E}_k & \bv{B} \\ \bv{O} & \bv{E}_l \end{pmatrix}
\begin{pmatrix} \bv{X} & \bv{Y} \\ \bv{Z} & \bv{W} \end{pmatrix} 
&= \begin{pmatrix}
    \bv{E}_k\bv{X} + \bv{B}\bv{Z} & \bv{E}_k\bv{Y} + \bv{B}\bv{W} \\
    \bv{O}\bv{X} + \bv{E}_l\bv{Z} & \bv{0}\bv{Y} + \bv{E}_l\bv{W}
\end{pmatrix} \\
&= \begin{pmatrix}
    \bv{X} + \bv{B}\bv{Z} & \bv{Y} + \bv{B}\bv{W} \\
    \bv{Z} & \bv{W}
\end{pmatrix}
\end{align*}
等式成立:
\[
\begin{pmatrix}
    \bv{X} + \bv{B}\bv{Z} & \bv{Y} + \bv{B}\bv{W} \\
    \bv{Z} & \bv{W}
\end{pmatrix} = 
\begin{pmatrix} \bv{E}_k & \bv{O} \\ \bv{O} & \bv{E}_l \end{pmatrix}
\]
得到方程组:
$\bv{X} + \bv{B}\bv{Z} = \bv{E}_k,\quad\bv{Y} + \bv{B}\bv{W} = \bv{O},\quad\bv{Z} = \bv{O},\quad\bv{W} = \bv{E}_l$,因此,逆矩阵为:
\[\bv{M}^{-1} = \begin{pmatrix} \bv{E}_k & -\bv{B} \\ \bv{O} & \bv{E}_l \end{pmatrix}\]
\end{solution}
\begin{example}{}{}
    设$\bv{A},\bv{B},\bv{C}$是同阶方阵,其中$\bv{A},\bv{B}$可逆,求下列矩阵的逆:\[\bv{D}=\begin{pmatrix}\bv{O}&\bv{A}\\\bv{B}&\bv{C}\end{pmatrix}\]
\end{example}
\begin{solution}
    设逆矩阵为:\[\bv{D}^{-1} = \begin{pmatrix} \bv{X} & \bv{Y} \\ \bv{Z} & \bv{W} \end{pmatrix}\]其中 \(\bv{X}, \bv{Y}, \bv{Z}, \bv{W}\) 是与 \(\bv{A}\) 同阶的方阵。计算矩阵乘积:
\begin{align*}
\begin{pmatrix} \bv{O} & \bv{A} \\ \bv{B} & \bv{C} \end{pmatrix}
\begin{pmatrix} \bv{X} & \bv{Y} \\ \bv{Z} & \bv{W} \end{pmatrix} 
&= \begin{pmatrix}
    \bv{O}\bv{X} + \bv{A}\bv{Z} & \bv{O}\bv{Y} + \bv{A}\bv{W} \\
    \bv{B}\bv{X} + \bv{C}\bv{Z} & \bv{B}\bv{Y} + \bv{C}\bv{W}
\end{pmatrix} \\
&= \begin{pmatrix} \bv{E} & \bv{O} \\ \bv{O} & \bv{E} \end{pmatrix}
\end{align*}
因此$\bv{A}\bv{Z} = \bv{E} ,\bv{A}\bv{W} = \bv{O} ,\bv{B}\bv{X} + \bv{C}\bv{Z} = \bv{O} ,\bv{B}\bv{Y} + \bv{C}\bv{W} = \bv{E}$,计算得到
\[\bv{D}^{-1} = \begin{pmatrix} 
    -\bv{B}^{-1}\bv{C}\bv{A}^{-1} & \bv{B}^{-1} \\
    \bv{A}^{-1} & \bv{O}
\end{pmatrix}\]
\end{solution}
\begin{example}{}{}
    若$\bv{A}^k=\bv{O}$,证明:\[(\bv{E}-\bv{A})^{-1}=\bv{E}+\bv{A}+\bv{A}^2+\cdots+\bv{A}^{k-1}\]
\end{example}
\begin{solution}
    设 $\bv{S} = \bv{E} + \bv{A} + \bv{A}^2 + \cdots + \bv{A}^{k-1}$,计算乘积:
\begin{align*}
(\bv{E} - \bv{A})\bv{S} &= \bv{E}(\bv{E} + \bv{A} + \bv{A}^2 + \cdots + \bv{A}^{k-1})- \bv{A}(\bv{E} + \bv{A} + \bv{A}^2 + \cdots + \bv{A}^{k-1}) \\
&= (\bv{E} + \bv{A} + \bv{A}^2 + \cdots + \bv{A}^{k-1}) - (\bv{A} + \bv{A}^2 + \bv{A}^3 + \cdots + \bv{A}^k)\\
&= \bv{E} + (\bv{A} - \bv{A}) + (\bv{A}^2 - \bv{A}^2) + \cdots + (\bv{A}^{k-1} - \bv{A}^{k-1}) - \bv{A}^k \\
&= \bv{E} + \bv{O} + \bv{O} + \cdots + \bv{O} - \bv{O} = \bv{E}
\end{align*}
\end{solution}
\begin{example}{}{}
    设$\bv{A}$为$n(n\geq 2)$阶方阵,证明$|\bv{A}^{*}|=|\bv{A}|^{n-1}$.
\end{example}
\begin{solution}
由伴随矩阵的定义,有:
\[\bv{A} \bv{A}^{*} = \bv{A}^{*} \bv{A} = |\bv{A}| \bv{E}\]
其中 $\bv{E}$ 是 $n$ 阶单位矩阵。对等式 $\bv{A} \bv{A}^{*} = |\bv{A}| \bv{E}$ 两边取行列式:
\[|\bv{A} \bv{A}^{*}| =|\bv{A}||\bv{A}^{*}|= ||\bv{A}| \bv{E}|=\begin{vmatrix}
|\bv{A}| & 0 & 0 & \cdots & 0 & 0 \\
0 & |\bv{A}| & 0 & \cdots & 0 & 0 \\
0 & 0 & |\bv{A}| & \cdots & 0 & 0 \\
\vdots & \vdots & \vdots & \ddots & \vdots & \vdots \\
0 & 0 & 0 & \cdots & |\bv{A}| & 0 \\
0 & 0 & 0 & \cdots & 0 & |\bv{A}|
\end{vmatrix}=|\bv{A}|^{n}\]
分情况讨论:当$ |\bv{A}|\neq 0$时可以直接得到$|\bv{A}^{*}|=|\bv{A}|^{n-1}$.当$ |\bv{A}|=0$时,则我们可以对矩阵$\bv{A}$进行初等变换使之变成阶梯型矩阵,那么最后一行元素一定全为$0$,因此$|\bv{A}|$中最多只有最后一行的元素对应的代数余子式非零,所以$|\bv{A}^{*}|$中一定有全为$0$的列,所以$|\bv{A}^{*}|=0=|\bv{A}|^{n-1}$,综上所述,无论 $|\bv{A}|$ 是否为零,都有:
\[|\bv{A}^{*}| = |\bv{A}|^{n-1}\]
\end{solution}