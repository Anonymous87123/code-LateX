\chapter{特征值和特征向量}
\newpage
\begin{example}{求特征值和特征向量}{}
    (1)$\left(\begin{array}{rrr}3&2&-1\\-2&-2&2\\3&6&-1\end{array}\right)$\quad(2)$\begin{pmatrix}0&0&0&1\\0&0&1&0\\0&1&0&0\\1&0&0&0\end{pmatrix}$
\end{example}
\begin{solution}
(1)计算 $\det(A - \lambda E) = 0$:
\begin{align*}
\det(\bm{A - \lambda E}) &=\begin{vmatrix}
3-\lambda & 2 & -1 \\
-2 & -2-\lambda & 2 \\
3 & 6 & -1-\lambda
\end{vmatrix}\\
&= (3-\lambda)\begin{vmatrix} -2-\lambda & 2 \\ 6 & -1-\lambda \end{vmatrix} 
- 2\begin{vmatrix} -2 & 2 \\ 3 & -1-\lambda \end{vmatrix} 
+ (-1)\begin{vmatrix} -2 & -2-\lambda \\ 3 & 6 \end{vmatrix} \\
&= (3-\lambda)[(-2-\lambda)(-1-\lambda) - 12] - 2[(-2)(-1-\lambda) - 6] - 1[(-2)(6) - (-2-\lambda)(3)] \\
&= (3-\lambda)[(2+3\lambda+\lambda^2) - 12] - 2[2+2\lambda - 6] - 1[-12 + 6 + 3\lambda] \\
&= (3-\lambda)(\lambda^2 + 3\lambda - 10) - 2(2\lambda - 4) - (3\lambda - 6) \\
&= (3-\lambda)(\lambda-2)(\lambda+5) - 7(\lambda-2)=(\lambda-2)(15-7-2\lambda-\lambda^2)\\
&= (\lambda-2)(8-2\lambda-\lambda^2)=-(\lambda-2)^2(\lambda+4)=0
\end{align*}
所以特征值是$\lambda_1=2$(二重)和$\lambda_2=-4$(一重)\\
先计算矩阵$\bm{A-2E}$并进行行变换:
\[\bm{A-2E}=\begin{pmatrix}
1 & 2 & -1 \\
-2 & -4 & 2 \\
3 & 6 & -3
\end{pmatrix}\rightarrow\begin{pmatrix}
1 & 2 & -1 \\
0 & 0 & 0 \\
0 & 0 & 0
\end{pmatrix}\]
所以选取$x_2,x_3$为自由变量,$x_1=x_3-2x_2$,则有基向量:
\[x_2 = 1, x_3 = 0 \Rightarrow x_1 = -2,\mathbf{v}_1 = \begin{pmatrix} -2 \\ 1 \\ 0 \end{pmatrix}\quad x_2 = 0, x_3 = 1 \Rightarrow x_1 = 1,\mathbf{v}_2 = \begin{pmatrix} 1 \\ 0 \\ 1 \end{pmatrix}\]
则特征值$2$对应的特征向量为$k_1\bm{v_1}+k_2\bm{v_2}$,其中$k_1,k_2$为任意常数.\\
计算矩阵$\bm{A+4E}$并进行行变换:
\[\bm{A+4E}=\begin{pmatrix}
7 & 2 & -1 \\
-2 & 2 & 2 \\
3 & 6 & 3
\end{pmatrix}\rightarrow 
\begin{pmatrix}
7 & 2 & -1 \\
-2\times 7+7\times 2 & 2\times 7+2\times 2 & 2\times 7-2\times (-1) \\
3\times 7-7\times 3 & 6\times 7-2\times 3 & 3\times 7+1\times 3
\end{pmatrix}\]
得到$\begin{pmatrix}
7 & 2 & -1 \\
0 & 18 & 12 \\
0 & 36 & 24\end{pmatrix}\rightarrow \begin{pmatrix}
7 & 2 & -1 \\
0 & 3 & 2 \\
0 & 0 & 0\end{pmatrix}$
选取$x_3$为自由变量,解得$x_2 = -\frac{2}{3}x_3$,代入得 $x_1 = \frac{1}{3}x_3$
取$x_3 = 3$:$\mathbf{v}_3 = \begin{pmatrix} 1 \\ -2 \\ 3 \end{pmatrix}$,则特征值$-4$对应的特征向量为$k_3\begin{pmatrix} 1 \\ -2 \\ 3 \end{pmatrix}$,其中$k_3$为任意常数.\\
(2)计算 $\det(A - \lambda E) = 0$:
\begin{align*}
|\bm{B - \lambda E}| &= \begin{vmatrix}
-\lambda & 0 & 0 & 1 \\
0 & -\lambda & 1 & 0 \\
0 & 1 & -\lambda & 0 \\
1 & 0 & 0 & -\lambda
\end{vmatrix}= (-\lambda)\begin{vmatrix} -\lambda & 1 & 0 \\ 1 & -\lambda & 0 \\ 0 & 0 & -\lambda \end{vmatrix} 
- 0 + 0 
- 1\begin{vmatrix} 0 & -\lambda & 1 \\ 0 & 1 & -\lambda \\ 1 & 0 & 0 \end{vmatrix}\\
&= (-\lambda)^2[(-\lambda)(-\lambda) - 1\cdot1]-(\lambda^2-1)=\lambda^2(\lambda^2-1)-(\lambda^2-1)=(\lambda^2-1)^2
\end{align*}
特征值:$\lambda = 1$(二重根),$\lambda = -1$(二重根),对于 $\lambda = 1$,构造矩阵$\bm{A-E}$进行初等行变换:
\[\bm{A-E}=\begin{pmatrix}
-1 & 0 & 0 & 1 \\
0 & -1 & 1 & 0 \\
0 & 1 & -1 & 0 \\
1 & 0 & 0 & -1
\end{pmatrix}\rightarrow\begin{pmatrix}
-1 & 0 & 0 & 1 \\
0 & -1 & 1 & 0 \\
0 & 0 & 0 & 0 \\
0 & 0 & 0 & 0
\end{pmatrix}\rightarrow\begin{cases}x_2=x_3\\x_1=x_4\end{cases}\]
基向量:$x_3 = 1, x_4 = 0\Rightarrow\bm{\xi_1} = \begin{pmatrix} 0 \\ 1 \\ 1 \\ 0 \end{pmatrix}\quad x_3 = 0, x_4 = 1\Rightarrow\bm{\xi_2} = \begin{pmatrix} 1 \\ 0 \\ 0 \\ 1 \end{pmatrix}$\\
所以特征值$\lambda_1=1$对应的特征向量为$k_1\bm{\xi_1}+k_2\bm{\xi_2}$,其中$k_1,k_2$为任意常数。\\
对于 $\lambda = -1$,构造矩阵$\bm{A+E}$进行初等行变换:
\[\bm{A+E}=\begin{pmatrix}
1 & 0 & 0 & 1 \\
0 & 1 & 1 & 0 \\
0 & 1 & 1 & 0 \\
1 & 0 & 0 & 1
\end{pmatrix}\rightarrow\begin{pmatrix}
1 & 0 & 0 & 1 \\
0 & 1 & 1 & 0 \\
0 & 0 & 0 & 0 \\
0 & 0 & 0 & 0
\end{pmatrix}\rightarrow\begin{cases}x_2=-x_3\\x_1=-x_4\end{cases}\]
基向量:$x_3 = 1, x_4 = 0\Rightarrow\bm{\xi_3} = \begin{pmatrix} 0 \\ -1 \\ 1 \\ 0 \end{pmatrix}\quad x_3 = 0, x_4 = 1\Rightarrow\bm{\xi_4} = \begin{pmatrix} -1 \\ 0 \\ 0 \\ 1 \end{pmatrix}$\\
所以特征值$\lambda_1=-1$对应的特征向量为$k_3\bm{\xi_3}+k_4\bm{\xi_4}$,其中$k_3,k_4$为任意常数.
\end{solution}
\begin{example}{求该$n$阶矩阵的特征值和特征向量}{}
    \[\bm{A}=\begin{pmatrix}0&1&1&\cdots&1\\1&0&1&\cdots&1\\1&1&0&\cdots&1\\\vdots&\vdots&\vdots&&\vdots\\1&1&1&\cdots&0\end{pmatrix}\]
\end{example}
\begin{solution}
    计算 $\det(\bm{A - \lambda E}) = 0$:
    \begin{align*}
|\bm{A - \lambda E}| &= \begin{vmatrix}
-\lambda & 1 & 1 & \cdots & 1 \\
1 & -\lambda & 1 & \cdots & 1 \\
1 & 1 & -\lambda & \cdots & 1 \\
\vdots & \vdots & \vdots & \ddots & \vdots \\
1 & 1 & 1 & \cdots & -\lambda
\end{vmatrix}=\begin{vmatrix}
n-1-\lambda & n-1-\lambda & n-1-\lambda & \cdots & n-1-\lambda \\
1 & -\lambda & 1 & \cdots & 1 \\
1 & 1 & -\lambda & \cdots & 1 \\
\vdots & \vdots & \vdots & \ddots & \vdots \\
1 & 1 & 1 & \cdots & -\lambda
\end{vmatrix}\\
&=(n-1-\lambda)\begin{vmatrix}
1 & 1 & 1 & \cdots & 1 \\
1 & -\lambda & 1 & \cdots & 1 \\
1 & 1 & -\lambda & \cdots & 1 \\
\vdots & \vdots & \vdots & \ddots & \vdots \\
1 & 1 & 1 & \cdots & -\lambda
\end{vmatrix}=(n-1-\lambda)\begin{vmatrix}
1 & 1 & 1 & \cdots & 1 \\
0 & -\lambda-1 & 0 & \cdots & 0 \\
0 & 0 & -\lambda-1 & \cdots & 0 \\
\vdots & \vdots & \vdots & \ddots & \vdots \\
0 & 0 & 0 & \cdots & -\lambda-1
\end{vmatrix}\\
&=(n-1-\lambda)(-\lambda-1)^{n-1}=0\Rightarrow \lambda_1=n-1,\lambda_2=-1
\end{align*}
特征值为$\lambda_1=n-1$(一重)和$\lambda_2=-1$($n-1$重)\\
对于特征值$\lambda_1=n-1$,我们根据相同的行变换过程发现此时每一列的和为0,因此此时这个矩阵的所有行向量线性相关,所以只有一个有效方程,而且$x_1,x_2,\cdots,x_n$应该相等,严谨的说明过程则需要写出:
\[\begin{cases}-(n-1)x_1+x_2+x_3+\cdots+x_n=0\\x_1-(n-1)x_2+x_3+\cdots+x_n=0\\x_1+x_2-(n-1)x_3+\cdots+x_n=0\\\cdots\\x_1+x_2+x_3+\cdots-(n-1)x_n=0\end{cases}\]
设 $S = x_1 + x_2 + \cdots + x_n$,则第 $i$ 个方程为:
\[-(n-1)x_i + (S - x_i) = 0 \Rightarrow S = n x_i\]
这对所有 $i = 1, 2, \ldots, n$ 成立,因此:
\[n x_1 = n x_2 = \cdots = n x_n \Rightarrow x_1 = x_2 = \cdots = x_n\]
故特征向量为 $\bm{\xi} = k(1, 1, \ldots, 1)^T$,$k \neq 0$.\\
下面考虑另一个特征向量$\lambda_2=-1$,将刚才进行的行变换作用于矩阵:
\begin{align*}
\bm{A - \lambda E}&= \begin{pmatrix}
-\lambda & 1 & 1 & \cdots & 1 \\
1 & -\lambda & 1 & \cdots & 1 \\
1 & 1 & -\lambda & \cdots & 1 \\
\vdots & \vdots & \vdots & \ddots & \vdots \\
1 & 1 & 1 & \cdots & -\lambda
\end{pmatrix}\to\begin{pmatrix}
n-1-\lambda & n-1-\lambda & n-1-\lambda & \cdots & n-1-\lambda \\
1 & -\lambda & 1 & \cdots & 1 \\
1 & 1 & -\lambda & \cdots & 1 \\
\vdots & \vdots & \vdots & \ddots & \vdots \\
1 & 1 & 1 & \cdots & -\lambda
\end{pmatrix}\\
&\to\begin{pmatrix}
1 & 1 & 1 & \cdots & 1 \\
1 & -\lambda & 1 & \cdots & 1 \\
1 & 1 & -\lambda & \cdots & 1 \\
\vdots & \vdots & \vdots & \ddots & \vdots \\
1 & 1 & 1 & \cdots & -\lambda
\end{pmatrix}\to\begin{pmatrix}
1 & 1 & 1 & \cdots & 1 \\
0 & -\lambda-1 & 0 & \cdots & 0 \\
0 & 0 & -\lambda-1 & \cdots & 0 \\
\vdots & \vdots & \vdots & \ddots & \vdots \\
0 & 0 & 0 & \cdots & -\lambda-1
\end{pmatrix}
\end{align*}
代入$\lambda=-1$,则可得到$x_1+x_2+\cdots+x_n=0$,一组基可以取为:
\[\bm{\xi_1} = (1, -1, 0, 0, \ldots, 0)^T,\bm{\xi_2} = (1, 0, -1, 0, \ldots, 0)^T,\cdots,\bm{\xi_{n-1}}= (1, 0, 0, \ldots, 0, -1)^T\]
故特征向量为:
\[k_1\bm{\xi_1}+k_2\bm{\xi_2}+\cdots+k_{n-1}\bm{\xi_{n-1}},\quad k_1,k_2,\cdots,k_{n-1}\text{为任意常数}\]
\end{solution}
\begin{example}{}{}
    设$\lambda$是$n$阶可逆矩阵$A$的一个特征值,$\alpha$是$A$的属于特征值$\lambda$的一个特征向量,证明$\dfrac{|A|}{\lambda}$是$A^*$的一个特征值,$\alpha$也是$A^*$属于此特征值的一个特征向量.
\end{example}
\begin{solution}
    由于矩阵$\bm{A}$可逆,所以$\bm{AA^*}=|\bm{A}|\bm{E}$,设列向量$\bm{\alpha}$满足$\bm{A\alpha}=\lambda\bm{\alpha}$,则$\bm{A^*A\alpha}=\lambda\bm{A^*\alpha}$,即$|\bm{A}|\bm{E\alpha}=|\bm{A}|\bm{\alpha}=\lambda\bm{A^*\alpha}$,即$\dfrac{|\bm{A}|}{\lambda}\alpha=\bm{A^*\alpha}$,所以$\dfrac{|\bm{A}|}{\lambda}$是$\bm{A^*}$的一个特征值,$\bm{\alpha}$也是$\bm{A^*}$属于此特征值的一个特征向量.
\end{solution}
\begin{example}{}{}
    设$\lambda_1,\lambda_2$是矩阵$\bm{A}$的两个不同的特征值,$\bm{\alpha_1,\alpha_2}$分别是属于$\lambda_1,\lambda_2$的特征向量.试证:$\bm{\alpha}_1+\bm{\alpha}_2$一定不是$\bm{A}$的特征向量
\end{example}
\begin{solution}
    由定义得到$\bm{A\alpha_1}=\lambda_1\bm{\alpha_1},\bm{A\alpha_2}=\lambda_2\bm{\alpha_2}$,相加有$\bm{A(\alpha_1+\alpha_2)}=\lambda_1\bm{\alpha_1}+\lambda_2\bm{\alpha_2}$,假如$\bm{\alpha}_1+\bm{\alpha}_2$是$\bm{A}$的特征向量,即$\bm{A(\alpha_1+\alpha_2)}=\lambda_3(\bm{\alpha_1+\alpha_2})$,与$\lambda_1\bm{\alpha_1}+\lambda_2\bm{\alpha_2}$联立得到$(\lambda_3-\lambda_1)\bm{\alpha_1}+(\lambda_3-\lambda_2)\bm{\alpha_2}=0$,由$\bm{\alpha_1,\alpha_2}$线性无关,得到$\lambda_3=\lambda_1=\lambda_2$,但是$\lambda_1,\lambda_2$是矩阵$\bm{A}$的两个不同的特征值,因此矛盾,$\bm{\alpha}_1+\bm{\alpha}_2$一定不是$\bm{A}$的特征向量.
\end{solution}
\begin{example}{}{}
    $\text{矩阵}\boldsymbol{A}=\begin{pmatrix}{3}&{2}&{-1}\\{-2}&{a}&{2}\\{3}&{6}&{-1}\end{pmatrix}\text{与矩阵}\boldsymbol{B}=\begin{pmatrix}{-4}&{0}&{0}\\{0}&{b}&{0}\\{0}&{0}&{2}\end{pmatrix}\text{相似},\text{试求}a,b\text{的值}.$
\end{example}
\begin{solution}
    由于矩阵 $\boldsymbol{A}$ 与 $\boldsymbol{B}$ 相似,它们具有相同的特征值。矩阵 $\boldsymbol{B}$ 是对角矩阵,其特征值即为对角线元素:$-4$、$b$ 和 $2$。因此,$\boldsymbol{A}$ 的特征值也应为 $-4$、$b$ 和 $2$,而且由相似矩阵的迹相等得到$3 + a + (-1) = a + 2 = -4 + b + 2 = b - 2\Rightarrow b = a + 4$,由特征多项式韦达定理得到$\boldsymbol{A}$的所有特征值的积为$|\bm{A}|=-4\times b\times 2=-8b$,下面计算行列式(按行展开):
    \begin{align*}
\det(\boldsymbol{A}) &= 3 \cdot 
\begin{vmatrix}
a & 2 \\
6 & -1
\end{vmatrix}
- 2 \cdot 
\begin{vmatrix}
-2 & 2 \\
3 & -1
\end{vmatrix}
+ (-1) \cdot 
\begin{vmatrix}
-2 & a \\
3 & 6
\end{vmatrix} \\
&= 3 \cdot (a \cdot (-1) - 2 \cdot 6) - 2 \cdot ((-2) \cdot (-1) - 2 \cdot 3) - 1 \cdot ((-2) \cdot 6 - a \cdot 3) \\
&= 3 \cdot (-a - 12) - 2 \cdot (2 - 6) - 1 \cdot (-12 - 3a) \\
&= -3a - 36 + 8 + 12 + 3a = -16
\end{align*}
所以$b=2,a=2-4=-2$
\end{solution}
\begin{example}{}{}
    已知$\lambda=1$是矩阵$\bm{A}=\begin{pmatrix}2&1&1\\1&2&t\\1&t&t+1\end{pmatrix}$的二重特征值,求$t$的值,并求是否存在可逆矩阵$\bm{T}$使得$\bm{T^{-1}AT}$为对角矩阵
\end{example}
\begin{solution}
    由于已知$\lambda=1$是矩阵$\bm{A}=\begin{pmatrix}2&1&1\\1&2&t\\1&t&t+1\end{pmatrix}$的二重特征值,所以直接计算$|\bm{A-E}|$的值:
    \begin{align*}
\det(\bm{A} - \bm{I}) &= 
\begin{vmatrix}
1 & 1 & 1 \\
1 & 1 & t \\
1 & t & t
\end{vmatrix} = 1 \cdot \begin{vmatrix} 1 & t \\ t & t \end{vmatrix} 
- 1 \cdot \begin{vmatrix} 1 & t \\ 1 & t \end{vmatrix} 
+ 1 \cdot \begin{vmatrix} 1 & 1 \\ 1 & t \end{vmatrix} \\
&= 1 \cdot (1 \cdot t - t \cdot t) - 1 \cdot (1 \cdot t - t \cdot 1) + 1 \cdot (1 \cdot t - 1 \cdot 1) \\
&= (t - t^2) - 0 + (t - 1) = -t^2 + 2t - 1 = -(t-1)^2
\end{align*}
必要条件是$\det(\bm{A-E})=0$,即$t=1$,当 $t=1$ 时,矩阵为:
\[
\bm{A} = \begin{pmatrix}
2 & 1 & 1 \\
1 & 2 & 1 \\
1 & 1 & 2
\end{pmatrix}
\]
计算特征多项式:
\begin{align*}
f(\lambda) &= \det(\bm{A} - \lambda \bm{I}) = 
\begin{vmatrix}
2-\lambda & 1 & 1 \\
1 & 2-\lambda & 1 \\
1 & 1 & 2-\lambda
\end{vmatrix}
=\begin{vmatrix}
4-\lambda & 4-\lambda & 4-\lambda \\
1 & 2-\lambda & 1 \\
1 & 1 & 2-\lambda
\end{vmatrix}
= (4-\lambda) \begin{vmatrix}
1 & 1 & 1 \\
1 & 2-\lambda & 1 \\
1 & 1 & 2-\lambda
\end{vmatrix}\\
&= (4-\lambda) \left[ 
\begin{vmatrix}
2-\lambda & 1 \\
1 & 2-\lambda
\end{vmatrix}
- \begin{vmatrix}
1 & 1 \\
1 & 2-\lambda
\end{vmatrix}
+ \begin{vmatrix}
1 & 2-\lambda \\
1 & 1
\end{vmatrix}
\right] \\
&= (4-\lambda) \left[ ((2-\lambda)^2 - 1) - ((2-\lambda) - 1) + (1 - (2-\lambda)) \right] \\
&= (4-\lambda) \left[ (\lambda^2 - 4\lambda + 3) - (1-\lambda) + (\lambda-1) \right] \\
&= (4-\lambda)(\lambda^2 - 4\lambda + 3+2\lambda-2)=(4-\lambda)(\lambda-1)^2
\end{align*}
因此特征值为 $\lambda = 1$(二重)和 $\lambda = 4$,验证了 $\lambda=1$ 是二重特征值。接下来算出$\bm{T}$,对于特征值 $\lambda = 1$,解方程组$(\bm{A} - \bm{I})\bm{x} = \bm{0}$:
\[
\bm{A} - \bm{I} = \begin{pmatrix}
1 & 1 & 1 \\
1 & 1 & 1 \\
1 & 1 & 1
\end{pmatrix} \xrightarrow{\text{行变换}} \begin{pmatrix}
1 & 1 & 1 \\
0 & 0 & 0 \\
0 & 0 & 0
\end{pmatrix}
\]
得到方程 $x_1 + x_2 + x_3 = 0$。选择两个线性无关的解:
\[
\bm{v}_1 = \begin{pmatrix} -1 \\ 1 \\ 0 \end{pmatrix}, \quad
\bm{v}_2 = \begin{pmatrix} -1 \\ 0 \\ 1 \end{pmatrix}
\]
因此$\lambda_1$的代数重数和几何重数相等,对于特征值 $\lambda = 4$,解方程组 $(\bm{A} - 4\bm{I})\bm{x} = \bm{0}$:
\[
\bm{A} - 4\bm{I} = \begin{pmatrix}
-2 & 1 & 1 \\
1 & -2 & 1 \\
1 & 1 & -2
\end{pmatrix} \xrightarrow{\text{行变换}} \begin{pmatrix}
1 & 0 & -1 \\
0 & 1 & -1 \\
0 & 0 & 0
\end{pmatrix}
\]
得到方程 $x_1 = x_3$, $x_2 = x_3$。取一个特解:
\[
\bm{v}_3 = \begin{pmatrix} 1 \\ 1 \\ 1 \end{pmatrix}
\]
令 $\bm{T} = (\bm{v}_1, \bm{v}_2, \bm{v}_3) = \begin{pmatrix} 1 & 1 & 1 \\ -1 & 0 & 1 \\ 0 & -1 & 1 \end{pmatrix}$。
且
$\det(\bm{T}) = \begin{vmatrix} 1 & 1 & 1 \\ -1 & 0 & 1 \\ 0 & -1 & 1 \end{vmatrix} 
=\begin{vmatrix} 3 & 2 & 1 \\ 0 & 1 & 1 \\ 0 & 0 & 1 \end{vmatrix} = 3 \neq 0$
可逆,满足:
\[
\bm{T}^{-1}\bm{A}\bm{T} = \begin{pmatrix} 1 & 0 & 0 \\ 0 & 1 & 0 \\ 0 & 0 & 4 \end{pmatrix}
\]
结果的三个对角线上的元素不用具体计算,因为$\lambda_1=\lambda_2=1,\lambda_3=4$.
\end{solution}