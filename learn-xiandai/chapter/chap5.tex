\chapter{特征值和特征向量}
\newpage
\begin{example}{求特征值和特征向量}{}
    (1)$\left(\begin{array}{rrr}3&2&-1\\-2&-2&2\\3&6&-1\end{array}\right)$\quad(2)$\begin{pmatrix}0&0&0&1\\0&0&1&0\\0&1&0&0\\1&0&0&0\end{pmatrix}$
\end{example}
\begin{solution}
(1)计算 $\det(A - \lambda E) = 0$:
\begin{align*}
\det(\bm{A - \lambda E}) &=\begin{vmatrix}
3-\lambda & 2 & -1 \\
-2 & -2-\lambda & 2 \\
3 & 6 & -1-\lambda
\end{vmatrix}\\
&= (3-\lambda)\begin{vmatrix} -2-\lambda & 2 \\ 6 & -1-\lambda \end{vmatrix} 
- 2\begin{vmatrix} -2 & 2 \\ 3 & -1-\lambda \end{vmatrix} 
+ (-1)\begin{vmatrix} -2 & -2-\lambda \\ 3 & 6 \end{vmatrix} \\
&= (3-\lambda)[(-2-\lambda)(-1-\lambda) - 12] - 2[(-2)(-1-\lambda) - 6] - 1[(-2)(6) - (-2-\lambda)(3)] \\
&= (3-\lambda)[(2+3\lambda+\lambda^2) - 12] - 2[2+2\lambda - 6] - 1[-12 + 6 + 3\lambda] \\
&= (3-\lambda)(\lambda^2 + 3\lambda - 10) - 2(2\lambda - 4) - (3\lambda - 6) \\
&= (3-\lambda)(\lambda-2)(\lambda+5) - 7(\lambda-2)=(\lambda-2)(15-7-2\lambda-\lambda^2)\\
&= (\lambda-2)(8-2\lambda-\lambda^2)=-(\lambda-2)^2(\lambda+4)=0
\end{align*}
所以特征值是$\lambda_1=2$(二重)和$\lambda_2=-4$(一重)\\
先计算矩阵$\bm{A-2E}$并进行行变换:
\[\bm{A-2E}=\begin{pmatrix}
1 & 2 & -1 \\
-2 & -4 & 2 \\
3 & 6 & -3
\end{pmatrix}\rightarrow\begin{pmatrix}
1 & 2 & -1 \\
0 & 0 & 0 \\
0 & 0 & 0
\end{pmatrix}\]
所以选取$x_2,x_3$为自由变量,$x_1=x_3-2x_2$,则有基向量:
\[x_2 = 1, x_3 = 0 \Rightarrow x_1 = -2,\mathbf{v}_1 = \begin{pmatrix} -2 \\ 1 \\ 0 \end{pmatrix}\quad x_2 = 0, x_3 = 1 \Rightarrow x_1 = 1,\mathbf{v}_2 = \begin{pmatrix} 1 \\ 0 \\ 1 \end{pmatrix}\]
则特征值$2$对应的特征向量为$k_1\bm{v_1}+k_2\bm{v_2}$,其中$k_1,k_2$为任意常数.\\
计算矩阵$\bm{A+4E}$并进行行变换:
\[\bm{A+4E}=\begin{pmatrix}
7 & 2 & -1 \\
-2 & 2 & 2 \\
3 & 6 & 3
\end{pmatrix}\rightarrow 
\begin{pmatrix}
7 & 2 & -1 \\
-2\times 7+7\times 2 & 2\times 7+2\times 2 & 2\times 7-2\times (-1) \\
3\times 7-7\times 3 & 6\times 7-2\times 3 & 3\times 7+1\times 3
\end{pmatrix}\]
得到$\begin{pmatrix}
7 & 2 & -1 \\
0 & 18 & 12 \\
0 & 36 & 24\end{pmatrix}\rightarrow \begin{pmatrix}
7 & 2 & -1 \\
0 & 3 & 2 \\
0 & 0 & 0\end{pmatrix}$
选取$x_3$为自由变量,解得$x_2 = -\frac{2}{3}x_3$,代入得 $x_1 = \frac{1}{3}x_3$
取$x_3 = 3$:$\mathbf{v}_3 = \begin{pmatrix} 1 \\ -2 \\ 3 \end{pmatrix}$,则特征值$-4$对应的特征向量为$k_3\begin{pmatrix} 1 \\ -2 \\ 3 \end{pmatrix}$,其中$k_3$为任意常数.\\
(2)计算 $\det(A - \lambda E) = 0$:
\begin{align*}
|\bm{B - \lambda E}| &= \begin{vmatrix}
-\lambda & 0 & 0 & 1 \\
0 & -\lambda & 1 & 0 \\
0 & 1 & -\lambda & 0 \\
1 & 0 & 0 & -\lambda
\end{vmatrix}= (-\lambda)\begin{vmatrix} -\lambda & 1 & 0 \\ 1 & -\lambda & 0 \\ 0 & 0 & -\lambda \end{vmatrix} 
- 0 + 0 
- 1\begin{vmatrix} 0 & -\lambda & 1 \\ 0 & 1 & -\lambda \\ 1 & 0 & 0 \end{vmatrix}\\
&= (-\lambda)^2[(-\lambda)(-\lambda) - 1\cdot1]-(\lambda^2-1)=\lambda^2(\lambda^2-1)-(\lambda^2-1)=(\lambda^2-1)^2
\end{align*}
特征值:$\lambda = 1$(二重根),$\lambda = -1$(二重根),对于 $\lambda = 1$,构造矩阵$\bm{A-E}$进行初等行变换:
\[\bm{A-E}=\begin{pmatrix}
-1 & 0 & 0 & 1 \\
0 & -1 & 1 & 0 \\
0 & 1 & -1 & 0 \\
1 & 0 & 0 & -1
\end{pmatrix}\rightarrow\begin{pmatrix}
-1 & 0 & 0 & 1 \\
0 & -1 & 1 & 0 \\
0 & 0 & 0 & 0 \\
0 & 0 & 0 & 0
\end{pmatrix}\rightarrow\begin{cases}x_2=x_3\\x_1=x_4\end{cases}\]
基向量:$x_3 = 1, x_4 = 0\Rightarrow\bm{\xi_1} = \begin{pmatrix} 0 \\ 1 \\ 1 \\ 0 \end{pmatrix}\quad x_3 = 0, x_4 = 1\Rightarrow\bm{\xi_2} = \begin{pmatrix} 1 \\ 0 \\ 0 \\ 1 \end{pmatrix}$\\
所以特征值$\lambda_1=1$对应的特征向量为$k_1\bm{\xi_1}+k_2\bm{\xi_2}$,其中$k_1,k_2$为任意常数。\\
对于 $\lambda = -1$,构造矩阵$\bm{A+E}$进行初等行变换:
\[\bm{A+E}=\begin{pmatrix}
1 & 0 & 0 & 1 \\
0 & 1 & 1 & 0 \\
0 & 1 & 1 & 0 \\
1 & 0 & 0 & 1
\end{pmatrix}\rightarrow\begin{pmatrix}
1 & 0 & 0 & 1 \\
0 & 1 & 1 & 0 \\
0 & 0 & 0 & 0 \\
0 & 0 & 0 & 0
\end{pmatrix}\rightarrow\begin{cases}x_2=-x_3\\x_1=-x_4\end{cases}\]
基向量:$x_3 = 1, x_4 = 0\Rightarrow\bm{\xi_3} = \begin{pmatrix} 0 \\ -1 \\ 1 \\ 0 \end{pmatrix}\quad x_3 = 0, x_4 = 1\Rightarrow\bm{\xi_4} = \begin{pmatrix} -1 \\ 0 \\ 0 \\ 1 \end{pmatrix}$\\
所以特征值$\lambda_1=-1$对应的特征向量为$k_3\bm{\xi_3}+k_4\bm{\xi_4}$,其中$k_3,k_4$为任意常数.
\end{solution}
\begin{example}{求该$n$阶矩阵的特征值和特征向量}{}
    \[\bm{A}=\begin{pmatrix}0&1&1&\cdots&1\\1&0&1&\cdots&1\\1&1&0&\cdots&1\\\vdots&\vdots&\vdots&&\vdots\\1&1&1&\cdots&0\end{pmatrix}\]
\end{example}
\begin{solution}
    计算 $\det(\bm{A - \lambda E}) = 0$:
    \begin{align*}
|\bm{A - \lambda E}| &= \begin{vmatrix}
-\lambda & 1 & 1 & \cdots & 1 \\
1 & -\lambda & 1 & \cdots & 1 \\
1 & 1 & -\lambda & \cdots & 1 \\
\vdots & \vdots & \vdots & \ddots & \vdots \\
1 & 1 & 1 & \cdots & -\lambda
\end{vmatrix}=\begin{vmatrix}
n-1-\lambda & n-1-\lambda & n-1-\lambda & \cdots & n-1-\lambda \\
1 & -\lambda & 1 & \cdots & 1 \\
1 & 1 & -\lambda & \cdots & 1 \\
\vdots & \vdots & \vdots & \ddots & \vdots \\
1 & 1 & 1 & \cdots & -\lambda
\end{vmatrix}\\
&=(n-1-\lambda)\begin{vmatrix}
1 & 1 & 1 & \cdots & 1 \\
1 & -\lambda & 1 & \cdots & 1 \\
1 & 1 & -\lambda & \cdots & 1 \\
\vdots & \vdots & \vdots & \ddots & \vdots \\
1 & 1 & 1 & \cdots & -\lambda
\end{vmatrix}=(n-1-\lambda)\begin{vmatrix}
1 & 1 & 1 & \cdots & 1 \\
0 & -\lambda-1 & 0 & \cdots & 0 \\
0 & 0 & -\lambda-1 & \cdots & 0 \\
\vdots & \vdots & \vdots & \ddots & \vdots \\
0 & 0 & 0 & \cdots & -\lambda-1
\end{vmatrix}\\
&=(n-1-\lambda)(-\lambda-1)^{n-1}=0\Rightarrow \lambda_1=n-1,\lambda_2=-1
\end{align*}
特征值为$\lambda_1=n-1$(一重)和$\lambda_2=-1$($n-1$重)\\
对于特征值$\lambda_1=n-1$,我们根据相同的行变换过程发现此时每一列的和为0,因此此时这个矩阵的所有行向量线性相关,所以只有一个有效方程,而且$x_1,x_2,\cdots,x_n$应该相等,严谨的说明过程则需要写出:
\[\begin{cases}-(n-1)x_1+x_2+x_3+\cdots+x_n=0\\x_1-(n-1)x_2+x_3+\cdots+x_n=0\\x_1+x_2-(n-1)x_3+\cdots+x_n=0\\\cdots\\x_1+x_2+x_3+\cdots-(n-1)x_n=0\end{cases}\]
设 $S = x_1 + x_2 + \cdots + x_n$,则第 $i$ 个方程为:
\[-(n-1)x_i + (S - x_i) = 0 \Rightarrow S = n x_i\]
这对所有 $i = 1, 2, \ldots, n$ 成立,因此:
\[n x_1 = n x_2 = \cdots = n x_n \Rightarrow x_1 = x_2 = \cdots = x_n\]
故特征向量为 $\bm{\xi} = k(1, 1, \ldots, 1)^T$,$k \neq 0$.\\
下面考虑另一个特征向量$\lambda_2=-1$,将刚才进行的行变换作用于矩阵:
\begin{align*}
\bm{A - \lambda E}&= \begin{pmatrix}
-\lambda & 1 & 1 & \cdots & 1 \\
1 & -\lambda & 1 & \cdots & 1 \\
1 & 1 & -\lambda & \cdots & 1 \\
\vdots & \vdots & \vdots & \ddots & \vdots \\
1 & 1 & 1 & \cdots & -\lambda
\end{pmatrix}\to\begin{pmatrix}
n-1-\lambda & n-1-\lambda & n-1-\lambda & \cdots & n-1-\lambda \\
1 & -\lambda & 1 & \cdots & 1 \\
1 & 1 & -\lambda & \cdots & 1 \\
\vdots & \vdots & \vdots & \ddots & \vdots \\
1 & 1 & 1 & \cdots & -\lambda
\end{pmatrix}\\
&\to\begin{pmatrix}
1 & 1 & 1 & \cdots & 1 \\
1 & -\lambda & 1 & \cdots & 1 \\
1 & 1 & -\lambda & \cdots & 1 \\
\vdots & \vdots & \vdots & \ddots & \vdots \\
1 & 1 & 1 & \cdots & -\lambda
\end{pmatrix}\to\begin{pmatrix}
1 & 1 & 1 & \cdots & 1 \\
0 & -\lambda-1 & 0 & \cdots & 0 \\
0 & 0 & -\lambda-1 & \cdots & 0 \\
\vdots & \vdots & \vdots & \ddots & \vdots \\
0 & 0 & 0 & \cdots & -\lambda-1
\end{pmatrix}
\end{align*}
代入$\lambda=-1$,则可得到$x_1+x_2+\cdots+x_n=0$,一组基可以取为:
\[\bm{\xi_1} = (1, -1, 0, 0, \ldots, 0)^T,\bm{\xi_2} = (1, 0, -1, 0, \ldots, 0)^T,\cdots,\bm{\xi_{n-1}}= (1, 0, 0, \ldots, 0, -1)^T\]
故特征向量为:
\[k_1\bm{\xi_1}+k_2\bm{\xi_2}+\cdots+k_{n-1}\bm{\xi_{n-1}},\quad k_1,k_2,\cdots,k_{n-1}\text{为任意常数}\]
\end{solution}
\begin{example}{}{}
    设$\lambda$是$n$阶可逆矩阵$A$的一个特征值,$\alpha$是$A$的属于特征值$\lambda$的一个特征向量,证明$\dfrac{|A|}{\lambda}$是$A^*$的一个特征值,$\alpha$也是$A^*$属于此特征值的一个特征向量.
\end{example}
\begin{solution}
    由于矩阵$\bm{A}$可逆,所以$\bm{AA^*}=|\bm{A}|\bm{E}$,设列向量$\bm{\alpha}$满足$\bm{A\alpha}=\lambda\bm{\alpha}$,则$\bm{A^*A\alpha}=\lambda\bm{A^*\alpha}$,即$|\bm{A}|\bm{E\alpha}=|\bm{A}|\bm{\alpha}=\lambda\bm{A^*\alpha}$,即$\dfrac{|\bm{A}|}{\lambda}\alpha=\bm{A^*\alpha}$,所以$\dfrac{|\bm{A}|}{\lambda}$是$\bm{A^*}$的一个特征值,$\bm{\alpha}$也是$\bm{A^*}$属于此特征值的一个特征向量.
\end{solution}
\begin{example}{}{}
    设$\lambda_1,\lambda_2$是矩阵$\bm{A}$的两个不同的特征值,$\bm{\alpha_1,\alpha_2}$分别是属于$\lambda_1,\lambda_2$的特征向量.试证:$\bm{\alpha}_1+\bm{\alpha}_2$一定不是$\bm{A}$的特征向量
\end{example}
\begin{solution}
    由定义得到$\bm{A\alpha_1}=\lambda_1\bm{\alpha_1},\bm{A\alpha_2}=\lambda_2\bm{\alpha_2}$,相加有$\bm{A(\alpha_1+\alpha_2)}=\lambda_1\bm{\alpha_1}+\lambda_2\bm{\alpha_2}$,假如$\bm{\alpha}_1+\bm{\alpha}_2$是$\bm{A}$的特征向量,即$\bm{A(\alpha_1+\alpha_2)}=\lambda_3(\bm{\alpha_1+\alpha_2})$,与$\lambda_1\bm{\alpha_1}+\lambda_2\bm{\alpha_2}$联立得到$(\lambda_3-\lambda_1)\bm{\alpha_1}+(\lambda_3-\lambda_2)\bm{\alpha_2}=0$,由$\bm{\alpha_1,\alpha_2}$线性无关,得到$\lambda_3=\lambda_1=\lambda_2$,但是$\lambda_1,\lambda_2$是矩阵$\bm{A}$的两个不同的特征值,因此矛盾,$\bm{\alpha}_1+\bm{\alpha}_2$一定不是$\bm{A}$的特征向量.
\end{solution}
\begin{example}{}{}
    $\text{矩阵}\bm{A}=\begin{pmatrix}{3}&{2}&{-1}\\{-2}&{a}&{2}\\{3}&{6}&{-1}\end{pmatrix}\text{与矩阵}\bm{B}=\begin{pmatrix}{-4}&{0}&{0}\\{0}&{b}&{0}\\{0}&{0}&{2}\end{pmatrix}\text{相似},\text{试求}a,b\text{的值}.$
\end{example}
\begin{solution}
    由于矩阵 $\bm{A}$ 与 $\bm{B}$ 相似,它们具有相同的特征值。矩阵 $\bm{B}$ 是对角矩阵,其特征值即为对角线元素:$-4$、$b$ 和 $2$。因此,$\bm{A}$ 的特征值也应为 $-4$、$b$ 和 $2$,而且由相似矩阵的迹相等得到$3 + a + (-1) = a + 2 = -4 + b + 2 = b - 2\Rightarrow b = a + 4$,由特征多项式韦达定理得到$\bm{A}$的所有特征值的积为$|\bm{A}|=-4\times b\times 2=-8b$,下面计算行列式(按行展开):
    \begin{align*}
\det(\bm{A}) &= 3 \cdot 
\begin{vmatrix}
a & 2 \\
6 & -1
\end{vmatrix}
- 2 \cdot 
\begin{vmatrix}
-2 & 2 \\
3 & -1
\end{vmatrix}
+ (-1) \cdot 
\begin{vmatrix}
-2 & a \\
3 & 6
\end{vmatrix} \\
&= 3 \cdot (a \cdot (-1) - 2 \cdot 6) - 2 \cdot ((-2) \cdot (-1) - 2 \cdot 3) - 1 \cdot ((-2) \cdot 6 - a \cdot 3) \\
&= 3 \cdot (-a - 12) - 2 \cdot (2 - 6) - 1 \cdot (-12 - 3a) \\
&= -3a - 36 + 8 + 12 + 3a = -16
\end{align*}
所以$b=2,a=2-4=-2$
\end{solution}
\begin{example}{}{}
    已知$\lambda=1$是矩阵$\bm{A}=\begin{pmatrix}2&1&1\\1&2&t\\1&t&t+1\end{pmatrix}$的二重特征值,求$t$的值,并求是否存在可逆矩阵$\bm{T}$使得$\bm{T^{-1}AT}$为对角矩阵
\end{example}
\begin{solution}
    由于已知$\lambda=1$是矩阵$\bm{A}=\begin{pmatrix}2&1&1\\1&2&t\\1&t&t+1\end{pmatrix}$的二重特征值,所以直接计算$|\bm{A-E}|$的值:
    \begin{align*}
\det(\bm{A} - \bm{I}) &= 
\begin{vmatrix}
1 & 1 & 1 \\
1 & 1 & t \\
1 & t & t
\end{vmatrix} = 1 \cdot \begin{vmatrix} 1 & t \\ t & t \end{vmatrix} 
- 1 \cdot \begin{vmatrix} 1 & t \\ 1 & t \end{vmatrix} 
+ 1 \cdot \begin{vmatrix} 1 & 1 \\ 1 & t \end{vmatrix} \\
&= 1 \cdot (1 \cdot t - t \cdot t) - 1 \cdot (1 \cdot t - t \cdot 1) + 1 \cdot (1 \cdot t - 1 \cdot 1) \\
&= (t - t^2) - 0 + (t - 1) = -t^2 + 2t - 1 = -(t-1)^2
\end{align*}
必要条件是$\det(\bm{A-E})=0$,即$t=1$,当 $t=1$ 时,矩阵为:
\[
\bm{A} = \begin{pmatrix}
2 & 1 & 1 \\
1 & 2 & 1 \\
1 & 1 & 2
\end{pmatrix}
\]
计算特征多项式:
\begin{align*}
f(\lambda) &= \det(\bm{A} - \lambda \bm{I}) = 
\begin{vmatrix}
2-\lambda & 1 & 1 \\
1 & 2-\lambda & 1 \\
1 & 1 & 2-\lambda
\end{vmatrix}
=\begin{vmatrix}
4-\lambda & 4-\lambda & 4-\lambda \\
1 & 2-\lambda & 1 \\
1 & 1 & 2-\lambda
\end{vmatrix}
= (4-\lambda) \begin{vmatrix}
1 & 1 & 1 \\
1 & 2-\lambda & 1 \\
1 & 1 & 2-\lambda
\end{vmatrix}\\
&= (4-\lambda) \left[ 
\begin{vmatrix}
2-\lambda & 1 \\
1 & 2-\lambda
\end{vmatrix}
- \begin{vmatrix}
1 & 1 \\
1 & 2-\lambda
\end{vmatrix}
+ \begin{vmatrix}
1 & 2-\lambda \\
1 & 1
\end{vmatrix}
\right] \\
&= (4-\lambda) \left[ ((2-\lambda)^2 - 1) - ((2-\lambda) - 1) + (1 - (2-\lambda)) \right] \\
&= (4-\lambda) \left[ (\lambda^2 - 4\lambda + 3) - (1-\lambda) + (\lambda-1) \right] \\
&= (4-\lambda)(\lambda^2 - 4\lambda + 3+2\lambda-2)=(4-\lambda)(\lambda-1)^2
\end{align*}
因此特征值为 $\lambda = 1$(二重)和 $\lambda = 4$,验证了 $\lambda=1$ 是二重特征值。接下来算出$\bm{T}$,对于特征值 $\lambda = 1$,解方程组$(\bm{A} - \bm{I})\bm{x} = \bm{0}$:
\[
\bm{A} - \bm{I} = \begin{pmatrix}
1 & 1 & 1 \\
1 & 1 & 1 \\
1 & 1 & 1
\end{pmatrix} \xrightarrow{\text{行变换}} \begin{pmatrix}
1 & 1 & 1 \\
0 & 0 & 0 \\
0 & 0 & 0
\end{pmatrix}
\]
得到方程 $x_1 + x_2 + x_3 = 0$。选择两个线性无关的解:
\[
\bm{v}_1 = \begin{pmatrix} -1 \\ 1 \\ 0 \end{pmatrix}, \quad
\bm{v}_2 = \begin{pmatrix} -1 \\ 0 \\ 1 \end{pmatrix}
\]
因此$\lambda_1$的代数重数和几何重数相等,对于特征值 $\lambda = 4$,解方程组 $(\bm{A} - 4\bm{I})\bm{x} = \bm{0}$:
\[
\bm{A} - 4\bm{I} = \begin{pmatrix}
-2 & 1 & 1 \\
1 & -2 & 1 \\
1 & 1 & -2
\end{pmatrix} \xrightarrow{\text{行变换}} \begin{pmatrix}
1 & 0 & -1 \\
0 & 1 & -1 \\
0 & 0 & 0
\end{pmatrix}
\]
得到方程 $x_1 = x_3$, $x_2 = x_3$。取一个特解:
\[
\bm{v}_3 = \begin{pmatrix} 1 \\ 1 \\ 1 \end{pmatrix}
\]
令 $\bm{T} = (\bm{v}_1, \bm{v}_2, \bm{v}_3) = \begin{pmatrix} 1 & 1 & 1 \\ -1 & 0 & 1 \\ 0 & -1 & 1 \end{pmatrix}$。
且
$\det(\bm{T}) = \begin{vmatrix} 1 & 1 & 1 \\ -1 & 0 & 1 \\ 0 & -1 & 1 \end{vmatrix} 
=\begin{vmatrix} 3 & 2 & 1 \\ 0 & 1 & 1 \\ 0 & 0 & 1 \end{vmatrix} = 3 \neq 0$
可逆,满足:
\[
\bm{T}^{-1}\bm{A}\bm{T} = \begin{pmatrix} 1 & 0 & 0 \\ 0 & 1 & 0 \\ 0 & 0 & 4 \end{pmatrix}
\]
结果的三个对角线上的元素不用具体计算,因为$\lambda_1=\lambda_2=1,\lambda_3=4$.
\end{solution}
\newpage
\begin{example}{}{}
    求出能使下列矩阵 $\bm{A}$ 相似于对角矩阵的正交矩阵 $\bm{T}$ :
$$(1) \bm{A} = \left(\begin{array}{lll}
6 & 2 & 4 \\
2 & 3 & 2 \\
4 & 2 & 6
\end{array}\right) ; (2) \bm{A} = \left.A=\left(\begin{array}{rrrr}4&-1&-1&1\\-1&4&1&-1\\-1&1&4&-1\\1&-1&-1&4\end{array}\right.\right)$$
\end{example}
\begin{solution}
(1) 对于矩阵 $\bm{A} = \begin{pmatrix} 6 & 2 & 4 \\ 2 & 3 & 2 \\ 4 & 2 & 6 \end{pmatrix}$,特征多项式为:
\[
|\lambda E - A| = \begin{vmatrix}
\lambda-6 & -2 & -4 \\
-2 & \lambda-3 & -2 \\
-4 & -2 & \lambda-6
\end{vmatrix} =\begin{vmatrix}
\lambda-6 & -2 & -4 \\
-2 & \lambda-3 & -2 \\
0 & 4-2\lambda & \lambda-2
\end{vmatrix}= (\lambda-2)^2(\lambda-11) = 0
\]
得特征值:$\lambda_1 = \lambda_2 = 2$(二重),$\lambda_3 = 11$,对于 $\lambda = 2$,解 $(\bm{A}-2\bm{E})\bm{x} = \bm{O}$:
\[
A-2E = \begin{pmatrix} 4 & 2 & 4 \\ 2 & 1 & 2 \\ 4 & 2 & 4 \end{pmatrix} \xrightarrow{\text{行化简}} \begin{pmatrix} 2 & 1 & 2 \\ 0 & 0 & 0 \\ 0 & 0 & 0 \end{pmatrix}
\]
得基础解系:$\bm{\alpha}_1 = (-1, 2, 0)^T$,$\bm{\alpha}_2 = (0, 2, -1)^T$,对于 $\lambda = 11$,解 $(\bm{A}-11\bm{E})\bm{x} = \bm{O}$:
\[
\bm{A}-11\bm{E} = \begin{pmatrix} -5 & 2 & 4 \\ 2 & -8 & 2 \\ 4 & 2 & -5 \end{pmatrix} \xrightarrow{\text{行变换}} \begin{pmatrix} -5 & 2 & 4 \\ 1 & -2 & 0 \\ 0 & 2 & -1 \end{pmatrix}
\]
此时已经很容易解得特征向量为:$\bm{\alpha}_3 = (2, 1, 2)^T$。对 $\lambda = 2$ 对应的特征向量进行正交化:取 $\bm{\beta}_1 = \bm{\alpha}_1 = (-1, 2, 0)^T$:
\[\bm{\beta}_2 = \bm{\alpha}_2 - \frac{(\bm{\alpha}_2, \bm{\beta}_1)}{(\bm{\beta}_1, \bm{\beta}_1)}\bm{\beta}_1 = (0, 2,-1)^T - \frac{4}{5}(-1, 2, 0)^T = (\frac{4}{5}, \frac{2}{5}, -1)^T\]
取 $\bm{\beta}_2 = (-4, -2, 5)^T$(乘以5简化)
单位化并构造正交矩阵:
\[
\bm{T} = \begin{pmatrix}
-\frac{1}{\sqrt{5}} & \frac{4}{3\sqrt{5}} & \frac{2}{3} \\
\frac{2}{\sqrt{5}} & \frac{2}{3\sqrt{5}} & \frac{1}{3} \\
0 & -\frac{5}{3\sqrt{5}} & \frac{2}{3}
\end{pmatrix}
\]
(2) 对于矩阵 $\bm{A} = \begin{pmatrix} 4 & -1 & -1 & 1 \\ -1 & 4 & 1 & -1 \\ -1 & 1 & 4 & -1 \\ 1 & -1 & -1 & 4 \end{pmatrix}$,下面证明一个引理:
如果矩阵 $\bm{A}$ 的每一行元素之和都等于同一个常数 $c$,那么 $c$ 是 $\bm{A}$ 的一个特征值。设 $\bm{A} = (a_{ij})$ 是一个 $n \times n$ 矩阵,且满足:
$$\sum_{j=1}^n a_{ij} = c \quad \text{对于所有 } i = 1, 2, \ldots, n$$
考虑向量 $\bm{v} = ( 1 , 1 , \cdots,1 )^{\bm{T}}$(全1向量)。计算矩阵乘积 $\bm{A}\bm{v}$ 的第 $i$ 个分量:
\[(\bm{A}\bm{v})_i = \sum_{j=1}^n a_{ij} \cdot v_j = \sum_{j=1}^n a_{ij} \cdot 1 = \sum_{j=1}^n a_{ij} = c \]
因此,对于所有 $i = 1, 2, \ldots, n$,都有 $(\bm{A}\bm{v})_i = c$,这意味着:$\bm{A}\bm{v} = c\bm{v}$,根据特征值和特征向量的定义,上式 $\bm{A}\bm{v} = c\bm{v}$ 表明 $c$ 是 $\bm{A}$ 的特征值,$\bm{v}$ 是对应的特征向量。\\
而矩阵 $\bm{A}$ 的每一行元素之和均为:
$$\sum_{j=1}^4 a_{ij} = 4 + (-1) + (-1) + 1 = 3 \quad \text{对于所有 } i = 1,2,3,4$$
因此,$\lambda = 3$ 是 $\bm{A}$ 的一个特征值,对应的特征向量为 $\bm{v} = (1,1,1,1)^T$。将 $\bm{A}$ 分解为:$\bm{A} = 3\bm{E} + \bm{C}$,其中 $\bm{E}$ 是 $4 \times 4$ 单位矩阵,$\bm{C} = \bm{A} - 3\bm{E}$。计算 $\bm{C}$:
$$\bm{C} = \begin{pmatrix} 
1 & -1 & -1 & 1 \\ 
-1 & 1 & 1 & -1 \\ 
-1 & 1 & 1 & -1 \\ 
1 & -1 & -1 & 1 
\end{pmatrix}$$
显然,$\text{rank}(\bm{C}) = 1$,即 $\bm{C}$ 是秩1矩阵。一个非零特征值:$\lambda_C = \text{tr}(\bm{C}) = 1 + 1 + 1 + 1 = 4$,三个零特征值:$0, 0, 0$,由于 $\bm{A} = 3\bm{E} + \bm{C}$,且矩阵加法保持特征向量的对应关系,有:$\lambda_A = 3 + \lambda_C$;因此:当 $\lambda_C = 4$ 时,$\lambda_A = 3 + 4 = 7$,当 $\lambda_C = 0$ 时,$\lambda_A = 3 + 0 = 3$;由于 $\bm{C}$ 有一个非零特征值和三个零特征值,$\bm{A}$ 对应地有一个特征值 $7$ 和三个特征值 $3$。\\
对于 $\lambda = 3$,解 $(\bm{A}-3\bm{E})\bm{x}=\bm{Cx} = \bm{0}$:
得基础解系:$\bm{\alpha}_1 = (1, 1, 0, 0)^T$,$\bm{\alpha}_2 = (1, 0, 1, 0)^T$,$\bm{\alpha}_3 = (-1, 0, 0, 1)^T$;对于 $\lambda = 7$,解 $(\bm{A}-7\bm{E})\bm{x} = \bm{0}$,得特征向量:$\bm{\alpha}_4 = (-1, 1, 1, -1)^T$
对 $\lambda = 3$ 对应的特征向量进行正交化:取 $\bm{\beta}_1 = \bm{\alpha}_1 = (1, 1, 0, 0)^T$
\[\bm{\beta}_2 = \bm{\alpha}_2 - \frac{(\bm{\alpha}_2, \bm{\beta}_1)}{(\bm{\beta}_1, \bm{\beta}_1)}\bm{\beta}_1 = (1, 0, 1, 0)^T - \frac{1}{2}(1, 1, 0, 0)^T = (\frac{1}{2}, -\frac{1}{2}, 1, 0)^T\]
取 $\bm{\beta}_2 = (1, -1, 2, 0)^T$(乘以2简化)
\[\bm{\beta}_3 = \bm{\alpha}_3 - \frac{(\bm{\alpha}_3, \bm{\beta}_1)}{(\bm{\beta}_1, \bm{\beta}_1)}\bm{\beta}_1 - \frac{(\bm{\alpha}_3, \bm{\beta}_2)}{(\bm{\beta}_2, \bm{\beta}_2)}\bm{\beta}_2\]
计算得 $\bm{\beta}_3 = (-1, 1, 1, 3)^T$,构造正交矩阵
\[
\bm{T} = \begin{pmatrix}
\frac{1}{\sqrt{2}} & \frac{1}{\sqrt{6}} & -\frac{1}{2\sqrt{3}} & -\frac{1}{2} \\
\frac{1}{\sqrt{2}} & -\frac{1}{\sqrt{6}} & \frac{1}{2\sqrt{3}} & \frac{1}{2} \\
0 & \frac{2}{\sqrt{6}} & \frac{1}{2\sqrt{3}} & \frac{1}{2} \\
0 & 0 & \frac{3}{2\sqrt{3}} & -\frac{1}{2}
\end{pmatrix}
\]
\end{solution}
\begin{example}{}{}
试证明:设$\bm{A}$为$n$阶实对称矩阵,且$\bm{A^2}=\bm{A}$,则存在正交矩阵$\bm{T}$,使得
$$\bm{T^{-1}AT}=\begin{pmatrix}\bm{E_r}&\mathbf{0}\\\mathbf{0}&\mathbf{0}\end{pmatrix},$$
其中$r$为秩,$\bm{E_r}$为$r$阶单位矩阵。
\end{example}
\begin{solution}
由于 $\bm{A}$ 是实对称矩阵,根据实对称矩阵的正交相似对角化定理,存在正交矩阵 $\bm{T}$,使得
\[
\bm{T}^{-1}\bm{A}\bm{T} = \operatorname{diag}(\lambda_1, \lambda_2, \ldots, \lambda_n),
\]
其中 $\lambda_1, \lambda_2, \ldots, \lambda_n$ 是 $\bm{A}$ 的特征值。又已知 $\bm{A}^2 = \bm{A}$,考虑正交相似变换:
\[
(\bm{T}^{-1}\bm{A}\bm{T})^2 = \bm{T}^{-1}\bm{A}^2\bm{T} = \bm{T}^{-1}\bm{A}\bm{T}\Leftrightarrow \operatorname{diag}(\lambda_1^2, \lambda_2^2, \ldots, \lambda_n^2) = \operatorname{diag}(\lambda_1, \lambda_2, \ldots, \lambda_n)
\]
因此,对于每个 $i$,有 $\lambda_i^2 = \lambda_i$,解得 $\lambda_i = 0$ 或 $\lambda_i = 1$。
由于 $\bm{A}$ 的秩为 $r$,而实对称矩阵的秩等于其非零特征值的个数,故特征值 $1$ 的个数为 $r$。通过适当排列正交矩阵 $\bm{T}$ 的列向量(即特征向量的顺序),可以使特征值 $1$ 位于前 $r$ 个位置,于是
\[
\bm{T}^{-1}\bm{A}\bm{T} = \begin{pmatrix} \bm{E}_r & \mathbf{0} \\ \mathbf{0} & \mathbf{0} \end{pmatrix},
\]
\end{solution}
\begin{example}{}{}
    $\text{设 }A\text{ 为 }n\text{ 阶方阵},\text{ 且 }A^2=A,\text{ 证明若 }\lambda\text{ 为 }A\text{ 的一个特征值},\text{则 }\lambda=0\text{ 或 }1.$
\end{example}
\begin{solution}
设 $\lambda$ 为 $A$ 的任意一个特征值,对应的特征向量为 $\bm{\alpha} \neq \bm{0}$,即
\[
\bm{A}\bm{\alpha} = \lambda \bm{\alpha}\Rightarrow \bm{A}^2\bm{\alpha} = \bm{A}(\lambda \bm{\alpha}) = \lambda\bm{A}\bm{\alpha} = \lambda^2 \bm{\alpha}
\]
由于 $A^2 = A$,有 $A^2\bm{\alpha} = A\bm{\alpha} = \lambda \bm{\alpha}$,因此
\[
\lambda^2 \bm{\alpha} = \lambda \bm{\alpha}.
\]
因为 $\bm{\alpha} \neq \bm{0}$,所以 $\lambda^2 = \lambda$,即 $\lambda(\lambda - 1) = 0$,解得 $\lambda = 0$ 或 $\lambda = 1$。由于 $\lambda$ 是 $A$ 的任意特征值,故 $A$ 的所有特征值都是 $0$ 或 $1$。
\end{solution}
\begin{example}{}{}
    $\text{设 }A\text{ 为 }2\times2\text{ 矩阵, 若 tr}\left(A\right)=8\text{ 且 }\left|A\right|=12,\text{ 求 }A\text{ 的特征值}.$
\end{example}
\begin{solution}
用定义,设$|\bm{A}-\lambda\bm{E}|=\begin{vmatrix}a-\lambda&b\\c&d-\lambda\end{vmatrix}=(ad-bc)-\lambda(a+d)+\lambda^2$,则代入数据得到$\lambda_1=2,\lambda_2=6$,因此$A$的特征值为$2$或$6$。
\end{solution}
\begin{example}{}{}
    设$\bm{A,B}$为两个$n$阶方阵,证明:\\
(1)若$\lambda$为$\bm{AB}$的一个非零特征值,则它也是$BA$的一个特征值;\\
(2)若$\lambda=0$为$\bm{AB}$的一个特征值,则$\lambda=0$也是$\bm{BA}$的一个特征值
\end{exmaple}
\begin{solution}
    (1)由定义,$\bm{AB}$的特征值$\lambda\ne0$与其对应的特征向量$\bm{\alpha}$满足$AB=\lambda\bm{E}_n$,
\end{solution}
\begin{example}{}{}
    设$A,B$为两个$n$阶方阵,证明若存在同一个可逆矩阵$P$同时对角化$A$和$B$,则有$AB=BA.$
\end{exmaple}

