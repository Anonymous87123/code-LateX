\chapter{空间向量}
\begin{example}{16}{}
    $\text{求过点 }M_0(2,9,-6)\text{ 且与连接坐标原点 }O\text{ 及点 }M_0\text{ 的线段 }OM_0\text{ 垂直的平面方程}.$
\end{example}
\begin{solution}
    根据题意,$\overrightarrow{OM_0}=(2,9,-6)$为平面法向量,设平面方程为$2x+9y-6z+D=0$,代入$M_0=(2,9,-6)$,有$2(2)+9(9)-6(-6)+D=0$,即$D=-121$,即$2x+9y-6z-121=0$为平面方程.
\end{solution}
\begin{example}{18}{}
    $\text{求过点 }(3,0,-1)\text{ 且与平面 }3x-7y+5z-12=0\text{ 平行的平面方程}.$
\end{example}
\begin{solution}
    设平面方程为$3x-7y+5z+D=0$,代入$(3,0,-1)$,有$3(3)-7(0)+5(-1)+D=0\Rightarrow D=-4$,即$3x-7y+5z-4=0$为平面方程.
\end{solution}
\begin{example}{20}{}
    $\text{求过点 }(4,0,-2)\text{ 和点 }(5,1,7)\text{ 且平行于 }x\text{ 轴的平面方程}.$
\end{example}
\begin{solution}
    设该平面的法向量为$\vecc{n}=(0,B,C)$,则平面方程可以写为$By+Cz+D=0$,代入$(4,0,-2)$和$(5,1,7)$,有$B=-\dfrac{9D}2$,$C=\dfrac{D}2$,取$D=-2,B=9,C=-1$,即$9y-z-2=0$为平面方程.
\end{solution}
\begin{example}{22}{}
    平面$x-2y+3z+D=0,-2x+4y+Cz+6=0$,分别求出当两个平面平行和重合时的$C,D$值.
\end{example}
\begin{solution}
    当两个平面重合,则它们等价,则$C=-6,D=-3$;若两个平面平行,则$C=-6,D\neq -3$;
\end{solution}
\begin{example}{25}{}
    求出经过点$P(3,1,2)$且平行于平面$x+y+z+3=0,y-z+1=0$的直线的对称式方程.
\end{example}
\begin{solution}
    直线的方向向量为$\vecc{n}=(a,b,c)$,则$\vecc{n}\cdot(1,1,1)=0,\vecc{n}\cdot(0,1,-1)=0$解得直线的方向向量为$\vecc{n}=(-2,1,1)$,则其对称式方程为$\dfrac{x-3}{-2}=\dfrac{y-1}{1}=\dfrac{z-1}{1}$.
\end{solution}
\begin{example}{27}{}
    (1)求过直线$\begin{cases}x=2+3t\\y=2+t\\z=1+2t\end{cases}$和点$(1,2,-1)$的平面方程.\\(2)求过直线$\begin{cases}x+3y-z=0\\x-y+z+1=0\end{cases}$且与平面$x+2z=1$垂直的平面方程.
\end{example}
\begin{solution}
    (1)该直线的方向向量为$\vecc{n_1}=(3,1,2)$,过点$A(2,2,1)$,设$B(1,2,-1)$,则$\vecc{AB}=(-1,0,-2)$,设平面的法向量为$\vecc{n}$,那么$\vecc{n}\cdot\vecc{AB}=0,\vecc{n}\cdot\vecc{{n_1}=0}$,解得$n=(-2,4,1)$,则$-2x+4y+z+D=0$为平面方程,代入点$A(2,2,1)$得到$D=-5$,所以$2x-4y-z+5=0$为平面方程.\\
    (2)设平面方程为$x+3y-z+\lambda(x-y+z+1)=0$,即\[(1+\lambda)x+(3-\lambda)y+(\lambda-1)z+\lambda=0\]法向量为$\vecc{n}=(1+\lambda,3-\lambda,\lambda-1)$,垂直于$\vecc{n_1}=(1,0,2)$,解得$\lambda=\dfrac13$,则平面方程为$4x+8y-2z+1=0$
\end{solution}
\begin{example}{28}{}
    证明下列两条直线 $l_{1}$ 和 $l_{2}$ 共面,并求它们所在的平面的方程.\\
$l_{1}:\dfrac{x-7}{3}=\dfrac{y-2}{2}=\dfrac{z-1}{-2}, \quad l_{2}:\begin{cases} x=1+2t, \\ y=-2-3t, \\ z=5+4t. \end{cases}$
\end{example}
\begin{solution}
    将$l_1$化为参数方程$\begin{cases}x=7+3u\\y=2+2u\\z=1-2u\end{cases}$,与$l_2$的参数方程联立得到$t=0,u=-2$,所以两条直线的交点为$(1,-2,5)$,所以两条直线 $l_{1}$ 和 $l_{2}$ 共面,设出平面的法向量为$\vecc{n},\vecc{n}\cdot(2,-3,4)=0,\vecc{n}\cdot(3,2,-2)=0$,解得$\vecc{n}=(2,-16,-13)$,则$2x-16y-13z+D=0$为平面方程,代入$(1,-2,5)$得到$D=31$,所以$2x-16y-13z+31=0$为平面方程.
\end{solution}