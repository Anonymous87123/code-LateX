\chapter{二次型的标准形与规范形}
\begin{example}{}{}
用配方法把下列二次型化成标准形:$x_{1}x_{2}+x_{1}x_{3}+x_{1}x_{4}+x_{2}x_{4}$
\end{example}
\begin{solution}
先用平方差公式代换,让式子中出现平方项,再根据系数配平方:设$2y_1=x_1+x_2,2y_2=x_1-x_2$,则有:
\begin{align*}
    &x_{1}x_{2}+x_{1}x_{3}+x_{1}x_{4}+x_{2}x_{4}=\dfrac{y_1^2}{4}-\dfrac{y_2^2}4+\dfrac{y_1+y_2}2x_3+y_1x_4\\
    =&\dfrac{y_1^2}4+y_1\left(\dfrac{x_3}{2}+x_4\right)-\dfrac{y_2^2}4+\dfrac{x_3y_2}2\\
    =&\left(\dfrac{y_1}{2}+\dfrac{x_3}{2}+x_4\right)^2-\left(\dfrac{x_3}{2}+x_4\right)^2-\left(\dfrac{y_2}{2}-\dfrac{x_3}2\right)^2+\left(\dfrac{x_3}2\right)^2\\
    =&z_1^2-z_2^2-z_3^2+z_4^2
\end{align*}
其中$z_1=\dfrac{y_1}{2}+\dfrac{x_3}{2}+x_4,z_2=\dfrac{x_3}{2}+x_4,z_3=\dfrac{y_2}{2}-\dfrac{x_3}{2},z_4=\dfrac{x_3}{2}$.
\end{solution}
\begin{example}{}{}
    用正交变换把下列实二次型化成标准形,并写出所作的正交变换:$2x_{1}x_{3}+x_{2}^{2}$
\end{example}
\begin{solution}将该二次型简写成矩阵乘积的形式:
\[2x_{1}x_{3}+x_{2}^{2}=(x_1,x_2,x_3)\begin{pmatrix}0&0&1\\0&1&0\\1&0&0\end{pmatrix}(x_1,x_2,x_3)^T\]
设$\bm{A}=\begin{pmatrix}0&0&1\\0&1&0\\1&0&0\end{pmatrix}$,则
\[
|\bm{A-\lambda\bm{E}}|=\begin{vmatrix}-\lambda&0&1\\0&1-\lambda&0\\1&0&-\lambda\end{vmatrix}=\begin{vmatrix}
0&0&1-\lambda^2\\0&1-\lambda&0\\1&0&-\lambda
\end{vmatrix}=-(\lambda-1)^2(\lambda+1)
\]
得到特征值$\lambda_1=1,\lambda_2=-1$,令$\lambda=1$,则
\[\begin{pmatrix}
    -1&0&1\\0&0&0\\1&0&-1
\end{pmatrix}\Rightarrow\begin{cases}x_1=x_3\\x_2\text{为自由变量}\end{cases}\Rightarrow\bm{v_1}=\begin{pmatrix}1\\0\\1\end{pmatrix},\bm{v_2}=\begin{pmatrix}0\\1\\0\end{pmatrix}\]
其中$\bm{v_1,v_2}$为基础解系,显然$(\bm{v_1},\bm{v_2})=0$.令$\lambda=-1$,则
\[\begin{pmatrix}
    1&0&1\\0&2&0\\1&0&1
\end{pmatrix}\Rightarrow\begin{cases}x_1=-x_3\\x_2=0\end{cases}\Rightarrow\bm{v_3}=\begin{pmatrix}-1\\0\\1\end{pmatrix}\]
则$\bm{v_3}$为基础解系,则将$\bm{v_1,v_2,v_3}$单位化,写出矩阵
$\bm{P}=\left(\dfrac1{\sqrt2}\bm{v_1},\bm{v_2},\dfrac1{\sqrt2}\bm{v_3}\right)$,令$\bm{X=PY}$得到规范型:$-y_1^2+y_2^2+y_3^2$
\end{solution}
\begin{example}{}{}
    求下列实二次型的正、负惯性指数:$x_1^2+ x_2^2+ 3x_3^2+ 4x_1x_2+ 2x_1x_3+ 2x_2x_3;$
\end{example}
\begin{solution}
将该二次型化成标准形:
\begin{align*}
    f(x_1,x_2,x_3)=&x_1^2+ x_2^2+ 3x_3^2+ 4x_1x_2+ 2x_1x_3+ 2x_2x_3\\
    =&(x_1,x_2,x_3)\begin{pmatrix}
    1&2&1\\2&1&1\\1&1&3
    \end{pmatrix}(x_1,x_2,x_3)^T\\
    =&(x_1,x_2,x_3)\bm{A}(x_1,x_2,x_3)^T\\
|\bm{A}-\lambda\bm{E}|=&\begin{vmatrix}
    1-\lambda&2&1\\2&1-\lambda&1\\1&1&3-\lambda
\end{vmatrix}=\begin{vmatrix}
    0&\lambda+1&-\lambda^2+4\lambda-2\\0&-1-\lambda&2\lambda-5\\1&1&3-\lambda
\end{vmatrix}\\
=&(\lambda+1)(\lambda^2-6\lambda+7)
\end{align*}
特征值为$\lambda_1=-1$ ,$\lambda_2=3+\sqrt{2}$ ,$\lambda_3=3-\sqrt{2}$。由于$3+\sqrt{2}>0$和$3-\sqrt{2}>0$ ,正特征值个数为 2,负特征值个数为 1。因
此,二次型的标准形可通过正交变换化为:
$$-1\cdot y_1^2+(3+\sqrt{2})y_2^2+(3-\sqrt{2})y_3^2$$
正惯性指数为 2,负惯性指数为 1。
\end{solution}
\begin{example}{}{}
    判断下列实二次型是不是正定二次型,或在何种条件下是正定二次型:
\[f= x_{1}^{2}+ x_{2}^{2}+ 5x_{3}^{2}+ 2\lambda x_{1}x_{2}- 2x_{1}x_{3}+ 4x_{2}x_{3}\]
\end{example}
\begin{solution}
    给定二次型:
\[
f = x_1^2 + x_2^2 + 5x_3^2 + 2\lambda x_1 x_2 - 2x_1 x_3 + 4x_2 x_3
\]

对应的对称矩阵为:
\[
A = \begin{pmatrix}
1 & \lambda & -1 \\
\lambda & 1 & 2 \\
-1 & 2 & 5
\end{pmatrix}
\]

二次型正定的充要条件是对称矩阵 \(A\) 的所有顺序主子式大于零。

1. 第一顺序主子式:$\Delta_1 = 1 > 0$恒成立。

2. 第二顺序主子式:$\Delta_2 = \begin{vmatrix}
1 & \lambda \\\lambda & 1\end{vmatrix} = 1 - \lambda^2
$,要求 \(\Delta_2 > 0\),即:$1 - \lambda^2 > 0 \quad \Rightarrow \quad -1 < \lambda < 1$

3. 第三顺序主子式(矩阵行列式):
\[
\begin{aligned}
\Delta_3 &= \det(A) = \begin{vmatrix}
1 & \lambda & -1 \\
\lambda & 1 & 2 \\
-1 & 2 & 5
\end{vmatrix} = 1 \cdot \begin{vmatrix} 1 & 2 \\ 2 & 5 \end{vmatrix} 
   - \lambda \cdot \begin{vmatrix} \lambda & 2 \\ -1 & 5 \end{vmatrix} 
   + (-1) \cdot \begin{vmatrix} \lambda & 1 \\ -1 & 2 \end{vmatrix} \\
&= 1 \cdot (5 - 4) - \lambda \cdot (5\lambda + 2) - (2\lambda + 1) \\
&= 1 - 5\lambda^2 - 2\lambda - 2\lambda - 1 = -5\lambda^2 - 4\lambda
\end{aligned}
\]
要求 \(\Delta_3 > 0\),即:
\[
-5\lambda^2 - 4\lambda > 0 \quad \Rightarrow \quad 5\lambda^2 + 4\lambda < 0\quad \Rightarrow \quad -\frac{4}{5} < \lambda < 0
\]
因此,当且仅当 \(\lambda \in \left( -\dfrac{4}{5}, 0 \right)\) 时,该二次型为正定二次型。
\end{solution}
\begin{example}{}{}
    设有二次型$f(x_1,x_2,x_3)=a(x_1^2+x_2^2+x_3^2)+2x_1x_2+2x_1x_3-2x_2x_3$,问\\
(1)当$a$取何值时,$f(x_1,x_2,x_3)$正定? \\(2)当$a$取何值时,$f(x_1,x_2,x_3)$负定?
\end{example}
\begin{solution}
该二次型对应的矩阵为:
\[\bm{A}=\begin{pmatrix}a&1&1\\1&a&-1\\1&-1&a\end{pmatrix}\]
    (1)第一,二阶顺序子式大于0,即$\begin{cases}\Delta_1=a>0\\\Delta_2=a^2-1>0\end{cases}\Rightarrow a>1$\\
    第三阶顺序子式:$\Delta_3=\begin{vmatrix}a&1&1\\1&a&-1\\1&-1&a\end{vmatrix}=(a-2)(a+1)^2>0\Rightarrow a>2$
    因此,当$a>2$时,二次型正定。
    \\
    (2)二次型$f$负定当且仅当$-f$正定。$-f$对应的矩阵为$-\bm{A}$:
    \[
    -\bm{A} = \begin{pmatrix}
    -a & -1 & -1 \\
    -1 & -a & 1 \\
    -1 & 1 & -a
    \end{pmatrix}
    \]
    
    第一顺序主子式:$\Delta_1' = -a > 0 \Rightarrow a < 0$
    
    第二顺序主子式:$\Delta_2' = \begin{vmatrix}
    -a & -1 \\
    -1 & -a
    \end{vmatrix} = a^2 - 1 > 0 \Rightarrow a < -1 \quad \text{或} \quad a > 1$
    
    第三顺序主子式:
    \[
    \begin{aligned}
    \Delta_3' &= \begin{vmatrix}
    -a & -1 & -1 \\
    -1 & -a & 1 \\
    -1 & 1 & -a
    \end{vmatrix} = -a \cdot \begin{vmatrix} -a & 1 \\ 1 & -a \end{vmatrix} 
       - (-1) \cdot \begin{vmatrix} -1 & 1 \\ -1 & -a \end{vmatrix} 
       + (-1) \cdot \begin{vmatrix} -1 & -a \\ -1 & 1 \end{vmatrix} \\
    &= -a(a^2 - 1) + 1(a + 1) - 1(-1 - a) = -a^3 + a + a + 1 + 1 + a \\
    &= -a^3 + 3a + 2=-(a - 2)(a + 1)^2 > 0
    \end{aligned}
    \]
综合三个条件:$a < 0$,$a < -1$,$a < 2$,取交集,因此,当$a < -1$时,$-\bm{A}$正定,即原二次型负定。
\end{solution}
\begin{example}{}{}
    $\text{设 }\bm{A,B}\text{ 为两个 }n\text{ 阶正定矩阵,证明:}\bm{A+B}\text{ 也是正定矩阵}.$
\end{example}
\begin{solution}
    设$\bm{A},\bm{B}$为两个$n\times n$的正定矩阵,显然由于定义,$\bm{A,B}$都是实对称阵,则验证定义$\bm{(A+B)^{T}}=\bm{A}^{T}+\bm{B}^{T}=\bm{A+B}$,所以$\bm{A+B}$也是实对称阵。则任意矩阵$\bm{X}\ne0,\bm{XAX^{T}}>0,\bm{XBX^{T}}>0$,则$\bm{XAX^{T}}+\bm{XBX^{T}}=\bm{X(A+B)X^{T}}>0$,即$\bm{A+B}$也是正定矩阵。
\end{solution}
\begin{example}{}{}
    设$A$为$n$阶正定矩阵,证明:\\
(1) $\bm{A}^\mathrm{T}$ 也是正定矩阵;\\
$(2)\bm{A}$的伴随矩阵 $\bm{A^*}$ 也是正定矩阵.
\end{example}
\begin{solution}
    (1)设$\bm{A}$为$n\times n$的正定矩阵,则$\bm{A}^\mathrm{T}$也是实对称阵,则任意矩阵$\bm{X}\ne0$,有$\bm{XAX^{T}}>0$,则$\bm{X A^{T} X^{T}}=(\bm{X A X^{T}})^{T}>0$,即$\bm{A}^\mathrm{T}$也是正定矩阵。\\
(2)由于$\bm{A}$为正定矩阵,则$|\bm{A}|>0$,则$\bm{A}$可逆,即$\bm{A}^{-1}$存在,且满足$\bm{A^*}=|\bm{A}|\bm{A}^{-1}$。又由于存在可逆矩阵$\bm{P}$,使得$\bm{A}=\bm{PP^T}$,那么$\bm{A^*}=|\bm{A}|\bm{A}^{-1}=|\bm{A}|(\bm{PP^T})^{-1}=|\bm{A}|(\bm{P}^T)^{-1}\bm{P^{-1}}=|\bm{A}|(\bm{P}^{-1})^{T}\bm{P}^{-1}$,即$\bm{A^{-1}}$也是正定矩阵。所以其各项特征值均大于0,因此$\bm{A^*}$各项特征值也大于0,所以$\bm{A^*}$也是正定矩阵。
\end{solution}