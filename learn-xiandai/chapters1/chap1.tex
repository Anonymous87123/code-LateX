\chapter{行列式}
\section{第1周作业}
习题一第一大题的第(1)(3)(5)问解答如下:
\begin{example}{(习题一第一大题)}{}
计算行列式的值
$\begin{vmatrix}
    \sin x & -\cos x \\
    \cos x & \sin x
\end{vmatrix}$
,
$\begin{vmatrix}
    1 & 2 & 3 \\
    4 & 5 & 6 \\
    7 & 8 & 9
\end{vmatrix}$
,
$\begin{vmatrix}
    x & y & y \\
    y & x & y \\
    y & y & x
\end{vmatrix}$
\end{example}
\begin{solution}{}{}
    (1)原式$=\sin^2x-(-\cos^2x)=1$.\\
    (2)由沙路法则:原式\vspace{-10pt}\\
    \vspace{-10pt}
    \begin{align*}
        =&1\times5\times9+2\times6\times7+3\times4\times8\\
        &-1\times6\times8-2\times4\times9-3\times5\times7\\
        =&45+84+96-48-72-105\\
        =&225-225=0.\square
    \end{align*}
    实际上第三行是第二行各数的两倍减去第一行各数得到的,因此第三行是第一行和第二行的线性组合,所以矩阵:
    \[
    \begin{pmatrix} 
        1 & 2 & 3 \\
        4 & 5 & 6 \\
        7 & 8 & 9
    \end{pmatrix}
    \]的行向量线性相关,行列式的值为0.\\
    (3)仍然由沙路法则:原式\vspace{=-10pt}
    \begin{align*}
        &=x^3+y^3+y^3-xy^2-xy^2-xy^2\\
        &=x^3+2y^3-3xy^2.\square
    \end{align*}
    实际上这个结果可以推广的。
\end{solution}
\begin{theorem}{}{}
    将$n$阶行列式$D$中每个元$a_{ij},(i,j=1,2,...,n)$都加上参数$t$,得到的行列式记为$D(t)$,则:
    \[D(t)=D+\sum_{i=1}^n\sum_{j=1}^nA_{ij}.\]
    其中$A_{ij}$是$a_{ij}$的代数余子式.
\end{theorem}
\begin{proof}{}{}
    将$D(t)$中的第一列拆开,得到的新行列式记为$D_1(t)$,则:
    \[D_1(t)=\begin{vmatrix}a_{11}&a_{12}+t&\cdots&a_{1n}+t\\
        a_{21}&a_{22}+t&\cdots&a_{2n}+t\\
        \vdots&\vdots&&\vdots\\
        a_{n1}&a_{n2}+t&\cdots&a_{nn}+t\end{vmatrix}\]
    \begin{align*}
        D(t)&=D_1(t)+\begin{vmatrix}t&a_{12}+t&\cdots&a_{1n}+t\
        t&a_{22}+t&\cdots&a_{2n}+t\\\vdots&\vdots&&\vdots\\
        t&a_{n2}+t&\cdots&a_{nn}+t\end{vmatrix}\\
        &=D_1(t)+
        \begin{vmatrix}t&a_{12}&\cdots&a_{1n}\\
            t&a_{22}&\cdots&a_{2n}\\
            \vdots&\vdots&&\vdots\\
            t&a_{n2}&\cdots&a_{nn}\end{vmatrix}\\
        &=t\sum_{i=1}^nA_{i1}+D_1(t).\\
        &=t\sum_{i=1}^nA_{i1}+t\sum_{i=1}^nA_{i2}+D_2(t).\\
        &=t\sum_{i=1}^nA_{i1}+t\sum_{i=1}^nA_{i2}+t\sum_{i=1}^nA_{i3}+D_3(t).\\
        &=\dots\\
        &=t\sum_{i=1}{n}\sum_{j=1}^nA_{ij}+D.
        \end{align*}
\end{proof}
