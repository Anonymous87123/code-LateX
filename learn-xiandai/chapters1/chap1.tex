\chapter{行列式}
\section{第1周作业}
习题一第一大题的第(1)(3)(5)问解答如下:
\begin{example}{(习题一第一大题)}{}
计算行列式的值
$\begin{vmatrix}
    \sin x & -\cos x \\
    \cos x & \sin x
\end{vmatrix}$
,
$\begin{vmatrix}
    1 & 2 & 3 \\
    4 & 5 & 6 \\
    7 & 8 & 9
\end{vmatrix}$
,
$\begin{vmatrix}
    x & y & y \\
    y & x & y \\
    y & y & x
\end{vmatrix}$
\end{example}
\begin{solution}{}{}
    (1)原式$=\sin^2x-(-\cos^2x)=1$.\\
    (2)由沙路法则:原式\vspace{-10pt}\\
    \vspace{-10pt}
    \begin{align*}
        =&1\times5\times9+2\times6\times7+3\times4\times8\\
        &-1\times6\times8-2\times4\times9-3\times5\times7\\
        =&45+84+96-48-72-105\\
        =&225-225=0.
    \end{align*}
    实际上第三行是第二行各数的两倍减去第一行各数得到的,因此第三行是第一行和第二行的线性组合,所以矩阵:
    \[
    \begin{pmatrix} 
        1 & 2 & 3 \\
        4 & 5 & 6 \\
        7 & 8 & 9
    \end{pmatrix}
    \]的行向量线性相关,行列式的值为0.\\
    (3)仍然由沙路法则:原式\vspace{=-10pt}
    \begin{align*}
        &=x^3+y^3+y^3-xy^2-xy^2-xy^2\\
        &=x^3+2y^3-3xy^2.
    \end{align*}
    实际上这个结果可以推广的,下面定理的证明运用了行列式的线性性质和递归分解的思想。
\end{solution}
\begin{theorem}{}{}
    将$n$阶行列式$D$中每个元$a_{ij},(i,j=1,2,...,n)$都加上参数$t$,得到新行列式$D(t)$,则:
    \[D(t)=D+\sum_{i=1}^n\sum_{j=1}^nA_{ij}\Leftrightarrow D=D(t)-t\sum_{i=1}^n\sum_{j=1}^nA_{ij}.\]
    其中$A_{ij}$是$a_{ij}$的代数余子式.
\end{theorem}
\begin{proof}{}{}
    将$D(t)$中的第一列拆开,得到的新行列式记为$D_1(t)$,则:
    \[D_1(t)=\begin{vmatrix}a_{11}&a_{12}+t&\cdots&a_{1n}+t\\
        a_{21}&a_{22}+t&\cdots&a_{2n}+t\\
        \vdots&\vdots&&\vdots\\
        a_{n1}&a_{n2}+t&\cdots&a_{nn}+t\end{vmatrix}\]
    \begin{align*}
        D(t)&=D_1(t)+\begin{vmatrix}t&a_{12}+t&\cdots&a_{1n}+t\\
        t&a_{22}+t&\cdots&a_{2n}+t\\\vdots&\vdots&&\vdots\\
        t&a_{n2}+t&\cdots&a_{nn}+t\end{vmatrix}
        =D_1(t)+
        \begin{vmatrix}t&a_{12}&\cdots&a_{1n}\\
            t&a_{22}&\cdots&a_{2n}\\
            \vdots&\vdots&&\vdots\\
            t&a_{n2}&\cdots&a_{nn}\end{vmatrix}\\
        &=t\sum_{i=1}^nA_{i1}+D_1(t)=t\sum_{i=1}^nA_{i1}+t\sum_{i=1}^nA_{i2}+D_2(t)\\
        &=t\sum_{i=1}^nA_{i1}+t\sum_{i=1}^nA_{i2}+t\sum_{i=1}^nA_{i3}+D_3(t)\\
        &=\dots=t\sum_{i=1}^{n}\sum_{j=1}^nA_{ij}+D.
        \end{align*}
\end{proof}
下面推广一下上面那道例题:
\begin{example}{}{}
    求$n$阶行列式的值$(a\ne b)$:
    \[D=\begin{vmatrix}x_1&a&\cdots&a\\b&x_2&\cdots&a\\
        \vdots&\vdots&&\vdots\\b&b&\cdots&x_n\end{vmatrix}=\begin{vmatrix}x_1&b&\cdots&b\\a&x_2&\cdots&b\\
        \vdots&\vdots&&\vdots\\a&a&\cdots&x_n\end{vmatrix}\]
\end{example}
\begin{solution}{}{}
    原题只有第一个行列式,笔者将其进行了转置,得到第二个行列式,目的是方便进行证明,这里运用了“对偶”的思想。这样就可以轻松写出:
    \[D(-a)=\begin{vmatrix}x_1-a&b-a&\cdots&b-a\\0&x_2-a&\cdots&b-a\\
        \vdots&\vdots&&\vdots\\0&0&\cdots&x_n-a\end{vmatrix}=\prod_{i=1}^n(x_i-a)\]
    \[D(-b)=\begin{vmatrix}x_1-b&a-b&\cdots&a-b\\0&x_2-b&\cdots&a-b\\
        \vdots&\vdots&&\vdots\\0&0&\cdots&x_n-b\end{vmatrix}=\prod_{i=1}^n(x_i-b)\]
    因此:\[D=\prod_{i=1}^n(x_i-a)+(-a)\sum_{i=1}^n\sum_{j=1}^nA_{ij}=\prod_{i=1}^n(x_i-b)+(-b)\sum_{i=1}^n\sum_{j=1}^nA_{ij}.\]
    容易发现,\[\sum_{i=1}^n\sum_{j=1}^nA_{ij}\]取决于原来的行列式本身,与引入的参数无关(因为我们是一列一列地拆出参数,而恰好在取代数余子式时避开了含有参数的列),因此上述公式构成二元一次线性方程组,由此可以解得:
    \[D=\frac{a}{a-b}\prod_{i=1}^{n}(x_{i}-b)-\frac{b}{a-b}\prod_{i=1}^{n}(x_{i}-a).\]
\end{solution}
\begin{example}{}{}
    求$n$阶行列式的值:
        \[D=\begin{vmatrix}x_1&a&\cdots&a\\a&x_2&\cdots&a\\
        \vdots&\vdots&&\vdots\\a&a&\cdots&x_n\end{vmatrix}\]
\end{example}
\begin{solution}{}{}
    直接套结论:
    \[D(-a)=\begin{vmatrix}x_1-a&0&\cdots&0\\0&x_2-a&\cdots&0\\
        \vdots&\vdots&&\vdots\\0&0&\cdots&x_n-a\end{vmatrix}=\prod_{i=1}^n(x_i-a)\]
    则:\[D=D(-a)+a\sum_{i=1}^n\sum_{j=1}^nA_{ij}.\]
    其中对于$A_{ij}$而言,由于只有对角线上的数不是0,因此如果不取对角线上的数写代数余子式,那么由于系数是0,行列式无论是什么都不会影响结果\[
    A_{ij} = \begin{cases} 
    0 & i \neq j \\
    \prod\limits_{k \neq i}^{n} (x_k - a) & i = j \end{cases}\]
    所以:\[
    D=D(-a)+a\sum_{i=1}^n\sum_{j=1}^nA_{ij}=D(-a)+a\sum_{i=1}^nA_{ii}=\prod_{i=1}^n(x_i-a)+a\sum_{i=1}^n\prod_{k\neq i}(x_k-a).
    \]
\end{solution}
我们再补充一个例子:
\begin{example}{}{}
    求$n$阶行列式的值($a\neq b$):\[D=\begin{vmatrix}
        a&a&\cdots&a&0\\
        a&a&\cdots&0&b\\
        \vdots&\vdots&&\vdots&\vdots\\
        a&0&\cdots&b&b\\
        0&b&\cdots&b&b\end{vmatrix}\]
\end{example}
\begin{solution}{}{}
    \[D=D(-a)+a\sum_{i=1}^n\sum_{j=1}^nA_{ij}=D(-b)+b\sum_{i=1}^n\sum_{j=1}^nA_{ij}\]
    其中:\[D(-a)=\begin{vmatrix}        
        0&0&\cdots&0&-a\\
        0&0&\cdots&-a&b-a\\
        \vdots&\vdots&&\vdots&\vdots\\
        0&-a&\cdots&b-a&b-a\\
        -a&b-a&\cdots&b-a&b-a\end{vmatrix}
    \]
    然后使用定义:
    \[\begin{vmatrix}a_{11}&a_{12}&\ldots&a_{1n}
        \\a_{21}&a_{22}&\ldots&a_{2n}\\
        \vdots&\vdots&\ddots&\vdots\\
        a_{n1}&a_{n2}&\ldots&a_{nn}\end{vmatrix}
        =\sum(-1)^{\tau(j_1j_2\ldots j_n)}a_{1j_1}a_{2j_2}\ldots a_{nj_n}\]
    就可以得到:\[D(-a)=(-a)^{1+2+3+4+...+n-1}(-a)^n=(-1)^{\frac{n(n+1)}{2}}a^n\]
    \[D(-b)=(-b)^{1+2+3+4+...+n-1}(-b)^n=(-1)^{\frac{n(n+1)}{2}}b^n\]
    当$a\neq b$时,\[D=(-1)^{\frac{n(n+1)}{2}}\frac{ab^n-ba^n}{a-b}=(-1)^{\frac{n^2+n+2}{2}}(a^{n-1}b+a^{n-2}b^2+\cdots+ab^{n-1})\]
\end{solution}
再补充一个例子:
\begin{example}{计算行列式}{}
    \[D_n=\begin{vmatrix}a&b&b&\cdots&b\\b&a&b&\cdots&b\\b&b&a&\cdots&b\\\vdots&\vdots&\vdots&\ddots&\vdots\\b&b&b&\cdots&a\end{vmatrix}\]
\end{example}
\begin{solution}
    每一列的和是$a+(n-1)b$,可把每一行都加到第一列,提取公因式$a+(n-1)b$:\[
    \begin{vmatrix}a+(n-1)b&a+(n-1)b&a+(n-1)b&\cdots&a+(n-1)b\\b&a&b&\cdots&b\\b&b&a&\cdots&b\\\vdots&\vdots&\vdots&\ddots&\vdots\\b&b&b&\cdots&a\end{vmatrix}    
    =[a+(n-1)b]\begin{vmatrix}1&1&1&\cdots&1\\b&a&b&\cdots&b\\b&b&a&\cdots&b\\\vdots&\vdots&\vdots&\ddots&\vdots\\b&b&b&\cdots&a\end{vmatrix}
    \]\[=
    [a+(n-1)b]\begin{vmatrix}1&1&1&\cdots&1\\0&a-b&0&\cdots&0\\0&0&a-b&\cdots&0\\\vdots&\vdots&\vdots&\ddots&\vdots\\0&0&0&\cdots&a-b\end{vmatrix}=[a+(n-1)b](a-b)^{n-1}
    \]
    或者我们可以使用加边法:\[
    D_n=\begin{vmatrix}1&b&b&b&\cdots&b\\0&a&b&b&\cdots&b\\0&b&a&b&\cdots&b\\0&b&b&a&\cdots&b\\\vdots&\vdots&\vdots&\ddots&\vdots\\0&b&b&b&\cdots&a\end{vmatrix}=\begin{vmatrix}1&1&1&\cdots&1\\-1&a-b&0&\cdots&0\\-1&0&a-b&\cdots&0\\\vdots&\vdots&\vdots&\ddots&\vdots\\-1&0&0&\cdots&a-b\end{vmatrix}\]\[
    =\begin{vmatrix}1+\frac{nb}{a-b}&1&1&\cdots&1\\0&a-b&0&\cdots&0\\0&0&a-b&\cdots&0\\\vdots&\vdots&\vdots&\ddots&\vdots\\0&0&0&\cdots&a-b\end{vmatrix}=[a+(n-1)b](a-b)^{n-1}
    \]
\end{solution}
习题一第二大题的第(1)(2)问解答如下:
\begin{example}{(习题一第二大题:证明恒等式)}{}
    \[
    (1)\begin{vmatrix}
        a&b+x\\
        c&d+y
    \end{vmatrix}=\begin{vmatrix}
        a&b\\
        c&d
    \end{vmatrix}+\begin{vmatrix}
        a&x\\
        c&y
    \end{vmatrix}\quad
    (2)\begin{vmatrix}
        0&b&a\\
        1&e&f\\
        0&d&c\end{vmatrix}=\begin{vmatrix}
        a&b\\
        c&d\end{vmatrix}\]
\end{example}
\begin{solution}{}{}
    (1)由于行列式阶数较低,考虑直接展开即可完成证明:\[\begin{vmatrix}
        a&b+x\\
        c&d+y
    \end{vmatrix}=a(d+y)-c(b+x)=ad-bc+ay-cx=\begin{vmatrix}
        a&b\\
        c&d
    \end{vmatrix}+\begin{vmatrix}
        a&x\\
        c&y\end{vmatrix}\]
    (2)直接按列展开即可完成证明:\[\begin{vmatrix}
        0&b&a\\
        1&e&f\\
        0&d&c\end{vmatrix}=-1\times\begin{vmatrix}
        b&a\\
        d&c\end{vmatrix}=\begin{vmatrix}
            a&b\\
            c&d\end{vmatrix}\]
\end{solution}
\begin{example}{(习题一第三大题)}{}
    借助行列式的知识解方程组:\[
    \begin{cases}
        3x_1+2x_2-4x_3=-6\\
        x_1+2x_2-x_3=3\\
        2x_1-x_2+x_3=17\end{cases}\]
\end{example}
\begin{solution}{}{}
    写出各个行列式并应用克莱默法则,每一个行列式都可以由沙路法则或者对其进行行变换得出结果:\[
    D=\begin{vmatrix}
        3&2&-4\\
        1&2&-1\\
        2&-1&1\end{vmatrix}=6-4+4-3-2+16=17\]
    \[
    D_1=\begin{vmatrix}
        -6&2&-4\\
        3&2&-1\\
        17&-1&1\end{vmatrix}= -12-34+12+6-6+136=102\]
    \[
    D_2=\begin{vmatrix}
        3&-6&-4\\
        1&3&-1\\
        2&17&1\end{vmatrix}=\begin{vmatrix}
        0&-26&-4\\
        1&3&-1\\
        2&17&1\end{vmatrix}=\begin{vmatrix}
            0&-26&-4\\
            1&3&-1\\
            0&11&3\end{vmatrix}=34\]
    \[D_3=\begin{vmatrix}
        3&2&-6\\
        1&2&3\\
        2&-1&17\end{vmatrix}=\begin{vmatrix}
        0&1&-26\\
        1&2&3\\
        2&-1&17\end{vmatrix}=\begin{vmatrix}
            0&1&-26\\
            1&2&3\\
            0&-5&11\end{vmatrix}=119
        \]
    故解方程组:\[x_1=\dfrac{D_1}{D}=\dfrac{102}{17}=6,\quad x_2=\dfrac{D_2}{D}=\dfrac{34}{17}=2,\quad x_3=\dfrac{D_3}{D}=\dfrac{119}{17}=7.\]
\end{solution}
\begin{example}{(习题一第5大题)}{}
    若排列$i_1,i_2,i_3,...,i_n$的逆序数为m,求排列$i_n,i_{n-1},i_{n-2},...,i_1$的逆序数。
\end{example}
\begin{solution}{}{}
    排列$i_1,i_2,i_3,...,i_n$的逆序数为$m$指这个排列的逆序数对的数量为$m$,而数对一共有
    \[\left(\begin{array}{c}n\\2\end{array}\right)=\frac{n(n-1)}{2}\]种,
    所以排列$i_n,i_{n-1},i_{n-2},...,i_1$的逆序数为\[\left(\begin{array}{c}n\\2\end{array}\right)-m=\frac{n(n-1)}{2}-m.\]
\end{solution}
\begin{example}{(习题一第6大题)}{}
    求下面各个排列的逆序数并判断排列的奇偶性:\vspace{-5pt}\[
    (1)26538417\quad(2)n(n-1)(n-2)...1\quad(3)2n(2n-2)(2n-4)...2(2n-1)(2n-3)...1.\]
\end{example}
\begin{solution}{}{}
    (1)逆序数为$1+4+3+1+3+1=13$,奇排列。
    (2)逆序数为\[(n-1)+(n-2)+...+1=\frac{n(n-1)}{2}\]
    设$k\in\mathbb{N}_+$当$n=4k,~4k+1$时,为偶排列,当$n=4k+2,~4k+3$时,为奇排列。
    (3)逆序数为\[(2n-1)+(2n-3)+...+1+(n-1)+(n-2)+...+1=n^2+\frac{n(n-1)}{2}=\frac{n(3n-1)}{2}\]
    当$n\equiv0(\mathrm{mod}4)$或$n\equiv3(\mathrm{mod}4)$时,为偶排列;
    当$n\equiv1(\mathrm{mod}4)$或$n\equiv2(\mathrm{mod}4)$时,为奇排列。
\end{solution}
\begin{example}{(习题一第10大题)}{}
    用行列式的定义计算:
    \[\begin{vmatrix}
        0&a&0&a\\
        b&0&0&0\\
        0&c&0&d\\
        0&0&e&0
    \end{vmatrix},\quad\begin{vmatrix}
        a_{11}&0&\cdots&0&0\\0&0&\cdots&0&a_{2n}\\
        \vdots&\vdots&&\vdots&\vdots\\
        0&0&\cdots&a_{n-1,n-1}&a_{n-1,n}\\
        0&a_{n2}&\cdots&a_{n,n-1}&a_{nn}\end{vmatrix},\quad\begin{vmatrix}
        a_1&a_2&\cdots&a_{n-2}&a_{n-1}&a_n\\
        1&0&\cdots&0&0&0\\0&1&\cdots&0&0&0\\
        \vdots&\vdots&&\vdots&\vdots&\vdots\\
        0&0&\cdots&1&0&0\\0&0&\cdots&0&1&0\end{vmatrix}\]
\end{example}
\begin{solution}{}{}
    (1)我们自上而下地按每一行来取元素,则结果为:
    \[\tau{(2143)}abde+\tau{(4123)abce}=abde-abce=abe(d-c)\]
    (2)我们自上而下地按每一列来取元素,则结果为:\vspace{-5pt}
    \begin{align*}
    D=&\tau{(1n(n-1)(n-2)...2)}a_{11}a_{2n}a_{3,n-1}...a_{n-1,n-1},a_{n,2}\\
    =&a_{11}(-1)^{\frac{(n-2)(n-1)}{2}}a_{2n}\cdots a_{n2}\end{align*}
    (3)我们自上而下地按每一行来取元素,则结果为:\vspace{-5pt}
    \begin{align*}
    D=&\tau{(n123...(n-1))}a_n\\
    =&(-1)^{(n-1)}a_n
    \end{align*}
\end{solution}\newpage
\section{第2周作业}
\begin{example}{计算行列式}{}
\[\begin{vmatrix}
    1&2&3&4\\
    3&6&12&5\\
    0&1&3&5\\
    0&4&7&9
\end{vmatrix},\quad
\begin{vmatrix}x&x&\cdots&x&a\\x&x&\cdots&a&x\\\vdots&\vdots&&\vdots&\vdots\\x&a&\cdots&x&x\\a&x&\cdots&x&x\end{vmatrix}_n,\quad
\begin{vmatrix}x&x&\cdots&x&a\\0&0&\cdots&a&x\\\vdots&\vdots&&\vdots&\vdots\\0&a&\cdots&0&x\\a&0&\cdots&0&x\end{vmatrix}_n\]
\end{example}
\begin{solution}
    (1)进行行列式的行变换,再按行展开:\vspace{-5pt}
\[\begin{vmatrix}
    1&2&3&4\\
    3&6&12&5\\
    0&1&3&5\\
    0&4&7&9
\end{vmatrix}=\begin{vmatrix}1&2&3&4\\0&0&3&-7\\0&1&3&5\\0&4&7&9\end{vmatrix}=\begin{vmatrix}0&3&-7\\1&3&5\\4&7&9\end{vmatrix}
=\begin{vmatrix}
    0&3&-7\\
    1&3&5\\
    0&-5&-11
\end{vmatrix}=\begin{vmatrix}
    -7&3\\
    -11&-5
\end{vmatrix}=68\]
(2)观察到所有列的元素和相同,所以同时相减:\\\vspace{-5pt}\[
\begin{vmatrix}x&x&\cdots&x&a\\x&x&\cdots&a&x\\\vdots&\vdots&&\vdots&\vdots\\x&a&\cdots&x&x\\a&x&\cdots&x&x\end{vmatrix}_n=
\begin{vmatrix}x&x&\cdots&x&a\\0&0&\cdots&a-x&x-a\\\vdots&\vdots&&\vdots&\vdots\\0&a-x&\cdots&0&x-a\\a-x&0&\cdots&0&x-a\end{vmatrix}_n=
\begin{vmatrix}x&x&\cdots&x&a+(n-1)x\\0&0&\cdots&a-x&0\\\vdots&\vdots&&\vdots&\vdots\\0&a-x&\cdots&0&0\\a-x&0&\cdots&0&0\end{vmatrix}_n\]
使用定义法计算可得答案为:\vspace{-5pt}\[
(-1)^{\frac{n(n-1)}{2}}\bigg(a+(n-1)x\bigg)(a-x)^{n-1}\]
(3)一样的道理,当$a\neq 0$,时,将前$n-1$列的元素乘以$-\dfrac{x}{a}$加到最后一列上,再利用定义计算:\\\vspace{-5pt}\[D=
\begin{vmatrix}x&x&\cdots&x&a\\0&0&\cdots&a&x\\\vdots&\vdots&&\vdots&\vdots\\0&a&\cdots&0&x\\a&0&\cdots&0&x\end{vmatrix}_n=
\begin{vmatrix}x&x&\cdots&x&a-(n-1)\dfrac{x^2}{a}\\0&0&\cdots&a&0\\\vdots&\vdots&&\vdots&\vdots\\0&a&\cdots&0&0\\a&0&\cdots&0&0\end{vmatrix}_n=(-1)^{\frac{n(n-1)}{2}}(a^n-(n-1)x^2a^{n-2}).\]
当$a=0$时,显然\[D=
\begin{cases}x&,n=1\\x^2&,n=2\\0&,n\ge 3\end{cases}\]
\end{solution}
\begin{example}{证明等式}{}
    \vspace{-5pt}\[(1)\begin{vmatrix}x_1+y_1&x_2+y_2&x_3+y_3\\y_1+z_1&y_2+z_2&y_3+z_3\\z_1+x_1&z_2+x_2&z_3+x_3\end{vmatrix}=2\begin{vmatrix}x_1&y_1&z_1\\x_2&y_2&z_2\\x_3&y_3&z_3\end{vmatrix}\]\\\vspace{-15pt}
    \[(2)\begin{vmatrix}a&2&3&\cdots&n\\1&a+1&3&\cdots&n\\1&2&a+2&\cdots&n\\\vdots&\vdots&\vdots&&\vdots\\1&2&3&\cdots&a+n-1\end{vmatrix}=\left[a+\frac{(n-1)(n+2)}{2}\right](a-1)^{n-1}\]
\end{example}
\begin{solution}
    (1)先转置,再一点点拆出来:\[
    \begin{vmatrix}x_1+y_1&x_2+y_2&x_3+y_3\\y_1+z_1&y_2+z_2&y_3+z_3\\z_1+x_1&z_2+x_2&z_3+x_3\end{vmatrix}
    =\begin{vmatrix}x_1+y_1&y_1+z_1&z_1+x_1\\x_2+y_2&y_2+z_2&z_2+x_2\\x_3+y_3&y_3+z_3&z_3+x_3\end{vmatrix}
    \]然后把最后一列数加到第一行,再减去第二列数:\[
    =\begin{vmatrix}2x_1&y_1+z_1&z_1+x_1\\2x_2&y_2+z_2&z_2+x_2\\2x_3&y_3+z_3&z_3+x_3\end{vmatrix}
    =2\begin{vmatrix}x_1&y_1+z_1&z_1+x_1\\x_2&y_2+z_2&z_2+x_2\\x_3&y_3+z_3&z_3+x_3\end{vmatrix}\]
    然后将第三列数减去第一列数,再乘以$-1$加到第二列数上:\\\vspace{-10pt}\[
    =2\begin{vmatrix}x_1&y_1+z_1&z_1\\x_2&y_2+z_2&z_2\\x_3&y_3+z_3&z_3\end{vmatrix}
    =2\begin{vmatrix}x_1&y_1&z_1\\x_2&y_2&z_2\\x_3&y_3&z_3\end{vmatrix}
    \]
    (2)套路式地将第二行,第三行...第n行分别减去第一行:\[
    \begin{vmatrix}a&2&3&\cdots&n\\1&a+1&3&\cdots&n\\1&2&a+2&\cdots&n\\\vdots&\vdots&\vdots&&\vdots\\1&2&3&\cdots&a+n-1\end{vmatrix}
    =\begin{vmatrix}a&2&3&\cdots&n\\1-a&a-1&0&\cdots&0\\1-a&0&a-1&\cdots&0\\\vdots&\vdots&\vdots&&\vdots\\1-a&0&0&\cdots&a-1\end{vmatrix}
    \]再将第2列到第n列地数依次加到第一列上:\[
    =\begin{vmatrix}a+\dfrac{(n+2)(n-1)}{2}&2&3&\cdots&n\\0&a-1&0&\cdots&0\\0&0&a-1&\cdots&0\\\vdots&\vdots&\vdots&&\vdots\\0&0&0&\cdots&a-1\end{vmatrix}
    =\bigg(a+\dfrac{(n-1)(n+2)}{2}\bigg)(a-1)^{n-1}\]
\end{solution}
\begin{example}{计算行列式}{}
    \vspace{-10pt}\[(1)\begin{vmatrix}7&49&1&1\\0&20&0&0\\-3&6&-1&5\\-2&11&-3&1\end{vmatrix},\quad(2)\begin{vmatrix}x&0&0&\cdots&0&y\\y&x&0&\cdots&0&0\\0&y&x&\cdots&0&0\\\vdots&\vdots&\vdots&&\vdots&\vdots\\0&0&0&\cdots&x&0\\0&0&0&\cdots&y&x\end{vmatrix}_n,\quad(3)\begin{vmatrix}x&z&0&\cdots&0&0\\y&x&z&\cdots&0&0\\0&y&x&\cdots&0&0\\\vdots&\vdots&\vdots&&\vdots&\vdots\\0&0&0&\cdots&x&z\\0&0&0&\cdots&y&x\end{vmatrix}_n\]
\end{example}
\begin{solution}
    (1)直接计算即可:
    \[\begin{vmatrix}7&49&1&1\\0&20&0&0\\-3&6&-1&5\\-2&11&-3&1\end{vmatrix}
    =-\begin{vmatrix}0&20&0&0\\7&49&1&1\\-3&6&-1&5\\-2&11&-3&1\end{vmatrix}
    =20\begin{vmatrix}7&1&1\\-3&-1&5\\-2&-3&1\end{vmatrix}\]\[
    =20\begin{vmatrix}0&-6&8\\-3&-1&5\\-2&-3&1\end{vmatrix}
    =20\begin{vmatrix}0&-6&8\\0&3.5&3.5\\-2&-3&1\end{vmatrix}
    =-40\begin{vmatrix}-6&8\\3.5&3.5\end{vmatrix}=1960.
    \]
    (2)使用分割法,分别构造上三角行列式和下三角行列式,则\[
    \begin{vmatrix}x&0&0&\cdots&0&y\\y&x&0&\cdots&0&0\\0&y&x&\cdots&0&0\\\vdots&\vdots&\vdots&&\vdots&\vdots\\0&0&0&\cdots&x&0\\0&0&0&\cdots&y&x\end{vmatrix}_n
    =x\begin{vmatrix}x&0&0&\cdots&0&0\\y&x&0&\cdots&0&0\\0&y&x&\cdots&0&0\\\vdots&\vdots&\vdots&&\vdots&\vdots\\0&0&0&\cdots&x&0\\0&0&0&\cdots&y&x\end{vmatrix}_{n-1}+(-1)^{n+1}y\begin{vmatrix}y&x&0&\cdots&0\\0&y&x&\cdots&0\\\vdots&\vdots&\vdots&&\vdots\\0&0&0&\cdots&x\\0&0&0&\cdots&y\end{vmatrix}_{n-1}
    \]\[
    =x^n+(-1)^{n+1}y^n=x^n-(-y)^n
    \]
    当然这题也可以用行变换先行处理,将第$k$行乘以$-\dfrac{y}{x}$再加到第$k+1$行:\[
    \begin{vmatrix}x&0&0&\cdots&0&y\\y&x&0&\cdots&0&0\\0&y&x&\cdots&0&0\\\vdots&\vdots&\vdots&&\vdots&\vdots\\0&0&0&\cdots&x&0\\0&0&0&\cdots&y&x\end{vmatrix}_n
    =\begin{vmatrix}x&0&0&\cdots&0&y\\0&x&0&\cdots&0&y(-\frac{y}{x})\\0&0&x&\cdots&0&y(-\frac{y}{x})^2\\\vdots&\vdots&\vdots&&\vdots&\vdots\\0&0&0&\cdots&x&y(\frac{y}{x})^{n-2}\\0&0&0&\cdots&0&x+y(-\frac{y}{x})^{n-1}\end{vmatrix}
    =x^{n-1}\bigg(x+y(-\dfrac{y}{x})^{n-1}\bigg)=x^n-(-y)^n\]
    (3)这题可以递推法来做:\[D_n=
    \begin{vmatrix}x&z&0&\cdots&0&0\\y&x&z&\cdots&0&0\\0&y&x&\cdots&0&0\\\vdots&\vdots&\vdots&&\vdots&\vdots\\0&0&0&\cdots&x&z\\0&0&0&\cdots&y&x\end{vmatrix}_n
    =x\begin{vmatrix}x&z&0&\cdots&0&0\\y&x&z&\cdots&0&0\\0&y&x&\cdots&0&0\\\vdots&\vdots&\vdots&&\vdots&\vdots\\0&0&0&\cdots&x&z\\0&0&0&\cdots&y&x\end{vmatrix}_{n-1}-y\begin{vmatrix}z&0&0&\cdots&0&0\\y&x&z&\cdots&0&0\\\vdots&\vdots&\vdots&&\vdots&\vdots\\0&0&0&\cdots&x&z\\0&0&0&\cdots&y&x\end{vmatrix}_{n-1}
    \]\[
    =x\begin{vmatrix}x&z&0&\cdots&0&0\\y&x&z&\cdots&0&0\\0&y&x&\cdots&0&0\\\vdots&\vdots&\vdots&&\vdots&\vdots\\0&0&0&\cdots&x&z\\0&0&0&\cdots&y&x\end{vmatrix}_{n-1}-yz\begin{vmatrix}x&z&\cdots&0&0\\\vdots&\vdots&&\vdots&\vdots\\0&0&\cdots&x&z\\0&0&\cdots&y&x\end{vmatrix}_{n-2}
    =xD_{n-1}-yzD_{n-2}
    \]
    而$D_1=x,D_2=x^2-yz$,列出特征方程:$r^2-xr+yz=0$,则分类:\newline
    当判别式为$0$时,根为$\dfrac{x}{2}$,则$D_n=(A+Bn)(\dfrac{x}{2})^n$,代入$D_1=x,D_2=x^2-yz=\dfrac{3}{4}x^2$得到\[
    D_n=(n+1)\bigg(\dfrac{x}{2}\bigg)^n\]
    当判别式不为$0$时,根为$r_1=\dfrac{x+\sqrt{x^2-4yz}}{2},r_2=\dfrac{x-\sqrt{x^2-4yz}}{2}$,则$D_n=Ar_1^2+Br_2^2$,代入$D_1=x,D_2=x^2-yz$得到\[
    \begin{cases}A=\dfrac{\sqrt{x^2-4yz}+x}{2\sqrt{x^2-4yz}}\\[2ex]B=\dfrac{\sqrt{x^2-4yz}-x}{2\sqrt{x^2-4yz}}\end{cases}
    \Rightarrow D_n=\dfrac{(x+\sqrt{x^2-4yz})^{n+1}+(x-\sqrt{x^2-4yz})^{n+1}}{2^{n+1}\sqrt{x^2-4yz}}
    \]
    综上:\[
    D_n=\begin{cases}(n+1)\bigg(\dfrac{x}{2}\bigg)^n&,x^2=4yz\\[2ex]
    \dfrac{(x+\sqrt{x^2-4yz})^{n+1}+(x-\sqrt{x^2-4yz})^{n+1}}{2^{n+1}\sqrt{x^2-4yz}}&,x^2\neq4yz\end{cases}
    \]
\end{solution}
\begin{example}{求解线性方程组}{}
    \[\begin{cases}3x_1+x_2+2x_3=0\\5x_2+x_3+4x_4=1\\4x_1+2x_3+x_4=0\\2x_1+x_2+x_4=1\end{cases}    \]
\end{example}
\begin{solution}
    分别计算可得:\[
    D=\begin{vmatrix}3&1&2&0\\-8&1&1&0\\4&0&2&1\\-2&1&-2&0\end{vmatrix}=-\begin{vmatrix}3&1&2\\-8&1&1\\-2&1&-2\end{vmatrix}=\begin{vmatrix}3&1&2\\-8&1&1\\2&-1&2\end{vmatrix}=\begin{vmatrix}1&2&0\\-8&1&1\\2&-1&2\end{vmatrix}=\begin{vmatrix}1&2&0\\-8&1&1\\18&-3&0\end{vmatrix}=39\]
    \[D_1=\begin{vmatrix}0&1&2&0\\1&5&1&4\\0&0&2&1\\1&1&0&1\end{vmatrix}=\begin{vmatrix}0&1&2&0\\1&5&1&4\\0&0&2&1\\0&-4&-1&-3\end{vmatrix}=\begin{vmatrix}1&2&0\\0&2&1\\4&1&3\end{vmatrix}=\begin{vmatrix}1&2&0\\0&2&1\\0&-7&3\end{vmatrix}=13\]
    \[D_2=\begin{vmatrix}3&0&2&0\\0&1&1&4\\4&0&2&1\\2&1&0&1\end{vmatrix}=\begin{vmatrix}0&2&3&0\\1&1&0&4\\0&2&4&1\\1&0&2&1\end{vmatrix}=\begin{vmatrix}0&2&3&0\\0&1&-2&3\\0&2&4&1\\1&0&2&1\end{vmatrix}=\begin{vmatrix}2&3&0\\1&-2&3\\2&4&1\end{vmatrix}=\begin{vmatrix}2&3&0\\-5&-14&0\\2&4&1\end{vmatrix}=-13\]\[
    D_3=\begin{vmatrix}3&1&0&0\\0&5&1&4\\4&0&0&1\\2&1&1&1\end{vmatrix}=\begin{vmatrix}0&1&0&0\\-15&5&1&4\\4&0&0&1\\-1&1&1&1\end{vmatrix}=\begin{vmatrix}1&-15&4\\0&4&1\\1&-1&1\end{vmatrix}=\begin{vmatrix}1&-15&4\\0&4&1\\0&14&-3\end{vmatrix}=-26\]\[
    D_4=\begin{vmatrix}3&1&2&0\\0&5&1&1\\4&0&2&0\\2&1&0&1\end{vmatrix}=\begin{vmatrix}3&1&2&0\\-2&4&1&0\\4&0&2&0\\2&1&0&1\end{vmatrix}=\begin{vmatrix}3&1&2\\-2&4&1\\4&0&2\end{vmatrix}=\begin{vmatrix}7&-7&0\\-2&4&1\\8&-8&0\end{vmatrix}=0    \]
    所以解为:\[x_1=\dfrac{1}{3},x_2=-\dfrac13,x_3=-\dfrac23,x_4=0\]
\end{solution}
\begin{example}{如果齐次线性方程组有非零解,求$\lambda$的取值}{}
\[\begin{cases}\lambda x_1+x_2+x_3=0\\x_1+\lambda x_2+2x_3=0\\\lambda^2x_1+x_2+\lambda x_3=0\end{cases}\]
\end{example}
\begin{solution}
    考虑系数行列式:\[
    \begin{vmatrix}\lambda&1&1\\1&\lambda&2\\\lambda^{2}&1&\lambda\end{vmatrix}=\begin{vmatrix}\lambda&1&1\\1&\lambda&2\\0&1-\lambda&0\end{vmatrix}=\begin{vmatrix}0&1-\lambda^2&1-2\lambda\\1&\lambda&2\\0&1-\lambda&0\end{vmatrix}=\begin{vmatrix}1-2\lambda&1-\lambda^2\\0&1-\lambda\end{vmatrix}=(1-2\lambda)(1-\lambda)=0\]
    所以$\lambda=\frac{1}{2}$或$1$.
\end{solution}