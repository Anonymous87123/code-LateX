\chapter{行列式}
\section{第1周作业}
习题一第一大题的第(1)(3)(5)问解答如下:
\begin{example}{(习题一第一大题)}{}
计算行列式的值
$\begin{vmatrix}
    \sin x & -\cos x \\
    \cos x & \sin x
\end{vmatrix}$
,
$\begin{vmatrix}
    1 & 2 & 3 \\
    4 & 5 & 6 \\
    7 & 8 & 9
\end{vmatrix}$
,
$\begin{vmatrix}
    x & y & y \\
    y & x & y \\
    y & y & x
\end{vmatrix}$
\end{example}
\begin{solution}{}{}
    (1)原式$=\sin^2x-(-\cos^2x)=1$.\\
    (2)由沙路法则:原式\vspace{-10pt}\\
    \vspace{-10pt}
    \begin{align*}
        =&1\times5\times9+2\times6\times7+3\times4\times8\\
        &-1\times6\times8-2\times4\times9-3\times5\times7\\
        =&45+84+96-48-72-105\\
        =&225-225=0.\square
    \end{align*}
    实际上第三行是第二行各数的两倍减去第一行各数得到的,因此第三行是第一行和第二行的线性组合,所以矩阵:
    \[
    \begin{pmatrix} 
        1 & 2 & 3 \\
        4 & 5 & 6 \\
        7 & 8 & 9
    \end{pmatrix}
    \]的行向量线性相关,行列式的值为0.\\
    (3)仍然由沙路法则:原式\vspace{=-10pt}
    \begin{align*}
        &=x^3+y^3+y^3-xy^2-xy^2-xy^2\\
        &=x^3+2y^3-3xy^2.\square
    \end{align*}
    实际上这个结果可以推广的,下面定理的证明运用了​​行列式的线性性质​​和​​递归分解​​的思想。
\end{solution}
\begin{theorem}{}{}
    将$n$阶行列式$D$中每个元$a_{ij},(i,j=1,2,...,n)$都加上参数$t$,得到新行列式$D(t)$,则:
    \[D(t)=D+\sum_{i=1}^n\sum_{j=1}^nA_{ij}\leftrightarrow D=D(t)-t\sum_{i=1}^n\sum_{j=1}^nA_{ij}.\]
    其中$A_{ij}$是$a_{ij}$的代数余子式.
\end{theorem}
\begin{proof}{}{}
    将$D(t)$中的第一列拆开,得到的新行列式记为$D_1(t)$,则:
    \[D_1(t)=\begin{vmatrix}a_{11}&a_{12}+t&\cdots&a_{1n}+t\\
        a_{21}&a_{22}+t&\cdots&a_{2n}+t\\
        \vdots&\vdots&&\vdots\\
        a_{n1}&a_{n2}+t&\cdots&a_{nn}+t\end{vmatrix}\]
    \begin{align*}
        D(t)&=D_1(t)+\begin{vmatrix}t&a_{12}+t&\cdots&a_{1n}+t\\
        t&a_{22}+t&\cdots&a_{2n}+t\\\vdots&\vdots&&\vdots\\
        t&a_{n2}+t&\cdots&a_{nn}+t\end{vmatrix}
        =D_1(t)+
        \begin{vmatrix}t&a_{12}&\cdots&a_{1n}\\
            t&a_{22}&\cdots&a_{2n}\\
            \vdots&\vdots&&\vdots\\
            t&a_{n2}&\cdots&a_{nn}\end{vmatrix}\\
        &=t\sum_{i=1}^nA_{i1}+D_1(t)=t\sum_{i=1}^nA_{i1}+t\sum_{i=1}^nA_{i2}+D_2(t)\\
        &=t\sum_{i=1}^nA_{i1}+t\sum_{i=1}^nA_{i2}+t\sum_{i=1}^nA_{i3}+D_3(t)\\
        &=\dots=t\sum_{i=1}^{n}\sum_{j=1}^nA_{ij}+D.
        \end{align*}
\end{proof}
下面推广一下:
\begin{example}{}{}
    求$n$阶行列式的值$(a\ne b)$:
    \[D=\begin{vmatrix}x_1&a&\cdots&a\\b&x_2&\cdots&a\\
        \vdots&\vdots&&\vdots\\b&b&\cdots&x_n\end{vmatrix}=\begin{vmatrix}x_1&b&\cdots&b\\a&x_2&\cdots&b\\
        \vdots&\vdots&&\vdots\\a&a&\cdots&x_n\end{vmatrix}\]
\end{example}
\begin{solution}{}{}
    原题只有第一个行列式,笔者将其进行了转置,得到第二个行列式,目的是方便进行证明,这里运用了“对偶”的思想。这样就可以轻松写出:
    \[D(-a)=\begin{vmatrix}x_1-a&b-a&\cdots&b-a\\0&x_2-a&\cdots&b-a\\
        \vdots&\vdots&&\vdots\\0&0&\cdots&x_n-a\end{vmatrix}=\prod_{i=1}^n(x_i-a)\]
    \[D(-b)=\begin{vmatrix}x_1-b&a-b&\cdots&a-b\\0&x_2-b&\cdots&a-b\\
        \vdots&\vdots&&\vdots\\0&0&\cdots&x_n-b\end{vmatrix}=\prod_{i=1}^n(x_i-b)\]
    因此:\[D=\prod_{i=1}^n(x_i-a)+(-a)\sum_{i=1}^n\sum_{j=1}^nA_{ij}=\prod_{i=1}^n(x_i-b)+(-b)\sum_{i=1}^n\sum_{j=1}^nA_{ij}.\]
    容易发现,\[\sum_{i=1}^n\sum_{j=1}^nA_{ij}\]取决于原来的行列式本身,与引入的参数无关(因为我们是一列一列地拆出参数,而恰好在取代数余子式时避开了含有参数的列),因此上述公式构成二元一次线性方程组,由此可以解得:
    \[D=\frac{a}{a-b}\prod_{i=1}^{n}(x_{i}-b)-\frac{b}{a-b}\prod_{i=1}^{n}(x_{i}-a).\]
\end{solution}
\begin{example}{}{}
    求$n$阶行列式的值:
        \[D=\begin{vmatrix}x_1&a&\cdots&a\\a&x_2&\cdots&a\\
        \vdots&\vdots&&\vdots\\a&a&\cdots&x_n\end{vmatrix}\]
\end{example}
\begin{solution}{}{}
    直接套结论:
    \[D(-a)=\begin{vmatrix}x_1-a&0&\cdots&0\\0&x_2-a&\cdots&0\\
        \vdots&\vdots&&\vdots\\0&0&\cdots&x_n-a\end{vmatrix}=\prod_{i=1}^n(x_i-a)\]
    则:\[D=D(-a)+a\sum_{i=1}^n\sum_{j=1}^nA_{ij}.\]
    其中对于$A_{ij}$而言,由于只有对角线上的数不是0,因此如果不取对角线上的数写代数余子式,那么由于系数是0,行列式无论是什么都不会影响结果\[
    A_{ij} = \begin{cases} 
    0 & i \neq j \\
    \prod\limits_{k \neq i}^{n} (x_k - a) & i = j \end{cases}\]
    所以:\[
    D=D(-a)+a\sum_{i=1}^n\sum_{j=1}^nA_{ij}=D(-a)+a\sum_{i=1}^nA_{ii}=\prod_{i=1}^n(x_i-a)+a\sum_{i=1}^n\prod_{k\neq i}(x_k-a).
    \]
\end{solution}
我们再补充一个例子:
\begin{example}{}{}
    求$n$阶行列式的值($a\neq b$):\[D=\begin{vmatrix}
        a&a&\cdots&a&0\\
        a&a&\cdots&0&b\\
        \vdots&\vdots&&\vdots&\vdots\\
        a&0&\cdots&b&b\\
        0&b&\cdots&b&b\end{vmatrix}\]
\end{example}
\begin{solution}{}{}
    \[D=D(-a)+a\sum_{i=1}^n\sum_{j=1}^nA_{ij}=D(-b)+b\sum_{i=1}^n\sum_{j=1}^nA_{ij}\]
    其中:\[D(-a)=\begin{vmatrix}        
        0&0&\cdots&0&-a\\
        0&0&\cdots&-a&b-a\\
        \vdots&\vdots&&\vdots&\vdots\\
        0&-a&\cdots&b-a&b-a\\
        -a&b-a&\cdots&b-a&b-a\end{vmatrix}
    \]
    然后使用定义:
    \[\begin{vmatrix}a_{11}&a_{12}&\ldots&a_{1n}
        \\a_{21}&a_{22}&\ldots&a_{2n}\\
        \vdots&\vdots&\ddots&\vdots\\
        a_{n1}&a_{n2}&\ldots&a_{nn}\end{vmatrix}
        =\sum(-1)^{\tau(j_1j_2\ldots j_n)}a_{1j_1}a_{2j_2}\ldots a_{nj_n}\]
    就可以得到:\[D(-a)=(-a)^{1+2+3+4+...+n-1}(-a)^n=(-1)^{\frac{n(n+1)}{2}}a^n\]
    \[D(-b)=(-b)^{1+2+3+4+...+n-1}(-b)^n=(-1)^{\frac{n(n+1)}{2}}b^n\]
    当$a\neq b$时,\[D=(-1)^{\frac{n(n+1)}{2}}\frac{ab^n-ba^n}{a-b}=(-1)^{\frac{n^2+n+2}{2}}(a^{n-1}b+a^{n-2}b^2+\cdots+ab^{n-1})\]
\end{solution}
习题一第二大题的第(1)(2)问解答如下:
\begin{example}{(习题一第二大题:证明恒等式)}{}
    \[
    (1)\begin{vmatrix}
        a&b+x\\
        c&d+y
    \end{vmatrix}=\begin{vmatrix}
        a&b\\
        c&d
    \end{vmatrix}+\begin{vmatrix}
        a&x\\
        c&y
    \end{vmatrix}
    \]\[
    (2)\begin{vmatrix}
        0&b&a\\
        1&e&f\\
        0&d&c\end{vmatrix}=\begin{vmatrix}
        a&b\\
        c&d\end{vmatrix}\]
\end{example}
\begin{proof}{}{}
\end{proof}