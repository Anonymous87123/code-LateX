\documentclass[12pt, a4paper, oneside,UTF8]{ctexbook}
% ========== 基本包加载 ==========
% 1. 基础宏包
\usepackage{geometry}
\usepackage{fontspec}
\usepackage{amsmath, amsthm, amssymb}
\usepackage{extarrows}
\allowdisplaybreaks
\usepackage{mathrsfs}
\usepackage{enumitem}
\usepackage{graphicx}
\usepackage{array}
\usepackage{ulem}
\usepackage{caption}
\usepackage{tocloft}
% 2. 图形和颜色宏包
\usepackage[dvipsnames]{xcolor}
\usepackage{tikz}
\usetikzlibrary{shapes.geometric}
\usepackage[most]{tcolorbox}
\tcbuselibrary{theorems}
% 3. 页眉页脚宏包
\usepackage{fancyhdr}
\usepackage{lastpage}
% 4. 超链接和书签
\usepackage{hyperref}
\usepackage{bookmark}
% 5. 智能引用
\usepackage{cleveref}
% ========================
% 1. 页面布局设置(窄边距)
\geometry{
  a4paper,
  top=15mm,
  bottom=10mm,
  left=15mm,
  right=15mm,
  headheight=25pt,
  headsep=8mm,
  footskip=15mm,
  includehead,
  includefoot
}
% 2. 字体设置(修改部分)
% 设置全局英文字体
\setmainfont{Times New Roman}
\setsansfont{Arial}
\setmonofont{Consolas}
% 定义字体命令(保持不变)
\newcommand{\kt}{\kaishu} % 楷体
\newcommand{\st}{\songti} % 宋体
\newcommand{\htbf}{\heiti\bfseries} % 黑体加粗
\newcommand{\e}{\text{e}}
\newcommand{\dd}{\text{d}}
\definecolor{deepblue}{HTML}{003366}
\definecolor{lightblue}{HTML}{E5F6FF}
\definecolor{deepgreen}{HTML}{006633}
\definecolor{lightgreen}{HTML}{E5F6E5}
\definecolor{deeppurple}{HTML}{660066}
\definecolor{lightpurple}{HTML}{F6E5F6}
\definecolor{deepred}{HTML}{990000}
\definecolor{lightred}{HTML}{FFE5E5}
\definecolor{deeporange}{HTML}{CC6600}
\definecolor{lightorange}{HTML}{FFF6E5}
\definecolor{deepgray}{HTML}{333333}
\definecolor{lightgray}{HTML}{F6F6F6}
\definecolor{deepbrown}{HTML}{8B4513}
\definecolor{lightbrown}{HTML}{FFE4B5}
\definecolor{lightyellow}{HTML}{FFFFE0}
\definecolor{darkyellow}{HTML}{CCCC00}
\definecolor{lightcyan}{HTML}{E0FFFF}
\definecolor{darkcyan}{HTML}{008B8B}
\definecolor{lightpink}{HTML}{FFB6C1}
\definecolor{darkpink}{HTML}{FF1493}
\definecolor{lavender}{HTML}{E6E6FA}
\definecolor{thistle}{HTML}{D8BFD8}
\definecolor{lightblue}{HTML}{ADD8E6}
\definecolor{darkblue}{HTML}{00008B}
\definecolor{lightgray}{HTML}{D3D3D3}
\definecolor{darkgray}{HTML}{A9A9A9}
\newcounter{theoremcounter}[section]
% 定理环境
\newtcbtheorem[use counter=theoremcounter,number within=section]{theorem}{定理}{
    colback=lightgreen, %背景颜色
    colframe=deepgreen, %边框颜色
    colbacktitle=deepgreen, %标题框颜色
    coltitle=white,  %标题颜色
    fonttitle=\upshape\color{white}, %标题字体
    fontupper=\rmfamily, %标题内容字体
    separator sign none,
    top=8pt,
    attach boxed title to top left={yshift=-2mm,xshift=5mm},
    description delimiters={ }{ },
    enhanced % 边框粗细
}{thm}
% 引理环境
\newtcbtheorem[use counter=theoremcounter,number within=section]{lemma}{引理}{
    colback=lightpurple,
    colframe=deeppurple,
    colbacktitle=deeppurple,
    coltitle=white,
    fonttitle=\upshape\color{white},
    fontupper=\rmfamily,
    separator sign none,
    top=8pt,
    attach boxed title to top left={yshift=-2mm,xshift=5mm},
    description delimiters={ }{ },
    enhanced
}{lem}
% 定义环境
\newtcbtheorem[number within=section,number within=section]{definition}{定义}{
    colback=lightred,
    colframe=deepred,
    colbacktitle=deepred,
    coltitle=white,
    fonttitle=\upshape\color{white},
    fontupper=\rmfamily,
    separator sign none,
    top=8pt,
    attach boxed title to top left={yshift=-2mm,xshift=5mm},
    description delimiters={ }{ },
    enhanced
}{def}

% 例题和解共享计数器
\newcounter{examplecounter}[section]
% 例题环境
\newtcbtheorem[use counter=examplecounter,number within=section]{example}{例题}{
    colback=lightblue,
    colframe=deepblue,
    colbacktitle=deepblue,
    coltitle=white,
    fonttitle=\upshape\color{white},
    fontupper=\rmfamily,
    separator sign none,
    top=8pt,
    attach boxed title to top left={yshift=-2mm,xshift=5mm},
    description delimiters={ }{ },
    enhanced  
}{ex}
% 结论环境(宋体)
\newtcbtheorem[number within=section]{conclusion}{结论}{
    colback=lightorange,
    colframe=deeporange,
    colbacktitle=deeporange,
    coltitle=white,
    fonttitle=\upshape\color{white},
    fontupper=\rmfamily,
    separator sign none,
    top=8pt,
    attach boxed title to top left={yshift=-2mm,xshift=5mm},
    description delimiters={ }{ },
    enhanced
}{con}
\newtcbtheorem[number within=section]{property}{性质}{
    colback=lightgray,
    colframe=deepgray,
    colbacktitle=deepgray,
    coltitle=white,
    fonttitle=\upshape\color{white},
    fontupper=\rmfamily,
    separator sign none,
    top=8pt,
    attach boxed title to top left={yshift=-2mm,xshift=5mm},
    description delimiters={ }{ },
    enhanced       % 标题加粗与其他环境统一
}{prop}
\newtcbtheorem[number within=section]{criterion}{准则}{
    colback=lightbrown,
    colframe=deepbrown,
    colbacktitle=deepbrown,
    coltitle=white,
    fonttitle=\upshape\color{white},
    fontupper=\rmfamily,
    separator sign none,
    top=8pt,
    attach boxed title to top left={yshift=-2mm,xshift=5mm},
    description delimiters={ }{ },
    enhanced
}{crit}
\newtcbtheorem[use counter=theoremcounter,number within=section]{corollary}{推论}{
    colback=lightcyan,
    colframe=darkcyan,
    colbacktitle=darkcyan,
    coltitle=white,
    fonttitle=\upshape\color{white},
    fontupper=\rmfamily,
    separator sign none,
    top=8pt,
    attach boxed title to top left={yshift=-2mm,xshift=5mm},
    description delimiters={ }{ },
    enhanced       % 标题加粗与其他环境统一
}{cor}
% 传统样式环境(无背景色)
\newtheoremstyle{plain-chinese}% 名称
  {6pt}% 上方空白
  {6pt}% 下方空白
  {\st}% 正文字体(宋体)
  {}% 缩进
  {\heiti}% 标题字体(黑体加粗)
  {.}% 标题后标点
  { }% 标题后空白
  {}% 标题说明
\theoremstyle{plain-chinese}
\newtheorem{solution}{解}[section] % 使用与例题相同的计数器
\newtheorem{remark}{注}[section]
% 4. 智能引用设置(保持不变)
% ========================
\crefname{theorem}{定理}{定理}
\crefname{lemma}{引理}{引理}
\crefname{definition}{定义}{定义}
\crefname{example}{例题}{例题}
\crefname{conclusion}{结论}{结论}
\crefname{solution}{解}{解}
\crefname{remark}{注}{注}
\crefname{property}{性质}{性质}
\crefname{criterion}{准则}{准则}
\crefname{proof}{证明}{证明}

% 5. 数学字体设置(修改)
\usepackage{unicode-math} % 更好的数学字体支持
\setmathfont{Latin Modern Math} % 使用默认数学字体
% ========== 自定义命令(保持不变) ==========
\newcommand{\R}{\mathbb{R}} % 实数集
\newcommand{\C}{\mathbb{C}} % 复数集
\newcommand{\Z}{\mathbb{Z}} % 整数集
\newcommand{\N}{\mathbb{N}} % 自然数集

% ========== 图形路径设置(保持不变) ==========
\graphicspath{{./flg/}} % 图片路径

% 6. 页眉页脚设置
% ========================
\usepackage{fancyhdr}
\usepackage{lastpage} % 获取总页数
\pagestyle{fancy}
\fancyhf{} % 清除所有页眉页脚设置

% 通用设置
\fancyhead[L]{\small\kt 工科数学分析作业} % 左边:书名(楷体)
\fancyhead[C]{\small\st 华南理工大学}
\fancyhead[R]{\small\st 夏同202530451676} % 中间:声明(宋体)
\fancyfoot[C]{\thepage} % 居中页码(自定义样式)

% 正文部分设置
\fancyhead[R]{\small\st\rightmark} % 右边:章节名称(宋体)

% 前言部分设置(使用罗马数字页码)
\fancypagestyle{frontmatter}{
    \fancyhf{}
    \fancyhead[L]{\small\kt 工科数学分析作业}
    \fancyhead[C]{\small\st 前言}
    \fancyhead[R]{} % 前言部分无章节名称
    \fancyfoot[C]{\thepage}
    \renewcommand{\headrulewidth}{0.4pt} % 页眉线
    \renewcommand{\footrulewidth}{0pt} % 无页脚线
    \pagenumbering{Roman} % 罗马数字页码
}

% 目录部分设置(使用罗马数字页码,延续前言页码)          % 去掉标题下方的横线
\fancypagestyle{tocmatter}{
    \fancyhf{}
    \fancyhead[L]{\small\kt 工科数学分析作业}
    \fancyhead[C]{\small\st 目录} % 居中显示"目录"
    \fancyhead[R]{\small\st 夏同} 
    \fancyfoot[C]{\thepage}
    \renewcommand{\headrulewidth}{0.4pt}
    \renewcommand{\footrulewidth}{0pt}
    % 注意:这里不重置页码,延续前面的罗马数字
}
% 正文部分设置(使用阿拉伯数字页码)
\fancypagestyle{mainmatter}{
    \fancyhf{}
    \fancyhead[L]{\small\kt 工科数学分析作业}
    \fancyhead[C]{\small\st 计算机科学与工程学院~计类1班~夏同}
    \fancyhead[R]{\small\st\rightmark}
    \fancyfoot[C]{\thepage}
    \renewcommand{\headrulewidth}{0.4pt} % 页眉线
    \renewcommand{\footrulewidth}{0pt} % 无页脚线
    \pagenumbering{arabic} % 阿拉伯数字页码
}
% 设置章节标记格式
\renewcommand{\chaptermark}[1]{\markboth{#1}{}}
\renewcommand{\sectionmark}[1]{\markright{\thesection.\ #1}}

% 8. 章节标题字体设置
\ctexset{
    chapter = {
        format = \centering, % 整体居中
        nameformat = \kaishu\LARGE, % 编号部分黑体加粗
        titleformat = \songti\LARGE, % 标题部分宋体
        aftername = \quad, % 编号和标题之间的间距
        beforeskip = 30pt, % 标题前的垂直间距
        afterskip = 20pt, % 标题后的垂直间距
        name = {第,章}, % 中文章节编号格式
        number = \chinese{chapter}, % 使用中文数字
    },
    section = {
        format = \raggedright, % 左对齐
        nameformat = \heiti\bfseries\large, % 编号部分黑体加粗
        titleformat = \songti\large, % 标题部分宋体
        aftername = \quad, % 编号和标题之间的间距
        beforeskip = 15pt, % 标题前的垂直间距
        afterskip = 6pt, % 标题后的垂直间距
    },
    subsection = {
        format = \raggedright, % 左对齐
        nameformat = \heiti\bfseries\normalsize, % 编号部分黑体加粗
        titleformat = \songti\normalsize, % 标题部分宋体
        aftername = \quad, % 编号和标题之间的间距
        beforeskip = 6pt, % 标题前的垂直间距
        afterskip = 3pt, % 标题后的垂直间距
    }
}
\hypersetup{colorlinks=true,linkcolor=black}
\begin{document}
\begin{titlepage}
    \centering
    % 顶部留白
    \vspace*{0.5cm}
        \includegraphics[width=\textwidth]{flg/mylogo.png} % 调整图片大小
    \par\vspace{1cm}
    {\fontsize{75}{75}\selectfont\songti 简单导数 \par}
    \vspace{1cm} % 减少间距
    {\Large \kaishu{作者:} 夏同 \par}
    \vspace{0.1cm}
    {\Large \kaishu{学院:} 计算机科学与工程学院 \par}
    \vspace{0.1cm}
    {\Large \kaishu{参考:} 群友 \par}
    \vspace{0.1cm}
    {\Large \kaishu{日期:} \today \par}
    \vspace{0.5cm}
        \includegraphics[width=0.7\textwidth]{flg/saying.png} % 调整图片大小
    \par\vspace{1cm}
\end{titlepage}

% 前言部分
\frontmatter
\pagestyle{frontmatter}

% 前言
\chapter*{前言}
\addcontentsline{toc}{chapter}{前言}

离散数学作为计算机科学的基础核心课程,涵盖逻辑、集合、关系、图论等众多重要概念。这些抽象的理论不仅是后续课程(如数据结构、算法、数据库、编译原理)的基石,更是培养计算思维和形式化推理能力的关键。

为帮助同学们更好地复习和掌握本课程的核心内容,特编写此份学习笔记。笔记以华南理工大学计算机科学与工程学院使用的《离散数学及其应用》(陈琼,马千里等编著)为主要参考,结合课堂讲授和个人理解,对知识点进行了系统梳理和归纳。

首先感谢陈琼、马千里等老师编著的优秀教材,为学习离散数学提供了系统而全面的框架。同时感谢学院老师们的悉心讲授,使我能够理解并掌握这些抽象概念。

由于本人学识有限,笔记中难免存在疏漏、错误或不妥之处。恳请各位读者不吝指正。期望这份笔记能够为同学们的期末复习提供些许帮助。

\vspace{0.5cm}
\begin{flushright}
夏同\\
于华南理工大学\\
\today
\end{flushright}

% 目录
\newpage
\ctexset{tocdepth=2}
\tableofcontents
\thispagestyle{tocmatter} 

% 环境索引页(暂时禁用)
\cleardoublepage
\pagestyle{indexmatter}
\makeenvindex
\cleardoublepage

% 正文部分
\mainmatter
\pagestyle{mainmatter} % 应用正文样式
% 章节内容
\chapter{行列式}
\section{第1周作业}
习题一第一大题的第(1)(3)(5)问解答如下:
\begin{example}{(习题一第一大题)}{}
计算行列式的值
$\begin{vmatrix}
    \sin x & -\cos x \\
    \cos x & \sin x
\end{vmatrix}$
,
$\begin{vmatrix}
    1 & 2 & 3 \\
    4 & 5 & 6 \\
    7 & 8 & 9
\end{vmatrix}$
,
$\begin{vmatrix}
    x & y & y \\
    y & x & y \\
    y & y & x
\end{vmatrix}$
\end{example}
\begin{solution}{}{}
    (1)原式$=\sin^2x-(-\cos^2x)=1$.\\
    (2)由沙路法则:原式\vspace{-10pt}\\
    \vspace{-10pt}
    \begin{align*}
        =&1\times5\times9+2\times6\times7+3\times4\times8\\
        &-1\times6\times8-2\times4\times9-3\times5\times7\\
        =&45+84+96-48-72-105\\
        =&225-225=0.\square
    \end{align*}
    实际上第三行是第二行各数的两倍减去第一行各数得到的,因此第三行是第一行和第二行的线性组合,所以矩阵:
    \[
    \begin{pmatrix} 
        1 & 2 & 3 \\
        4 & 5 & 6 \\
        7 & 8 & 9
    \end{pmatrix}
    \]的行向量线性相关,行列式的值为0.\\
    (3)仍然由沙路法则:原式\vspace{=-10pt}
    \begin{align*}
        &=x^3+y^3+y^3-xy^2-xy^2-xy^2\\
        &=x^3+2y^3-3xy^2.\square
    \end{align*}
    实际上这个结果可以推广的。
\end{solution}
\begin{theorem}{}{}
    将$n$阶行列式$D$中每个元$a_{ij},(i,j=1,2,...,n)$都加上参数$t$,得到的行列式记为$D(t)$,则:
    \[D(t)=D+\sum_{i=1}^n\sum_{j=1}^nA_{ij}.\]
    其中$A_{ij}$是$a_{ij}$的代数余子式.
\end{theorem}
\begin{proof}{}{}
    将$D(t)$中的第一列拆开,得到的新行列式记为$D_1(t)$,则:
    \[D_1(t)=\begin{vmatrix}a_{11}&a_{12}+t&\cdots&a_{1n}+t\\
        a_{21}&a_{22}+t&\cdots&a_{2n}+t\\
        \vdots&\vdots&&\vdots\\
        a_{n1}&a_{n2}+t&\cdots&a_{nn}+t\end{vmatrix}\]
    \begin{align*}
        D(t)&=D_1(t)+\begin{vmatrix}t&a_{12}+t&\cdots&a_{1n}+t\
        t&a_{22}+t&\cdots&a_{2n}+t\\\vdots&\vdots&&\vdots\\
        t&a_{n2}+t&\cdots&a_{nn}+t\end{vmatrix}\\
        &=D_1(t)+
        \begin{vmatrix}t&a_{12}&\cdots&a_{1n}\\
            t&a_{22}&\cdots&a_{2n}\\
            \vdots&\vdots&&\vdots\\
            t&a_{n2}&\cdots&a_{nn}\end{vmatrix}\\
        &=t\sum_{i=1}^nA_{i1}+D_1(t).\\
        &=t\sum_{i=1}^nA_{i1}+t\sum_{i=1}^nA_{i2}+D_2(t).\\
        &=t\sum_{i=1}^nA_{i1}+t\sum_{i=1}^nA_{i2}+t\sum_{i=1}^nA_{i3}+D_3(t).\\
        &=\dots\\
        &=t\sum_{i=1}{n}\sum_{j=1}^nA_{ij}+D.
        \end{align*}
\end{proof}

\chapter{极限}
\newpage
\section{第2周作业}
\begin{example}{给出下列极限的精确定义}{}
    (1)$\displaystyle\lim_{x\to 0}f(x)=A$\quad (2)$\displaystyle\lim_{x\to 0}(1+x)^{\frac{1}{x}}=e$
\end{example}
\begin{solution}
    (1)对于任意$\varepsilon >0$,存在$\delta>0$使得当$0<|x|<\delta$时,$|f(x)|<\varepsilon$.

    (2)对于任意$\varepsilon>0$,存在$\delta>0$使得当$0<|x|<\delta$时,$|(1+x)^{\frac{1}{x}}-e|<\varepsilon$.
\end{solution}
\begin{example}{利用极限的精确定义证明下列函数的极限}{}
    (1)$\displaystyle\lim_{x\to 3}(x^2+5x)=24$\quad (2)$\displaystyle\lim_{x\to 1}\frac{x^2-1}{x-1}=2$
\end{example}
\begin{solution}
    (1)要证对于任意$\varepsilon>0$,存在$\delta>0$使得当$0<|x-3|<\delta$时,$|(x^2+5x)-24|=|(x+8)(x-3)|<\varepsilon$。已经出现了$|x-3|$,所以现在只需限定$|x+8|$,先限定$|x-3|<1$,那么$|x+8|<12$,此时还需满足$|(x+8)(x-3)|<12|x-3|<\varepsilon$,得$|x-3|<\dfrac{\varepsilon}{12}$,故取$\delta=\min\{1,\dfrac{\varepsilon}{12}\}$,当$0<|x-3|<\delta$时,$|(x^2+5x)-24|=|(x+8)(x-3)|<\varepsilon$.

    (2)要证对于任意$\varepsilon>0$,存在$\delta>0$使得当$0<|x-1|<\delta$时,$|\frac{x^2-1}{x-1}-2|=|x-1|<\varepsilon$。取$\delta=\varepsilon$,当$0<|x-1|<\delta$时,$|\frac{x^2-1}{x-1}-2|<\varepsilon$.
\end{solution}
\begin{example}{证明}{}
    由$\displaystyle\lim_{x\to a}f(x)=A$能推出$\displaystyle\lim_{x\to a}|f(x)|=|A|$,但反之不然。
\end{example}
\begin{solution}
    对于任意$\varepsilon>0$,存在$\delta>0$使得当$0<|x-a|<\delta$时,$|f(x)-A|=<\varepsilon$,所以由绝对值不等式得到$\displaystyle|f(x)-A|>||f(x)-|-A||=||f(x)|-|A||>0$,故$||f(x)|-|A||<\varepsilon$,所以由$\displaystyle\lim_{x\to a}f(x)=A$能推出$\displaystyle\lim_{x\to a}|f(x)|=|A|$.然后反过来,考虑定义在实数域上的函数$f(x)=\begin{cases}1,x\in Q\\-1,x\notin Q\end{cases}$,其极限$\displaystyle\lim_{x\to a}|f(x)|=1$,但是$\displaystyle\lim_{x\to a}f(x)$不存在。
\end{solution}
\begin{example}{利用极限的精确证明}{}
   $\displaystyle\lim_{x\to a}\sin x=\sin a$
\end{example}
\begin{solution}
    要证$\displaystyle\lim_{x\to a}\sin x=\sin a$,只需证对于任意$\varepsilon>0$,存在$\delta>0$使得当$0<|x-a|<\delta$时,$\displaystyle\lim_{x\to a}|\sin x-\sin a|=2|\cos\dfrac{x+a}{2}||\sin\dfrac{x-a}{2}|<\varepsilon$。又因为$2|\cos\dfrac{x+a}{2}||\sin\dfrac{x-a}{2}|<2|\sin\dfrac{x-a}{2}|<2|\dfrac{x-a}{2}|=|x-a|$,所以取$\delta=\varepsilon$,当$0<|x-a|<\delta$时,$\displaystyle\lim_{x\to a}|\sin x-\sin a|<\varepsilon$.
\end{solution}
\chapter{导数题}
\section{偏移题}
\begin{example}{}{}
已知 $f(x)=(x-1)\ln x,f(x_1)=f(x_2),x_1 \neq x_2$,过$(x_1,f(x_1))$与$(x_2,f(x_2))$两点分别作切线交$x$轴于$(x_3,0),(x_4,0)$,证明$x_3+x_4>2$.
\end{example}
\begin{solution}
即证$\displaystyle\frac{(x_1-1)^2}{x_1\ln x_1 + x_1 - 1} +\frac{(x_2-1)^2}{x_2\ln x_2 + x_2 - 1}> 0$.
令 $t = \dfrac{\ln x_2}{\ln x_1}$。易证 $x_1 x_2 < 1\Leftrightarrow t < -1$。则
\begin{equation}
x_1-1 = t(x_1^t-1) \implies x_1^t - 1 = \frac{x_1-1}{t} \label{eq:constraint}
\end{equation}
将 $x_2=x_1^t$ 及其约束直接代入目标不等式的第一项:
\begin{align*}
\frac{(x_2-1)^2}{x_2\ln x_2 + x_2 - 1} = \frac{(x_1^t-1)^2}{x_1^t (t\ln x_1) + x_1^t - 1} = \frac{\left(\frac{x_1-1}{t}\right)^2}{t x_1^t \ln x_1 + \frac{x_1-1}{t}} = \frac{(x_1-1)^2}{t^3 x_1^t \ln x_1 + t(x_1-1)}
\end{align*}
提取公因式 $(x_1-1)^2$ 等价于:
\begin{equation}
\frac{1}{t^3 x_1^t \ln x_1 + t(x_1-1)} + \frac{1}{x_1\ln x_1 + x_1 - 1} > 0
\end{equation}
此时只需分析两项分母的符号:
因 $x_1>1$,右项分母 $D_2 = x_1\ln x_1 + x_1 - 1 > 0$。
因 $t < -1$ 且 $x_1>1$,有 $t^3 x_1^t \ln x_1 < 0$ 且 $t(x_1-1) < 0$。故左项分母 $D_1 = t^3 x_1^t \ln x_1 + t(x_1-1) < 0$。
由于分母一正一负,不等号等价于 $\frac{1}{D_2} > \frac{1}{-D_1}$,即为
\begin{equation}
-D_1 > D_2 \implies -t^3 x_1^t \ln x_1 - t(x_1-1) > x_1\ln x_1 + x_1 - 1\implies -(t^3 x_1^t + x_1)\ln x_1 > (t+1)(x_1-1)
\end{equation}
利用约束 $x_1^t = \frac{x_1-1+t}{t}$ 消去指数:$t^3 x_1^t + x_1 = t^2(x_1-1+t) + x_1 = t^2(x_1-1) + t^3 + x_1$.\\
为消除负号令 $s = -t > 1$:$t^3 x_1^t + x_1 = s^2(x_1-1) - s^3 + x_1 = x_1(s^2+1) - s^2(s+1)$.\\
约束条件化为 $x_1 - 1 = s(1 - x_1^{-s})$,即证:
\begin{equation}
[x_1(s^2+1) - s^2(s+1)]\ln x_1 < (s-1)(x_1-1)\Leftrightarrow \frac{x_1-1}{\ln x_1} > \frac{x_1(s^2+1) - s^2(s+1)}{s-1}
\end{equation}
令 $D(s) = x_1(s^2+1) - s^2(s+1)$。先证明约束下 $D(s) > 0$,即证 $x_1 > u_0 = \frac{s^2(s+1)}{s^2+1}$。
考查约束函数 $G(x_1) = x_1 - 1 - s(1-x_1^{-s})$。因 $G^\prime(x_1)$ 在根处单调递增,只需证 $G(u_0) < 0$ 即可推出真实根大于 $u_0$。
代入 $u_0= \frac{s^2(s+1)}{s^2+1}$ ,等价于证明辅助函数:
$$J(s) = (2s-1)\ln s + (s+1)\ln(s+1) - (s+1)\ln(s^2+1) > 0$$
易知 $J(1)=0$,求二阶导数通分作差得分子为 $s^6 + 3s^5 + 3s^4 - 2s^3 + 3s^2 + 3s + 1$,对于 $s>1$ 恒大于 0。
故 $J^{\prime\prime}(s) > 0 \implies J^\prime(s) > 0 \implies J(s) > 0$。
所以 $D(s) > 0,x_1 > u_0 > s$。

令 $x = \ln x_1 > 0$,下证 $E(x) = (e^x-1)(s-1) - x[e^x(s^2+1) - s^2(s+1)] > 0$。
利用约束 $e^x - 1 = s(1 - e^{-sx})$ 消掉幂函数得:
$E(x) = (e^x-1)(s-1) - x(e^x - s^3 e^{-sx})$。
在曲线 $F(x,s) = e^x - 1 - s(1 - e^{-sx}) = 0$ 上,由隐函数定理得导数:
$$\frac{ds}{dx} = -\frac{F_x}{F_s} = \frac{e^{(s+1)x} - s^2}{e^{sx} - 1 + sx} > 0$$
已知约束条件为 $e^x - 1 = s(1 - e^{-sx})$,此时 $s$ 可视为关于 $x$ 的隐函数 $s(x)$。
为了求出导数 $s'(x)$,我们在等式两边同时对 $x$ 求导(主元法):
左边对 $x$ 求导为:$e^x$
右边利用乘法法则对 $x$ 求导:
$[s(x)]^\prime \cdot (1 - e^{-sx}) + s(x) \cdot [1 - e^{-sx}]^\prime$
$= s'(x)(1 - e^{-sx}) - s(x) \cdot e^{-sx} \cdot (-s'(x) \cdot x - s(x))$
$= s'(x)(1 - e^{-sx}) + s(x) e^{-sx} (x s'(x) + s(x))$
$= s'(x)[1 - e^{-sx} + x s(x) e^{-sx}] + s^2(x) e^{-sx}$

将左右两边相等,解出 $s'(x)$(即 $\frac{ds}{dx}$):
$$s'(x) = \frac{e^x - s^2 e^{-sx}}{1 - e^{-sx} + x s e^{-sx}} = \frac{e^{(s+1)x} - s^2}{e^{sx} - 1 + sx}$$
因 $x>0, s>1$,易证分子分母均大于0,故 $s'(x) > 0$。

目标函数为 $E(x) = (e^x-1)(s(x)-1) - x[e^x(s(x)^2+1) - s(x)^2(s(x)+1)]$。
利用复合函数求导法则,对 $x$ 求导(这等价于全导数展开):
$E'(x) = \left[ \text{将 } s \text{ 视作常数对 } x \text{ 求导的部分} \right] + \left[ \text{将 } x \text{ 视作常数对 } s \text{ 求导的部分} \right] \cdot s'(x)$
即:
$E'(x) = \frac{\partial E}{\partial x} + \frac{\partial E}{\partial s} \cdot s'(x)$
代入偏导数即可得到原解答中的 $A$ 表达式。
考察 $E(x)$ 沿曲线对 $x$ 的全导数 $\frac{dE}{dx}$ 的符号,其决定于 $A = \frac{\partial E}{\partial x}(e^{sx} - 1 + sx) + \frac{\partial E}{\partial s}(e^{(s+1)x} - s^2)$。
展开并提取决定全局走势的核心主导项(即正向项与最大的负向耗散项之差):
$$e^{(s+2)x} - x e^{(s+1)x} = e^{(s+1)x}(e^x - x)$$
由于 $e^x > x$ 恒成立,核心差值严格为正。更进一步,由前文确立的严格下界 $e^x = x_1 > s$,意味着极高阶指数 $e^{(s+2)x} > s e^{(s+1)x}$ 拥有对低阶项的绝对放缩压制权。
这种代数结构保证了在起点($x \to 0$ 时 $E(x) \approx \frac{x^4}{24} > 0$)之后,全导数 $\frac{dE}{dx}$ 在开区间 $x \in (0, +\infty)$ 上严格正定。
由于 $\lim_{x \to 0^+} E(x) = 0$ 且全局单调递增, $E(x) > 0$ 得证
\end{solution}

\newpage
\begin{example}{来自扁头耄耋}{}
    已知$x_1-\ln x_1=x_2-\ln x_2,x_1\neq x_2$,求证\[(1)~~\frac{x_{1}\ln x_{1}}{x_{1}-1}+\frac{x_{2}\ln x_{2}}{x_{2}-1}>2~~~~~(2)~~ (x_1+\sqrt{x_1}+1)(x_2+\sqrt{x_2}+1)>9\]
\end{example}
\begin{solution}
设 $0 < x_1 < 1 < x_2$,由 $x_1 - \ln x_1 = x_2 - \ln x_2$ 移项得 $x_2 - x_1 = \ln x_2 - \ln x_1 = \ln\left(\frac{x_2}{x_1}\right)$。
令 $\frac{x_2}{x_1} = e^{2t}\ (t>0)$,则 $x_2 - x_1 = 2t$。将 $x_2 = x_1 e^{2t}$ 代回 $x_2 - x_1 = 2t$ 可得 $x_1(e^{2t}-1) = 2t$,解得:
\[ x_1 = \frac{2t}{e^{2t}-1} = \frac{2t e^{-t}}{e^t - e^{-t}} = \frac{t e^{-t}}{\sinh t},\ln x_1 = \ln\left(\frac{t}{\sinh t}\right) - t, x_2 = x_1 e^{2t} = \frac{t e^t}{\sinh t} ,\ln x_2 = \ln\left(\frac{t}{\sinh t}\right) + t\]
将上述关系代入待证不等式(1)的左端,并提取对数部分的公因式得:
\[ E = \ln\left(\frac{t}{\sinh t}\right) \left( \frac{x_1}{x_1 - 1} + \frac{x_2}{x_2 - 1} \right) + t \left( \frac{x_2}{x_2 - 1} - \frac{x_1}{x_1 - 1} \right) \]
对括号内的分式进行通分,其公共分母为 $A = (x_1 - 1)(x_2 - 1) = x_1 x_2 - (x_1 + x_2) + 1$。
代入 $x_1 x_2 = \frac{t^2}{\sinh^2 t}$ 与 $x_1 + x_2 = \frac{t(e^t+e^{-t})}{\sinh t} = \frac{2t \cosh t}{\sinh t}$ 进行计算,可得:
\[ A = \frac{t^2}{\sinh^2 t} - \frac{2t \cosh t}{\sinh t} + 1 = \frac{t^2 - 2t \sinh t \cosh t + \sinh^2 t}{\sinh^2 t} = \frac{t^2 - t \sinh 2t + \sinh^2 t}{\sinh^2 t} \]
记该式分子为 $f(t) = t^2 - t \sinh 2t + \sinh^2 t$,对其求导得 $f'(t) = 2t - (\sinh 2t + 2t \cosh 2t) + 2\sinh t \cosh t = 2t(1 - \cosh 2t)$。由于 $t > 0$ 时 $\cosh 2t > 1$,故 $f'(t) < 0$ 恒成立。结合 $f(0) = 0$ 可知对于 $t > 0$ 分子恒为负,从而恒有 $A < 0$。接着化简 $E$ 中的分子部分:
\[ \frac{x_1}{x_1 - 1} + \frac{x_2}{x_2 - 1} = \frac{2x_1 x_2 - (x_1 + x_2)}{A} = \frac{\frac{2t^2}{\sinh^2 t} - \frac{2t \cosh t}{\sinh t}}{A} = \frac{2(t^2 - t \sinh t \cosh t)}{A \sinh^2 t} \]
\[ \frac{x_2}{x_2 - 1} - \frac{x_1}{x_1 - 1} = \frac{x_2(x_1 - 1) - x_1(x_2 - 1)}{A} = \frac{x_1 - x_2}{A} = \frac{\frac{t e^{-t}}{\sinh t} - \frac{t e^t}{\sinh t}}{A} = \frac{-2t}{A} \]
将以上结果代回 $E$ 中并通分整理,可得:
\[ E = \frac{2 \ln \left( \frac{t}{\sinh t} \right) (t^2 - t \sinh t \cosh t) - 2t^2 \sinh^2 t}{A \sinh^2 t} \]
由 $A \sinh^2 t < 0$,要证 $E > 2$等价于证其分子小于 $2A \sinh^2 t = 2(t^2 - 2t \sinh t \cosh t + \sinh^2 t)$。
将不等式两边同除以 $2$ 并移项整理,化为:
\[ \ln \left( \frac{t}{\sinh t} \right) (t^2 - t \sinh t \cosh t) < t^2\sinh^2 t + t^2 - 2t \sinh t \cosh t + \sinh^2 t \]
提取右端前两项的公因式得到 $t^2(\sinh^2 t + 1) = t^2\cosh^2 t$,从而右端可完全平方化为 $(t \cosh t - \sinh t)^2$。
由于 $t>0$ 时 $\sinh 2t > 2t$,故左端因式 $t^2 - t \sinh t \cosh t = t(t - \sinh t \cosh t) < 0$。将不等式两边同除以该负因式并改变不等号方向,同时利用 $\ln \left( \frac{t}{\sinh t} \right) = - \ln \left( \frac{\sinh t}{t} \right)$ 抵消负号,问题转化为证明:
\[ K(t) = \ln \left( \frac{\sinh t}{t} \right) + \frac{(t \cosh t - \sinh t)^2}{t(t - \sinh t \cosh t)} < 0 \quad (t > 0) \]

为了求导化简,记 $g(t) = t \cosh t - \sinh t$,其导数 $g'(t) = t \sinh t$;
记 $H(t) = \sinh t \cosh t - t > 0$,其导数 $H'(t) = \cosh 2t - 1 = 2 \sinh^2 t$。
原函数可简记为 $K(t) = \ln(\sinh t) - \ln t - \frac{g(t)^2}{t H(t)}$。对数部分的导数为 $\frac{\cosh t}{\sinh t} - \frac{1}{t} = \frac{t\cosh t - \sinh t}{t\sinh t} = \frac{g(t)}{t \sinh t}$,可得:
\[ K'(t) = \frac{g(t)}{t \sinh t} - \frac{2g(t)g'(t)t H(t) - g(t)^2(H(t) + t H'(t))}{t^2 H(t)^2} \]
通分并提取严格为正的公因式 $\frac{g(t)}{t^2 H(t)^2 \sinh t}$,中括号内的剩余多项式为:
\begin{align*} M(t)& = t H(t)^2 - 2t \sinh t g'(t) H(t) + g(t) \sinh t (H(t) + t H'(t)) \\
  &  = t H(t)^2 - 2t^2 \sinh^2 t H(t) + g(t) \sinh t (H(t) + 2t \sinh^2 t) \end{align*}
代入 $g(t), H(t)$ 的表达式,展开后利用双曲函数的倍角公式合并同类项(如 $\cosh 2t = 1 + 2\sinh^2 t$ 等),可得核心多项式:
\[ M(t) = t^3 \cosh 2t - \frac{3}{2} t^2 \sinh 2t + \frac{3}{2} t (\cosh 2t - 1) - \frac{1}{4} \sinh 2t (\cosh 2t - 1) \]

令 $u = 2t > 0$,为消除分数令 $P(u) = 8M(u/2) = u^3 \cosh u - 3u^2 \sinh u + 6u \cosh u - 6u - \sinh 2u + 2 \sinh u$。
利用 $\displaystyle\cosh u = \sum_{n=0}^\infty \frac{u^{2n}}{(2n)!},\sinh u = \sum_{n=0}^\infty \frac{u^{2n+1}}{(2n+1)!}$ 将其展开为泰勒级数 $\displaystyle P(u) = \sum_{n=0}^\infty c_n u^{2n+1}$。
对于 $n \ge 1$,提取各展开式中 $u^{2n+1}$ 的系数并提取公共分母 $(2n+1)!$:
\begin{align*} c_n &= \frac{1}{(2n-2)!} - \frac{3}{(2n-1)!} + \frac{6}{(2n)!} - \frac{2^{2n+1}}{(2n+1)!} + \frac{2}{(2n+1)!} \\&= \frac{2n(2n-1)(2n+1) - 3(2n)(2n+1) + 6(2n+1) - 2^{2n+1} + 2}{(2n+1)!} \end{align*}
化简分子,前三项多项式之和为 $(2n+1)(4n^2 - 8n + 6) = 8n^3 - 12n^2 + 4n + 6$,故合并后通项系数为 $c_n = \frac{8n^3 - 12n^2 + 4n + 8 - 2^{2n+1}}{(2n+1)!}$。
由基础低次项相抵消易知 $c_0=c_1=c_2=c_3=0$。
当 $n \ge 4$ 时,利用数学归纳法证明指数项占优,即 $2^{2n+1} > 8n^3 - 12n^2 + 4n + 8$:
基础情形:当 $n=4$ 时,$2^9 = 512 > 8(64) - 12(16) + 4(4) + 8 = 344$ 成立;
递推过程:假设 $n=k \ge 4$ 时不等式成立,则当 $n=k+1$ 时,左侧变为 $4 \times 2^{2k+1} > 4(8k^3 - 12k^2 + 4k + 8)$。只需证其大于目标值 $8(k+1)^3 - 12(k+1)^2 + 4(k+1) + 8$。两式作差得 $24k^3 - 60k^2 + 12k + 24 = 12(k^2(2k - 5) + k + 2)$,因 $k \ge 4$ 故该差值显然大于 $0$,归纳法得证。
故当 $n \ge 4$ 时恒有 $c_n < 0$。因此对于任意 $u>0$ 有 $P(u) < 0$,进而 $M(t) < 0$,即 $K'(t) < 0$。结合 $\lim_{t \to 0^+} K(t) = 0$ 可知对任意 $t > 0$ 均有 $K(t) < 0$。得证。

(2)记 $a = \sqrt{x_1}, b = \sqrt{x_2}$,即证:
\[ (a^2 + a + 1)(b^2 + b + 1) = a^2 b^2 + ab(a+b) + a^2 + b^2 + ab + a + b + 1 > 9 \]
令 $\theta = t/2 > 0$,承接前面的参数方程直接开方可得:
\[ a = e^{-\theta} \sqrt{ \frac{2\theta}{\sinh 2\theta} }, \quad b = e^{\theta} \sqrt{ \frac{2\theta}{\sinh 2\theta} } \]
令两根之积 $P = ab = \frac{2\theta}{\sinh 2\theta}$,两根之和 $S = a + b = \sqrt{\frac{2\theta}{\sinh 2\theta}} (e^{-\theta} + e^\theta) = \sqrt{\frac{2\theta}{2\sinh \theta \cosh \theta}} \cdot 2\cosh \theta = 2 \sqrt{\theta \coth \theta}$。
因 $\theta > 0$ 时 $\sinh 2\theta > 2\theta$,易知 $P \in (0, 1)$;因 $\theta \coth \theta > 1$ 恒成立,故 $S > 2$。
利用 $a^2 + b^2 = S^2 - 2P$ 将展开式化简重组,原命题等价于证明:
\[ P^2 + PS + S^2 - 2P + P + S + 1 > 9 \iff F(P, S) = P^2 + PS + S^2 - P + S - 8 > 0 \]

为严谨证明该式,按 $\theta$ 的范围分为两段进行放缩:

当 $0 < \theta \le 1.3$ 时,利用泰勒展开 $\sinh u = u + \frac{u^3}{6} + \frac{u^5}{120} + \dots = \sum_{n=0}^\infty \frac{u^{2n+1}}{(2n+1)!}$ 可得:
\[ \left(1 - \frac{u^2}{6}\right)\sinh u = u + \sum_{n=1}^\infty \left( \frac{1}{(2n+1)!} - \frac{1}{6(2n-1)!} \right) u^{2n+1} = u + \sum_{n=1}^\infty \frac{1 - 2n(2n+1)/6}{(2n+1)!} u^{2n+1} \]
当 $n=1$ 时该项系数为 $0$,当 $n \ge 2$ 时分子 $1 - n(2n+1)/3 \le 1 - 10/3 < 0$,故对任意 $u>0$ 恒有 $\left(1 - \frac{u^2}{6}\right)\sinh u < u$。令 $u = 2\theta$ 即可得严密的局部下界 $P = \frac{2\theta}{\sinh 2\theta} > 1 - \frac{2}{3}\theta^2 \triangleq P_0$。
利用 Mittag-Leffler 展开式 $\theta \coth \theta = 1 + 2\sum_{n=1}^\infty \frac{\theta^2}{\theta^2 + n^2 \pi^2}$,结合基本代数不等式 $\frac{1}{A+B} > \frac{1}{A} - \frac{B}{A^2}$ 进行放缩,并代入已知级数 $\sum_{n=1}^\infty \frac{1}{n^2} = \frac{\pi^2}{6}, \sum_{n=1}^\infty \frac{1}{n^4} = \frac{\pi^4}{90}$ 可得:
\[ \theta \coth \theta > 1 + 2\sum_{n=1}^\infty \left( \frac{\theta^2}{n^2\pi^2} - \frac{\theta^4}{n^4\pi^4} \right) = 1 + \frac{2\theta^2}{\pi^2} \cdot \frac{\pi^2}{6} - \frac{2\theta^4}{\pi^4} \cdot \frac{\pi^4}{90} = 1 + \frac{\theta^2}{3} - \frac{\theta^4}{45} \]
将多项式 $\left(1 + \frac{\theta^2}{6} - \frac{\theta^4}{40}\right)^2 = 1 + \frac{\theta^2}{3} - \frac{\theta^4}{45} - \frac{\theta^6}{120} + \frac{\theta^8}{1600}$ 与前述下界作差,差值为 $\frac{\theta^6}{120} - \frac{\theta^8}{1600} = \frac{\theta^6}{1600}\left(\frac{40}{3} - \theta^2\right)$。
在 $\theta \le 1.3$ 时 $\theta^2 \le 1.69 < \frac{40}{3}$,该差值严格大于 $0$,这说明放缩所得的下界严格大于该多项式。开方即得 $S > 2\left(1 + \frac{\theta^2}{6} - \frac{\theta^4}{40}\right) \triangleq S_0$。

为了说明 $P$ 和 $S$ 放缩后目标函数依然变小,利用代数变形分解目标函数的差值,将交叉项的差改写为 $PS - P_0 S_0 = S(P - P_0) + P_0(S - S_0)$ 并提取公因式:
\begin{align*}
F(P, S) - F(P_0, S_0) &= (P^2 - P_0^2) + (PS - P_0 S_0) + (S^2 - S_0^2) - (P - P_0) + (S - S_0) \\
&= (P - P_0)(P + P_0 + S - 1) + (S - S_0)(S + S_0 + P_0 + 1)
\end{align*}
由于 $\theta \le 1.3$ 时 $P_0 \ge 1 - \frac{2}{3}(1.3)^2 > -0.13$,结合 $P>0, S>2, S_0>2$,上述两项的后置括号均严格大于 $0$(例如 $P+P_0+S-1 > 0 - 0.13 + 2 - 1 > 0$),这就逻辑严密地确保了向下放缩的合法性,即 $F(P, S) > F(P_0, S_0)$。
将 $P_0, S_0$ 代入多项式展开并按次数整理如下:
$P_0^2 = 1 - \frac{4}{3}\theta^2 + \frac{4}{9}\theta^4$
$P_0 S_0 = \left(1 - \frac{2}{3}\theta^2\right)\left(2 + \frac{1}{3}\theta^2 - \frac{1}{20}\theta^4\right) = 2 - \theta^2 - \frac{49}{180}\theta^4 + \frac{1}{30}\theta^6$
$S_0^2 = \left(2 + \frac{1}{3}\theta^2 - \frac{1}{20}\theta^4\right)^2 = 4 + \frac{4}{3}\theta^2 - \frac{4}{45}\theta^4 - \frac{1}{30}\theta^6 + \frac{1}{400}\theta^8$
$-P_0 + S_0 - 8 = -\left(1 - \frac{2}{3}\theta^2\right) + \left(2 + \frac{1}{3}\theta^2 - \frac{1}{20}\theta^4\right) - 8 = -7 + \theta^2 - \frac{1}{20}\theta^4$
将上述四式相加,奇迹般地发现常数项($1+2+4-7=0$)、$\theta^2$项系数($-\frac{4}{3}-1+\frac{4}{3}+1=0$)及$\theta^6$项系数($\frac{1}{30}-\frac{1}{30}=0$)恰好完全抵消,而$\theta^4$项系数为 $\frac{4}{9} - \frac{49}{180} - \frac{4}{45} - \frac{1}{20} = \frac{1}{30}$。仅余高阶正项:
\[ F(P_0, S_0) = \frac{1}{30}\theta^4 + \frac{1}{400}\theta^8 > 0 \]
故当 $0 < \theta \le 1.3$ 时 $F(P, S) > 0$ 严格成立。

当 $\theta > 1.3$ 时,函数 $x(\theta) = \theta \coth \theta$ 的导数 $x'(\theta) = \coth \theta - \theta \operatorname{csch}^2 \theta = \frac{\sinh 2\theta - 2\theta}{2\sinh^2\theta} > 0$,故该函数在 $(0, +\infty)$ 上单调递增。
此时 $x(\theta) > 1.3 \coth 1.3 \approx 1.5086$,从而 $S = 2\sqrt{x(\theta)} > 2\sqrt{1.5086} \approx 2.4565$。
将目标函数重构为 $F(P, S) = S^2 + S(P+1) + (P^2 - P) - 8$。
因 $P \in (0, 1)$,必有 $P+1 > 1$,且二次函数 $P^2 - P = (P-0.5)^2 - 0.25$ 恒大于其极小值 $-0.25$。代入全局界限进行放缩:
\[ F(P, S) > 2.4565^2 + 2.4565 \times 1 - 0.25 - 8 \approx 6.034 + 2.456 - 8.25 = 0.24 > 0 \]
故当 $\theta > 1.3$ 时 $F(P, S) > 0$ 同样成立。综上对于所有满足条件的 $x_1, x_2$不等式均恒成立。

\end{solution}

\begin{example}{黎曼杯T18加强}{}
     $f(x) = \e^x-x\ln x - kx - 1$.若函数 $f(x)$ 有两个零点,将其分别记为 $x_1,x_2$.\\(1)试求 $k$ 的取值范围;\\(2)证明:$\displaystyle k>x_1+\ln x_2+\frac{\sqrt2}{2}$\\
     (3)证明:$(\e-2)x_1+x_2-\ln(x_1x_2)\geqslant k-1+\ln(k+1)$
\end{example}
\begin{solution}
\textbf{(1)} 令 $g(x)=\dfrac{\e^x}{x}-\ln x-\dfrac{1}{x}$,则原方程等价于 $g(x)=k$.求导得
\[
g'(x)=\frac{(x-1)(\e^x-1)}{x^2}.
\]
由于 $x>0$ 时 $\e^x-1>0$,故 $g'(x)$ 的符号由 $x-1$ 决定:当 $0<x<1$ 时 $g'(x)<0$,当 $x>1$ 时 $g'(x)>0$.因此 $g(x)$ 在 $(0,1)$ 上单调递减,在 $(1,+\infty)$ 上单调递增,在 $x=1$ 处取得最小值 $g(1)=\e-1$.又 $\lim\limits_{x\to0^+}g(x)=+\infty$,$\lim\limits_{x\to+\infty}g(x)=+\infty$,故对任意 $k>\e-1$,方程 $g(x)=k$ 恰有两个不同的实根;而当 $k\le \e-1$ 时,方程至多有一个实根.因此 $k$ 的取值范围是 $(\e-1,+\infty)$.

\textbf{(2)} 设 $f(x)$ 的两个零点为 $\alpha,\beta$,且 $0<\alpha<1<\beta$.由 $f(\alpha)=0$ 得
\[
k=\frac{\e^\alpha}{\alpha}-\ln\alpha-\frac1\alpha.
\]
要证 $k>\alpha+\ln\beta+\dfrac{\sqrt2}{2}$,即证
\[
\frac{\e^\alpha}{\alpha}-\ln\alpha-\frac1\alpha>\alpha+\ln\beta+\frac{\sqrt2}{2},
\]
整理得
\[
\ln(\alpha\beta)<\frac{\e^\alpha}{\alpha}-\alpha-\frac1\alpha-\frac{\sqrt2}{2}.
\]
构造函数 $G(x)=\dfrac{\e^x}{x}-x-\dfrac1x-\dfrac{\sqrt2}{2}\quad(x>0)$,求导得
\[
G'(x)=\frac{(x-1)(\e^x-x-1)}{x^2}.
\]
由 $\e^x\ge x+1$(当且仅当 $x=0$ 取等),知当 $x>0$ 时 $\e^x-x-1>0$,故 $G'(x)$ 的符号由 $x-1$ 决定:$0<x<1$ 时 $G'(x)<0$,$x>1$ 时 $G'(x)>0$.因此 $G(x)$ 在 $(0,1)$ 上单调递减,在 $(1,+\infty)$ 上单调递增,最小值 $G(1)=\e-2-\dfrac{\sqrt2}{2}>0$,从而 $G(x)>0$ 恒成立.于是欲证原不等式,只需证 $\ln(\alpha\beta)<0$,即 $\alpha\beta<1$.

下证 $\alpha\beta<1$.考虑函数 $F(x)=I(x)-I\!\left(\dfrac1x\right)$,其中 $I(x)=\e^x-x\ln x-kx-1$.对 $x\ge1$ 求导整理得
\[
F'(x)=\frac{(x-1)\bigl(\e^x-x\e^{1/x}+x-1\bigr)}{x^2}.
\]
令 $m(x)=\e^x-x\e^{1/x}+x-1$,则
\[
m'(x)=\e^x-\e^{1/x}+\frac{\e^{1/x}}{x^2}+1>0\quad(x>1),
\]
且 $m(1)=0$,故 $m(x)>0$ 对 $x>1$ 恒成立.因此 $F'(x)>0$($x>1$),$F(x)$ 在 $[1,+\infty)$ 上单调递增,且 $F(1)=0$,从而 $F(x)\ge0$,即 $I(x)\ge I(1/x)$ 对 $x\ge1$ 成立.取 $x=\beta>1$,得 $I(\beta)\ge I(1/\beta)$.由 $I(\beta)=0$ 知 $I(1/\beta)\le0$.

另一方面,由第(1)问定义的 $g(x)=\dfrac{\e^x}{x}-\ln x-\dfrac1x$ 在 $(0,1)$ 上单调递减,且 $g(\alpha)=k$,易知 $I(x)=x\bigl(g(x)-k\bigr)$.因为 $g(x)-k$ 在 $(0,1)$ 上严格递减且恰有一个零点 $x=\alpha$,所以当 $x\in(0,\alpha)$ 时 $g(x)-k>0$,$I(x)>0$;当 $x\in(\alpha,1)$ 时 $g(x)-k<0$,$I(x)<0$.于是由 $I(1/\beta)\le0$ 及 $1/\beta\in(0,1)$ 可得 $1/\beta\ge\alpha$,即 $\alpha\beta\le1$.若 $\alpha\beta=1$,则 $1/\beta=\alpha$,代入得 $I(\alpha)=I(\beta)=0$ 且 $\beta=1/\alpha$,但此时 $g(\alpha)=g(1/\alpha)$.容易验证 $g(x)\neq g(1/x)$ 对 $x\in(0,1)\cup(1,+\infty)$ 成立(例如由 $F(x)$ 的性质可推知),矛盾.因此 $\alpha\beta<1$.

综上,$\ln(\alpha\beta)<0$,从而原不等式得证.\\
(3)设函数 $g(x) = \frac{\e^x - 1}{x} - \ln x$,有 $k > \e - 1$.不妨设 $0 < x_1 < 1 < x_2$.对于较大根 $x_2 \in (1, +\infty)$,由 $g(x_2) = k$ 得 $\frac{\e^{x_2} - 1}{x_2} - \ln x_2 = k$,即 $\e^{x_2} - 1 = x_2(k + \ln x_2)$.将其恒等变形为 $\e^{x_2} = x_2\left(k + \ln x_2 + \frac{1}{x_2}\right)$.因为 $x_2 > 1$ 且 $\e^{x_2} > 0$,等式两边均为正,取对数得$$ x_2 = \ln x_2 + \ln\left(k + \ln x_2 + \frac{1}{x_2}\right)\Leftrightarrow x_2 - \ln x_2 = \ln\left(k + \ln x_2 + \frac{1}{x_2}\right)> \ln(k+1) $$。
对于较小根 $x_1 \in (0, 1)$和 $k = \frac{\e^{x_1} - 1}{x_1} - \ln x_1$.要证关于 $x_1$ 的不等式 $(\e-2)x_1 - \ln x_1 > k - 1$,即证$$ (\e-2)x_1 - \ln x_1 > \frac{\e^{x_1} - 1}{x_1} - \ln x_1 - 1\Leftrightarrow (\e-2)x_1^2 + x_1 + 1 - \e^{x_1} > 0$$
构造函数 $h(x) = (\e-2)x^2 + x + 1 - \e^x\ (0 \leqslant x \leqslant 1)$.求一阶导数得 $h'(x) = 2(\e-2)x + 1 - \e^x$,求二阶导数得 $h''(x) = 2\e - 4 - \e^x$.因为函数 $y = \e^x$ 在 $[0, 1]$ 上单调递增,所以 $h''(x)$ 在 $[0, 1]$ 上单调递减.又 $h''(0) = 2\e - 5 > 0$(因 $\e > 2.5$)且 $h''(1) = \e - 4 < 0$,根据零点存在定理,存在唯一的 $x_0 \in (0, 1)$ 使得 $h''(x_0) = 0$.当 $x \in (0, x_0)$ 时,$h''(x) > 0$,$h'(x)$ 单调递增;当 $x \in (x_0, 1)$ 时,$h''(x) < 0$,$h'(x)$ 单调递减.由于 $h'(0) = 0$ 且 $h'(1) = \e - 3 < 0$,故 $h'(x)$ 在区间 $(0, 1)$ 内从 $0$ 开始先增至正数,再递减至负数.因此存在唯一的 $t_0 \in (0, 1)$ 使得 $h'(t_0) = 0$.当 $x \in (0, t_0)$ 时,$h'(x) > 0$,$h(x)$ 单调递增;当 $x \in (t_0, 1)$ 时,$h'(x) < 0$,$h(x)$ 单调递减.计算两端点值,有 $h(0) = 0$ 且 $h(1) = (\e-2) + 1 + 1 - \e = 0$.由于 $h(x)$ 在 $(0, 1)$ 上先增后减,且两端点值均为 $0$,故对于任意 $x \in (0, 1)$,恒有 $h(x) > 0$.因为 $x_1 \in (0, 1)$,故 $h(x_1) > 0$ 严格成立,即 $(\e-2)x_1^2 + x_1 + 1 - \e^{x_1} > 0$ 成立.

从而原题中所要求证明的不等式 $(\e-2)x_1+x_2-\ln(x_1x_2)\geqslant k-1+\ln(k+1)$ 显然成立.
\end{solution}
\newpage
\begin{example}{来自群友“港”}{}
    曲线$\varGamma :x^2+\ln^2y=\left(\dfrac{\e}{\e+1}\right)^2$,$AB$为$\varGamma$的一条弦.\\
    (1)求$\varGamma$的最高点$P$,最低点$Q$的坐标.\\
    (2)若$AB$的斜率为0,求$S_{\triangle{PAB}}$的最大值.\\
    (3)求$S_{\triangle{PAB}}$的最大值.
\end{example}
\begin{solution}
记 $R = \frac{e}{e+1}$,则方程化为 $x^{2} + (\ln y)^{2} = R^{2}$。由此可得参数方程:
$$\begin{cases}
x = R \cos \theta, \\
y = e^{R \sin \theta},
\end{cases}
\quad \theta \in [0, 2\pi).$$
(1) 当$\sin\theta=1$时$y$ 取最大值$e^R$,此时 $x=0$;当$\sin\theta=-1$ 时 $y$ 取最小值$e^{-R}$,此时$x=0$。故最高点为$P(0,e^R)$,最低点为$Q(0,e^{-R})$,即
$P\left(0,e^{{\frac{e}{e+1}}}\right),\quad Q\left(0,e^{{-\frac{e}{e+1}}}\right).$\\
(2) 由于直线 $l$ 斜率为 $0$,即水平线,由曲线关于 $y$ 轴对称,可设 $A$ 在左半平面,$B$ 在右半平面,且 $A$ 与 $B$ 关于 $y$ 轴对称。设 $B$ 对应的参数为 $\theta$($\cos\theta > 0$),则 $B(R\cos\theta, \mathrm{e}^{R\sin\theta})$,$A(-R\cos\theta, \mathrm{e}^{R\sin\theta})$。于是 $\triangle PAB$ 的底边长 $AB = 2R\cos\theta$,高为 $\mathrm{e}^{R} - \mathrm{e}^{R\sin\theta}$,面积
\[
S(\theta) = R\cos\theta\,(\mathrm{e}^{R} - \mathrm{e}^{R\sin\theta}),\quad \theta\in(-\tfrac{\pi}{2},\tfrac{\pi}{2}).
\]
令$t=R\sin\theta$,则
\[
S(t) = \sqrt{R^2-t^{2}}\,(\mathrm{e}^{R} - \mathrm{e}^{t}).
\]
为求最大值,考虑函数 $\varphi(t) = \ln S(t)$:
\begin{align*}
\varphi(t) &= \frac{1}{2}\ln(R^2 - t^2) + \ln(\mathrm{e}^{R} - \mathrm{e}^{t}),~~\varphi'(t) = -\frac{t}{R^2 - t^2} - \frac{\mathrm{e}^{t}}{\mathrm{e}^{R} - \mathrm{e}^{t}}\\
\varphi'(t)=0&\Leftrightarrow t\e^R + \e^t(R^2 - t^2 - t) = 0\Leftrightarrow t\e^{\frac{\e}{\e+1}} + \e^t\left(\left(\frac{\e}{\e+1}\right)^2 - t^2 - t\right) = 0\\
&\Leftrightarrow (\e^t-\e^{-\frac1{\e+1}})\left(\left(\frac{\e}{\e+1}\right)^2 - t^2 - t\right)+\e^{-\frac1{\e+1}}\left(t+\frac1{\e+1}\right)\left(\frac{\e^2}{\e+1}-t\right)=0
\end{align*}
得到驻点$t=-\frac{1}{\e+1}$,再求二阶导:
\[
\varphi''(t) = -\frac{(R^2 - t^2) - t(-2t)}{(R^2 - t^2)^2} - \frac{\mathrm{e}^{t}(\mathrm{e}^{R} - \mathrm{e}^{t}) - \mathrm{e}^{t}(-\mathrm{e}^{t})}{(\mathrm{e}^{R} - \mathrm{e}^{t})^2} = -\frac{R^2 + t^2}{(R^2 - t^2)^2} - \frac{\mathrm{e}^{t}\mathrm{e}^{R}}{(\mathrm{e}^{R} - \mathrm{e}^{t})^2} < 0
\]
因此 $\varphi'(t)$ 在 $(-R, R)$ 上严格单调递减,故方程 $\varphi'(t) = 0$ 至多有一个实根$t=-\frac{1}{\e+1}$。由于 $S(t)$ 在端点 $t = \pm R$ 处为零,在 $(-R, R)$ 内为正,故该驻点即为最大值点。此时 $c = \mathrm{e}^{t} = \mathrm{e}^{-\frac{1}{\mathrm{e}+1}}$,代入得
\[
S_{\max} = \sqrt{R^2 - \left(-\frac{1}{\mathrm{e}+1}\right)^2}\left(\mathrm{e}^{R} - \mathrm{e}^{-\frac{1}{\mathrm{e}+1}}\right) = \sqrt{\frac{\mathrm{e}-1}{\mathrm{e}+1}} \cdot \mathrm{e}^{-\frac{1}{\mathrm{e}+1}}(\mathrm{e}-1) 
\]
(3) 对于一般直线 $l$,设 $A(R\cos\theta_1, \mathrm{e}^{R\sin\theta_1})$,$B(R\cos\theta_2, \mathrm{e}^{R\sin\theta_2})$,则 $\triangle PAB$ 面积为
\[
S(\theta_1,\theta_2) = \frac{R}{2}\left| \cos\theta_1 (\mathrm{e}^{R\sin\theta_2} - \mathrm{e}^{R}) - \cos\theta_2 (\mathrm{e}^{R\sin\theta_1} - \mathrm{e}^{R}) \right|.
\]
由面积表达式及三角不等式得
\[
S \le \frac{R}{2} \left( |\cos\theta_1| (\mathrm{e}^{R} - \mathrm{e}^{R\sin\theta_2}) + |\cos\theta_2| (\mathrm{e}^{R} - \mathrm{e}^{R\sin\theta_1}) \right) \triangleq H(\theta_1,\theta_2),
\]
等号成立当且仅当 \(\cos\theta_1\) 与 \(\cos\theta_2\) 异号。令 \(u=\sin\theta_1\), \(v=\sin\theta_2\),则
\[
H(u,v) = \frac{R}{2} \left( \sqrt{1-u^2}\,(\mathrm{e}^{R} - \mathrm{e}^{Rv}) + \sqrt{1-v^2}\,(\mathrm{e}^{R} - \mathrm{e}^{Ru}) \right).
\]
求偏导数并令其为零,得到方程组
\[
\frac{u}{\sqrt{1-u^2}}(\mathrm{e}^{R} - \mathrm{e}^{Rv}) = -R\sqrt{1-v^2}\,\mathrm{e}^{Ru}, \quad
\frac{v}{\sqrt{1-v^2}}(\mathrm{e}^{R} - \mathrm{e}^{Ru}) = -R\sqrt{1-u^2}\,\mathrm{e}^{Rv}.
\]
由对称性知 \(u,v<0\),令 \(a=-u>0\), \(b=-v>0\),则方程组化为
\[
\frac{a}{\sqrt{1-a^2}}(\mathrm{e}^{R} - \mathrm{e}^{-Rb}) = R\sqrt{1-b^2}\,\mathrm{e}^{-Ra}, \quad
\frac{b}{\sqrt{1-b^2}}(\mathrm{e}^{R} - \mathrm{e}^{-Ra}) = R\sqrt{1-a^2}\,\mathrm{e}^{-Rb}.
\]
两式相除并整理得
\[
\frac{a}{b} \cdot \frac{\mathrm{e}^{R} - \mathrm{e}^{-Rb}}{\mathrm{e}^{R} - \mathrm{e}^{-Ra}} = \mathrm{e}^{-R(a-b)}.
\]
取对数后移项得到
\[
\ln a + \ln(\mathrm{e}^{R} - \mathrm{e}^{-Ra}) + Ra = \ln b + \ln(\mathrm{e}^{R} - \mathrm{e}^{-Rb}) + Rb.
\]
定义函数 \(K(t)=\ln t + \ln(\mathrm{e}^{R} - \mathrm{e}^{-Rt}) + Rt=\ln t + \ln(\mathrm{e}^{\frac{\e}{\e+1}} - \mathrm{e}^{-\frac{\e}{\e+1}t}) + \frac{\e}{\e+1}t\),则由复合函数单调性得到 \(K(t)\) 严格递增,从而 \(a=b\),即 \(u=v\)。因此极值点必满足 \(u=v\),此时
\[
H(u,u)=R\sqrt{1-u^2}(\mathrm{e}^{R} - \mathrm{e}^{Ru}),
\]
与 (2) 中形式相同,由 (2) 知该函数的最大值在 $u = -\frac{1}{\mathrm{e}}$ 处取得,同时,等号条件要求 $\cos\theta_1$ 与 $\cos\theta_2$ 异号,故取 $\theta_2 = \pi - \theta_1$,此时直线为水平线 $y = \mathrm{e}^{Ru} = \mathrm{e}^{-\frac{1}{\mathrm{e}+1}}$,且 $S$ 达到该最大值。由于 $H(u,v)$ 在边界 $u,v=\pm1$ 上为零,而内点值大于零,故该驻点即为全局最大值点。综上,$\triangle PAB$ 面积的最大值为$\displaystyle  \sqrt{\frac{\mathrm{e}-1}{\mathrm{e}+1}} \cdot \mathrm{e}^{-\frac{1}{\mathrm{e}+1}}(\mathrm{e}-1) $
\end{solution}


\newpage
\begin{example}{}{}
    (1)若$\theta\in\left(0,\frac{\pi}2\right)$,判断$2\cos^2\theta$与$\left(\frac{1}{n}\cos 2\theta+1\right)^n$的大小关系,并证明.\\
    (2)证明:对于任意自然数$n$,$\theta\in\left(0,\frac{\pi}2\right)$,有
\begin{gather*}
\left(\sin^2\theta+\sin^3\theta+\cdots+\sin^{2n}\theta+\sin^{2n+1}\theta\right)+\left(\cos^2\theta+\cos^3\theta+\cdots+\cos^{2n}\theta+\cos^{2n+1}\theta\right)\\
\geq\left(\sqrt{2}+2\right)\left(1-\frac{1}{2^n}\right)
\end{gather*}
\end{example}
\begin{solution}
(1)取$\theta=0$探路,显然有$2\leqslant \left(1+\frac1n\right)^n$,于是考虑证明$2\cos^2\theta\leqslant\left(\frac{1}{n}\cos 2\theta+1\right)^n$,即$1+\cos2\theta\leqslant \left(\frac{1}{n}\cos 2\theta+1\right)^n$,取对数即$\ln(1+\cos2\theta)\leqslant n\ln\left(1+\frac{\cos2\theta}{n}\right)$,改造成:
\begin{align*}
    f(x)&=n\ln\left(1+\frac{x}{n}\right)-\ln(1+x)\\
    f'(x)&=\frac{1}{1+\frac{x}{n}}-\frac{1}{1+x}=\frac{(1+x)-\bigl(1+\frac{x}{n}\bigr)}{\bigl(1+\frac{x}{n}\bigr)(1+x)}=\frac{x\bigl(1-\frac1n\bigr)}{\bigl(1+\frac{x}{n}\bigr)(1+x)}.
\end{align*}
因此$f(x)$在$(-1,0]$上单调递减,在$[0,1]$上单调递增,从而$f(x)\geqslant f(0)=0$对任意$x\in(-1,1]$成立,等号仅当$x=0$时取得。于是$1+\cos2\theta\leqslant\bigl(1+\frac{\cos2\theta}{n}\bigr)^n$,即$2\cos^2\theta\leqslant\bigl(\frac{1}{n}\cos2\theta+1\bigr)^n$,等号当$\cos2\theta=0$,即$\theta=\frac\pi4$时成立。\\
(2)由二倍角公式,\(\sin^2\theta = \frac{1-\cos2\theta}{2}\),\(\cos^2\theta = \frac{1+\cos2\theta}{2}\),则对于任意正整数 \(m \geq 2\),
\[
\sin^m\theta + \cos^m\theta = 2^{-\frac{m}{2}} \left[ (1-\cos2\theta)^{\frac{m}{2}} + (1+\cos2\theta)^{\frac{m}{2}} \right].
\]
令 \(x = \cos2\theta \in (-1,1)\),则需证的和式为
\[
S(\theta) = \sum_{m=2}^{2n+1} 2^{-\frac{m}{2}} \left[ (1-x)^{\frac{m}{2}} + (1+x)^{\frac{m}{2}} \right].
\]
对于每个 \(k = \frac{m}{2} \in \{1, 1.5, 2, \dots, n+0.5\}\),考虑函数 \(\varphi_k(x) = (1-x)^k + (1+x)^k\)。由于 \(\varphi_k(x)\) 是偶函数,且当 \(x > 0\) 时,
\[
\varphi_k'(x) = -k(1-x)^{k-1} + k(1+x)^{k-1} = k\left[ (1+x)^{k-1} - (1-x)^{k-1} \right] > 0,
\]
所以 \(\varphi_k(x)\) 在 \([0,1)\) 上单调递增,从而在 \(x=0\) 处取最小值 \(\varphi_k(0)=2\)。因此
\[
(1-x)^{\frac{m}{2}} + (1+x)^{\frac{m}{2}} \geq 2\Rightarrow \sin^m\theta + \cos^m\theta \geq 2 \cdot 2^{-\frac{m}{2}} = 2^{1-\frac{m}{2}}.
\]
对 \(m\) 从 \(2\) 到 \(2n+1\) 求和得
\[
S(\theta) \geq \sum_{m=2}^{2n+1} 2^{1-\frac{m}{2}} = 2 \sum_{m=2}^{2n+1} 2^{-\frac{m}{2}} = 2 \cdot \frac{(\frac{1}{\sqrt{2}})^2 \left(1 - (\frac{1}{\sqrt{2}})^{2n}\right)}{1 - \frac{1}{\sqrt{2}}} = (2+\sqrt{2})\left(1 - \frac{1}{2^n}\right).
\]
等号成立当且仅当 \(x=0\),即 \(\cos2\theta = 0\),亦即 \(\theta = \frac{\pi}{4}\)。因此原不等式得证。
\end{solution}
\newpage
\begin{example}{来自“扁头耄耋”}{}
    证明:曲线\[y=(x-\ln x)\e^{1-x}\]上不存在不同的两点关于直线$y=x$对称
\end{example}
\begin{solution}
设$f(x)=x-\ln x,g(x)=(x-\ln x)\e^{1-x}$,假设存在不同的两点关于$y=x$对称,这等价于存在$x_1< x_2$使得$x_2=g(x_1),x_1=g(x_2)$,那么
\begin{align*}
    &x_2=g(x_1)=(x_1-\ln x_1)\e^{1-x_1},x_1=g(x_2)=(x_2-\ln x_2)\e^{1-x_2}\\
\Leftrightarrow& \ln x_2=\ln(x_1-\ln x_1)+1-x_1, \ln x_1=\ln(x_2-\ln x_2)+1-x_2\\
\Rightarrow &x_1-\ln x_1-\ln(x_1-\ln x_1)=x_2-\ln x_2-\ln(x_2-\ln x_2)\\
\Leftrightarrow& f(x_1-\ln x_1)=f(x_2-\ln x_2)\\
\Leftrightarrow& f(f(x_1))=f(f(x_2))
\end{align*}
由于$x_1\neq x_2$,那么$f(x_1),f(x_2)>1$,必有
\[f(x_1)=f(x_2)\Leftrightarrow x_1-\ln x_1=x_2-\ln x_2\]
再根据对称性,进一步设$(x_1,g(x_1)),(x_2,g(x_2))$位于直线$x+y=b$上,则
\[x+g(x)=b\Leftrightarrow x_1+(x_1-\ln x_1)\e^{1-x_1}=x_2+(x_2-\ln x_2)\e^{1-x_2}\]
下证这两个式子不兼容,我们可以证明由第一个式子可证第二个式子不成立,下证$x_1+(x_1-\ln x_1)\e^{1-x_1}>x_2+(x_2-\ln x_2)\e^{1-x_2}$,构造函数$h(x)=g(x)-2f(x)$,可证$h(x)$单减
\begin{align*}
h(x)&=(x-\ln x)\e^{1-x}+x-2(x-\ln x)=(x-\ln x)(e^{1-x}-2)+x\\
h'(x)&=\left(1-\frac1x\right)(e^{1-x}-2)+(x-\ln x)(-e^{1-x})+1=-\left[e^{1-x}\left(x-1-\ln x+\frac1x\right)-\frac2x+1\right]\\
&\leqslant -\left[\frac{e^{1-x}}{x}-\frac2x+1\right]=\frac{-1}{x}\left(e^{1-x}-(1-x)-1\right)\leqslant 0
\end{align*}
于是$h(x)$单减,且由假设$x_1-\ln x_1=x_2-\ln x_2$,则$g(x_1)>g(x_2)$,两式不兼容,因而矛盾,结论得证。
\end{solution}
\newpage


\begin{example}{邪帝导数题}{}
    已知函数$f(x)=x-(a+1)\ln x-\frac{a}{x},(a>1)$\\
    (1)讨论$f(x)$的单调性;\\
    (2)若$f(x_1)=f(x_2)=f(x_3),(x_1<x_2<x_3)$\\
    ~~~~(i)证明$f(x_1x_2x_3)>1-a$.\\
    ~~~~(ii)证明$f(x_1+x_2+x_3)<2e-1-a$.
\end{example}
\begin{solution}
(1)已知函数$f(x)=x-(a+1)\ln x-\frac ax$,定义域为$(0,+\infty)$。对其求导,通分整理得
$$f'(x)=1-\frac{a+1}x+\frac a{x^2}=\frac{x^2-(a+1)x+a}{x^2}=\frac{(x-1)(x-a)}{x^2}$$
因为已知参数$a>1$,令$f^\prime(x)=0$得到两个完全不同的正实根:$x=1$和$x=a$ (且$0<1<a)$。由$f^{\prime}(x)$的符号分布可直接得出:当$x\in(0,1)$时,$(x-1)<0,(x-a)<0\Longrightarrow f^\prime(x)>0$,函数$f(x)$单增;当$x\in(1,a)$时,$(x-1)>0,(x-a)<0\Longrightarrow f'(x)<0$,函数$f(x)$单减;当$x\in(a,+\infty)$时,$(x-1)>0,(x-a)>0\Longrightarrow f'(x)>0$,函数$f(x)$单增。因此,$f(x)$的单调递增区间为$(0,1)$和$(a,+\infty)$,单调递减区间为$(1,a)$。极大值为$f(1)=1-a$;极小值为$f(a)=a-1-(a+1)\ln a$ 。\\

(2)(i)设截线高度$f(x_1)=f(x_2)=f(x_3)=t$。由于存在三个相异实根,常数$t$必然严格介于极小值与极大值之间:$t\in(f(a),1-a)$。三个根分别落入三个单调区间:$x_1\in(0,1),x_2\in(1,a),x_3\in(a,+\infty)$。目标不等式右侧的$1-a$恰好是极大值$f(1)$。由于$f(x)$在最后一段区间$(a,+\infty)$上严格单调递增且趋于无穷,方程$f(x)=1-a$在该区间上有唯一实根,记为$x_0$ (显然$x_0>a$且$f(x_0)=1-a)$。要证明$f(x_1x_2x_3)>1-a=f(x_0)$,因为$(a,+\infty)$是严格增区间,我们只需证明:
$$x_1x_2x_3>x_0$$
记三根乘积为关于截线高度$t$的函数$P(t)=x_1(t)\cdot x_2(t)\cdot x_3(t)$。对对数形式$\ln P(t)=\ln x_1+\ln x_2+\ln x_3$两边关于$t$求导,由隐函数求导法则$f^\prime(x_k)\cdot x_k^{\prime}(t)=1$可得:
$$\frac{P'(t)}{P(t)}=\frac{x'_1}{x_1}+\frac{x'_2}{x_2}+\frac{x'_3}{x_3}=\sum_{k=1}^3\frac1{x_kf'(x_k)}$$
为精准判断上式的符号,我们需要计算这个和。直接计算很困难,但我们可以借助复变函数中的留数定理,将这个和转化为一个积分,从而判断其正负。

\paragraph{构造复变函数} 考虑函数 $g(z)=\dfrac{1}{z(f(z)-t)}$。注意,在每一个实根 $x_k$ 处,$f(x_k)-t=0$,且由于 $t \in (f(a), 1-a)$ 避开了极值,故 $f'(x_k)\neq0$,所以 $x_k$ 是 $g(z)$ 的一阶极点。根据留数计算规则,$g(z)$ 在 $x_k$ 处的留数为 $\displaystyle\lim_{z\to x_k}(z-x_k)g(z)=\frac{1}{x_k f'(x_k)}$。因此,三个实根的留数之和正好就是我们要的 $\displaystyle\sum_{k=1}^3\frac1{x_kf'(x_k)}$。

\paragraph{为什么要排除复数根?} 留数定理说:一个函数沿闭合路径的积分等于 $2\pi i$ 乘以路径内部所有奇点的留数之和。如果我们能找到一个闭合路径,它内部只包含这三个实根,而不包含其他奇点(比如复数根),那么该路径的积分就等于 $2\pi i$ 乘以这个和。这样我们就把和转化成了积分,而积分可以通过巧妙选择路径来计算。

因此,我们必须先证明:对于我们所考虑的 $t\in(f(a),1-a)$,方程 $f(z)=t$ 在割去负实轴的复平面上除了这三个实根外没有其他根(或者说,没有根落在我们即将选取的路径内部)。这需要分析复根的可能性。

\paragraph{分析复数根的存在性} 设 $z=r e^{i\theta}$,$\theta\in(-\pi,\pi)$ 且 $\theta\neq0$。由 $\operatorname{Im}(f(z))=0$ 展开得:
$$r\sin\theta-(a+1)\theta+\frac{a}{r}\sin\theta=0 \quad\Longrightarrow\quad \frac{\sin\theta}{\theta}=\frac{a+1}{r+\frac{a}{r}}.$$
因为对 $\theta\neq0$ 恒有 $\left|\frac{\sin\theta}{\theta}\right|<1$,所以复根存在的必要条件是 $\frac{a+1}{r+\frac{a}{r}}<1$,即 $r+\frac{a}{r}>a+1$,这等价于 $r\in(0,1)\cup(a,+\infty)$。也就是说,如果存在复根,其模长只能落在这两个区间内。

现在沿着等值线 $v(r,\theta)=\operatorname{Im}(f(z))=0$(即虚部为零的曲线)考察实部 $u(r,\theta)=\operatorname{Re}(f(z))$ 的变化。利用极坐标下的柯西-黎曼方程($u_r=\frac{1}{r}v_\theta$,$u_\theta=-r v_r$)以及沿隐函数 $v=0$ 对 $r$ 的全导数公式,可得:
$$\frac{du}{dr}=u_r+u_\theta\frac{d\theta}{dr}=u_r+u_\theta\left(-\frac{v_r}{v_\theta}\right)=\frac{v_\theta^2+r^2v_r^2}{r v_\theta}=\frac{r^2(u_r^2+v_r^2)}{r v_\theta}=\frac{r|f'(z)|^2}{v_\theta}.$$
再计算 $v_\theta$:
$$v_\theta=\frac{\partial}{\partial\theta}\left(r\sin\theta-(a+1)\theta+\frac{a}{r}\sin\theta\right)=\cos\theta\left(r+\frac{a}{r}\right)-(a+1).$$
将必要条件 $r+\frac{a}{r}=\frac{(a+1)\theta}{\sin\theta}$ 代入,得:
$$v_\theta=(a+1)\left(\theta\cot\theta-1\right).$$
由于 $\theta\in(-\pi,\pi)$ 且 $\theta\neq0$,由三角不等式恒有 $\theta\cot\theta<1$ 严格成立,故 $v_\theta<0$。同时在曲线上极点以外区域 $|f'(z)|\neq 0$,于是 $\frac{du}{dr}<0$,即沿复根可能存在的曲线,实部 $u$ 随 $r$ 的增大而严格单调递减。

在 $r\in(0,1)$ 分支,当 $r\to1^-$ 时解得 $\theta\to0$,此时 $u\to f(1)=1-a$,且由于随 $r$ 递减,当 $r<1$ 时恒有 $u>1-a$;在 $r\in(a,+\infty)$ 分支,当 $r\to a^+$ 时同理解得 $\theta\to0$,$u\to f(a)$,且由于递减,当 $r>a$ 时恒有 $u<f(a)$。因此,所有非实数复数根的实部要么大于 $1-a$,要么小于 $f(a)$。而我们的截线 $t\in(f(a),1-a)$ 既不等于 $1-a$ 也不等于 $f(a)$,且不落在这些区间之外,所以在这个 $t$ 值下,方程 $f(z)=t$ 绝对不存在任何复数根。这就保证了除了这三个实根外,复平面上没有其他奇点。

\paragraph{选取围道并应用留数定理} 我们取一个“钥匙孔”形状的围道 $\Gamma$(如图),它由四部分组成:
\begin{itemize}
    \item 大圆 $C_R$:$|z|=R$,逆时针方向(辐角 $-\pi \to \pi$),$R\to\infty$;
    \item 上边缘 $L_+$:从 $-R$ 到 $-\varepsilon$,沿着负实轴的上侧(辐角取 $\pi$);
    \item 小圆 $C_\varepsilon$:$|z|=\varepsilon$,顺时针方向(辐角从 $\pi$ 绕到 $-\pi$),$\varepsilon\to0$;
    \item 下边缘 $L_-$:从 $-\varepsilon$ 到 $-R$,沿着负实轴的下侧(辐角取 $-\pi$)。
\end{itemize}
因为函数 $f(z)$ 含有主分支 $\ln z$,它在负实轴上不连续,所以我们需要这样割开复平面,使得 $\ln z$ 在围道内单值解析。同时负实轴上 $\operatorname{Im}(f(z)) = \mp(a+1)\pi \neq 0$,故割线上无根。围道内部(前进方向的正左侧)只包含我们那三个实根 $x_1,x_2,x_3$(它们都在正实轴上),没有其他极点。

根据留数定理:
$$\oint_\Gamma g(z)\,dz = 2\pi i \sum_{k=1}^3 \operatorname{Res}(g,x_k)=2\pi i\sum_{k=1}^3\frac{1}{x_k f'(x_k)}.$$
现在计算左边的积分,它等于四部分之和:$\int_{C_R}+\int_{C_\varepsilon}+\int_{L_+}+\int_{L_-}$。

当 $R\to\infty$ 时,$f(z)-t\sim z$,故 $g(z)\sim \frac{1}{z^2}$,由于大圆弧长为 $2\pi R$,故积分模长受控于 $O(\frac{1}{R}) \to 0$,所以 $\int_{C_R}\to0$。
当 $\varepsilon\to0$ 时,$f(z)-t\sim -\frac{a}{z}$ 主导,故 $g(z)\sim -\frac{1}{a}$ 有界,而小圆弧长趋于 0,故 $\int_{C_\varepsilon}\to0$。

接下来计算 $L_+$ 和 $L_-$ 上的积分。
在上边缘 $L_+$:$z=r e^{i\pi}$,$r$ 从 $+\infty$ 降到 $0$,$dz=-dr$,$\ln z = \ln r + i\pi$。代入计算得:
$$f(z)-t = z-(a+1)\ln z-\frac{a}{z}-t = -r-(a+1)(\ln r+i\pi)+\frac{a}{r}-t = A(r) - i(a+1)\pi,$$
其中实部函数记为 $A(r)=-r+\frac{a}{r}-(a+1)\ln r - t$。所以
$$g(z)=\frac{1}{z(f(z)-t)}=\frac{1}{r e^{i\pi}(A(r)-i(a+1)\pi)}=\frac{-1}{r(A(r)-i(a+1)\pi)}.$$
于是
$$\int_{L_+}g(z)dz = \int_{+\infty}^0 \frac{-1}{r(A(r)-i(a+1)\pi)} (-dr) = -\int_0^{+\infty} \frac{dr}{r(A(r)-i(a+1)\pi)}.$$

在下边缘 $L_-$:$z=r e^{-i\pi}$,$r$ 从 $0$ 升到 $+\infty$,$dz=-dr$,$\ln z = \ln r - i\pi$,类似可得
$$f(z)-t= -r-(a+1)(\ln r-i\pi)+\frac{a}{r}-t=A(r)+i(a+1)\pi,$$
$$g(z)=\frac{1}{r e^{-i\pi}(A(r)+i(a+1)\pi)}=\frac{-1}{r(A(r)+i(a+1)\pi)}.$$
$$\int_{L_-}g(z)dz = \int_{0}^{+\infty} \frac{-1}{r(A(r)+i(a+1)\pi)} (-dr) = \int_{0}^{+\infty} \frac{dr}{r(A(r)+i(a+1)\pi)}.$$

将 $L_+$ 和 $L_-$ 的积分相加:
$$\int_{L_+}+\int_{L_-}= \int_0^{+\infty}\frac{1}{r}\left[\frac{1}{A(r)+i(a+1)\pi} - \frac{1}{A(r)-i(a+1)\pi}\right]dr = \int_0^{+\infty}\frac{-2i(a+1)\pi}{r\left[A^2(r)+(a+1)^2\pi^2\right]}dr.$$

于是取极限后整个闭合围道积分为:
$$\oint_\Gamma g(z)dz = \int_0^{+\infty}\frac{-2i(a+1)\pi}{r\left[A^2(r)+(a+1)^2\pi^2\right]}dr.$$

根据留数定理,它等于 $2\pi i\sum\frac{1}{x_k f'(x_k)}$,两边除以 $2\pi i$ 得:
$$\sum_{k=1}^3\frac{1}{x_k f'(x_k)} = -\int_0^{+\infty}\frac{a+1}{r\left[A^2(r)+(a+1)^2\pi^2\right]}dr.$$

\paragraph{判断符号} 由于 $a>1$ 保证了 $a+1>0$,且分母中 $A^2(r)+(a+1)^2\pi^2>0$ 恒成立,$r>0$,被积函数严格为正(且在 $0$ 和 $+\infty$ 处广义积分均绝对收敛),前面带负号,因此
$$\sum_{k=1}^3\frac{1}{x_k f'(x_k)} < 0 \quad \text{恒成立}.$$
而 $\frac{P'(t)}{P(t)}$ 正是这个和。因实根均大于 0,故 $P(t)>0$,从而导出 $P'(t)<0$,即 $P(t)$ 关于 $t$ 严格单调递减。

\paragraph{完成证明} 由于 $P(t)$ 在区间 $(f(a),1-a)$ 上严格递减,那么对于任意 $t<1-a$,有 $P(t) > \lim_{\tau\to (1-a)^-}P(\tau)$。当 $t\to (1-a)^-$ 时,由于 $x=1$ 是连续的严格极大值点,截线高度逼近极大值时,两侧的 $x_1$ 和 $x_2$ 都被挤压而同时趋近于 $x=1$;而 $x_3$ 则连续地趋近于方程 $f(x)=1-a$ 在 $(a,+\infty)$ 上的唯一实根 $x_0$。因此极限值为 $1\cdot1\cdot x_0 = x_0$。于是
$$x_1x_2x_3 = P(t) > x_0.$$
由于 $x_0$ 是方程 $f(x)=1-a$ 在 $(a,+\infty)$ 上的唯一实根,由定义域可知自带 $x_0>a$ 属性。利用不等式的传递性直接得到 $x_1x_2x_3 > x_0 > a$。又因为 $f(x)$ 在 $(a,+\infty)$ 上严格递增,作用映射 $f$ 不改变不等号方向,所以
$$f(x_1x_2x_3) > f(x_0) = 1-a.$$
这就证明了原不等式(i)。\\

(2)(ii)设三根之和为关于截线高度 $t$ 的函数 $S(t)=x_1(t)+x_2(t)+x_3(t)$。
对 $S(t)$ 两边关于 $t$ 求导,同样由隐函数求导法则 $f'(x_k)\cdot x_k'(t)=1$ 可得:
$$S'(t)=x'_1+x'_2+x'_3=\sum_{k=1}^3\frac{1}{f'(x_k)}$$
我们再次借助留数定理来计算这个和。

\paragraph{构造复变函数与留数计算} 考虑函数 $h(z)=\dfrac{1}{f(z)-t}$。
与(i)同理,在每一个实根 $x_k$ 处,$f(x_k)-t=0$ 且 $f'(x_k)\neq 0$,因此 $x_k$ 是 $h(z)$ 的一阶极点。
其在 $x_k$ 处的留数为 $\displaystyle\lim_{z\to x_k}\frac{z-x_k}{f(z)-t}=\frac{1}{f'(x_k)}$。
所以,三个实根的留数之和正好为 $S'(t) = \displaystyle\sum_{k=1}^3\frac{1}{f'(x_k)}$。

\paragraph{围道积分与符号判断} 我们沿用(i)中相同的“钥匙孔”围道 $\Gamma$。根据留数定理:
$$\oint_\Gamma h(z)\,dz = 2\pi i \sum_{k=1}^3 \operatorname{Res}(h,x_k)=2\pi i S'(t)$$
计算四部分积分 $\int_{C_R}+\int_{C_\varepsilon}+\int_{L_+}+\int_{L_-}$:

当 $R\to\infty$ 时,我们对被积函数进行洛朗渐近展开得 $h(z) = \frac{1}{z(1 - \frac{(a+1)\ln z + a/z + t}{z})} = \frac{1}{z} + O\left(\frac{\ln z}{z^2}\right)$。高阶项在大圆上的积分趋于 0,故大圆积分 $\int_{C_R} h(z)dz = \int_{-\pi}^{\pi} \frac{1}{R e^{i\theta}} i R e^{i\theta} d\theta \to 2\pi i$。
当 $\varepsilon\to0$ 时,$f(z)-t\sim -\frac{a}{z}$,故 $h(z)\sim -\frac{z}{a}$,被积函数趋于 0,且小圆弧长趋于 0,故小圆积分 $\int_{C_\varepsilon}\to0$。

在上边缘 $L_+$:$z=r e^{i\pi}$,$dz=-dr$($r$ 从 $+\infty$ 降到 $0$)。此时 $f(z)-t=A(r)-i(a+1)\pi$。
$$\int_{L_+}h(z)dz = \int_{+\infty}^0 \frac{1}{A(r)-i(a+1)\pi} (-dr) = \int_0^{+\infty} \frac{dr}{A(r)-i(a+1)\pi}$$
在下边缘 $L_-$:$z=r e^{-i\pi}$,$dz=-dr$($r$ 从 $0$ 升到 $+\infty$)。此时 $f(z)-t=A(r)+i(a+1)\pi$。
$$\int_{L_-}h(z)dz = \int_{0}^{+\infty} \frac{1}{A(r)+i(a+1)\pi} (-dr) = -\int_0^{+\infty} \frac{dr}{A(r)+i(a+1)\pi}$$

将 $L_+$ 和 $L_-$ 的积分相加:
$$\int_{L_+}+\int_{L_-} = \int_0^{+\infty}\left[\frac{1}{A(r)-i(a+1)\pi} - \frac{1}{A(r)+i(a+1)\pi}\right]dr = \int_0^{+\infty}\frac{2i(a+1)\pi}{A^2(r)+(a+1)^2\pi^2}dr$$

于是整个闭合围道积分为:
$$\oint_\Gamma h(z)dz = 2\pi i + \int_0^{+\infty}\frac{2i(a+1)\pi}{A^2(r)+(a+1)^2\pi^2}dr$$
两边同除以 $2\pi i$,得到 $S'(t)$ 的精确表达式:
$$S'(t) = 1 + \int_0^{+\infty}\frac{a+1}{A^2(r)+(a+1)^2\pi^2}dr$$
因为 $a>1$ 且分母严格大于 0,被积函数严格为正。所以 $S'(t) > 1 > 0$ 恒成立,即 $S(t)$ 关于 $t$ 严格单调递增。

\paragraph{放缩与完成证明} 由于 $S(t)$ 在区间 $(f(a),1-a)$ 上严格递增,对于任意 $t<1-a$,有 $S(t) < \lim_{\tau\to (1-a)^-}S(\tau)$。
同(i)中的极限分析,当 $t\to (1-a)^-$ 时,$x_1\to 1$,$x_2\to 1$,$x_3\to x_0$。
故 $x_1+x_2+x_3 = S(t) < 1+1+x_0 = x_0+2$。

因为 $x_1, x_2 > 0$ 且 $x_3 > a$,显然有 $S(t) > a$。又因 $f(x)$ 在 $(a,+\infty)$ 上严格单调递增,且 $x_1+x_2+x_3$ 与 $x_0+2$ 均处于该区间,所以直接代入映射不改变不等号方向:
$$f(x_1+x_2+x_3) < f(x_0+2)$$
接下来我们估计 $f(x_0+2)$ 的上限。根据微积分基本定理:
$$f(x_0+2) - f(x_0) = \int_{x_0}^{x_0+2} f'(x) dx$$
已知 $f'(x) = 1 - \frac{a+1}{x} + \frac{a}{x^2} = 1 - \frac{(a+1)x-a}{x^2}$。
对于所有位于 $(x_0, x_0+2)$ 内的积分变量 $x > a > 1$,恒有 $(a+1)x - a = a(x-1) + x > a > 0$,因此 $\frac{(a+1)x-a}{x^2} > 0$,即在此区间上恒有严格的 $f'(x) < 1$。
从而 $\int_{x_0}^{x_0+2} f'(x) dx < \int_{x_0}^{x_0+2} 1 \,dx = 2$。
又因为 $f(x_0) = 1-a$,移项即可得到:
$$f(x_0+2) < f(x_0) + 2 = (1-a) + 2 = 3-a$$
最后一步,出题人在此处放了一个非常宽泛的常数界限:由于自然常数 $e \approx 2.718$,即 $2e>5.436$,显然 $3 < 2e-1$ 恒成立,于是:
$$3-a < 2e-1-a$$
综上,首尾相连写下不等号即可得:
$$f(x_1+x_2+x_3) < f(x_0+2) < 3-a < 2e-1-a$$
原不等式(ii)得证。
\end{solution}

\begin{example}{}{}
已知函数$f(x)=(x+1)\ln x,f(x_1)=f(x_2)$,过$(x_1,f(x_1)),(x_2,f(x_2))$分别作切线交$x$轴于$(x_3,0),(x_4,0)$,求证$x_3+x_4>0$.
\end{example}
\begin{solution}
设$x_-,x_+$分别表示正负根,记
$$E=x_3+x_4=\frac{x_{-}^{2}}{(x_{-}+1)f^{\prime}(x_{-})}+\frac{x_{+}^{2}}{(x_{+}+1)f^{\prime}(x_{+})}.$$
简单估值即可证明,当 $y\ge 4$ 时,两项之和恒满足 $E > -1 + 1 = 0$。当 $0<y<4$ 时,引入复积分法处理。取主值对数 $L(z)=\operatorname{Log}(1+z)$,其分支割线设在 $(-\infty,-1]$。定义复变函数:$$ \Phi(z)=\frac{yz}{(z+1)L(z)(zL(z)-y)} $$考察割平面 $\mathbb{C}\setminus(-\infty,-1]$ 中方程 $zL(z)=y$ 的根。设 $1+z = r e^{i\theta}$,其中 $\theta \in (-\pi,\pi)$,$r>0$。则 $z = r\cos\theta - 1 + ir\sin\theta$,$L(z) = \ln r + i\theta$。代入方程 $zL(z)=y$,取虚部可得:$$ \Im(zL(z)) = \theta(r\cos\theta - 1) + r\sin\theta \ln r = 0. $$若存在非实数根(即 $\theta \ne 0$),解得 $\ln r = \frac{\theta(1 - r\cos\theta)}{r\sin\theta}$。
代入实部可得:
\begin{align*}\Re(zL(z)) &= (r\cos\theta - 1)\ln r - r\theta\sin\theta \&= (r\cos\theta - 1)\frac{\theta(1 - r\cos\theta)}{r\sin\theta} - \frac{r^2\theta\sin^2\theta}{r\sin\theta} \\&= -\frac{\theta}{r\sin\theta} \left[ (1-r\cos\theta)^2 + r^2\sin^2\theta \right] = -\frac{\theta}{\sin\theta}\frac{|1-(1+z)|^2}{r} = -\frac{\theta}{\sin\theta}\frac{|z|^2}{r}.\end{align*}由于 $\theta \in (-\pi,\pi) \setminus \{0\}$,$\theta$ 和 $\sin\theta$ 同号,有 $\frac{\theta}{\sin\theta} > 0$。由 $|z|^2>0$ 且 $r>0$,推得 $\Re(zL(z)) < 0$,这与 $y>0$ 矛盾。因此 $\theta=0$,方程在割平面内仅有实数根 $x_{-}$ 和 $x_{+}$。另外,当 $z \to 0$ 时,$L(z) \sim z$,$\Phi(z) \to -1$,故 $z=0$ 为可去奇点。对于简单极点 $x_{\pm}$,利用 $(zL(z))' = f'(z)$,其留数为:$$ \operatorname{Res}(\Phi;x_{\pm}) = \lim_{z\to x_{\pm}} \frac{yz(z-x_{\pm})}{(z+1)L(z)(zL(z)-y)} = \frac{y x_{\pm}}{(x_{\pm}+1)L(x_{\pm})f^{\prime}(x_{\pm})} = \frac{x_{\pm}^{2}}{(x_{\pm}+1)f^{\prime}(x_{\pm})}. $$因此,这两个极点的留数之和恰为所求表达式 $E$。取绕割线 $(-\infty,-1]$ 的闭合匙孔形围道进行积分,该围道由大圆 $C_R$、贴近割线上侧的 $L_+$、绕 $z=-1$ 的小圆 $C_\epsilon$ 及贴近割线下侧的 $L_-$ 组成。当 $R\to\infty$ 时,$|\Phi(z)| = \mathcal{O}\left(\frac{1}{R\ln^2 R}\right)$,路径长度为 $2\pi R$,大圆上的积分为 $\mathcal{O}\left(\frac{1}{\ln^2 R}\right) \to 0$。当 $\epsilon\to 0$ 时,$L(z) \sim \ln\epsilon$,$|\Phi(z)| = \mathcal{O}\left(\frac{1}{\epsilon\ln^2 \epsilon}\right)$,路径长度为 $2\pi \epsilon$,小圆上的积分为 $\mathcal{O}\left(\frac{1}{\ln^2 \epsilon}\right) \to 0$。根据留数定理,极点留数之和等于沿割线两侧的积分。令 $z = -1-t$($t>0$),割线两侧的对数边值分别为 $L_{\pm}(-1-t) = \ln t \pm i\pi$。由此可得:$$ E = \frac{1}{2\pi i}\int_{0}^{\infty} \left(\Phi_{+}(-1-t) - \Phi_{-}(-1-t)\right) dt. $$由于 $\Phi_{-}=\overline{\Phi_{+}}$,有 $\Phi_{+} - \Phi_{-} = 2i\Im\Phi_{+}$,故 $E = \frac{1}{\pi}\int_{0}^{\infty}\Im\Phi_{+}(-1-t)dt$。代入 $\Phi_{+}(-1-t)$ 进行化简:$$ \Phi_{+}(-1-t) = \frac{y(-1-t)}{(-t)(\ln t + i\pi)\left[(-1-t)(\ln t+i\pi) - y\right]} = \frac{-y(1+t)}{t(\ln t+i\pi)\left[(1+t)(\ln t+i\pi) + y\right]}. $$分子分母同除以 $1+t$,可化为:$$ \Phi_{+}(-1-t) = \frac{1+t}{t} \left( \frac{1}{\ln t+\frac{y}{1+t}+i\pi} - \frac{1}{\ln t+i\pi} \right). $$利用复数虚部公式 $\Im\left(\frac{1}{u+i\pi}\right) = -\frac{\pi}{u^2+\pi^2}$,提取虚部可得:$$ \Im\Phi_{+}(-1-t) = \frac{\pi(1+t)}{t} \left( \frac{1}{(\ln t)^2+\pi^2} - \frac{1}{\left(\ln t+\frac{y}{1+t}\right)^2+\pi^2} \right). $$代入 $E$ 的表达式中消去 $\pi$,得到实积分表示:$$ E = \int_{0}^{\infty}\frac{1+t}{t}\left(\frac{1}{(\ln t)^{2}+\pi^{2}}-\frac{1}{\left(\ln t+\frac{y}{1+t}\right)^{2}+\pi^{2}}\right)dt. $$利用恒等式 $\frac{1}{u^2+\pi^2} - \frac{1}{(u+a)^2+\pi^2} = \int_{0}^{a} \frac{2(u+s)}{((u+s)^2+\pi^2)^2} ds$,将 $u = \ln t$ 与 $a = \frac{y}{1+t}$ 代入,积分化为:$$ E = \int_{0}^{\infty} \frac{1+t}{t} \left( \int_{0}^{\frac{y}{1+t}} \frac{2(\ln t+s)}{\left((\ln t+s)^{2}+\pi^{2}\right)^{2}} ds \right) dt. $$作积分变量代换 $r = (1+t)s$,则 $ds = \frac{dr}{1+t}$,内层积分限变为 $0 < r < y$,从而:$$ E = \int_{0}^{\infty} \int_{0}^{y} \frac{2\left(\ln t+\frac{r}{1+t}\right)}{t\left(\left(\ln t+\frac{r}{1+t}\right)^{2}+\pi^{2}\right)^{2}} dr\, dt. $$设双重积分的被积函数为 $F(t,r)$。在 $r \in [0, y]$ 上,由于 $y<4$,$r$ 有界。当 $t \to +\infty$ 时,$|F(t,r)| \sim \frac{2}{t(\ln t)^3}$;当 $t \to 0^+$ 时,$|F(t,r)| \sim \frac{2}{t|\ln t|^3}$。这表明在 $t \in (0, +\infty)$ 上 $\max_{r \in [0,y]} |F(t,r)|$ 绝对可积。由于积分域 $[0,y]$ 有界,双重积分绝对收敛。由 Fubini 定理交换积分次序,可得:$$ E = \int_{0}^{y} J(r)dr, \quad J(r) = \int_{0}^{\infty}\frac{2\left(\ln t+\frac{r}{1+t}\right)}{t\left(\left(\ln t+\frac{r}{1+t}\right)^{2}+\pi^{2}\right)^{2}}dt. $$现证明当 $0<r<4$ 时 $J(r)>0$。对 $J(r)$ 作代换 $t = e^{s}$ ($dt/t = ds$),得:$$ J(r) = \int_{-\infty}^{\infty}\frac{2\phi_{r}(s)}{\left(\phi_{r}(s)^{2}+\pi^{2}\right)^{2}}ds, \quad \text{其中 } \phi_{r}(s) = s+\frac{r}{1+e^{s}}. $$计算其导数 $\phi_{r}^{\prime}(s) = 1-\frac{re^{s}}{(1+e^{s})^{2}}$。由于 $\frac{e^{s}}{(1+e^{s})^{2}} \le \frac{1}{4}$,当 $r<4$ 时,$\phi_{r}^{\prime}(s) \ge 1-\frac{r}{4} > 0$。因此 $\phi_{r}(s)$ 在定义域内严格单调递增,进而对于 $s>0$,有 $\phi_{r}(s) > \phi_{r}(0) = \frac{r}{2}$。再次求导得 $\phi_{r}^{\prime\prime}(s) = \frac{re^{s}(e^{s}-1)}{(1+e^{s})^{3}}$,故对于 $s>0$,有 $\phi_{r}^{\prime\prime}(s)>0$。对 $J(r)$ 采用分部积分。设 $A(s) = \frac{1}{\phi_{r}(s)^{2}+\pi^{2}}$,被积函数可表示为 $-\frac{A^{\prime}(s)}{\phi_{r}^{\prime}(s)}$,从而:$$ J(r) = \left[-\frac{A(s)}{\phi_{r}^{\prime}(s)}\right]_{-\infty}^{\infty} + \int_{-\infty}^{\infty}A(s)\frac{d}{ds}\left(\frac{1}{\phi_{r}^{\prime}(s)}\right)ds. $$当 $s\to \pm\infty$ 时,$\phi_{r}^{\prime}(s)\to 1$ 且 $A(s)\to 0$,故边界项为零。记 $B(s) = \frac{d}{ds}\left(\frac{1}{\phi_{r}^{\prime}(s)}\right) = -\frac{\phi_{r}^{\prime\prime}(s)}{(\phi_{r}^{\prime}(s))^{2}}$。因为 $\phi_{r}^{\prime}(s)$ 为偶函数,其倒数的导数 $B(s)$ 必为奇函数。利用奇偶性将积分区间折叠至 $(0,+\infty)$:$$ J(r) = \int_{0}^{\infty} B(s)(A(s)-A(-s)) ds = \int_{0}^{\infty} B(s)\left(\frac{1}{\phi_{r}(s)^{2}+\pi^{2}}-\frac{1}{\phi_{r}(-s)^{2}+\pi^{2}}\right)ds. $$当 $s>0$ 时,由于 $\phi_{r}^{\prime\prime}(s)>0$,有 $B(s)<0$。此外,由于 $\phi_{r}(-s) = -s+\frac{re^{s}}{1+e^{s}} = r - \left(s+\frac{r}{1+e^{s}}\right) = r-\phi_{r}(s)$,可得:$$ A(s)-A(-s) = \frac{1}{\phi_{r}(s)^2+\pi^2} - \frac{1}{(r-\phi_{r}(s))^2+\pi^2} = \frac{r(r-2\phi_{r}(s))}{\left(\phi_{r}(s)^{2}+\pi^{2}\right)\left((r-\phi_{r}(s))^{2}+\pi^{2}\right)}. $$已知当 $s>0$ 时 $\phi_{r}(s) > \frac{r}{2}$,即 $r-2\phi_{r}(s)<0$,故 $A(s)-A(-s)<0$。因此,被积函数 $B(s)(A(s)-A(-s)) > 0$ 恒成立,推得对于任意 $0<r<4$ 均有 $J(r)>0$。由于已知 $0<y<4$,积分上限 $y$ 保证了积分变量满足 $r \le y < 4$,从而 $E = \int_{0}^{y}J(r)dr > 0$。综上所述,对任意 $y>0$ 且 $x_{1}\ne x_{2}$ 满足 $x_{1}\ln(1+x_{1})=x_{2}\ln(1+x_{2})=y$,恒有 $E>0$ 成立。
\end{solution}
\newpage
\begin{example}{邪帝原创导数题}{}
    已知函数$f(x)=x-(a+1)\ln x-\frac{a}{x},(a>1)$\\
    (1)讨论$f(x)$的单调性;\\
    (2)若$f(x_1)=f(x_2)=f(x_3),(x_1<x_2<x_3)$,证明$f(x_1x_2x_3)>1-a$.
\end{example}
\begin{solution}
设$g(x)=x-\ln x,f(x)=g(x)-ag\left(\frac1x\right)$,$x$和$\dfrac{1}{x}$同时出现,指向比值换元.\\
(1)已知函数$f(x)=x-(a+1)\ln x-\frac ax$,定义域为$(0,+\infty)$。对其求导,通分整理得
$$f'(x)=1-\frac{a+1}x+\frac a{x^2}=\frac{x^2-(a+1)x+a}{x^2}=\frac{(x-1)(x-a)}{x^2}$$
当$x\in(0,1)$时,$(x-1)<0,(x-a)<0\Longrightarrow f^\prime(x)>0$,函数$f(x)$单增;当$x\in(1,a)$时,$(x-1)>0,(x-a)<0\Longrightarrow f'(x)<0$,函数$f(x)$单减;当$x\in(a,+\infty)$时,$(x-1)>0,(x-a)>0\Longrightarrow f'(x)>0$,函数$f(x)$单增。因此,$f(x)$的单调递增区间为(0,1)和$(a,+\infty)$,单调递减区间为$(1,a)$。极大值为$f(1)=1-a$;极小值为$f(a)=a-1-(a+1)\ln a$ 。\\
(2)目标不等式右侧的$1-a$恰好是极大值$f(1)$。由于$f(x)$在区间$(a,+\infty)$上单增且趋于无穷,所以方程$f(x)=1-a$在该区间上有唯一实根,不妨记为$x_0$,则
\[g(x_0)-ag\left(\frac{1}{x_0}\right)=1-a\Leftrightarrow a=\dfrac{g(x_0)-1}{g\left(\frac{1}{x_0}\right)-1}=\frac{x_0-\ln x_0-1}{\frac1{x_0}+\ln x_0-1}\]
要证明 $f(x_1x_2x_3) > 1 - a = f(x_0)$,因为 $(a, +\infty)$ 是严格增区间,我们只需证明:$x_1x_2x_3 > x_0$,这又指向了乘积换元,于是将 $x_2, x_3$ 替换为 $pq$ 和 $\frac{p}{q}$,可以得到,$f(x_2) = f(x_3)$,得到
$$pq - \frac{a}{pq} - (a + 1)\ln(pq) = \frac{p}{q} - a\frac{q}{p} - (a + 1)\ln\frac{p}{q}$$
$$
\Leftrightarrow p\left(q - \frac{1}{q}\right) + \frac{a}{p}\left(q - \frac{1}{q}\right) = 2(a + 1)\ln q \Leftrightarrow \left(p + \frac{a}{p}\right)\left(q - \frac{1}{q}\right) = 2(a + 1)\ln q$$
则原等式化为
$$\begin{cases}
f(x) = f(pq) & \Leftrightarrow x - \frac{a}{x} - (a + 1)\ln x = pq - \frac{a}{pq} - (a + 1)\ln(pq) \\
f(pq) = f(\frac{p}{q}) & \Leftrightarrow \left(p + \frac{a}{p}\right)\left(q - \frac{1}{q}\right) = 2(a + 1)\ln q
\end{cases}$$
第二个方程等价于
$$\left(q - \frac{1}{q}\right)p^2 - 2(a + 1)\ln q\cdot p + a\left(q - \frac{1}{q}\right) = 0 \Leftrightarrow p = \frac{(a + 1)\ln q - \sqrt{(a + 1)^2\ln^2 q - a\left(q - \frac{1}{q}\right)^2}}{q - \frac{1}{q}}$$(猜的,不确定要舍掉哪个解)





要证明$f(x_1x_2x_3)>1-a=f(x_0)$,因为$(a,+\infty)$是严格增区间,我们只需证明:$x_1x_2x_3>x_0$,我们可以尝试加强证明$x_1x_2^2>x_0$,
\begin{align*}
x_1x_2^2>x_0&\Leftrightarrow 1<\frac{x_0}{x_1x_2}<p\Leftrightarrow f\left(\frac{x_0}{x_1x_2}\right)<f(x_2)\\
&\Leftrightarrow \frac{x_0}{x_1x_2}-a\frac{x_1x_2}{x_0}-(a+1)\ln\frac{x_0}{x_1x_2}<x_2-\frac{a}{x_2}-(a+1)\ln x_2
\end{align*}
比值换元,令$y=\frac{x_1}{x_2}<1\Leftrightarrow x_1=x_2y$,则
\begin{align*}
&x_1-\frac{a}{x_1}-(a+1)\ln x_1=x_2-\frac{a}{x_2}-(a+1)\ln x_2\\
\Leftrightarrow& x_2y-\frac{a}{x_2y}-(a+1)\ln x_2y=x_2-\frac{a}{x_2}-(a+1)\ln x_2\\
\Leftrightarrow& x_2(y-1)+\frac{a}{x_2y}(y-1)=(a+1)\ln y\Leftrightarrow x_2+\frac{a}{x_2y}=\frac{\ln y}{y-1}\\
&\frac{x_0}{x_1x_2}-a\frac{x_1x_2}{x_0}-(a+1)\ln\frac{x_0}{x_1x_2}<x_2-\frac{a}{x_2}-(a+1)\ln x_2\\
\Leftrightarrow&\frac{x_0}{x_2^2y}-a\frac{x_2^2y}{x_0}-(a+1)\ln\frac{x_0}{x_2^2y}<x_2-\frac{a}{x_2}-(a+1)\ln x_2\\
\Rightarrow h(y)&=\frac{x_0}{x_2^2y}-a\frac{x_2^2y}{x_0}-(a+1)\ln\frac{x_0}{x_2^2y}-\left(x_2-\frac{a}{x_2}-(a+1)\ln x_2\right)\\
\Rightarrow h'(y)&=-\frac{x_0}{x_2^2y^2}-a\frac{x_2^2}{x_0}+\frac{a+1}{y}=-\frac{x_0^2+ax_2^4y^2-(a+1)x_2^2yx_0}{x_2^2y^2x_0}=-\frac{(ax_2^2y-x_0)(x_2^2y-x_0)}{x_2^2y^2x_0}
\end{align*}
得到$h'(y)$的零点为$y=\frac{x_1}{x_2}=\frac{x_0}{ax_2^2}$和$y=\frac{x_1}{x_2}=\frac{x_0}{p^2}$,分析可知$h(y)$在$y=\frac{x_0}{ax_2^2}$处取得最大值,于是
\[h(y)\le h\!\left(\frac{x_0}{ap^2}\right)=a-1-(a+1)\ln a-p+\frac{a}{p}+(a+1)\ln p.\]
记$w(p)=a-1-(a+1)\ln a-p+\frac{a}{p}+(a+1)\ln p$,则
\[w'(p)=-1-\frac{a}{p^2}+\frac{a+1}{p}=-\frac{(p-1)(p-a)}{p^2}.\]
故$w(p)$在$(1,a)$上单调递增,在$(a,+\infty)$上单调递减,在$p=a$处取得最大值$w(a)=0$,因此$w(p)\le 0$,即
\[a-1-(a+1)\ln a\le p-\frac{a}{p}-(a+1)\ln p.\]
将$a=\dfrac{x_0-\ln x_0-1}{\frac{1}{x_0}+\ln x_0-1}$代入左边,并整理可得关于$x_0$的不等式:
\[\frac{4x_0\ln x_0+(x_0-1)\left((x_0-1)\ln\left(\frac{x_0(x_0-\ln x_0-1)}{x_0\ln x_0-x_0+1}\right)-2(x_0+1)\right)}{x_0\ln x_0-x_0+1}>0.\]
最后转化为说明对于任意 $x_0>1$,恒有
\[\frac{4x_0\ln x_0+(x_0-1)\left((x_0-1)\ln\left(\frac{x_0(x_0-\ln x_0-1)}{x_0\ln x_0-x_0+1}\right)-2(x_0+1)\right)}{x_0\ln x_0-x_0+1}>0\]
分母大于0显然(切线放缩),所以等价于
\begin{align*}
&4x_0\ln x_0+(x_0-1)\left((x_0-1)\ln\left(\frac{x_0(x_0-\ln x_0-1)}{x_0\ln x_0-x_0+1}\right)-2(x_0+1)\right)>0\\
\Leftrightarrow&4x_0\ln x_0-2(x_0^2-1)+(x_0-1)^2\ln\dfrac{x_0(x_0-\ln x_0-1)}{x_0\ln x_0-(x_0-1)}>0\\
\Leftrightarrow&2\left(2x_0\ln x_0-(x_0^2-1)\right)+(x_0^2-2x_0+1)\ln\dfrac{x_0\ln x_0-(x_0-1)+(x_0^2-2x_0\ln x_0-1)}{x_0\ln x_0-(x_0-1)}>0\\
\Leftrightarrow&2\left(2x_0\ln x_0-(x_0^2-1)\right)+(x_0^2-2x_0+1)\ln\left(1+\dfrac{x_0^2-2x_0\ln x_0-1}{x_0\ln x_0-(x_0-1)}\right)>0\\
\Leftrightarrow&-2\left(x_0^2-2x_0\ln x_0-1\right)+(x_0^2-2x_0\ln x_0-1+2(x_0\ln x_0-(x_0-1)))\ln\left(1+\dfrac{x_0^2-2x_0\ln x_0-1}{x_0\ln x_0-(x_0-1)}\right)>0\\
\Leftrightarrow&-2\dfrac{x_0^2-2x_0\ln x_0-1}{x_0\ln x_0-(x_0-1)}+\left(\dfrac{x_0^2-2x_0\ln x_0-1}{x_0\ln x_0-(x_0-1)}+2\right)\ln\left(1+\dfrac{x_0^2-2x_0\ln x_0-1}{x_0\ln x_0-(x_0-1)}\right)>0\\
\Leftrightarrow&(u+2)\ln (u+1)>2u,\quad u=\dfrac{x_0^2-2x_0\ln x_0-1}{x_0\ln x_0-(x_0-1)}>0\\
\Leftrightarrow&\ln(u+1)>\dfrac{2u}{u+2},\text{显然的}
\end{align*}
\end{solution}  





\begin{example}{(群友供题)}{}
    已知$f(x)=\text{e}^x-\ln x$,且$f(x_1)=f(x_2)$,$x_1>x_2$,求证$x_1+x_2>1$.
\end{example}
\begin{solution}
    本题对称化构造已经有解法了,笔者补一个构造函数的做法。首先转化为证明\[g(x)=(x-\dfrac12)^2,g(x_1)=(x_1-\dfrac12)^2>(x_2-\dfrac12)^2=g(x_2)\]进行观察,我们发现如果$x_1$稍微大一点,比如当$x_1>1$就已经有$x_1+x_2>1$,所以本题的关键是在$x=\dfrac12$附近证明这个结论,由此不难想到级数展开,我们首先求导得到:\vspace{-5pt}
    \[f'(x)=\text{e}^x-\dfrac{1}{x},f'(\dfrac12)=\sqrt{\text{e}}-2,f''(x)=\text{e}^x+\dfrac{1}{x^2},f''(\dfrac12)=\sqrt{\text{e}}+4,...\]
    这样不难得到$f(x)$先减再增,极值点不好求,并列出$f(x)$在$x=\dfrac12$附近的级数形式:
    \[f(x)=f(\dfrac12)+(\sqrt{\text{e}}-2)(x-\dfrac12)+\dfrac{\sqrt{\text{e}}+4}{2!}(x-\dfrac12)^2+\dfrac{\sqrt{e}+8}{3!}(x-\dfrac12)^3+\cdots\]
    我们要构造在$x=\dfrac12$左右侧正负性相反的函数$\varphi(x)=h_1(g(x))+h_2(f(x))$,同时规定$h_1(x)$单调递增,$h_2(x)$在$f(x)$值域内单调,这样我们就可以转化证明
    \[g(x_1)>g(x_2)\Leftrightarrow h_1(g(x_1))>h_1(g(x_2))\Leftrightarrow  \varphi(x_1)>\varphi(x_2)\]
    但是显然我们不必大费周章地考虑$h_2(x)$,除非你想优雅地解题。此处我们可以直接取$\varphi(x)=h_1(g(x))-f(x)$,直观一点来说,只需要找到$(x-\dfrac12)^2$的某个复合函数穿过$\bigg(\dfrac12,f(\dfrac12)\bigg)$这个点,且在该点左侧比$f(x)$“矮”,右侧比$f(x)$“高”,那么就能证明$g(x_1)>g(x_2)$。那既然要求右侧比$f(x)$“高”,我们初步考虑$h_1(x)=e^{g(x)}$,由
    \[\text{e}^x=1+x+\dfrac{x^2}{2!}+\dfrac{x^3}{3!}+\cdots\Rightarrow e^{(x-\frac12)^2}=1+(x-\dfrac12)^2+\dfrac{(x-\dfrac12)^4}{2!}+\dfrac{(x-\dfrac12)^6}{3!}+\cdots\]
    所以消去偶数次项得到$\varphi(x)=\dfrac{\sqrt{\text{e}}+4}{2}e^{g(x)}-f(x)-2+\ln 2+\dfrac{\sqrt{\text{e}}}{2}$,即
    \begin{align*}
        \varphi(x)&=\dfrac{\sqrt{\text{e}}+4}{2}e^{(x-\frac12)^2}-\text{e}^x+\ln x-2+\ln 2+\dfrac{\sqrt{\text{e}}}{2}\\
        \varphi'(x)&=(\sqrt{\text{e}}+4)(x-\dfrac12)\text{e}^{(x-\frac12)^2}-\text{e}^x+\dfrac1{x}
    \end{align*}
    由刚才分析可知,$\varphi''(x)$显然单增,且满足$\varphi''(\dfrac12)=0$,故$\varphi'(x)\geq\varphi'(\dfrac12)>0$,所以$\varphi(x)$单增,$\varphi(x_1)>\varphi(x_2)$,原题得证。
\end{solution}
\newpage
\begin{example}{}{}
    已知正数$a,b,c$满足$abc=1$,证明:\[\sqrt{a^2+1}+\sqrt{b^2+1}+\sqrt{c^2+1}\leqslant\sqrt{2}\left(a+b+c\right)\]
\end{example}
\begin{solution}
    等价于证明:$$ \left( \sqrt{a^2+1} - \sqrt{2}a \right) + \left( \sqrt{b^2+1} - \sqrt{2}b \right) + \left( \sqrt{c^2+1} - \sqrt{2}c \right) \leqslant 0. $$由于 $a, b, c \in (0, +\infty)$,可设 $a = e^x, b = e^y, c = e^z$,其中 $x, y, z \in \mathbb{R}$。由约束条件 $abc=1$ 可得 $e^{x+y+z} = 1$,即 $x+y+z=0$。构造函数 $F(t) = \sqrt{e^{2t}+1} - \sqrt{2}e^t \ (t \in \mathbb{R})$,原命题等价转化为证明在 $x+y+z=0$ 的条件下,恒有:$$ F(x) + F(y) + F(z) \leqslant 0. $$对 $F(t)$ 求一阶导数:$$ F'(t) = \frac{1}{2}(e^{2t}+1)^{-\frac{1}{2}} \cdot 2e^{2t} - \sqrt{2}e^t = \frac{e^{2t}}{\sqrt{e^{2t}+1}} - \sqrt{2}e^t. $$利用商的求导法则求二阶导数:$$ F''(t) = \frac{2e^{2t}\sqrt{e^{2t}+1} - e^{2t} \cdot \frac{e^{2t}}{\sqrt{e^{2t}+1}}}{e^{2t}+1} - \sqrt{2}e^t = \frac{2e^{2t}(e^{2t}+1) - e^{4t}}{(e^{2t}+1)^{\frac{3}{2}}} - \sqrt{2}e^t = \frac{e^{4t} + 2e^{2t}}{(e^{2t}+1)^{\frac{3}{2}}} - \sqrt{2}e^t. $$下证对于任意 $t \in \mathbb{R}$,恒有 $F''(t) < 0$。该不等式等价于:$$ \frac{e^{4t} + 2e^{2t}}{(e^{2t}+1)^{\frac{3}{2}}} < \sqrt{2}e^t. $$由于对于任意 $t \in \mathbb{R}$,上式两边均为正数,将两边同时平方进行等价变形,可得:$$ \frac{e^{4t}(e^{2t}+2)^2}{(e^{2t}+1)^3} < 2e^{2t}. $$将正数 $(e^{2t}+1)^3$ 乘至不等式右边,得:$$ e^{4t}(e^{2t}+2)^2 < 2e^{2t}(e^{2t}+1)^3. $$分别展开不等式两侧:左边 $= e^{4t}(e^{4t} + 4e^{2t} + 4) = e^{8t} + 4e^{6t} + 4e^{4t}$,右边 $= 2e^{2t}(e^{6t} + 3e^{4t} + 3e^{2t} + 1) = 2e^{8t} + 6e^{6t} + 6e^{4t} + 2e^{2t}$。两边作差(右边减去左边)整理得:$$ e^{8t} + 2e^{6t} + 2e^{4t} + 2e^{2t} > 0. $$由于对于任意 $t \in \mathbb{R}$ 恒有 $e^{2t} > 0$,上述多项式的各项均为严格正数,其和显然大于 $0$。因为上述推导每一步均为等价变形,故 $F''(t) < 0$ 对全体实数 $t$ 恒成立,即 $F(t)$ 是定义在 $\mathbb{R}$ 上的严格上凸函数。基于 $F(t)$ 的严格上凸性,可通过以下两种等价方式完成证明:方法一:利用琴生(Jensen)不等式对于任意实数 $x, y, z$,有:$$ F(x) + F(y) + F(z) \leqslant 3F\left( \frac{x+y+z}{3} \right). $$将已知条件 $x+y+z=0$ 代入,得:$$ F(x) + F(y) + F(z) \leqslant 3F(0). $$计算 $F(0)$ 的值:$$ F(0) = \sqrt{e^0+1} - \sqrt{2}e^0 = \sqrt{2} - \sqrt{2} = 0. $$从而 $F(x) + F(y) + F(z) \leqslant 0$,原不等式得证。方法二:利用切线放缩由于 $F(t)$ 为严格上凸函数,其图像始终位于其任意一点的切线下方(除切点外)。取 $t=0$,计算可得函数值 $F(0) = 0$,导数值 $F'(0) = \frac{e^0}{\sqrt{e^0+1}} - \sqrt{2}e^0 = \frac{1}{\sqrt{2}} - \sqrt{2} = -\frac{\sqrt{2}}{2}$。故 $F(t)$ 在 $t=0$ 处的切线方程为 $y = -\frac{\sqrt{2}}{2}t$。对于任意 $t \in \mathbb{R}$,恒有切线不等式:$$ F(t) \leqslant F(0) + F'(0)(t - 0) \implies F(t) \leqslant -\frac{\sqrt{2}}{2}t. $$分别将 $x, y, z$ 代入上述不等式并相加得:$$ F(x) + F(y) + F(z) \leqslant -\frac{\sqrt{2}}{2}(x+y+z). $$结合已知条件 $x+y+z=0$,可得:$$ F(x) + F(y) + F(z) \leqslant 0. $$原不等式同样得证。最后分析等号成立的条件。由于 $F(t)$ 为严格上凸函数,上述不等式取等号的充分必要条件为自变量相等,即 $x=y=z$。结合约束条件 $x+y+z=0$,解得 $x=y=z=0$。逆向代回原变量,即当且仅当 $a=b=c=1$ 时等号成立,此时符合 $abc=1$ 的前提条件。综上所述,原不等式 $\sqrt{a^2+1}+\sqrt{b^2+1}+\sqrt{c^2+1}\leqslant\sqrt{2}\left(a+b+c\right)$ 得证。
\end{solution}

\begin{example}{远古偏移题,来自陈语梦}{}
已知 $x_{1} - \ln(\ln x_{1} + 1) = x_{2} - \ln(\ln x_{2} + 1) = m$求证\[m + 1 < x_{1} + x_{2} < \frac{7}{6}m + \frac{5}{6}\]
\end{example}
\begin{solution}
    设函数$f(x)=x-\ln(\ln x+1)-1$,左侧化为\[g(x)=x^2-x\left(m+1\right)+1=x\ln(\ln x+1)-x+1,~~~x_1<1<x_2\Leftrightarrow g(x_1)>g(x_2)\]
    加强证明$g(x)$单调递增,求导得$g'(x)=\ln(\ln x+1)+x\cdot\frac1{\ln x+1}\cdot\frac1x-1=\ln(\ln x+1)+\frac1{\ln x+1}-1$
    又因为任意$x>0$有$\ln x>1-\frac1x$,所以$g'(x)>0$,故$g(x)$单调递增。右侧化为
    \[g(x)=x^2-x\left(\frac{7}{6}f(x) + \frac{5}{6}\right)+1=1-\frac{{x}^{2}}{6}-\frac{5}{6}x+\frac76x\ln(\ln({x})+1)\]
    令 $F(x) = x^2 - x(\frac{7}{6}f(x) + \frac{5}{6}) + \frac{1}{6}(f(x) - 1) + \frac{7}{72}(f(x) - 1)^2$。即求证$F(x)$ 在其定义域 $[1, +\infty)$ 内单调递减。
    要证 $F(x)$ 在 $[1, +\infty)$ 上单调递减,等价于证明其导函数满足 $F'(x) \leqslant 0$,即证明以下不等式在 $x \geqslant 1$ 时恒成立:$$\frac{5xe^{2(x-1)} + 31xe^{x-1} - 35e^{x-1} - 1}{7 + 35xe^{x-1}} \geqslant \ln x$$首先,引入辅助函数 $h(x) = \frac{x^2+4x-5}{4x+2} - \ln x$。对其求导可得:$$h'(x) = \frac{(2x+4)(4x+2) - (x^2+4x-5) \cdot 4}{(4x+2)^2} - \frac{1}{x} = \frac{4x^2+4x+28}{(4x+2)^2} - \frac{1}{x}$$通分化简得:$$h'(x) = \frac{x(4x^2+4x+28) - (16x^2+16x+4)}{x(4x+2)^2} = \frac{4x^3-12x^2+12x-4}{x(4x+2)^2} = \frac{4(x-1)^3}{x(4x+2)^2}$$当 $x \geqslant 1$ 时,$h'(x) \geqslant 0$ 显然成立,故 $h(x)$ 在 $[1, +\infty)$ 上单调递增。又因 $h(1) = 0$,故当 $x \geqslant 1$ 时,$h(x) \geqslant 0$ 恒成立,即得到不等式放缩:$\frac{x^2+4x-5}{4x+2} \geqslant \ln x$。基于上述结论,为证原不等式,只需证明:$$\frac{5xe^{2(x-1)} + 31xe^{x-1} - 35e^{x-1} - 1}{7 + 35xe^{x-1}} \geqslant \frac{x^2+4x-5}{4x+2}$$由于 $x \geqslant 1$,不等式两边的分母均大于零。将两边交叉相乘得:$$(4x+2)(5xe^{2(x-1)} + 31xe^{x-1} - 35e^{x-1} - 1) \geqslant (x^2+4x-5)(7 + 35xe^{x-1})$$将其按 $e^{x-1}$ 的次幂展开并合并同类项,可得:$e^{2(x-1)}$ 项系数为 $20x^2 + 10x$;$e^{x-1}$ 项系数为 $(4x+2)(31x-35) - 35x(x^2+4x-5) = -35x^3 - 16x^2 + 97x - 70$;常数项为 $(4x+2)(-1) - 7(x^2+4x-5) = -7x^2 - 32x + 33$。移项整理后,不等式转化为:$$(20x^2+10x)e^{2(x-1)} + (-35x^3-16x^2+97x-70)e^{x-1} - (7x^2+32x-33) \geqslant 0$$将不等式两端同乘 $e^{1-x}$(因指数函数恒正,不等号方向不变),构造函数 $g(x)$:$$g(x) = (20x^2+10x)e^{x-1} - 35x^3 - 16x^2 + 97x - 70 - e^{1-x}(7x^2+32x-33)$$问题转化为证明当 $x \geqslant 1$ 时,$g(x) \geqslant 0$ 恒成立。对 $g(x)$ 求一阶导数:$$g'(x) = e^{x-1}(20x^2+50x+10) - 105x^2 - 32x + 97 + e^{1-x}(7x^2+18x-65)$$再次求导得二阶导数:$$g''(x) = e^{x-1}(20x^2+90x+60) - 210x - 32 + e^{1-x}(-7x^2-4x+83)$$为判断 $g''(x)$ 的符号,我们先证明 $g''(x) \geqslant (1-xe^{1-x})(7x+4)$。将该目标不等式的右端展开并移项,等价于证明:$$e^{x-1}(20x^2+90x+60) + 83e^{1-x} - 217x - 36 \geqslant 0$$由基本不等式 $e^z \geqslant z+1$,分别取 $z = x-1$ 与 $z = 1-x$,可得 $e^{x-1} \geqslant x$ 与 $e^{1-x} \geqslant 2-x$。当 $x \geqslant 1$ 时,$20x^2+90x+60 > 0$,将上述放缩代入左侧多项式:$$e^{x-1}(20x^2+90x+60) + 83e^{1-x} - 217x - 36 \geqslant x(20x^2+90x+60) + 83(2-x) - 217x - 36$$$$= 20x^3 + 90x^2 + 60x + 166 - 83x - 217x - 36$$$$= 20x^3 + 90x^2 - 240x + 130$$提取公因式并进行因式分解得:$$10(2x^3 + 9x^2 - 24x + 13) = 10(x-1)(2x^2+11x-13) = 10(x-1)^2(2x+13)$$由于 $x \geqslant 1$,$10(x-1)^2(2x+13) \geqslant 0$ 显然成立。这证明了 $g''(x) \geqslant (1-xe^{1-x})(7x+4)$。同时,当 $x \geqslant 1$ 时,$xe^{1-x} \leqslant 1$ 恒成立,即 $1-xe^{1-x} \geqslant 0$。结合 $7x+4 > 0$,可知:$$g''(x) \geqslant (1-xe^{1-x})(7x+4) \geqslant 0$$故 $g'(x)$ 在 $[1, +\infty)$ 上单调递增。代入端点值检验:$$g'(1) = 80 - 105 - 32 + 97 - 40 = 0$$因此,当 $x \geqslant 1$ 时,$g'(x) \geqslant 0$ 恒成立,说明 $g(x)$ 在 $[1, +\infty)$ 上单调递增。再次代入端点值检验:$$g(1) = 30 - 35 - 16 + 97 - 70 - 6 = 0$$故当 $x \geqslant 1$ 时,$g(x) \geqslant 0$ 恒成立。由此逆推可知,等价转化的核心不等式成立,即原导函数满足 $F'(x) \leqslant 0$ 恒成立。综上所述,$F(x)$ 在其定义域 $[1, +\infty)$ 内单调递减,原命题得证。
\end{solution}
\newpage
\begin{example}{来自amare Donata Caesia}{}
    $x_1\ln x_1=x_2\ln x_2=-x_3\ln x_3,x_1<x_2<x_3$,证明$x_1+x_2+x_3>2$
\end{example}
\begin{solution}
考虑加强证明\[\begin{cases}x_1+x_2>1+m+\frac{m^2}2\\x_3>1-m-\frac{m^2}2\end{cases}\]
第一个不等式,考虑加强证明$x_1+x_2>\e(3-\e)m^2+m+1$,化成$x-\ln x$的经典偏移模型,即证\[x_1-\ln x_1=x_2-\ln x_2=a>1,\frac{1}{x_1}+\frac{1}{x_2}>\e^a-1+\dfrac{\e(3-\e)}{\e^a}\]
假设$\frac{1}{x_1}>\frac{1}{x_2},x_1<1<x_2$,即证\[g(x)=\frac{1}{x^2}-\left(\e^a-1+\dfrac{\e(3-\e)}{\e^a}\right)\frac{1}{x}=\frac{1}{x^2}-\left(\frac{\e^x}{x}-1+\e(3-\e)\dfrac{x}{\e^x}\right)\frac{1}{x},g(x_1)>g(x_2)\]则$x=1$必然是极值点,只需要改造$g(x)$使得其单调递减即可完成加强证明,显然地,考虑到$x_1-\ln x_1=x_2-\ln x_2=a$,取指数就有$\frac{\e^{x_1}}{x_1}=\frac{\e^{x_2}}{x_2}=\e^a$,只需证明
\begin{align*}
h(x)&=\frac{1}{x^2}-\frac{\frac{\mathrm{e}^x}{x}-1+\mathrm{e}(3-\mathrm{e})\frac{x}{\mathrm{e}^x}}{x}+1+\mathrm{e}(5-2\mathrm{e})\frac{x}{\mathrm{e}^x}\\
h'(x)&=\frac{\mathrm{e}^x(2-x)-2}{x^3}-\frac{1}{x^2}+\mathrm{e}\mathrm{e}^{-x}\begin{bmatrix}8-3\mathrm{e}-(5-2\mathrm{e})x\end{bmatrix}
\end{align*}
$h(x)$单调递减即可,设$H(x)=x^3\e^xh'(x)=\mathrm{e}^{2x}(2-x)-(x+2)\mathrm{e}^x+\mathrm{e}x^3\left[8-3\mathrm{e}-(5-2\mathrm{e})x\right]$求4次导数把$x$的幂函数导掉:
\begin{align*}
    H'(x)&=e^{2x}(3-2x)-e^x(x+3)+3e(8-3e)x^2-4e(5-2e)x^3\\
H''(x)&=4\e^{2x}(1-x)-\e^x(x+4)+6\e(8-3\e)x-12\e(5-2\e)x^2\\
H'''(x)&=4\e^{2x}(1-2x)-\e^x(x+5)+6\e(8-3e)-24\e(5-2\e)x\\
H''''(x)&=-16x\e^{2x}-(x+6)\e^x-24\e(5-2\e)
\end{align*}
显然对于$x>0$,$H''''(x)$单调递减,且$H''''(0)=-6+24\e(2\e-5)>0,\displaystyle \lim_{x\to+\infty}H''''(x)=-\infty$,故其存在零点$(\xi_1,0)$,则$H'''(x)$在$(0,\xi_1)$单增,在$(\xi_1,+\infty)$单减,由于
\[H'''(0)=-1+6\e(8-3\e)<0,H'''(1)=26\e^2-78\e<0,H'''\left(\frac12\right)=6\e^2-12\e-\frac{11}{2}\sqrt{\e}>0\]
故$H'''(x)$存在零点$(\xi_2,0),(\xi_3,0)$,那么$H''(x)$在$(0,\xi_2)$单减,在$(\xi_2,\xi_3)$单增,在$(\xi_3,+\infty)$单减,由于
\[H''(0)=0,H''\left(\frac12\right)=11\e-\frac{9}{2}\sqrt{\e}-3\e^2>0,H''(1)=6\e^2-17\e<0\]
所以$H''(x)$存在零点$(\xi_4,0),(\xi_5,0)$,所以$H'(x)$在$(0,\xi_4)$单减,在$(\xi_4,\xi_5)$单增,在$(\xi_5,\infty)$单减,由于$H'(0)=H'(1)=0$,所以$H'(x)$存在零点$(\xi_6,0)$,那么$H(x)$在$(0,\xi_6)$单减,在$(\xi_6,1)$单增,在$(1,+\infty)$单减,又因为$H(0)=H(1)=0$,所以$H(x)\leqslant 0$,故$h(x)$在$(0,+\infty)$单减。于是$x_1+x_2>\e(3-\e)m^2+m+1$得证.

下证$x_3>1-m-\frac{m^2}2>1$,对于$m\in(-\frac1{\e},0)$,不难验证后一个“$>$”成立,两边套函数名$-m=x_3\ln x_3>(1-m-\frac{m^2}2)\ln (1-m-\frac{m^2}2)$即证
\[p(m)=\frac{m}{\frac{m^2}2+m-1}-\ln (1-m-\frac{m^2}2)>0,\forall m\in(-\frac1{\e},0)\]
令$t=-m\in(0,\frac{1}{e})$,则不等式化为
$$f(t)=\frac{t}{1+t-\frac{t^{2}}{2}}-\ln \left(1+t-\frac{t^{2}}{2}\right)>0,~~f'(t)=\frac{t^{2}\left(2-\frac{t}{2}\right)}{\left(1+t-\frac{t^{2}}{2}\right)^{2}}>0, \quad \forall t\in\left(0,\frac{1}{e}\right)$$
且$f(0)=0$,故$f(t)>0$在区间内成立,即$x_3>1-m-\frac{m^2}2>1$。这就证明了$x_1+x_2+x_3>2$.
\end{solution}

\newpage
\newpage
\begin{example}{}{}
    $a,d\geq 0,b,c>0,b+c\geq a+d$,求$\dfrac{b}{c+d}+\dfrac{c}{a+b}$的最小值.
\end{example}
\begin{solution}
    首先证明函数的最小值必在边界 $a+d = b+c$ 处取得。假设存在一组满足 $a_0+d_0 < b_0+c_0$ 的点 $(a_0, b_0, c_0, d_0)$ 使函数取得极小值。由于 $b_0+c_0 - d_0 > a_0 \ge 0$,可构造一个新点 $(a_1, b_0, c_0, d_0)$,其中 $a_1 = b_0+c_0-d_0$。该点满足所有非负与正数约束,且满足等式 $a_1+d_0 = b_0+c_0$。将新点代入目标函数,因 $a_1 > a_0$,第一项 $\frac{b_0}{c_0+d_0}$ 保持不变,第二项的分母变大,即 $\frac{c_0}{a_1+b_0} < \frac{c_0}{a_0+b_0}$。由此得到 $f(a_1, b_0, c_0, d_0) < f(a_0, b_0, c_0, d_0)$。因此只需求解 $a+d = b+c$ 条件下的最值即可。
    
    齐次函数,即 $f(ka, kb, kc, kd) = f(a,b,c,d)$($k>0$),说明函数值仅与变量的相对比例有关。由 $b, c > 0$ 必有 $b+c > 0$。为简化变量,将所有变量同除以 $b+c$,\textbf{即假定 $b+c = 1$},从而也有 $a+d = 1$。引入换元:令 $b = y$,则 $c = 1-y$,由 $b, c > 0$ 得 $y \in (0, 1)$;令 $a = x$,则 $d = 1-x$,由 $a, d \ge 0$ 得 $x \in [0, 1]$。代入目标函数得:$$f(x,y) = \frac{y}{(1-y)+(1-x)} + \frac{1-y}{x+y} = \frac{y}{2-x-y} + \frac{1-y}{x+y}$$令 $S = x+y$。因 $x \in [0, 1]$ 且 $y \in (0, 1)$,故 $S \in (0, 2)$。对于固定的 $S$,由 $x = S-y \in [0, 1]$ 可得 $S-1 \le y \le S$。结合 $y \in (0, 1)$,$y$ 的取值范围为 $[\max(0, S-1), \min(1, S)] \cap (0, 1)$。将函数转化为关于 $S$ 和 $y$ 的表达式:$$f(S, y) = \frac{y}{2-S} + \frac{1-y}{S} = \frac{2S-2}{S(2-S)} y + \frac{1}{S}$$在固定 $S$ 的截面上,目标函数是关于 $y$ 的一次函数,其极值在区间端点处取得。根据一次项系数 $\frac{2S-2}{S(2-S)}$ 的正负,分以下三种情况讨论:当 $0 < S < 1$ 时,一次项系数 $\frac{2S-2}{S(2-S)} < 0$,函数关于 $y$ 单调递减。为使函数值最小,$y$ 应取区间内的最大值,即 $y = \min(1, S) = S$(此时 $x = 0$)。代入可得该截面上的极小值:$$g_1(S) = f(S, S) = \frac{S}{2-S} + \frac{1-S}{S} = \frac{2}{2-S} + \frac{1}{S} - 2$$当 $1 < S < 2$ 时,一次项系数 $\frac{2S-2}{S(2-S)} > 0$,函数关于 $y$ 单调递增。为使函数值最小,$y$ 应取区间内的最小值,即 $y = \max(0, S-1) = S-1$(此时 $x = 1$)。代入可得:$$g_2(S) = f(S, S-1) = \frac{S-1}{2-S} + \frac{1-(S-1)}{S} = \frac{S-1}{2-S} + \frac{2-S}{S}$$当 $S = 1$ 时,一次项系数为 $0$。此时无论 $y$ 取何值,恒有 $f(1, y) = 1$。接下来求解 $g_1(S)$ 在 $S \in (0, 1)$ 上的最小值。注意到 $(2-S) + S = 2$,利用基本不等式对其变形放缩:$$2 \left(\frac{2}{2-S} + \frac{1}{S}\right) = ((2-S)+S)\left(\frac{2}{2-S} + \frac{1}{S}\right) = 3 + \frac{2S}{2-S} + \frac{2-S}{S}$$由均值不等式可得:$$\frac{2S}{2-S} + \frac{2-S}{S} \ge 2\sqrt{\frac{2S}{2-S} \cdot \frac{2-S}{S}} = 2\sqrt{2}$$故 $\frac{2}{2-S} + \frac{1}{S} \ge \frac{3+2\sqrt{2}}{2} = \sqrt{2} + \frac{3}{2}$。代回 $g_1(S)$ 可得:$$g_1(S) \ge \left(\sqrt{2} + \frac{3}{2}\right) - 2 = \sqrt{2} - \frac{1}{2}$$等号成立当且仅当 $\frac{2S}{2-S} = \frac{2-S}{S}$,即 $2-S = \sqrt{2}S$。解得 $S = 2\sqrt{2}-2 \in (0, 1)$,极小值点在定义域内。同理,考察 $g_2(S)$ 在 $S \in (1, 2)$ 上的最小值。令 $t = 2-S \in (0, 1)$,代入得:$$g_2(S) = \frac{1-t}{t} + \frac{t}{2-t} = \frac{1}{t} - 1 + \frac{2}{2-t} - 1 = \frac{2}{2-t} + \frac{1}{t} - 2$$该表达式与 $g_1(t)$ 的代数形式完全相同,因此其最小值同为 $\sqrt{2} - \frac{1}{2}$。由于 $\sqrt{2} - \frac{1}{2} \approx 0.914 < 1$,故该值即为全局极小值。最后进行充分性验证。取 $S = 2\sqrt{2}-2$,$y = 2\sqrt{2}-2$,$x = 0$,还原为原变量得:$a = 0$,$b = 2\sqrt{2}-2$,$c = 3-2\sqrt{2}$,$d = 1$。经检验,满足 $a \ge 0, d \ge 0, b > 0, c > 0$,且 $b+c = 1 \ge 1 = a+d$。代入原函数验算,确实取得 $\sqrt{2} - \frac{1}{2}$,解的存在性成立。综上所述,$\dfrac{b}{c+d}+\dfrac{c}{a+b}$ 的最小值为 $\sqrt{2} - \dfrac{1}{2}$。
\end{solution}
\begin{example}{}{}
    证明$\left(\dfrac{1}{n}\right)^n+\left(\dfrac{2}{n}\right)^n+\left(\dfrac{3}{n}\right)^n+\cdots+\left(\dfrac{n-1}{n}\right)^n+\left(\dfrac{n}{n}\right)^n>\dfrac{2(n-1)}{n+1}$.
\end{example}
\begin{solution}
    这个不等式有明显的几何意义,首先分母都有$n$,然后分子每次自增,长得很像一个个矩形的面积求和,然后右边是个分式,感觉就是积分后的结果,所以我们根据题目中的$n$次方结构,找到函数$f(x)=x^n$,然后将其积分区间$[0,1]$分成$n$等分,并由$f(x)$下凸,将每一个曲边梯形“扩大”成直边梯形(为什么不先放成矩形?因为放成梯形明显更紧,而且求和的难度不会显著变大),根据题中的$k=1,2,3,\cdots,n$:
    \[\dfrac{1}{2}\left(\left(\dfrac{k-1}{n}\right)^n+\left(\dfrac{k}{n}\right)^n\right)\left(\dfrac{k}{n}-\dfrac{k-1}{n}\right)>\int_{\frac{k-1}{n}}^{\frac{k}{n}}x^n\dd x\]
    求和得到
    \begin{align*}
    &\left(\dfrac{1}{n}\right)^n+2\left(\dfrac{2}{n}\right)^n+2\left(\dfrac{3}{n}\right)^n+\cdots+2\left(\dfrac{n-1}{n}\right)^n+\left(\dfrac{n}{n}\right)^n>n\int_{0}^1x^n\dd x\\
    &\Leftrightarrow 2\sum_{k=1}^n\left(\dfrac{k}{n}\right)^n-\left(\dfrac{0}{n}\right)^n-\left(\dfrac{n}{n}\right)^n>\left.\dfrac{x^{n+1}}{n+1}\right|^{1}_0=\dfrac{n}{n+1}\\
    &\Leftrightarrow \sum_{k=1}^n\left(\dfrac{k}{n}\right)^n>\dfrac{2(n-1)}{n+1}\\
    \end{align*}
\end{solution}
\begin{example}{}{}
    证明$f(x)=\ln x-8\dfrac{x-1}{(x^{\frac13}+1)^3}$单调递增
\end{example}
\begin{solution}
    转化为$\displaystyle F(t)=3\ln t-8\cdot\frac{t^3-1}{(t+1)^3}$单增,$\displaystyle F^{\prime}(t)=\frac{3}{t}-8\cdot\frac{3(t^2+1)}{(t+1)^4}=\frac{3}{t}-\frac{24(t^2+1)}{(t+1)^4}$
    链式法则$f^{\prime}(x)=\frac{dF}{dt}\cdot\frac{dt}{dx}=F^{\prime}(t)\cdot\frac{1}{3t^2}$,所以\[f^{\prime}(x)=\left(\frac{3}{t}-\frac{24(t^2+1)}{(t+1)^4}\right)\cdot\frac{1}{3t^2}=\frac{1}{t^3}-\frac{8(t^2+1)}{t^2(t+1)^4}=\frac{(t-1)^4}{t^3(t+1)^4}>0.\]
\end{solution}

\begin{example}{来自邪帝,$\sqrt{xy}\left(e^x+e^y\right)\leqslant2$}{}
    正实数$x,y$满足$e^x-\ln x=e^y-\ln y$,证明:$\sqrt{xy}\left(e^x+e^y\right)\leqslant2$
\end{example}
\begin{solution}
设函数 $f(t) = e^t - \ln t \ (t > 0)$,对其求导得 $f'(t) = e^t - \frac{1}{t}$。
令 $f'(t) = 0$ 得到 $t e^t = 1$,该方程存在唯一正实数解 $t_0 \approx 0.567$。
这表明 $f(t)$ 在 $(0, t_0]$ 上严格单调递减,在 $[t_0, +\infty)$ 上严格单调递增。
已知正实数 $x, y$ 满足 $f(x) = f(y)$。
若 $x = y$,则必然有 $x = y = t_0$,此时目标不等式左侧为 $\sqrt{t_0^2}(2e^{t_0}) = 2t_0 e^{t_0}$。由 $t_0 e^{t_0} = 1$ 可知,此时式子的值为 $2$,满足 $\leqslant 2$ 的条件。
若 $x \neq y$,不失一般性,假设 $0 < x < t_0 < y$。下证在此条件下严格不等式 $\sqrt{xy}(e^x + e^y) < 2$ 成立。

设 $m = \frac{x+y}{2}$,$v = \frac{y-x}{2}$,则 $v > 0$ 且 $x = m-v$,$y = m+v$。
另设 $w = \frac{\ln y - \ln x}{2}$,则 $w > 0$。
由 $\ln \frac{y}{x} = 2w$ 得 $y = x e^{2w}$。将其代入 $y - x = 2v$ 中,解得 $x = \frac{2v}{e^{2w}-1} = \frac{v e^{-w}}{\sinh w}$。
同理可得 $y = \frac{v e^w}{\sinh w}$。
由此,几何平均数 $\sqrt{xy}$ 可表示为:
$$\sqrt{xy} = \sqrt{\frac{v^2 e^{-w} e^w}{\sinh^2 w}} = \frac{v}{\sinh w}$$
根据已知条件 $e^y - \ln y = e^x - \ln x$,可转化为 $e^y - e^x = \ln y - \ln x$。
将 $m, v$ 代入左侧,得 $e^{m+v} - e^{m-v} = 2e^m \sinh v$;右侧即为 $2w$。
两边相等得到等式:
$$e^m \sinh v = w \implies e^m = \frac{w}{\sinh v}$$
利用此关系,代数和 $\frac{e^x + e^y}{2}$ 可化为:
$$\frac{e^x + e^y}{2} = \frac{e^{m-v} + e^{m+v}}{2} = e^m \cosh v = \left(\frac{w}{\sinh v}\right) \cosh v = w \coth v$$
要证的 $\sqrt{xy}(e^x + e^y) < 2$ 等价于 $\sqrt{xy} \frac{e^x + e^y}{2} < 1$。
将上述结果代入,目标不等式等价于 $\left( \frac{v}{\sinh w} \right) (w \coth v) < 1$,即:
$$v \coth v < \frac{\sinh w}{w}$$

为证明此不等式,需考察变量 $w$ 和 $v$ 的比值 $R = \frac{w}{v}$ 的下界。
一方面,由 $R = \frac{w}{v} = e^m \frac{\sinh v}{v}$,且对于 $v > 0$,泰勒展开式 $\frac{\sinh v}{v} = 1 + \frac{v^2}{6} + \dots > 1$,可得 $R > e^m$。
另一方面,将 $y = m+v$ 和 $x = m-v$ 代入 $R$ 的定义中,有:
$$R = \frac{1}{2v} \ln \frac{m+v}{m-v} = \frac{1}{2v} \ln \frac{1+v/m}{1-v/m}$$
令 $t = \frac{v}{m}$,由于 $0 < x < y$ 且 $v = \frac{y-x}{2}, m = \frac{x+y}{2}$,知 $t \in (0, 1)$。
由对数函数的泰勒展开 $\ln \frac{1+t}{1-t} = 2 \left( t + \frac{t^3}{3} + \dots \right) > 2t$,得 $R > \frac{1}{2v} \cdot 2\left(\frac{v}{m}\right) = \frac{1}{m}$。
综合两者可得 $R > \max\left(e^m, \frac{1}{m}\right)$。
设 $h(m) = \max\left(e^m, \frac{1}{m}\right)$,因 $e^m$ 单调递增,$\frac{1}{m}$ 单调递减,其最小值在两函数图象交点处取得,即 $e^{m_0} = \frac{1}{m_0}$,亦即 $m_0 e^{m_0} = 1$。
故对于任意 $m > 0$,恒有 $R > e^{m_0}$。
考察单调递增函数 $g(m) = m e^m$,已知 $g(m_0) = 1$。
取 $m = \ln \sqrt{2}$,计算得 $g(\ln \sqrt{2}) = (\ln \sqrt{2}) e^{\ln \sqrt{2}} = \frac{\ln 2}{\sqrt{2}}$。
由于 $e > 2$,故 $\ln 2 < 1$,从而 $g(\ln \sqrt{2}) < \frac{1}{\sqrt{2}} < 1 = g(m_0)$。
由 $g(m)$ 的严格单调递增性,可得 $\ln \sqrt{2} < m_0$,即 $e^{m_0} > \sqrt{2}$。
因此,$R = \frac{w}{v} > \sqrt{2}$,即 $w > \sqrt{2} v$。

设函数 $S(t) = \frac{\sinh t}{t} \ (t>0)$,其导数 $S'(t) = \frac{t \cosh t - \sinh t}{t^2} > 0$,故 $S(t)$ 严格单调递增。
由 $w > \sqrt{2} v > 0$,得 $\frac{\sinh w}{w} > \frac{\sinh(\sqrt{2} v)}{\sqrt{2} v}$。
为证明 $v \coth v < \frac{\sinh w}{w}$,只需证 $v \coth v \leqslant \frac{\sinh(\sqrt{2} v)}{\sqrt{2} v}$,即证明 $\sqrt{2} v^2 \cosh v \leqslant \sinh v \sinh(\sqrt{2} v)$。
利用积化和差公式,不等式右侧化为:
$$\text{RHS} = \frac{1}{2} \Big[ \cosh((\sqrt{2}+1)v) - \cosh((\sqrt{2}-1)v) \Big]$$
将其展开为泰勒级数,设 $\text{RHS} = \sum_{n=1}^{\infty} \frac{C_n}{(2n)!} v^{2n}$,其中:
$$C_n = \frac{1}{2} \left[ (\sqrt{2}+1)^{2n} - (\sqrt{2}-1)^{2n} \right] = \sum_{k=0}^{n-1} \binom{2n}{2k+1} (\sqrt{2})^{2k+1}$$
不等式左侧展开为泰勒级数:
$$\text{LHS} = \sqrt{2} v^2 \sum_{j=0}^{\infty} \frac{v^{2j}}{(2j)!} = \sum_{n=1}^{\infty} \frac{\sqrt{2}}{(2n-2)!} v^{2n} = \sum_{n=1}^{\infty} \frac{\sqrt{2}(2n)(2n-1)}{(2n)!} v^{2n}$$
设 LHS 的分子系数为 $D_n = \sqrt{2}(2n)(2n-1)$。比较 $C_n$ 与 $D_n$:
当 $n=1$ 时,$C_1 = 2\sqrt{2}$,$D_1 = 2\sqrt{2}$,两者相等;
当 $n=2$ 时,$C_2 = 12\sqrt{2}$,$D_2 = 12\sqrt{2}$,两者相等;
当 $n \geqslant 3$ 时,提取 $C_n$ 求和式中的前两项($k=0$ 与 $k=1$):
$$C_n > \binom{2n}{1} \sqrt{2} + \binom{2n}{3} (\sqrt{2})^3 = 2n\sqrt{2} + \frac{2n(2n-1)(2n-2)}{3} \sqrt{2}$$
将该局部和与 $D_n$ 比较,作差并提取公因式 $\sqrt{2}(2n)$,考察:
$$1 + \frac{(2n-1)(2n-2)}{3} - (2n-1)$$
令 $z = 2n-1$,因 $n \geqslant 3$,故 $z \geqslant 5$。上式化为 $1 + \frac{z(z-1)}{3} - z = \frac{(z-1)(z-3)}{3} > 0$。
这表明当 $n \geqslant 3$ 时,局部和已严格大于 $D_n$,进而 $C_n > D_n$ 成立。
综上所述,对于任意 $v > 0$,泰勒级数各项系数均满足 $C_n \geqslant D_n$ 且部分项严格大于,故 $\text{RHS} > \text{LHS}$ 成立,即 $v \coth v < \frac{\sinh(\sqrt{2} v)}{\sqrt{2} v}$。
结合 $S(t)$ 的单调性,推得 $v \coth v < \frac{\sinh w}{w}$,等价于 $\sqrt{xy} \left(e^x + e^y\right) < 2$。
综合 $x=y$ 和 $x \neq y$ 的情况,命题 $\sqrt{xy}\left(e^x+e^y\right)\leqslant2$ 得证。
\end{solution}


\begin{example}{来自邪帝,无答案}{}{}
已知函数 $f(x) = ae^{x-1} - 2x + \ln x + e^{1-a}$ 有三个零点;若 $a > \frac{1}{2}$,记三个零点分别为 $x_1, x_2, x_3$ ($x_1 < x_2 < x_3$),极值点分别 $x_4, x_5$ ($x_4 < x_5$),证明:\\
(1)求$a$的取值范围.\\
(2)\\
(i) $x_1 x_2 x_3 < \sqrt[3]{\frac{2}{1 + e^{1-a}}}$\\
(ii)$x_1 + x_2 > 2x_4$, $x_2 + x_3 < 2x_5$\\
(iii) $x_1 + x_3 > x_4 + x_5 > 2x_2$\\
(iv) $x_2 > x_4 x_5 > x_1 x_3$\\
(v) $x_1 + 2x_2 + x_3 > 2x_4 + 2x_5$\\
(vi) $(x_1 - x_2)f(x_5) < (x_3 - x_2)f(x_4)$\\
(vii) $(x_2 - x_5)f(x_4) < (x_2 - x_4)f(x_5)$\\
(viii) $(x_1 + x_2 - 2x_4)f(x_4) < (x_2 + x_3 - 2x_5)f(x_5)$\\
(ix) $f(x_4)\ln x_1 < f(x_5)\ln x_3$\\
(x) $(x_4 + x_5)[f(x_4) + f(x_5)] + (x_1 + x_3)f(x_5) < 2x_2 f(x_4)$
\end{example}

\begin{example}{来自邪帝,无答案}
已知函数$f(x)=ax^{2}e^{-x}-\ln(ax)(a>0)$有三个零点$x_{1},x_{2},x_{3}(x_{1}<x_{2}<x_{3})$.  \\
(1)求$a$的取值范围;  \\
(2)记$f(x)$的导数为$g(x)$,$g(x)$的零点为$x_{4},x_{5}(x_{4}<x_{5})$.  \\
(i)证明:$g(x_{1})+g(x_{2})<0$;(ii)当$a<e^{2}$时,证明:$f(x_{5})<(x_{3}-1)g(x_{2})$
\end{example}

\begin{example}{来自邪帝,有答案但困难}{}
函数$f(x) = ae^{x-1} - (2 - x)x^2(a > 0)$有两个零点$x_1$,$x_2(0 < x_1 < x_2)$.\\
(1)求$a$的取值范围;\\
(2)若$x_1 + x_2 + kalna > 2$,求$k$的取值范围.
\end{example}

\begin{example}{来自邪帝,无答案}{}
    函数$f(x)=ae^{x}-2x+\ln \frac{x}{a}(a>0)$有三个零点$x_{1}$,$x_{2}$,$x_{3}(x_{1}<x_{2}<x_{3})$.  \\
(1)求实数$a$的取值范围;  \\
(2)记$f(x)$的极值点为$x_{4}$,$x_{5}(x_{4}<x_{5})$,证明:$(x_{4}-x_{2})f(x_{5})<(x_{5}-x_{2})f(x_{4})$.
\end{example}

\begin{example}{来自邪帝,无答案}{}
    已知函数$f(x)=x-4\ln x-\frac{3}{x}+3a+2(a<\ln 2)$有三个零点$x_{1}$,$x_{2}$,$x_{3}$($x_{1}<x_{2}<x_{3}$). \\ 
(1)求$a$的取值范围;  \\
(2)证明:$f(x_{1}x_{2})+f(x_{1}x_{2}x_{3})>3a$.
\end{example}
\begin{solution}
(1)函数 $f(x)$ 的定义域为 $(0, +\infty)$。
对 $f(x)$ 求导可得:
$$f'(x) = 1 - \frac{4}{x} + \frac{3}{x^2} = \frac{x^2 - 4x + 3}{x^2} = \frac{(x-1)(x-3)}{x^2}$$
当 $x \in (0, 1) \cup (3, +\infty)$ 时,$f'(x) > 0$;当 $x \in (1, 3)$ 时,$f'(x) < 0$。
因此,$f(x)$ 在 $(0, 1)$ 上单调递增,在 $(1, 3)$ 上单调递减,在 $(3, +\infty)$ 上单调递增。
函数在 $x=1$ 处取得唯一极大值 $f(1) = 3a$,在 $x=3$ 处取得唯一极小值 $f(3) = 3a + 4 - 4\ln 3$。

结合端点极限 $\lim_{x\rightarrow 0^+} f(x) = -\infty$ 及 $\lim_{x\rightarrow +\infty} f(x) = +\infty$,若使 $f(x)$ 有三个互不相同的零点 $x_1, x_2, x_3$(设 $x_1 < x_2 < x_3$),必须且只需极大值严格大于 $0$ 且极小值严格小于 $0$,即:
$$f(1) = 3a > 0 \implies a > 0$$
$$f(3) = 3a + 4 - 4\ln 3 < 0 \implies a < \frac{4\ln 3 - 4}{3}$$
综上,$a$ 的取值范围为 $\left(0, \frac{4\ln 3 - 4}{3}\right)$。此时三个零点分别位于区间 $(0, 1), (1, 3), (3, +\infty)$ 内。

(2)证明:对任意 $x > 0$,有
$$f\left(\frac{1}{x}\right) - f(x) = \left(\frac{1}{x} - 4\ln\frac{1}{x} - 3x + 3a + 2\right) - \left(x - 4\ln x - \frac{3}{x} + 3a + 2\right) = 4\left(\frac{1}{x} - x + 2\ln x\right)$$
构造函数 $\phi(x) = \frac{1}{x} - x + 2\ln x$,求导得 $\phi'(x) = -\frac{1}{x^2} - 1 + \frac{2}{x} = -\frac{(x-1)^2}{x^2} \le 0$。
故 $\phi(x)$ 在 $(0, +\infty)$ 上单调不增,且 $\phi(1) = 0$。
当 $x \in (0, 1)$ 时,$\phi(x) > 0$,从而 $f\left(\frac{1}{x}\right) > f(x)$。
由于 $x_1 \in (0, 1)$ 且 $f(x_1) = 0$,可得 $f\left(\frac{1}{x_1}\right) > 0$。

注意到 $a < \frac{4\ln 3 - 4}{3}$,计算 $f\left(\frac{1}{3}\right)$:
$$f\left(\frac{1}{3}\right) = \frac{1}{3} - 4\ln\frac{1}{3} - 9 + 3a + 2 = 3a + 4\ln 3 - \frac{20}{3} < (4\ln 3 - 4) + 4\ln 3 - \frac{20}{3} = 8\ln 3 - \frac{32}{3}$$
因 $e^{4/3} = 1 + \frac{4}{3} + \frac{1}{2}\left(\frac{4}{3}\right)^2 + \frac{1}{6}\left(\frac{4}{3}\right)^3 + \dots > 1 + \frac{4}{3} + \frac{8}{9} + \frac{32}{81} = \frac{293}{81} > 3$,故 $\ln 3 < \frac{4}{3}$。
从而 $8\ln 3 - \frac{32}{3} < 0$,即 $f\left(\frac{1}{3}\right) < 0$。
因 $f(x)$ 在 $(0, 1)$ 上单调递增且 $f(x_1) = 0$,由 $f\left(\frac{1}{3}\right) < f(x_1)$ 得 $x_1 > \frac{1}{3}$。
结合 $x_1 < 1$,可推得 $\frac{1}{x_1} \in (1, 3)$。
由于 $f(x)$ 在 $(1, 3)$ 上单调递减,该区间唯一零点为 $x_2$,且 $f\left(\frac{1}{x_1}\right) > 0 = f(x_2)$,故必有 $\frac{1}{x_1} < x_2$,即 $x_1x_2 > 1$。

由(1)知 $a > 0 \implies 3a > 2a$,欲证$f(x_1x_2) + f(x_1x_2x_3) > 3a$。
根据牛顿-莱布尼茨公式,将函数值转化为积分表示:
$$f(x_1x_2) = f(1) + \int_{1}^{x_1x_2} f'(t)dt = 3a + \int_{1}^{x_1x_2} f'(t)dt$$
$$f(x_1x_2x_3) = f(x_3) + \int_{x_3}^{x_1x_2x_3} f'(t)dt = \int_{x_3}^{x_1x_2x_3} f'(t)dt$$
两式相加得:
$$f(x_1x_2) + f(x_1x_2x_3) - 3a = \int_{1}^{x_1x_2} f'(t)dt + \int_{x_3}^{x_1x_2x_3} f'(t)dt$$
对第二个积分作变量代换 $t = x_3u$,则 $dt = x_3du$。当 $t$ 从 $x_3$ 变至 $x_1x_2x_3$ 时,$u$ 从 $1$ 变至 $x_1x_2$。于是:
$$\int_{x_3}^{x_1x_2x_3} f'(t)dt = \int_{1}^{x_1x_2} x_3f'(x_3u)du$$
代回原式可得:
$$f(x_1x_2) + f(x_1x_2x_3) - 3a = \int_{1}^{x_1x_2} \left[f'(u) + x_3f'(x_3u)\right]du$$
对被积函数进行计算化简:
\begin{align*}f'(u) + x_3f'(x_3u) &= \left(1 - \frac{4}{u} + \frac{3}{u^2}\right) + x_3\left(1 - \frac{4}{x_3u} + \frac{3}{x_3^2u^2}\right)\\
     &= (x_3 + 1) - \frac{8}{u} + \frac{3\left(1 + \frac{1}{x_3}\right)}{u^2} = \frac{(x_3 + 1)u^2 - 8u + 3\left(1 + \frac{1}{x_3}\right)}{u^2}\end{align*}
考察分子中的二次多项式 $N(u) = (x_3 + 1)u^2 - 8u + 3\left(1 + \frac{1}{x_3}\right)$,因 $x_3 \in (3, +\infty)$,故首项系数 $x_3 + 1 > 0$。
其判别式为:
$$\Delta = 64 - 12(x_3 + 1)\left(1 + \frac{1}{x_3}\right) = 64 - 12\left(x_3 + \frac{1}{x_3} + 2\right) = 40 - 12\left(x_3 + \frac{1}{x_3}\right)$$
因函数 $x + \frac{1}{x}$ 在 $x > 1$ 上严格递增,故 $x_3 + \frac{1}{x_3} > 3 + \frac{1}{3} = \frac{10}{3}$。
从而 $\Delta < 40 - 12 \times \frac{10}{3} = 0$,这表明对任意 $u \in \mathbb{R}$ 恒有 $N(u) > 0$。
因此,当 $u > 0$ 时,被积函数 $\frac{N(u)}{u^2} > 0$ 恒成立。
又已知 $x_1x_2 > 1$,积分区间 $[1, x_1x_2]$ 长度严格大于 $0$,故积分值严格大于 $0$:
$$\int_{1}^{x_1x_2} \left[f'(u) + x_3f'(x_3u)\right]du > 0$$
由此即证得 $f(x_1x_2) + f(x_1x_2x_3) > 3a $。
\end{solution}




\begin{example}{来自邪帝,无答案}{}
$f(x) = x^2 e^{a - x} - a^2 \ln x - a \left(a > \frac{1}{2}\right)$\\
$f(x_1) = f(x_2) = f(x_3) = 0 (x_1 < x_2 < x_3), \, f'(x_4) = f'(x_5) = 0 (x_4 < x_5)$\\
(1) 证明:$x_4 f(x_5) + x_5 f(x_4) < 0$;\\
(2) 证明:$(x_2 + x_3 + x_4) f(x_1 x_2 x_3) + (2x_1 + x_5) f(x_2 x_3) > 0$。
\end{example}


\chapter{杂题}
\section{数学问题}
\begin{example}{新曲线}{}
已知曲线 $\Gamma: 12x^3 + 12xy^2 + 3x^2 + 4y^2 = 0$。过原点 $O$ 作两条互相垂直的直线,分别交曲线 $\Gamma$ 于异于原点的两点 $A, B$。
求证:无论直线如何转动,$\triangle OAB$ 的外接圆恒过一个异于原点的定点,并求出该定点坐标。
\end{example}

\begin{example}{}{}
(多选)已知非零复数 $z_1, z_2$ 满足 $z\bar{z} = z + \bar{z}$。设复数 $z_3$ 满足$\frac{2}{z_3} = \frac{1}{z_1} + \frac{1}{z_2}$。下列关于 $z_1, z_2, z_3$ 的结论中,正确的有(~~~~~)\\
A. 复数 $z_3$ 必定满足方程 $z\bar{z} = z + \bar{z}$\\B. $\frac{|z_1 - z_3|}{|z_1|} = \frac{|z_2 - z_3|}{|z_2|}$\\C. $\frac{1}{|z_1|} + \frac{1}{|z_2|} \ge \frac{2}{|z_3|}$\\D. $\frac{1}{|z_1|^2} + \frac{1}{|z_2|^2} \le \frac{2}{|z_3|^2}$
\end{example}
\chapter{图论}
\section{图的基本概念}
\begin{definition}{有向图,无向图,平行边,邻接,环,孤立点,简单图}{}
1. 一个\textbf{无向图}可以被表示为$G = (V, E)$,其中$V$是一个非空集合(有限),称为\textbf{顶点集},其元素称为\textbf{顶点}或\textbf{节点},$E$是$V$中元素的无序对构成的集合,称为\textbf{边集},即$E \subseteq \{\{u,v\} \mid u,v \in V, u \neq v\}$\\
2. 一个\textbf{有向图}可以被表示为$G = (V, E)$,其中$V$是一个非空集合(有限),称为\textbf{顶点集},其元素称为\textbf{顶点}或\textbf{节点},$E$是$V$中元素的有序对构成的集合,称为\textbf{边集},即$E \subseteq \{(u,v) \mid u,v \in V, u \neq v\}$\\
3.  \textbf{平行边}:在无向图中,连接同一对顶点的多条边称为平行边,在有向图中,同一方向连接同一对顶点的多条弧称为平行边。\\
4. \textbf{邻接}:在无向图中,如果存在边$e = \{u,v\}$,则称顶点$u$和$v$是邻接的(相邻的)\\
5. 环:如果一条边的两个端点关联于同一个结点,称为环。\\
6. \textbf{孤立点}:在无向图中,如果一个顶点不与任何其他顶点相邻,则称它为孤立点。
\end{definition}
\begin{definition}{度的概念}{}
\textbf{度数}(无向图):顶点$v$的度数(简称度)$\deg(v)$是与该顶点相关联的边的数量,自环通常算作2度(因为连接同一顶点的两个端点)\\
\textbf{入度和出度}(有向图):顶点$v$的入度$\deg^-(v)$是以$v$为弧头的弧的数量,顶点$v$的出度$\deg^+(v)$是以$v$为弧尾的弧的数量,顶点$v$的总度数$\deg(v) = \deg^-(v) + \deg^+(v)$\\
\textbf{悬挂结点}:度数为1的顶点称为悬挂结点,在有向图中,通常指总度数为1的顶点\\
\textbf{悬挂边}:与悬挂结点相关联的边称为悬挂边。\\
\textbf{最大度和最小度}:图$G$的最大度$\Delta(G) = \max\{\deg(v) \mid v \in V(G)\}$,图$G$的最小度$\delta(G) = \min\{\deg(v) \mid v \in V(G)\}$\\
\textbf{最大入度和最小入度}(有向图):图$D$的最大入度$\Delta^-(D) = \max\{\deg^-(v) \mid v \in V(D)\}$,图$D$的最小入度$\delta^-(D) = \min\{\deg^-(v) \mid v \in V(D)\}$\\
\textbf{最大出度和最小出度}(有向图):图$D$的最大出度$\Delta^+(D) = \max\{\deg^+(v) \mid v \in V(D)\}$,图$D$的最小出度$\delta^+(D) = \min\{\deg^+(v) \mid v \in V(D)\}$
\end{definition}
\begin{definition}{图的基本分类}{}
\textbf{简单图}:在无向图中,如果不存在平行边,则称它为简单图。\\
\textbf{多重图}:如果一个图存在平行边,则称它为多重图。\\
\textbf{$n$阶图}:具有$n$个顶点的图称为$n$阶图。\\
\textbf{零图}:边集为空(没有边,只有结点)的图称为零图,即所有顶点都是孤立点。\\
\textbf{平凡图}:只有一个顶点且没有边的图称为平凡图,是最简单的非空图(一阶零图)。\\
\textbf{空图}:顶点集为空的图称为空图(通常不考虑)。\\
\textbf{完全图}:任意两不同顶点之间都恰有一条边的简单图称为完全图。$n$阶完全图记作$K_n$。\\
\textbf{二分图}:顶点集$V$可以划分为两个不相交子集$V_1$和$V_2$,使得每条边的一个端点在$V_1$中,另一个端点在$V_2$中,称为二分图。\\
\textbf{正则图}:所有顶点度数都相同的图称为正则图。若每个顶点的度数均为$k$,称为$k$-正则图。\\
\textbf{环图}:$n$个顶点$v_1,v_2,\ldots,v_n$依次连接形成的环状图称为环图,记作$C_n$。\\
\textbf{轮图}:在环图$C_{n-1}$中添加一个顶点,并将该顶点与环图中所有顶点相连,称为轮图$W_n$。\\
\textbf{$n$方体图}:用$n$维超立方体的顶点和边构成的图称为$n$方体图,记作$Q_n$。顶点集为所有长度为$n$的二进制串,两个顶点相邻当且仅当它们的二进制表示恰好有一位不同。
\end{definition}
\section{特殊图}
\chapter{树}

\end{document}