\chapter{谓词逻辑}
\section{谓词逻辑基本概念}
\begin{definition}{个体词~谓词~n元谓词~个体常元变元~谓词常项变项}{}
在谓词逻辑中,命题被分解为个体词和谓词两部分。\\
个体词是命题中表示具体或抽象对象的词,包括表示特定个体的个体常元(如$a, b, c$)和表示不确定个体的个体变元(如$x, y, z$)。\\
谓词是用来说明个体性质或个体间关系的词,包括表示特定性质或关系的谓词常项(如$P, Q, R$)和表示不确定性质或关系的谓词变项(如$F, G, H$)。n元谓词是涉及$n$个个体的谓词,表示$n$元关系,如一元谓词$P(x)$描述个体性质,二元谓词$R(x,y)$描述两个体间关系,n元谓词$F(x_1,x_2,\ldots,x_n)$描述n个个体的关系。
\end{definition}
\begin{definition}{谓词表达式,命题函数,个体域,论述域}{}
在谓词逻辑中,谓词表达式是由谓词和个体词组成的符号串,用于表示个体的性质或个体间的关系,例如$P(x)$、$R(a, b)$等。\\
命题函数是以个体变元(其取值范围就是个体域,也称作论述域)为自变量的函数,其取值是一个命题,当个体变元被特定个体常元替换时,命题函数转化为一个具体的命题。例如,$P(x)$是一个命题函数,当$x$取值为$a$时,$P(a)$成为一个命题。命题函数的真值取决于个体变元的取值和谓词的含义。
\end{definition}
从上面的定义可以看出,谓词表达式可以判断真值,命题函数不可以。个体域可以是无限或者有限的,没有特别说明时,个体变元的论述域指的是把整个宇宙中一切事物都作为对象的集合,所以当大题没有说个体域是什么时,最好不要想当然的自我设想个体域,比如说有的题目说“所以人都是学生”,那么最好还是设置命题函数“$M(x):x$是人”,或者声明个体域$x$是人,不要直接不管了。

