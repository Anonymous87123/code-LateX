\chapter{谓词逻辑}
\section{谓词逻辑基本概念}
\begin{definition}{个体词~谓词~n元谓词~个体常元变元~谓词常项变项}{}
在谓词逻辑中,命题被分解为个体词和谓词两部分。\\
个体词是命题中表示具体或抽象对象的词,包括表示特定个体的个体常元(如$a, b, c$)和表示不确定个体的个体变元(如$x, y, z$)。\\
谓词是用来说明个体性质或个体间关系的词,包括表示特定性质或关系的谓词常项(如$P, Q, R$)和表示不确定性质或关系的谓词变项(如$F, G, H$)。n元谓词是涉及$n$个个体的谓词,表示$n$元关系,如一元谓词$P(x)$描述个体性质,二元谓词$R(x,y)$描述两个体间关系,n元谓词$F(x_1,x_2,\ldots,x_n)$描述n个个体的关系。
\end{definition}
\begin{definition}{谓词表达式,命题函数,个体域,论述域}{}
在谓词逻辑中,谓词表达式是由谓词和个体词组成的符号串,用于表示个体的性质或个体间的关系,例如$P(x)$、$R(a, b)$等。\\
命题函数是以个体变元(其取值范围就是个体域,也称作论述域)为自变量的函数,其取值是一个命题,当个体变元被特定个体常元替换时,命题函数转化为一个具体的命题。例如,$P(x)$是一个命题函数,当$x$取值为$a$时,$P(a)$成为一个命题。命题函数的真值取决于个体变元的取值和谓词的含义。
\end{definition}
可以看出,谓词表达式可以判断真值,命题函数不可以。个体域可以是无限或者有限的,没有特别说明时,个体变元的论述域指的是把整个宇宙中一切事物都作为对象的集合,所以当大题没有说个体域是什么时,最好不要想当然的自我设想个体域,比如说有的题目说“所以人都是学生”,那么最好还是设置命题函数“$M(x):x$是人”,或者声明个体域$x$是人,不要直接不管了。
\begin{definition}{量词}{}
量词是用于表示个体变元在个体域中取值范围的逻辑符号。\\
\textbf{全称量词}($\forall$):表示"对所有"或"任意",如$\forall x P(x)$表示"对所有$x$,$P(x)$成立"。\\
\textbf{存在量词}($\exists$):表示"存在"或"至少有一个",如$\exists x P(x)$表示"存在$x$使得$P(x)$成立"。\\
除了这两种基本量词外,还有一些扩展的量词表示法:\\
\textbf{唯一存在量词}($\exists!$):表示"存在唯一的",如$\exists! x P(x)$表示"存在唯一的$x$使得$P(x)$成立"\\
\textbf{计数量词}:如$\exists_n x P(x)$表示"恰好有$n$个$x$满足$P(x)$"
\end{definition}
注意,在使用全称量词时,表示个体范围的特性谓词和表示个体性质的谓词构成条件关系式:$\forall x (P(x) \rightarrow Q(x))$;使用存在量词时,表示个体范围的特性谓词和表示个体性质的谓词构成合取关系式:$\exists x (P(x) \land Q(x))$。

考虑命题"所有大学生都年轻"。如果我们错误地用合取式:$\forall x (Student(x) \land Young(x))$这表示"每个个体都是大学生并且年轻"——要求宇宙中\textbf{所有个体}都必须是大学生!这显然过强了。正确的条件式:$\forall x (Student(x) \rightarrow Young(x))$只对"是大学生"的那些个体施加"年轻"的要求,对非大学生个体没有要求,这才是符合题意的。

考虑命题"有大学生聪明"。如果我们错误地用条件式:$\exists x (Student(x) \rightarrow Smart(x))$这个公式在逻辑上等价于$\exists x (\neg Student(x) \lor Smart(x))$,只要存在\textbf{任意一个非大学生},这个公式就为真,完全偏离原意。而$\exists x (Student(x) \land Smart(x))$表达了"存在一个个体既是大学生又聪明"的含义。

实际上,两种表达在特定条件下可以转换:$\forall x (P(x) \rightarrow Q(x)) \Leftrightarrow \neg \exists x (P(x) \land \neg Q(x))$
\begin{definition}{谓词演算的合式公式,简称谓词合式公式}{}
合式公式是谓词逻辑中形式推理和语义解释的基本单位,其递归定义如下:\\
1. 原子公式是合式公式:如果$P$是$n$元谓词,$t_1, t_2, \ldots, t_n$是项(个体常元、个体变元或函数),则$P(t_1, t_2, \ldots, t_n)$是合式公式\\
2. 逻辑联结词组合:如果$A$和$B$是合式公式,则$(\neg A)$、$(A \land B)$、$(A \lor B)$、$(A \rightarrow B)$、$(A \leftrightarrow B)$也是合式公式\\
3. 量词组合:如果$A$是合式公式,$x$是个体变元,则$(\forall x A)$和$(\exists x A)$也是合式公式\\
4. 只有有限次应用上述规则构成的表达式才是合式公式
\end{definition}
\begin{definition}{指导变元,辖域,约束变元,约束出现,自由变元,自由出现,闭式}{}
在谓词逻辑中,量词和变元的使用涉及以下重要概念:\\
\textbf{指导变元}:紧跟在量词后面的个体变元称为指导变元,如$\forall x$中的$x$和$\exists y$中的$y$。\\
\textbf{辖域}:量词所作用的公式范围称为该量词的辖域。如$\forall x (P(x) \rightarrow Q(x))$中,$(P(x) \rightarrow Q(x))$是$\forall x$的辖域。\\
\textbf{约束变元}:在量词辖域内出现且与该量词指导变元相同的个体变元称为约束变元。\\
\textbf{约束出现}:个体变元在公式中的某次出现如果处于某个量词的辖域内,且与该量词的指导变元相同,则称为约束出现。\\
\textbf{自由变元}:在公式中不被任何量词约束的个体变元称为自由变元。\\
\textbf{自由出现}:个体变元在公式中的某次出现如果不是约束出现,则称为自由出现。\\
\textbf{闭式}:不包含任何自由出现的个体变元的谓词公式称为闭式。闭式中所有个体变元都是约束出现的,这样的公式具有确定的真值。
\end{definition}
例如,在公式$\forall x (P(x,y) \rightarrow \exists z Q(x,z))$中:$x$和$z$是指导变元,$(P(x,y) \rightarrow \exists z Q(x,z))$是$\forall x$的辖域,$Q(x,z)$是$\exists z$的辖域,第一个$x$和第二个$x$是约束出现(被$\forall x$约束),$y$是自由出现(自由变元),$z$是约束出现(被$\exists z$约束),该公式不是闭式,因为包含自由变元$y$。

\begin{definition}{谓词公式的解释和分类}{}
谓词公式的解释是指为公式中的符号赋予具体含义的过程,包括以下四个组成部分:\\
1. \textbf{个体域} $D$:一个非空集合,规定个体变元的取值范围\\
2. \textbf{个体常元的指定}:为每个个体常元指定$D$中的一个特定元素\\
3. \textbf{函数符号的指定}:为每个$n$元函数符号指定$D^n$到$D$的映射\\
4. \textbf{谓词符号的指定}:为每个$n$元谓词符号指定$D^n$到$\{\text{真,假}\}$的映射\\
给定解释$I$和公式$A$,可以通过递归方式计算$A$在$I$下的真值。闭式在任意解释下的真值是确定的,而包含自由变元的公式真值取决于自由变元的取值。公式$A$是永真式当且仅当$A$在所有解释下为真;$A$是可满足式当且仅当存在解释使$A$为真;$A$是永假式当且仅当$A$在所有解释下为假。
\end{definition}
谓词公式判断类型是比较困难的,为了判断一些简单的情形,可以定义代换实例:
\begin{definition}{谓词公式的代换实例}{}
谓词公式的代换实例,是指通过将命题逻辑中的命题公式中的命题变元替换为谓词公式而得到的谓词公式。具体来说,设$A$是一个命题公式,其中包含命题变元$P_1, P_2, \ldots, P_n$。如果用一个谓词公式$B_i$($i = 1, 2, \ldots, n$)替换$A$中的每个命题变元$P_i$,则得到的新公式$A'$称为$A$的一个代换实例。\\
代换实例保持了原命题公式的逻辑结构,只是将原子命题替换为谓词公式。如果原命题公式是永真式(矛盾式),则其所有代换实例也是永真式(矛盾式)。
\end{definition}
例如,命题公式$P \rightarrow (Q \land P)$的一个代换实例可以是:$\forall x P(x) \rightarrow (\exists y Q(y) \land \forall x P(x))$。而在代换实例的视角来看,有些谓词公式可以通过像命题公式那样进行等价变换,从而证明一些命题公式之间有等价关系或者蕴含关系。这就需要用到之前的推理定律。详见$2.3$
\section{换名,前束范式}
换名的动机是让谓词公式中尽量不要出现同样的变元,以便于区分和进行逻辑推理。

约束变元换名规则允许在公式中更改约束变元的名称,而不改变公式的逻辑含义。比如将 $\forall x (P(x) \rightarrow \exists y Q(x,y))$ 中的 $x$ 换为 $z$,得到 $\forall z (P(z) \rightarrow \exists y Q(z,y))$,要求如下:\\
1. 只能更改约束变元的名称,不能更改自由变元;\\
2. 换名必须在整个量词的辖域内一致进行\\
3. 新变元名称不能与公式中已有的自由变元同名\\
4. 多个量词约束的同名变元必须同时换名

自由变元代入规则允许用项替换公式中的自由变元。例如将 $\forall x P(x,y)$ 中的自由变元 $y$ 用常元 $a$ 代入,得到 $\forall x P(x,a)$,要求如下:\\
1. 代入必须对自由变元的所有自由出现同时进行\\
2. 代入项中不能含有在公式中受约束的变元\\
3. 代入后原公式的逻辑含义保持不变

注意,约束变元换名,改变的是变元符号,不改变变元的性质(仍是约束变元),自由变元代入是用具体的项(常元、函数或其他变元)替换自由变元换名规则保持公式的逻辑等价性。

举一个例子,比如谓词公式:$\forall x (P(x,y) \rightarrow \exists y Q(x,y))$,这个公式存在两个问题:一是自由变元$y$与约束变元$y$同名,容易引起混淆,二是量词$\forall x$和$\exists y$的约束变元$x$和$y$在公式中混合使用,不利于逻辑推理。下面换名:

\textbf{第一步:自由变元代入}:将自由变元$y$用常元$a$代入:$\forall x (P(x,a) \rightarrow \exists y Q(x,y))$

\textbf{第二步:约束变元换名}:将存在量词$\exists y$的约束变元$y$换名为$z$:$\forall x (P(x,a) \rightarrow \exists z Q(x,z))$。经过两个规则的应用,我们得到的新公式:自由变元$y$已被具体化为常元$a$,约束变元$y$被换名为$z$,避免了名称冲突,现在公式中每个变元的角色清晰:$x$是$\forall$的约束变元,$z$是$\exists$的约束变元,$a$是常元,新公式与原公式逻辑等价,但更便于进行逻辑推理。

常常利用换名将谓词公式化为与其自身等价的规范形式,成为前束范式:
\begin{definition}{前束范式,首标,母式}{}
前束范式是谓词逻辑中的一种标准形式,其中所有量词都出现在公式的最前端,且其辖域延伸到公式的末尾。形式化地,一个公式是前束范式当且仅当它具有以下形式:\vspace{-10pt}
$$Q_1x_1Q_2x_2\cdots Q_nx_nM,\text{其中}Q_1x_1Q_2x_2\text{为首标},M\text{为母式}\vspace{-10pt}$$
其中:$Q_i$ ($i=1,2,\ldots,n$) 是量词符号($\forall$ 或 $\exists$),$x_i$ ($i=1,2,\ldots,n$) 是个体变元。
\end{definition}
将任意谓词公式化为前束范式的步骤通常包括:消去蕴含联结词 $\rightarrow$ 和等价联结词 $\leftrightarrow$,将否定符号 $\neg$ 内移,使其只作用于原子公式,将量词向左移动,并适当换名避免变元冲突。

例如谓词公式$\forall x\forall y(\exists z(P(x,z)\land P(y,z))\to\exists uQ(x,y,u))$化为前束范式的过程是:
\begin{align*}
&\forall x\forall y(\exists z(P(x,z)\land P(y,z))\to\exists uQ(x,y,u))\\
&\Leftrightarrow \forall x\forall y(\neg\exists z(P(x,z)\land P(y,z))\lor\exists uQ(x,y,u))&\text{置换规则}\\
&\Leftrightarrow \forall x\forall y(\forall z(\neg P(x,z)\lor\neg P(y,z))\lor\exists uQ(x,y,u))&\text{置换规则}\\
&\Leftrightarrow \forall x\forall y\forall z\exists u(\neg P(x,z)\lor\neg P(y,z)\lor Q(x,y,u))&\text{量词辖域扩张}
\end{align*}
由于上面的$x,y$的辖域覆盖到公式末尾,而且$z$和$u$的虽然辖域不同,但名称不同,无需换名,下面再看一个需要换名的例子。
求下列谓词公式的前束范式:$\neg\exists xA(x)\to\forall yB(x,y)$
\begin{align*}
    &\neg\exists xA(x)\to\forall yB(x,y)\\
    &\Leftrightarrow \forall x\neg A(x)\to\forall yB(x,y)&\text{否定等价转换}\\
    &\Leftrightarrow \forall z\neg A(z)\to\forall yB(x,y)&\text{换名}\\
    &\Leftrightarrow \forall z\forall y(\neg A(z)\to B(x,y))&\text{量词辖域扩张}\\
    &\Leftrightarrow \forall z\forall y(A(z)\lor B(x,y))&\text{置换规则}
\end{align*}
\section{谓词公式的等价和蕴含关系式}
与上一章类似,当两个谓词公式$A$和$B$满足$A\leftrightarrow B$为永真式时,我们说$A$和$B$是等价的,等价关系式的形式为:$A\Leftrightarrow B$;当两个谓词公式$A$和$B$满足$A\rightarrow B$为永真式时,我们说$A$蕴含$B$,蕴含关系式的形式为:$A\Rightarrow B$。
\begin{theorem}{常见的等价关系式}{}
1. 去括号($B$含$x$):$\exists x(A(x)\to B(x))\Leftrightarrow \forall xA(x)\to \exists xB(x)$\\
2. 去括号($B$含$x$):$\forall y(A(y)\to B(x))\Leftrightarrow \exists xA(x)\to B(x)$\\
3. 去括号($B$不含$x$):$\forall x(\exists x)~(B(\land,\lor,\to)A(x))\Leftrightarrow \forall x(\exists x)~B(\land,\lor,\to)A(x) $\\
4. 去括号($B$不含$x$):$\forall x(A(x)\to B)\Leftrightarrow \exists xA(x)\to B\quad  \exists x(A(x)\to B)\Leftrightarrow \forall xA(x)\to B$\\
5. 分配律:$\forall x(A(x)\land B(x))\Leftrightarrow \forall xA(x)\land \forall xB(x)\quad \exists x(A(x)\lor B(x))\Leftrightarrow \exists xA(x)\lor \exists xB(x)$\\
6. 交换律:$\forall x\forall y A(x,y)\Leftrightarrow \forall y\forall x A(x,y)\quad \exists x\exists y A(x,y)\Leftrightarrow \exists y\exists x A(x,y)$
\end{theorem}
3和4即书上给出的量词辖域扩张与收缩律,2虽然说$B$含$x$,但是由于此时蕴含前件不含$x$,而是含有$y$,且指导变元是$y$,所以此时和4一样。上面的公式利用等价变形都可以解决,下面给出理解和记忆的方式。

1. 左边说“至少有一个$x$,能够确保:如果它满足$A$,那么它也满足$B$”。右边说“如果所有的$x$都满足$A$,那么至少有一个$x$满足$B$”。例如:“假设你是老师,你承诺说你会找一些幸运儿学生,如果他们来上课就能及格”,这等价于“如果所有学生都来上课,那么有学生能及格”,因为你总要挑选学生给到及格。

2. 需要注意在这个公式中,$B(x)$中的变元是自由变元(没有被任何量词约束),因此两边都表示关于某个特定个体$x$的陈述。左边说"对任何一个$y$,如果$y$具有性质$A$,那么$x$具有性质$B$";右边说"如果存在某个个体具有性质$A$,那么$x$具有性质$B$"。例如,"如果任何人都是医生,那么张三健康"等价于"如果有医生存在,那么张三健康"。由于$x$是自由变元,两边讨论的都是同一个特定的$x$(这里也可以视作存在量词引入),$B(x)$描述的是这个特定个体的性质,不受量词影响。

3. 当B是与x无关的事实陈述时,可以把B从量词中“提出来”。例如,“对所有学生,今天下雨且他们要考试”等价于“今天下雨且所有学生都要考试”。

4. 左边说“每个A都导致B”;右边说“只要有A存在就会导致B”。例如,“如果每个学生都作弊,就取消考试”等价于“只要有学生作弊,就取消考试”。

5. “每个人都既聪明又勤奋”等价于“每个人都聪明且每个人都勤奋”。“有人会唱歌或跳舞”等价于“有人会唱歌或者有人会跳舞”。

6. “任何两个人都是朋友”等价于“对任何人来说,与任何人都是朋友”,量词顺序不影响含义。“存在两个人是朋友”等价于“存在这样的两个人,他们是朋友”,存在量词的顺序也不影响含义。
\begin{theorem}{常见的蕴含关系式(结论的弱化)}{}
1. $\exists xA(x)\to\exists xB(x)\Rightarrow\exists x(A(x)\to B(x))$\\
2. $\forall xA(x)\to \exists xB(x)\Rightarrow \exists x(A(x)\to B(x))$\\
3. $\forall xA(x)\lor \forall xB(x)\Rightarrow \forall x(A(x)\lor B(x))\Rightarrow \neg\forall xA(x)\to\exists xB(x)$\\
4. $\exists x(A(x)\land B(x))\Rightarrow \exists xA(x)\land \exists xB(x)$\\
5. $\forall x(A(x)\to B(x))\Rightarrow \forall xA(x)\to \forall xB(x)$\\
6. $\exists x(A(x)\to B(x))\Rightarrow \exists xA(x)\to \exists xB(x)$\\
7. $\forall x\forall yA(x,y)\Rightarrow \forall xA(x,x)$\\
8. $\exists xA(x,x)\Rightarrow \exists x\exists yA(x,y)$\\
9. $\forall x\forall yA(x,y)\Rightarrow \exists y\forall xA(x,y)\Rightarrow \forall x\exists yA(x,y)\Rightarrow \exists x\exists yA(x,y)$
\end{theorem}
1. 如果存在某个$x$满足$A(x)$,那么存在某个$x$满足$B(x)$。注意,这里的两个$x$可能不是同一个个体,也可能是同一个个体,当为后一种情况,就得到了蕴含后件。所以蕴含关系 $\exists xA(x)\to\exists xB(x)\Rightarrow\exists x(A(x)\to B(x))$ 成立。

2. 如果所有$x$满足$A(x)$,那么存在某个$x$满足$B(x)$。单独考虑这个对$B(x)$量身定制的$x$也会满足$A(x)$,此时$A(x)\to B(x)$成立,蕴含关系 $\forall xA(x)\to \exists xB(x)\Rightarrow \exists x(A(x)\to B(x))$ 成立。

3. 左边是一个强条件:要么所有个体都满足$A$,要么所有个体都满足$B$。右边是一个弱条件:每个个体至少满足$A$或$B$之一。显然,如果左边成立,那么右边必须成立,因为如果所有个体都满足$A$,那么每个个体都满足$A$,所以满足$A$或$B$;类似如果所有个体都满足$B$,也一样。但反过来不一定成立:可能有些个体满足$A$但不满足$B$,有些满足$B$但不满足$A$,所以右边真但左边假,因此蕴含关系 $\forall xA(x)\lor \forall xB(x)\Rightarrow \forall x(A(x)\lor B(x))$ 成立。而另一个蕴含关系式作为例题安排在后面了。

4. 如果存在一个个体同时满足$A$和$B$,那么当然存在满足$A$的个体(就是这个个体)和存在满足$B$的个体(也是这个个体),但是右边的两个$x$可能一样也可能不一样,蕴含关系 $\exists x(A(x)\land B(x))\Rightarrow \exists xA(x)\land \exists xB(x)$ 显然成立。

5. 每一个满足$A$的$x$都同时满足$B$,那当$x$恒满足$A$时就会恒满足$B$,但当存在不满足$A$的个体时,前提不施加任何约束,因此可能存在某些个体满足$A$但不满足$B$,只要不是所有个体都满足$A$,前提仍然为真。蕴含关系 $\forall x(A(x)\to B(x))\Rightarrow \forall xA(x)\to \forall xB(x)$ 显然成立。

6. 对于某特定的$x$,如果$A(x)$成立,那么$B(x)$也成立。但是右边的两个$x$可能一样也可能不一样,蕴含关系 $\exists x(A(x)\to B(x))\Rightarrow \exists xA(x)\to \exists xB(x)$ 显然成立。

7. 左边说$A$关系对所有个体对都成立。右边说$A$关系对所有自对(即个体与自身)都成立。如果左边成立,特别地,当$y=x$时,$A(x,x)$成立,因此蕴含关系 $\forall x\forall yA(x,y)\Rightarrow \forall xA(x,x)$ 成立。

8. 如果存在一个个体与自身满足$A$,那么取个体$x$和$y$(取$y=x$)满足$A$。所以左边蕴含右边,即 $\exists xA(x,x)\Rightarrow \exists x\exists yA(x,y)$ 。但可能$A(x,y)$成立对于$x$和$y$不同,$A(x,x)$不成立。

9. 从“全局$A$”到“存在万能$y$”(万能钥匙)到“每个$x$有对应$y$”(专用钥匙)到“存在一对”(碰运气),强度递减,蕴含关系成立。

上面的叙述其实已经用到了四大谓词逻辑推理规则:
\begin{theorem}{四大谓词逻辑推理规则}{}
1. 全称实例化 (Universal Instantiation, UI):从全称命题推出特定实例。\vspace{-10pt}
$$\forall x P(x) \Rightarrow P(c),\text{其中}c\text{是个体域中的任意个体}\vspace{-10pt}$$
使用要求与注意事项:$c$必须属于个体域中的个体,且$c$不能在$A(x)$中约束出现,可以对同一个全称命题进行多次实例化,得到不同的实例,实例化时保持变元的一致性。例如,从“所有人都会死”推出“苏格拉底会死”。\\
2. 存在实例化 (Existential Instantiation, EI)从存在命题推出特定实例。\vspace{-10pt}
$$\exists x P(x) \Rightarrow P(c),\text{其中}c\text{是个体域中的某个特定个体}\vspace{-10pt}$$
使用要求与注意事项:除了$x$以外,$A(x)$不能出现其它自由出现的个体变元,同时$c$必须是新引入的常元,不能和公式或推理证明过程中的其它常元或变元相同;$c$不能出现在结论中(因为不知道具体是哪个个体);只能对同一个存在量词使用一次EI。例如,从"有人迟到了"推出"令小王是那个迟到的人"\\
3. 全称泛化 (Universal Generalization, UG)从特定实例推出全称命题。\vspace{-10pt}
$$P(c) \Rightarrow \forall x P(x),\text{其中}c\text{是个体域中的任意个体}\vspace{-10pt}$$
使用要求与注意事项:$c$必须是自由出现的个体变元,不能有特殊性质,且$c$不能在$A(x)$中约束出现,必须确保对个体域中所有个体都成立。示例:通过证明任意一个三角形的内角和为180度,推出所有三角形的内角和为180度。\\
4. 存在泛化 (Existential Generalization, EG):从特定实例推出存在命题。\vspace{-10pt}
$$P(c) \Rightarrow \exists x P(x),\text{其中}c\text{是个体域中的某个特定个体}\vspace{-10pt}$$
使用要求与注意事项:$c$可以是任意已知个体,但是取代$c$的$x$不可以在$A(c)$中出现过,即不能和已有的量词绑定。示例:从“苏格拉底是哲学家”推出“存在哲学家”。\\
5. \textbf{易错点:在使用EI引入新常元后,不能再对该常元使用UG}
\end{theorem}
例题:证明$\forall x(A(x)\lor B(x))\Rightarrow \neg\forall xA(x)\to\exists xB(x)$
\begin{align*}
(1)&\quad \forall x(A(x)\lor B(x)) & \text{前提} \\
(2)&\quad \neg\forall xA(x) & \text{附加前提} \\
(3)&\quad \exists x\neg A(x) & \text{由(2)量词否定转换} \\
(4)&\quad \neg A(c) & \text{由(3)存在实例化(EI),c为新常量} \\
(5)&\quad A(c)\lor B(c) & \text{由(1)全称实例化(UI)} \\
(6)&\quad B(c) & \text{由(4)(5)析取三段论} \\
(7)&\quad \exists xB(x) & \text{由(6)存在泛化(EG)} \\
(8)&\quad \neg\forall xA(x)\to\exists xB(x) & \text{由(2)(7)条件证明(CP)}
\end{align*}
