\chapter{函数}
\section{自然数}
\begin{definition}{后继集与自然数集的定义}{}
\textbf{后继集}:设 $A$ 是一个集合,$A$ 的\textbf{后继集}定义为 $A^{+} = A \cup \{A\}$。\\
\textbf{自然数集的递归定义}(冯·诺依曼定义):

1. \textbf{基础}:$0 = \varnothing$

2. \textbf{递归步骤}:$n + 1 = n^{+} = n \cup \{n\}$

3.\textbf{极限情况}:自然数集 $\mathbb{N}$ 是包含 $0$ 且在后继运算下封闭的最小集合\\
按照这个定义,自然数可以具体构造为:\vspace{-10pt}
\begin{align*}
0 &= \varnothing \\
1 &= 0^{+} = \varnothing \cup \{\varnothing\} = \{\varnothing\} \\
2 &= 1^{+} = \{\varnothing\} \cup \{\{\varnothing\}\} = \{\varnothing, \{\varnothing\}\} \\
3 &= 2^{+} = \{\varnothing, \{\varnothing\}\} \cup \{\{\varnothing, \{\varnothing\}\}\} = \{\varnothing, \{\varnothing\}, \{\varnothing, \{\varnothing\}\}\} \\
&\vdots
\vspace{-10pt}\end{align*}
\textbf{性质}:每个自然数都是它前面所有自然数的集合;$n = \{0, 1, 2, \ldots, n-1\}$;这种定义方式使得 $m < n$ 当且仅当 $m \in n$;自然数集 $\mathbb{N}$ 是一个归纳集
\end{definition}