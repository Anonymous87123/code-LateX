\chapter{集合与关系}
\section{集合}
\begin{definition}{空集,全集,有限集,无限集,元素个数}{}
\textbf{空集}:不含任何元素的集合,记作$\nothing$或$\{x|x\ne x\}$。空集是任何集合的子集。空集是唯一的\\
\textbf{全集}:特定讨论中包含所有对象的集合,记作$U$。全集是相对的,取决于讨论的上下文。\\
\textbf{有限集}:元素个数有限的集合。如果集合$A$有$n$个元素,其中$n$是非负整数,则$A$是有限集。\\
\textbf{无限集}:不是有限集的集合,即元素个数无限多的集合。如自然数集$\mathbb{N}$、实数集$\mathbb{R}$等。\\
\textbf{元素个数}:有限集$A$中元素的数目,记作$|A|$或$card(A)$。无限集的元素个数用基数概念描述,如可数无限集、不可数无限集等。
\end{definition}
\begin{definition}{集合的包含关系}{}
    \textbf{子集}:如果集合$A$中的每个元素都是集合$B$中的元素,则称$A$是$B$的\textbf{子集},记作$A \subseteq B$。空集是空集的子集\\
    \textbf{真子集}:如果$A \subseteq B$且$A \neq B$,则称$A$是$B$的\textbf{真子集},记作$A \subset B$。即\vspace{-10pt}
    \[\forall x(A(x) \rightarrow B(x)) \land \exists x(B(x) \land \neg A(x))\]
\end{definition}
\begin{definition}{集合的基本运算}{}
\textbf{交集}:集合$A$和$B$的交集是由所有既属于$A$又属于$B$的元素组成的集合,记作$A \cap B$。\vspace{-10pt}
$$A \cap B = \{x \mid x \in A \land x \in B\}\vspace{-10pt}$$
\textbf{并集}:集合$A$和$B$的并集是由所有属于$A$或属于$B$的元素组成的集合,记作$A \cup B$。\vspace{-10pt}
$$A \cup B = \{x \mid x \in A \lor x \in B\}\vspace{-10pt}$$
\textbf{补集}:集合$A$的补集是由全集中所有不属于$A$的元素组成的集合,记作$\sim  A$。\vspace{-10pt}
$$\sim A = \{x \in U \mid x \notin A\}\vspace{-10pt}$$
\textbf{差集}:集合$A$与$B$的差集是由所有属于$A$但不属于$B$的元素组成的集合:\vspace{-10pt}
$$A-B = \{x \mid x \in A \land x \notin B\}\vspace{-10pt}$$
\textbf{对称差}:$A$和$B$的对称差是由所有属于$A$或属于$B$但不同时属于两者的元素组成的集合\vspace{-10pt}
$$A\oplus B = (A - B) \cup (B -A) = (A \cup B)- (A \cap B)\vspace{-10pt}$$
\textbf{幂集}:集合$A$的幂集是由$A$的所有子集构成的集合,记作$P(A)$:\vspace{-10pt}
$$P(A) = \{X \mid X \subseteq A\}$$
\end{definition}
\begin{theorem}{容斥原理}{}
对于有限集合$A_1, A_2, \ldots, A_n$,有:
\begin{align*}
\left|\bigcup_{i=1}^n A_i\right| &= \sum_{i=1}^n |A_i| 
                               - \sum_{1 \leq i < j \leq n} |A_i \cap A_j| + \sum_{1 \leq i < j < k \leq n} |A_i \cap A_j \cap A_k| \\
                               &- \cdots + (-1)^{n-1} |A_1 \cap A_2 \cap \cdots \cap A_n|
\end{align*}
\end{theorem}
然后主要写一些差集,对称差,幂集的性质。
\begin{theorem}{差集的性质}{}
分配律:$A - (B \cup C) = (A - B) \cap (A - C)=(A - B) - C$\quad$A - (B \cap C) = (A - B) \cup (A - C)$\\
分配律:$(A \cup B) - C = (A - C) \cup (B - C)$\quad$(A \cap B) - C = (A - C) \cap (B - C)$\\
去括号:$A - (B - C) = (A - B) \cup (A \cap C)$\\
传递性:如果$A \subseteq B$,则$A - C \subseteq B - C$
\end{theorem}
第二条恒等变形就可以了比较简单,第四条也好理解,$x\in A\land x\notin C$显然蕴含$x\in B\land x\notin C$

第一条和第三条有点难,一个很霸道的方式是,对于全集$U$内的元素$x$,考虑真值表
\begin{center}
\begin{tabular}{ccc|cccccc}
$x \in A$ & $x \in B$ & $x \in C$ & $x \in (B \cup C)$ & $x \in (B - C)$ & $x \in A - B$ & $x \in A - C$ & $x \in A \cap C$ \\
\hline
0 & 0 & 0 & 0 & 0 & 0 & 0 & 0 \\
0 & 0 & 1 & 1 & 0 & 0 & 0 & 0 \\
0 & 1 & 0 & 1 & 1 & 0 & 0 & 0 \\
0 & 1 & 1 & 1 & 0 & 0 & 0 & 0 \\
1 & 0 & 0 & 0 & 0 & 1 & 1 & 0 \\
1 & 0 & 1 & 1 & 0 & 1 & 0 & 1 \\
1 & 1 & 0 & 1 & 1 & 0 & 1 & 0 \\
1 & 1 & 1 & 1 & 0 & 0 & 0 & 1 \\
\end{tabular}
\end{center}
然后,目标表达式比较
\begin{center}
\begin{tabular}{ccc|ccc}
$x \in A$ & $x \in B$ & $x \in C$ & $x \in A - (B \cup C)$ & $x \in (A - B) \cap (A - C)$ & $x \in (A - B) - C$ \\
\hline
0 & 0 & 0 & 0 & 0 & 0 \\
0 & 0 & 1 & 0 & 0 & 0 \\
0 & 1 & 0 & 0 & 0 & 0 \\
0 & 1 & 1 & 0 & 0 & 0 \\
1 & 0 & 0 & 1 & 1 & 1 \\
1 & 0 & 1 & 0 & 0 & 0 \\
1 & 1 & 0 & 0 & 0 & 0 \\
1 & 1 & 1 & 0 & 0 & 0 \\
\end{tabular}
\end{center}
从表2可以看出第4,5,6列的真值完全相同,因此:$A - (B \cup C) = (A - B) \cap (A - C) = (A - B) - C$
\begin{center}
\begin{tabular}{ccc|cc}
$x \in A$ & $x \in B$ & $x \in C$ & $x \in A - (B - C)$ & $x \in (A - B) \cup (A \cap C)$ \\
\hline
0 & 0 & 0 & 0 & 0 \\
0 & 0 & 1 & 0 & 0 \\
0 & 1 & 0 & 0 & 0 \\
0 & 1 & 1 & 0 & 0 \\
1 & 0 & 0 & 1 & 1 \\
1 & 0 & 1 & 1 & 1 \\
1 & 1 & 0 & 0 & 0 \\
1 & 1 & 1 & 1 & 1 \\
\end{tabular}
\end{center}
从表3可以看出,第4列和第5列的真值完全相同,因此$A - (B - C) = (A - B) \cup (A \cap C)$
\begin{theorem}{对称差的性质}{}
    结合律:$(A \oplus B) \oplus C = A \oplus (B \oplus C)$\\
    分配律:$A \cap (B \oplus C) = (A \cap B) \oplus (A \cap C)$\\
    差形式:$A \oplus B = (A \cup B) - (A \cap B)$\\
    取补集:$\sim(A \oplus B) = (\sim A \oplus B) = (A \oplus \sim B)$\\
    常见等价关系:$A \oplus B = \varnothing$ 当且仅当 $A = B$,$A \oplus B = A \oplus C$ 当且仅当 $B = C$\\
    消去律:$(A \oplus B) \oplus (B \oplus C) = A \oplus C$(可以推广)
\end{theorem}
证明:$(A \oplus B) \oplus C = A \oplus (B \oplus C)$
\begin{center}
\begin{tabular}{ccc|cccccc}
$A$ & $B$ & $C$ & $A \oplus B$ & $(A \oplus B) \oplus C$ & $B \oplus C$ & $A \oplus (B \oplus C)$ & 是否相等 \\
\hline
0 & 0 & 0 & 0 & 0 & 0 & 0 & 是 \\
0 & 0 & 1 & 0 & 1 & 1 & 1 & 是 \\
0 & 1 & 0 & 1 & 1 & 1 & 1 & 是 \\
0 & 1 & 1 & 1 & 0 & 0 & 0 & 是 \\
1 & 0 & 0 & 1 & 1 & 0 & 1 & 是 \\
1 & 0 & 1 & 1 & 0 & 1 & 0 & 是 \\
1 & 1 & 0 & 0 & 0 & 1 & 0 & 是 \\
1 & 1 & 1 & 0 & 1 & 0 & 1 & 是 \\
\end{tabular}
\end{center}
如果要求必须使用等价变形,我们依然可以使用真值表写出两者的主析取范式,然后比较。两者的的主析取范式完全相同,因此:$(A \oplus B) \oplus C = A \oplus (B \oplus C)$,具体的解答过程则是先将左右两边化为交并补运算,真值表可以起到检查是否正确的作用。
\begin{align*}
(A \oplus B) \oplus C = & (\sim A \cap \sim B \cap C) \cup 
                        (\sim A \cap B \cap \sim C) \cup (A \cap \sim B \cap \sim C) \cup 
                        (A \cap B \cap C)\\
A \oplus (B \oplus C) = & (\sim A \cap \sim B \cap C) \cup 
                         (\sim A \cap B \cap \sim C) \cup 
                         (A \cap \sim B \cap \sim C) \cup 
                         (A \cap B \cap C)
\end{align*}
证明:$A \cap (B \oplus C) = (A \cap B) \oplus (A \cap C)$
\begin{center}
\begin{tabular}{ccc|cccccc}
$A$ & $B$ & $C$ & $B \oplus C$ & $A \cap (B \oplus C)$ & $A \cap B$ & $A \cap C$ & $(A \cap B) \oplus (A \cap C)$ & 相等 \\
\hline
0 & 0 & 0 & 0 & 0 & 0 & 0 & 0 & 是 \\
0 & 0 & 1 & 1 & 0 & 0 & 0 & 0 & 是 \\
0 & 1 & 0 & 1 & 0 & 0 & 0 & 0 & 是 \\
0 & 1 & 1 & 0 & 0 & 0 & 0 & 0 & 是 \\
1 & 0 & 0 & 0 & 0 & 0 & 0 & 0 & 是 \\
1 & 0 & 1 & 1 & 1 & 0 & 1 & 1 & 是 \\
1 & 1 & 0 & 1 & 1 & 1 & 0 & 1 & 是 \\
1 & 1 & 1 & 0 & 0 & 1 & 1 & 0 & 是 \\
\end{tabular}
\end{center}
证明:$(A\cup B)\oplus(A\cup C)\subseteq A\cup(B\oplus C)$
\begin{center}
\begin{tabular}{ccc|cccccc}
$A$ & $B$ & $C$ & $B \oplus C$ & $A \cup (B \oplus C)$ & $A \cup B$ & $A \cup C$ & $(A \cup B) \oplus (A \cup C)$ & 蕴含关系 \\
\hline
0 & 0 & 0 & 0 & 0 & 0 & 0 & 0 & 相等 \\
0 & 0 & 1 & 1 & 1 & 0 & 1 & 1 & 相等 \\
0 & 1 & 0 & 1 & 1 & 1 & 0 & 1 & 相等 \\
0 & 1 & 1 & 0 & 0 & 1 & 1 & 0 & 相等 \\
1 & 0 & 0 & 0 & 1 & 1 & 1 & 0 & 左真右假 \\
1 & 0 & 1 & 1 & 1 & 1 & 1 & 0 & 左真右假 \\
1 & 1 & 0 & 1 & 1 & 1 & 1 & 0 & 左真右假 \\
1 & 1 & 1 & 0 & 1 & 1 & 1 & 0 & 左真右假 \\
\end{tabular}
\end{center}
可以发现,当全集内的某个元素$x$满足$x\in (A \cup B) \oplus (A \cup C)$时,$x$必然满足$x\in A\cup(B\oplus C)$\\
证明:$(A \oplus B) \oplus (B \oplus C) = A \oplus C$
\begin{center}
\begin{tabular}{ccc|cccccc}
$A$ & $B$ & $C$ & $A \oplus B$ & $B \oplus C$ & $(A \oplus B) \oplus (B \oplus C)$ & $A \oplus C$ & 是否相等 \\
\hline
0 & 0 & 0 & 0 & 0 & 0 & 0 & 是 \\
0 & 0 & 1 & 0 & 1 & 1 & 1 & 是 \\
0 & 1 & 0 & 1 & 1 & 0 & 0 & 是 \\
0 & 1 & 1 & 1 & 0 & 1 & 1 & 是 \\
1 & 0 & 0 & 1 & 0 & 1 & 1 & 是 \\
1 & 0 & 1 & 1 & 1 & 0 & 0 & 是 \\
1 & 1 & 0 & 0 & 1 & 1 & 1 & 是 \\
1 & 1 & 1 & 0 & 0 & 0 & 0 & 是 \\
\end{tabular}
\end{center}
然而,此题用真值表,是杀鸡用牛刀,其实最简单的方法是利用结合律:
\begin{align*}
(A_1 \oplus A_2) \oplus (A_2 \oplus A_3) = A_1 \oplus (A_2 \oplus A_2) \oplus A_3 = A_1 \oplus \varnothing \oplus A_3  = A_1 \oplus A_3 
\end{align*}
\begin{theorem}{对称差的代数结构}{}
    对于任意集合序列 $A_1, A_2, \ldots, A_n$,有\vspace{-10pt}
$$(A_1 \oplus A_2) \oplus (A_2 \oplus A_3) \oplus \cdots \oplus (A_{n-1} \oplus A_n) = A_1 \oplus A_n$$
\end{theorem}
对于任意形式的对称差链,只要每个中间项出现偶数次,就会相互抵消。例如:
$$(A \oplus B \oplus C) \oplus (B \oplus C \oplus D) = A \oplus D$$
因为通过结合律和交换律,$B$ 和 $C$ 实际上各出现两次,因此抵消。\\
证明:$\sim(A \oplus B) = (\sim A \oplus B)$
\begin{align*}
      \sim(A \oplus B) &= \sim((A\cap \sim B) \cup (\sim A \cap B))=(\sim A\cup B)\cap (A\cup \sim B) \\
      &=(\sim A\cap A)\cup(\sim A\cap\sim B)\cup(A\cap B)\cup(B\cap\sim B)\\
      &=(\sim A\cap\sim B)\cup(A\cap B)= (\sim A \oplus B)
\end{align*}
\begin{theorem}{幂集的性质}{}
设$X$是一个集合,$P(X)$表示$X$的幂集(即$X$的所有子集构成的集合)。幂集具有以下性质:
1. 如果$|X| = n$(有限),则$|P(X)| = 2^n$,$P(\varnothing) = \{\varnothing\}$,有$1 = 2^0$个元素\\
2. 单调性:如果$A \subseteq B$,则$P(A) \subseteq P(B)$\\
3. $P(A) \cup P(B) \subseteq P(A \cup B)$,当且仅当 $A \subseteq B$ 或 $B \subseteq A$取等\\
4. $P(A) \cap P(B) = P(A \cap B)$\\
5. $ P(A - B) \subseteq (P(A) - P(B))\cup\{\varnothing\}$
\end{theorem}
性质3的证明比较简单:\vspace{-10pt}
\[P(A) \cup P(B) \subseteq P(A \cup B)\cup P(A\cup B)=P(A\cup B)\vspace{-10pt}\]
分别放缩左右两侧即可,而且这样的操作还有一个好处,就是可以看清蕴含关系变成等价关系时的特殊条件(当且仅当 $A \subseteq B$ 或 $B \subseteq A$取等)。我们不妨再看一个例子:

证明或证伪:$A\subset B\land C\subset D\Rightarrow (A\cup C)\subset (B\cup D)$\\
解:考虑$B=A\cup C,D=A\cup C$,则
\vspace{-10pt}\[A\subset(A\cup C),C\subset(A\cup C)\Rightarrow (A\cup C)=(B\cup D)\vspace{-10pt}\]显然该命题错误,改成$A\subset B\land C\subset D\Rightarrow (A\cup C)\subseteq (B\cup D)$即可。

证明或证伪:$A\subset B\land C\subset D\Rightarrow (A\cap C)\subset (B\cap D)$\\
解:同样错误,考虑$A\cap C\subset A,C-A\ne\varnothing,B\cap D\subset B,D-B\ne\varnothing$,则转化为\vspace{-10pt}\[(A\cap C)\cap(B\cap D)=(A\cap B)\cap(C\cap D)\subset A\cap B\vspace{-10pt}\]
显然,这里$(A\cap B)=(C\cap D)$完全是可能的,所以仍然需要改成$\subseteq$

性质4显然,一个集合既是$A$的子集,又是$B$的子集,那么其元素都在$A,B$中,即在$A\cap B$中,反过来,一个集合的元素在$A\cap B$中,那么也在$A,B$中,故既是$A$的子集又是$B$的子集。

下面证明性质5:$ P(A - B) \subseteq (P(A) - P(B))\cup\{\varnothing\}$
\begin{align*}
P(A - B) &= \{x|x\subseteq (A\cap\sim B)\}=\{x|x\subseteq A\land x\subseteq\sim B\}\\
&\subseteq\{x|x\subseteq A\land x\not\subseteq B\}=(P(A)-P(B))\cup\{\varnothing\}
\end{align*}
需要说明的是,涉及到幂集的性质的有关证明不可以使用真值表,或者说用真值表比较麻烦,因为我们考虑集合之间的关系的时候,由于集合可能有不止一个元素,所以$x$中的元素全在$A$中的反面,不是“$x$中的元素全不在$A$中”,而是“$x$中的元素不全在$A$中”。这就导致了\vspace{-10pt}\[x\not\subseteq A\land x\not\subseteq B\Leftarrow x\not\subseteq A\cup B\vspace{-10pt}\]和之前的情形大不相同,此时要分多种情形讨论,复杂度明显升高。但是,真值表还是可以帮助我们进行集合的等价运算,如:
\begin{example}{}{}
\[(A\cap B\cap C)\cup(\sim A\cap B\cap C)\cup(A\cap B\cap\sim C)\]\end{example}
\begin{solution}设全集为$U$,观察到三个并集中都有$B$,所以合并:
    \begin{align*}
& (A \cap B \cap C) \cup (\sim A \cap B \cap C) \cup (A \cap B \cap \sim C) \\
&= [B \cap (A \cap C)] \cup [B \cap (\sim A \cap C)] \cup [B \cap (A \cap \sim C)] \quad \text{(结合律)} \\
&= B \cap [(A \cap C) \cup (\sim A \cap C) \cup (A \cap \sim C)] \quad \text{(分配律)} \\
&= B \cap \{[(A \cup \sim A) \cap C] \cup (A \cap \sim C)\} \quad \text{(分配律)} \\
&= B \cap [(U \cap C) \cup (A \cap \sim C)] \quad \text{(补集律:$A \cup \sim A = U$)} \\
&= B \cap [C \cup (A \cap \sim C)] \quad \text{(同一律:$U \cap C = C$)} \\
&= B \cap [(C \cup A) \cap (C \cup \sim C)] \quad \text{(分配律)} \\
&= B \cap [(A \cup C) \cap U] \quad \text{(交换律和补集律)} \\
&= B \cap (A \cup C) \quad \text{(同一律)}
\end{align*}
另一种方式就是使用真值表:我们通过分析元素 $x$ 的属于关系来简化该表达式。定义真值表,其中 1 表示 $x$ 属于集合,0 表示 $x$ 不属于集合:
\begin{center}
\begin{tabular}{ccc|cccc}
$x \in A$ & $x \in B$ & $x \in C$ & $x \in A \cap B \cap C$ & $x \in \sim A \cap B \cap C$ & $x \in A \cap B \cap \sim C$ & $x \in \text{整体表达式}$ \\
\hline
0 & 0 & 0 & 0 & 0 & 0 & 0 \\
0 & 0 & 1 & 0 & 0 & 0 & 0 \\
0 & 1 & 0 & 0 & 0 & 0 & 0 \\
0 & 1 & 1 & 0 & 1 & 0 & 1 \\
1 & 0 & 0 & 0 & 0 & 0 & 0 \\
1 & 0 & 1 & 0 & 0 & 0 & 0 \\
1 & 1 & 0 & 0 & 0 & 1 & 1 \\
1 & 1 & 1 & 1 & 0 & 0 & 1 \\
\end{tabular}
\end{center}
从真值表可知,$x$ 属于整体表达式当且仅当满足以下条件:$x \in B$ 为真, 同时 $x \in A$ 或 $x \in C$ 为真,这等价于 $x \in B \cap (A \cup C)$。

伪装的方式是:\\
设$x$是全集中的任意元素,我们分析$x$属于该并集的条件。\\
若$x \in A \cap B \cap C$,则$x \in A$,$x \in B$,$x \in C$\\
若$x \in \sim A \cap B \cap C$,则$x \notin A$,$x \in B$,$x \in C$\\
若$x \in A \cap B \cap \sim C$,则$x \in A$,$x \in B$,$x \notin C$\\
综合三种情况,$x$属于该并集当且仅当同时满足:$x \in B$,$x \in A$ 或 $x \in C$(即$x \in A \cup C$),因此,$x$属于该并集当且仅当$x \in B \cap (A \cup C)$
\end{solution}
这种等价变形的题目比较简单,首先化成交并补,然后只需要观察重复出现的结构(有的时候还需要进行适当的换元)接着等价运算。题目一难或者看不出来了,就利用真值表来证明。
\section{集合的笛卡尔积}
\begin{definition}{集合的笛卡尔积}{}
设$A$和$B$是两个集合,$A$与$B$的\textbf{笛卡尔积}(或称直积)定义为所有有序对$(a,b)$组成的集合,其中$a \in A$,$b \in B$,记作$A \times B$:\vspace{-10pt}
$$A \times B = \{(a,b) \mid a \in A, b \in B\}\vspace{-10pt}$$
\textbf{推广到$n$个集合}:对于$n$个集合$A_1, A_2, \ldots, A_n$,它们的笛卡尔积定义为所有$n$元有序组$(a_1, a_2, \ldots, a_n)$组成的集合,其中$a_i \in A_i$:
\vspace{-10pt}$$A_1 \times A_2 \times \cdots \times A_n = \{(a_1, a_2, \ldots, a_n) \mid a_i \in A_i, i = 1,2,\ldots,n\}\vspace{-10pt}$$
~~~~当$A = B$时,$A \times A$可简写为$A^2$,$n$个相同集合$A$的笛卡尔积可简写为$A^n$\\
~~~~空集与任何集合的笛卡尔积为空集:$\varnothing \times A = A \times \varnothing = \varnothing$
\end{definition}
\begin{theorem}{笛卡尔积的性质}{}
不交换:$(A\times B=B\times A)\Leftrightarrow (A=\varnothing)\lor(B=\varnothing)\lor(A=B)$\\
分配律:$A\times(B\cup C)=(A\times B)\cup(A\times C)~~~~~(A\cup B)\times C=(A\times C)\cup(B\times C)$\\
分配律:$A\times(B\cap C)=(A\times B)\cap(A\times C)~~~~~(A\cap B)\times C=(A\times C)\cap(B\times C)$\\
分配律:$A\times(B-C)=(A\times B)-(A\times C)~~~~~(A-B)\times C=(A\times C)-(B\times C)$\\
分配律:$(A\oplus B)\times C=(A\times C)\oplus(B\times C)~~~~~~A\times(B\oplus C)=(A\times B)\oplus(A\times C)$\\
积的交等于交的积:$(A\times C)\cap(B\times D)=(A\cap B)\times(C\cap D)$\\
积的并蕴含并的积:$(A\times C)\cup(B\times D)\subseteq(A\cup B)\times(C\cup D)$\\
差的积蕴含积的差:$(A-B)\times(C-D)\subseteq (A\times C)-(B\times D)$(后者元素数量更多,基数更大)\\
存在一个单射:$P(A) \times P(B)\to P(A \times B)$\\
单调性:$A\subseteq C\land B\subseteq D\Rightarrow A\times B\subseteq C\times D$
\end{theorem}
注意到性质4可以分解为性质2和3,性质5可以分解为性质234,下面给出证明:\vspace{-10pt}
\begin{align*}
(x,y)\in A\times(B\cup C)&\Leftrightarrow x\in A\land y\in (B\cup C)\Leftrightarrow x\in A\land (y\in B\lor y\in C)\\
&\Leftrightarrow (x\in A\land y\in B)\lor(x\in A\land y\in C)\\&\Leftrightarrow (x,y)\in A\times B\cup A\times C\\
(x,y)\in A\times(B\cap C)&\Leftrightarrow x\in A\land y\in (B\cap C)\Leftrightarrow x\in A\land (y\in B\land y\in C)\\
&\Leftrightarrow (x\in A\land y\in B\land y\in C)\\&\Leftrightarrow (x,y)\in A\times B\cap A\times C\\
(x,y)\in A\times(B-C)&\Leftrightarrow x\in A\land y\in (B-C)\Leftrightarrow x\in A\land (y\in B\land y\not\in C)\\
&\Leftrightarrow (x\in A\land y\in B\land y\not\in C)\\&\Leftrightarrow (x,y)\in A\times B-(A\times C)\\
(x,y)\in (A\oplus B)\times C&\Leftrightarrow x\in (A\oplus B)\land y\in C\\
&\Leftrightarrow (x\in A\land x\not\in B\land y\in C)\lor(x\in B\land x\not\in A\land y\in C)\\
&\Leftrightarrow (x,y)\in(A\times C)\oplus(B\times C)\\
(x,y) \in (A \times C) \cap (B \times D) &\Leftrightarrow (x,y) \in A \times C \land (x,y) \in B \times D \\
&\Leftrightarrow (x \in A \land y \in C) \land (x \in B \land y \in D) \\
&\Leftrightarrow (x \in A \land x \in B) \land (y \in C \land y \in D) \\
&\Leftrightarrow x \in A \cap B \land y \in C \cap D \\
&\Leftrightarrow (x,y) \in (A \cap B) \times (C \cap D)\\
(x,y) \in (A \times C) \cup (B \times D) &\Leftrightarrow (x,y) \in A \times C \lor (x,y) \in B \times D \\
&\Leftrightarrow (x \in A \land y \in C) \lor (x \in B \land y \in D) \\
&\Leftrightarrow (x\in A\lor x\in B)\land(x\in A\lor y\in D)\\ &\land(y\in C\lor x\in B)\land(y\in C\lor y\in D)\\
&\Rightarrow (x \in A \lor x \in B) \land (y \in C \lor y \in D)\\
&\Leftrightarrow x \in A \cup B \land y \in C \cup D \\
&\Leftrightarrow (x,y) \in (A \cup B) \times (C \cup D)\\
(x,y) \in (A - B) \times (C - D) &\Leftrightarrow x \in A - B \land y \in C - D \\
&\Leftrightarrow (x \in A \land x \notin B) \land (y \in C \land y \notin D) \\
&\Leftrightarrow (x \in A \land y \in C) \land (x \notin B \land y \notin D) \\
&\Rightarrow (x,y) \in A \times C \land (x,y) \notin B \times D \\
&\Leftrightarrow (x,y) \in (A \times C) - (B \times D)
\end{align*}
最后一个蕴含符号是因为$(x \notin B \land y \notin D)\lor(x \in B \land y \notin D)\lor(x \notin B \land y \in D)\Leftrightarrow (x,y) \notin B \times D$\\
最后一个性质当且仅当$A=B=\varnothing$或$A,B\ne\varnothing$时,等号成立。
\section{关系}
\begin{definition}{关系}{}
在集合论中,一个\textbf{二元关系}是指元素都是有序对的非空集合,或者空集。二元关系也可以简称为关系。对于二元关系$R$,如果$(x,y)\in R$,则称$x,y$有$R$关系,反之,则称$x,y$没有$R$关系,写作$x\not Ry$,其中$A=\text{dom}~R,B=\text{ran}~R$,分别称作定义域和值域。\\
\textbf{全域关系}:从$A$到$B$的\textbf{全域关系}是完整的笛卡尔积$A \times B$,即包含所有可能有序对的关系。\\
\textbf{恒等关系}:在集合$A$上的\textbf{恒等关系}(或称对角线关系)是所有形如$(a,a)$的有序对组成的集合,其中$a \in A$\\
\textbf{空关系}:从$A$到$B$的\textbf{空关系}是空集$\varnothing$,即不包含任何有序对的关系。记作$R = \varnothing$\\
一个 $n$ 元关系是 $n$ 个集合 $A_1, A_2, \dots, A_n$ 的笛卡尔积的一个子集,即 $R \subseteq A_1 \times A_2 \times \dots \times A_n$。
\end{definition}
集合$A,B$,如果$|A|=n,|B|=m$,则$|A\times B|=nm$,子集有$2^{mn}$个,所以从$A$到$B$的关系个数为$2^{mn}$。$A$上有$2^{n^2}$个不同的二元关系。比如整除关系,$(2,6)$就属于自然数集上的整数关系。

可以用列举法,描述法来写出集合表示二元关系,如$>=\{(x,y)|x,y\in\mathbb{R}\land x>y\}$
\begin{definition}{关系图}{}
    设$A$是一个有限集合,$R$是$A$上的二元关系(即$R \subseteq A \times A$)。$R$的\textbf{关系图}是一个有向图:\\
\textbf{顶点集}:集合$A$中的每个元素对应图中的一个顶点\\
\textbf{边集}:对于每个有序对$(a,b) \in R$,在图中添加一条从$a$指向$b$的有向边\\
\textbf{自环}:如果$(a,a) \in R$,则在顶点$a$处添加一个自环\\
\textbf{关系图的性质}\\
如果$R$是自反关系,则每个顶点都有自环\\
如果$R$是对称关系,则任意两个顶点之间要么没有边,要么有双向边\\
如果$R$是反对称关系,则任意两个不同顶点之间最多只有一条有向边\\
如果$R$是传递关系,则图中任意两条连续的有向边$a \to b$和$b \to c$都对应一条边$a \to c$
\end{definition}
\begin{definition}{二元关系的邻接矩阵}{}
设 $A = \{a_1, a_2, \ldots, a_m\}$ 和 $B = \{b_1, b_2, \ldots, b_n\}$ 是有限集合,$R \subseteq A \times B$ 是一个从 $A$ 到 $B$ 的二元关系。$R$ 的\textbf{邻接矩阵}是一个 $m \times n$ 的矩阵 $M_R = [m_{ij}]$,其中元素定义为:\vspace{-10pt}
\[m_{ij} = \begin{cases}1, & \text{如果 } (a_i, b_j) \in R \\0, & \text{如果 } (a_i, b_j) \notin R\end{cases}\vspace{-10pt}\]
\textbf{特殊情况}:当 $A = B$ 时,邻接矩阵是方阵,当 $R$ 是空关系时,邻接矩阵是全零矩阵\\
当 $R$ 是全域关系时,邻接矩阵是全一矩阵,当 $R$ 是恒等关系 $I_A$ 时,邻接矩阵是单位矩阵\\
\textbf{性质}:关系的并、交、补运算对应矩阵的布尔运算,关系的复合运算对应矩阵的布尔乘法,关系的逆对应矩阵的转置
\end{definition}
\begin{definition}{关系的逆运算}{}
设 \( R \) 是一个从集合 \( A \) 到集合 \( B \) 的二元关系,即 \( R \subseteq A \times B \)。关系 \( R \) 的\textbf{逆运算}定义为一个新关系 \( R^{-1} \),它是从 \( B \) 到 \( A \) 的关系,用集合描述法表示为:\vspace{-10pt}
\[R^{-1} = \{ (b, a) \mid (a, b) \in R \}\vspace{-10pt}\]
其中,\( R^{-1} \subseteq B \times A \)。逆运算将原关系中的每个有序对的顺序反转。
\end{definition}
由于逆运算只是反转了有序对的顺序,所以有以下性质成立:
\begin{theorem}{关系的逆运算的性质}{}
设$R$和$S$是从集合$A$到集合$B$的二元关系,$Q$是从$B$到集合$C$的二元关系:\\
\textbf{1. 双重逆定理}:$(R^{-1})^{-1} = R$\\
\textbf{2. 并集的逆}:$(R \cup S)^{-1} = R^{-1} \cup S^{-1}$\\
\textbf{3. 交集的逆}:$(R \cap S)^{-1} = R^{-1} \cap S^{-1}$\\
\textbf{4. 差集的逆}:$(R - S)^{-1} = R^{-1} - S^{-1}$\\
\textbf{5. 补集的逆}:$(\sim R)^{-1} = \sim (R^{-1})$,其中$\sim R = (A \times B) - R$是$R$的补集\\
\textbf{6. 复合关系的逆}:$(Q \circ R)^{-1} = R^{-1} \circ Q^{-1}$,其中$Q \circ R$是关系复合运算\\
\textbf{7. 包含关系的逆}:如果$R \subseteq S$,则$R^{-1} \subseteq S^{-1}$\\
\textbf{8. 域和值域的关系}:$\text{dom}(R^{-1}) = \text{ran}(R)$,$\text{ran}(R^{-1}) = \text{dom}(R)$
\end{theorem}
\begin{definition}{关系的复合运算}{}
设 $R$ 是从集合 $A$ 到集合 $B$ 的二元关系,$S$ 是从集合 $B$ 到集合 $C$ 的二元关系,即 $R \subseteq A \times B$,$S \subseteq B \times C$。$R$ 与 $S$ 的\textbf{复合关系}(或称合成关系)是一个从 $A$ 到 $C$ 的二元关系,记作 $S \circ R$\vspace{-10pt}
\[
S \circ R = \{(a, c)\in A \times C \mid \exists b \in B : (a, b) \in R \land (b, c) \in S\}
\]
\end{definition}
如果使用关系图表示关系的复合运算,就是将两张关系图“合并”,\textbf{步骤1:绘制原始关系图},绘制$R$的关系图:$A$中元素指向$B$中元素的有向边,绘制$S$的关系图:$B$中元素指向$C$中元素的有向边。\textbf{步骤2:寻找连接路径}:对于每个$a \in A$和$c \in C$,检查是否存在中间元素$b \in B$,如果存在路径$a \rightarrow b$(属于$R$)和$b \rightarrow c$(属于$S$),则在复合关系图中添加边$a \rightarrow c$。\textbf{步骤3:构建复合关系图}:复合关系图包含顶点集$A \cup C$(中间集合$B$的顶点不出现),边集由所有满足条件的$a \rightarrow c$组成,这相当于在原始图中寻找长度为2的路径并直接连接起点和终点。
\begin{theorem}{关系的复合运算的性质}{}
    $(S\cup P)\circ R=(S\circ R)\cup(P\circ R)\quad R\circ(S\cup P)=(R\circ S)\cup(R\circ P)$\\
    $(S\cap P)\circ R\subseteq(S\circ R)\cap(P\circ R)\quad R\circ(S\cap P)\subseteq(R\circ S)\cap(R\circ P)$\\
    $(Q \circ R)^{-1} = R^{-1} \circ Q^{-1}$
\end{theorem}
为什么一个相等,另一个只有包含关系?其实,是因为把复合运算用谓词逻辑表达后,量词是存在量词,辖域为整个公式,所以如果括号内部是析取,就可以直接等价(分配律),如果括号内部是合取,则需要弱化:$\exists x(P(x) \land Q(x))\Rightarrow \exists xP(x)\land \exists xQ(x)$,下面给出证明:
\begin{align*}
\forall (x,y)\in(S\cup P)\circ R &\Leftrightarrow \exists z((x,z)\in R\land(z,y)\in S\cup P)\\
&\Leftrightarrow \exists z((x,z)\in R\land((z,y)\in S\lor (z,y)\in P))\\
&\Leftrightarrow \exists z(((x,z)\in R\land(z,y)\in S)\lor((x,z)\in R\land(z,y)\in P))\\
&\Leftrightarrow \exists z((x,z)\in R\land(z,y)\in S)\lor \exists z((x,z)\in R\land(z,y)\in P)\\
&\Leftrightarrow (x,y)\in S\circ R\lor (x,y)\in P\circ R\\
&\Leftrightarrow (x,y)\in (S\circ R)\cup(P\circ R)\\
\forall (x,y)\in R\circ(S\cup P) &\Leftrightarrow \exists z((x,z)\in S\cup P\land(z,y)\in R)\\
&\Leftrightarrow \exists z(((x,z)\in S\lor(x,z)\in P)\land(z,y)\in R)\\
&\Leftrightarrow \exists z(((x,z)\in S\land(z,y)\in R)\lor((x,z)\in P\land(z,y)\in R))\\
&\Leftrightarrow \exists z((x,z)\in S\land(z,y)\in R)\lor\exists z((x,z)\in P\land(z,y)\in R)\\
&\Leftrightarrow (x,y)\in R\circ S\lor(x,y)\in R\circ P\\
&\Leftrightarrow (x,y)\in (R\circ S)\cup(R\circ P)\\
\forall (x,y)\in(S\cap P)\circ R &\Leftrightarrow \exists z((x,z)\in R\land(z,y)\in S\cap P)\\
&\Leftrightarrow \exists z((x,z)\in R\land((z,y)\in S\land(z,y)\in P))\\
&\Rightarrow \exists z((x,z)\in R\land(z,y)\in S)\land\exists z((x,z)\in R\land(z,y)\in P)\\
&\Leftrightarrow (x,y)\in S\circ R\land(x,y)\in P\circ R\\
&\Leftrightarrow (x,y)\in (S\circ R)\cap(P\circ R)\\
\forall (x,y)\in R\circ(S\cap P) &\Leftrightarrow \exists z((x,z)\in S\cap P\land(z,y)\in R)\\
&\Leftrightarrow \exists z(((x,z)\in S\land(x,z)\in P)\land(z,y)\in R)\\
&\Rightarrow \exists z((x,z)\in S\land(z,y)\in R)\land\exists z((x,z)\in P\land(z,y)\in R)\\
&\Leftrightarrow (x,y)\in R\circ S\land(x,y)\in R\circ P\\
&\Leftrightarrow (x,y)\in (R\circ S)\cap(R\circ P)
\end{align*}
设 $R \subseteq A \times B$ 和 $Q \subseteq B \times C$ 是二元关系。证明 $(Q \circ R)^{-1} = R^{-1} \circ Q^{-1}$,对于任意有序对 $(c, a)$,有:
\begin{align*}
(c, a) \in (Q \circ R)^{-1} &\Leftrightarrow (a, c) \in Q \circ R \\
&\Leftrightarrow \exists b \in B ((a, b) \in R \land (b, c) \in Q) \\
&\Leftrightarrow \exists b \in B ((b, a) \in R^{-1} \land (c, b) \in Q^{-1}) \\
&\Leftrightarrow (c, a) \in R^{-1} \circ Q^{-1}
\end{align*}
因此,$(Q \circ R)^{-1} = R^{-1} \circ Q^{-1}$ 成立。

证明:设$A$是$R$上的关系,则$R\circ I_A=I_A\circ R=R$
\begin{align*}
(x, y) \in R \circ I_A &\Leftrightarrow \exists z \in A ,((x, z) \in I_A \land (z, y) \in R) \\
&\Leftrightarrow \exists z  (z\in A\land x = z \land (z, y) \in R) \\
&\Rightarrow (x, y) \in R\quad\text{存在量词的合取性质} \\
(x,y)\in R&\Rightarrow (x,x)\in I_A\land(x,y)\in R\\
&\Rightarrow (x,y)\in R
\end{align*}
\begin{definition}{关系的幂}{}
设 \( R \) 是集合 \( A \) 上的二元关系(即 \( R \subseteq A \times A \))。关系 \( R \) 的幂 \( R^n \)(其中 \( n \) 是非负整数)递归定义如下,其中 \( I_A = \{ (a, a) \mid a \in A \} \) 是 \( A \) 上的恒等关系,\( \circ \) 表示关系的复合运算。
关系幂的性质包括:\( R^m \circ R^n = R^{m+n} \),\( (R^m)^n = R^{mn} \):\vspace{-10pt}
\begin{align*}
R^0 &= I_A \quad \text{(恒等关系)} \\\vspace{-10pt}
R^n &= R \circ R^{n-1} \quad \text{对于 } n \geq 1
\end{align*}
\end{definition}
\textbf{性质1:} \( R^m \circ R^n = R^{m+n} \),对 \( n \) 进行数学归纳法。\\
\textbf{基础步骤:} 当 \( n = 0 \)时,$R^m \circ R^0 = R^m \circ I_A = R^m = R^{m+0}$\\
\textbf{归纳假设:} 假设对于 \( n = k \) 有 \( R^m \circ R^k = R^{m+k} \),当 \( n = k+1 \)时
\begin{align*}
R^m \circ R^{k+1} &= R^m \circ (R \circ R^k) \\
&= (R^m \circ R) \circ R^k \quad \text{(复合运算结合律)} \\
&= R^{m+1} \circ R^k \\
&= R^{(m+1)+k} \quad \text{(归纳假设)} \\
&= R^{m+(k+1)}
\end{align*}
因此,由数学归纳法,性质1成立。\\
\textbf{性质2:} \( (R^m)^n = R^{mn} \)对 \( n \) 进行数学归纳法。\\
\textbf{基础步骤:} 当 \( n = 0 \),$(R^m)^0 = I_A = R^0 = R^{m \cdot 0}$\\
\textbf{归纳假设:} 假设对于 \( n = k \) 有 \( (R^m)^k = R^{mk} \),当 \( n = k+1 \)时
\begin{align*}
(R^m)^{k+1} &= (R^m)^k \circ R^m \\
&= R^{mk} \circ R^m \quad \text{(归纳假设)} \\
&= R^{mk + m} \quad \text{(性质1)} \\
&= R^{m(k+1)}
\end{align*}
因此,由数学归纳法,性质2成立。
\begin{theorem}{关系的幂的性质}{}
设$R_1,R_2,R_3$是集合$A$上的二元关系,如果$R_1\subseteq R_2$那么:\\
(1)$R_1\circ R_3\subseteq R_2\circ R_3$\\
(2)$R_3\circ R_1\subseteq R_3\circ R_2$
\end{theorem}
\textbf{证明(1)}:
设$(x,z)\in R_1\circ R_3$,则存在$y\in A$使得$(x,y)\in R_1$且$(y,z)\in R_3$。
由于$R_1\subseteq R_2$,有$(x,y)\in R_2$,因此$(x,z)\in R_2\circ R_3$。
故$R_1\circ R_3\subseteq R_2\circ R_3$。

\textbf{证明(2)}:
设$(x,z)\in R_3\circ R_1$,则存在$y\in A$使得$(x,y)\in R_3$且$(y,z)\in R_1$。
由于$R_1\subseteq R_2$,有$(y,z)\in R_2$,因此$(x,z)\in R_3\circ R_2$。
故$R_3\circ R_1\subseteq R_3\circ R_2$。
\begin{theorem}{关系的幂的性质}{}
    设\( R \) 是集合 \( A \) 上的二元关系(即 \( R \subseteq A \times A \)),且$|A|=n$,则存在自然数$s,t$,使得\\
    (1)$R^s=R^t,\quad 0\leq s<t\leq 2^{n^2}$\quad(2)对任意$k\in\mathbb{N},R^{s+k}=R^{t+k}$
\end{theorem}
设 $|A| = n$,则 $A \times A$ 的子集个数为 $2^{n^2}$,即不同的二元关系最多有 $2^{n^2}$ 个。

考虑序列 $R^0, R^1, R^2, \ldots, R^{2^{n^2}}$,共 $2^{n^2} + 1$ 个关系。由鸽巢原理,必存在 $0 \leq s < t \leq 2^{n^2}$ 使得 $R^s = R^t$,这就证明了性质(1)。对于性质(2),对任意 $k \in \mathbb{N}$,有:
\begin{align*}
R^{s+k} &= R^s \circ R^k \quad \text{(幂的定义)} \\
&= R^t \circ R^k \quad \text{(由 $R^s = R^t$)} \\
&= R^{t+k} \quad \text{(幂的定义)}
\end{align*}
因此,$R^{s+k} = R^{t+k}$ 对任意 $k \in \mathbb{N}$ 成立。
\begin{definition}{布尔加和布尔乘,用矩阵表示关系的复合}{}
\textbf{布尔加(逻辑或运算)}:
定义两个布尔值$a$和$b$的布尔加为:
\[
a \vee b = 
\begin{cases}
1, & \text{如果 } a = 1 \text{ 或 } b = 1 \\
0, & \text{否则}
\end{cases}
\]
布尔加对应逻辑或运算,满足交换律、结合律和幂等律。

\textbf{布尔乘(逻辑与运算)}:
定义两个布尔值$a$和$b$的布尔乘为:\vspace{-10pt}
\[
a \wedge b = 
\begin{cases}
1, & \text{如果 } a = 1 \text{ 且 } b = 1 \\
0, & \text{否则}
\end{cases}\vspace{-10pt}
\]
布尔乘对应逻辑与运算,满足交换律、结合律和分配律。

\textbf{邻接矩阵布尔乘法}:
设$A$和$B$是两个布尔矩阵,其乘积$C = A \cdot B$通过布尔运算定义:\vspace{-10pt}
\[
c_{ij} = \bigvee_{k=1}^{n} (a_{ik} \wedge b_{kj})\vspace{-10pt}
\]
其中:$\wedge$表示布尔乘(逻辑与),用于计算对应元素的乘积,$\vee$表示布尔加(逻辑或),用于累积部分结果,运算结果$c_{ij} = 1$当且仅当存在$k$使得$a_{ik} = 1$且$b_{kj} = 1$,这种布尔矩阵乘法正好对应关系复合的邻接矩阵计算:$M_{S \circ R} = M_R \cdot M_S$。
\end{definition}
翻译成人话,就是:确定$A,B$相乘之后的矩阵$C$的位于某一行某一列的元素,就需要把$A$的对应的行向量和$B$的对应的列向量(先转置,变成横的)抽出来做布尔点积(当我们定义了布尔加和布尔乘,布尔点积就被定义了)。考虑以下示例中使用的矩阵:
\[
A = \begin{bmatrix}
1 & 0 & 1 \\
0 & 1 & 0 \\
1 & 1 & 0
\end{bmatrix}, \quad
B = \begin{bmatrix}
1 & 0 \\
0 & 1 \\
1 & 1
\end{bmatrix}, \quad
AB = \begin{bmatrix}
1 & 1 \\
0 & 1 \\
1 & 1
\end{bmatrix}
\]
\[
\begin{array}{c@{\hspace{-2pt}}c}
 & 
\begin{bmatrix}
1 & 0 \\
0 & 1 \\
1 & 1
\end{bmatrix} \\[8pt]
\begin{bmatrix}
1 & 0 & 1 \\
0 & 1 & 0 \\
1 & 1 & 0
\end{bmatrix} & 
\begin{bmatrix}
1 & 1 \\
0 & 1 \\
1 & 1
\end{bmatrix}
\end{array}
\]
\section{自反,对称,传递关系,闭包}
\begin{definition}{自反,对称,传递关系}{}
设$R$是集合$A$上的二元关系(即$R \subseteq A \times A$)。

\textbf{自反关系}:
$R$是自反的当且仅当对于所有$a \in A$,都有$(a,a) \in R$。
即:$\forall a \in A, (a,a) \in R$

\textbf{对称关系}:
$R$是对称的当且仅当对于所有$a,b \in A$,如果$(a,b) \in R$,则$(b,a) \in R$。
即:$\forall a,b \in A, (a,b) \in R \Rightarrow (b,a) \in R$

\textbf{传递关系}:
$R$是传递的当且仅当对于所有$a,b,c \in A$,如果$(a,b) \in R$且$(b,c) \in R$,则$(a,c) \in R$。
即:$\forall a,b,c \in A, [(a,b) \in R \land (b,c) \in R] \Rightarrow (a,c) \in R$

\textbf{反自反关系}:
$R$是反自反的当且仅当对于所有$a \in A$,都有$(a,a) \notin R$。
即:$\forall a \in A, (a,a) \notin R$

\textbf{反对称关系}:
$R$是反对称的当且仅当对于所有$a,b \in A$,如果$(a,b) \in R$且$(b,a) \in R$,则$a = b$。
即:$\forall a,b \in A, [(a,b) \in R \land (b,a) \in R] \Rightarrow a = b$
\end{definition}
引入关系图,那么自反关系当且仅当每一个结点都有环,反自反关系当且仅当每一个结点都没有环,对称关系当且仅当每两个结点之间要么有2条反向的有向边(可以视作无向边),要么没有边;反对称关系当且仅当每两个结点之间要么没有边,要么只可以有一条有向边。传递关系当且仅当在关系图中若存在结点$a$到$b$的有向边和结点$b$到$c$的有向边,则有结点$a$到$c$的有向边。

如果引入邻接矩阵的话,自反关系对应矩阵的对角线元素全为1,反自反关系对应矩阵的对角线元素全为0,对称关系对应矩阵是对称矩阵,反对称关系对应矩阵是反对称矩阵,传递关系满足其矩阵的2次幂所对应的关系包含于1次幂所对应的关系。

但是要注意:一个关系不可能同时是自反或者反自反关系,但是也可能既不是自反的,也不是反自反的。一个矩阵可以同时是对称的和反对称的(如空关系),但是也可能既不是对称的,也不是反对称的。

空关系和单元素关系(只包含零个,一个有序对的关系)总是传递的,因为当一个命题的前提为假的时候,结论总为真。
\begin{theorem}{自反对称传递关系的性质}{}
(1)$R$是自反的当且仅当$I_A\subseteq R$,$R$是反自反的当且仅当$R-I_A=R$\\
(2)$R$是对称的当且仅当$R=R^T$,$R$是反对称的当且仅当$R\cap R^T\subseteq I_A$\\
(3)$R$是传递的当且仅当$R^2\subseteq R$
\end{theorem}
\begin{theorem}{关系运算对关系属性的保持性}{}
设$R_1, R_2$是集合$A$上的二元关系。各种关系运算对关系属性的保持性如下:
\begin{center}
\begin{tabular}{|c|c|c|c|c|c|c|}
\hline
运算 & 自反性 & 反自反性 & 对称性 & 反对称性 & 传递性 & 等价性 \\
\hline
逆运算 $R^{-1}$ & 保持 & 保持 & 保持 & 保持 & 保持 & 保持 \\
\hline
交集 $R_1 \cap R_2$ & 保持 & 保持 & 保持 & 保持 & 保持 & 保持 \\
\hline
并集 $R_1 \cup R_2$ & 保持 & 保持 & 保持 & 不一定 & 不一定 & 不一定 \\
\hline
差集 $R_1 - R_2$ & 不一定 & 保持 & 保持 & 保持 & 保持 & 不一定 \\
\hline
对称差 $R_1 \oplus R_2$ & 不一定 & 保持 & 保持 & 不一定 & 不一定 & 不一定 \\
\hline
复合 $R_1 \circ R_2$ & 保持 & 不一定 & 不一定 & 不一定 & 不一定 & 不一定 \\
\hline
幂运算 $R^n$ & 保持 & 保持 & 不一定 & 不一定 & 保持 & 不一定 \\
\hline
\end{tabular}
\end{center}
\end{theorem}
\textbf{详细说明与反例}:

\textbf{自反性}:自反关系是否能维持,取决于每一个结点的自环有没有受到影响,显然,并集交集、逆、复合、幂运算保持自反性,差集不一定保持(注意不是一定不保持,因为我们可以让邻接矩阵的大小不同,做差后主对角线上还会有1)

\textbf{反自反性}:和自反关系类似的是,并集交集、逆、幂运算保持反自反性,但是,差集、对称差、也会保持,这是因为两个集合中都没有形如$(x,x)$的有序对,所以不管怎么交怎么并怎么做差,都不会影响到反自反性。但是,复合不一定保持:如$R_1 = \{(1,2)\}$,$R_2 = \{(2,1)\}$,则$R_1 \circ R_2 = \{(1,1)\}$不是反自反的

\textbf{对称性}:并集交集、逆、差集(显然,如果做差会消去元素的话,那一定是成对地消去有序对)、对称差(差集和并集保持,那么对称差保持)保持对称性,复合、幂运算不一定保持:如$A = \{1,2,3\}$,$R = \{(1,2),(2,1),(2,3)\}$对称,但$R^2 = \{(1,1),(1,3),(2,2)\}$不对称

\textbf{反对称性}:交集、差集保持反对称性;并集、对称差、复合、幂运算不一定保持:如$A = \{1,2\}$,$R_1 = \{(1,2)\}$,$R_2 = \{(2,1)\}$都反对称,但$R_1 \cup R_2$不反对称,并集的情况很好理解,因为$R_1,R_2$两个矩阵合在一起,明显可能会导致有元素关于对角线对称。由此,对称差因为含有并集运算,所以也显然不一定维持,幂运算可能会出现$R_1 = \{(1,2),(3,2)\}$,$R_2 = \{(2,3),(2,1)\}$,则$R_1 \circ R_2 = \{(1,1),(1,3),(3,3),(3,1)\}$,不是反对称的

\textbf{传递性}:交集、幂运算保持传递性,并集、对称差、复合不一定保持:如$A = \{1,2\}$,$R_1 = \{(1,2)\}$,$R_2 = \{(2,1)\}$都传递,但$R_1 \cup R_2$不传递,进而对称差也不一定维持,考虑集合 $A = \{1,2,3,4\}$,定义两个关系:$R = \{(1,2), (3,4)\} \quad S = \{(2,3), (4,1)\}$,计算复合关系 $R \circ S$:$R \circ S = \{(1,3), (3,1)\}$,因此,$R \circ S$ 不满足传递性。此反例证明传递关系的复合不一定传递。
\begin{definition}{等价关系}{}
设$R$是集合$A$上的二元关系(即$R \subseteq A \times A$)。如果$R$同时满足是自反的,对称的,传递的,则称$R$为\textbf{等价关系}。若$(x,y)$属于此集合中,记作$x\sim y$
\end{definition}
\begin{definition}{自反/对称/传递闭包}{}
    设$R$是集合$A$上的二元关系。若$R'$满足$R' \subseteq R$且$R'$是自反/对称/传递的,且对$A$上任何包含$R$的关系$R''$,都有$R'\subseteq R''$,则称$R'$是$R$的自反/对称/传递闭包。

\textbf{自反闭包}:
$R$的自反闭包是包含$R$的最小自反关系,记作$r(R)$,定义为:\vspace{-10pt}
$$r(R) = R \cup I_A\vspace{-10pt}$$
其中$I_A = \{(a,a) \mid a \in A\}$是$A$上的恒等关系。

\textbf{对称闭包}:
$R$的对称闭包是包含$R$的最小对称关系,记作$s(R)$,定义为:\vspace{-10pt}
$$s(R) = R \cup R^{-1}\vspace{-10pt}$$
其中$R^{-1} = \{(b,a) \mid (a,b) \in R\}$是$R$的逆关系。

\textbf{传递闭包}:
$R$的传递闭包是包含$R$的最小传递关系,记作$t(R)$,定义为:\vspace{-10pt}
$$t(R) = \bigcup_{n=1}^{\infty} R^n = R \cup R^2 \cup R^3 \cup \cdots\vspace{-10pt}$$
其中$R^n$表示关系$R$的$n$次幂(关系的复合)。 对于元素个数为$n$的集合$A$来说,$A$上的关系$R$的传递闭包又可以写作:
\vspace{-10pt}
$$t(R) = \bigcup_{n=1}^{\infty} R^n = R \cup R^2 \cup R^3 \cup \cdots\cup R^{2^{n^2}}\vspace{-10pt}$$
$R$的自反对称传递闭包写作$tsr(R)=t(s(r(R)))$.
\end{definition}
要证明 $\bigcup_{n=1}^{\infty} R^n = R \cup R^2 \cup R^3 \cup \cdots$ 是 $R$ 的传递闭包,需要证明3个性质:

\textbf{1. 包含性}:$R \subseteq \bigcup_{n=1}^{\infty} R^n$,显然当 $n=1$ 时,$R^1 = R$,所以 $R \subseteq \bigcup_{n=1}^{\infty} R^n$。

\textbf{2. 传递性}:$\bigcup_{n=1}^{\infty} R^n$ 是传递的,设 $(a,b) \in \bigcup_{n=1}^{\infty} R^n$ 且 $(b,c) \in \bigcup_{n=1}^{\infty} R^n$,则存在 $m,n \geq 1$ 使得:$(a,b) \in R^m ,(b,c) \in R^n$,由关系幂的性质,$(a,c) \in R^m \circ R^n = R^{m+n}$,而 $R^{m+n} \subseteq \bigcup_{n=1}^{\infty} R^n$,所以 $(a,c) \in \bigcup_{n=1}^{\infty} R^n$。

\textbf{3. 最小性}:$\bigcup_{n=1}^{\infty} R^n$ 是最小的传递关系

用数学归纳法证明:设 $T$ 是任意包含 $R$ 的传递关系,则对任意 $n \geq 1$,$R^n \subseteq T$。

\textbf{基础情况}:$n=1$ 时,$R^1 = R \subseteq T$。

\textbf{归纳假设}:假设 $R^k \subseteq T$ 对某个 $k \geq 1$ 成立。

\textbf{归纳步骤}:对于 $R^{k+1} = R^k \circ R$,
\begin{align*}
(a,b) \in R^{k+1} &\Rightarrow \exists c \text{ 使得 } (a,c) \in R^k \text{ 且 } (c,b) \in R \\
&\Rightarrow (a,c) \in T \text{ 且 } (c,b) \in T \quad \text{(归纳假设和 $R \subseteq T$)} \\
&\Rightarrow (a,b) \in T \quad \text{(因为 $T$ 是传递的)}
\end{align*}
所以 $R^{k+1} \subseteq T$。

由数学归纳法,对所有 $n \geq 1$,$R^n \subseteq T$,因此 $\bigcup_{n=1}^{\infty} R^n \subseteq T$。

\textbf{结论}:$\bigcup_{n=1}^{\infty} R^n$ 是包含 $R$ 的最小传递关系,即传递闭包。
\begin{theorem}{闭包的性质}{}
设$R_1,R_2$是集合$A$上的二元关系,则:\\
(1)$R_1\subseteq R_2\Rightarrow r(R_1)\subseteq r(R_2)$\\
(2)$R_1\subseteq R_2\Rightarrow s(R_1)\subseteq s(R_2)$\\
(3)$R_1\subseteq R_2\Rightarrow t(R_1)\subseteq t(R_2)$\\
(4)$r(R_1)\cup r(R_2)=r(R_1\cup R_2)\quad r(R_1)\cap r(R_2)=r(R_1\cap R_2)$\\
(5)$s(R_1)\cup s(R_2)=s(R_1\cup R_2)\quad s(R_1\cap R_2)\subseteq s(R_1)\cap s(R_2)$\\
(6)$t(R_1)\cup t(R_2)\subseteq t(R_1\cup R_2)\quad t(R_1\cap R_2)\subseteq t(R_1)\cap t(R_2)$
\end{theorem}
性质1:由于$R_1 \subseteq R_2$,且$I_A \subseteq I_A$,故$r(R_1) = R_1 \cup I_A \subseteq R_2 \cup I_A = r(R_2)$。

性质2:由于$R_1 \subseteq R_2$,则$R_1^{-1} \subseteq R_2^{-1}$,故$s(R_1) = R_1 \cup R_1^{-1} \subseteq R_2 \cup R_2^{-1} = s(R_2)$。

性质3:由于$R_1 \subseteq R_2$,则对任意$n \geq 1$有$R_1^n \subseteq R_2^n$,故$t(R_1) = \bigcup_{n=1}^{\infty} R_1^n \subseteq \bigcup_{n=1}^{\infty} R_2^n = t(R_2)$。

性质4:\vspace{-20pt}\begin{align*}
r(R_1) \cup r(R_2) &= (R_1 \cup I_A) \cup (R_2 \cup I_A) = (R_1 \cup R_2) \cup I_A = r(R_1 \cup R_2) \\
r(R_1) \cap r(R_2) &= (R_1 \cup I_A) \cap (R_2 \cup I_A) = (R_1 \cap R_2) \cup I_A = r(R_1 \cap R_2)
\end{align*}
性质5:
由于$R_1 \cap R_2 \subseteq R_1$且$R_1 \cap R_2 \subseteq R_2$,由单调性得$s(R_1 \cap R_2) \subseteq s(R_1) \cap s(R_2)$。另外:\vspace{-10pt}
\begin{align*}
s(R_1) \cup s(R_2) &= (R_1 \cup R_1^{-1}) \cup (R_2 \cup R_2^{-1}) \\
&= (R_1 \cup R_2) \cup (R_1^{-1} \cup R_2^{-1}) \\
&= (R_1 \cup R_2) \cup (R_1 \cup R_2)^{-1} = s(R_1 \cup R_2)
\end{align*}
性质6:由单调性,$t(R_1) \subseteq t(R_1 \cup R_2)$且$t(R_2) \subseteq t(R_1 \cup R_2)$,故并集包含于右边。同理,由$R_1 \cap R_2 \subseteq R_1$和$R_1 \cap R_2 \subseteq R_2$,得$t(R_1 \cap R_2) \subseteq t(R_1) \cap t(R_2)$。
\begin{theorem}{闭包运算的复合}{}
(1)$s(r(R)) = r(s(R))$\quad(2)$r(t(R)) = t(r(R))$\quad(3)$s(t(R)) \subseteq t(s(R))$
\end{theorem}
(1)性质1证明如下:
\begin{align*}
s(r(R)) &= s(R \cup I_A) = (R \cup I_A) \cup (R \cup I_A)^{-1} \quad \text{(对称闭包定义)} \\
&= R \cup I_A \cup R^{-1} \cup I_A^{-1} \quad \text{(逆运算分配律)} = R \cup R^{-1} \cup I_A \quad \text{(因为 $I_A^{-1} = I_A$)} \\
r(s(R)) &= r(R \cup R^{-1}) = (R \cup R^{-1}) \cup I_A \quad \text{(自反闭包定义)} = R \cup R^{-1} \cup I_A
\end{align*}
(2)性质2证明如下:
\begin{align*}
r(t(R)) = t(R) \cup I_A \quad 
t(r(R)) = t(R \cup I_A) 
\end{align*}
由于$R \cup I_A \subseteq t(R) \cup I_A$,且$t(R) \cup I_A$是传递的,所以$t(R \cup I_A) \subseteq t(R) \cup I_A$。由于$R \subseteq R \cup I_A \subseteq t(R \cup I_A)$且$I_A \subseteq t(R \cup I_A)$,所以$t(R) \subseteq t(R \cup I_A)$,故$t(R) \cup I_A \subseteq t(R \cup I_A)$\\
(3)性质3证明如下:
\[s(t(R)) = t(R) \cup t(R)^{-1} \quad
t(s(R)) = t(R \cup R^{-1}) \]
由于$t(R) \subseteq t(R \cup R^{-1}) = t(s(R))$,且$t(R)^{-1} = t(R^{-1}) \subseteq t(R^{-1} \cup R) = t(s(R))$(因为$R^{-1} \subseteq s(R)$),所以:
\[
s(t(R)) = t(R) \cup t(R)^{-1} \subseteq t(s(R)) \cup t(s(R)) = t(s(R))
\]
因此,$s(t(R)) \subseteq t(s(R))$。这个结论告诉我们;\textbf{传递关系的对称闭包可能会丢失传递性,但是对称关系的传递闭包不会丢失对称性}。比如关系$R=\{(1,2),(3,2)\}$是传递的,但是其对称闭包$s(R)=\{(1,2),(2,1),(2,3),(3,2)\}$并不是传递的.
\begin{theorem}{自反传递对称闭包}
    设$R$是非空集合上的二元关系,则$tsr(R)=t(s(r(R)))$是包含$R$的最小等价关系。
\end{theorem}
设$E = t(s(r(R)))$。需要证明以下三点:

\textbf{1. 包含性}:
由于$r(R) \supseteq R$,$s(r(R)) \supseteq r(R) \supseteq R$,且$t(s(r(R))) \supseteq s(r(R)) \supseteq R$,故$E \supseteq R$。

\textbf{2. $E$是等价关系}:

\textbf{自反性}:$r(R)$是自反的,$s(r(R))$保持自反性(对称闭包不破坏自反性),$t(s(r(R)))$也保持自反性(传递闭包不破坏自反性),故$E$是自反的。

\textbf{对称性}:$s(r(R))$是对称的(因为对称闭包)。若$S$是对称关系,则$t(S)$也是对称的(因为若$(x,y) \in t(S)$,则存在路径$x=x_0,x_1,\ldots,x_n=y$使得$(x_i,x_{i+1}) \in S$;由对称性,路径反向$y=x_n,x_{n-1},\ldots,x_0=x$也在$S$中,故$(y,x) \in t(S)$)。因此$E$是对称的。

\textbf{传递性}:由传递闭包的定义,$E$是传递的。

\textbf{3. 最小性}:
设$E'$是任意包含$R$的等价关系。由于$E'$是自反的,$E' \supseteq r(R)$;由于$E'$是对称的,$E' \supseteq s(r(R))$;由于$E'$是传递的,$E' \supseteq t(s(r(R))) = E$。故$E$是包含$R$的最小等价关系。
\section{Warshell算法}
设$A = \{1,2,3,4,5\}$,关系$R$的邻接矩阵为:
\[
M = \begin{bmatrix}
0 & 1 & 0 & 0 & 0 \\
0 & 0 & 1 & 0 & 0 \\
0 & 0 & 0 & 1 & 0 \\
0 & 0 & 0 & 0 & 1 \\
1 & 0 & 0 & 0 & 0
\end{bmatrix}
\]
这个关系表示一个有向环:$1 \to 2 \to 3 \to 4 \to 5 \to 1$。

\textbf{应用Warshall算法}:

\textbf{迭代1 ($k=1$)}:先关注第一列,因为我们需要关注是否存在某个有序对的后一个数是1,然后,再关注是否存在某个有序对前一个数是1,显然此时就要看第一行了。循环往复这个过程知道遍历完成。
\begin{align*}
M^*[5,1] \land M^*[1,2] &= 1 \land 1 = 1 \Rightarrow M^*[5,2] = 0 \lor 1 = 1 \\
M^*[5,1] \land M^*[1,3] &= 1 \land 0 = 0 \Rightarrow M^*[5,3] = 0 \\
M^*[5,1] \land M^*[1,4] &= 1 \land 0 = 0 \Rightarrow M^*[5,4] = 0 \\
M^*[5,1] \land M^*[1,5] &= 1 \land 0 = 0 \Rightarrow M^*[5,5] = 0
\end{align*}
更新后矩阵:
\[
M^* = \begin{bmatrix}
0 & 1 & 0 & 0 & 0 \\
0 & 0 & 1 & 0 & 0 \\
0 & 0 & 0 & 1 & 0 \\
0 & 0 & 0 & 0 & 1 \\
1 & 1 & 0 & 0 & 0
\end{bmatrix}
\]
\textbf{迭代2 ($k=2$)}:先关注第二列,因为我们需要关注是否存在某个有序对的后一个数是2,然后,再关注是否存在某个有序对前一个数是2,显然此时就要看第二行了。循环往复这个过程知道遍历完成。
\begin{align*}
M^*[1,2] \land M^*[2,3] &= 1 \land 1 = 1 \Rightarrow M^*[1,3] = 0 \lor 1 = 1 \\
M^*[5,2] \land M^*[2,3] &= 1 \land 1 = 1 \Rightarrow M^*[5,3] = 0 \lor 1 = 1 \\
M^*[1,2] \land M^*[2,4] &= 1 \land 0 = 0 \Rightarrow M^*[1,4] = 0 \\
M^*[1,2] \land M^*[2,5] &= 1 \land 0 = 0 \Rightarrow M^*[1,5] = 0
\end{align*}
更新后矩阵:
\[
M^* = \begin{bmatrix}
0 & 1 & 1 & 0 & 0 \\
0 & 0 & 1 & 0 & 0 \\
0 & 0 & 0 & 1 & 0 \\
0 & 0 & 0 & 0 & 1 \\
1 & 1 & 1 & 0 & 0
\end{bmatrix}
\]
\textbf{迭代3 ($k=3$)}:先关注第三列,因为我们需要关注是否存在某个有序对的后一个数是3,然后,再关注是否存在某个有序对前一个数是3,显然此时就要看第三行了。循环往复这个过程知道遍历完成。
\begin{align*}
M^*[1,3] \land M^*[3,4] &= 1 \land 1 = 1 \Rightarrow M^*[1,4] = 0 \lor 1 = 1 \\
M^*[2,3] \land M^*[3,4] &= 1 \land 1 = 1 \Rightarrow M^*[2,4] = 0 \lor 1 = 1 \\
M^*[5,3] \land M^*[3,4] &= 1 \land 1 = 1 \Rightarrow M^*[5,4] = 0 \lor 1 = 1
\end{align*}
更新后矩阵:
\[
M^* = \begin{bmatrix}
0 & 1 & 1 & 1 & 0 \\
0 & 0 & 1 & 1 & 0 \\
0 & 0 & 0 & 1 & 0 \\
0 & 0 & 0 & 0 & 1 \\
1 & 1 & 1 & 1 & 0
\end{bmatrix}
\]
\textbf{迭代4 ($k=4$)}:先关注第四列,因为我们需要关注是否存在某个有序对的后一个数是4,然后,再关注是否存在某个有序对前一个数是4,显然此时就要看第四行了。循环往复这个过程知道遍历完成。
\begin{align*}
M^*[1,4] \land M^*[4,5] &= 1 \land 1 = 1 \Rightarrow M^*[1,5] = 0 \lor 1 = 1 \\
M^*[2,4] \land M^*[4,5] &= 1 \land 1 = 1 \Rightarrow M^*[2,5] = 0 \lor 1 = 1 \\
M^*[3,4] \land M^*[4,5] &= 1 \land 1 = 1 \Rightarrow M^*[3,5] = 0 \lor 1 = 1 \\
M^*[5,4] \land M^*[4,5] &= 1 \land 1 = 1 \Rightarrow M^*[5,5] = 0 \lor 1 = 1
\end{align*}
更新后矩阵:
\[
M^* = \begin{bmatrix}
0 & 1 & 1 & 1 & 1 \\
0 & 0 & 1 & 1 & 1 \\
0 & 0 & 0 & 1 & 1 \\
0 & 0 & 0 & 0 & 1 \\
1 & 1 & 1 & 1 & 1
\end{bmatrix}
\]
\textbf{迭代5 ($k=5$)}:先关注第五列,因为我们需要关注是否存在某个有序对的后一个数是5,然后,再关注是否存在某个有序对前一个数是5,显然此时就要看第五行了。循环往复这个过程知道遍历完成。
\textbf{迭代5 ($k=5$)}:
\begin{align*}
M^*[1,5] \land M^*[5,1] &= 1 \land 1 = 1 \Rightarrow M^*[1,1] = 0 \lor 1 = 1 \\
M^*[2,5] \land M^*[5,2] &= 1 \land 1 = 1 \Rightarrow M^*[2,2] = 0 \lor 1 = 1 \\
M^*[3,5] \land M^*[5,3] &= 1 \land 1 = 1 \Rightarrow M^*[3,3] = 0 \lor 1 = 1 \\
M^*[4,5] \land M^*[5,4] &= 1 \land 1 = 1 \Rightarrow M^*[4,4] = 0 \lor 1 = 1
\end{align*}
最终传递闭包矩阵:
\[
M^* = \begin{bmatrix}
1 & 1 & 1 & 1 & 1 \\
1 & 1 & 1 & 1 & 1 \\
1 & 1 & 1 & 1 & 1 \\
1 & 1 & 1 & 1 & 1 \\
1 & 1 & 1 & 1 & 1
\end{bmatrix}
\]
\textbf{结果分析}:由于原关系构成一个有向环,传递闭包是全关系,即任意两个顶点之间都存在路径。这个例子展示了Warshall算法如何处理复杂的有向图结构。
\section{等价关系,等价类,覆盖和划分}
\begin{definition}{等价类,商集,覆盖,划分,类,块}{}
设$R$是集合$A$上的一个等价关系(即$R$满足自反性、对称性和传递性)。对于任意元素$a \in A$,$a$的\textbf{等价类}定义为:\vspace{-10pt}
\[
[a]_R = \{x \in A \mid (a,x) \in R\}
\]
\textbf{商集}:所有等价类构成的集合称为$A$关于$R$的商集,记作:\vspace{-5pt}
\[
A/R = \{[a]_R \mid a \in A\}\vspace{-5pt}
\]
设$A$是一个非空集合,$\mathcal{C} = \{A_1, A_2, \ldots, A_n\}$是$A$的一族\boxed{\textbf{非空}}子集。\\
\textbf{覆盖}:如果$\mathcal{C}$满足$\bigcup_{i=1}^{n} A_i = A$,则称$\mathcal{C}$是$A$的一个\textbf{覆盖}。\\
\textbf{划分}:如果$\mathcal{C}$满足以下两个条件:$\bigcup_{i=1}^{n} A_i = A$(覆盖性),对任意$i \neq j$,有$A_i \cap A_j = \varnothing$(互不相交性),则称$\mathcal{C}$是$A$的一个\textbf{划分}。此时$\mathcal{C} = \{A_1, A_2, \ldots, A_n\}$中的任意一个元素为$A$的一个类或划分的一个块。
\end{definition}
设集合 $A = \{1, 2, 3, 4, 5, 6, 7, 8\}$,定义等价关系 $R$ 为模三同余关系,即对于任意 $a, b \in A$,$a \, R \, b$ 当且仅当 $a \equiv b \pmod{3}$。则 $A$ 的等价类如下:余数为 $0$ 的等价类:$[3] = \{3, 6\}$,余数为 $1$ 的等价类:$[1] = \{1, 4, 7\}$,余数为 $2$ 的等价类:$[2] = \{2, 5, 8\}$。注意:等价类的代表元选择不唯一,例如 $[3] = [6]$,$[1] = [4] = [7]$,$[2] = [5] = [8]$。

商集为:\vspace{-5pt}
\[
A/R = \{[1], [2], [3]\} = \{\{1,4,7\}, \{2,5,8\}, \{3,6\}\}\vspace{-5pt}
\]

\textbf{商集 $A/R$ 是 $A$ 的一个划分},第一点是因为\textbf{覆盖性}:$[1] \cup [2] \cup [3] = \{1,4,7\} \cup \{2,5,8\} \cup \{3,6\} = A$;第二点是因为\textbf{互不相交性}:$[1] \cap [2] = \varnothing$,$[1] \cap [3] = \varnothing$,$[2] \cap [3] = \varnothing$。\\
\textbf{覆盖但不构成划分的例子}:考虑 $\mathcal{C} = \{\{1,2,3\}, \{3,4,5\}, \{5,6,7\}, \{7,8,1\}\}$\\先看\textbf{覆盖性}:$\{1,2,3\} \cup \{3,4,5\} \cup \{5,6,7\} \cup \{7,8,1\} = A$,是覆盖
\\但不构成划分,因为 $\{1,2,3\} \cap \{3,4,5\} = \{3\} \neq \varnothing$
\begin{theorem}{等价类的性质}{}
(1)$\forall x\in A,[x]_R\ne\varnothing$,且$[x]_R\subseteq A$。\\
(2)$\forall x,y\in A,(x,y)\in R\Rightarrow [x]_R=[y]_R$。\\
(3)$\forall x,y\in A,(x,y)\notin R\Rightarrow [x]_R\cap[y]_R=\varnothing$。\\
(4)$\bigcup_{x\in A} [x]_R = A$。
\end{theorem}
\textbf{证明(1)}:
由于$R$是自反的,对任意$x \in A$,有$(x,x) \in R$,所以$x \in [x]_R$,故$[x]_R \neq \varnothing$。
由等价类的定义$[x]_R = \{y \in A \mid (x,y) \in R\}$,显然$[x]_R \subseteq A$。

\textbf{证明(2)}:
设$(x,y) \in R$。要证$[x]_R = [y]_R$。

先证$[x]_R \subseteq [y]_R$:对任意$z \in [x]_R$,有$(x,z) \in R$。
由对称性,$(y,x) \in R$,再由传递性,$(y,x) \in R$且$(x,z) \in R$推出$(y,z) \in R$,所以$z \in [y]_R$。

再证$[y]_R \subseteq [x]_R$:对任意$z \in [y]_R$,有$(y,z) \in R$。
由对称性,$(x,y) \in R$,再由传递性,$(x,y) \in R$且$(y,z) \in R$推出$(x,z) \in R$,所以$z \in [x]_R$。因此$[x]_R = [y]_R$。

\textbf{证明(3)}:
用反证法。假设存在$x,y \in A$满足$(x,y) \notin R$,但$[x]_R \cap [y]_R \neq \varnothing$,
则存在$z \in [x]_R \cap [y]_R$,即$(x,z) \in R$且$(y,z) \in R$。由对称性,$(z,y) \in R$,再由传递性,$(x,z) \in R$且$(z,y) \in R$推出$(x,y) \in R$,与假设矛盾。故$[x]_R \cap [y]_R = \varnothing$。

\textbf{证明(4)}:
显然$\bigcup_{x\in A} [x]_R \subseteq A$。
另一方面,对任意$a \in A$,有$a \in [a]_R \subseteq \bigcup_{x\in A} [x]_R$,所以$A \subseteq \bigcup_{x\in A} [x]_R$。因此$\bigcup_{x\in A} [x]_R = A$。
\begin{theorem}{划分是等价关系的另一种描述}{}
    假设$\{A_1, A_2, \ldots, A_n\}$是$A$的一个划分,则$R$是$A$上的关系,$R=\{(x,y) \mid x,y\in A_i,i=1,2,\ldots,n\}$,则$R$是等价关系。
\end{theorem}
要证明$R$是等价关系,需验证自反性、对称性和传递性。

\textbf{1. 自反性}:对于任意$x \in A$,由于$\{A_1, A_2, \ldots, A_n\}$是$A$的划分,存在$i$使得$x \in A_i$。由$R$的定义,$(x, x) \in R$,故$R$是自反的。

\textbf{2. 对称性}:对于任意$x, y \in A$,若$(x, y) \in R$,则存在$i$使得$x, y \in A_i$。因此$y, x \in A_i$,故$(y, x) \in R$,$R$是对称的。

\textbf{3. 传递性}:对于任意$x, y, z \in A$,若$(x, y) \in R$且$(y, z) \in R$,则存在$i, j$使得$x, y \in A_i$且$y, z \in A_j$。由于$y \in A_i \cap A_j$,且划分中的集合互不相交,故$A_i = A_j$。因此$x, z \in A_i$,从而$(x, z) \in R$,$R$是传递的。

综上,$R$是等价关系。
\begin{theorem}{等价关系与划分的一一对应}{}
设$A$是一个非空集合,$\{A_1, A_2, \ldots, A_m\}$是$A$的一个划分。则:

划分的每一个划分块$A_i$($i=1,2,\ldots,m$)中的任意两个元素都有关系$R$,因此每一个划分块都是$R$的一个等价类。
    
由此可见,集合$A$上的等价关系是对集合$A$中的元素做划分,使得同一划分块中的元素之间有等价关系。
    
反之,由一个划分可以确定唯一的一个等价关系,这个等价关系的等价类就是划分的划分块。也就是说,集合$A$上的等价关系与集合$A$的划分是一一对应的。
    
在非空集合$A$上给定一个划分,可以找出由该划分所唯一确定的$A$上的等价关系,方法如下:把$A$的划分$\{A_1, A_2, \ldots, A_m\}$的每一个划分块$A_i$,求笛卡儿积$A_i \times A_i$,然后求这些笛卡儿积的并集:
    \[
    R = (A_1 \times A_1) \cup (A_2 \times A_2) \cup \cdots \cup (A_m \times A_m)
    \]
这个并集$R$即为所求的等价关系。
\end{theorem}
给定一个等价关系$R$,它的等价类构成$A$的一个划分,给定$A$的一个划分,通过上述方法构造的关系$R$是一个等价关系,这两种构造是互逆的,建立了等价关系与划分之间的一一对应。

设集合$A = \{1, 2, 3, 4, 5\}$,考虑$A$的一个划分:
\[
\mathcal{P} = \{\{1, 2, 3\}, \{4, 5\}\}
\]
按照定理中的方法,我们构造等价关系$R$:
\begin{align*}
R &= (A_1 \times A_1) \cup (A_2 \times A_2) \\
  &= (\{1,2,3\} \times \{1,2,3\}) \cup (\{4,5\} \times \{4,5\}) \\
  &= \{(1,1),(1,2),(1,3),(2,1),(2,2),(2,3),(3,1),(3,2),(3,3)\} \\
  &\quad \cup \{(4,4),(4,5),(5,4),(5,5)\}
\end{align*}
如果从等价关系$R$出发,可以得到划分$A/R$,这个划分恰好就是原来的划分$\mathcal{P}$。
\begin{example}{写出一个集合所有的等价关系}{}
    求集合$A = \{1, 2, 3\}$的所有的等价关系及其对应的商集
\end{example}
集合$A$的等价关系与$A$的划分一一对应。$A$的划分共有5种,如下所示:
\begin{enumerate}
    \item \textbf{划分:} $\{\{1,2,3\}\}$ \\
          \textbf{等价关系:} $R_1 = A \times A = \{(1,1),(1,2),(1,3),(2,1),(2,2),(2,3),(3,1),(3,2),(3,3)\}$ \\
          \textbf{商集:} $A/R_1 = \{\{1,2,3\}\}$

    \item \textbf{划分:} $\{\{1,2\}, \{3\}\}$ \\
          \textbf{等价关系:} $R_2 = (\{1,2\} \times \{1,2\}) \cup (\{3\} \times \{3\}) = \{(1,1),(1,2),(2,1),(2,2),(3,3)\}$ \\
          \textbf{商集:} $A/R_2 = \{\{1,2\}, \{3\}\}$

    \item \textbf{划分:} $\{\{1,3\}, \{2\}\}$ \\
          \textbf{等价关系:} $R_3 = (\{1,3\} \times \{1,3\}) \cup (\{2\} \times \{2\})= \{(1,1),(1,3),(3,1),(3,3),(2,2)\}$ \\
          \textbf{商集:} $A/R_3 = \{\{1,3\}, \{2\}\}$

    \item \textbf{划分:} $\{\{2,3\}, \{1\}\}$ \\
          \textbf{等价关系:} $R_4 = (\{2,3\} \times \{2,3\}) \cup (\{1\} \times \{1\})= \{(2,2),(2,3),(3,2),(3,3),(1,1)\}$ \\
          \textbf{商集:} $A/R_4 = \{\{2,3\}, \{1\}\}$

    \item \textbf{划分:} $\{\{1\}, \{2\}, \{3\}\}$ \\
          \textbf{等价关系:} $R_5 = (\{1\} \times \{1\}) \cup (\{2\} \times \{2\}) \cup (\{3\} \times \{3\})= \{(1,1),(2,2),(3,3)\}$ \\
          \textbf{商集:} $A/R_5 = \{\{1\}, \{2\}, \{3\}\}$
\end{enumerate}
因此,集合$A$共有5个不同的等价关系,分别对应5种划分。
\begin{theorem}{等价关系之间的关系等价于划分之间的关系}{}
    设$R_1$和$R_2$是集合$A$上的两个等价关系,则以下两个条件等价:\\
    (1)$R_1 \subseteq R_2$\\
    (2)$R_1$的划分$C_1$的每一个等价类都是$R_2$的划分$C_2$的一个等价类
\end{theorem}
\textbf{(1) $\Rightarrow$ (2)}:
假设$R_1 \subseteq R_2$。设$[x]_{R_1}$是$R_1$的任意一个等价类。对任意$y \in [x]_{R_1}$,有$(x,y) \in R_1$。由于$R_1 \subseteq R_2$,所以$(x,y) \in R_2$,即$y \in [x]_{R_2}$。因此$[x]_{R_1} \subseteq [x]_{R_2}$,即$R_1$的每个等价类都包含在$R_2$的某个等价类中。

\textbf{(2) $\Rightarrow$ (1)}:
假设$R_1$的每个等价类都是$R_2$的某个等价类的子集。设$(x,y) \in R_1$,则$x$和$y$属于$R_1$的同一个等价类$[x]_{R_1}$。由假设,存在$R_2$的等价类$C$使得$[x]_{R_1} \subseteq C$,所以$x,y \in C$,即$(x,y) \in R_2$。因此$R_1 \subseteq R_2$。

综上,两个条件等价。
\section{偏序关系,全序关系,良序关系}
\begin{definition}{偏序关系,可比,覆盖,全序关系}{}
设$R$是集合$A$上的二元关系。\\
\textbf{偏序关系}:如果$R$满足自反性,反对称性,传递性,则称$R$为$A$上的偏序关系,通常用符号"$\preccurlyeq $"表示偏序关系,记作$(A, \preccurlyeq )$。若$(x,y)\in R$,记作$x \preccurlyeq  y$\\
\textbf{可比}:设$(A, \preccurlyeq )$是偏序集,$x,y \in A$。如果$x \preccurlyeq  y$或$y \preccurlyeq  x$,则称$x$和$y$是\textbf{可比的};否则称$x$和$y$是\textbf{不可比的}。\\
\textbf{覆盖}:设$(A, \preccurlyeq )$是偏序集,$x,y \in A$且$x \neq y$。如果$x \preccurlyeq  y,x\neq y$(即$x\prec y$)且不存在$z \in A$使得$x \prec z\prec  y$,则称$y$ \textbf{覆盖} $x$(或称$x$是$y$的\textbf{直接前驱},$y$是$x$的\textbf{直接后继})。\\
\textbf{全序关系}:设$(A, \preccurlyeq)$是偏序集。如果对于任意$x, y \in A$,$x$和$y$都是可比的,即$\forall x, y \in A, x \preccurlyeq y \quad \text{或} \quad y \preccurlyeq x$,则称$\preccurlyeq$是$A$上的\textbf{全序关系}(或称\textbf{线性序关系}),此时$(A, \preccurlyeq)$称为\textbf{全序集}。
\end{definition}
考虑集合 $A = \{1, 2, 3, 4, 5, 6\}$,定义整除关系 $\preccurlyeq$ 为:$
x \preccurlyeq y \quad \text{当且仅当} \quad x \mid y \quad (\text{即 } x \text{ 整除 } y)$,则 $(A, \preccurlyeq)$ 构成一个偏序集,因为整除关系满足:

\textbf{自反性}:$\forall x \in A, x \mid x$

\textbf{反对称性}:若 $x \mid y$ 且 $y \mid x$,则 $x = y$

\textbf{传递性}:若 $x \mid y$ 且 $y \mid z$,则 $x \mid z$

\textbf{覆盖关系分析}:在偏序集中,$y$ \textbf{覆盖} $x$ 当且仅当$x \prec y$,且不存在 $z \in A$($z \neq x, z \neq y$)使得 $x \prec z \prec y$,考虑元素 $1$ 和 $4$:$1 \preccurlyeq 4$ 成立,因为 $1 \mid 4$
,但存在 $z = 2$,使得 $1 \preccurlyeq 2 \preccurlyeq 4$(即 $1 \mid 2$ 且 $2 \mid 4$),因此,$4$ 不覆盖 $1$,类似地,$4$ 覆盖 $2$,因为 $2 \mid 4$ 且不存在 $z$ 满足 $2 \prec z \prec 4$

\textbf{在判断偏序关系时,自反性是容易被忽视的,比如小于和小于等于。}
\begin{definition}{哈斯图}{}
哈斯图(Hasse diagram)是有限偏序集的一种直观图示方法

\textbf{顶点表示}:用平面上的点表示偏序集中的每个元素

\textbf{位置安排}:如果$x \preccurlyeq y$,则将$x$放在$y$的下方(保持偏序的方向性)

\textbf{边的关系}:仅当$y$覆盖$x$(即$y$是$x$的直接后继)时,才在$x$和$y$之间画一条线段

\textbf{省略规则}:省略自反关系(不画自环),省略传递关系(如果$x \preccurlyeq y$且$y \preccurlyeq z$,但不画$x$到$z$的边),尽量避免边的交叉,保持图的清晰性
\end{definition}
\begin{figure}[!ht]
\centering
\resizebox{0.2\textwidth}{!}{%
\begin{circuitikz}
\tikzstyle{every node}=[font=\small]
\draw (4.75,12.75) to[short, -o] (4.75,11.5) ;
\draw (4.75,12.75) to[short, -o] (6,14) ;
\draw (4.75,11.5) to[short, -o] (6,12.75) ;
\draw (6,14) to[short, -o] (6,12.75) ;
\draw (6,12.75) to[short, -o] (6,11.5) ;
\draw (4.75,11.5) to[short, -o] (4.75,12.75) ;
\node [font=\small, rotate around={-2:(0,0)}] at (6,14.5) {a};
\node [font=\small, rotate around={-2:(0,0)}] at (6.5,12.75) {c};
\node [font=\small, rotate around={-2:(0,0)}] at (6.5,11.75) {e};
\node [font=\small, rotate around={-2:(0,0)}] at (4,11.75) {d};
\node [font=\small, rotate around={-2:(0,0)}] at (4,12.75) {b};
\end{circuitikz}
}%
\end{figure}
如上图,集合$A=\{1,2,3,4,5\}$,集合$A$上的偏序关系\[(A,\preccurlyeq)=\{(d,b),(d,c),(d,a),(e,c),(e,a),(b,a),(c,a),(a,a),(b,b),(c,c),(d,d),(e,e)\}\]窍门是从下往上一个个地遍历所有结点,这样能保证不重不漏。

以下给出$\{2,3,6,12,24,36\}$的整除关系的哈斯图:
\begin{figure}[!ht]
\centering
\resizebox{0.15\textwidth}{!}{%
\begin{circuitikz}
\tikzstyle{every node}=[font=\normalsize]
\draw (3.25,15.5) to[short, -o] (2.5,16.25) ;
\draw (3.25,15.5) to[short, -o] (4,16.25) ;
\draw (3.25,15.5) to[short, -o] (3.25,14.75) ;
\draw (3.25,14.75) to[short, -o] (2.5,14) ;
\draw (3.25,14.75) to[short, -o] (4,14) ;
\draw (3.25,14.75) to[short, -o] (3.25,15.5) ;
\node [font=\normalsize] at (2,16.75) {24};
\node [font=\normalsize] at (4,16.75) {36};
\node [font=\normalsize] at (2,13.75) {2};
\node [font=\normalsize] at (4,13.75) {3};
\node [font=\normalsize] at (2.5,14.75) {6};
\node [font=\normalsize] at (3.5,15.25) {12};
\end{circuitikz}
}%
\end{figure}
\[
R = \left\{
\begin{aligned}
& (2,2), (3,3), (6,6), (12,12), (24,24), (36,36), \\
& (2,6), (2,12), (2,24), (2,36), \\
& (3,6), (3,12), (3,24), (3,36), \\
& (6,12), (6,24), (6,36), \\
& (12,24), (12,36)
\end{aligned}
\right\}
\]
\begin{definition}{偏序集中的特殊元素}{}
设$(S, \preccurlyeq)$是一个偏序集,$A \subseteq S$是一个非空子集。

\textbf{最小元}:
元素$a_0 \in A$称为$A$的\textbf{最小元},如果对于所有$a \in A$,有$a_0 \preccurlyeq a$。

\textbf{最大元}:
元素$a_1 \in A$称为$A$的\textbf{最大元},如果对于所有$a \in A$,有$a \preccurlyeq a_1$。

\textbf{极小元}:
元素$a_m \in A$称为$A$的\textbf{极小元},如果不存在$a \in A$使得$a \prec a_m$(即不存在$a \in A$满足$a \preccurlyeq a_m$且$a \neq a_m$)。

\textbf{极大元}:
元素$a_M \in A$称为$A$的\textbf{极大元},如果不存在$a \in A$使得$a_M \prec a$(即不存在$a \in A$满足$a_M \preccurlyeq a$且$a_M \neq a$)。

\textbf{上界}:
元素$s \in S$称为$A$的\textbf{上界},如果对于所有$a \in A$,有$a \preccurlyeq s$。

\textbf{下界}:
元素$s \in S$称为$A$的\textbf{下界},如果对于所有$a \in A$,有$s \preccurlyeq a$。

\textbf{最小上界(上确界)}:
$A$的\textbf{最小上界}是$A$的所有上界中的最小元,记作$\sup A$或$\bigvee A$。

\textbf{最大下界(下确界)}:
$A$的\textbf{最大下界}是$A$的所有下界中的最大元,记作$\inf A$或$\bigwedge A$。
\end{definition}
最小元和最大元如果存在则唯一,但极小元和极大元可以有多个,上界和下界不一定属于$A$,它们可以是$S$中的任意元素,最小上界和最大下界如果存在则唯一,符号$\prec$表示严格偏序:$x \prec y \Leftrightarrow (x \preccurlyeq y \land x \neq y)$,如下图是偏序关系$(A,R)$的哈斯图:
\begin{figure}[!ht]
\centering
\resizebox{0.2\textwidth}{!}{%
\begin{circuitikz}
\tikzstyle{every node}=[font=\normalsize]
\draw (3.5,16.5) to[short, -o] (3.5,14.5) ;
\draw (5.5,16.5) to[short, -o] (5.5,14.5) ;
\draw (3.5,14.5) to[short, -o] (5.5,16.5) ;
\draw (3.5,14.5) to[short, -o] (3.5,16.5) ;
\draw (3.5,16.5) to[short, -o] (4.5,17.5) ;
\draw (4.5,17.5) to[short, -o] (5.5,16.5) ;
\draw (3.5,14.5) to[short, -o] (4.5,13.5) ;
\draw (5.5,14.5) to[short, -o] (4.5,13.5) ;
\draw (5.5,14.5) to[short, -o] (6.5,13.5) ;
\draw (3.5,14.5) to[short, -o] (2.5,13.5) ;
\node [font=\normalsize] at (4.5,18.25) {h};
\node [font=\normalsize] at (2.5,16.75) {f};
\node [font=\normalsize] at (6,16.75) {g};
\node [font=\normalsize] at (6,14.75) {e};
\node [font=\normalsize] at (4.5,13.25) {b};
\node [font=\normalsize] at (6.5,13.25) {c};
\node [font=\normalsize] at (2,13.25) {a};
\node [font=\normalsize] at (3,14.75) {d};
\end{circuitikz}
}%
\end{figure}
子集$\{a,b,c\}$的上界是$g,h$,最小上界是$g$,无下界和最大下界。子集$\{f,g,h\}$的上界是$h$,最小上界是$h$,下界是$a,b,d$,最大下界是$d$。子集$\{e,g\}$的上界是$g,h$,最小上界是$g$,下界是$b,c,e$,最大下界是$e$。偏序集
\begin{align*}
R = &(a,a), (b,b), (c,c), (d,d), (e,e), (f,f), (g,g), (h,h),\\
&(a,d), (a,f), (a,g), (a,h),\\
&(b,d), (b,e), (b,f), (b,g), (b,h),\\
&(c,e), (c,g), (c,h),\\
&(d,f), (d,g), (d,h),\\
&(e,g), (e,h),\\
&(f,h),\\
&(g,h)\\
\end{align*}
\begin{definition}{良序集}{}
设$(S, \preccurlyeq)$是一个全序集(即$\preccurlyeq$是$S$上的全序关系)。如果$S$的每个非空子集都有最小元,则称$(S, \preccurlyeq)$为\textbf{良序集},$\preceq$称为$S$上的\textbf{良序关系}。
\end{definition}
$(S, \preccurlyeq)$是良序集当且仅当:$\preccurlyeq$是$S$上的全序关系;对$S$的任意非空子集$A \subseteq S$,存在$a_0 \in A$使得对所有的$a \in A$,有$a_0 \preccurlyeq a$。

良序集一定是全序集,但全序集不一定是良序集,\textbf{有限全序集一定是良序集},自然数集$(\mathbb{N}, \leq)$是良序集(良序原理),整数集$(\mathbb{Z}, \leq)$是全序集但不是良序集(因为负整数子集没有最小元),实数上的小于等于关系$(\mathbb{R}, \leq)$是全序集但不是良序集(因为开区间$(0,1)$没有最小元)。
\begin{definition}{严格偏序关系}{}
    反自反,反对称和传递关系,称作严格偏序关系。严格偏序关系和偏序关系有密切联系,区别仅在于$(x,x)$是否属于这个关系。
\end{definition}
比如实数集上的小于关系不是偏序关系,因为其不满足自反性,但是却是严格偏序关系。
