\chapter{集合与关系}
\section{集合}
\begin{definition}{空集,全集,有限集,无限集,元素个数}{}
\textbf{空集}:不含任何元素的集合,记作$\nothing$或$\{x|x\ne x\}$。空集是任何集合的子集。空集是唯一的\\
\textbf{全集}:特定讨论中包含所有对象的集合,记作$U$。全集是相对的,取决于讨论的上下文。\\
\textbf{有限集}:元素个数有限的集合。如果集合$A$有$n$个元素,其中$n$是非负整数,则$A$是有限集。\\
\textbf{无限集}:不是有限集的集合,即元素个数无限多的集合。如自然数集$\mathbb{N}$、实数集$\mathbb{R}$等。\\
\textbf{元素个数}:有限集$A$中元素的数目,记作$|A|$或$card(A)$。无限集的元素个数用基数概念描述,如可数无限集、不可数无限集等。
\end{definition}
\begin{definition}{集合的包含关系}{}
    \textbf{子集}:如果集合$A$中的每个元素都是集合$B$中的元素,则称$A$是$B$的\textbf{子集},记作$A \subseteq B$。空集是空集的子集\\
    \textbf{真子集}:如果$A \subseteq B$且$A \neq B$,则称$A$是$B$的\textbf{真子集},记作$A \subset B$。即\vspace{-10pt}
    \[\forall x(A(x) \rightarrow B(x)) \land \exists x(B(x) \land \neg A(x))\]
\end{definition}
\begin{definition}{集合的基本运算}{}
\textbf{交集}:集合$A$和$B$的交集是由所有既属于$A$又属于$B$的元素组成的集合,记作$A \cap B$。\vspace{-10pt}
$$A \cap B = \{x \mid x \in A \land x \in B\}\vspace{-10pt}$$
\textbf{并集}:集合$A$和$B$的并集是由所有属于$A$或属于$B$的元素组成的集合,记作$A \cup B$。\vspace{-10pt}
$$A \cup B = \{x \mid x \in A \lor x \in B\}\vspace{-10pt}$$
\textbf{补集}:集合$A$的补集是由全集中所有不属于$A$的元素组成的集合,记作$\sim  A$。\vspace{-10pt}
$$\sim A = \{x \in U \mid x \notin A\}\vspace{-10pt}$$
\textbf{差集}:集合$A$与$B$的差集是由所有属于$A$但不属于$B$的元素组成的集合:\vspace{-10pt}
$$A-B = \{x \mid x \in A \land x \notin B\}\vspace{-10pt}$$
\textbf{对称差}:$A$和$B$的对称差是由所有属于$A$或属于$B$但不同时属于两者的元素组成的集合\vspace{-10pt}
$$A\oplus B = (A - B) \cup (B -A) = (A \cup B)- (A \cap B)\vspace{-10pt}$$
\textbf{幂集}:集合$A$的幂集是由$A$的所有子集构成的集合,记作$P(A)$:\vspace{-10pt}
$$P(A) = \{X \mid X \subseteq A\}$$
\end{definition}
\begin{theorem}{容斥原理}{}
对于有限集合$A_1, A_2, \ldots, A_n$,有:
\begin{align*}
\left|\bigcup_{i=1}^n A_i\right| &= \sum_{i=1}^n |A_i| 
                               - \sum_{1 \leq i < j \leq n} |A_i \cap A_j| + \sum_{1 \leq i < j < k \leq n} |A_i \cap A_j \cap A_k| \\
                               &- \cdots + (-1)^{n-1} |A_1 \cap A_2 \cap \cdots \cap A_n|
\end{align*}
\end{theorem}
然后主要写一些差集,对称差,幂集的性质。
\begin{theorem}{差集的性质}{}
分配律:$A - (B \cup C) = (A - B) \cap (A - C)=(A - B) - C$\quad$A - (B \cap C) = (A - B) \cup (A - C)$\\
分配律:$(A \cup B) - C = (A - C) \cup (B - C)$\quad$(A \cap B) - C = (A - C) \cap (B - C)$\\
去括号:$A - (B - C) = (A - B) \cup (A \cap C)$\\
传递性:如果$A \subseteq B$,则$A - C \subseteq B - C$
\end{theorem}
第二条恒等变形就可以了比较简单,第四条也好理解,$x\in A\land x\notin C$显然蕴含$x\in B\land x\notin C$

第一条和第三条有点难,一个很霸道的方式是,对于全集$U$内的元素$x$,考虑真值表
\begin{center}
\begin{tabular}{ccc|cccccc}
$x \in A$ & $x \in B$ & $x \in C$ & $x \in (B \cup C)$ & $x \in (B - C)$ & $x \in A - B$ & $x \in A - C$ & $x \in A \cap C$ \\
\hline
0 & 0 & 0 & 0 & 0 & 0 & 0 & 0 \\
0 & 0 & 1 & 1 & 0 & 0 & 0 & 0 \\
0 & 1 & 0 & 1 & 1 & 0 & 0 & 0 \\
0 & 1 & 1 & 1 & 0 & 0 & 0 & 0 \\
1 & 0 & 0 & 0 & 0 & 1 & 1 & 0 \\
1 & 0 & 1 & 1 & 0 & 1 & 0 & 1 \\
1 & 1 & 0 & 1 & 1 & 0 & 1 & 0 \\
1 & 1 & 1 & 1 & 0 & 0 & 0 & 1 \\
\end{tabular}
\end{center}
然后,目标表达式比较
\begin{center}
\begin{tabular}{ccc|ccc}
$x \in A$ & $x \in B$ & $x \in C$ & $x \in A - (B \cup C)$ & $x \in (A - B) \cap (A - C)$ & $x \in (A - B) - C$ \\
\hline
0 & 0 & 0 & 0 & 0 & 0 \\
0 & 0 & 1 & 0 & 0 & 0 \\
0 & 1 & 0 & 0 & 0 & 0 \\
0 & 1 & 1 & 0 & 0 & 0 \\
1 & 0 & 0 & 1 & 1 & 1 \\
1 & 0 & 1 & 0 & 0 & 0 \\
1 & 1 & 0 & 0 & 0 & 0 \\
1 & 1 & 1 & 0 & 0 & 0 \\
\end{tabular}
\end{center}
从表2可以看出第4,5,6列的真值完全相同,因此:$A - (B \cup C) = (A - B) \cap (A - C) = (A - B) - C$
\begin{center}
\begin{tabular}{ccc|cc}
$x \in A$ & $x \in B$ & $x \in C$ & $x \in A - (B - C)$ & $x \in (A - B) \cup (A \cap C)$ \\
\hline
0 & 0 & 0 & 0 & 0 \\
0 & 0 & 1 & 0 & 0 \\
0 & 1 & 0 & 0 & 0 \\
0 & 1 & 1 & 0 & 0 \\
1 & 0 & 0 & 1 & 1 \\
1 & 0 & 1 & 1 & 1 \\
1 & 1 & 0 & 0 & 0 \\
1 & 1 & 1 & 1 & 1 \\
\end{tabular}
\end{center}
从表3可以看出,第4列和第5列的真值完全相同,因此$A - (B - C) = (A - B) \cup (A \cap C)$
\begin{theorem}{对称差的性质}{}
    结合律:$(A \oplus B) \oplus C = A \oplus (B \oplus C)$\\
    分配律:$A \cap (B \oplus C) = (A \cap B) \oplus (A \cap C)$\\
    差形式:$A \oplus B = (A \cup B) - (A \cap B)$\\
    取补集:$\sim(A \oplus B) = (\sim A \oplus B) = (A \oplus \sim B)$\\
    常见等价关系:$A \oplus B = \varnothing$ 当且仅当 $A = B$,$A \oplus B = A \oplus C$ 当且仅当 $B = C$\\
    消去律:$(A \oplus B) \oplus (B \oplus C) = A \oplus C$(可以推广)
\end{theorem}
证明:$(A \oplus B) \oplus C = A \oplus (B \oplus C)$
\begin{center}
\begin{tabular}{ccc|cccccc}
$A$ & $B$ & $C$ & $A \oplus B$ & $(A \oplus B) \oplus C$ & $B \oplus C$ & $A \oplus (B \oplus C)$ & 是否相等 \\
\hline
0 & 0 & 0 & 0 & 0 & 0 & 0 & 是 \\
0 & 0 & 1 & 0 & 1 & 1 & 1 & 是 \\
0 & 1 & 0 & 1 & 1 & 1 & 1 & 是 \\
0 & 1 & 1 & 1 & 0 & 0 & 0 & 是 \\
1 & 0 & 0 & 1 & 1 & 0 & 1 & 是 \\
1 & 0 & 1 & 1 & 0 & 1 & 0 & 是 \\
1 & 1 & 0 & 0 & 0 & 1 & 0 & 是 \\
1 & 1 & 1 & 0 & 1 & 0 & 1 & 是 \\
\end{tabular}
\end{center}
如果要求必须使用等价变形,我们依然可以使用真值表写出两者的主析取范式,然后比较。两者的的主析取范式完全相同,因此:$(A \oplus B) \oplus C = A \oplus (B \oplus C)$,具体的解答过程则是先将左右两边化为交并补运算,真值表可以起到检查是否正确的作用。
\begin{align*}
(A \oplus B) \oplus C = & (\sim A \cap \sim B \cap C) \cup 
                        (\sim A \cap B \cap \sim C) \cup (A \cap \sim B \cap \sim C) \cup 
                        (A \cap B \cap C)\\
A \oplus (B \oplus C) = & (\sim A \cap \sim B \cap C) \cup 
                         (\sim A \cap B \cap \sim C) \cup 
                         (A \cap \sim B \cap \sim C) \cup 
                         (A \cap B \cap C)
\end{align*}
证明:$A \cap (B \oplus C) = (A \cap B) \oplus (A \cap C)$
\begin{center}
\begin{tabular}{ccc|cccccc}
$A$ & $B$ & $C$ & $B \oplus C$ & $A \cap (B \oplus C)$ & $A \cap B$ & $A \cap C$ & $(A \cap B) \oplus (A \cap C)$ & 相等 \\
\hline
0 & 0 & 0 & 0 & 0 & 0 & 0 & 0 & 是 \\
0 & 0 & 1 & 1 & 0 & 0 & 0 & 0 & 是 \\
0 & 1 & 0 & 1 & 0 & 0 & 0 & 0 & 是 \\
0 & 1 & 1 & 0 & 0 & 0 & 0 & 0 & 是 \\
1 & 0 & 0 & 0 & 0 & 0 & 0 & 0 & 是 \\
1 & 0 & 1 & 1 & 1 & 0 & 1 & 1 & 是 \\
1 & 1 & 0 & 1 & 1 & 1 & 0 & 1 & 是 \\
1 & 1 & 1 & 0 & 0 & 1 & 1 & 0 & 是 \\
\end{tabular}
\end{center}
证明:$(A\cup B)\oplus(A\cup C)\subseteq A\cup(B\oplus C)$
\begin{center}
\begin{tabular}{ccc|cccccc}
$A$ & $B$ & $C$ & $B \oplus C$ & $A \cup (B \oplus C)$ & $A \cup B$ & $A \cup C$ & $(A \cup B) \oplus (A \cup C)$ & 蕴含关系 \\
\hline
0 & 0 & 0 & 0 & 0 & 0 & 0 & 0 & 相等 \\
0 & 0 & 1 & 1 & 1 & 0 & 1 & 1 & 相等 \\
0 & 1 & 0 & 1 & 1 & 1 & 0 & 1 & 相等 \\
0 & 1 & 1 & 0 & 0 & 1 & 1 & 0 & 相等 \\
1 & 0 & 0 & 0 & 1 & 1 & 1 & 0 & 左真右假 \\
1 & 0 & 1 & 1 & 1 & 1 & 1 & 0 & 左真右假 \\
1 & 1 & 0 & 1 & 1 & 1 & 1 & 0 & 左真右假 \\
1 & 1 & 1 & 0 & 1 & 1 & 1 & 0 & 左真右假 \\
\end{tabular}
\end{center}
可以发现,当全集内的某个元素$x$满足$x\in (A \cup B) \oplus (A \cup C)$时,$x$必然满足$x\in A\cup(B\oplus C)$\\
证明:$(A \oplus B) \oplus (B \oplus C) = A \oplus C$
\begin{center}
\begin{tabular}{ccc|cccccc}
$A$ & $B$ & $C$ & $A \oplus B$ & $B \oplus C$ & $(A \oplus B) \oplus (B \oplus C)$ & $A \oplus C$ & 是否相等 \\
\hline
0 & 0 & 0 & 0 & 0 & 0 & 0 & 是 \\
0 & 0 & 1 & 0 & 1 & 1 & 1 & 是 \\
0 & 1 & 0 & 1 & 1 & 0 & 0 & 是 \\
0 & 1 & 1 & 1 & 0 & 1 & 1 & 是 \\
1 & 0 & 0 & 1 & 0 & 1 & 1 & 是 \\
1 & 0 & 1 & 1 & 1 & 0 & 0 & 是 \\
1 & 1 & 0 & 0 & 1 & 1 & 1 & 是 \\
1 & 1 & 1 & 0 & 0 & 0 & 0 & 是 \\
\end{tabular}
\end{center}
然而,此题用真值表,是杀鸡用牛刀,其实最简单的方法是利用结合律:
\begin{align*}
(A_1 \oplus A_2) \oplus (A_2 \oplus A_3) = A_1 \oplus (A_2 \oplus A_2) \oplus A_3 = A_1 \oplus \varnothing \oplus A_3  = A_1 \oplus A_3 
\end{align*}
\begin{theorem}{对称差的代数结构}{}
    对于任意集合序列 $A_1, A_2, \ldots, A_n$,有\vspace{-10pt}
$$(A_1 \oplus A_2) \oplus (A_2 \oplus A_3) \oplus \cdots \oplus (A_{n-1} \oplus A_n) = A_1 \oplus A_n$$
\end{theorem}
对于任意形式的对称差链,只要每个中间项出现偶数次,就会相互抵消。例如:
$$(A \oplus B \oplus C) \oplus (B \oplus C \oplus D) = A \oplus D$$
因为通过结合律和交换律,$B$ 和 $C$ 实际上各出现两次,因此抵消。\\
证明:$\sim(A \oplus B) = (\sim A \oplus B)$
\begin{align*}
      \sim(A \oplus B) &= \sim((A\cap \sim B) \cup (\sim A \cap B))=(\sim A\cup B)\cap (A\cup \sim B) \\
      &=(\sim A\cap A)\cup(\sim A\cap\sim B)\cup(A\cap B)\cup(B\cap\sim B)\\
      &=(\sim A\cap\sim B)\cup(A\cap B)= (\sim A \oplus B)
\end{align*}
\begin{theorem}{幂集的性质}{}
设$X$是一个集合,$P(X)$表示$X$的幂集(即$X$的所有子集构成的集合)。幂集具有以下性质:
1. 如果$|X| = n$(有限),则$|P(X)| = 2^n$,$P(\varnothing) = \{\varnothing\}$,有$1 = 2^0$个元素\\
2. 单调性:如果$A \subseteq B$,则$P(A) \subseteq P(B)$\\
3. $P(A) \cup P(B) \subseteq P(A \cup B)$,当且仅当 $A \subseteq B$ 或 $B \subseteq A$取等\\
4. $P(A) \cap P(B) = P(A \cap B)$\\
5. $ P(A - B) \subseteq (P(A) - P(B))\cup\{\varnothing\}$
\end{theorem}
性质3的证明比较简单:\vspace{-10pt}
\[P(A) \cup P(B) \subseteq P(A \cup B)\cup P(A\cup B)=P(A\cup B)\vspace{-10pt}\]
分别放缩左右两侧即可,而且这样的操作还有一个好处,就是可以看清蕴含关系变成等价关系时的特殊条件(当且仅当 $A \subseteq B$ 或 $B \subseteq A$取等)。我们不妨再看一个例子:

证明或证伪:$A\subset B\land C\subset D\Rightarrow (A\cup C)\subset (B\cup D)$\\
解:考虑$B=A\cup C,D=A\cup C$,则
\vspace{-10pt}\[A\subset(A\cup C),C\subset(A\cup C)\Rightarrow (A\cup C)=(B\cup D)\vspace{-10pt}\]显然该命题错误,改成$A\subset B\land C\subset D\Rightarrow (A\cup C)\subseteq (B\cup D)$即可。

证明或证伪:$A\subset B\land C\subset D\Rightarrow (A\cap C)\subset (B\cap D)$\\
解:同样错误,考虑$A\cap C\subset A,C-A\ne\varnothing,B\cap D\subset B,D-B\ne\varnothing$,则转化为\vspace{-10pt}\[(A\cap C)\cap(B\cap D)=(A\cap B)\cap(C\cap D)\subset A\cap B\vspace{-10pt}\]
显然,这里$(A\cap B)=(C\cap D)$完全是可能的,所以仍然需要改成$\subseteq$

性质4显然,一个集合既是$A$的子集,又是$B$的子集,那么其元素都在$A,B$中,即在$A\cap B$中,反过来,一个集合的元素在$A\cap B$中,那么也在$A,B$中,故既是$A$的子集又是$B$的子集。

下面证明性质5:$ P(A - B) \subseteq (P(A) - P(B))\cup\{\varnothing\}$
\begin{align*}
P(A - B) &= \{x|x\subseteq (A\cap\sim B)\}=\{x|x\subseteq A\land x\subseteq\sim B\}\\
&\subseteq\{x|x\subseteq A\land x\not\subseteq B\}=(P(A)-P(B))\cup\{\varnothing\}
\end{align*}
需要说明的是,涉及到幂集的性质的有关证明不可以使用真值表,或者说用真值表比较麻烦,因为我们考虑集合之间的关系的时候,由于集合可能有不止一个元素,所以$x$中的元素全在$A$中的反面,不是“$x$中的元素全不在$A$中”,而是“$x$中的元素不全在$A$中”。这就导致了\vspace{-10pt}\[x\not\subseteq A\land x\not\subseteq B\Leftarrow x\not\subseteq A\cup B\vspace{-10pt}\]和之前的情形大不相同,此时要分多种情形讨论,复杂度明显升高。但是,真值表还是可以帮助我们进行集合的等价运算,如:
\begin{example}{}{}
\[(A\cap B\cap C)\cup(\sim A\cap B\cap C)\cup(A\cap B\cap\sim C)\]\end{example}
\begin{solution}设全集为$U$,观察到三个并集中都有$B$,所以合并:
    \begin{align*}
& (A \cap B \cap C) \cup (\sim A \cap B \cap C) \cup (A \cap B \cap \sim C) \\
&= [B \cap (A \cap C)] \cup [B \cap (\sim A \cap C)] \cup [B \cap (A \cap \sim C)] \quad \text{(结合律)} \\
&= B \cap [(A \cap C) \cup (\sim A \cap C) \cup (A \cap \sim C)] \quad \text{(分配律)} \\
&= B \cap \{[(A \cup \sim A) \cap C] \cup (A \cap \sim C)\} \quad \text{(分配律)} \\
&= B \cap [(U \cap C) \cup (A \cap \sim C)] \quad \text{(补集律:$A \cup \sim A = U$)} \\
&= B \cap [C \cup (A \cap \sim C)] \quad \text{(同一律:$U \cap C = C$)} \\
&= B \cap [(C \cup A) \cap (C \cup \sim C)] \quad \text{(分配律)} \\
&= B \cap [(A \cup C) \cap U] \quad \text{(交换律和补集律)} \\
&= B \cap (A \cup C) \quad \text{(同一律)}
\end{align*}
另一种方式就是使用真值表:我们通过分析元素 $x$ 的属于关系来简化该表达式。定义真值表,其中 1 表示 $x$ 属于集合,0 表示 $x$ 不属于集合:
\begin{center}
\begin{tabular}{ccc|cccc}
$x \in A$ & $x \in B$ & $x \in C$ & $x \in A \cap B \cap C$ & $x \in \sim A \cap B \cap C$ & $x \in A \cap B \cap \sim C$ & $x \in \text{整体表达式}$ \\
\hline
0 & 0 & 0 & 0 & 0 & 0 & 0 \\
0 & 0 & 1 & 0 & 0 & 0 & 0 \\
0 & 1 & 0 & 0 & 0 & 0 & 0 \\
0 & 1 & 1 & 0 & 1 & 0 & 1 \\
1 & 0 & 0 & 0 & 0 & 0 & 0 \\
1 & 0 & 1 & 0 & 0 & 0 & 0 \\
1 & 1 & 0 & 0 & 0 & 1 & 1 \\
1 & 1 & 1 & 1 & 0 & 0 & 1 \\
\end{tabular}
\end{center}
从真值表可知,$x$ 属于整体表达式当且仅当满足以下条件:$x \in B$ 为真, 同时 $x \in A$ 或 $x \in C$ 为真,这等价于 $x \in B \cap (A \cup C)$。

伪装的方式是:\\
设$x$是全集中的任意元素,我们分析$x$属于该并集的条件。\\
若$x \in A \cap B \cap C$,则$x \in A$,$x \in B$,$x \in C$\\
若$x \in \sim A \cap B \cap C$,则$x \notin A$,$x \in B$,$x \in C$\\
若$x \in A \cap B \cap \sim C$,则$x \in A$,$x \in B$,$x \notin C$\\
综合三种情况,$x$属于该并集当且仅当同时满足:$x \in B$,$x \in A$ 或 $x \in C$(即$x \in A \cup C$),因此,$x$属于该并集当且仅当$x \in B \cap (A \cup C)$
\end{solution}
这种等价变形的题目比较简单,首先化成交并补,然后只需要观察重复出现的结构(有的时候还需要进行适当的换元)接着等价运算。题目一难或者看不出来了,就利用真值表来证明。
\section{集合的笛卡尔积}
\begin{definition}{集合的笛卡尔积}{}
设$A$和$B$是两个集合,$A$与$B$的\textbf{笛卡尔积}(或称直积)定义为所有有序对$(a,b)$组成的集合,其中$a \in A$,$b \in B$,记作$A \times B$:\vspace{-10pt}
$$A \times B = \{(a,b) \mid a \in A, b \in B\}\vspace{-10pt}$$
\textbf{推广到$n$个集合}:对于$n$个集合$A_1, A_2, \ldots, A_n$,它们的笛卡尔积定义为所有$n$元有序组$(a_1, a_2, \ldots, a_n)$组成的集合,其中$a_i \in A_i$:
\vspace{-10pt}$$A_1 \times A_2 \times \cdots \times A_n = \{(a_1, a_2, \ldots, a_n) \mid a_i \in A_i, i = 1,2,\ldots,n\}\vspace{-10pt}$$
\textbf{特殊情形}:\\
~~~~当$A = B$时,$A \times A$可简写为$A^2$\\
~~~~$n$个相同集合$A$的笛卡尔积可简写为$A^n$\\
~~~~实数集$\mathbb{R}$与自身的笛卡尔积$\mathbb{R}^2$表示平面直角坐标系\\
~~~~空集与任何集合的笛卡尔积为空集:$\varnothing \times A = A \times \varnothing = \varnothing$
\end{definition}


