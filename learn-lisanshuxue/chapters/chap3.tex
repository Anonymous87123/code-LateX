\chapter{集合与关系}
\section{集合}
\begin{definition}{空集,全集,有限集,无限集,元素个数}{}
\textbf{空集}:不含任何元素的集合,记作$\nothing$或$\{x|x\ne x\}$。空集是任何集合的子集。空集是唯一的\\
\textbf{全集}:特定讨论中包含所有对象的集合,记作$U$。全集是相对的,取决于讨论的上下文。\\
\textbf{有限集}:元素个数有限的集合。如果集合$A$有$n$个元素,其中$n$是非负整数,则$A$是有限集。\\
\textbf{无限集}:不是有限集的集合,即元素个数无限多的集合。如自然数集$\mathbb{N}$、实数集$\mathbb{R}$等。\\
\textbf{元素个数}:有限集$A$中元素的数目,记作$|A|$或$card(A)$。无限集的元素个数用基数概念描述,如可数无限集、不可数无限集等。
\end{definition}
\begin{definition}{集合的包含关系}{}
    \textbf{子集}:如果集合$A$中的每个元素都是集合$B$中的元素,则称$A$是$B$的\textbf{子集},记作$A \subseteq B$。空集是空集的子集\\
    \textbf{真子集}:如果$A \subseteq B$且$A \neq B$,则称$A$是$B$的\textbf{真子集},记作$A \subset B$。即\vspace{-10pt}
    \[\forall x(A(x) \rightarrow B(x)) \land \exists x(B(x) \land \neg A(x))\]
\end{definition}
\begin{definition}{集合的基本运算}{}
\textbf{交集}:集合$A$和$B$的交集是由所有既属于$A$又属于$B$的元素组成的集合,记作$A \cap B$。\vspace{-10pt}
$$A \cap B = \{x \mid x \in A \land x \in B\}\vspace{-10pt}$$
\textbf{并集}:集合$A$和$B$的并集是由所有属于$A$或属于$B$的元素组成的集合,记作$A \cup B$。\vspace{-10pt}
$$A \cup B = \{x \mid x \in A \lor x \in B\}\vspace{-10pt}$$
\textbf{补集}:集合$A$的补集是由全集中所有不属于$A$的元素组成的集合,记作$\sim  A$。\vspace{-10pt}
$$\sim A = \{x \in U \mid x \notin A\}\vspace{-10pt}$$
\textbf{差集}:集合$A$与$B$的差集是由所有属于$A$但不属于$B$的元素组成的集合:\vspace{-10pt}
$$A-B = \{x \mid x \in A \land x \notin B\}\vspace{-10pt}$$
\textbf{对称差}:$A$和$B$的对称差是由所有属于$A$或属于$B$但不同时属于两者的元素组成的集合\vspace{-10pt}
$$A\oplus B = (A - B) \cup (B -A) = (A \cup B)- (A \cap B)\vspace{-10pt}$$
\textbf{幂集}:集合$A$的幂集是由$A$的所有子集构成的集合,记作$P(A)$:\vspace{-10pt}
$$P(A) = \{X \mid X \subseteq A\}$$
\end{definition}
\begin{theorem}{容斥原理}{}
对于有限集合$A_1, A_2, \ldots, A_n$,有:
\begin{align*}
\left|\bigcup_{i=1}^n A_i\right| &= \sum_{i=1}^n |A_i| 
                               - \sum_{1 \leq i < j \leq n} |A_i \cap A_j| + \sum_{1 \leq i < j < k \leq n} |A_i \cap A_j \cap A_k| \\
                               &- \cdots + (-1)^{n-1} |A_1 \cap A_2 \cap \cdots \cap A_n|
\end{align*}
\end{theorem}
然后主要写一些差集,对称差,幂集的性质。
\begin{theorem}{差集的性质}{}
分配律:$A - (B \cup C) = (A - B) \cap (A - C)=(A - B) - C$\quad$A - (B \cap C) = (A - B) \cup (A - C)$\\
分配律:$(A \cup B) - C = (A - C) \cup (B - C)$\quad$(A \cap B) - C = (A - C) \cap (B - C)$\\
去括号:$A - (B - C) = (A - B) \cup (A \cap C)$\\
传递性:如果$A \subseteq B$,则$A - C \subseteq B - C$
\end{theorem}
第二条恒等变形就可以了比较简单,第四条也好理解,$x\in A\land x\notin C$显然蕴含$x\in B\land x\notin C$

第一条和第三条有点难,一个很霸道的方式是,对于全集$U$内的元素$x$,考虑真值表
\begin{center}
\begin{tabular}{ccc|cccccc}
$x \in A$ & $x \in B$ & $x \in C$ & $x \in (B \cup C)$ & $x \in (B - C)$ & $x \in A - B$ & $x \in A - C$ & $x \in A \cap C$ \\
\hline
0 & 0 & 0 & 0 & 0 & 0 & 0 & 0 \\
0 & 0 & 1 & 1 & 0 & 0 & 0 & 0 \\
0 & 1 & 0 & 1 & 1 & 0 & 0 & 0 \\
0 & 1 & 1 & 1 & 0 & 0 & 0 & 0 \\
1 & 0 & 0 & 0 & 0 & 1 & 1 & 0 \\
1 & 0 & 1 & 1 & 0 & 1 & 0 & 1 \\
1 & 1 & 0 & 1 & 1 & 0 & 1 & 0 \\
1 & 1 & 1 & 1 & 0 & 0 & 0 & 1 \\
\end{tabular}
\end{center}
然后,目标表达式比较
\begin{center}
\begin{tabular}{ccc|ccc}
$x \in A$ & $x \in B$ & $x \in C$ & $x \in A - (B \cup C)$ & $x \in (A - B) \cap (A - C)$ & $x \in (A - B) - C$ \\
\hline
0 & 0 & 0 & 0 & 0 & 0 \\
0 & 0 & 1 & 0 & 0 & 0 \\
0 & 1 & 0 & 0 & 0 & 0 \\
0 & 1 & 1 & 0 & 0 & 0 \\
1 & 0 & 0 & 1 & 1 & 1 \\
1 & 0 & 1 & 0 & 0 & 0 \\
1 & 1 & 0 & 0 & 0 & 0 \\
1 & 1 & 1 & 0 & 0 & 0 \\
\end{tabular}
\end{center}
从表2可以看出第4,5,6列的真值完全相同,因此:$A - (B \cup C) = (A - B) \cap (A - C) = (A - B) - C$
\begin{center}
\begin{tabular}{ccc|cc}
$x \in A$ & $x \in B$ & $x \in C$ & $x \in A - (B - C)$ & $x \in (A - B) \cup (A \cap C)$ \\
\hline
0 & 0 & 0 & 0 & 0 \\
0 & 0 & 1 & 0 & 0 \\
0 & 1 & 0 & 0 & 0 \\
0 & 1 & 1 & 0 & 0 \\
1 & 0 & 0 & 1 & 1 \\
1 & 0 & 1 & 1 & 1 \\
1 & 1 & 0 & 0 & 0 \\
1 & 1 & 1 & 1 & 1 \\
\end{tabular}
\end{center}
从表3可以看出,第4列和第5列的真值完全相同,因此$A - (B - C) = (A - B) \cup (A \cap C)$
\begin{theorem}{对称差的性质}{}
    
\end{theorem}
