\chapter{图论}
\section{图的基本概念}
\begin{definition}{有向图,无向图,平行边,邻接,环,孤立点,简单图}{}
1. 一个\textbf{无向图}可以被表示为$G = (V, E)$,其中$V$是一个非空集合(有限),称为\textbf{顶点集},其元素称为\textbf{顶点}或\textbf{节点},$E$是$V$中元素的无序对构成的集合,称为\textbf{边集},即$E \subseteq \{\{u,v\} \mid u,v \in V, u \neq v\}$\\
2. 一个\textbf{有向图}可以被表示为$G = (V, E)$,其中$V$是一个非空集合(有限),称为\textbf{顶点集},其元素称为\textbf{顶点}或\textbf{节点},$E$是$V$中元素的有序对构成的集合,称为\textbf{边集},即$E \subseteq \{(u,v) \mid u,v \in V, u \neq v\}$\\
3.  \textbf{平行边}:在无向图中,连接同一对顶点的多条边称为平行边,在有向图中,同一方向连接同一对顶点的多条弧称为平行边。\\
4. \textbf{邻接}:在无向图中,如果存在边$e = \{u,v\}$,则称顶点$u$和$v$是邻接的(相邻的)\\
5. 环:如果一条边的两个端点关联于同一个结点,称为环。\\
6. \textbf{孤立点}:在无向图中,如果一个顶点不与任何其他顶点相邻,则称它为孤立点。
\end{definition}
\begin{definition}{度的概念}{}
\textbf{度数}(无向图):顶点$v$的度数(简称度)$\deg(v)$是与该顶点相关联的边的数量,自环通常算作2度(因为连接同一顶点的两个端点)\\
\textbf{入度和出度}(有向图):顶点$v$的入度$\deg^-(v)$是以$v$为弧头的弧的数量,顶点$v$的出度$\deg^+(v)$是以$v$为弧尾的弧的数量,顶点$v$的总度数$\deg(v) = \deg^-(v) + \deg^+(v)$\\
\textbf{悬挂结点}:度数为1的顶点称为悬挂结点,在有向图中,通常指总度数为1的顶点\\
\textbf{悬挂边}:与悬挂结点相关联的边称为悬挂边。\\
\textbf{最大度和最小度}:图$G$的最大度$\Delta(G) = \max\{\deg(v) \mid v \in V(G)\}$,图$G$的最小度$\delta(G) = \min\{\deg(v) \mid v \in V(G)\}$\\
\textbf{最大入度和最小入度}(有向图):图$D$的最大入度$\Delta^-(D) = \max\{\deg^-(v) \mid v \in V(D)\}$,图$D$的最小入度$\delta^-(D) = \min\{\deg^-(v) \mid v \in V(D)\}$\\
\textbf{最大出度和最小出度}(有向图):图$D$的最大出度$\Delta^+(D) = \max\{\deg^+(v) \mid v \in V(D)\}$,图$D$的最小出度$\delta^+(D) = \min\{\deg^+(v) \mid v \in V(D)\}$
\end{definition}
\begin{definition}{图的基本分类}{}
\textbf{简单图}:在无向图中,如果不存在平行边,则称它为简单图。\\
\textbf{多重图}:如果一个图存在平行边,则称它为多重图。\\
\textbf{$n$阶图}:具有$n$个顶点的图称为$n$阶图。\\
\textbf{零图}:边集为空(没有边,只有结点)的图称为零图,即所有顶点都是孤立点。\\
\textbf{平凡图}:只有一个顶点且没有边的图称为平凡图,是最简单的非空图(一阶零图)。\\
\textbf{空图}:顶点集为空的图称为空图(通常不考虑)。\\
\textbf{完全图}:任意两不同顶点之间都恰有一条边的简单图称为完全图。$n$阶完全图记作$K_n$。\\
\textbf{二分图}:顶点集$V$可以划分为两个不相交子集$V_1$和$V_2$,使得每条边的一个端点在$V_1$中,另一个端点在$V_2$中,称为二分图。\\
\textbf{正则图}:所有顶点度数都相同的图称为正则图。若每个顶点的度数均为$k$,称为$k$-正则图。\\
\textbf{环图}:$n$个顶点$v_1,v_2,\ldots,v_n$依次连接形成的环状图称为环图,记作$C_n$。\\
\textbf{轮图}:在环图$C_{n-1}$中添加一个顶点,并将该顶点与环图中所有顶点相连,称为轮图$W_n$。\\
\textbf{$n$方体图}:用$n$维超立方体的顶点和边构成的图称为$n$方体图,记作$Q_n$。顶点集为所有长度为$n$的二进制串,两个顶点相邻当且仅当它们的二进制表示恰好有一位不同。
\end{definition}
\section{特殊图}
\chapter{树}
