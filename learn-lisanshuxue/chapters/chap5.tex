\chapter{图论}
\section{图的基本概念}
\begin{definition}{有向图,无向图,平行边,邻接,环,孤立点,简单图}{}
1. 一个\textbf{无向图}可以被表示为$G = (V, E)$,其中$V$是一个非空集合(有限),称为\textbf{顶点集},其元素称为\textbf{顶点}或\textbf{节点},$E$是$V$中元素的无序对构成的集合,称为\textbf{边集},即$E \subseteq \{\{u,v\} \mid u,v \in V, u \neq v\}$\\
2. 一个\textbf{有向图}可以被表示为$G = (V, E)$,其中$V$是一个非空集合(有限),称为\textbf{顶点集},其元素称为\textbf{顶点}或\textbf{节点},$E$是$V$中元素的有序对构成的集合,称为\textbf{边集},即$E \subseteq \{(u,v) \mid u,v \in V, u \neq v\}$\\
3.  \textbf{平行边}:在无向图中,连接同一对顶点的多条边称为平行边,在有向图中,同一方向连接同一对顶点的多条弧称为平行边。\\
4. \textbf{邻接}:在无向图中,如果存在边$e = \{u,v\}$,则称顶点$u$和$v$是邻接的(相邻的)\\
5. \textbf{环}:如果一条边的两个端点关联于同一个结点,称为环。\\
6. \textbf{孤立点}:在无向图中,如果一个顶点不与任何其他顶点相邻,则称它为孤立点。
\end{definition}
不管是无向图还是有向图,边集允许重复的元素多次出现,即在一对结点之间存在多条边。如果关联一对结点的有向边的方向相同且有多个,则称为平行边。
\begin{definition}{度的概念}{}
\textbf{度数}(无向图):顶点$v$的度数(简称度)$\deg(v)$是与该顶点相关联的边的数量,自环通常算作2度(因为连接同一顶点的两个端点)\\
\textbf{入度和出度}(有向图):顶点$v$的入度$\deg^-(v)$是以$v$为弧头的弧的数量,顶点$v$的出度$\deg^+(v)$是以$v$为弧尾的弧的数量,顶点$v$的总度数$\deg(v) = \deg^-(v) + \deg^+(v)$\\
\textbf{悬挂结点}:度数为1的顶点称为悬挂结点,在有向图中,通常指总度数为1的顶点\\
\textbf{悬挂边}:与悬挂结点相关联的边称为悬挂边。\\
\textbf{最大度和最小度}:图$G$的最大度$\Delta(G) = \max\{\deg(v) \mid v \in V(G)\}$,图$G$的最小度$\delta(G) = \min\{\deg(v) \mid v \in V(G)\}$\\
\textbf{最大入度和最小入度}(有向图):图$D$的最大入度$\Delta^-(D) = \max\{\deg^-(v) \mid v \in V(D)\}$,图$D$的最小入度$\delta^-(D) = \min\{\deg^-(v) \mid v \in V(D)\}$\\
\textbf{最大出度和最小出度}(有向图):图$D$的最大出度$\Delta^+(D) = \max\{\deg^+(v) \mid v \in V(D)\}$,图$D$的最小出度$\delta^+(D) = \min\{\deg^+(v) \mid v \in V(D)\}$
\end{definition}
\begin{definition}{图的基本分类}{}
\textbf{带权图}:每个结点或者每条边都带有数值的图称为带权图。\\
\textbf{简单图}:如果一个图不存在平行边也不存在环,则称它为简单图。\\
\textbf{多重图}:如果一个图存在平行边,则称它为多重图。\\
\textbf{$n$阶图}:具有$n$个顶点的图称为$n$阶图。\\
\textbf{零图}:边集为空(没有边,只有结点)的图称为零图,即所有顶点都是孤立点。\\
\textbf{平凡图}:只有一个顶点且没有边的图称为平凡图,是最简单的非空图(一阶零图)。\\
\textbf{空图}:顶点集为空的图称为空图(通常不考虑)。\\
\textbf{完全图}:任意两不同顶点之间都恰有一条边的简单图称为完全图。$n$阶完全图记作$K_n$。\\
\textbf{二分图}:顶点集$V$可以划分为两个不相交子集$V_1$和$V_2$,使得每条边的一个端点在$V_1$中,另一个端点在$V_2$中,称为二分图。\\
\textbf{完全二分图}:顶点集$V$可以划分为两个不相交子集$V_1$和$V_2$,使得每条边的一个端点在$V_1$中,另一个端点在$V_2$中,且 $V_1$ 中的每个顶点都与 $V_2$ 中的每个顶点相连,且图中没有其他边,则称 $G$ 为完全二分图,记作 $K_{m,n}$,其中 $m = |X|$,$n = |Y|$
\textbf{正则图}:所有顶点度数都相同的图称为正则图。若每个顶点的度数均为$k$,称为$k$-正则图。\\
\textbf{环图}:$n$个顶点$v_1,v_2,\ldots,v_n$依次连接形成的环状图称为环图,记作$C_n$。\\
\textbf{轮图}:在环图$C_{n-1}$中添加一个顶点,并将该顶点与环图中所有顶点相连,称为轮图$W_n$。\\
\textbf{$n$方体图}:用$n$维超立方体的顶点和边构成的图称为$n$方体图,记作$Q_n$。顶点集为所有长度为$n$的二进制串,两个顶点相邻当且仅当它们的二进制表示恰好有一位不同。
\end{definition}
\begin{theorem}{图的基本定理}{}
握手定理:设 $G = (V, E)$ 是一个无向图,则所有顶点的度数之和满足$\sum_{v \in V} \deg(v) = 2|E|$
\end{theorem}
\textbf{证明:} 由于每条边连接两个顶点,每条边在度数求和中被计算了两次(一次为每个端点)。因此,总度数之和等于边数的两倍。具体地,对于每条边 $e = \{u, v\} \in E$,它贡献 1 到 $\deg(u)$ 和 1 到 $\deg(v)$,所以在求和 $\sum_{v \in V} \deg(v)$ 中,每条边恰好被计算两次。故
\[
\sum_{v \in V} \deg(v) = 2|E|
\]
\begin{theorem}{有向图握手定理}{}
设 $D = (V, A)$ 是有向图,则所有顶点的入度之和等于所有顶点的出度之和,且都等于边数:
\[
\sum_{v \in V} \deg^-(v) = \sum_{v \in V} \deg^+(v) = |A|
\]
\end{theorem}
\textbf{证明:} 每条有向边有一个起点和一个终点,为起点的出度贡献 1,为终点的入度贡献 1。因此,总入度之和等于总出度之和,都等于边数。
\begin{theorem}{握手定理推论}{}
任意图中,度数为奇数的顶点个数为偶数。\\
设 $G$ 是 $n$ 阶无向图,则 $G$ 中最大度数满足 $\Delta(G) \leq n-1$。\\
在任意无向图中,所有顶点的平均度数为 $\dfrac{2|E|}{|V|}$。\\
如果无向图 $G$ 是 $k$-正则图(每个顶点度数均为 $k$),则 $k|V| = 2|E|$
\end{theorem}
设 $V_1$ 为度数为奇数的顶点集合,$V_2$ 为度数为偶数的顶点集合。由握手定理,总度数和为偶数。$V_2$ 中顶点度数之和为偶数,故 $V_1$ 中顶点度数之和也必为偶数,因此 $|V_1|$ 必为偶数。

\begin{theorem}{可简单图化的判定定理(Havel-Hakimi算法)}{}
一个非负整数序列 $d = (d_1, d_2, \ldots, d_n)$(其中 $d_1 \geq d_2 \geq \cdots \geq d_n$,和为偶数)是可简单图化的,当且仅当可以通过\textbf{Havel-Hakimi算法}判定:

1.将序列按非递增顺序排序\\
2.如果序列全为零,则是可简单图化的\\
3.如果序列中有负数,或最大度数 $d_1 \geq n$,则不可简单图化\\
4.否则,删除第一个元素 $d_1$,然后将接下来的 $d_1$ 个元素每个减1\\
5.返回步骤1,重复上述过程

如果序列可简单图化,那么存在一个简单图以该序列为度序列,如果在某一步出现负数,说明无法构造满足条件的简单图,如果算法最终得到全零序列,则原序列是可简单图化的
\end{theorem}
\textbf{示例:}
判定序列 $(3, 3, 2, 2, 2)$ 是否可简单图化:
\begin{align*}
& (3, 3, 2, 2, 2) \quad \text{删除3,后面3个数各减1} \\
\to & (2, 1, 1, 2) \quad \text{重新排序为}(2, 2, 1, 1) \\
\to & (2, 2, 1, 1) \quad \text{删除2,后面2个数各减1} \\
\to & (1, 0, 1) \quad \text{重新排序为}(1, 1, 0) \\
\to & (1, 1, 0) \quad \text{删除1,后面1个数减1} \\
\to & (0, 0) \quad \text{全为零,可简单图化}
\end{align*}
\begin{definition}{补图}{}
设 $G = (V, E)$ 是一个简单无向图(无自环和无重边),其顶点集 $V$ 有 $n$ 个顶点。令 $K_n$ 表示完全图,其顶点集为 $V$,边集为 $E(K_n)$(即所有可能的无序顶点对)。则图 $G$ 的补图记为 $\sim G$ :
\[
\sim G = (V, E(K_n) -E)
\]
换句话说,$\sim G$ 的边集由 $K_n$ 中所有不在 $G$ 中的边组成。
\end{definition}
\begin{definition}{图的基本运算}{}

\textbf{1. 删除边运算}
设 $G = (V, E)$ 是一个图,$e \in E$ 是 $G$ 的一条边。从 $G$ 中删除边 $e$ 得到的图记为 $G - e$,定义为:
\[
G - e = (V, E - \{e\})
\]
即顶点集不变,仅从边集中移除边 $e$。

\textbf{2. 删除边集运算}
设 $G = (V, E)$ 是一个图,$F \subseteq E$ 是 $G$ 的一个边子集。从 $G$ 中删除边集 $F$ 得到的图记为 $G - F$,定义为:
\[
G - F = (V, E- F)
\]
即顶点集不变,从边集中移除所有属于 $F$ 的边。

\textbf{3. 删除结点运算}
设 $G = (V, E)$ 是一个图,$v \in V$ 是 $G$ 的一个顶点。从 $G$ 中删除顶点 $v$ 得到的图记为 $G - v$,定义为:
\[
G - v = (V - \{v\}, E-\{e \in E \mid v \in e\})
\]
即从顶点集中移除 $v$,同时移除所有与 $v$ 相关联的边。

\textbf{4. 删除结点集运算}
设 $G = (V, E)$ 是一个图,$U \subseteq V$ 是 $G$ 的一个顶点子集。从 $G$ 中删除顶点集 $U$ 得到的图记为 $G - U$,定义为:
\[
G - U = (V - U, E - \{e \in E \mid e \cap U \neq \varnothing\})
\]
即从顶点集中移除所有属于 $U$ 的顶点,同时移除所有与 $U$ 中顶点相关联的边。

\textbf{5. 边的收缩运算}
设 $G = (V, E)$ 是一个图,$e = uv \in E$ 是 $G$ 的一条边。收缩边 $e$ 得到的图记为 $G/e$,定义为:
将顶点 $u$ 和 $v$ 合并为一个新顶点 $w$,所有与 $u$ 或 $v$ 相关联的边(除 $e$ 外)改为与 $w$ 相关联,删除边 $e$,如果合并后产生重边,通常保留一条(简单图)或全部保留(多重图)。形式化地,$G/e = (V', E')$,其中:
\[
V' = (V - \{u, v\}) \cup \{w\}, \quad E' = E -\{e\} 
\]

\textbf{6. 加新边运算}
设 $G = (V, E)$ 是一个图,$e = (u,v)$ 是连接 $G$ 中两个顶点 $u, v \in V$ 但 $e \notin E$ 的边。向 $G$ 中添加新边 $e$ 得到的图记为 $G + e$,定义为:
\[
G + e = (V, E \cup \{e\})
\]
即顶点集不变,向边集中添加新边 $e$。
\end{definition}
注意,删除边通常会保留结点,但是删除结点通常不会保留边,加新边不会引入新的结点,边的收缩会影响结点的数量。
\begin{definition}{子图及相关概念}{}
\textbf{1. 子图}
设图 $G = (V, E)$ 和图 $H = (V', E')$。如果 $V' \subseteq V$ 且 $E' \subseteq E$,并且 $H$ 中每条边的端点都在 $V'$ 中,则称 $H$ 是 $G$ 的子图,记作 $H \subseteq G$。

\textbf{2. 母图}
如果 $H$ 是 $G$ 的子图,则称 $G$ 是 $H$ 的母图。

\textbf{3. 真子图}
如果 $H$ 是 $G$ 的子图,且 $H \neq G$(即 $V' \subset V$ 或 $E' \subset E$),则称 $H$ 是 $G$ 的真子图。

\textbf{4. 生成子图}
如果 $H$ 是 $G$ 的子图,且 $V' = V$(即顶点集相同),则称 $H$ 是 $G$ 的生成子图。

\textbf{5. 导出的子图记作}
导出子图分为两类:\\
\textbf{结点集导出子图}:对于 $G$ 的顶点子集 $V_1 \subseteq V$,由 $S$ 导出的子图记作 $G(V_1)$,其顶点集为 $S$,边集为 $G$ 中所有端点都在 $S$ 中的边,即 $E(G(V_1)) = \{ e \in E \mid e \subseteq S \}$。\\
\textbf{边集导出子图}:对于 $G$ 的边子集 $F \subseteq E$,由 $F$ 导出的子图记作 $G[F]$,其顶点集为 $F$ 中所有边的端点的并集,边集为 $F$。
\end{definition}
注意,生成子图保留了全部的结点,以及某些边(可选)。但是结点集导出子图仅保留部分的结点,而其边集则是由其保留的结点来确定。边集导出子图仅保留部分的边,而其顶点集则是由其保留的边来确定。

再加上上面介绍的删除结点,删除边等运算,下面讨论它们的区别和联系。

$G(V-V_1)$是由结点集$V-V_1$导出的子图,也就是在图$G$中删除结点集$V_1$以及所有它们关联的边所得到的子图。所以可以将$G(V-V_1)$视作$G-V_1$。而$G(E-E_1)$是由边集$E-E_1$导出的子图,也就是在图$G$中删除边集$E_1$以及里面的所有边所导出的子图,反观$G-E_1$,\textbf{由于删除边集时要保留结点不可以删除},所以删除边集运算不会改变结点数量,而删除边集导出的子图会改变结点数量,因为导出过程关注边(及其关联的结点)而非可能存在的孤立结点,所以要注意区别$G(E-E_1)$和$G-E_1$。
\begin{definition}{图的同构,自互补图}{}
设 $G = (V, E)$ 和 $H = (U, F)$ 是两个简单图(无向图或有向图)。如果存在一个双射函数 $f: V \to U$,使得对于 $G$ 中的任意两个顶点 $u, v \in V$:\vspace{-10pt}
\[
(u,v) \in E(G) \iff (f(u), f(v)) \in E(H)\vspace{-10pt}
\]
则称图 $G$ 和图 $H$ 是同构的,记作 $G \cong H$,并称 $f$ 是 $G$ 到 $H$ 的一个同构映射。\\
\textbf{自互补图}:如果一个图与自身的补图同构,则称该图为自互补图。
\end{definition}
\begin{theorem}{同构的不变量}{}
如果两个图同构,则它们必须共享以下性质(同构不变量):\\
顶点数相同,边数相同,结点度数序列相同(不计顺序),连通分支数和每个分支的大小相同,围长(最短环长度)相同,直径相同,色数相同,特征多项式相同(邻接矩阵的特征多项式)\\
\textbf{注意:}同构关系是等价关系(自反、对称、传递)
\end{theorem}
\begin{theorem}{自互补图的性质}{}
设 $G$ 是一个 $n$ 阶自互补图(即 $G \cong \sim G$),则以下性质成立:\\
自互补图对应的完全图边数为偶数\\
顶点数性质:$n \equiv 0 \pmod{4}$ 或 $n \equiv 1 \pmod{4}$\\
边数性质:$G$ 的边数为 $\dfrac{n(n-1)}{4}$\\
正则性:如果 $d$ 是 $G$ 中某个顶点的度数,则 $n-1-d$ 也必须是 $G$ 中某个顶点的度数\\
直径性质:自互补图的直径最大为 3(直径的定义后文会给出)\\
连通性:所有自互补图都是连通的
\end{theorem}
边数性质:由于 $G$ 和 $\sim G$ 同构,它们边数相等,所以自互补图对应的完全图边数为偶数。完全图 $K_n$ 有 $\frac{n(n-1)}{2}$ 条边,这些边被平分给 $G$ 和 $\sim G$,所以 $G$ 的边数为 $\frac{n(n-1)}{4}$,这必须是整数,故 $n(n-1)$ 必须被 4 整除。

在同构映射下,度数为 $d$ 的顶点映射到度数为 $n-1-d$ 的顶点。如果直径大于3,则存在距离为4的顶点对,在补图中距离为1,矛盾。如果不连通,则补图连通,但自互补图必须与补图有相同的连通性。
\begin{example}{}{}
    证明:6个人聚在一起,必然有3个人互相认识彼此(双向认识),或者至少有3个人互相都不认识彼此(双向不认识)。
\end{example}
将6个人视为6个顶点,构建一个完全图 $K_6$。定义图 $G$:顶点表示人,边表示两人互相认识。则补图 $\sim G$ 表示两人互相不认识。问题转化为证明:要么 $G$ 包含一个三角形(即3个顶点两两相连),要么 $\sim G$ 包含一个三角形。

考虑任意顶点 $v$。在完全图 $K_6$ 中,$v$ 与其他5个顶点相连。由于边要么在 $G$ 中,要么在 $\sim G$ 中,有:
\[\deg_G(v) + \deg_{\sim G}(v) = 5\]
因此,$\deg_G(v)$ 和 $\deg_{\sim G}(v)$ 中至少有一个大于等于3。不妨设 $\deg_G(v) \geq 3$(如果 $\deg_{\sim G}(v) \geq 3$,证明类似,只需交换 $G$ 和 $\sim G$ 的角色)。

设 $v$ 在 $G$ 中的三个邻居为 $a, b, c$。现在考虑 $a, b, c$ 之间的边:

如果 $a, b, c$ 中有一对在 $G$ 中相连,比如 $ab \in E(G)$,则 $v, a, b$ 构成一个三角形在 $G$ 中(即三人互相认识)。

如果 $a, b, c$ 中没有任何一对在 $G$ 中相连,则 $a, b, c$ 之间的所有边都在 $\sim G$ 中,即 $ab, ac, bc \in E(\sim G)$,所以 $a, b, c$ 构成一个三角形在 $\sim G$ 中(即三人互相不认识)。

因此,无论哪种情况,都存在一个三角形在 $G$ 或 $\sim G$ 中,即必有3个人互相认识或3个人互相不认识。证毕。


\section{特殊图}
\section{树}
