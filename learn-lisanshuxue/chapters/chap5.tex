\chapter{图论}
\section{图的基本概念}
\begin{definition}{有向图,无向图,平行边,邻接,环,孤立点,简单图}{}
1. 一个\textbf{无向图}可以被表示为$G = (V, E)$,其中$V$是一个非空集合(有限),称为\textbf{顶点集},其元素称为\textbf{顶点}或\textbf{节点},$E$是$V$中元素的无序对构成的集合,称为\textbf{边集},即$E \subseteq \{\{u,v\} \mid u,v \in V, u \neq v\}$\\
2. 一个\textbf{有向图}可以被表示为$G = (V, E)$,其中$V$是一个非空集合(有限),称为\textbf{顶点集},其元素称为\textbf{顶点}或\textbf{节点},$E$是$V$中元素的有序对构成的集合,称为\textbf{边集},即$E \subseteq \{(u,v) \mid u,v \in V, u \neq v\}$\\
3.  \textbf{平行边}:在无向图中,连接同一对顶点的多条边称为平行边,在有向图中,同一方向连接同一对顶点的多条弧称为平行边。\\
4. \textbf{邻接}:在无向图中,如果存在边$e = \{u,v\}$,则称顶点$u$和$v$是邻接的(相邻的)\\
5. \textbf{环}:如果一条边的两个端点关联于同一个结点,称为环。\\
6. \textbf{孤立点}:在无向图中,如果一个顶点不与任何其他顶点相邻,则称它为孤立点。
\end{definition}
不管是无向图还是有向图,边集允许重复的元素多次出现,即在一对结点之间存在多条边。如果关联一对结点的有向边的方向相同且有多个,则称为平行边。
\begin{definition}{度的概念}{}
\textbf{度数}(无向图):顶点$v$的度数(简称度)$\deg(v)$是与该顶点相关联的边的数量,自环通常算作2度(因为连接同一顶点的两个端点)\\
\textbf{入度和出度}(有向图):顶点$v$的入度$\deg^-(v)$是以$v$为弧头的弧的数量,顶点$v$的出度$\deg^+(v)$是以$v$为弧尾的弧的数量,顶点$v$的总度数$\deg(v) = \deg^-(v) + \deg^+(v)$\\
\textbf{悬挂结点}:度数为1的顶点称为悬挂结点,在有向图中,通常指总度数为1的顶点\\
\textbf{悬挂边}:与悬挂结点相关联的边称为悬挂边。\\
\textbf{最大度和最小度}:图$G$的最大度$\Delta(G) = \max\{\deg(v) \mid v \in V(G)\}$,图$G$的最小度$\delta(G) = \min\{\deg(v) \mid v \in V(G)\}$\\
\textbf{最大入度和最小入度}(有向图):图$D$的最大入度$\Delta^-(D) = \max\{\deg^-(v) \mid v \in V(D)\}$,图$D$的最小入度$\delta^-(D) = \min\{\deg^-(v) \mid v \in V(D)\}$\\
\textbf{最大出度和最小出度}(有向图):图$D$的最大出度$\Delta^+(D) = \max\{\deg^+(v) \mid v \in V(D)\}$,图$D$的最小出度$\delta^+(D) = \min\{\deg^+(v) \mid v \in V(D)\}$
\end{definition}
\begin{definition}{图的基本分类}{}
\textbf{带权图}:每个结点或者每条边都带有数值的图称为带权图。\\
\textbf{简单图}:如果一个图不存在平行边也不存在环,则称它为简单图。\\
\textbf{多重图}:如果一个图存在平行边,则称它为多重图。\\
\textbf{$n$阶图}:具有$n$个顶点的图称为$n$阶图。\\
\textbf{零图}:边集为空(没有边,只有结点)的图称为零图,即所有顶点都是孤立点。\\
\textbf{平凡图}:只有一个顶点且没有边的图称为平凡图,是最简单的非空图(一阶零图)。\\
\textbf{空图}:顶点集为空的图称为空图(通常不考虑)。\\
\textbf{完全图}:任意两不同顶点之间都恰有一条边的简单图称为完全图。$n$阶完全图记作$K_n$。\\
\textbf{二分图}:顶点集$V$可以划分为两个不相交子集$V_1$和$V_2$(互补结点集),使得每条边的一个端点在$V_1$中,另一个端点在$V_2$中,称为二分图。($n$阶零图是二分图)\\
\textbf{完全二分图}:顶点集$V$可以划分为两个不相交子集$V_1$和$V_2$,使得每条边的一个端点在$V_1$中,另一个端点在$V_2$中,且 $V_1$ 中的每个顶点都与 $V_2$ 中的每个顶点相连,且图中没有其他边,则称 $G$ 为完全二分图,记作 $K_{m,n}$,其中 $m = |X|$,$n = |Y|$
\textbf{正则图}:所有顶点度数都相同的图称为正则图。若每个顶点的度数均为$k$,称为$k$-正则图。\\
\textbf{环图}:$n$个顶点$v_1,v_2,\ldots,v_n$依次连接形成的环状图称为环图,记作$C_n$。\\
\textbf{轮图}:在环图$C_{n-1}$中添加一个顶点,并将该顶点与环图中所有顶点相连,称为轮图$W_n$。\\
\textbf{$n$方体图}:用$n$维超立方体的顶点和边构成的图称为$n$方体图,记作$Q_n$。顶点集为所有长度为$n$的二进制串,两个顶点相邻当且仅当它们的二进制表示恰好有一位不同。
\end{definition}
\begin{theorem}{图的基本定理}{}
握手定理:设 $G = (V, E)$ 是一个无向图,则所有顶点的度数之和满足$\displaystyle\sum_{v \in V} \deg(v) = 2|E|$
\end{theorem}
\textbf{证明:} 由于每条边连接两个顶点,每条边在度数求和中被计算了两次(一次为每个端点)。因此,总度数之和等于边数的两倍。具体地,对于每条边 $e = \{u, v\} \in E$,它贡献 1 到 $\deg(u)$ 和 1 到 $\deg(v)$,所以在求和 $\sum_{v \in V} \deg(v)$ 中,每条边恰好被计算两次。故$\displaystyle\sum_{v \in V} \deg(v) = 2|E|$.
\begin{theorem}{有向图握手定理}{}
设 $D = (V, A)$ 是有向图,则所有顶点的入度之和等于所有顶点的出度之和,且都等于边数:
\[
\sum_{v \in V} \deg^-(v) = \sum_{v \in V} \deg^+(v) = |A|
\]
\end{theorem}
\textbf{证明:} 每条有向边有一个起点和一个终点,为起点的出度贡献 1,为终点的入度贡献 1。因此,总入度之和等于总出度之和,都等于边数。
\begin{theorem}{握手定理推论}{}
任意图中,度数为奇数的顶点个数为偶数。\\
设 $G$ 是 $n$ 阶无向简单图,则 $G$ 中最大度数满足 $\Delta(G) \leqslant n-1$。\\
在任意无向图中,所有顶点的平均度数为 $\dfrac{2|E|}{|V|}$。\\
如果无向图 $G$ 是 $k$-正则图(每个顶点度数均为 $k$),则 $k|V| = 2|E|$
\end{theorem}
设 $V_1$ 为度数为奇数的顶点集合,$V_2$ 为度数为偶数的顶点集合。由握手定理,总度数和为偶数。$V_2$ 中顶点度数之和为偶数,故 $V_1$ 中顶点度数之和也必为偶数,因此 $|V_1|$ 必为偶数。

\begin{theorem}{可简单图化的判定定理(Havel-Hakimi算法)}{}
一个非负整数序列 $d = (d_1, d_2, \ldots, d_n)$(其中 $d_1 \geqslant d_2 \geqslant \cdots \geqslant d_n$,和为偶数)是可简单图化的,当且仅当可以通过\textbf{Havel-Hakimi算法}判定:

1.将序列按非递增顺序排序\\
2.如果序列全为零,则是可简单图化的\\
3.如果序列中有负数,或最大度数 $d_1 \geqslant n$,则不可简单图化\\
4.否则,删除第一个元素 $d_1$,然后将接下来的 $d_1$ 个元素每个减1\\
5.返回步骤1,重复上述过程

如果序列可简单图化,那么存在一个简单图以该序列为度序列,如果在某一步出现负数,说明无法构造满足条件的简单图,如果算法最终得到全零序列,则原序列是可简单图化的
\end{theorem}
\textbf{示例:}
判定序列 $(3, 3, 2, 2, 2)$ 是否可简单图化:
\begin{align*}
& (3, 3, 2, 2, 2) \quad \text{删除3,后面3个数各减1} \\
\to & (2, 1, 1, 2) \quad \text{重新排序为}(2, 2, 1, 1) \\
\to & (2, 2, 1, 1) \quad \text{删除2,后面2个数各减1} \\
\to & (1, 0, 1) \quad \text{重新排序为}(1, 1, 0) \\
\to & (1, 1, 0) \quad \text{删除1,后面1个数减1} \\
\to & (0, 0) \quad \text{全为零,可简单图化}
\end{align*}
\begin{definition}{补图}{}
设 $G = (V, E)$ 是一个简单无向图(无自环和无重边),其顶点集 $V$ 有 $n$ 个顶点。令 $K_n$ 表示完全图,其顶点集为 $V$,边集为 $E(K_n)$(即所有可能的无序顶点对)。则图 $G$ 的补图记为 $\sim G$ :
\[
\sim G = (V, E(K_n) -E)
\]
换句话说,$\sim G$ 的边集由 $K_n$ 中所有不在 $G$ 中的边组成。
\end{definition}
\begin{definition}{图的基本运算}{}

\textbf{1. 删除边运算}
设 $G = (V, E)$ 是一个图,$e \in E$ 是 $G$ 的一条边。从 $G$ 中删除边 $e$ 得到的图记为 $G - e$,定义为:
\[
G - e = (V, E - \{e\})
\]
即顶点集不变,仅从边集中移除边 $e$。

\textbf{2. 删除边集运算}
设 $G = (V, E)$ 是一个图,$F \subseteq E$ 是 $G$ 的一个边子集。从 $G$ 中删除边集 $F$ 得到的图记为 $G - F$,定义为:
\[
G - F = (V, E- F)
\]
即顶点集不变,从边集中移除所有属于 $F$ 的边。

\textbf{3. 删除结点运算}
设 $G = (V, E)$ 是一个图,$v \in V$ 是 $G$ 的一个顶点。从 $G$ 中删除顶点 $v$ 得到的图记为 $G - v$,定义为:
\[
G - v = (V - \{v\}, E-\{e \in E \mid v \in e\})
\]
即从顶点集中移除 $v$,同时移除所有与 $v$ 相关联的边。

\textbf{4. 删除结点集运算}
设 $G = (V, E)$ 是一个图,$U \subseteq V$ 是 $G$ 的一个顶点子集。从 $G$ 中删除顶点集 $U$ 得到的图记为 $G - U$,定义为:
\[
G - U = (V - U, E - \{e \in E \mid e \cap U \neq \varnothing\})
\]
即从顶点集中移除所有属于 $U$ 的顶点,同时移除所有与 $U$ 中顶点相关联的边。

\textbf{5. 边的收缩运算}
设 $G = (V, E)$ 是一个图,$e = uv \in E$ 是 $G$ 的一条边。收缩边 $e$ 得到的图记为 $G/e$,定义为:
将顶点 $u$ 和 $v$ 合并为一个新顶点 $w$,所有与 $u$ 或 $v$ 相关联的边(除 $e$ 外)改为与 $w$ 相关联,删除边 $e$,如果合并后产生重边,通常保留一条(简单图)或全部保留(多重图)。形式化地,$G/e = (V', E')$,其中:
\[
V' = (V - \{u, v\}) \cup \{w\}, \quad E' = E -\{e\} 
\]

\textbf{6. 加新边运算}
设 $G = (V, E)$ 是一个图,$e = (u,v)$ 是连接 $G$ 中两个顶点 $u, v \in V$ 但 $e \notin E$ 的边。向 $G$ 中添加新边 $e$ 得到的图记为 $G + e$,定义为:
\[
G + e = (V, E \cup \{e\})
\]
即顶点集不变,向边集中添加新边 $e$。
\end{definition}
注意,删除边通常会保留结点,但是删除结点通常不会保留边,加新边不会引入新的结点,边的收缩会影响结点的数量。
\begin{definition}{子图及相关概念}{}
\textbf{1. 子图}
设图 $G = (V, E)$ 和图 $H = (V', E')$。如果 $V' \subseteq V$ 且 $E' \subseteq E$,并且 $H$ 中每条边的端点都在 $V'$ 中,则称 $H$ 是 $G$ 的子图,记作 $H \subseteq G$。

\textbf{2. 母图}
如果 $H$ 是 $G$ 的子图,则称 $G$ 是 $H$ 的母图。

\textbf{3. 真子图}
如果 $H$ 是 $G$ 的子图,且 $H \neq G$(即 $V' \subset V$ 或 $E' \subset E$),则称 $H$ 是 $G$ 的真子图。

\textbf{4. 生成子图}
如果 $H$ 是 $G$ 的子图,且 $V' = V$(即顶点集相同),则称 $H$ 是 $G$ 的生成子图。

\textbf{5. 导出的子图记作}
导出子图分为两类:\\
\textbf{结点集导出子图}:对于 $G$ 的顶点子集 $V_1 \subseteq V$,由 $S$ 导出的子图记作 $G(V_1)$,其顶点集为 $S$,边集为 $G$ 中所有端点都在 $S$ 中的边,即 $E(G(V_1)) = \{ e \in E \mid e \subseteq S \}$。\\
\textbf{边集导出子图}:对于 $G$ 的边子集 $F \subseteq E$,由 $F$ 导出的子图记作 $G[F]$,其顶点集为 $F$ 中所有边的端点的并集,边集为 $F$。
\end{definition}
注意,生成子图保留了全部的结点,以及某些边(可选)。但是结点集导出子图仅保留部分的结点,而其边集则是由其保留的结点来确定。边集导出子图仅保留部分的边,而其顶点集则是由其保留的边来确定。

再加上上面介绍的删除结点,删除边等运算,下面讨论它们的区别和联系。

$G(V-V_1)$是由结点集$V-V_1$导出的子图,也就是在图$G$中删除结点集$V_1$以及所有它们关联的边所得到的子图。所以可以将$G(V-V_1)$视作$G-V_1$。而$G(E-E_1)$是由边集$E-E_1$导出的子图,也就是在图$G$中删除边集$E_1$以及里面的所有边所导出的子图,反观$G-E_1$,\textbf{由于删除边集时要保留结点不可以删除},所以删除边集运算不会改变结点数量,而删除边集导出的子图会改变结点数量,因为导出过程关注边(及其关联的结点)而非可能存在的孤立结点,所以要注意区别$G(E-E_1)$和$G-E_1$。
\begin{definition}{图的同构,自互补图}{}
设 $G = (V, E)$ 和 $H = (U, F)$ 是两个简单图(无向图或有向图)。如果存在一个双射函数 $f: V \to U$,使得对于 $G$ 中的任意两个顶点 $u, v \in V$:\vspace{-10pt}
\[
(u,v) \in E(G) \iff (f(u), f(v)) \in E(H)\vspace{-10pt}
\]
则称图 $G$ 和图 $H$ 是同构的,记作 $G \cong H$,并称 $f$ 是 $G$ 到 $H$ 的一个同构映射。\\
\textbf{自互补图}:如果一个图与自身的补图同构,则称该图为自互补图。
\end{definition}
\begin{theorem}{同构的不变量}{}
如果两个图同构,则它们必须共享以下性质(同构不变量):\\
顶点数相同,边数相同,结点度数序列相同(不计顺序),连通分支数和每个分支的大小相同,围长(最短环长度)相同,直径相同,色数相同,特征多项式相同(邻接矩阵的特征多项式)\\
\textbf{注意:}同构关系是等价关系(自反、对称、传递)
\end{theorem}
设 $\mathcal{G}$ 表示所有简单图的集合。图同构关系“$\cong$”是 $\mathcal{G}$ 上的一个等价关系,即满足以下性质:

\textbf{1. 自反性:} 对于任意简单图 $G \in \mathcal{G}$,有 $G \cong G$。

\textbf{证明:}
定义恒等映射 $I_d: V(G) \to V(G)$,其中 $I_d(v) = v$ 对于所有 $v \in V(G)$。显然,$I_d$ 是双射,且对于任意 $u, v \in V(G)$,有 $\{u, v\} \in E(G)$ 当且仅当 $\{I_d(u), I_d(v)\} \in E(G)$。因此,$I_d$ 是 $G$ 到自身的同构映射,故 $G \cong G$。

\textbf{2. 对称性:} 对于任意简单图 $G, H \in \mathcal{G}$,如果 $G \cong H$,则 $H \cong G$。

\textbf{证明:}
假设 $G \cong H$,则存在同构映射 $f: V(G) \to V(H)$,即 $f$ 是双射,且对于任意 $u, v \in V(G)$,有 $\{u, v\} \in E(G)$ 当且仅当 $\{f(u), f(v)\} \in E(H)$。考虑 $f$ 的逆映射 $f^{-1}: V(H) \to V(G)$。由于 $f$ 是双射,$f^{-1}$ 也是双射。对于任意 $x, y \in V(H)$,令 $u = f^{-1}(x)$, $v = f^{-1}(y)$,则 $\{x, y\} \in E(H)$ 当且仅当 $\{f(u), f(v)\} \in E(H)$,这等价于 $\{u, v\} \in E(G)$,即 $\{f^{-1}(x), f^{-1}(y)\} \in E(G)$。因此,$f^{-1}$ 是 $H$ 到 $G$ 的同构映射,故 $H \cong G$。

\textbf{3. 传递性:} 对于任意简单图 $G, H, K \in \mathcal{G}$,如果 $G \cong H$ 且 $H \cong K$,则 $G \cong K$。

\textbf{证明:}
假设 $G \cong H$ 且 $H \cong K$,则存在同构映射 $f: V(G) \to V(H)$ 和 $g: V(H) \to V(K)$。考虑复合映射 $g \circ f: V(G) \to V(K)$。由于 $f$ 和 $g$ 都是双射,$g \circ f$ 也是双射。对于任意 $u, v \in V(G)$,有:
\[
\{u, v\} \in E(G) \iff \{f(u), f(v)\} \in E(H) \iff \{g(f(u)), g(f(v))\} \in E(K)
\]
即 $\{u, v\} \in E(G)$ 当且仅当 $\{(g \circ f)(u), (g \circ f)(v)\} \in E(K)$。因此,$g \circ f$ 是 $G$ 到 $K$ 的同构映射,故 $G \cong K$。

\textbf{结论:}
由于图同构关系满足自反性、对称性和传递性,因此它是等价关系。
\begin{theorem}{自互补图的性质}{}
设 $G$ 是一个 $n$ 阶自互补图(即 $G \cong \overline{G}$),则以下性质成立:
\begin{enumerate}[itemsep=2pt, parsep=2pt, topsep=5pt, label=(\arabic*)]
    \item 自互补图对应的完全图边数为偶数。
    \item 顶点数性质:$n \equiv 0 \pmod{4}$ 或 $n \equiv 1 \pmod{4}$。
    \item 边数性质:$G$ 的边数为 $\dfrac{n(n-1)}{4}$。
    \item 正则性:如果 $d$ 是 $G$ 中某个顶点的度数,则 $n-1-d$ 也必须是 $G$ 中某个顶点的度数。
    \item 直径性质:自互补图的直径最大为 3(直径的定义后文会给出)。
    \item 连通性:所有自互补图都是连通的。
\end{enumerate}
\end{theorem}
边数性质:由于 $G$ 和 $\sim G$ 同构,它们边数相等,所以自互补图对应的完全图边数为偶数。完全图 $K_n$ 有 $\frac{n(n-1)}{2}$ 条边,这些边被平分给 $G$ 和 $\sim G$,所以 $G$ 的边数为 $\frac{n(n-1)}{4}$,这必须是整数,故 $n(n-1)$ 必须被 4 整除。

在同构映射下,度数为 $d$ 的顶点映射到度数为 $n-1-d$ 的顶点。如果直径大于3,则存在距离为4的顶点对,在补图中距离为1,矛盾。如果不连通,则补图连通,但自互补图必须与补图有相同的连通性。
\begin{example}{}{}
    证明:6个人聚在一起,必然有3个人互相认识彼此(双向认识),或者至少有3个人互相都不认识彼此(双向不认识)。
\end{example}
将6个人视为6个顶点,构建一个完全图 $K_6$。定义图 $G$:顶点表示人,边表示两人互相认识。则补图 $\sim G$ 表示两人互相不认识。问题转化为证明:要么 $G$ 包含一个三角形(即3个顶点两两相连),要么 $\sim G$ 包含一个三角形。

考虑任意顶点 $v$。在完全图 $K_6$ 中,$v$ 与其他5个顶点相连。由于边要么在 $G$ 中,要么在 $\sim G$ 中,有:
\[\deg_G(v) + \deg_{\sim G}(v) = 5\]
因此,$\deg_G(v)$ 和 $\deg_{\sim G}(v)$ 中至少有一个大于等于3。不妨设 $\deg_G(v) \geqslant 3$(如果 $\deg_{\sim G}(v) \geqslant 3$,证明类似,只需交换 $G$ 和 $\sim G$ 的角色)。

设 $v$ 在 $G$ 中的三个邻居为 $a, b, c$。现在考虑 $a, b, c$ 之间的边:

如果 $a, b, c$ 中有一对在 $G$ 中相连,比如 $ab \in E(G)$,则 $v, a, b$ 构成一个三角形在 $G$ 中(即三人互相认识)。

如果 $a, b, c$ 中没有任何一对在 $G$ 中相连,则 $a, b, c$ 之间的所有边都在 $\sim G$ 中,即 $ab, ac, bc \in E(\sim G)$,所以 $a, b, c$ 构成一个三角形在 $\sim G$ 中(即三人互相不认识)。

因此,无论哪种情况,都存在一个三角形在 $G$ 或 $\sim G$ 中,即必有3个人互相认识或3个人互相不认识。证毕。
\begin{example}{}{}
    含5个结点,3条边的不同构的简单图有多少个?
\end{example}
我们需要枚举所有具有5个顶点和3条边的简单图(无自环、无重边)的不同构类。由于顶点数较少,可以通过度序列分类讨论。

\noindent \textbf{步骤1:列出所有可能的度序列。}
由于边数为3,总度数为6。度序列是5个非负整数的非增序列,每个不超过4,且总和为6。可能的非增序列有:
\[
(3,1,1,1,0),\quad (2,2,2,0,0),\quad (2,2,1,1,0),\quad (2,1,1,1,1).
\]
其他如$(3,3,0,0,0)$、$(4,1,1,0,0)$等均不可图(边数约束或可图性不满足)。

\textbf{步骤2:对每个度序列构造图并判断同构类。}

\textbf{度序列 $(3,1,1,1,0)$}:有一个度为3的顶点,它必须与其余三个度为1的顶点相连,另一个顶点孤立。图的结构是星形$K_{1,3}$加上一个孤立点。所有此类图同构。

\textbf{度序列 $(2,2,2,0,0)$}:三个度为2的顶点必须构成一个三角形(每对之间相连),两个顶点孤立。图的结构是三角形加两个孤立点。所有此类图同构。

\textbf{度序列 $(2,2,1,1,0)$}:两个度为2的顶点之间必有一条边(否则若不相连,每个需连两个度为1的顶点,导致度为1的顶点度数增加,矛盾)。于是,两个度为2的顶点相连,并各连接一个度为1的顶点,形成一条长度为3的路径$P_4$,再加一个孤立点。所有此类图同构于$P_4$加孤立点。

\textbf{度序列 $(2,1,1,1,1)$}:唯一度为2的顶点连接两个度为1的顶点,剩下两个度为1的顶点必须彼此相连(否则无法满足度数)。于是图由一条长度为2的路径$P_3$和一条独立边$K_2$组成,即$P_3 \cup K_2$。所有此类图同构。

每个度序列对应唯一的同构类,且不同度序列对应的图不同构。因此,共有4个不同构的图。分别为:三角形,三节棍,折线加直线(2个分支),鸡爪子状。

\begin{example}{}{}
    完全图 $K_4$ 中非同构的生成子图的数量为?
\end{example}
完全图 $K_4$ 中非同构的生成子图的数量为 \textbf{11}。这些生成子图对应4个顶点的所有非同构简单图,按边数分类如下:
\begin{enumerate}
    \item 0 条边:1 种(4阶零图)
    \item 1 条边:1 种
    \item 2 条边:2 种(两条边共点、两条边不交)
    \item 3 条边:3 种(路径 $P_4$、星 $K_{1,3}$、三角形加孤立点)
    \item 4 条边:2 种(四边形、三角形加悬挂边)
    \item 5 条边:1 种($K_4$ 删去一条边)
    \item 6 条边:1 种(完全图 $K_4$)
\end{enumerate}
一般 $n$ 的计数公式:对于一般的 $n$,设 $a(n)$ 为完全图 $K_n$ 中非同构的生成子图的数量(即 $n$ 个顶点的非同构图的数量)。利用波利亚计数定理,考虑对称群 $S_n$ 在边集上的作用,可得:
\[
a(n) = \frac{1}{n!} \sum_{\substack{j_1 + 2j_2 + \cdots + n j_n = n \\ j_k \ge 0}} 
\frac{n!}{\prod_{k=1}^n k^{j_k} j_k!} \; 
2^{\sum_{k=1}^n \left\lfloor \frac{k}{2} \right\rfloor j_k \;+\; 
\sum_{1 \le r < s \le n} \gcd(r, s) \, j_r j_s}
\]
其中求和取遍所有满足 $1j_1 + 2j_2 + \cdots + n j_n = n$ 的非负整数解 $(j_1, \ldots, j_n)$,指数中的项对应于轮换型为 $1^{j_1}2^{j_2}\cdots n^{j_n}$ 的置换下边轨道的数目。该公式提供了计算 $a(n)$ 的递推方法,依次可求得:
\[
a(0)=1,\; a(1)=1,\; a(2)=2,\; a(3)=4,\; a(4)=11,\; a(5)=34,\; a(6)=156,\; \ldots
\]


\begin{definition}{图的路径与回路}{}
设图 $G = (V, E)$,定义以下概念:
\begin{enumerate}
    \item \textbf{通路}:一个顶点和边的交替序列 $v_0e_1v_1e_2\cdots e_kv_k$,
          其中 $e_i = (v_{i-1}, v_i)$。$v_0$ 称为起点,$v_k$ 称为终点,$k$ 称为通路的长度。
    
    \item \textbf{回路}:如果通路的起点和终点相同(即 $v_0 = v_k$),则称该通路为回路。
    
    \item \textbf{简单通路}:一条不重复经过同一边的通路。
    
    \item \textbf{基本通路/初级通路/路径}:一条经过的顶点互不相同(从而边也互不相同)的通路。
    
    \item \textbf{基本回路/初级回路/圈}:除起点和终点外,经过的其余顶点互不相同的回路。
    
    \item \textbf{长度}:一条回路或通路所包含的边的数目被称为回路或者通路的长度,
          若长度为奇数/偶数,则称为奇回路/偶回路。
    
    \item \textbf{距离}:设 $G = (V, E)$ 是一个无向图。对于任意两顶点 $u, v \in V$,
          定义距离 $d(u, v)$ :
          \begin{itemize}
              \item 若 $u$ 和 $v$ 间存在至少一条路径, $d(u, v)$ 是 $u$ 到 $v$ 的
                    最短路径(短程线)的长度;
              \item 如果 $u$ 和 $v$ 之间不存在路径(即 $u$ 和 $v$ 属于不同的连通分支),
                    则 $d(u, v) = \infty$。
          \end{itemize}
\end{enumerate}
\end{definition}
基本回路(通路)一定是简单回路(通路),但反之则不然。
\begin{definition}{连通图,非连通图,连通关系}{}
\begin{enumerate}
    \item \textbf{连通图}:一个无向图 $G = (V, E)$ 称为连通图,如果对于任意两个顶点 $u, v \in V$,都存在一条从 $u$ 到 $v$ 的路径。即图中任意两个顶点都是连通的。
    \item \textbf{非连通图}:一个无向图如果不是连通图,则称为非连通图。即存在至少两个顶点之间没有路径相连。
    \item \textbf{连通关系}:无向图 $G = (V, E)$ ,定义顶点集 $V$ 上的二元关系 $R$:\vspace{-10pt}
\[
uRv \iff \text{存在一条从 } u \text{ 到 } v \text{ 的路径}\vspace{-10pt}
\]
这个关系称为连通关系。
\item\textbf{连通分支:}设 $G = (V, E)$ 是一个无向图。图 $G$ 的连通分支是指 $G$ 的一个极大连通子图$H$满足$H$是连通的;$H$ 是极大的,即不存在 $G$ 的另一个连通子图 $H'$,使得 $H$ 是 $H'$ 的真子图。
\end{enumerate}
\end{definition}
\begin{theorem}{连通图只能有偶数个奇度数顶点}{}
若某个图$G$恰好有两个奇度数顶点,则这两个顶点一定连通
\end{theorem}
在任何无向图 $G$ 中,奇度数顶点的个数是偶数。图 $G$ 的奇度数顶点个数为偶数,故恰好有两个奇度数顶点是可能的。设这两个奇度数顶点为 $u$ 和 $v$。考虑 $G$ 的各个连通分支。在每个连通分支中,所有顶点的度数之和为偶数(因为该分支内部的边贡献2度),因此每个连通分支中奇度数顶点的个数必为偶数(类似定理1的论证)。若 $u$ 和 $v$ 分别属于两个不同的连通分支,则每个分支将各自含有奇数个奇度数顶点(1个),这与每个连通分支中奇度数顶点个数为偶数矛盾。因此,$u$ 和 $v$ 必属于同一个连通分支,从而它们之间存在一条路径。
\begin{definition}{点割集、割点、边割集、割边}{}
\begin{enumerate}
    \item\textbf{点割集:}设 $G = (V, E)$ 是一个无向图。如果删除结点子集 $S$ 中的所有顶点(同时删除与这些顶点相关联的边)后,图的连通分支数增加。即:$W(G - S) > W(G)$,其中 $W(G)$ 表示图 $G$ 的连通分支数。 $S \subset V$ 称为 $G$ 的一个点割集。同时还要求是极小的点割集(本书定义如此)。
    \item \textbf{割点:}单独构成一个点割集的结点 $v \in V$为图 $G$ 的割点。
    \item \textbf{边割集:}设 $G = (V, E)$ 是一个无向图。边子集 $F \subset E$ 称为 $G$ 的一个边割集,如果删除 $F$ 中的所有边后,图的连通分支数增加。即:$W(G - F) > W(G)$。同时还要求是极小的边割集(本书定义如此)。
\item\textbf{割边/桥:}单独构成一个边割集的边 $e\in E$ 为图 $G$ 的割边或桥。
\end{enumerate}
\end{definition}
\begin{definition}{有向图中的可达性}{}
设 $D = (V, A)$ 是一个有向图。
\begin{enumerate}
\item\textbf{可达}:对于顶点 $u, v \in V$,如果存在一条从 $u$ 到 $v$ 的有向路径(即顶点序列 $u = v_0, v_1, \ldots, v_k = v$,其中对每个 $i$,$(v_i, v_{i+1}) \in A$),则称 $v$ 是从 $u$ 可达的,记作 $u , v$。规定一个节点到自身总是可达的。
\item\textbf{不可达}:如果不存在从 $u$ 到 $v$ 的有向路径,则称 $v$ 是从 $u$ 不可达的。
\item\textbf{相互可达}:如果 $u , v$ 且 $v , u$,即存在从 $u$ 到 $v$ 的有向路径和从 $v$ 到 $u$ 的有向路径,则称 $u$ 和 $v$ 是相互可达的,记作 $u \leftrightarrow v$。
\item\textbf{(相互)可达关系:}有向图中结点的可达关系具有自反性,传递性,但不具有对称性。相互可达关系是等价关系(自反、对称、传递)。它将顶点集划分为等价类,每个等价类称为一个强连通分支。
\end{enumerate}
\end{definition}
\begin{definition}{有向图的连通性}{}
对于有向图 $D = (V, A)$ ,$\text{强连通} \subset \text{单向连通} \subset \text{弱连通}$
\begin{enumerate}
    \item \textbf{强连通}:如果对于该有向图其中任意两个顶点 $u, v \in V$,满足$u$ 到 $v$ 是可达的,并且 $v$ 到 $u$ 也是可达的。则该有向图是强连通的。
\item\textbf{弱连通}:如果将该有向图中所有边的方向忽略后得到的无向图是连通的。则该有向图是弱连通的。
\item\textbf{单向连通}:如果对于该有向图中任意两个顶点 $u, v \in V$, $u$ 到 $v$ 可达或者 $v$ 到 $u$ 可达(或两者都可达)。则称该有向图是单向连通的。
\end{enumerate}
\end{definition}
\begin{theorem}{连通图中不同结点之间的通路长度不会是$n$}{}
    设 $G$ 是一个连通图,$n$ 为 $G$ 的顶点数。则对于任意两个顶点 $u, v$,$u$ 到 $v$ 之间的通路长度至多为 $n-1$。
\end{theorem}
    假设不然,即存在两个顶点 $u$ 和 $v$,以及一条从 $u$ 到 $v$ 的通路 $P$,其长度 $L \geqslant n$(其中长度定义为通路中的边数)。
    由于通路 $P$ 有 $L$ 条边,因此包含 $L+1$ 个顶点(包括起点和终点)。但图中只有 $n$ 个顶点,且 $L+1 \geqslant n+1$,根据鸽巢原理,通路 $P$ 中至少有两个顶点相同(即顶点重复)。这意味着 $P$ 中存在一个回路(cycle)。
    
    设重复的顶点为 $w$,则 $P$ 中从 $w$ 到 $w$ 的子通路是一个回路。
    因此,$P$ 中至少有 $L+1$ 个顶点,且至少有一个顶点重复。
    现在,从 $P$ 中删除这个回路,得到一条新的从 $u$ 到 $v$ 的通路 $P'$,其长度小于 $L$。如果 $P'$ 的长度仍然 $\geqslant n$,我们可以重复上述过程,继续删除回路。
    
    但由于每次删除都会减少通路长度,而通路长度是非负整数,这个过程最终会停止,得到一条长度小于 $n$ 的通路。假设中存在长度 $\geqslant n$ 的通路,这与最短通路的长度至多为 $n-1$ 矛盾。故假设不成立,定理得证。
\begin{definition}{强分图(强连通分量)}{}
    设有向图 $G = (V, E)$。在顶点集 $V$ 上定义关系 $R$:对于任意 $u, v \in V$,$u R v$ 当且仅当 $u$ 与 $v$ 相互可达,即存在从 $u$ 到 $v$ 的有向路径,也存在从 $v$ 到 $u$ 的有向路径。容易验证 $R$ 是 $V$ 上的等价关系(满足自反性、对称性和传递性)。该等价关系将 $V$ 划分为若干个等价类 $V_1, V_2, \dots, V_k$,每个等价类 $V_i$ 连同 $G$ 中两端均在 $V_i$ 内的所有有向边构成的诱导子图称为 $G$ 的一个强分图(Strongly Connected Component, SCC)。每个强分图本身是强连通的,并且是极大的(不能再添加任何顶点而仍保持强连通)。
\end{definition}
\begin{theorem}{简单无向图的连通分支,结点数和边数的关系}{}
设 $G$ 是一个有 $n$ 个顶点、$e$ 条边的简单无向图,且 $G$ 有 $W$ 个连通分支,则以下关系成立:
\[n - W\leqslant e\leqslant \dfrac{(n-W)(n-W+1)}{2}\]
\end{theorem}
设 $G$ 的 $W$ 个连通分支分别为 $G_1, G_2, \ldots, G_W$,其中 $G_i$ 有 $n_i$ 个顶点和 $e_i$ 条边。由于每个连通分支是连通图,考虑边数最少的情形,也就是每个连通分支都是树,有 $e_i \geqslant n_i - 1$。因此,总边数:
   \[
   e = \sum_{i=1}^W e_i \geqslant \sum_{i=1}^W (n_i - 1) = \left(\sum_{i=1}^W n_i\right) - W = n - W
   \]
等号成立当且仅当每个连通分支都是一棵树。现在考虑$e$的上界,由于每个连通分支 $G_i$ 最多有 $\frac{n_i(n_i-1)}{2}$ 条边(当 $G_i$ 是完全图时),因此总边数满足:
   \[
   e \leqslant \sum_{i=1}^W \frac{n_i(n_i-1)}{2},\quad\sum_{i=1}^W n_i = n
   \]
   定义函数 $f(k) = \frac{k(k-1)}{2}$,其中 $k \geqslant 1$。计算二阶导数:
   \[
   f'(k) = k - \frac{1}{2}, \quad f''(k) = 1 > 0
   \]
   所以 $f(k)$ 是凸函数。由于凸函数在固定和下的和最大化当变量取极值,即当 $n_1, n_2, \ldots, n_W$ 中一个尽可能大、其余尽可能小时达到最大值。具体地,令 $n_1 = n-W+1$,$n_2 = n_3 = \cdots = n_W = 1$,则:
   \[
   \sum_{i=1}^W f(n_i) = f(n-W+1) + \sum_{i=2}^W f(1) = \frac{(n-W+1)(n-W)}{2} + (W-1) \cdot 0 = \frac{(n-W+1)(n-W)}{2}
   \]
   对于任何其他分配,由于 $f(k)$ 的凸性,和不会更大。因此:
   \[
   e \leqslant \frac{(n-W+1)(n-W)}{2} = \frac{(n-W)(n-W+1)}{2}
   \]
   等号成立当且仅当 $G$ 由一个完全图 $K_{n-W+1}$ 和 $W-1$ 个孤立顶点组成。
\begin{theorem}{简单图连通的最小边数条件}{}
    简单图$G$有$n$个结点,$e$条边,如果$2e>(n-1)(n-2)$,则$G$是连通的
\end{theorem}
\textbf{证明:}假设$G$不连通,则$G$至少有两个连通分支。设$G$的连通分支为$G_1, G_2, \ldots, G_k$,其中$k \geqslant 2$。每个连通分支$G_i$有$n_i$个顶点,满足$\sum_{i=1}^k n_i = n$。为了最大化$G$的边数,各连通分支应尽可能完整,即每个$G_i$应为完全图$K_{n_i}$。此时$G$的边数最大值为:
\[
e \leqslant \sum_{i=1}^k \binom{n_i}{2} = \sum_{i=1}^k \frac{n_i(n_i-1)}{2}
\]

考虑函数$f(x) = x(x-1)$,它是凸函数。由凸性,在固定$\sum n_i = n$的条件下,当$n_1 = n-1$,$n_2 = 1$,$n_3 = \cdots = n_k = 0$时(实际上$k=2$),和$\sum n_i(n_i-1)$取得最大值。因此:
\[
e \leqslant \binom{n-1}{2} + \binom{1}{2} = \frac{(n-1)(n-2)}{2} + 0 = \frac{(n-1)(n-2)}{2}
\]

但已知$2e > (n-1)(n-2)$,即$e > \frac{(n-1)(n-2)}{2}$,与上式矛盾。故假设错误,$G$必须是连通的。

\textbf{注:}这个定理给出了保证图连通的最小边数条件。完全图$K_n$有$\frac{n(n-1)}{2}$条边,而完全图去掉一个顶点$K_{n-1}$有$\frac{(n-1)(n-2)}{2}$条边。因此,只要边数超过$K_{n-1}$的边数,图就必须包含所有$n$个顶点且是连通的。
\begin{example}{}{}
    若一个无向简单图有$2n$个结点,每一个结点的度数都大于等于$n$,则该图是连通图。
\end{example}
假设图 $G$ 不连通,则 $G$ 至少有两个连通分支。由于 $G$ 有 $2n$ 个顶点,根据鸽巢原理,至少有一个连通分支 $H$ 的顶点数 $k$ 满足 $1 \leqslant k \leqslant n$。
    
考虑连通分支 $H$。由于 $G$ 是简单图,$H$ 也是简单图,因此 $H$ 中每个顶点的度数最多为 $k-1\leqslant n-1$。但根据条件,$H$ 中每个顶点的度数至少为 $n$。矛盾,图 $G$ 必须是连通的。

若将条件改为$n-1$,考虑图$G$由两个不相交的完全图$K_n$组成,即$G = K_n \cup K_n$,其中两个$K_n$之间没有边连接。则$G$有$2n$个结点,每个结点的度数均为$n-1$(因为每个$K_n$中结点的度数为$n-1$),但$G$不是连通图,因为两个$K_n$之间没有路径。
\begin{theorem}{平均度数介于最小度数和最大度数之间}{}
    设图$G$的结点数为$v$,边数为$e$,证明$d(v_{min})\leqslant 2e/v\leqslant d(v_{max})$,其中$d(v)$表示$G$中顶点$v$的度数。
\end{theorem}
该定理表明图的平均度数介于最小度数和最大度数之间。由最小度和最大度的定义,对于任意顶点 $u \in V(G)$,有:
    \[
    d(v_{\text{min}}) \leqslant d(u) \leqslant d(v_{\text{max}})
    \]
    因此,对所有顶点的度数求和可得:
    \[
    \sum_{u \in V(G)} d(v_{\text{min}}) \leqslant \sum_{u \in V(G)} d(u) \leqslant \sum_{u \in V(G)} d(v_{\text{max}})
    \]
    即:
    \[
    v \cdot d(v_{\text{min}}) \leqslant 2e \leqslant v \cdot d(v_{\text{max}})
    \]
    由于顶点数 $v > 0$,将上式除以 $v$,即得:
    \[
    d(v_{\text{min}}) \leqslant \frac{2e}{v} \leqslant d(v_{\text{max}})
    \]
\begin{theorem}{有向图强连通的判定定理}{}
    一个有向图是强连通的当且仅当该图中存在经过每个结点的回路
\end{theorem}
设 $D = (V, A)$ 是一个有向图,$|V| = n$。

\textbf{必要性:} 如果存在一个经过每个结点的回路 $C$(即 $C$ 是一个闭路径覆盖所有顶点),则对于任意两个顶点 $u, v \in V$,由于 $C$ 包含所有顶点,存在 $C$ 上从 $u$ 到 $v$ 的子路径和从 $v$ 到 $u$ 的子路径。因此,$u$ 到 $v$ 和 $v$ 到 $u$ 都是可达的,故 $D$ 是强连通的。

\textbf{充分性:} 如果 $D$ 是强连通的,则对于任意顶点 $v \in V$,存在从 $v$ 到其他所有顶点的路径。构造一个闭路径覆盖所有顶点:从任意顶点 $v_1$ 开始,由于强连通性,存在从 $v_1$ 到 $v_2$ 的路径($v_2$ 为未访问的顶点),然后从 $v_2$ 到 $v_3$ 的路径,依此类推,直到访问所有顶点。最后,从最后一个顶点返回 $v_1$(强连通性保证路径存在)。这样得到的闭路径经过每个结点至少一次,即存在经过每个结点的回路(闭路径)。因此,定理得证。

\section{图的表示法}
\begin{definition}{邻接表}{}
    列出图的每一个结点与其邻接的结点的表称作邻接表,注意邻接表的局限性是不适合表示多重边的无向图或者有向图。
\end{definition}
考虑一个无向加权图 $G$,顶点集为 $\{A, B, C, D, E\}$,边及其权重如下:
\begin{center}
\begin{tabular}{c|c|c}
边 & 权重 & 描述 \\
\hline
AB & 5 & A到B,权重5 \\
AC & 3 & A到C,权重3 \\
AD & 7 & A到D,权重7 \\
BC & 2 & B到C,权重2 \\
BD & 4 & B到D,权重4 \\
BE & 6 & B到E,权重6 \\
CE & 8 & C到E,权重8 \\
DE & 1 & D到E,权重1 \\
\end{tabular}
\end{center}
该图的邻接表表示如下:
\begin{center}
\begin{tabular}{l|l}
顶点 & 邻接表 \\
\hline
A & $B(5) , C(3) , D(7)$ \\
B & $A(5) , C(2) , D(4) , E(6)$ \\
C & $A(3) , B(2) , E(8)$ \\
D & $A(7) , B(4) , E(1)$ \\
E & $B(6) , C(8) , D(1)$ \\
\end{tabular}
\end{center}
\begin{definition}{图的邻接矩阵}{}
\textbf{1. 无向图的邻接矩阵}:
设无向图 $G = (V, E)$ 有 $n$ 个顶点,顶点集 $V = \{v_1, v_2, \cdots, v_n\}$。$G$ 的邻接矩阵是一个 $n \times n$ 的矩阵 $A = [a_{ij}]$,其中$a_{ij} = \text{顶点 } v_i \text{ 与 } v_j \text{ 之间的边数}$\\
\textbf{2. 有向图的邻接矩阵}:
设有向图 $D = (V, A)$ 有 $n$ 个顶点,顶点集 $V = \{v_1, v_2, \ldots, v_n\}$。$D$ 的邻接矩阵是一个 $n \times n$ 的矩阵 $A = [a_{ij}]$,其中:$a_{ij} = \text{从顶点 } v_i \text{ 到 } v_j \text{ 的边数}$\\
\textbf{注意自环}:在无向图中,每个自环使对应对角线元素增加2;在有向图中,每个自环使对应对角线元素增加1
\end{definition}
无向图 $G$:顶点集 $\{1,2,3\}$,边集 $\{12, 13, 13\}$(两条13边) ,邻接矩阵:\[
A = \begin{bmatrix}
0 & 1 & 2 \\
1 & 0 & 0 \\
2 & 0 & 0
\end{bmatrix}\]
注意,对于一个$n$阶图,由于$n$个顶点有$n!$种不同的排列方式,所以该图就有$n!$个不同的邻接矩阵。从有向图的邻接矩阵定义可以看出,其某一行的元素之和表示该行对应的结点的出度,某一列的元素之和表示该列对应的结点的入度。矩阵中所有元素求和等于总度数,也等于边数。
\begin{theorem}{邻接矩阵的性质}{}
设$A$是有向图$G=(V,E)$的邻接矩阵,矩阵幂 $A^k$ 的元素 $a_{ij}^{(k)}$ 表示从 $v_i$ 到 $v_j$ 长度为 $k$ 的路径数(考虑平行边时计算的是路径的数量)
\end{theorem}
\textbf{基础情况:} 当 $k=1$ 时,$A^1 = A$,根据邻接矩阵的定义,$a_{ij}$ 表示从 $v_i$ 到 $v_j$ 的边数,即长度为1的路径数。基础情况成立。

\textbf{归纳假设:} 假设当 $k = m$ 时,$A^m$ 的元素 $a_{ij}^{(m)}$ 表示从 $v_i$ 到 $v_j$ 长度为 $m$ 的路径数。

\textbf{归纳步骤:} 考虑 $k = m+1$ 的情况。根据矩阵乘法定义:
\[
a_{ij}^{(m+1)} = \sum_{l=1}^{n} a_{il}^{(m)} \cdot a_{lj}
\]
其中$a_{il}^{(m)}$ 表示从 $v_i$ 到 $v_l$ 长度为 $m$ 的路径数,$a_{lj}$ 表示从 $v_l$ 到 $v_j$ 长度为1的路径数(即边数)

因此,对于每个中间顶点 $v_l$,从 $v_i$ 到 $v_j$ 且经过 $v_l$ 作为第 $m$ 个顶点的长度为 $m+1$ 的路径数为 $a_{il}^{(m)} \cdot a_{lj}$。对所有可能的中间顶点 $v_l$ 求和,即得到从 $v_i$ 到 $v_j$ 长度为 $m+1$ 的总路径数。
\begin{theorem}{包含所有可能的路径信息}{}
    设$A$是有向图$G=(V,E)$的邻接矩阵,$|V|=n$。考虑矩阵$S_k = I + A + A^2 + \cdots + A^k$,其中$I$是单位矩阵,则当$k \geqslant n-1$时,$S_k$包含了所有可能的路径信息,即如果从$v_i$到$v_j$是可达的,则存在$m \leqslant n-1$使得$a_{ij}^{(m)} > 0$
\end{theorem}
\begin{theorem}{邻接矩阵判断有向图连通性}{}
设$D=(V,A)$是一个有$n$个顶点的有向图,$A$是其邻接矩阵,$I$是$n\times n$单位矩阵。

\textbf{1. 强连通性判定:}
$D$是强连通的当且仅当矩阵$\bm{(I+A)^{n-1}}$的所有非对角线元素都大于0

\textbf{2. 单向连通性判定:}
$D$是单向连通的当且仅当矩阵$\bm{(I+A)^{n-1} + (I+A^T)^{n-1}}$的所有非对角线元素都大于0。


\textbf{3. 弱连通性判定:}
$D$是弱连通的当且仅当矩阵$\bm{(I+A+A^T)^{n-1}}$的所有非对角线元素大于0
\end{theorem}
考虑一个有向图,一个结点到另一个结点的路径长小于结点数,所以只需将幂次设置为结点数减一,如果这样操作后得到的矩阵满足所有非对角线元素都大于0,所以所有非对角元素为正意味着任意两个不同顶点互相可达,所以该有向图是强连通的。

如果基础无向图是连通的,则对于任意$i \neq j$,存在从$i$到$j$的路径。单向连通性使用$\bm{(I+A)^{n-1} + (I+A^T)^{n-1}}$,分别考虑原图和转置图的可达性,如果两者的和(这里可以看作析取,或者布尔和)满足非对角线元素大于0,则表明任意不同的两个结点之间都至少存在一条路径,该图是单向连通的。

矩阵$\bm{A+A^T}$表示有向图$D$的基础无向图的邻接矩阵,对于任意两个顶点$i$和$j$,$\bm{(A+A^T)_{ij}}$表示在基础无向图中$i$和$j$之间的边数。矩阵$\bm{(I+A+A^T)^{n-1}}$中的元素$(i,j)$表示从顶点$i$到顶点$j$的长度不超过$n-1$的路径数。由于基础无向图中简单路径的最大长度为$n-1$,这个矩阵包含了所有可能的连通性信息
\begin{theorem}{同构图与邻接矩阵的关系}{}
设$G$和$H$是两个简单图(有向图或无向图),$\bm{A_G}$和$\bm{A_H}$分别是它们的邻接矩阵。$G$和$H$是同构的,记作$G \cong H$,等且仅当存在一个置换矩阵(每一行和每一列恰好有一个1,其余为0的矩阵)$P$使得:
\[
\bm{A_G = P A_H P^T}
\]
其中$P^T$是$P$的转置矩阵。也就是说,两个图$G$和$H$同构当且仅当它们的邻接矩阵可以通过一系列相同的行变换和列变换相互转化。
\end{theorem}
在图同构问题中,左乘置换矩阵 $P$ 对应于对图的顶点进行重新标号。具体来说:如果 $A_H$ 是图 $H$ 的邻接矩阵,那么 $PA_H$ 对应于将 $H$ 的顶点按照置换 $P$ 重新标号后的图的邻接矩阵。而完整的变换 $PA_HP^T$ 则同时进行了顶点重标号和相应的边端点调整

\begin{definition}{可达矩阵}{}
设 $G = (V, E)$ 是一个有向图,其中 $V = \{v_1, v_2, \ldots, v_n\}$ 是顶点集。图 $G$ 的可达矩阵是一个 $n \times n$ 的 0-1 矩阵 $P(\bm{G}) = [r_{ij}]$,定义为:
\[
r_{ij} = 
\begin{cases}
1, & \text{如果从顶点 } v_i \text{ 到 } v_j \text{ 存在一条路径(长度可以为0)} \\
0, & \text{否则}
\end{cases}
\]
通常约定每个顶点到自身是可达的,即 $r_{ii} = 1$(对应长度为0的路径)。对于无向图,可达矩阵是对称矩阵,对于有向图,可达矩阵可能不对称
\end{definition}
所以,可以先计算$\bm{B}=\bm{A}+\bm{A^2}+\cdots+\bm{A^{n-1}}$,然后将这个矩阵的大于0的元素全部替换成1,得到的矩阵就是可达矩阵。

定义矩阵 $\bm{P'} = \bm{P} \land \bm{P^T}$,其中 $\land$ 表示逐元素的逻辑与运算(即对应位置同时为1时结果为1,否则为0)。那么矩阵 $\bm{P'}$ 是实对称的,且具有以下重要性质:

矩阵 $\bm{P'}$ 的第 $i$ 行中所有为1的列指标 $j$ 对应的顶点 $v_j$ 与顶点 $v_i$ 属于同一个强连通分支。具体来说:对于任意 $i,j$,$\bm{P'}[i,j] = 1$ 当且仅当 $\bm{P}[i,j] = 1$ 且 $\bm{P}[j,i] = 1$,这意味着从顶点 $v_i$ 到 $v_j$ 可达,且从 $v_j$ 到 $v_i$ 也可达。正是强连通性的定义:顶点 $v_i$ 和 $v_j$ 相互可达

\begin{definition}{关联矩阵}{}
\textbf{1. 无向图的关联矩阵}
设无向图 $G = (V, E)$,其中 $V = \{v_1, v_2, \ldots, v_m\}$ 是顶点集,$E = \{e_1, e_2, \ldots, e_n\}$ 是边集。图 $G$ 的关联矩阵是一个 $m \times n$ 的矩阵 $M = [m_{ij}]$,定义为:
\[
m_{ij} = 
\begin{cases}
1, & \text{如果顶点 } v_i \text{ 与边 } e_j \text{ 相关联} \\
0, & \text{否则}
\end{cases}
\]
\textbf{2. 有向图的关联矩阵}
设有向图 $D = (V, A)$,其中 $V = \{v_1, v_2, \ldots, v_m\}$ 是顶点集,$A = \{a_1, a_2, \ldots, a_n\}$ 是弧集。图 $D$ 的关联矩阵是一个 $m \times n$ 的矩阵 $M = [m_{ij}]$,定义为:
\[
m_{ij} = 
\begin{cases}
1, & \text{如果弧 } a_j \text{ 以顶点 } v_i \text{ 为起点} \\
-1, & \text{如果弧 } a_j \text{ 以顶点 } v_i \text{ 为终点} \\
0, & \text{否则}
\end{cases}
\]
\end{definition}
在实际解题过程中,要么按结点去写关联矩阵,要么按边去写关联矩阵。

无向图的关联矩阵的特点:每列恰好有两个1(因为每条边连接两个顶点,但是如果有自环,则这条自环对应的列只会出现一个2而不是两个1,其他元素为0),每行中1的个数等于对应顶点的度数,所有行向量之和为零向量(模2运算下),关联矩阵的秩为 $m - k$,其中 $k$ 是连通分支数。孤立点对应的行全为0,多重边对应的列相同。

有向图的关联矩阵的特点:每列恰好有一个1和一个-1(因为每条弧有一个起点和一个终点,自环就相互抵消)。每行中1的个数等于对应顶点的出度,$-1$的个数等于入度,所有行向量之和为零向量。关联矩阵的秩为 $m - k$,其中 $k$ 是连通分支数














\newpage
\section{特殊图}
\begin{definition}{欧拉图及相关概念}{}
\textbf{1. 欧拉通路}:
图 $G$ 中的一条通路经过图 $G$ 中的每条边恰好一次,则称该通路为欧拉通路。

\textbf{2. 欧拉回路:}
图 $G$ 中的一条回路经过图 $G$ 中的每条边恰好一次,且起点和终点相同,,则称该通路为欧拉回路。

\textbf{3. 欧拉图:}
包含欧拉回路的图称为欧拉图。

\textbf{4. 半欧拉图:}
包含欧拉通路但不包含欧拉回路的图称为半欧拉图。

\textbf{5. 有向欧拉通路:}
有向图 $D$ 中的一条有向通路经过图 $D$ 中的每条有向边恰好一次,则称为有向欧拉通路。

\textbf{6. 有向欧拉回路:}
有向图 $D$ 中的一条有向回路经过图 $D$ 中的每条有向边恰好一次,且起点和终点相同,则称为有向欧拉回路。

\textbf{7. 有向欧拉图:}
包含有向欧拉回路的有向图称为有向欧拉图。

\textbf{8. 有向半欧拉图:}
包含有向欧拉通路但不包含有向欧拉回路的有向图称为有向半欧拉图。
\end{definition}
\begin{theorem}{欧拉图和半欧拉图的判定定理}{}
设 $G$ 是一个连通的无向图,$D$ 是一个(强或弱)连通的有向图。

\textbf{1. 无向图欧拉图判定定理:}
连通无向图 $G$ 是欧拉图(存在欧拉回路)当且仅当 $G$ 中每个顶点的度数都是偶数。

\textbf{2. 无向图半欧拉图判定定理:}
连通无向图 $G$ 是半欧拉图(存在欧拉通路但不存在欧拉回路)当且仅当 $G$ 中恰好有两个顶点的度数是奇数,其余顶点的度数都是偶数。

\textbf{3. 有向图欧拉图判定定理:}
强连通有向图 $D$ 是欧拉图(存在有向欧拉回路)当且仅当 $D$ 中每个顶点的入度等于出度。(强连通的有向图不一定是欧拉图)

\textbf{4. 有向图半欧拉图判定定理:}
有向图 $D$ 是半欧拉图(存在有向欧拉通路但不存在有向欧拉回路)当且仅当:$D$ 是弱连通的(基础无向图连通);恰好有一个顶点的出度比入度大1(作为通路起点);恰好有一个顶点的入度比出度大1(作为通路终点);其余顶点的入度等于出度。
\end{theorem}
\begin{theorem}{奇度数结点的性质}{}
    设 $G$ 是一个连通无向图,有$k$个奇度数结点,则$G$的边集可以划分为$\dfrac{k}2$条基本通路,而不可能划分成比这个更少的简单通路。
\end{theorem}
\textbf{证明:}首先证明可以划分为 $\frac{k}{2}$ 条简单通路,由于图中奇度顶点的个数 $k$ 必为偶数(握手定理的推论),我们可以将 $k$ 个奇度顶点配对,得到 $\frac{k}{2}$ 对顶点。对于每一对奇度顶点,我们可以在它们之间添加一条新边(这些新边构成边集 $E'$)。这样得到的新图 $G' = G \cup E'$ 中所有顶点的度数都是偶数。

由于 $G$ 连通,$G'$ 也连通,且每个顶点度数为偶数,因此 $G'$ 存在欧拉回路。在这个欧拉回路中,移除添加的边集 $E'$,则这条欧拉回路被分割成 $\frac{k}{2}$ 条简单通路,这些通路覆盖了 $G$ 的所有边。
    
\textbf{最优性证明(不能划分成少于 $\frac{k}{2}$ 条简单通路):}考虑任意一种将 $G$ 的边集划分为简单通路的划分。每条简单通路的端点必须是奇度顶点(因为通路中非端点的顶点度数为偶数)。由于 $G$ 有 $k$ 个奇度顶点,而每条简单通路最多消耗 2 个奇度顶点作为端点,因此至少需要 $\frac{k}{2}$ 条简单通路才能覆盖所有奇度顶点。
    
因此,不可能用少于 $\frac{k}{2}$ 条简单通路覆盖 $G$ 的所有边。
\begin{theorem}{连通有向图与欧拉图的联系}{}
若一个有向图是欧拉图(即存在一条有向欧拉回路),则它必是强连通的。\\
反之不成立:存在强连通的有向图不是欧拉图。\\
若一个简单有向图是欧拉图(即存在一条有向欧拉回路),则它必是强连通的。\\
反之不成立:存在强连通的简单有向图不是欧拉图。
\end{theorem}
设 $D$ 是一个有向欧拉图,则存在一条有向欧拉回路 $C$,它经过 $D$ 的每条弧恰好一次,且起点和终点相同。对于任意两个顶点 $u$ 和 $v$,由于 $C$ 经过所有顶点,$u$ 和 $v$ 都在 $C$ 上。在 $C$ 上,从 $u$ 出发沿着 $C$ 的方向可以到达 $v$,从 $v$ 出发沿着 $C$ 的方向也可以到达 $u$(因为 $C$ 是回路)。因此,$u$ 和 $v$ 相互可达,故 $D$ 是强连通的。

考虑一个简单有向图,它包含一个回路,这个回路经过所有顶点(即哈密顿回路),从而保证强连通性。但是,我们可以在某些顶点上额外添加出边或入边,使得入度和出度不相等。例如,考虑一个具有三个顶点的有向图:顶点A、B、C。我们构造一个哈密顿回路:A→B→C→A,这样图是强连通的。然后,我们添加一条额外的弧,比如从A到B。这样,顶点A的出度比入度大1,顶点B的入度比出度大1,而顶点C的入度和出度相等。因此,这个图是强连通的,但不是欧拉图。
\begin{definition}{哈密顿图及相关概念}{}
    \textbf{1. 哈密顿通路:}
    在图 $G$ 中,一条经过每个顶点恰好一次的通路称为哈密顿通路。(规定平凡图是哈密顿图)

    \textbf{2. 哈密顿回路:}
    在图 $G$ 中,一条经过每个顶点恰好一次的回路称为哈密顿回路。该回路中,起点和终点重合,但在计数时起点只计算一次。

    \textbf{3. 哈密顿图:}
    若图 $G$ 中存在一条哈密顿回路,则称 $G$ 为哈密顿图。

    \textbf{4. 半哈密顿图:}
    若图 $G$ 中存在哈密顿通路,但不存在哈密顿回路,则称 $G$ 为半哈密顿图。
\end{definition}
上述定义适用于无向图与有向图。对于有向图,哈密顿通路或回路必须遵循有向边的方向。

哈密顿通路(回路)要求遍历所有顶点恰好一次,而边可以不被全部使用,每条边至多被使用一次。这与欧拉通路(回路)要求遍历所有边恰好一次形成对比。
\begin{theorem}{哈密顿图的性质(用于证明一个图不是哈密顿图)}{}
    设无向图 $G=(V,E)$是哈密顿图,则满足:对于结点集 $V$ 中的真子集 $S$,均有$W(G-S)\leqslant |S|$,其中,$W(G-S)$表示$G-S$的导出子图的连通分支数。

    所以如果我们要证明一个图不是哈密顿图,则只需要删除某些特定的结点,使得删除后该图的连通分支数多于删除的结点数量即可。
\end{theorem}
\textbf{证明:}
    设 $C$ 是 $G$ 中的一个哈密顿回路,因此 $C$ 包含 $G$ 的所有顶点。从 $C$ 中删除顶点集 $S$ 后,回路 $C$ 被分割成至多 $|S|$ 条不交的路径。即
    \[
    W(C - S) \leqslant |S|
    \]
删除 $S$ 后,$C - S$ 是 $G - S$ 的生成子图(包含 $G - S$ 的所有顶点)。因此,$G - S$ 的连通分支数不大于 $C - S$ 的连通分支数,即
    \[
    W(G - S) \leqslant W(C - S)
    \]
    结合以上两式,得
    \[
    W(G - S) \leqslant |S|
    \]
\begin{example}{推论}{}
    设$G$是无向连通图,如果$G$中有桥或者割点,则$G$不是哈密顿图。
\end{example}
\begin{theorem}{哈密顿图的充分条件1(用于证明一个图是哈密顿图)}{}
设图 $G$ 是结点数为$n$的简单无向图,对于$G$的每一对不邻接结点$u$和$v$,满足$d(u)+d(v)\geqslant n$,那么图$G$是哈密顿图。
\end{theorem}
\textbf{证明:}
采用反证法。假设 $G$ 满足定理条件但不是哈密顿图。在 $G$ 上添加边,直到 $G$ 变成一个极大非哈密顿图,即 $G$ 本身不是哈密顿图,但添加任意一条新边后得到的图都是哈密顿图。

设 $u$ 和 $v$ 是 $G$ 中一对不相邻的顶点。由于 $G$ 是极大非哈密顿图,添加边 $uv$ 后得到的图 $G'$ 是哈密顿图。设 $C$ 是 $G'$ 的一个哈密顿回路,则 $uv$ 必在 $C$ 中(否则 $G$ 就是哈密顿图)。于是,$C$ 可表示为:
\[
C: u = v_1 \to v_2 \to \cdots \to v_{n-1} \to v_n = v \to u
\]
其中 $v_i$ 是 $G$ 的顶点,且 $v_1 v_n$ 是 $C$ 中的边。

考虑与 $u$ 相邻的顶点集 $S = \{v_i \mid uv_{i} \in E(G)\}$ 和与 $v$ 相邻的顶点集 $T = \{v_{i-1} \mid v v_i \in E(G)\}$。由于 $G$ 中不含边 $uv$,$S$ 和 $T$ 是 $V(G)$ 的两个子集,且 $v_n = v \notin S$,$v_1 = u \notin T$。

由条件 $d(u) + d(v) \ge n$ 知 $|S| + |T| \ge n$。另一方面,$|S \cup T| \le n-1$(因为 $v_n = v \notin S$ 且 $v_1 = u \notin T$),因此
\[
|S \cap T| = |S| + |T| - |S \cup T| \ge n - (n-1) = 1
\]
故存在某个 $v_k \in S \cap T$。这意味着 $uv_k \in E(G)$ 且 $v v_{k+1} \in E(G)$(由 $T$ 的定义)。于是,$G$ 中存在哈密顿回路:
\[
u = v_1 \to v_2 \to \cdots \to v_{k} \to v_n = v \to v_{n-1} \to \cdots \to v_{k+1} \to u
\]
与 $G$ 是非哈密顿图矛盾。因此假设不成立,$G$ 是哈密顿图。
\begin{theorem}{哈密顿图的充分条件2(用于证明一个图是哈密顿图)}{}
设图 $G$ 是结点数为$n$的简单无向图,$G$中每个结点的度数都至少为$\dfrac{n}2$,则图$G$是哈密顿图。
\end{theorem}
\textbf{证明:}由于 $G$ 中每个结点的度数至少为 $\dfrac{n}{2}$,则对于任意两个不相邻的结点 $u$ 和 $v$,有
    \[
    d(u) + d(v) \geqslant \frac{n}{2} + \frac{n}{2} = n.
    \]
由上一个定理可知,$G$ 是哈密顿图。
\begin{example}{}{}
    设$G$是$n$阶图无向简单图,边数的两倍$2e=(n-1)(n-2)+4$,则$G$是哈密顿图。
\end{example}
\textbf{证明:} 假设 $G$ 不是哈密顿图。由上面定理的否定,存在不相邻的顶点 $u$ 和 $v$,使得
    \[
    d(u) + d(v) \le n-1.
    \]
    考虑子图 $G' = G - \{u, v\}$,即删除顶点 $u, v$ 及与其关联的边。$G'$ 有 $n-2$ 个顶点。由于 $u$ 和 $v$ 不相邻,删除的边数恰好为 $d(u) + d(v)$,因此 $G'$ 的边数
    \[
    e' = e - (d(u) + d(v)) \ge e - (n-1).
    \]
    另一方面,$G'$ 是 $n-2$ 阶简单图,其边数至多为 $\binom{n-2}{2}$,即
    \[
    e' \le \binom{n-2}{2}.
    \]
    由条件 $2e = (n-1)(n-2) + 4$,得
    \[
    e = \frac{(n-1)(n-2)}{2} + 2.
    \]
    因此,
    \[
    e' \ge \frac{(n-1)(n-2)}{2} + 2 - (n-1) = \frac{(n-1)(n-2) - 2(n-1)}{2} + 2 = \frac{(n-1)(n-4)}{2} + 2.
    \]
    同时,
    \[
    \binom{n-2}{2} = \frac{(n-2)(n-3)}{2}.
    \]
    于是有
    \[
    \frac{(n-1)(n-4)}{2} + 2 \le \frac{(n-2)(n-3)}{2}.
    \]
    等价变形知道此式一定不成立。因此假设不成立,$G$ 是哈密顿图。
\begin{theorem}{半哈密顿图的充分条件}{}
    设图 $G$ 是结点数为$n$的简单无向图,对于$G$的每一对不邻接结点$u$和$v$,满足$d(u)+d(v)\geqslant n-1$,那么图$G$是半哈密顿图。
\end{theorem}
\textbf{证明:}
    分两步进行。

    \textbf{第一步:证明 $G$ 是连通图。}
    用反证法。假设 $G$ 不连通,则存在至少两个连通分支 $C_1$ 和 $C_2$。设 $|C_1| = a$,$|C_2| = b$,则 $a + b \le n$。在 $C_1$ 和 $C_2$ 中分别取顶点 $u$ 和 $v$,显然 $u$ 和 $v$ 不相邻。由于 $C_1$ 和 $C_2$ 是 $G$ 的连通分支,$u$ 的度数不超过 $a-1$,$v$ 的度数不超过 $b-1$,于是
    \[
    d(u) + d(v) \le (a-1) + (b-1) = a + b - 2 \le n-2
    \]
    这与条件 $d(u) + d(v) \ge n-1$ 矛盾。故 $G$ 是连通图。

    \textbf{第二步:构造哈密顿通路。}
    在 $G$ 中添加一个新顶点 $w$,并将 $w$ 与 $G$ 中所有顶点相连,得到新图 $G'$。显然 $G'$ 有 $n+1$ 个顶点。对 $G'$ 中任意两个不相邻的顶点 $x$ 和 $y$,分两种情况讨论:
         
    若 $x, y$ 均为 $G$ 中的顶点,则在 $G$ 中 $x$ 和 $y$ 不相邻(否则在 $G'$ 中相邻)。由条件 $d_G(x) + d_G(y) \ge n-1$,而 $d_{G'}(x) = d_G(x) + 1$,$d_{G'}(y) = d_G(y) + 1$,故
        \[
        d_{G'}(x) + d_{G'}(y) = d_G(x) + d_G(y) + 2 \ge (n-1) + 2 = n+1
        \]
         若 $x$ 和 $y$ 中有一个是 $w$,不妨设 $x = w$,则 $y$ 是 $G$ 中顶点。由于 $w$ 与所有顶点相邻,$w$ 和 $y$ 在 $G'$ 中相邻,故无需检验条件。

    因此,$G'$ 满足 Ore 条件:对任意不相邻的顶点 $x$ 和 $y$,有 $d_{G'}(x) + d_{G'}(y) \ge n+1$。由上面定理,$G'$ 是哈密顿图,存在哈密顿回路 $C'$。

    在 $C'$ 中删除顶点 $w$,由于 $w$ 在 $C'$ 中与两个顶点相邻,删除 $w$ 后得到 $G$ 的一条哈密顿通路。因此 $G$ 是半哈密顿图。
\begin{definition}{格雷码与n方体图}{}
\textbf{格雷码:}
格雷码是一种二进制编码方式,其中任意两个相邻的码字(包括首尾码字)之间只有一位不同。n位格雷码是长度为$2^n$的循环序列,包含所有$n$位二进制串,且相邻码字仅有一位不同。

\textbf{n方体图:}
n方体图$Q_n$是一个无向图,定义如下:
顶点集$V(Q_n)$是所有长度为$n$的二进制串,共$2^n$个顶点。  
边集$E(Q_n)$:两个顶点$u$和$v$之间有一条边当且仅当它们仅有一位不同。  
$Q_n$是$n$正则图,每个顶点的度数为$n$。
\end{definition}
n位格雷码对应于$Q_n$中的一条哈密顿回路。  
格雷码序列中的每个码字对应$Q_n$的一个顶点。  
格雷码的相邻性(仅一位不同)对应$Q_n$中的边。  
格雷码的循环性(首尾相邻)对应回路。  
由于格雷码包含所有$2^n$个可能的$n$位二进制串,该回路经过每个顶点恰好一次,即哈密顿回路。  
因此,寻找n位格雷码等价于在$Q_n$中寻找哈密顿回路。
\begin{theorem}{不是所有完全图都是哈密顿图}{}
    不是所有完全图都是哈密顿图,如$K_2$不是哈密顿图(没有回路)。
\end{theorem}
\begin{theorem}{格雷码与n方体图的关系}{}
    对于任意$n \geqslant 2$,$Q_n$是哈密顿图,即存在n位格雷码。
\end{theorem}
\textbf{证明:}基础:$n=2$时,$Q_2$是一个四边形,序列00,01,11,10构成2位格雷码。  
归纳:假设已构造出$n$位格雷码$G_n$,则$n+1$位格雷码可构造为$0G_n$后接$1G_n^R$($G_n^R$是$G_n$的逆序)。这对应在$Q_{n+1}$中复制两个$Q_n$并在对应顶点间添加边形成哈密顿回路。
\begin{theorem}{}{}
    所有的$n$方体图都是二分图。
\end{theorem}
\textbf{证明:}
    设 $Q_n$ 的顶点集 $V$ 是所有长度为 $n$ 的二进制串,即
    \[
    V = \{ (x_1, x_2, \dots, x_n) \mid x_i \in \{0, 1\} \text{ 对所有 } i \}.
    \]
    两个顶点相邻当且仅当它们的二进制串恰好有一位不同。

    定义顶点集合的划分如下:
    \begin{align*}
    A &= \{ v \in V \mid v \text{ 的二进制表示中 1 的个数为偶数} \}, \\
    B &= \{ v \in V \mid v \text{ 的二进制表示中 1 的个数为奇数} \}.
    \end{align*}
    显然 $A \cup B = V$ 且 $A \cap B = \varnothing$。

    对任意一条边 $e = uv \in E(Q_n)$,$u$ 和 $v$ 的二进制串恰好有一位不同。因此,$u$ 和 $v$ 的二进制串中 1 的个数相差 1,即一者为偶数,另一者为奇数。故 $u$ 和 $v$ 分别属于 $A$ 和 $B$ 中的不同集合。

    由二分图的定义可知,$Q_n$ 是二分图,且 $(A, B)$ 是其一个二部划分。
\begin{theorem}{}{}
    完全二分图$k_{m,n}$是哈密顿图当且仅当$m=n\geqslant 2$。
\end{theorem}
设 $K_{m,n}$ 的顶点划分为 $X$ 和 $Y$,其中 $|X| = m$,$|Y| = n$。

    \textbf{必要性:} 假设 $K_{m,n}$ 是哈密顿图,则存在哈密顿回路 $C$。由于 $K_{m,n}$ 是二分图,$C$ 上的顶点必须交替属于 $X$ 和 $Y$。因此 $C$ 中 $X$ 和 $Y$ 的顶点数相等,即 $m = n$。又因为哈密顿回路至少包含 3 个顶点,故 $m = n \ge 2$(当 $m=n=1$ 时,$K_{1,1}$ 是一条边,不构成回路)。

    \textbf{充分性:} 假设 $m = n \ge 2$。令 $X = \{x_1, x_2, \dots, x_m\}$,$Y = \{y_1, y_2, \dots, y_m\}$。构造回路
    \[
    C = x_1 y_1 x_2 y_2 \cdots x_m y_m x_1.
    \]
    由于 $K_{m,m}$ 是完全二分图,相邻顶点之间均有边,故 $C$ 是 $K_{m,m}$ 中的一个哈密顿回路。因此 $K_{m,m}$ 是哈密顿图。
\section{旅行商问题}
\begin{definition}{旅行商问题的图论描述}{}
设 $G=(V,E)$ 是一个完全无向图,其中 $V=\{v_1,v_2,\ldots,v_n\}$ 是顶点集,表示城市;$E$ 是边集,每一条边 $\{v_i,v_j\}$ 有一个非负权值 $w_{ij}$,表示城市 $v_i$ 和 $v_j$ 之间的距离。旅行商问题要求找到一个哈密顿回路 $C$,即经过每个顶点恰好一次并回到起点的回路,使得回路的总权值 $\displaystyle w(C)=\sum_{\{v_i,v_j\}\in C} w_{ij}$ 最小。

若 $G$ 不是完全图,通常先将其补全为完全图,缺失的边权设为无穷大或一个足够大的数。

旅行商问题的判定形式:给定一个完全图 $G$ 和正整数 $k$,是否存在一个哈密顿回路,其总权值不超过 $k$?
\end{definition}
\section{最短路径问题与Dijkstra算法}
\begin{definition}{最短路径问题的图论描述}{}
    给定一个带权图 $G=(V,E,w)$,其中 $V$ 是顶点集,$E$ 是边集,$w: E \to \mathbb{R}$ 是边的权值函数(通常 $w(e) \geqslant 0$,但某些算法允许负权值)。设路径 $P = (v_0, v_1, \ldots, v_k)$ 的权值为 $w(P) =\displaystyle \sum_{i=0}^{k-1} w(v_i, v_{i+1})$。最短路径问题旨在找到两点之间权值最小的路径。
\end{definition}
下面介绍一种求解最短路径问题的算法——Dijkstra算法(边权不可以为负数)。

Dijkstra算法用于求解带非负权边的图的单源最短路径问题。给定一个带权图$G=(V,E,w)$,其中$w(e)\geqslant 0$,以及源顶点$s\in V$,算法求出从$s$到所有其他顶点的最短距离。

\textbf{算法步骤:}\\
1. 初始化距离数组$dist$:$dist[s]=0$,$dist[v]=\infty$($v\neq s$)。\\
2. 初始化优先队列$Q$,将$s$入队。\\
3. 当$Q$非空时,从$Q$中取出距离最小的顶点$u$。对于$u$的每个邻接点$v$,计算新路径长度$dist[u]+w(u,v)$。如果$dist[u]+w(u,v)<dist[v]$,则更新$dist[v]=dist[u]+w(u,v)$,将$v$的前驱设为$u$,并将$v$加入$Q$。\\
4. 算法结束时,$dist[v]$即为$s$到$v$的最短距离。

\textbf{算法性质:}仅适用于边权非负的图。每次从队列中取出的顶点,其最短距离已确定。可记录前驱顶点以重构最短路径。

\textbf{示例:}计算从A到各顶点的最短路径,图如下:\\
顶点:A, B, C, D\quad 边:A→B(4), A→D(1), B→C(2), D→C(3), D→B(2)\\
步骤:\\
1. 初始:dist[A]=0, 其他为$\infty$。\\
2. 取出A,更新邻接点:dist[B]=4, dist[D]=1。
\\
3. 取出D(距离最小),更新邻接点:dist[C]=4, dist[B]=3(D到B的距离为1+2=3,小于4,更新)。\\
4. 取出B,更新邻接点C:dist[C]=min(4, 3+2)=4。\\
5. 取出C,无更新。\\
最终最短距离:A→B:3, A→D:1, A→C:4。

\section{中国邮路问题}
\begin{definition}{中国邮路问题}{}
    中国邮路问题是指:给定一个连通的无向图 $G=(V,E)$,每条边 $e\in E$ 有一个非负权值 $w(e)$(表示长度)。求一个回路,使其经过每条边至少一次,且总权值最小。

    \textbf{问题描述:}
    设邮递员从邮局出发,走遍他所负责的投递区域的每一条街道,最后返回邮局。问如何选择路线,使得总路程最短?

    \textbf{图论模型:}
    将街道视为边,街道交叉点视为顶点,街道长度视为边权。问题转化为在带权连通无向图中寻找一条经过每条边至少一次的闭途径,且权值最小。

    \textbf{欧拉图情形:}
    如果图 $G$ 是欧拉图(每个顶点的度数都是偶数),则存在欧拉回路,欧拉回路经过每条边恰好一次,即为最优解。

    \textbf{非欧拉图情形:}
    如果图 $G$ 不是欧拉图,则奇度顶点的个数必为偶数。此时,邮递员需要重复走过某些边,使得最终走过的图中每个顶点的度数(包括重复走过的边)都是偶数。问题转化为:添加一些重复边(即重复走过某些边),使得得到的图是欧拉图,并且添加的边的总权值最小。
\end{definition}
设 $G=(V,E,w)$ 是一个连通的带权无向图,其中 $w: E \to \mathbb{R}^+$ 是边权函数。中国邮路问题要求找到一条经过每条边至少一次的闭回路,且其总权值最小。

由握手定理,图中奇度结点的个数必为偶数。记奇度结点集为 $V_{\text{odd}}$,$|V_{\text{odd}}|$ 为偶数。

\textbf{基本思路:} 通过添加重复边(称为附加边)使所有结点度数变为偶数,从而得到欧拉图,再求其欧拉回路。添加的重复边应使其总权值最小。

\textbf{解法步骤:}
1. 找出 $G$ 中所有奇度结点,构成集合 $V_{\text{odd}}$。\\
    2. 将 $V_{\text{odd}}$ 中的结点两两配对。对于每一对结点,求出它们之间的最短路(权值最小的路径)。\\
    3. 将所有最短路上的边作为附加边添加到原图 $G$ 中,每条边每出现在一条最短路上就增加一条重复边,形成多重图 $G_1 = G + E_1$,其中 $E_1$ 是附加边的集合。此时 $G_1$ 中所有结点度数均为偶数,因此 $G_1$ 是欧拉图。\\
    4. 在 $G_1$ 上求一条欧拉回路,该回路经过原图 $G$ 的每条边至少一次(重复边代表重复经过),回路的总权值为原图所有边权加上附加边权值之和。\\
    5. 为了使总权值最小,需要选择配对方式使得附加边的总权值 $w(E_1)$ 最小。这等价于将奇度结点配对,使得每对结点之间最短路长度之和最小。

\begin{theorem}{最优性条件}{}
    设 $G_1 = G + E_1$ 是通过添加附加边 $E_1$ 得到的欧拉图,则 $w(E_1)$ 最小的充分必要条件是:\\$G_1$ 中任意一条边至多被重复一次(即原图的每条边在 $E_1$ 中至多出现一次)。\\$G_1$ 中任意一个回路上,属于 $E_1$ 的边的权值之和不大于该回路总权值的一半。
\end{theorem}
    这个条件保证了添加的重复边是最优的:如果某个回路上重复边的权值和超过回路总权值的一半,则可以将该回路上原属于 $E_1$ 的边从 $E_1$ 中移除,而将回路上原本不属于 $E_1$ 的边加入 $E_1$,这样仍使图变为欧拉图,但 $w(E_1)$ 减小,矛盾。

    \textbf{调整方法:}
    基于上述条件,可以通过以下步骤调整附加边集 $E_1$:
    
    检查 $G_1$ 中每条边是否被重复多于一次,若是,则减少重复次数至一次。
    
    检查 $G_1$ 中的每个回路,若某个回路上重复边的权值和大于回路总权值的一半,则进行“翻转”:将该回路上属于 $E_1$ 的边移出 $E_1$,而将回路上不属于 $E_1$ 的边加入 $E_1$。
    
    这样调整后,每个奇度结点的度数奇偶性不变(因为回路上每个结点的度数变化为0),图仍为欧拉图,但 $w(E_1)$ 减小。
         
    重复上述检查与调整,直到满足定理条件时为止。此时得到的 $E_1$ 即为权值最小的附加边集。

    \textbf{最后,} 在调整后的欧拉图 $G_1 = G + E_1$ 上求一条欧拉回路,即为中国邮路问题的最优解。

    \textbf{注:} 上述调整过程本质上是在寻找奇度结点之间的最优配对,使得每对结点之间最短路长度之和最小。这等价于在奇度结点构成的完全图上求最小权完美匹配,但此处避免了“完美匹配”的术语,而通过回路条件进行调整。
\begin{theorem}{最少添加边数}{}
    当连通图$G$有$2k$个奇度结点时,最少需要添加的边数为$k$.
\end{theorem}
\begin{center}
\pagestyle{empty}
\definecolor{ududff}{rgb}{0.30196078431372547,0.30196078431372547,1.}
\begin{tikzpicture}[line cap=round,line join=round,>=triangle 45,x=1.0cm,y=1.0cm]
% 紧凑的裁剪框
\clip(-10,-4) rectangle (0,2);
\draw [line width=2.pt] (-9.28,-1.1)-- (-7.4,0.88);
\draw [line width=2.pt] (-9.28,-1.1)-- (-7.48,-1.1);
\draw [line width=2.pt] (-7.4,0.88)-- (-7.48,-1.1);
\draw [line width=2.pt] (-7.48,-1.1)-- (-7.48,-3.06);
\draw [line width=2.pt] (-9.28,-1.1)-- (-7.48,-3.06);
\draw [line width=2.pt] (-7.48,-1.1)-- (-4.8,-1.08);
\draw [line width=2.pt] (-7.48,-3.06)-- (-4.8,-3.02);
\draw [line width=2.pt] (-7.4,0.88)-- (-4.72,0.92);
\draw [line width=2.pt] (-4.72,0.92)-- (-4.8,-1.08);
\draw [line width=2.pt] (-4.8,-1.08)-- (-7.48,-3.06);
\draw [line width=2.pt] (-4.8,-1.08)-- (-4.8,-3.02);
\draw [line width=2.pt] (-4.8,-1.08)-- (-2.36,-1.04);
\draw [line width=2.pt] (-4.8,-3.02)-- (-2.36,-1.04);
\draw [line width=2.pt] (-2.36,-1.04)-- (-2.36,-2.97);
\draw [line width=2.pt] (-4.72,0.92)-- (-0.68,0.93);
\draw [line width=2.pt] (-4.8,-1.08)-- (-0.68,0.93);
\draw [line width=2.pt] (-0.68,0.93)-- (-2.36,-1.04);
\draw [line width=2.pt] (-0.68,0.93)-- (-2.36,-2.97);
\draw [line width=2.pt] (-4.8,-3.02)-- (-2.36,-2.97);

% 重新调整数字标签位置,让它们不覆盖线段
% 每条边的权重放在中点偏移一点的位置
\draw (-8.6,-0.2) node {7};  % 边AB
\draw (-8.7,-2.1) node {1};  % 边AD
\draw (-8.5,-0.9) node {4};  % 边AC
\draw (-6.6,1.2) node {8};   % 边BE
\draw (-7.6,-0.1) node {2};  % 边BC
\draw (-6.3,-0.9) node {2};  % 边CF
\draw (-5.0,-0.1) node {3};  % 边EF
\draw (-6.2,-1.8) node {2};  % 边DF
\draw (-7.7,-1.8) node {3};  % 边CD
\draw (-6.2,-3.2) node {5};  % 边DG
\draw (-3.6,-3.2) node {1};  % 边GK
\draw (-3.6,-2.1) node {1};  % 边GF
\draw (-2.6,-2.0) node {5};  % 边KM
\draw (-3.8,-1.1) node {3};  % 边FM
\draw (-5.1,-1.9) node {6};  % 边FG
\draw (-3.2,0.1) node {6};   % 边EH
\draw (-2.1,-0.3) node {4};  % 边FH
\draw (-1.4,-1.3) node {9};  % 边HM
\draw (-2.9,1.3) node {8};   % 边GH? 这个看起来是GH边

\begin{scriptsize}
\draw [fill=ududff] (-9.28,-1.1) circle (2.5pt);
\draw[color=ududff] (-9.62,-0.88) node {$A$};
\draw [fill=ududff] (-7.4,0.88) circle (2.5pt);
\draw[color=ududff] (-7.26,1.24) node {$B$};
\draw [fill=ududff] (-7.48,-1.1) circle (2.5pt);
\draw[color=ududff] (-7.34,-0.74) node {$C$};
\draw [fill=ududff] (-7.48,-3.06) circle (2.5pt);
\draw[color=ududff] (-7.8,-3.16) node {$D$};
\draw [fill=ududff] (-4.72,0.92) circle (2.5pt);
\draw[color=ududff] (-4.58,1.28) node {$E$};
\draw [fill=ududff] (-4.8,-1.08) circle (2.5pt);
\draw[color=ududff] (-4.66,-0.72) node {$F$};
\draw [fill=ududff] (-4.8,-3.02) circle (2.5pt);
\draw[color=ududff] (-4.8,-3.28) node {$G$};
\draw [fill=ududff] (-0.68,0.93) circle (2.5pt);
\draw[color=ududff] (-0.54,1.3) node {$H$};
\draw [fill=ududff] (-2.36,-1.04) circle (2.5pt);
\draw[color=ududff] (-2.46,-0.62) node {$M$};
\draw [fill=ududff] (-2.36,-2.97) circle (2.5pt);
\draw[color=ududff] (-2.02,-3.16) node {$K$};
\end{scriptsize}
\end{tikzpicture}
\end{center}
奇度顶点为 $A, B, E, K$(共4个)。要求从$A$到$H$的欧拉迹,需使最终只有$A$和$H$为奇度,其他为偶度。因此需将 $B, E, K$ 由奇变偶,$H$ 由偶变奇。这需在四顶点间进行两两配对,每对间添加最短路径(重复边走),使端点奇偶性翻转。

3. 计算最短路径距离
利用Dijkstra算法或手工计算,得相关顶点间最短距离:
\begin{align*}
&d(B,H)=10 \quad (\text{路径 } B-C-F-H),\\
&d(E,H)=8 \quad (\text{路径 } E-H),\\
&d(K,H)=6 \quad (\text{路径 } H-M-G-K),\\
&d(B,E)=7 \quad (\text{路径 } B-C-F-E),\\
&d(B,K)=11 \quad (\text{路径 } B-C-F-G-K),\\
&d(E,K)=10 \quad (\text{路径 } E-F-G-K).
\end{align*}

4. 配对方案与最小添加权重
共有三种配对方式:
\begin{enumerate}
    \item $(B,H)$ 与 $(E,K)$:添加权重 $10+10=20$;
    \item $(E,H)$ 与 $(B,K)$:添加权重 $8+11=19$;
    \item $(K,H)$ 与 $(B,E)$:添加权重 $6+7=13$。
\end{enumerate}

方案3添加权重最小,为13。具体添加路径:
\begin{enumerate}
    \item $K$与$H$:路径 $H-M-G-K$,边权 $4+1+1=6$;
    \item $B$与$E$:路径 $B-C-F-E$,边权 $2+2+3=7$。
\end{enumerate}
添加边权总和 $6+7=13$。

5. 最短路程
最短路程 = 原图边权总和 + 添加边权总和 = $80 + 13 = 93$。
\newpage
\section{匹配与二分图}
\begin{definition}{匹配及相关概念}{}
设 $G=(V,E)$ 是一个无向图,$M \subseteq E$ 是边集的一个子集。

\textbf{1. 匹配:}如果 $M$ 中任意两条边都没有公共顶点,称 $M$ 是 $G$ 的一个匹配。

\textbf{2. 极大匹配:}如果 $M$ 是一个匹配,且不存在 $e \in E- M$ 使得 $M \cup \{e\}$ 也是一个匹配,则称 $M$ 是极大匹配。即 $M$ 是不能再通过添加任意一条边来扩大的匹配。

\textbf{3. 最大匹配:}
如果 $M$ 是一个匹配,且 $G$ 中不存在匹配 $M'$ 使得 $|M'| > |M|$,则称 $M$ 是最大匹配。即 $M$ 是边数最多的匹配。

\textbf{4. 匹配数:}
图 $G$ 的最大匹配的边数称为 $G$ 的匹配数,记作 $\nu(G)$ 或 $\mu(G)$。

\textbf{5. 饱和点:}
设 $M$ 是 $G$ 的一个匹配,如果顶点 $v$ 与 $M$ 中的某条边关联,则称 $v$ 是 $M$-饱和的,简称饱和点。

\textbf{6. 非饱和点:}
设 $M$ 是 $G$ 的一个匹配,如果顶点 $v$ 不与 $M$ 中任何边关联,则称 $v$ 是 $M$-非饱和的,简称非饱和点。

\textbf{7. 完美匹配:}
如果匹配 $M$ 饱和了图 $G$ 的所有顶点,即 $G$ 的每个顶点都是 $M$-饱和的,则称 $M$ 是完美匹配。显然,完美匹配的边数为 $\dfrac{|V|}2$,因此 $|V|$ 必须为偶数。

\textbf{8. 完备匹配:}
设 $G$ 是一个二分图,其顶点集划分为 $X$ 和 $Y$。如果存在一个匹配 $M$ 饱和 $X$ 中的所有顶点,则称 $M$ 是从 $X$ 到 $Y$ 的完备匹配。注意:完备匹配不要求饱和 $Y$ 的所有顶点,因此 $|X| \leqslant |Y|$。

\textbf{9. 匹配边:}属于匹配 $M$ 的边称为匹配边。

\textbf{10. 非匹配边:}不属于匹配 $M$ 的边称为非匹配边。

\textbf{11. 交错路:}
设 $P$ 是 $G$ 中的一条路径。如果 $P$ 的边交替地属于 $M$ 和 $E- M$,则称 $P$ 为关于匹配 $M$ 的交错路。

\textbf{12. 可扩充的交错路:}
设 $P$ 是关于匹配 $M$ 的一条交错路。如果 $P$ 的两个端点都是 $M$-非饱和点,则称 $P$ 为关于匹配 $M$ 的可扩充的交错路。
\end{definition}
\textbf{说明:}
极大匹配不一定是最大匹配,但最大匹配一定是极大匹配。完美匹配是最大匹配的特例,要求所有顶点都被饱和。在二分图中,完备匹配是指饱和了其中一个部集的所有顶点。
\begin{theorem}{最大匹配的充要条件}{}
    设 $M$ 是 $G$ 的一个匹配,$M$是最大匹配当且仅当$G$中不存在$M$可扩充路。
\end{theorem}
 \textbf{必要性:} 如果 $M$ 是最大匹配,但存在一条关于 $M$ 的可扩充路 $P$,则可以通过将 $P$ 上的匹配边与非匹配边交换,得到一个新的匹配 $M' = M \oplus E(P)$,其中 $E(P)$ 是 $P$ 的边集。由于可扩充路的两个端点是非饱和点,且非匹配边比匹配边多一条,因此 $|M'| = |M| + 1$,与 $M$ 的最大性矛盾。故不存在可扩充路。

\textbf{充分性:} 假设 $G$ 中不存在关于 $M$ 的可扩充路,但 $M$ 不是最大匹配。设 $M^*$ 是一个最大匹配,且 $|M^*| > |M|$。考虑对称差 $D = M \oplus M^*$,则 $D$ 中的每个连通分量要么是偶长度交替环,要么是交替路。由于 $|M^*| > |M|$,$D$ 中必有一条连通分量是交替路 $P$,且 $P$ 中 $M^*$ 的边比 $M$ 的边多一条。因此,$P$ 的两个端点在 $M$ 中是非饱和的,即 $P$ 是关于 $M$ 的可扩充路,矛盾。故 $M$ 是最大匹配。
\begin{theorem}{最大匹配与最小边覆盖的关系}{}
最小边覆盖:一个图的边覆盖是一个边子集,使得每个顶点都至少与这个子集中的一条边关联。最小边覆盖是指边数最小的边覆盖。

设 $G=(V,E)$ 是一个无孤立顶点的图,$|V|=n$。记 $\alpha'(G)$ 为最大匹配的边数,$\beta'(G)$ 为最小边覆盖的边数。则有:
\[
\alpha'(G) + \beta'(G) = n
\]
\textbf{推论:}
在无孤立点的图中,最大匹配的边数与最小边覆盖的边数之和等于顶点数。因此,已知最大匹配可构造最小边覆盖,反之亦然。
\end{theorem}
\textbf{从最大匹配构造最小边覆盖:}
设 $M$ 是 $G$ 的一个最大匹配,$|M| = \alpha'(G)$。记 $U$ 为 $M$-非饱和点的集合。由于 $G$ 无孤立点,每个 $v \in U$ 至少与一条边关联。对每个 $v \in U$,任取一条与 $v$ 关联的边 $e_v$,记 $F = \{ e_v \mid v \in U \}$。则 $C = M \cup F$ 是一个边覆盖,且 $|C| = |M| + |U|$。因为 $M$ 饱和了 $2|M|$ 个顶点,故 $|U| = n - 2|M|$,于是 $|C| = |M| + n - 2|M| = n - |M|$。

下面证明 $C$ 是最小边覆盖。若存在更小的边覆盖 $C'$,则考虑 $C'$ 的极大匹配 $M'$(从 $C'$ 中删去相邻边直到成为匹配),有 $|M'| \ge |C'| - (n - 2|M'|)$,可得 $|C'| \ge n - |M'|$。但 $|M'| \le |M|$,故 $|C'| \ge n - |M| = |C|$,矛盾。因此 $C$ 是最小边覆盖,且 $\beta'(G) = n - \alpha'(G)$。

 \textbf{从最小边覆盖构造最大匹配:}
设 $C$ 是 $G$ 的一个最小边覆盖,$|C| = \beta'(G)$。考虑 $C$ 的极大匹配 $M$(从 $C$ 中删去相邻边直到成为匹配)。类似地,可证 $|M| = n - \beta'(G)$,且 $M$ 是最大匹配。因此 $\alpha'(G) = n - \beta'(G)$。

综上,$\alpha'(G) + \beta'(G) = n$。

\begin{theorem}{连通无向平面图完美匹配的判定定理}{}
    对于一个连通无向平面图,完美匹配的存在性判定依赖于一般图的 Tutte 定理,但平面图由于其特殊结构,有一些更强的结论。以下是几个重要的判定定理:

    \textbf{1. Tutte 定理(一般图)}:
    图 $G$ 有完美匹配的充分必要条件是:对于任意顶点子集 $S \subseteq V(G)$,$G-S$ 的奇连通分支数 $o(G-S) \le |S|$。

    \textbf{2. 平面图的偶数面条件}:
    设 $G$ 没有割点,且每个面的边界都是偶长回路,则 $G$ 是二分图。此时,$G$ 存在完美匹配的充分必要条件是 $|V(G)|$ 为偶数,且对于任意 $S \subseteq V(G)$,$o(G-S) \le |S|$(即 Tutte 条件)。

    \textbf{3. 平面图的特殊性质}:任何无桥的 3-正则图都有完美匹配。
\end{theorem}
\begin{theorem}{偶面平面图是二分图}{}
    设 $G$ 是连通平面图,且每个面的边界都是偶长回路,则 $G$ 是二分图。
\end{theorem}
    由于 $G$ 连通,可定义顶点染色:任取一顶点 $v$,染黑色;对任意顶点 $u$,若在 $G$ 中存在一条从 $v$ 到 $u$ 长度为偶数的路径,则 $u$ 染黑色,否则染白色。下证此染色是合法的二染色。

    假设存在边 $xy$ 使得 $x$ 和 $y$ 同色。则存在两条从 $v$ 到 $x$ 和 $v$ 到 $y$ 的路径,长度奇偶性相同。考虑 $v$ 到 $x$ 的最短路径 $P_x$ 和 $v$ 到 $y$ 的最短路径 $P_y$,以及边 $xy$,它们构成一个闭途径。由于平面嵌入,该闭途径可分解为若干面的边界。每个面边界长度为偶数,故闭途径长度为偶数。但 $P_x$ 和 $P_y$ 长度奇偶性相同,$P_x$ 加 $xy$ 加 $P_y$ 的反向的长度为奇数,矛盾。因此相邻顶点颜色不同,$G$ 是二分图。

\begin{theorem}{二分图的判定(相异性条件)}{}
    一个$n$阶无向图$G$是二分图,当且仅当$G$中不存在长度为奇数的回路。
\end{theorem}
设 $G=(V,E)$ 是一个 $n$ 阶无向图。

    \textbf{必要性:} 假设 $G$ 是二分图,则存在顶点集的一个划分 $V = X \cup Y$,使得 $G$ 的每条边的两个端点分别属于 $X$ 和 $Y$。考虑 $G$ 中任意一条回路 $C: v_1 v_2 \cdots v_k v_1$。不失一般性,设 $v_1 \in X$。由于边 $v_1v_2$ 的端点在不同部集,故 $v_2 \in Y$。同理,$v_3 \in X$,$v_4 \in Y$,如此交替。一般地,$v_i \in X$ 当 $i$ 为奇数,$v_i \in Y$ 当 $i$ 为偶数。由于 $v_1$ 和 $v_k$ 相邻,且 $v_1 \in X$,则 $v_k \in Y$,故 $k$ 为偶数。因此回路 $C$ 的长度为偶数。所以,二分图中不存在长度为奇数的回路。

    \textbf{充分性:} 假设 $G$ 中不存在长度为奇数的回路。我们证明 $G$ 是二分图。任取一个顶点 $u$,定义染色函数 $c: V \to \{0,1\}$ 如下:从 $u$ 开始进行广度优先搜索(BFS),令 $c(u)=0$,对于任意边 $(x,y)$,若 $x$ 已染色,则令 $c(y)=1-c(x)$。由于 $G$ 连通(若 $G$ 不连通,对每个连通分量分别考虑),此染色可覆盖所有顶点。我们证明此染色是合法的二染色,即任意边 $(x,y) \in E$,有 $c(x) \neq c(y)$。

    假设存在一条边 $(x,y)$ 使得 $c(x)=c(y)$。考虑从 $u$ 到 $x$ 的最短路径 $P_x$ 和从 $u$ 到 $y$ 的最短路径 $P_y$。设 $w$ 是 $P_x$ 和 $P_y$ 的最后一个公共顶点。则 $P_x$ 中从 $w$ 到 $x$ 的段与 $P_y$ 中从 $w$ 到 $y$ 的段,以及边 $(x,y)$ 构成一个回路。由于 $P_x$ 和 $P_y$ 都是最短路径,从 $w$ 到 $x$ 和从 $w$ 到 $y$ 的长度相等或相差 $1$。但 $c(x)=c(y)$ 意味着从 $w$ 到 $x$ 和从 $w$ 到 $y$ 的路径长度奇偶性相同(因为染色由距离奇偶性决定)。因此,这两段路径长度之和为偶数,加上边 $(x,y)$ 后得到奇数长度的回路,矛盾。所以,不存在这样的边 $(x,y)$,即染色合法。故 $G$ 是二分图。

    若 $G$ 不连通,对每个连通分量分别进行上述染色,同样可得 $G$ 是二分图。

\begin{theorem}{Hall定理(婚姻定理)}{}
    设$G=(X,Y,E)$是一个二分图,其中$X$和$Y$是两部分顶点集。$G$中存在一个匹配覆盖$X$的所有顶点(即存在从$X$到$Y$的完备匹配)的充要条件是:$X$中的任意$k$个结点至少邻接$Y$的$k$个结点。
\end{theorem}
\textbf{必要性:} 如果存在一个匹配$M$覆盖了$X$的所有顶点,那么$X$中任意$k$个顶点,每一个都在$Y$中有一个不同的匹配对象。由于匹配中的边没有公共端点,这$k$个不同的匹配对象都属于这$k$个顶点在$Y$中的邻点集,因此$Y$中至少有$k$个结点与这$k$个结点相邻。

\textbf{充分性:} 假设条件成立,但$G$中不存在覆盖$X$的完备匹配。设$M$是$G$的一个最大匹配,它没有覆盖$X$的所有顶点。在$X$中取一个未被$M$覆盖的顶点$u$。从$u$出发,寻找一条“交错路”(由非匹配边和匹配边交替构成),可以找到一个顶点集合$S \subseteq X$,使得$S$的邻点集$N(S)$中的顶点数严格少于$S$的顶点数($|N(S)| < |S|$)。这与“任意$k$个结点至少邻接$k$个结点”的条件矛盾。因此,假设不成立,$G$中必然存在覆盖$X$的完备匹配。
\begin{theorem}{二分图的完备匹配存在性}{}
    设$G=(V,E)$是二分图,$V_1,V_2$是$G$的互补结点集,若存在正整数$t$,使得\\
    (1)$V_1$中每个结点至少关联$t$条边\\
    (2)$V_2$中每个结点至多关联$t$条边\\
    则$G$中存在从$V_1$到$V_2$的完备匹配。
\end{theorem}
考虑$V_1$的任一子集$S\subseteq V_1$,记$N(S)$为$S$的邻点集。设$S$与$N(S)$之间的边数为$e(S,N(S))$。由条件(1),$S$中每个结点至少关联$t$条边,而这些边都关联到$N(S)$中的结点,故
    \[
    e(S,N(S)) \ge t|S|.
    \]
    由条件(2),$N(S)$中每个结点至多关联$t$条边,故
    \[
    e(S,N(S)) \le t|N(S)|.
    \]
    联合两式得
    \[
    t|S| \le e(S,N(S)) \le t|N(S)|,
    \]
    即$|S|\le |N(S)|$。由Hall定理,$G$中存在从$V_1$到$V_2$的完备匹配。
\section{平面图,同胚,对偶图}
\begin{definition}{平面图及相关概念}{}
\textbf{1. 平面图:}
如果图 $G$ 可以画在平面上,使得任意两条边仅在端点处相交,则称 $G$ 是\textbf{可平面图}。已经以这种方式画在平面上的图称为\textbf{平面嵌入},此时 $G$ 也称为\textbf{平面图}。

\textbf{2. 平面嵌入:}
将可平面图 $G$ 画在平面上,使得任意两条边仅在端点处相交,这样得到的图形称为 $G$ 的一个\textbf{平面嵌入}。

\textbf{3. 面:}
在平面嵌入中,由边所围成的连通区域称为\textbf{面}。其中,无界的区域称为\textbf{外部面}(无限面),有界的区域称为\textbf{内部面}(有限面)。

\textbf{4. 面的边界:}
给定平面图的一个面 $f$,包围 $f$ 的闭路径(或闭回路)称为面 $f$ 的边界。如果边界是简单回路,则回路的长度称为该面的次数;如果边界不是简单回路(例如,图中有割边),则规定每条割边在计算次数时计为 2。

\textbf{5. 面的次数:}
面 $f$ 的边界上边的数目(割边计两次)称为面 $f$ 的次数,记作 $\deg(f)$。

\textbf{6. 极大平面图:}
设 $G$ 是简单平面图,如果 $G$ 中任意两个不相邻的顶点之间加一条边都会破坏其平面性,则称 $G$ 为极大平面图。极大平面图的每个面的次数均为 3,也称为三角剖分图。

\textbf{7. 极小非平面图:}
如果图 $G$ 是非平面图,但删除任意一条边(或删除任意一个顶点)后得到的图都是平面图,则称 $G$ 为极小非平面图。例如,$K_5$ 和 $K_{3,3}$ 是极小非平面图。
\end{definition}
如果绕着面的边界走一圈,显然割边要走两次,所以要计算两次。
\begin{theorem}{平面图的第二种握手定理}{}
设 $G$ 是一个平面图(或平面嵌入),$F$ 是 $G$ 的所有面的集合,则所有面的次数之和等于边数的两倍,即
\[
\sum_{f \in F} \deg(f) = 2|E|
\]
其中 $\deg(f)$ 表示面 $f$ 的次数
\end{theorem}
考虑每条边 $e \in E$ 对 $\sum_{f \in F} \deg(f)$ 的贡献:若 $e$ 不是割边,则 $e$ 恰好是两个不同面的公共边界,因此 $e$ 对总次数贡献 $2$。若 $e$ 是割边,则 $e$ 只属于一个面,但根据定义,在计算该面的次数时,割边被计算两次,因此 $e$ 对总次数贡献也为 $2$。因此,每条边对总次数的贡献都是 $2$,故所有面的次数之和等于 $2|E|$。
\begin{theorem}{平面图的欧拉公式}{}
设 $G$ 是一个连通平面图,有 $n$ 个顶点、$e$ 条边和 $f$ 个面,则欧拉公式成立:\vspace{-10pt}
\[
n - e + f = 2\vspace{-10pt}
\]
更一般地,如果 $G$ 有 $w$ 个连通分支,则欧拉公式推广为:\vspace{-10pt}
\[
n - e + f = w + 1
\]
\end{theorem}
设 $G$ 有 $w$ 个连通分支 $G_1, G_2, \dots, G_w$,每个连通分支 $G_i$ 的顶点数、边数和面数分别为 $n_i, e_i, f_i$。对于每个连通分支,应用连通平面图的欧拉公式有 $n_i - e_i + f_i = 2$。将所有分支的等式相加:
\[
\sum_{i=1}^w (n_i - e_i + f_i) =\left(\sum_{i=1}^w n_i\right) - \left(\sum_{i=1}^w e_i\right) + \left(\sum_{i=1}^w f_i\right) = 2w
\]
注意 $\sum n_i = n$,$\sum e_i = e$,而 $\sum f_i$ 并不是 $f$,因为每个分支的外部面是同一个(即整个图的外部面)。设 $f'$ 为所有内部面之和,则 $f = f' + 1$(外部面计为一个)。而 $\sum f_i = f' + w$(每个分支有一个外部面,但实际共享一个外部面)。因此
\[
n - e + (f' + w) = 2w
\]
由于 $f = f' + 1$,代入得
\[
n - e + f = w + 1
\]
特别地,当 $w=1$ 时,得到 $n - e + f = 2$。
\begin{theorem}{欧拉公式的推论}{}
    设 $G$ 是一个 $n$ 个顶点、$e$ 条边的连通平面图,且有 $f$ 个面,$n-e+f=2$ 成立。

\textbf{推论1(最大边数):}若 $G$ 是简单平面图且 $n \geqslant 3$,则\vspace{-10pt}
\[
e \leqslant 3n - 6\vspace{-10pt}
\]
若 $G$ 是简单二分平面图,则\vspace{-10pt}
\[
e \leqslant 2n - 4\vspace{-10pt}
\]
\textbf{推论2(存在次数不超过5的顶点):}
若 $G$ 是简单平面图,则存在顶点 $v$ 使得 $\deg(v) \leqslant 5$。

\textbf{推论3(面的最小次数):}
若 $G$ 是简单平面图,则存在面 $f$ 使得 $\deg(f) \leqslant 5$.

\textbf{推论4(平面图的边数下界):}
若 $G$ 是连通平面图,则 $e \geqslant n-1$。

\textbf{推论5(极大平面图的性质):}
若 $G$ 是 $n \geqslant 3$ 的极大平面图,则 $e = 3n-6$,$f = 2n-4$,且每个面次数为 3。

\textbf{推论6(设连通分支数为w):}设$G$有$n$个结点,$e$条边,$w$个连通分支,每个面至少由$k$条边组成,则\vspace{-10pt}\[e\leqslant \dfrac{k(n-w-1)}{k-2}\]
\end{theorem}
推论1:对简单平面图,每个面的次数至少为 3,由第二种握手定理 $\sum \deg(f) = 2e$ 得 $3f \leqslant 2e$,代入欧拉公式 $f = 2 - n + e$ 得 $3(2-n+e) \leqslant 2e$,即 $e \leqslant 3n-6$。

对简单二分平面图,由于无奇圈,每个面的次数至少为 4,类似可得 $4f \leqslant 2e$,代入欧拉公式得 $e \leqslant 2n-4$。

推论2:假设所有顶点度数 $\geqslant 6$,由握手定理 $2e = \sum \deg(v) \geqslant 6n$,即 $e \geqslant 3n$,与 $e \leqslant 3n-6$ 矛盾。

推论3:由 $3f \leqslant 2e$ 和 $e \leqslant 3n-6$ 得 $f \leqslant 2n-4$。若所有面次数 $\geqslant 6$,则 $6f \leqslant 2e$ 即 $e \geqslant 3f$,代入 $f = 2-n+e$ 得矛盾。

推论4:由连通性,$G$ 有生成树,边数为 $n-1$,故 $e \geqslant n-1$。

推论5:极大平面图每个面为三角形,故 $3f=2e$,代入欧拉公式即得。

推论6:设 $G$ 有 $f$ 个面。由推广的欧拉公式 $n - e + f = w + 1$,得
    \[
    f = e - n + w + 1.
    \]
    由于每个面至少由 $k$ 条边组成,所有面的次数之和满足
    \[
    \sum_{i=1}^{f} \deg(f_i) \ge k f.
    \]
    由平面图的第二种握手定理,$\sum_{i=1}^{f} \deg(f_i) = 2e$,故
    \[
    2e \ge k f.
    \]
    将 $f = e - n + w + 1$ 代入上式,得
    \[
    2e \ge k (e - n + w + 1).
    \]
    由于 $k \ge 3$,$k-2 > 0$,故
    \[
    e \le \frac{k(n - w - 1)}{k-2}.
    \]
    证毕。
\begin{example}{}{}
    完全图 $K_5$ 和完全二分图 $K_{3,3}$ 不是平面图。它们各自至少需要删除 $1$ 条边才能变成可平面图。
\end{example}
对于 $K_5$:有 $5$ 个顶点,$10$ 条边。由平面图的必要条件:对于简单连通平面图,当顶点数 $n \geqslant 3$ 时,边数 $e \leqslant 3n-6$。代入 $n=5$ 得 $3 \times 5 - 6 = 9$。$K_5$ 有 $10$ 条边,因此至少需要删除 $10-9=1$ 条边。事实上,删除任意一条边后得到的图 $K_5 - e$ 是极大平面图(三角剖分),因此是可平面图。

对于 $K_{3,3}$:有 $6$ 个顶点,$9$ 条边。由于 $K_{3,3}$ 是二分图,由二分平面图的必要条件:对于简单连通二分平面图,当顶点数 $n \geqslant 3$ 时,边数 $e \leqslant 2n-4$。代入 $n=6$ 得 $2 \times 6 - 4 = 8$。$K_{3,3}$ 有 $9$ 条边,因此至少需要删除 $9-8=1$ 条边。事实上,$K_{3,3}$ 是极小非平面图,删除任意一条边后得到的图 $K_{3,3} - e$ 是可平面图。

\begin{example}{}{}
    小于30条边的简单连通平面图至少有一个度数小于等于4的结点。
\end{example}
由$e\leqslant 3n-6$,以及使用握手定理,先反设结论不成立,这样的话边数最少的情况就是每个结点的度数都是5,然后对结点度数求和得到$5n\leqslant 2e$,联立得到$e\geqslant 30$,矛盾。

\begin{definition}{插入两度结点、删除两度结点、同胚}{}
\textbf{1. 插入两度结点:}
在图 $G$ 的一条边 $e = (u, v)$ 上插入一个新的结点 $w$,即将边 $e$ 删除,然后添加新结点 $w$ 以及两条新边 $(u, w)$ 和 $(w, v)$。这一操作称为在边 $e$ 上插入一个两度结点。

\textbf{2. 删除两度结点:}
设 $w$ 是图 $G$ 的一个度数为 2 的结点,且与 $w$ 相邻的两个结点为 $u$ 和 $v$。删除结点 $w$ 以及与之关联的两条边 $(u, w)$ 和 $(w, v)$,然后添加一条新边 $(u, v)$(如果 $u$ 和 $v$ 之间原来没有边)。这一操作称为删除一个两度结点。

\textbf{3. 同胚:}
两个图 $G_1$ 和 $G_2$ 称为同胚的,如果它们可以通过一系列插入或删除两度结点的操作变成同构的图。即,存在图 $G$,使得 $G_1$ 和 $G_2$ 分别与 $G$ 同胚,且 $G_1$ 和 $G_2$ 同构。
\end{definition}

\begin{theorem}{库拉托夫斯基定理}{}
    一个图是平面图当且仅当它不包含同胚于 $K_5$ 或 $K_{3,3}$ 的子图。
\end{theorem}
彼得森图(Petersen graph)包含同胚于 $K_{3,3}$ 的子图,因此是非平面图。任何树都是平面图,因为树不包含任何圈,更不可能包含同胚于 $K_5$ 或 $K_{3,3}$ 的子图。
\begin{definition}{对偶图(Dual Graph)}{}
设 $G=(V,E)$ 是一个平面图(已嵌入平面),$F$ 是 $G$ 的面的集合。$G$ 的对偶图 $G^*=(V^*,E^*)$ 构造如下:

对于 $G$ 的每个面 $f \in F$,在 $f$ 的内部放置一个顶点 $f^*$,这些顶点构成 $V^*$。对于 $G$ 的每条边 $e \in E$,若 $e$ 是 $G$ 中两个不同面 $f_1$ 和 $f_2$ 的公共边界,则在对偶图 $G^*$ 中连接对应的顶点 $f_1^*$ 和 $f_2^*$ 得到边 $e^*$,且 $e^*$ 与 $e$ 相交一次(不穿过其他边和顶点)。若 $e$ 是 $G$ 中某个面 $f$ 的独有边界(即 $e$ 是割边),则在 $G^*$ 中对应顶点 $f^*$ 处添加一条自环 $e^*$,且 $e^*$ 与 $e$ 相交一次。

这样得到的图 $G^*$ 称为 $G$ 的对偶图。$G^*$ 也是平面图,且有一个自然的平面嵌入。对偶图 $G^*$ 的面与 $G$ 的顶点一一对应。
\end{definition}
对偶图 $G^*$ 的顶点数等于原图 $G$ 的面数,即 $|V^*| = |F|$。\\
对偶图 $G^*$ 的边数等于原图 $G$ 的边数,即 $|E^*| = |E|$。\\
对偶图 $G^*$ 的面数等于原图 $G$ 的顶点数,即 $|F^*| = |V|$。\\
对偶图 $G^*$ 中顶点的度数等于原图 $G$ 中对应面的边界的长度(割边计两次)。\\
\textbf{原图的环对应对偶图的桥,对偶图的环对应原图的桥,平面图的对偶图一定连通,反之不然。}
\begin{theorem}{对偶图的对偶图}{}
(1)同构平面图的对偶图不一定是同构的。

(2)平面图的对偶图的对偶图不一定同构于原图。但是若 $G$ 是连通平面图,则 $(G^*)^* \cong G$,即对偶的对偶同构于原图。
\end{theorem}
设 $G$ 是不连通平面图,则 $G$ 有至少两个连通分支。$G$ 的对偶图 $G^*$ 的构造依赖于嵌入,但通常 $G^*$ 是连通的(因为每个面在 $G^*$ 中对应一个顶点,且边根据面的相邻关系连接)。而 $(G^*)^*$ 是连通的,但 $G$ 不连通,因此不同构。例如,$G$ 为两个不相交的三角形,$G^*$ 为一条两个顶点间的多重边(具体结构依赖于嵌入),$(G^*)^*$ 为一个单点加自环(或类似结构),不与 $G$ 同构。

设 $G$ 是连通平面图,已给定平面嵌入。按对偶图的定义,$G^*$ 的顶点对应 $G$ 的面,$G^*$ 的边对应 $G$ 的边,且 $G^*$ 的嵌入是自然的:$G^*$ 的每个顶点放在 $G$ 的对应面内,$G^*$ 的每条边与 $G$ 的对应边交叉一次。由于 $G$ 连通,$G^*$ 的每个面包含 $G$ 的一个顶点,且 $G^*$ 的面与 $G$ 的顶点一一对应。构造 $(G^*)^*$ 时,顶点对应 $G^*$ 的面,即 $G$ 的顶点;边对应 $G^*$ 的边,即 $G$ 的边;且关联关系保持不变。因此,$(G^*)^*$ 与 $G$ 作为抽象图同构。具体地,映射 $\varphi: V(G) \to V((G^*)^*)$,将 $G$ 的顶点 $v$ 映到 $G^*$ 中包含 $v$ 的面所对应的顶点,是双射,且保持边的关联关系。故 $(G^*)^* \cong G$。
\begin{definition}{地图,面着色,边着色,点着色}{}
\textbf{1. 地图(Map)}
地图是指一个连通平面图的平面嵌入,它将平面划分为若干个连通区域,每个区域称为一个面。地图通常还要求没有割边(即每条边属于两个不同面的边界)和没有孤立顶点。

\textbf{2. 面着色:}
对地图的面进行着色,使得任意两个有公共边(非仅公共顶点)的面颜色不同。所需的最少颜色数称为地图的面色数(或地图色数)。如果能用$k$种颜色给图进行边着色,称该图是$k-$可边着色的。

\textbf{3. 边着色:}
对图的边进行着色,使得任意两条相邻的边(即有公共顶点的边)颜色不同。所需的最少颜色数称为边色数。

\textbf{4. 点着色:}
对图的顶点进行着色,使得任意两个相邻的顶点颜色不同。所需的最少颜色数称为点色数(通常简称色数)。
\end{definition}
\begin{theorem}{Vizing 定理}{}
    对于简单图 $G$,其边色数 $\chi'(G)$ 满足
    \[
    \Delta(G) \le \chi'(G) \le \Delta(G) + 1,
    \]
    其中 $\Delta(G)$ 是 $G$ 的最大度。
\end{theorem}
下界是显然的:由于最大度顶点关联 $\Delta(G)$ 条边,这些边必须用不同的颜色着色,故 $\chi'(G) \ge \Delta(G)$。

    上界的证明:用归纳法。设 $G$ 是简单图,$\Delta = \Delta(G)$。对边数 $m$ 进行归纳。当 $m=0$ 时,显然成立。假设对边数小于 $m$ 的简单图成立,考虑 $G$ 有 $m$ 条边。任取一边 $e=uv$,考虑 $G' = G - e$。由归纳假设,$G'$ 可用 $\Delta$ 或 $\Delta+1$ 种颜色正常边着色。我们试图将 $e$ 着色,使得 $G$ 的着色仍正常且颜色数不超过 $\Delta+1$。

    设 $C$ 是用于 $G'$ 的着色颜色集,$|C| = \Delta+1$。对每个顶点 $x$,记 $S(x)$ 为 $x$ 处已用的颜色集合,$T(x) = C -S(x)$ 为 $x$ 处缺失的颜色。由于每个顶点最多关联 $\Delta$ 条边,故 $|T(x)| \ge 1$。

    若 $T(u) \cap T(v) \neq \varnothing$,则存在颜色 $c$ 在 $u$ 和 $v$ 处都缺失,将 $e$ 着以 $c$ 即得 $G$ 的 $(\Delta+1)$-边着色。

    若 $T(u) \cap T(v) = \varnothing$,则取 $a \in T(u)$,$b \in T(v)$($a \neq b$)。考虑从 $u$ 开始的极大交替路径 $P$,其边颜色交替为 $a$ 和 $b$。由于 $a$ 在 $u$ 处缺失,路径 $P$ 的第一条边颜色为 $b$(如果存在)。路径 $P$ 不会形成环,且因为 $a$ 和 $b$ 在顶点处交替出现,路径可能在某个顶点结束,该顶点缺失 $a$ 或 $b$。

    分两种情况:
    
    若 $P$ 终止于 $v$,则 $P$ 是一条以 $v$ 结束的交替路径,且 $v$ 处缺失 $a$ 或 $b$。由于 $b \in T(v)$,$v$ 处缺失 $b$,但 $P$ 的最后一条边颜色可能是 $a$ 或 $b$。实际上,可以调整路径上的颜色,使得 $v$ 处缺失 $a$,然后将 $e$ 着以 $a$。详细调整需交换路径上的颜色。

    若 $P$ 终止于某个不同于 $v$ 的顶点 $w$,则 $w$ 处缺失 $a$ 或 $b$。交换路径 $P$ 上的颜色 $a$ 和 $b$,则 $u$ 处缺失 $b$,$v$ 处仍缺失 $b$(因为 $b \in T(v)$ 且路径调整不影响 $v$ 的缺失色集),于是 $T(u) \cap T(v)$ 包含 $b$,可将 $e$ 着以 $b$。
    
    因此,总能将 $e$ 着色,得到 $G$ 的 $(\Delta+1)$-边着色。由归纳法,定理得证。

\begin{theorem}{五色定理(Five Color Theorem)}{}
    任何平面图都可以用 5 种颜色进行正常的点着色,即其色数满足 $\chi(G) \le 5$。
\end{theorem}
\textbf{证明:} 对顶点数 $n$ 进行数学归纳法。

    \textbf{归纳基础:} 当 $n \le 5$ 时,显然可用 5 种颜色着色,结论成立。

    \textbf{归纳步骤:} 假设对任意 $n-1$ 个顶点的平面图,结论成立。考虑 $n$ 个顶点的平面图 $G$。由平面图的推论(欧拉公式推论)可知,$G$ 中存在一个顶点 $v$,其度数 $d(v) \le 5$。从 $G$ 中删除 $v$ 得到图 $G' = G - v$。由归纳假设,$G'$ 可用 5 种颜色正常着色。

    将 $G'$ 的着色固定。现在考虑将 $v$ 加回图中。若 $v$ 的邻点用了少于 5 种颜色,则 $v$ 可用剩下的一种颜色着色,得到 $G$ 的 5-着色。否则,$v$ 的邻点恰用了 5 种不同的颜色,记为 $c_1, c_2, c_3, c_4, c_5$,且按平面嵌入中顺时针顺序与 $v$ 相邻的顶点为 $v_1, v_2, v_3, v_4, v_5$,其中 $v_i$ 着颜色 $c_i$。

    考虑颜色对 $\{c_1, c_3\}$。在 $G'$ 中,从 $v_1$ 出发,只走颜色为 $c_1$ 和 $c_3$ 的顶点,得到诱导子图 $H$(即 Kempe 链)。分两种情况:

    \textbf{情况 1:} $v_1$ 和 $v_3$ 在 $H$ 中不连通。则在包含 $v_1$ 的连通分支中,交换颜色 $c_1$ 和 $c_3$。由于 $v_1$ 和 $v_3$ 不连通,$v_3$ 的颜色不变。交换后,$v_1$ 的颜色变为 $c_3$,而 $v$ 的邻点中不再有颜色 $c_1$,于是可将 $v$ 着以 $c_1$,得到 $G$ 的 5-着色。

    \textbf{情况 2:} $v_1$ 和 $v_3$ 在 $H$ 中连通。则存在一条从 $v_1$ 到 $v_3$ 的路径,其顶点颜色交替为 $c_1$ 和 $c_3$。这条路径与边 $v v_1$ 和 $v v_3$ 一起构成一个圈,将 $v_2$ 和 $v_4$ 分隔在圈的两侧。现在考虑颜色对 $\{c_2, c_4\}$,从 $v_2$ 出发的 Kempe 链(颜色为 $c_2$ 和 $c_4$ 的顶点)不可能与 $v_4$ 连通,因为这条链必须穿过上述圈,但圈上的顶点颜色为 $c_1$ 和 $c_3$,与链的颜色冲突。因此,在包含 $v_2$ 的连通分支中交换颜色 $c_2$ 和 $c_4$,交换后 $v_2$ 的颜色变为 $c_4$,$v$ 的邻点中不再有颜色 $c_2$,于是可将 $v$ 着以 $c_2$,得到 $G$ 的 5-着色。

    综上,在所有情况下,$G$ 均可 5-着色。由归纳法,定理得证。
\section{韦尔奇鲍威尔算法}
    韦尔奇鲍威尔算法是一种用于图着色的贪心算法,旨在用尽可能少的颜色对图的顶点进行着色,使得相邻顶点颜色不同。该算法并不保证得到最优解(即色数),但在实践中通常能得到较好的近似解。

    \textbf{算法步骤:}
    给定简单无向图 $G=(V,E)$,顶点集 $V=\{v_1,v_2,\ldots,v_n\}$。
    
         将顶点按度数递减顺序排列(如果度数相同,可以任意排列)。设排序后的顶点序列为\\$u_1,u_2,\ldots,u_n$,满足 $\deg(u_1) \ge \deg(u_2) \ge \cdots \ge \deg(u_n)$。
         初始化颜色编号 $c=1$。

         从 $u_1$ 开始,依次考察每个未着色的顶点 $u_i$,将颜色 $c$ 分配给 $u_i$,然后对于序列中排在 $u_i$ 之后的所有未着色的顶点 $u_j$,如果 $u_j$ 与 $u_i$ 不相邻,且与已着颜色 $c$ 的任何顶点都不相邻,则将颜色 $c$ 分配给 $u_j$。

         当不存在可以着颜色 $c$ 的未着色顶点时,令 $c=c+1$,重复步骤 3,直到所有顶点都被着色。
    
\begin{example}{}{}
    某大学有5门课程,现在课程1和2,课程1和3,课程1和4,课程2和4,课程2和5,课程3和4,课程3和5均有人同时修读,试问安排这5门课程至少需要几个时间段?
\end{example}
建立图模型。以课程为顶点,若两门课程有学生同时修读,则在对应顶点间连边,得图$G$:
\[
V(G)=\{v_1,v_2,v_3,v_4,v_5\},\quad
E(G)=\{v_1v_2,v_1v_3,v_1v_4,v_2v_4,v_2v_5,v_3v_4,v_3v_5\}.
\]
问题转化为求$G$的点色数$\chi(G)$,即最少颜色数,使得相邻顶点颜色不同。

\textbf{韦尔奇鲍威尔算法验证:}
 按度数降序排列顶点(度数相同任意):$v_1(3),v_2(3),v_3(3),v_4(3),v_5(2)$。
 颜色1:着$v_1$,$v_5$与$v_1$不相邻且与已着颜色1的顶点不相邻,故$v_5$也着颜色1。
 颜色2:$v_2$未着色,着颜色2;$v_3$与$v_2$不相邻,且与$v_1$(颜色1)相邻,但$v_3$与$v_2$不相邻,故$v_3$可着颜色2。
 颜色3:$v_4$与$v_1$(颜色1)、$v_2$(颜色2)、$v_3$(颜色2)相邻,需着颜色3。

共用3种颜色,故至少需要3个时间段。
\begin{example}{}{}
把平面图分成 $n$ 个区域,每两个区域都相邻,则 $n$ 最大为?
\end{example}
设平面图有 $n$ 个区域(面),且任意两个区域都相邻(即共享一条边)。考虑该平面图的对偶图:以每个区域为顶点,若两区域相邻则在对应顶点间连边。由条件,对偶图是完全图 $K_n$。由于原图是平面图,其对偶图也是平面图,因此 $K_n$ 必须是平面图。

但由$K_5$ 不是平面图,故 $n \le 4$。当 $n=4$ 时,可以构造满足条件的平面图,例如:取一个三角形及其外部区域,并将三角形内部用三条线段从一个内点连至三个顶点,将三角形分成三个小三角形区域,则四个区域(三个小三角形区域和外部无限区域)两两相邻。因此 $n$ 的最大值为 $4$。

\newpage
\section{树的相关概念}
\begin{definition}{无向树及其相关概念}{}
    \textbf{1. 无向树}:一个连通且无回路的无向图称为树。

    \textbf{2. 树枝}:树中的边称为树枝。

    \textbf{3. 树叶}:树中度数为1的顶点称为树叶。

    \textbf{4. 分支点}:树中度数大于1的顶点称为分支点(或内点)。

    \textbf{5. 平凡树}:只有一个顶点的树称为平凡树。

    \textbf{6. 森林}:一个无回路的无向图称为森林。即,森林的每个连通分支都是树。
\end{definition}
\begin{theorem}{树的判定}{}
    1. $G$是无回路的连通图\\
    2. $G$连通,且删去任意一条边后就不连通\\
    3. $G$连通,且边数等于结点数减去一\\
    4. $G$无回路,且边数等于结点数减去一\\
    5. 在$G$的每一对结点之间都有唯一的一条简单路\\
    6. $G$无回路,但是任意两个结点之间增加一条边,可以得到一条且仅一条的回路
\end{theorem}
我们采用循环证明:
(1)$\Rightarrow$(2)$\Rightarrow$(3)$\Rightarrow$(4)$\Rightarrow$(5)$\Rightarrow$(6)$\Rightarrow$(1)。

\textbf{(1)$\Rightarrow$(2)}:假设$G$是无回路的连通图。若存在边$e$使得$G-e$仍然连通,则在$G-e$中存在一条连接$e$的两个端点的简单路$P$。于是$P+e$构成$G$中的一个回路,与$G$无回路矛盾。故$G$连通且删去任意一条边后就不连通。

\textbf{(2)$\Rightarrow$(3)}:假设$G$连通,且删去任意一条边后就不连通。用数学归纳法证明$e=n-1$。当$n=1$时,$e=0$,结论成立。假设对少于$n$个结点的图结论成立。在$G$中任取一条边$e$,由条件知$G-e$不连通,设其连通分支为$G_1$和$G_2$,结点数分别为$n_1$和$n_2$,边数分别为$e_1$和$e_2$。显然$G_1$和$G_2$也满足条件(2)(连通且删去任意边后不连通)。由归纳假设,$e_1=n_1-1$,$e_2=n_2-1$。于是$e=e_1+e_2+1=(n_1-1)+(n_2-1)+1=n_1+n_2-1=n-1$。

\textbf{(3)$\Rightarrow$(4)}:假设$G$连通且$e=n-1$。只需证$G$无回路。若$G$有回路,则删去回路中一条边,图仍连通。重复此过程直到无回路,得到生成树$T$。$T$是连通无回路图,由(1)$\Rightarrow$(3)知$T$的边数为$n-1$,但$T$的边数小于$e=n-1$,矛盾。故$G$无回路。

\textbf{(4)$\Rightarrow$(5)}:假设$G$无回路且$e=n-1$。先证$G$连通。若$G$有$k$个连通分支,每个分支是无回路连通图,由(1)$\Rightarrow$(3)知每个分支边数等于结点数减一。设各分支结点数为$n_i$,则$\sum_{i=1}^k (n_i-1)=n-k=e=n-1$,得$k=1$,故$G$连通。任取两结点$u,v$,因$G$连通,存在一条$u$到$v$的简单路。若存在两条不同的简单路$P_1$和$P_2$,则$P_1\cup P_2$包含回路,矛盾。故存在唯一简单路。

\textbf{(5)$\Rightarrow$(6)}:假设在$G$的每一对结点之间都有唯一的一条简单路。显然$G$无回路(否则回路上两点间有两条简单路)。在$G$中任取两结点$u,v$,若$uv$不是边,则增加边$uv$后,$G+uv$中$u$到$v$有两条路:原唯一简单路和新边$uv$,这两条路构成一个回路。若还有另一个回路,则删去$uv$后$G$中有回路,矛盾。故得到一条且仅一条回路。

\textbf{(6)$\Rightarrow$(1)}:假设$G$无回路,但任意两结点间加一边后得到唯一回路。只需证$G$连通。若$G$不连通,则存在两结点$u,v$在不同连通分支中。增加边$uv$后,$G+uv$中无回路(因两连通分支间加一边不产生回路),与条件矛盾。故$G$连通。

综上,六个条件等价。
\begin{example}{}{}
    已知$n$阶$e$条边的无向图是由$w$棵树组成的森林,证明$e=n-w$
\end{example}
设森林由 $w$ 棵树组成,记第 $i$ 棵树有 $n_i$ 个顶点和 $e_i$ 条边,$i=1,2,\dots,w$。由于每棵树都是树,满足树的性质:边数等于顶点数减 $1$,于是,总顶点数 $n = \sum_{i=1}^w n_i$,总边数
    \[
    e = \sum_{i=1}^w e_i = \sum_{i=1}^w (n_i - 1) = \sum_{i=1}^w n_i - \sum_{i=1}^w 1 = n - w.
    \]
    因此,$e = n - w$。
\begin{example}{}{}
列出所有5阶无向树的结点度数序列
\end{example}
设5阶树的度数序列为$d_1 \ge d_2 \ge d_3 \ge d_4 \ge d_5 \ge 1$,且$\displaystyle\sum_{i=1}^5 d_i = 2(5-1) = 8$。由树的性质,至少有两个$d_i = 1$(树叶)。枚举所有可能的正整数序列:

最大度数为4:序列为$(4,1,1,1,1)$。
 
最大度数为3:设序列为$(3, d_2, d_3, d_4, d_5)$,则$d_2 + d_3 + d_4 + d_5 = 5$,且$d_2 \le 3$,$d_5 \ge 1$。满足非增的正整数解只有$(3,2,1,1,1)$。

最大度数为2:设序列为$(2,2,2,1,1)$,其和为$2+2+2+1+1=8$。

最大度数为1:全1序列和为5,不满足。

因此,所有可能的度数序列为:
\[
(4,1,1,1,1), \quad (3,2,1,1,1), \quad (2,2,2,1,1).
\]
\begin{theorem}{无向树的欧拉通路与哈密顿通路条件}{}
设 $T$ 是一个无向树(连通无环图),$n$ 表示顶点数,$e$ 表示边数。\\
\textbf{欧拉通路:} $T$ 存在欧拉通路当且仅当 $T$ 是平凡树($n=1$)或是一条路径(即 $T$ 中至多有两个叶子结点)。等价地,$T$ 中奇度顶点的个数为 0 或 2。\\
\textbf{哈密顿通路:} $T$ 存在哈密顿通路当且仅当 $T$ 是一条路径(即 $T$ 中至多有两个叶子结点)。\\
\textbf{既是欧拉图又是哈密顿图:} $T$ 既是欧拉图(存在欧拉回路)又是哈密顿图(存在哈密顿回路)当且仅当 $T$ 是平凡树($n=1$)。
\end{theorem}
欧拉通路的充要条件是图连通且奇度顶点个数为 0 或 2。树是连通的。对于平凡树($n=1$,$e=0$),所有顶点度数为 0(偶数),故存在欧拉通路(平凡路径)。对于非平凡树($n\ge 2$),由树的性质,叶子结点的度数为 1(奇数),且树中至少有两个叶子。若树中恰有两个叶子,则其余结点度数均为偶数(由于树中无环,非叶子结点的度数至少为 2,且由握手定理可推知其余结点度数必为偶数),此时存在欧拉通路(从一叶子到另一叶子)。若叶子数多于 2,则奇度顶点数大于 2,不存在欧拉通路。恰有两个叶子的树即为路径。

哈密顿通路要求经过每个顶点恰好一次。若树 $T$ 中存在度数大于 2 的结点,则从该结点出发访问其分支时,必会遗漏某些分支或重复访问,无法形成哈密顿通路。因此,$T$ 中每个结点的度数不超过 2,即 $T$ 是一条路径。反之,路径显然存在哈密顿通路。

欧拉图要求所有顶点度数为偶数。非平凡树至少有两个叶子(度数为 1,奇数),故不可能是欧拉图。平凡树($n=1$)中顶点度数为 0(偶数),存在平凡的欧拉回路。哈密顿图要求存在哈密顿回路。非平凡树无环,故不存在哈密顿回路。平凡树可视为存在平凡的哈密顿回路。因此,同时满足两者的只有平凡树。
\begin{theorem}{}{}
结点数大于等于2的树都是二分图.
\end{theorem}
设 \(T = (V, E)\) 是一棵结点数 \(|V| \ge 2\) 的树。由于树是连通无圈图,我们可以用如下的染色法构造二分划分:

任取一个结点 \(u \in V\),令
\[
U = \{ v \in V \mid \operatorname{dist}(u, v) \text{ 为偶数} \},
\quad W = \{ v \in V \mid \operatorname{dist}(u, v) \text{ 为奇数} \},
\]
其中 \(\operatorname{dist}(u, v)\) 表示 \(u\) 到 \(v\) 在 \(T\) 中的唯一路径的边数。显然 \(U \cap W = \varnothing\) 且 \(U \cup W = V\)。

对任意边 \(xy \in E\),考虑 \(x\) 与 \(y\) 到 \(u\) 的距离。由于 \(T\) 中任意两点间有唯一路径,若 \(x\) 和 \(y\) 到 \(u\) 的距离同奇偶,则 \(u\) 到 \(x\) 的路径加上边 \(xy\) 再加上 \(y\) 到 \(u\) 的路径会构成一个圈,与树的无圈性矛盾。因此 \(x\) 和 \(y\) 到 \(u\) 的距离奇偶性不同,即 \(x\) 和 \(y\) 分属 \(U\) 和 \(W\)。

故 \((U, W)\) 构成 \(T\) 的一个二部划分,从而 \(T\) 是二分图。
\section{生成树}
\begin{definition}{生成树、树枝、弦、余树}{}
    \textbf{1. 生成树}:设 $G = (V, E)$ 是一个连通无向图,$G$ 的一个生成子图 $T$ 如果是一棵树,则称 $T$ 是 $G$ 的一棵生成树。(生成树唯一)

    \textbf{2. 树枝}:生成树 $T$ 中的边称为树枝。

    \textbf{3. 弦}:图 $G$ 中不属于生成树 $T$ 的边称为弦(或连枝)。

    \textbf{4. 余树}:所有弦的集合称为余树,记作 $E - E(T)$,其中 $E(T)$ 是生成树 $T$ 的边集。

    \textbf{5.最小生成树}:设 $G=(V,E,w)$ 是一个带权连通无向图,其中 $w:E \to \mathbb{R}^+$ 是边的权值函数。$G$ 的一棵生成树 $T$ 是所有边权和最小的生成树,即下式取最小值,这样的生成树称为 $G$ 的最小生成树。
    \[
    w(T) = \sum_{e \in E(T)} w(e)
    \]
\end{definition}
\begin{example}{}{}
    $K_4$有多少种非同构的生成树?
\end{example}
$K_4$ 有 4 个顶点,其生成树是包含所有 4 个顶点的树。由于 $K_4$ 包含所有可能的边,因此 $K_4$ 的非同构生成树就是 4 个顶点的不同构的树。

4 个顶点的树有 3 条边,总度数为 6。可能的度序列(非递增)只有两种:$(3,1,1,1)$ 和 $(2,2,1,1)$。前者对应星形 $K_{1,3}$,后者对应路径 $P_4$。这两种树互不同构,且不存在其他可能的 4 顶点树。因此,$K_4$ 的非同构的生成树共有 2 个。
\begin{definition}{基本回路与基本割集}{}
设 $G=(V,E)$ 是连通无向图,$T$ 是 $G$ 的一棵生成树。

\textbf{1. 基本回路(Fundamental Circuit)}
对于每条弦 $e \in E(G)- E(T)$,将 $e$ 加入 $T$ 会产生唯一的回路 $C_e$,称为对应弦 $e$ 的基本回路。基本回路 $C_e$ 由 $e$ 和 $T$ 中唯一的 $u$-$v$ 路径组成,其中 $e=uv$。

\textbf{2. 基本割集(Fundamental Cutset)}
对于每条树枝 $e \in E(T)$,从 $T$ 中删除 $e$ 会将 $T$ 分成两个连通分支,设对应的顶点划分为 $(S, V- S)$。则 $G$ 中所有一个端点在 $S$ 中、另一个端点在 $V-S$ 中的边构成一个边割集 $B_e$,称为对应树枝 $e$ 的基本割集。$B_e$ 是 $G$ 的极小边割集,且 $B_e \cap E(T) = \{e\}$。
\end{definition}
对于生成树 $T$,图 $G$ 中有 $|E|-|V|+1$ 条弦,每条弦对应一个唯一的基本回路。这 $|E|-|V|+1$ 个基本回路构成图 $G$ 的回路空间的基。

生成树 $T$ 有 $|V|-1$ 条树枝,每条树枝对应一个唯一的基本割集。这 $|V|-1$ 个基本割集构成图 $G$ 的割集空间的基。
\begin{theorem}{}{}
图$G$有生成树当且仅当$G$连通
\end{theorem}
\textbf{必要性:} 若 $G$ 有生成树 $T$,则 $T$ 是 $G$ 的连通生成子图。因为 $T$ 连通且包含 $G$ 的所有顶点,所以 $G$ 是连通的。

    \textbf{充分性:} 若 $G$ 连通,则可以用以下两种方法之一构造生成树:
    
    \textbf{破圈法:} 从 $G$ 出发,若 $G$ 中有回路,则删除回路中的一条边,重复此过程直到图中无回路。由于每次删除边不破坏连通性,最终得到的图是连通的且无回路,即为一棵生成树。
    
    \textbf{避圈法(Kruskal 算法思想):} 从空边集开始,依次添加 $G$ 的边,要求每次添加的边不形成回路,直到添加了 $|V(G)|-1$ 条边。由于 $G$ 连通,最终得到的图是连通的且无回路,即为一棵生成树。

    因此,连通图 $G$ 必存在生成树。
\begin{theorem}{}{}
$G$中的任何一座桥都在$G$的任何一个生成树中。
\end{theorem}
用反证法证明。设 $e$ 是 $G$ 的一条桥,且存在 $G$ 的一个生成树 $T$ 不包含 $e$。由于 $e$ 是桥,$G-e$ 不连通。设 $G-e$ 的两个连通分支为 $C_1$ 和 $C_2$,则 $e$ 的两个端点分别在 $C_1$ 和 $C_2$ 中。因为 $T$ 是生成树,$T$ 包含 $G$ 的所有顶点,且 $T$ 连通。但 $T$ 不包含 $e$,因此 $T$ 是 $G-e$ 的生成子图。由于 $G-e$ 不连通,$T$ 作为其生成子图也不连通,这与 $T$ 是树(连通)矛盾。故假设不成立,$e$ 必在 $T$ 中。
\begin{theorem}{}{}
设$T_1$和$T_2$是$G$的两个生成树,$a$是在$T_1$中但是不在$T_2$中的一条边,则存在边$b$在$T_2$中但是不在$T_1$中,使得$(T_1-\{a\})\cup\{b\}$和$(T_2-\{b\})\cup\{a\}$都是$G$的生成树。
\end{theorem}
设 $G$ 有 $n$ 个顶点。由于 $T_1$ 和 $T_2$ 都是生成树,$|E(T_1)|=|E(T_2)|=n-1$。

考虑边 $a \in T_1 - T_2$。从 $T_1$ 中删除 $a$ 后,$T_1-a$ 分为两个连通分支,记为 $U$ 和 $W$,即 $V(G)=U \cup W$ 且 $U \cap W = \varnothing$,且 $a$ 连接 $U$ 和 $W$ 中的顶点。

由于 $T_2$ 是连通图,$T_2$ 中必有连接 $U$ 和 $W$ 的边。设\vspace{-10pt}
\[
B = \{ e \in E(T_2) \mid e \text{ 的一个端点在 } U \text{ 中,另一个在 } W \text{ 中} \}.\vspace{-10pt}
\]
显然 $B \neq \varnothing$。任取 $b \in B$。由于 $b$ 连接 $U$ 和 $W$,而 $T_1$ 中连接 $U$ 和 $W$ 的唯一一条边是 $a$(否则 $T_1-a$ 仍连通),且 $b \neq a$,故 $b \notin E(T_1)$。因此 $b \in T_2 -T_1$。

现证 $(T_1-\{a\})\cup\{b\}$ 是生成树。首先,$|E(T_1-\{a\})\cup\{b\}| = (n-1)-1+1 = n-1$。其次,$T_1-\{a\}$ 不连通,其两个连通分支为 $U$ 和 $W$,添加边 $b$(连接 $U$ 和 $W$)后,图变为连通。又因为边数为 $n-1$,故该图是树,从而是 $G$ 的生成树。

同理,证 $(T_2-\{b\})\cup\{a\}$ 是生成树。因为 $b \in T_2$,删除 $b$ 后 $T_2$ 分为两个连通分支(由 $b$ 连接的 $U$ 和 $W$ 中的顶点分别位于这两个分支),添加边 $a$(连接 $U$ 和 $W$)后恢复连通性,且边数仍为 $n-1$,故也是生成树。

综上,满足条件的 $b$ 存在。
\begin{theorem}{}{}
    简单连通无向图的任何一条边都是该图的某一棵生成树的边。
\end{theorem}
使用构造法:对 $G$ 的任意一条边 $e \in E$,构造一棵包含 $e$ 的生成树如下:
    
由于 $G$ 连通,$G$ 有生成树。若 $e$ 是桥,则 $e$ 属于 $G$ 的每一棵生成树,结论自然成立。
    
若 $e$ 不是桥,则 $G-e$ 仍连通。考虑图 $G' = G-e$,它仍是连通的。任取 $G'$ 的一棵生成树 $T'$,则 $T'$ 有 $|V|-1$ 条边且不含 $e$。将 $e$ 加入 $T'$ 中得到 $T' \cup \{e\}$,由于 $e$ 不是桥,$e$ 的两个端点在 $T'$ 中已连通,故 $T' \cup \{e\}$ 包含一个圈。在该圈中删去任意一条不同于 $e$ 的边 $e'$,得到 $T = (T' \cup \{e\}) - \{e'\}$。则 $T$ 连通且有 $|V|-1$ 条边,且不含圈,故 $T$ 是 $G$ 的一棵生成树,且包含 $e$。
    
因此,$G$ 的每条边都属于某棵生成树。

\begin{theorem}{最小生成树的 Kruskal 算法}{}
    Kruskal 算法是一种用于求解带权连通无向图的最小生成树的贪心算法。其基本思想是:按边权从小到大的顺序选择边,如果加入该边不会与已选择的边形成环,则加入该边,直到选出 $n-1$ 条边($n$ 为顶点数)为止。

    算法步骤如下:设 $G=(V,E,w)$ 是一个有 $n$ 个顶点的带权连通无向图,$w:E\to \mathbb{R}^+$ 是边权函数。首先,将边集 $E$ 按边权从小到大排序。然后,初始化生成树 $T$ 的边集为空,并初始化一个不相交集合数据结构(用于判断是否形成环)。接着,依次考虑排序后的每条边 $e$:如果当前 $T$ 中的边数已达到 $n-1$,则终止循环;否则,检查边 $e$ 的两个端点是否属于同一个集合(即是否在已选边的同一个连通分量中)。如果不属于同一个集合,则将 $e$ 加入 $T$,并将两个端点所在的集合合并;如果属于同一个集合,则跳过 $e$(加入 $e$ 会形成环)。算法结束时,$T$ 即为图 $G$ 的一棵最小生成树。
\end{theorem}
\begin{example}{}{}
    设$G$结点数为$n$,边数为$m$,证明:$G$中每对结点之间具有唯一的通路,等价于$G$连通且$n=m+1$
\end{example}
(必要性)假设 $G$ 中任意两个不同结点之间存在唯一的简单通路。首先,由定义知 $G$ 是连通的。其次,我们证明 $G$ 中不含圈。若 $G$ 包含圈,则圈上任意两个不同的结点之间存在两条不同的简单通路(沿圈的两个方向),矛盾。故 $G$ 是无圈连通图,即 $G$ 是树。由树的性质,边数 $m = n-1$,即 $n = m+1$。

(充分性)假设 $G$ 连通且 $n = m+1$。由于 $G$ 连通,任意两个不同结点之间至少存在一条通路。下证至多存在一条通路。假设存在两个不同的结点 $u$ 和 $v$,它们之间有两条不同的简单通路 $P$ 和 $Q$。考虑 $P \cup Q$,这是一个闭迹,其中必包含一个圈 $C$。在 $G$ 中删除 $C$ 上的一条边 $e$,得到图 $G' = G - e$。因为 $e$ 在圈上,$G'$ 仍连通($u$ 到 $v$ 的路径可通过 $C$ 的另一部分绕过 $e$)。但 $G'$ 有 $n$ 个结点和 $m-1$ 条边,且连通,故其边数满足 $m-1 \ge n-1$(连通图至少需要 $n-1$ 条边),即 $m \ge n$,与 $m = n-1$ 矛盾。因此,任意两个不同结点之间至多存在一条通路。结合连通性,恰存在唯一一条通路。

\begin{example}{}{}
    一个有5个城市的网络中,在城市i和城市j之间修一条路的成本为下列矩阵中的元素$c_{ij}$的值。无穷大表示无法修路,确定使得所有城市相互连通的最小修路成本。
    \[\begin{pmatrix}
        0&3&5&11&9\\
        3&0&3&9&8\\
        5&3&0&\infty&10\\
        11&9&\infty&0&7\\
        9&8&10&7&0
    \end{pmatrix}\]
\end{example}
\textbf{解:} 该问题等价于在给定带权无向图中求最小生成树(MST)。图中顶点为城市1,2,3,4,5,边权为修路成本,$\infty$表示无边。列出所有有效边及其权值(忽略自环和$\infty$):(1,2): 3;~~~(2,3): 3;~~~(1,3): 5;~~~(4,5): 7;~~~(2,5): 8;~~~(1,5): 9;~~~(2,4): 9;~~~(3,5): 10;~~~(1,4): 11。

使用Kruskal算法(按边权从小到大选择,不产生环):选择边(1,2),权值3,再选择边(2,3),权值3,然后选择边(1,3),结果发现会成环,所以不行,然后选择边(4,5),权值7,下一条边(2,5),权值8。顶点2和5未连通(2在集合$\{1,2,3\}$,5在集合$\{4,5\}$),加入后连通所有顶点。所选边集为$\{(1,2),(2,3),(4,5),(2,5)\}$,总权值为$3+3+7+8=21$。
\begin{example}{}{}
    一棵树有$n_1$个树叶,$n_2$个2度结点,$n_3$个3度结点,$n_4$个4度结点,$\cdots$,$n_k$个$k$度结点,求$n_1$
\end{example}
\vspace{-20pt}\[n_1+2n_2+3n_3+\cdots+kn_k=2e=2(n_1+n_2+n_3+\cdots+n_k-1)\Rightarrow n_1 = 2 + \sum_{i=3}^k (i-2)n_i\]
\section{根树}
\begin{definition}{有向树,有向根树,树根,树叶,分支点,内点,层数,高度}{}
\textbf{有向树:} 如果一个有向图的基础无向图是一棵树,则称该有向图为有向树。

\textbf{有向根树:} 存在一个顶点称为根,使得从根到每个其他顶点都有唯一的有向路径,这样的有向树称为有向根树。

\textbf{树根:} 在有向根树中,入度为0的顶点。

\textbf{树叶:} 在有向根树中,出度为0的顶点。

\textbf{分支点(内点):} 在有向根树中,出度不为0的顶点。

\textbf{层数:} 从树根到某一顶点的有向路径的长度(即路径上的边数)。树根的层数为0。

\textbf{高度:} 有向根树中所有顶点的最大层数。

\textbf{祖先:} 在有向根树中,如果从结点 $u$ 到结点 $v$ 存在一条有向路径,则称 $u$ 是 $v$ 的祖先。

\textbf{后代:} 在有向根树中,如果从结点 $u$ 到结点 $v$ 存在一条有向路径,则称 $v$ 是 $u$ 的后代。

\textbf{父亲:} 在有向根树中,如果结点 $u$ 到结点 $v$ 有一条有向边,则称 $u$ 是 $v$ 的父亲。

\textbf{儿子:} 在有向根树中,如果结点 $u$ 到结点 $v$ 有一条有向边,则称 $v$ 是 $u$ 的儿子。

\textbf{兄弟:} 在有向根树中,如果两个结点有相同的父亲,则称这两个结点互为兄弟。

\textbf{根子树:} 在有向根树中,以某个结点为根的子树称为该结点的根子树,它包括该结点及其所有后代,以及连接这些结点的边。

\textbf{m 元树:} 每个结点最多有 $m$ 个儿子的有向根树称为 $m$ 元树。

\textbf{二元树:} 每个结点最多有两个儿子的有向根树称为二元树。

\textbf{二叉树:} 二叉树是每个结点最多有两个子树的树结构,子树有左右之分,次序不能颠倒。二叉树是有序树。

\textbf{m 元正则树:} 每个结点恰好有 $m$ 个儿子(或为叶子)的有向根树称为 $m$ 元正则树。通常,$m$ 元正则树要求所有内点都有 $m$ 个儿子。

\textbf{完全 m 元正则树:} 所有叶子都在同一层,且每个内点都有 $m$ 个儿子的有向根树称为完全 $m$ 元正则树。

\textbf{有序根树:} 在根树中,如果为每个结点的所有儿子规定了一个次序(通常从左到右),则称该根树为有序根树。在有序根树中,结点的儿子有固定的顺序,不能随意交换。

\textbf{左儿子:} 在有序二叉树中,一个结点的左子结点称为该结点的左儿子。

\textbf{右儿子:} 在有序二叉树中,一个结点的右子结点称为该结点的右儿子。

\textbf{左子树:} 在有序二叉树中,以某个结点的左儿子为根的子树称为该结点的左子树。

\textbf{右子树:} 在有序二叉树中,以某个结点的右儿子为根的子树称为该结点的右子树。
\end{definition}
注意不要混淆层数与高度,层数是针对某个结点而言的,而高度是指整个树。

有向根数的结点的入度为0或者1.

\begin{theorem}{}{}
    高度为$h$的$m$元树至多有$n=m^h$个树叶。
\end{theorem}
对高度 $h$ 进行归纳证明。当 $h=0$ 时,$T$ 只有一个结点(根),该结点是叶结点,故叶结点数为 $1 = m^0$,结论成立。

假设对于高度不超过 $h-1$ 的 $m$ 元树,结论成立。考虑高度为 $h$ 的 $m$ 元树 $T$。设根有 $k$ 个子树 $T_1, T_2, \dots, T_k$,其中 $k \le m$。由于 $T$ 的高度为 $h$,每个子树 $T_i$ 的高度不超过 $h-1$。由归纳假设,每个子树 $T_i$ 的叶结点数不超过 $m^{h-1}$。$T$ 的叶结点是各子树的叶结点之和,故 $T$ 的叶结点数不超过
    \[
    k \cdot m^{h-1} \le m \cdot m^{h-1} = m^h.
    \]
等号成立当且仅当 $k=m$ 且每个子树都是高度为 $h-1$ 的满 $m$ 元树(即每个子树恰有 $m^{h-1}$ 个叶结点)。由归纳法,定理得证。
\begin{theorem}{$m$ 元正则树的结点与数量关系}{}
设 $T$ 为 $m$ 元正则树(每个分支点恰有 $m$ 个儿子),并设:$m$为正则树的元数(每个分支点的儿子数),$n$为$T$ 的结点总数,$i$为$T$ 的分支点数(内点数),$l$为$T$ 的叶子数,$e$为$T$ 的边数。则以下关系成立:
\[\begin{cases}
    e &= n - 1 \quad (\text{树的性质}) \\[-3pt]
    e &= m \cdot i \quad (\text{每个分支点引出 $m$ 条边}) \\[-3pt]
    n &= i + l \\[-3pt]
    n &= m \cdot i + 1
\end{cases}
\Rightarrow
\begin{cases}
    n = m \cdot i + 1,\\[-3pt]
    l = n - i = (m-1)i + 1,\\
    i = \dfrac{n-1}{m} = \dfrac{l-1}{m-1}.
\end{cases}
\]
已知其中任意两个量,即可求出其余量。
\end{theorem}
\begin{example}{}{}
    设$T$是正则二叉树,有$t$片树叶,则其阶数为$n=2t-1$
\end{example}
\textbf{证明:} 设 $i$ 为内部节点数(非树叶结点的数量)。由于 $T$ 是正则二叉树,每个内部节点有两个子节点。总节点数 $n = i + t$。在树中,边数 $e = n - 1$。另一方面,从子节点视角,总边数等于所有内部节点的子节点数之和,即 $2i$(因为每个内部节点有两个子节点),但根节点不是任何节点的子节点,所以 $e = 2i$。因此,$2i = n - 1$。代入 $n = i + t$,得 $2i = i + t - 1$,即 $i = t - 1$。所以 $n = (t - 1) + t = 2t - 1$。

\textbf{另解:}用握手定理,$T$的树根是2度,所有内部结点都是3度,树叶为1度,设$e$为树的边数,$i$是非树叶结点的数量,那么$2e=2+3(i-1)+t,n=e+1=i+t$,所以$n=2t-1$.
\begin{example}{}{}
    二元正则树的树叶数为$L$,则其边数为$2L-2$
\end{example}
已知$n=2i+1$,$i$为二元树的分支点数,代入$n=e+1,i=n-L=e+1-L$,解得边数为$2L-2$。
\begin{definition}{有序根树的遍历}{}
前序遍历算法:首先访问根结点,然后递归地对左子树执行前序遍历,最后递归地对右子树执行前序遍历。其访问顺序遵循“根-左-右”的规则。

中序遍历算法:首先递归地对左子树执行中序遍历,然后访问根结点,最后递归地对右子树执行中序遍历。其访问顺序遵循“左-根-右”的规则。

后序遍历算法:首先递归地对左子树执行后序遍历,然后递归地对右子树执行后序遍历,最后访问根结点。其访问顺序遵循“左-右-根”的规则。
\end{definition}
\begin{theorem}{确定二叉树}{}
    已知前序和后序不能唯一确定二叉树,除非二叉树是满二叉树(每个结点都有0或2个子结点)
\end{theorem}
设二叉树的前序序列为 $P$,后序序列为 $Q$。已知:
    \begin{align*}
        P &= \text{根} + (\text{左子树的前序}) + (\text{右子树的前序}), \\
        Q &= (\text{左子树的后序}) + (\text{右子树的后序}) + \text{根}.
    \end{align*}
    若根结点有唯一子结点(即度为1),则无法判断该子结点是左子结点还是右子结点,从而导致歧义。具体地:
 
    若根有唯一左子结点,则左子树的前序序列等于 $P$ 去掉根,左子树的后序序列等于 $Q$ 去掉根。
      
    若根有唯一右子结点,则右子树的前序序列等于 $P$ 去掉根,右子树的后序序列等于 $Q$ 去掉根。
    
    在这两种情况下,去掉根后的前序序列和后序序列相同,但对应的子树结构可能不同(左子树或右子树),因此无法唯一确定原二叉树。

    \textbf{ 满二叉树的情形}:若二叉树是满二叉树,则每个结点有0个或2个子结点。此时,前序序列的第一个结点是根,第二个结点是左子树的根;后序序列的最后一个结点是根,倒数第二个结点是右子树的根。因此,我们可以唯一地划分出左子树和右子树的前序、后序序列,从而递归地唯一确定二叉树。

    \vspace{10pt}

二叉树的三种遍历序列:前序(根-左-右)、中序(左-根-右)和后序(左-右-根)。已知其中两个序列,可以推导出第三个序列,但并非所有情况都唯一。

    \textbf{1. 已知前序和中序,求后序}:
    前序的第一个结点是根结点,在中序中找到根结点,其左边是左子树的中序,右边是右子树的中序。根据左子树的结点数,可从前序中分出左子树和右子树的前序。递归处理左、右子树,得到左、右子树的后序,然后按左-右-根的顺序拼接。

    \textbf{示例:}
    前序:ABDCE,中序:DBAEC \\
    根为A,左子树中序:DB,右子树中序:EC。 \\
    左子树前序:BD(前序中A后的两个),右子树前序:CE。 \\
    递归:左子树的前序BD、中序DB → 后序DB;右子树的前序CE、中序EC → 后序EC。 \\
    整体后序:左后序(DB)+右后序(EC)+根(A)= DBECA。

    \textbf{2. 已知中序和后序,求前序}:
    后序的最后一个结点是根结点,在中序中找到根结点,其左边是左子树的中序,右边是右子树的中序。根据左子树的结点数,从后序中分出左、右子树的后序。递归处理左、右子树,得到左、右子树的前序,然后按根-左-右的顺序拼接。

    \textbf{示例:}
    中序:DBAEC,后序:DBECA \\
    根为A,左子树中序:DB,右子树中序:EC。 \\
    左子树后序:DB(后序中前两个),右子树后序:EC(后序中接着的两个)。 \\
    递归:左子树的中序DB、后序DB → 前序BD;右子树的中序EC、后序EC → 前序CE。 \\
    整体前序:根(A)+左前序(BD)+右前序(CE)= ABDCE。

\begin{definition}{前缀码,二元前缀码}{}
\textbf{前缀码:} 设 $C = \{\alpha_1, \alpha_2, \dots, \alpha_n\}$ 是一个码(即一组字符串的集合),其中每个 $\alpha_i$ 是某个字母表上的非空字符串。如果对任意 $i \neq j$,$\alpha_i$ 都不是 $\alpha_j$ 的前缀(即不存在字符串 $\beta$ 使得 $\alpha_i \beta = \alpha_j$),则称 $C$ 是前缀码(亦称前缀无关码)。

\textbf{二元前缀码:} 若前缀码 $C$ 中的码字仅由两个符号(通常为0和1)组成,则称 $C$ 为二元前缀码。二元前缀码是信息论与编码理论中最常用的一类前缀码,例如霍夫曼编码即为最优的二元前缀码。
\end{definition}
前缀码不是通过$m$元树的前序遍历得到的编码,而是通过 $m$ 元树中从根到叶子的路径标记得到的编码。

前缀码可以通过任何的$m$元树构造,不一定是满二元树。每个叶子结点对应一个码字。从根到叶子的路径上,每条边标记一个符号(例如,二进制中左枝标记0,右枝标记1)。路径上的符号序列即为该叶子对应字符的码字。
\begin{definition}{波兰记法与逆波兰记法的二叉树遍历定义}{}
波兰记法(前缀表示法)和逆波兰记法(后缀表示法)可以通过表达式二叉树的前序遍历和后序遍历来定义。

设有一个表达式二叉树,其中每个内部结点表示一个运算符(如 $+$、$-$、$\times$、$/$),每个叶子结点表示一个操作数(如常数或变量)。则:

\textbf{波兰记法(Polish Notation)}:对该二叉树进行前序遍历(先访问根结点,然后递归地前序遍历左子树,最后前序遍历右子树)所得到的序列即为波兰记法。例如,表达式 $(a + b) \times c$ 对应的二叉树前序遍历序列为 $\times + a\, b\, c$。

\textbf{逆波兰记法(Reverse Polish Notation)}:对该二叉树进行后序遍历(先递归地后序遍历左子树,然后后序遍历右子树,最后访问根结点)所得到的序列即为逆波兰记法。例如,表达式 $(a + b) \times c$ 对应的二叉树后序遍历序列为 $a\, b + c \times$。

这两种记法都不需要括号来指定运算顺序,因为二叉树的遍历顺序隐含了运算的优先级和结合性。
\end{definition}
\begin{definition}{Huffman 算法}{}
霍夫曼算法(Huffman Algorithm)是一种用于构造最优前缀码(霍夫曼编码)的贪心算法。该算法由大卫·霍夫曼于1952年提出,用于根据字符出现频率构造最小平均长度的前缀码。

\textbf{算法步骤:}
设字符集为 $C$,每个字符 $c_i$ 有一个频率 $f_i$。

将每个字符 $c_i$ 看作一棵单结点二叉树,其权值为频率 $f_i$,将这些树放入优先队列 $Q$ 中(按权值升序排列)。

当 $|Q| > 1$ 时,重复以下操作:从 $Q$ 中取出两棵权值最小的二叉树 $T_1$ 和 $T_2$。创建一棵新树 $T$,以 $T_1$ 和 $T_2$ 为左右子树,$T$ 的根结点权值为 $T_1$ 和 $T_2$ 的权值之和。将 $T$ 插入优先队列 $Q$ 中。

当 $|Q| = 1$ 时,队列中剩下的树即为霍夫曼树。从霍夫曼树的根到每个叶子结点的路径(通常规定左分支为0,右分支为1)即得到每个字符的霍夫曼编码。
\end{definition}
设字符集 $\{A,B,C\}$ 的频率分别为 0.5, 0.3, 0.2。构造霍夫曼树:

初始队列:$(A,0.5), (B,0.3), (C,0.2)$。取出 $B$ 和 $C$,合并为 $D$,权值 0.5,队列:$(A,0.5), (D,0.5)$。取出 $A$ 和 $D$,合并为 $E$,权值 1.0,队列:$(E,1.0)$。得到霍夫曼树,编码:$A:0$, $B:10$, $C:11$。
\begin{definition}{Huffman编码与平均码长}{}
\textbf{Huffman编码:} Huffman编码是一种最优前缀码,用于无损数据压缩。给定一个字符集 $C = \{c_1, c_2, \dots, c_n\}$ 和每个字符对应的出现频率 $f_i$(或概率 $p_i$),Huffman编码通过构建一棵二叉树(Huffman树)来为每个字符分配一个变长码字,使得出现频率高的字符具有较短的码字,出现频率低的字符具有较长的码字。Huffman编码的构造算法(Huffman算法)是一种贪心算法,每次选择频率最小的两个节点合并,直到形成一棵树。从根到叶子的路径上,左分支标记0,右分支标记1,路径上的0/1序列即为该叶子对应字符的码字。Huffman编码具有前缀性质,即任意一个码字都不是另一个码字的前缀。

\textbf{平均码长:} 设字符 $c_i$ 的出现概率为 $p_i$(或频率 $f_i$,满足 $\sum_i p_i = 1$),其对应的Huffman码字长度为 $l_i$,则平均码长定义为
\[
L = \sum_{i=1}^n p_i l_i.
\]
对于给定的概率分布,Huffman编码使得平均码长最小,即它是所有前缀码中平均码长最短的编码。
\end{definition}
\begin{theorem}{}{}
任何一个前缀码都对应一个二元树。任何一个二元树的树叶都对应一个前缀码
\end{theorem}
任何一个前缀码都对应一个二元树:给定一个前缀码,我们可以构造一个二叉树,其中每个叶子节点表示一个码字,从根到叶子的路径上的边标记(0/1)表示该码字的二进制串。由于是前缀码,没有码字是另一个码字的前缀,所以每个码字对应一个叶子节点,不会出现在中间节点上。

任何一个二叉树的叶子都对应一个前缀码:给定一个二叉树,我们可以为每个叶子分配一个从根到该叶子的路径上的边标记(0/1)组成的串。这些串构成一个前缀码,因为任何一个叶子对应的串不会是另一个叶子对应串的前缀(否则一个叶子会在另一个叶子的路径上,这与叶子定义矛盾)。
因此,前缀码和二叉树(叶子节点)之间有一一对应关系。