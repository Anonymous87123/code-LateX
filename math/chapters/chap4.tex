\chapter{函数}
\section{函数(映射)的定义与表示}
\begin{definition}{函数,定义域,值域,陪域,像,原像}{function-terms}
\textbf{函数}:设 $A$ 和 $B$ 是两个集合。一个从 $A$ 到 $B$ 的函数 $f$ 是一个特殊的二元关系,它对 $A$ 中的每个元素 $x$ 指定 $B$ 中唯一的一个元素,记为 $f(x)$。记作 $f: A \to B$。

\textbf{定义域}:函数 $f: A \to B$ 的定义域是集合 $A$,即函数输入的全体。

\textbf{陪域}:函数 $f: A \to B$ 的陪域是集合 $B$,即函数输出的可能取值范围。

\textbf{值域}:函数 $f: A \to B$ 的值域是 $f$ 在 $A$ 上所有取值的集合,即 $\{f(x) \mid x \in A\}$,陪域的子集。

\textbf{像}:对于子集 $S \subseteq A$,$S$ 在 $f$ 下的像是集合 $f(S) = \{f(x) \mid x \in S\}$。

\textbf{原像}:对于子集 $T \subseteq B$,$T$ 在 $f$ 下的原像是集合 $f^{-1}(T) = \{x \in A \mid f(x) \in T\}$。
\end{definition}
\textbf{注}:函数也称为映射。函数必须满足单值性:对每个 $x \in A$,有且仅有一个 $f(x) \in B$ 与之对应。而且定义域中的每个元素都必须有对应的值。做题时关注定义域,如果有些定义域的元素没有对应的值,则该关系不是函数。

如:$\{(x_1,x_2)|x_1,x_2\in\mathbb{N},x_1+x_2=10\}$就不是函数
\begin{definition}{计算机科学中的特殊函数类型}{special-functions}
设 $A$ 和 $B$ 是集合,定义以下特殊函数类型:

\textbf{常函数}:函数 $f: A \to B$ 称为常函数,如果存在 $b \in B$ 使得对所有 $a \in A$,有 $f(a) = b$。即函数值不随自变量变化。

\textbf{恒等函数}:函数 $f_A: A \to A$ ,如果对于所有$x\in A$都有$f(x)=x$,则称为恒等函数。

\textbf{特征函数}:设 $S \subseteq A$,$S$ 的特征函数 $f_S: A \to \{0,1\}$ 定义为:
\[
f_S(a) = 
\begin{cases}
1 & \text{如果 } a \in S \\
0 & \text{如果 } a \notin S
\end{cases}
\]
\textbf{自然映射}:设 $\sim$ 是 $A$ 上的等价关系,$A/\sim$ 是商集。自然映射 $\pi: A \to A/\sim$ 定义为 $\pi(a) = [a]$,其中 $[a]$ 是 $a$ 的等价类。

\textbf{上取整函数}:函数 $\lceil \cdot \rceil: \mathbb{R} \to \mathbb{Z}$ 定义为 $\lceil x \rceil = \min\{n \in \mathbb{Z} \mid n \geqslant x\}$,即不小于 $x$ 的最小整数。

\textbf{下取整函数}:函数 $\lfloor \cdot \rfloor: \mathbb{R} \to \mathbb{Z}$ 定义为 $\lfloor x \rfloor = \max\{n \in \mathbb{Z} \mid n \leqslant x\}$,即不大于 $x$ 的最大整数。
\end{definition}
\begin{theorem}{函数经过集合运算后的函数性}{function-operations}
设 $f, g: A \to B$ 是两个函数。将 $f$ 和 $g$ 视为 $A \times B$ 的子集(即它们的图像)。那么:\\
$f \cap g$ 是从 $A$ 到 $B$ 的函数当且仅当 $f = g$。\\
$f \cup g$ 是从 $A$ 到 $B$ 的函数当且仅当 $f = g$。\\
$\sim f = A \times B - f$ 不是从 $A$ 到 $B$ 的函数,除非 $A = \varnothing$。\\
$f - g$ 是从 $A$ 到 $B$ 的函数当且仅当对于所有 $a \in A$,有 $f(a) \neq g(a)$。\\
$f \oplus g$ 是从 $A$ 到 $B$ 的函数当且仅当 $A = \varnothing$。
\end{theorem}

对于 $f \cap g$:如果 $f = g$,则 $f \cap g = f$,是从 $A$ 到 $B$ 的函数。反之,如果 $f \cap g$ 是从 $A$ 到 $B$ 的函数,则对任意 $a \in A$,存在唯一的 $b \in B$ 使得 $(a, b) \in f \cap g$,故 $(a, b) \in f$ 且 $(a, b) \in g$,即 $f(a) = b$ 且 $g(a) = b$,所以 $f(a) = g(a)$ 对所有 $a \in A$ 成立,即 $f = g$。

对于 $f \cup g$:如果 $f = g$,则 $f \cup g = f$,是从 $A$ 到 $B$ 的函数。反之,如果 $f \cup g$ 是从 $A$ 到 $B$ 的函数,则对任意 $a \in A$,存在唯一的 $b \in B$ 使得 $(a, b) \in f \cup g$。如果存在 $a \in A$ 使得 $f(a) \neq g(a)$,则 $(a, f(a)) \in f$ 和 $(a, g(a)) \in g$ 都在 $f \cup g$ 中,但 $f(a) \neq g(a)$,违反函数的单值性,矛盾。故对所有 $a \in A$,有 $f(a) = g(a)$,即 $f = g$。

对于 $\sim f$:如果 $A \neq \varnothing$,取 $a \in A$,则存在 $b = f(a)$ 使得 $(a, b) \in f$,故 $(a, b) \notin \sim f$。但对 $b' \neq b$,有 $(a, b') \in \sim f$,且 $b'$ 有多个选择(因为 $B$ 至少有两个元素,或如果 $|B|=1$ 则 $b'$ 唯一,但通常 $B$ 非空),但 $\sim f$ 中包含所有 $(a, b')$ 对于 $b' \neq b$,所以对同一个 $a$,有多个 $b'$ 对应,违反单值性。如果 $A = \varnothing$,则 $f$ 是空函数,$\sim f = \varnothing$,也是空函数(从空集到 $B$ 的函数)。

对于 $f - g$:如果对于所有 $a \in A$,有 $f(a) \neq g(a)$,则对每个 $a \in A$,有 $(a, f(a)) \in f$ 但 $(a, f(a)) \notin g$,故 $(a, f(a)) \in f - g$,且唯一,所以 $f - g = f$,是从 $A$ 到 $B$ 的函数。反之,如果 $f - g$ 是从 $A$ 到 $B$ 的函数,则对每个 $a \in A$,存在唯一的 $b \in B$ 使得 $(a, b) \in f - g$,故 $(a, b) \in f$ 且 $(a, b) \notin g$,所以 $b = f(a)$ 且 $b \neq g(a)$,即 $f(a) \neq g(a)$。

对于 $f \oplus g$:如果 $A = \varnothing$,则 $f$ 和 $g$ 都是空函数,$f \oplus g = \varnothing$,是从空集到 $B$ 的函数。如果 $A \neq \varnothing$,假设 $f \oplus g$ 是从 $A$ 到 $B$ 的函数。则对任意 $a \in A$,存在唯一的 $b \in B$ 使得 $(a, b) \in f \oplus g$。但由对称差定义,$(a, b) \in f \oplus g$ 当且仅当 $(a, b)$ 在 $f$ 或 $g$ 中但不同时在两者中。如果 $f(a) = g(a)$,则 $(a, f(a))$ 不在 $f \oplus g$ 中,矛盾于存在性。如果 $f(a) \neq g(a)$,则 $(a, f(a))$ 和 $(a, g(a))$ 都在 $f \oplus g$ 中,违反唯一性。故当 $A \neq \varnothing$ 时,$f \oplus g$ 不是函数。

\begin{theorem}{函数关系的性质}{}
设 $A$ 和 $B$ 是两个集合。所有从 $A$ 到 $B$ 的函数的集合记为 $B^A$,即:\vspace{-5pt}
\[
B^A = \{f \mid f: A \to B\}\vspace{-5pt}
\]
如果 $A$ 和 $B$ 都是有限集,且 $|A| = m$,$|B| = n$,则 $|B^A| = n^m$。\\
如果 $A = \varnothing$,则 $B^A = \{\varnothing\}$(空函数)。如果 $B = \varnothing$ 且 $A \neq \varnothing$,则 $B^A = \varnothing$。\\
特别地,$2^A$ 表示 $A$ 的幂集 $\mathcal{P}(A)$,因为 $2^A$ 与 $A$ 的特征函数集合 $\{f: A \to \{0,1\}\}$ 一一对应。
\end{theorem}
\begin{theorem}{函数定义域的性质}{function-domain-properties}
设 $f: A \to B$ 是一个函数,$S, T \subseteq A$,则:

(1) 若 $S \subseteq T$,则 $f(S) \subseteq f(T)$\quad (2) $f(S \cap T) \subseteq f(S) \cap f(T)$

(3) $f(S) - f(T) \subseteq f(S - T)$\quad (4) $f(S) \oplus f(T) \subseteq f(S \oplus T)$
\end{theorem}
(1) 设 $y \in f(S)$,即存在 $x \in S$ 使得 $f(x) = y$。那么由于 $S \subseteq T$,所以 $x \in T$,因此 $y = f(x) \in f(T)$。故 $f(S) \subseteq f(T)$。

(2) 设 $y \in f(S \cap T)$,则存在 $x \in S \cap T$ 使得 $f(x) = y$。由于 $x \in S$ 且 $x \in T$,所以 $y \in f(S)$ 且 $y \in f(T)$,即 $y \in f(S) \cap f(T)$。故 $f(S \cap T) \subseteq f(S) \cap f(T)$。

(3) 设 $y \in f(S) - f(T)$,则 $y \in f(S)$ 且 $y \notin f(T)$。由 $y \in f(S)$,存在 $x \in S$ 使得 $f(x) = y$。假设 $x \in T$,则 $y = f(x) \in f(T)$,矛盾。故 $x \notin T$,即 $x \in S - T$,所以 $y \in f(S - T)$。

(4) 由对称差的定义和(3)的结果:
\[
f(S) \oplus f(T) = (f(S) - f(T)) \cup (f(T) - f(S)) \subseteq f(S - T) \cup f(T - S)
\]
由于 $f(S - T) \cup f(T - S) = f((S - T) \cup (T - S)) = f(S \oplus T)$,所以 $f(S) \oplus f(T) \subseteq f(S \oplus T)$。
\begin{theorem}{等号成立的条件}{equality-conditions}
设 $f: A \to B$ 是一个函数,$S, T \subseteq A$,则:

(1) $f(S \cap T) = f(S) \cap f(T)$ 当且仅当对于任意 $s \in S$ 和 $t \in T$,如果 $f(s) = f(t)$,则存在 $x \in S \cap T$ 使得 $f(x) = f(s)$(等于没说)。特别的,如果$f$ 是单射,就满足此条件。

(2) $f(S) - f(T) = f(S - T)$ 当且仅当 $f$ 是单射。

(3) $f(S) \oplus f(T) = f(S \oplus T)$ 当且仅当 $f$ 是单射。
\end{theorem}
(1) 我们已经知道 $f(S \cap T) \subseteq f(S) \cap f(T)$ 总是成立。要证明反向包含,设 $y \in f(S) \cap f(T)$,则存在 $s \in S$ 和 $t \in T$ 使得 $f(s) = y$ 且 $f(t) = y$。如果 $f$ 是单射,则 $s = t \in S \cap T$,所以 $y \in f(S \cap T)$。更一般地,如果存在 $x \in S \cap T$ 使得 $f(x) = f(s) = y$,则 $y \in f(S \cap T)$。

(2) 我们已经知道 $f(S) - f(T) \subseteq f(S - T)$ 总是成立。要证明反向包含,设 $y \in f(S - T)$,则存在 $x \in S - T$ 使得 $f(x) = y$。由于 $x \in S$,所以 $y \in f(S)$。如果 $f$ 是单射,则 $y \notin f(T)$(否则存在 $z \in T$ 使得 $f(z) = y$,由单射性 $x = z \in T$,矛盾),所以 $y \in f(S) - f(T)$。

(3) 由对称差的定义和(1)(2)的结果,当 $f$ 是单射时,有:\vspace{-10pt}
\[
f(S) \oplus f(T) = (f(S) - f(T)) \cup (f(T) - f(S)) = f(S - T) \cup f(T - S) = f((S - T) \cup (T - S)) = f(S \oplus T)\vspace{-10pt}
\]
反之,如果 $f$ 不是单射,等号可能不成立。

\begin{remark}
当 $f$ 不是单射时,等号可能不成立。例如,考虑函数 $f: \{1,2\} \to \{1\}$,$f(1) = f(2) = 1$,取 $S = \{1\}$,$T = \{2\}$:

$f(S \cap T) = f(\varnothing) = \varnothing$,但 $f(S) \cap f(T) = \{1\} \cap \{1\} = \{1\}$

$f(S) - f(T) = \{1\} - \{1\} = \varnothing$,但 $f(S - T) = f(\{1\}) = \{1\}$

$f(S) \oplus f(T) = \{1\} \oplus \{1\} = \varnothing$,但 $f(S \oplus T) = f(\{1,2\}) = \{1\}$
\end{remark}
\section{单/满/双射,反函数,复合}
\begin{definition}{单射、满射、双射}{injection-surjection-bijection}
设 $f: A \to B$ 是一个函数。

\textbf{单射(injection)}:$f$ 是单射当且仅当对于任意 $x_1, x_2 \in A$,如果 $f(x_1) = f(x_2)$,则 $x_1 = x_2$。即不同的输入对应不同的输出。

\textbf{满射(surjection)}:$f$ 是满射当且仅当对于任意 $b \in B$,存在 $a \in A$ 使得 $f(a) = b$。即函数的值域等于陪域 $B$。

\textbf{双射(bijection)}:$f$ 是双射当且仅当它既是单射又是满射。即存在一一对应关系。

\textbf{等价定义}:

$f$ 是单射 $\Leftrightarrow$ $f$ 有左逆(存在 $g: B \to A$ 使得 $g \circ f = \text{id}_A$)

$f$ 是满射 $\Leftrightarrow$ $f$ 有右逆(存在 $h: B \to A$ 使得 $f \circ h = \text{id}_B$)

$f$ 是双射 $\Leftrightarrow$ $f$ 有逆函数(存在 $f^{-1}: B \to A$ 使得 $f^{-1} \circ f = \text{id}_A$ 且 $f \circ f^{-1} = \text{id}_B$)
\end{definition}
\begin{definition}{反函数}{}
设函数 $f: A \to B$ 是一个从定义域 $A$ 到值域 $B$ 的双射函数。$f$ 的\textbf{反函数},记作 $f^{-1}: B \to A$,定义为满足以下条件的函数:\vspace{-10pt}
\[
f^{-1}(b) = a \quad \text{当且仅当} \quad f(a) = b\vspace{-10pt}
\]
其中 $b \in B$,$a \in A$。
\end{definition}
\begin{definition}{函数的复合}{composition-of-functions}
设 $f: A \to B$ 和 $g: B \to C$ 是两个函数。函数 $f$ 和 $g$ 的复合函数 $g \circ f: A \to C$ 定义为\vspace{-10pt}
\[
(g \circ f)(x) = \{(x,z)|x\in A\land z\in C\land\exists y(y\in B\land (x,y)\in f\land(y,z)\in g)\}\vspace{-5pt}
\]
\end{definition}

注意如果$f$的值域不是$g$的定义域的子集,则复合函数$g\circ f$无法定义。下面证明:若$f: A \to B$ 和 $g: B \to C$ 是两个函数,则 $g \circ f$ 是 $A \to C$ 的函数

1. \textbf{存在性}:对于任意 $x \in A$,存在 $z \in C$ 使得 $(x,z) \in g \circ f$。

由于 $f: A \to B$ 是函数,对于任意 $x \in A$,存在唯一的 $y \in B$ 使得 $(x,y) \in f$。
由于 $g: B \to C$ 是函数,对于这个 $y \in B$,存在唯一的 $z \in C$ 使得 $(y,z) \in g$。
因此,存在 $z \in C$ 使得 $\exists y (y \in B \land (x,y) \in f \land (y,z) \in g)$,
即 $(x,z) \in g \circ f$。

2. \textbf{唯一性}:对于任意 $x \in A$,如果 $(x,z_1) \in g \circ f$ 且 $(x,z_2) \in g \circ f$,则 $z_1 = z_2$。

假设 $(x,z_1) \in g \circ f$ 且 $(x,z_2) \in g \circ f$。

则存在 $y_1, y_2 \in B$ 使得:$(x,y_1) \in f$ 且 $(y_1,z_1) \in g$,$(x,y_2) \in f$ 且 $(y_2,z_2) \in g$。

由于 $f$ 是函数,对于给定的 $x$,有唯一的 $y$ 使得 $(x,y) \in f$,所以 $y_1 = y_2$。

令 $y = y_1 = y_2$。由于 $g$ 是函数,对于给定的 $y$,有唯一的 $z$ 使得 $(y,z) \in g$,所以 $z_1 = z_2$。

综上,$g \circ f$ 满足函数的定义,因此是 $A \to C$ 的函数。
\begin{theorem}{函数复合的性质}{properties-of-function-composition}
设 $f: A \to B$, $g: B \to C$ 和 $h: C \to D$ 是函数。

(1) 结合律:$(h \circ g) \circ f = h \circ (g \circ f)$

(2) 单射、满射、双射的传递性:\\
如果 $f$ 和 $g$ 都是单/满/双射,则 $g \circ f$ 是单/满/双射。\\
(3) 通过复合函数确定原函数的性质:\\
如果 $g \circ f$ 是单射,则 $f$ 是单射。如果 $g \circ f$ 是满射,则 $g$ 是满射。\\
如果 $g \circ f$ 是双射,则 $f$ 是单射且 $g$ 是满射。\\
(4)衍生结论:\\
如果 $g \circ f$ 是满射,且$g$是单射,则$f$是满射。 \\
如果 $g \circ f$ 是单射,且$f$是满射,则$g$是单射。
\end{theorem}
(1) 对任意 $x \in A$,有:\vspace{-10pt}
\[
((h \circ g) \circ f)(x) = (h \circ g)(f(x)) = h(g(f(x)))\vspace{-10pt}
\]
\[
(h \circ (g \circ f))(x) = h((g \circ f)(x)) = h(g(f(x)))\vspace{-10pt}
\]
所以 $(h \circ g) \circ f = h \circ (g \circ f)$。

(2) 首先证明单射的传递性:设 $x_1, x_2 \in A$,若 $(g \circ f)(x_1) = (g \circ f)(x_2)$,即 $g(f(x_1)) = g(f(x_2))$。由于 $g$ 是单射,有 $f(x_1) = f(x_2)$,又因 $f$ 是单射,有 $x_1 = x_2$。故 $g \circ f$ 是单射。

其次证明满射的传递性:对任意 $c \in C$,由于 $g$ 是满射,存在 $b \in B$ 使得 $g(b) = c$。又因 $f$ 是满射,存在 $a \in A$ 使得 $f(a) = b$。于是 $(g \circ f)(a) = g(f(a)) = g(b) = c$。故 $g \circ f$ 是满射。

双射的传递性由前两个结果直接可得。

(3) 首先证明:如果 $g \circ f$ 是单射,则 $f$ 是单射。设 $x_1, x_2 \in A$,若 $f(x_1) = f(x_2)$,则 $g(f(x_1)) = g(f(x_2))$,即 $(g \circ f)(x_1) = (g \circ f)(x_2)$。由于 $g \circ f$ 是单射,有 $x_1 = x_2$。故 $f$ 是单射。

其次证明:如果 $g \circ f$ 是满射,则 $g$ 是满射。对任意 $c \in C$,由于 $g \circ f$ 是满射,存在 $a \in A$ 使得 $(g \circ f)(a) = c$,即 $g(f(a)) = c$。令 $b = f(a) \in B$,则 $g(b) = c$。故 $g$ 是满射。

最后,如果 $g \circ f$ 是双射,则它既是单射又是满射,由前两个结果可知 $f$ 是单射且 $g$ 是满射。

(4)(1) 假设 $g \circ f$ 是满射,且 $g$ 是单射。要证明 $f$ 是满射,即对任意 $b \in B$,存在 $a \in A$ 使得 $f(a) = b$。由于 $g \circ f$ 是满射,对于任意$b \in B$,令 $c = g(b)$,则存在 $a \in A$ 使得 $(g \circ f)(a) =g(f(a))= c=g(b)$。由于 $g$ 是单射,由 $g(f(a)) = g(b)$ 可得 $f(a) = b$。因此对任意 $b \in B$,存在 $a \in A$ 使得 $f(a) = b$,即 $f$ 是满射。

(4)(2) 假设 $g \circ f$ 是单射,且 $f$ 是满射。要证明 $g$ 是单射,即对任意 $b_1, b_2 \in B$,如果 $g(b_1) = g(b_2)$,则 $b_1 = b_2$。由于 $f$ 是满射,存在 $a_1, a_2 \in A$ 使得 $f(a_1) = b_1$ 和 $f(a_2) = b_2$。由 $g(b_1) = g(b_2)$ 得 $g(f(a_1)) = g(f(a_2))$,即 $(g \circ f)(a_1) = (g \circ f)(a_2)$。由于 $g \circ f$ 是单射,有 $a_1 = a_2$,从而 $b_1 = f(a_1) = f(a_2) = b_2$。因此 $g$ 是单射。
\vspace{20pt}

考虑一个快递运输的比喻来解释函数复合的性质:

设 $A$ = 寄件人集合,$B$ = 中转站集合,$C$ = 收件人集合。

函数 $f: A \to B$ 表示从寄件人到中转站的运输过程。
函数 $g: B \to C$ 表示从中转站到收件人的运输过程。
复合函数 $g \circ f: A \to C$ 表示从寄件人到收件人的完整运输过程。

结合律:如果还有第四个地点 $D$(例如国际转运中心)和函数 $h: C \to D$,那么无论是先组合 $g$ 和 $h$ 再与 $f$ 组合,还是先组合 $f$ 和 $g$ 再与 $h$ 组合,最终结果都是将快递从 $A$ 经 $B$ 和 $C$ 运送到 $D$。运输路径是确定的,与组合方式无关。

单射、满射、双射的传递性:如果从寄件人到中转站($f$)和中转站到收件人($g$)都是"一对一"运输(单射),那么整个运输过程($g \circ f$)也是"一对一"的。每个寄件人的快递最终会送到不同的收件人手中。如果两个阶段都能覆盖所有可能的地址(满射),那么整个运输过程也能覆盖所有收件人。如果两个阶段都是完美的"一对一且全覆盖"(双射),那么整个运输过程也是完美的。

通过复合函数确定原函数的性质:如果整个运输过程($g \circ f$)是"一对一"的(单射),那么第一阶段($f$)必须是"一对一"的。因为如果两个寄件人的快递在中转站就混在一起了,那么最终不可能送到不同的收件人手中。如果整个运输过程能覆盖所有收件人(满射),那么第二阶段($g$)必须能覆盖所有收件人。因为如果中转站有些地址无法送达最终收件人,那么这些收件人就收不到快递。如果整个运输过程完美无缺(双射),那么第一阶段必须保证不混件(单射),第二阶段必须保证全覆盖(满射)。
\begin{theorem}{复合函数的反函数}{}
设 $f: B \to C$ 和 $g: A \to B$ 都是双射函数,则复合函数 $f \circ g: A \to C$ 也是双射函数\vspace{-10pt}
\[(f \circ g)^{-1} = g^{-1} \circ f^{-1}\]
\end{theorem}
要证明 $(f \circ g)^{-1} = g^{-1} \circ f^{-1}$,需要验证两个条件$f\circ f^{-1}=f^{-1}\circ f$均为恒等函数;设定义在集合$A$上的恒等函数为$\text{id}_A$:

1. 左复合等于恒等映射:
\[
(f \circ g) \circ (g^{-1} \circ f^{-1}) = f \circ (g \circ g^{-1}) \circ f^{-1} = f \circ \text{id}_B \circ f^{-1} = f \circ f^{-1} = \text{id}_C
\]

2. 右复合等于恒等映射:
\[
(g^{-1} \circ f^{-1}) \circ (f \circ g) = g^{-1} \circ (f^{-1} \circ f) \circ g = g^{-1} \circ \text{id}_B \circ g = g^{-1} \circ g = \text{id}_A
\]

因此,$(f \circ g)^{-1} = g^{-1} \circ f^{-1}$。
\begin{example}{}{}
    设$f:A\to B$和$g:B\to P(A)$,其中$P(A)$表示$A$的幂集,且对于任意$b\in B$有$g(b)=\{x|(x\in A)\land(f(x)=b)\}$。证明:若$f$是满射,则$g$是单射
\end{example}
要证明 $g$ 是单射,即对于任意 $b_1, b_2 \in B$,如果 $g(b_1) = g(b_2)$,则 $b_1 = b_2$。由于 $f$ 是满射,对于任意 $b \in B$,存在至少一个 $x \in A$ 使得 $f(x) = b$,现在假设 $g(b_1) = g(b_2)$。由于 $g(b_1)$ 非空,存在 $x \in g(b_1)$。因为 $g(b_1) = g(b_2)$,所以 $x \in g(b_2)$。

根据 $g$ 的定义:$x \in g(b_1)$ 意味着 $f(x) = b_1$,$x \in g(b_2)$ 意味着 $f(x) = b_2$,因此,$b_1 = f(x) = b_2$,即 $b_1 = b_2$。故 $g$ 是单射。

\section{自然数}
\begin{definition}{后继集与自然数集的定义}{}
\textbf{后继集}:设 $A$ 是一个集合,$A$ 的\textbf{后继集}定义为 $A^{+} = A \cup \{A\}$。\\
\textbf{自然数集的递归定义}(冯·诺依曼定义):

1. \textbf{基础}:$0 = \varnothing$

2. \textbf{递归步骤}:$n + 1 = n^{+} = n \cup \{n\}$

3.\textbf{极限情况}:自然数集 $\mathbb{N}$ 是包含 $0$ 且在后继运算下封闭的最小集合\\
按照这个定义,自然数可以具体构造为:\vspace{-10pt}
\begin{align*}
0 &= \varnothing \\
1 &= 0^{+} = \varnothing \cup \{\varnothing\} = \{\varnothing\} \\
2 &= 1^{+} = \{\varnothing\} \cup \{\{\varnothing\}\} = \{\varnothing, \{\varnothing\}\} \\
3 &= 2^{+} = \{\varnothing, \{\varnothing\}\} \cup \{\{\varnothing, \{\varnothing\}\}\} = \{\varnothing, \{\varnothing\}, \{\varnothing, \{\varnothing\}\}\} \\
&\vdots
\vspace{-10pt}\end{align*}
\textbf{性质}:每个自然数都是它前面所有自然数的集合;$n = \{0, 1, 2, \ldots, n-1\}$;这种定义方式使得 $m < n$ 当且仅当 $m \in n$;自然数集 $\mathbb{N}$ 是一个归纳集
\end{definition}
\begin{definition}{第一数学归纳法}{}
    设 $P(n)$ 是一个与自然数 $n$ 相关的命题。如果满足:\\
    \textbf{基础步骤}:$P(0)$ 成立(即 $P(\varnothing)$ 成立)\\
    \textbf{归纳步骤}:对于任意自然数 $n$,如果 $P(n)$ 成立,则 $P(n^{+})$ 也成立\\
    其中 $n^{+} = n \cup \{n\}$ 是 $n$ 的后继集,那么,对于所有自然数 $n$,$P(n)$ 都成立。即\vspace{-10pt}
\[P(0) \land \forall n \in \mathbb{N}(P(n) \rightarrow P(n^{+}))] \rightarrow \forall n \in \mathbb{N}, P(n)\vspace{-10pt}\]
\textbf{原理的集合论基础}:数学归纳法原理等价于自然数集的良序原理
\end{definition}
证明:任何自然数都不是自己的元素。

使用数学归纳法证明。\textbf{基础步骤}:考虑 $n = 0$。根据定义,$0 = \varnothing$,即空集。空集不包含任何元素,因此显然 $0 \notin 0$。\textbf{归纳步骤}:假设对于某个自然数 $n$,有 $n \notin n$(归纳假设)。我们需要证明 $n+1 \notin n+1$。根据自然数的定义,$n+1 = n \cup \{n\}$,因此 $n+1$ 的元素是 $n$ 的所有元素以及 $n$ 本身。

反设$n+1 \in n+1$。那么由于 $n+1 = n \cup \{n\}$,我们有两种可能:$n+1 \in n$,或者 $n+1 = n$。考虑第一种情况:如果 $n+1 \in n$,那么根据引理1(自然数的传递性),由于 $n$ 是自然数,它是传递的,所以 $n+1 \subset n$。但 $n \in n+1$(由定义),因此 $n \in n$,这与归纳假设 $n \notin n$ 矛盾。
考虑第二种情况:如果 $n+1 = n$,那么这意味着 $n \cup \{n\} = n$,从而 $\{n\} \subseteq n$,即 $n \in n$。但这与归纳假设 $n \notin n$ 矛盾。

两种情况均导致矛盾。因此,原假设 $n+1 \in n+1$ 不成立,故 $n+1 \notin n+1$。
\begin{definition}{集合的特征函数}{characteristic-function}
设 $A$ 是全集$E$的子集,$A$ 的特征函数 $f_A$ 定义为:
\[
f_A:E\to\{0,1\},f_A(x) = 
\begin{cases}
1 & \text{如果 } x \in A \\
0 & \text{如果 } x \notin A
\end{cases}
\]
特征函数完全确定了集合 $A$,因为 $A = \{x \mid f_A(x) = 1\}$。
\end{definition}
\begin{theorem}{特征函数与集合运算}{characteristic-properties}
设 $A$ 和 $B$ 是集合,它们的特征函数分别为 $f_A$ 和 $f_B$,则有以下关系:\vspace{-5pt}
\begin{align*}
f_{A \cup B}(x) &= \max(f_A(x), f_B(x)) = f_A(x) + f_B(x) - f_A(x)f_B(x) \\
f_{A \cap B}(x) &= \min(f_A(x), f_B(x)) = f_A(x)f_B(x) \\
f_{~A}(x) &= 1 - f_A(x) \\
f_{A -B}(x) &= f_A(x)(1 - f_B(x)) \\
f_{A \oplus B}(x) &= |f_A(x) - f_B(x)| = f_A(x) + f_B(x) - 2f_A(x)f_B(x)
\end{align*}
\end{theorem}
\section{集合的基数,等势}
\begin{definition}{基数,比较两个集合的基数,等势}{cardinality}
集合 $A$ 的基数(或称势)是衡量集合大小的概念,记为 $|A|$。对于有限集,基数就是集合中元素的个数。无限集的基数概念更为复杂,需要用到选择公理等相关理论。

$A$ 和 $B$ 称为等势(记作 $|A| = |B|,A\sim B$),即存在一个从 $A$ 到 $B$ 的双射(一一对应)。

$|A| \leqslant |B|$ 表示存在从 $A$ 到 $B$ 的单射\\
$|A| < |B|$ 表示 $|A| \leqslant |B|$ 但 $|A| \neq |B|$\\
$|A| \geqslant |B|$ 表示存在从 $A$ 到 $B$ 的满射
\end{definition}
\begin{theorem}{基数的性质}{}
设$M$是一个集合,$P(M)$是其幂集,$P(M)$的基数记为$|P(M)|$。则:$|P(M)|>|M|$
\end{theorem}
假设存在双射 $f: M \to P(M)$。这个双射将集合$M$中的每个元素和$M$的每一个子集建立了一一对应关系,再定义集合 $A = \{ x \in M \mid x \notin f(x) \}$,意思是找出来所有这样的元素,满足它们分别所对应的那一个子集中不含有它自身。由于 $A \subseteq M$,有 $A \in P(M)$,即它是幂集中的一个元素,所以根据假设,$A$理应和$M$中的某一个元素对应,设这个元素为$a$,即$f(a) = A$。如果$a \in A$,那么根据$A$的定义,$a \notin f(a)$,而$f(a) = A$,所以$a \notin A$,矛盾。如果$a \notin A$,即$a \notin f(a)$,那么根据$A$的定义,$a$满足$A$的条件,所以$a \in A$,矛盾。

这个构造的核心思想是自指悖论,类似于"理发师悖论":理发师宣称:我给所有不给自己理发的人理发。那么理发师给自己理发吗?在我们的证明中:集合$A$被定义为“所有不包含在自己的像中的元素”所形成的集合,那么它是否包含在它自己的像中?
\begin{definition}{有限集与无限集}{finite-infinite-sets}
\textbf{有限集}:一个集合 $A$ 称为有限集,如果存在某个自然数 $n$ 和双射 $f: A \to \{0, 1, 2, \dots, n-1\}$,即 $A$ 与某个自然数等势。此时 $|A| = n$。

\textbf{无限集}:不是有限集的一个集合 $A$ 称为无限集。即不存在任何自然数 $n$ 使得$A$与某自然数等势。

\textbf{可数无限集}:与自然数集 $\mathbb{N}$ 等势的集合,即存在从 $\mathbb{N}$ 到 $A$ 的双射。例如$\mathbb{Z},\mathbb{Q}$。

\textbf{不可数无限集}:无限但不可数的集合,即其基数严格大于可数无限集的基数。例如:实数集 $\mathbb{R}$、复数集 $\mathbb{C}$。

\textbf{注}:可数无限集的基数记为 $\aleph_0$(阿列夫零),实数集的基数记为 $\aleph$(连续统基数)。
\end{definition}
\begin{theorem}{有限集和无限集的基数}{}
    有限集不和其任何一个子集等势\\一个集合是无限集当且仅当它与它自己的某一个真子集等势
\end{theorem}
设 $S$ 是一个有限集,假设存在 $S$ 的一个真子集 $T \subset S$ 使得 $|T| = |S|$。但由于 $T$ 是 $S$ 的真子集,$|T| < |S|$,矛盾。因此有限集不能与它的任何真子集等势。

设 $S$ 是无限集。我们可以构造一个单射但不是满射的函数 $f: S \to S$,例如通过选择可数无限子集并建立双射。这样 $f(S)$ 就是 $S$ 的一个真子集,且与 $S$ 等势。反过来,如果 $S$ 与它的某个真子集等势,假设 $S$ 是有限集,则与第一部分矛盾。因此 $S$ 必须是无限集。
\begin{theorem}{无限集必然含有可数的子集}{}
    任意一个无限集必然含有可数的子集。
\end{theorem}
这个证明使用了选择公理(通过递归选择元素):

设 $X$ 是一个无限集。我们需要证明 $X$ 包含一个可数无限子集。由于 $X$ 是无限的,它非空,因此我们可以选择一个元素 $a_1 \in X$。考虑集合 $X -\{a_1\}$,由于 $X$ 是无限的,$X - \{a_1\}$ 仍是非空的(否则 $X$ 将是有限的),所以我们可以选择一个元素 $a_2 \in X - \{a_1\}$。类似地,假设我们已经选择了互不相同的元素 $a_1, a_2, \dots, a_n \in X$,由于 $X$ 是无限的,$X- \{a_1, a_2, \dots, a_n\}$ 非空,因此我们可以选择 $a_{n+1} \in X-\{a_1, a_2, \dots, a_n\}$。通过数学归纳法,我们构造了一个序列 $\{a_n\}_{n=1}^\infty$,其中所有 $a_n$ 互不相同。则集合 $\{a_n \mid n \in \mathbb{N}\}$ 是 $X$ 的一个可数无限子集。
\begin{theorem}{自然数集与有理数集等势}{}
    (1)自然数集与有理数集等势\\ 
    (2)证明存在双射 $g: \mathbb{N} \to \mathbb{N} \times \mathbb{N}$
\end{theorem}
证明 $\mathbb{N} \times \mathbb{N}$ 与 $\mathbb{N}$ 等势。

定义函数 $g: \mathbb{N} \to \mathbb{N} \times \mathbb{N}$ 如下:我们使用对角线枚举法。将 $\mathbb{N} \times \mathbb{N}$ 中的元素排列成一个二维网格,按照对角线顺序枚举:
\begin{align*}
&(0,0), \\
&(0,1), (1,0), \\
&(0,2), (1,1), (2,0), \\
&(0,3), (1,2), (2,1), (3,0), \\
&\vdots
\end{align*}
更形式化地,我们可以定义配对函数(这里和书上略有不同,因为这里是从$(0,0)$开始的):
\[
g(n) = (i,j) \quad \text{其中} \quad n = \frac{(i+j)(i+j+1)}{2} + j
\]
这个函数是双射,因为不同的 $n$ 对应不同的对 $(i,j)$,而且每个对 $(i,j) \in \mathbb{N} \times \mathbb{N}$ 都有对应的 $n \in \mathbb{N}$,因此,$\mathbb{N} \times \mathbb{N}$ 与 $\mathbb{N}$ 等势。

第二步:证明 $\mathbb{Q}$ 与 $\mathbb{N}$ 等势。

考虑有理数集 $\mathbb{Q}$,每个有理数可以唯一表示为既约分数 $\frac{p}{q}$,其中 $p \in \mathbb{Z}$, $q \in \mathbb{N}^+$,且 $\gcd(p,q) = 1$。

定义函数 $h: \mathbb{Q} \to \mathbb{Z} \times \mathbb{N}^+$ 为 $h\left(\frac{p}{q}\right) = (p,q)$,其中 $\frac{p}{q}$ 是既约分数。

由于 $\mathbb{Z}$ 与 $\mathbb{N}$ 等势(可以通过交错排列正负整数来构造双射),且 $\mathbb{N}^+$ 与 $\mathbb{N}$ 等势,所以 $\mathbb{Z} \times \mathbb{N}^+$ 与 $\mathbb{N} \times \mathbb{N}$ 等势。

结合第一步的结果,$\mathbb{Z} \times \mathbb{N}^+$ 与 $\mathbb{N}$ 等势,因此 $\mathbb{Q}$ 与 $\mathbb{N}$ 等势。

综上,自然数集 $\mathbb{N}$ 与有理数集 $\mathbb{Q}$ 等势。
\begin{theorem}{可数个可数集合的并集仍然是可数集合}{}
    可数个可数集合的并集仍然是可数集合。
    
    设有一族可数集合 $\{A_n\}_{n=1}^\infty$,其中每个 $A_n$ 是可数集合(即至多可数)。考虑并集 $S = \bigcup_{n=1}^\infty A_n$。需要证明 $S$ 是可数集合。
\end{theorem}
由于每个 $A_n$ 是可数的,存在满射 $f_n: \mathbb{N} \to A_n$(如果 $A_n$ 有限,则通过重复元素构造满射;如果 $A_n$ 可数无限,则存在双射)。

现在,考虑笛卡尔积 $\mathbb{N} \times \mathbb{N}$。由于 $\mathbb{N} \times \mathbb{N}$ 是可数集合(存在双射 $g: \mathbb{N} \to \mathbb{N} \times \mathbb{N}$),定义映射 $f: \mathbb{N} \times \mathbb{N} \to S$ 为:
\[
f(n, m) = f_n(m).
\]
由于每个 $f_n$ 是满射,$f$ 是满射从 $\mathbb{N} \times \mathbb{N}$ 到 $S$。又因为存在满射 $g: \mathbb{N} \to \mathbb{N} \times \mathbb{N}$,复合映射 $f \circ g: \mathbb{N} \to S$ 是满射。因此,存在满射从 $\mathbb{N}$ 到 $S$,这意味着 $S$ 至多可数。

如果至少有一个 $A_n$ 是无限集合,则 $S$ 是无限集合,因此 $S$ 是可数无限集合;否则,如果所有 $A_n$ 有限,则 $S$ 可能有限或无限,但至多可数。综上,$S$ 是可数集合。
\begin{theorem}{开区间不可列}{}
    开区间$(0,1)$是不可列的
\end{theorem}
使用康托尔对角线法证明。假设 $(0,1)$ 是可数集合,则存在双射 $f: \mathbb{N} \to (0,1)$。
我们可以将 $(0,1)$ 中的所有实数枚举为:
\[
x_1, x_2, x_3, \dots
\]
每个 $x_n$ 可以写成十进制小数形式(为避免表示不唯一,约定不使用无限连续的9的表示):
\[
\begin{aligned}
x_1 &= 0.a_{11}a_{12}a_{13}\dots \\
x_2 &= 0.a_{21}a_{22}a_{23}\dots \\
x_3 &= 0.a_{31}a_{32}a_{33}\dots \\
&\vdots
\end{aligned}
\]
其中每个 $a_{ij} \in \{0,1,\dots,9\}$。现在构造实数 $y = 0.b_1b_2b_3\dots \in (0,1)$,其中:
\[
b_n = 
\begin{cases}
1 & \text{如果 } a_{nn} \neq 1 \\
2 & \text{如果 } a_{nn} = 1
\end{cases}
\]
这样确保 $b_n \neq a_{nn}$ 且 $b_n \neq 9$(避免表示歧义)。

则 $y$ 与每个 $x_n$ 都不同,\textbf{因为 $y$ 的第 $n$ 位小数 $b_n$ 与 $x_n$ 的第 $n$ 位小数 $a_{nn}$ 不同}。因此 $y$ 不在枚举中,与 $f$ 是满射矛盾。假设错误,$(0,1)$ 是不可数集合。
\begin{theorem}{实数集是不可列的}{}
实数集是不可列的
\end{theorem}
本题只需构造一个一一对应映射将$\mathbb{R}$映射到$(0,1)$上即可。或者构造一个一一对应映射将$(0,1)$映射到$\mathbb{R}$上即可。比如$f: (0,1) \to \mathbb{R}$可以定义为:$f(x)=\tan\left(\pi x-\dfrac{\pi}{2}\right)$
\begin{theorem}{$\mathbb{R}$ 和 $\mathbb{R} \times \mathbb{R}$ 等势}{}
    实数集 $\mathbb{R}$ 与其笛卡尔积 $\mathbb{R} \times \mathbb{R}$ 等势,即存在从 $\mathbb{R}$ 到 $\mathbb{R} \times \mathbb{R}$ 的双射。
\end{theorem}
我们可以通过构造显式的双射来证明。已知 $|\mathbb{R}| = \mathfrak{c}$,且:
\[
|\mathbb{R} \times \mathbb{R}| = \mathfrak{c} \times \mathfrak{c} = \max(\mathfrak{c}, \mathfrak{c}) = \mathfrak{c}
\]

具体构造双射 $f: \mathbb{R} \to \mathbb{R} \times \mathbb{R}$ 的一种方法是通过交错小数展开:

对任意 $x \in \mathbb{R}$,考虑其十进制展开(为避免表示不唯一,约定不使用无限连续的9的表示)。将 $x$ 的小数部分按奇偶位分开:
\[
x = a_0.a_1a_2a_3a_4\dots
\]
定义:
\[
f(x) = (0.a_1a_3a_5\dots, 0.a_2a_4a_6\dots)
\]
即用奇数位小数构造第一个坐标,偶数位小数构造第二个坐标。

可以验证这是一个双射,因此 $\mathbb{R}$ 和 $\mathbb{R} \times \mathbb{R}$ 等势。